\chapter{Factorizations}
\label{factorizations}

\section*{A}

In Chapter \ref{braces} we found that groups with an exact factorization 
produce braces. In this chapter we will study a different relationship between
factorizations and braces. 

A group $G$ is said to be 
\textbf{factorized through subgroups} $A$ and $B$ if $G=AB$. We remark that we do not assume that
$A\cap B=\{1\}$. 

\index{Group!metabelian}
A group $G$ is \textbf{metabelian} if $[G,G]$ is abelian. 
Thus $G$ is metabelian if and only if there is a normal subgroup 
$K$ of $G$ such that $K$ and $G/K$ are abelian. 
The groups $\Sym_3$ and $\Alt_4$ are metabelian. 

\begin{exercise}
	Let $G$ be a metabelian group.  
	\begin{enumerate}
		\item If $H$ is a subgroup of $G$, then $H$ is metabelian. 
		\item If $f\colon G\to H$ is a group homomorphism, then $f(H)$ is metabelian.
	\end{enumerate}
\end{exercise}

\begin{exercise}
	Prove that $\SL_2(3)$ is metabelian. 
\end{exercise}

A straightforward calculation shows that 
the following formulas hold: 
\begin{align*}
&[a,bc]=[a,b]b[a,c]b^{-1},\\ 
&[ab,c]=a[b,c]a^{-1}[a,c].
\end{align*}

The following theorem is considered the most satisfying result about group factorization. The proof
is based on a surprisingly short and smart calculation with commutators. 

\begin{theorem}[It\^o]
	\label{theorem:Ito}
    \index{Theorem!of It\^o}	
	Let $G=AB$ be a factorization of $G$ through the abelian subgroups $A$ and $B$. 
	Then $G$ is metabelian.
\end{theorem}

\begin{proof}
	Since $G=AB$, it follows that $AB=BA$. Let us prove that $[A,B]$ is a normal 
	subgroup of $G$. Let $a,a_1,\alpha,\alpha_1\in A$ and $b,b_1,\beta,\beta_1\in B$ 
	be such that $\alpha b\alpha^{-1}=b_1a_1$, $\beta
	a\beta^{-1}=a_2b_2$. Since 
	\begin{align*}
		&\alpha[a,b]\alpha^{-1}=a(\alpha b\alpha^{-1})a^{-1}(\alpha b^{-1}\alpha^{-1})=ab_1a_1a^{-1}a_1^{-1}b_1^{-1}=[a,b_1]\in [A,B]\\
		&\beta[a,b]\beta^{-1}=(\beta a\beta^{-1})\beta b\beta^{-1}(\beta a^{-1}\beta^{-1})b^{-1}=a_2b_2bb_2^{-1}a_2^{-1}b^{-1}=[a_2,b]\in [A,B],
	\end{align*}
	it follows that $[A,B]$ is a normal subgroup of $G$. 

	Now we prove that $[A,B]$ is abelian. Since  
	\begin{align*}
		&\beta\alpha[a,b]\alpha^{-1}\beta^{-1} = \beta[a,b_1]\beta^{-1}=(\beta a\beta^{-1})b_1(\beta a^{-1}\beta^{-1})b_1^{-1}=[a_2,b_1],\\
		&\alpha\beta[a,b]\beta^{-1}\alpha^{-1} = \alpha[a_2,b]\alpha^{-1}=a_2(\alpha b\alpha^{-1})a_2^{-1}(\alpha b\alpha^{-1})=[a_2,b_1],
	\end{align*}
	a direct calculation shows that  
	\[
		[\alpha^{-1},\beta^{-1}][a,b][\alpha^{-1},\beta^{-1}]^{-1}=[a,b].
	\]
	Since two arbitrary generators of $[A,B]$ commute, the group $[A,B]$ is abelian.
	
	Finally we note that $[G,G]=[A,B]$. Since $[A,B]$ is normal in $G$,  
	\[
	[a_1b_1,a_2b_2]=a_1[a_2,b_1]^{-1}a_1^{-1}a_2[a_1,b_2]a_2^{-1}\subseteq [A,B].\qedhere 
	\]
\end{proof}


Sysak found the following generalization of It\^o's theorem:

\begin{theorem}[Sysak]
\index{Theorem!of Sysak}
If $A$ and $B$ are abelian subgroups of $G$ and
$H$ is a subgroup of $G$ contained in the set $AB$, then
$H$ is metabelian. 
\end{theorem}

The proof appears in \cite{MR988177}.  

There are several other interesting results in the theory of factorizable groups. 
Another important result that is worth mentioning is the following
theorem. 

\begin{theorem}[Kegel--Wielandt]
\index{Theorem!of Kegel--Wielandt}
Let $G$ be a finite group. If there are nilpotent subgroups $A$ and $B$ of $G$ such that 
$G=AB$, then $G$ is solvable. 
\end{theorem}

The proof appears for example in~\cite[Theorem 2.4.3]{MR1211633}. 

The theorem of Kegel--Wielandt turns out to be the main tool in the proof of the following result 
on the structure of finite braces. 
This proves a conjecture 
of Byott, see~\cite{MR3425626}.

\begin{theorem}
\label{thm:mul_nilpotent}
Let $A$ be a finite brace with nilpotent multiplicative group. Then 
the additive group of $A$ is solvable.
\end{theorem}

\begin{proof}
    Let $K$ be the additive group of $A$ and $G$ be the multiplicative group of $A$. The group
    $\Gamma=K\rtimes G$ has multiplication
    \[
    (g,\lambda_g)(h,\lambda_h)=(g+\lambda_g(h),\lambda_g\lambda_h)=(g\circ h,\lambda_{g\circ h}).
    \]
    Let $f\colon G\to\Gamma$, $g\mapsto(g,\lambda_g)$. 
    Then $f$ is a group homomorphism and $f(G)$ is nilpotent. 
    Since $\lambda(G)$ is nilpotent, the finite group 
    $K\rtimes\lambda(G)=f(G)\lambda(G)$ is a product of nilpotent groups. 
    By the theorem of Kegel--Wielandt, $K\rtimes\lambda(G)$ 
    is solvable. Hence $K$ is solvable. 
\end{proof}

\section*{B}

It turns out to be interesting to study factorization of braces. 

\begin{definition}
    Let $A$ be a brace and let $B$ and $C$ be left ideals of $A$. We say
    that $A$ admits a \emph{factorization} through $B$ and $C$ if $A=B+C$.
\end{definition}

Note that if a brace $A$ admits a factorization through $B$ and $C$, then  
it follows that
\[
A=B+C=C+B=B\circ C=C\circ B.
\]

Now we prove an analog of It\^o's theorem in the context of braces. 
It turns out that one needs to consider factorizations through strong left ideals. We also need
the following
definition: 

\begin{definition}
\index{Brace!meta-trivial}
A brace $A$ is said to be \emph{meta-trivial} if $A^{(2)}$ is a trivial brace. 
\end{definition}

Clearly a brace $A$ is meta-trivial if and only there is 
an ideal $I$ of $A$ such that $I$ and $A/I$ are trivial as braces.

\begin{lemma} 
\label{lem:calcbraces}
Let $A$ be a brace. For any $x,y,z \in A$ the following statements hold:
\begin{enumerate}
    \item $x*(y+z) = x*y +y +x*z -y$,
    \item $(x\circ y)*z = x*(y*z) + y*z + x*z$.
\end{enumerate}
\end{lemma}

\begin{lemma}\label{lem:calculations}
    Let $A$ be a brace such that $A = B+C$, where $B$ and $C$ are left ideals. If $B$ and $C$ are trivial as braces then,
    for any $b,\beta \in B$ and $c,\gamma \in C$, the following statements hold:
	\begin{enumerate}
	    \item $\lambda_{\beta \circ \gamma} = \lambda_{\gamma \circ \beta}$,
	    \item $(c + b) \circ \beta - \beta = c + b + c* \beta $,
	    \item $b\circ c \circ b' \circ c' = b \circ c - c\circ b = b + \lambda_b(c) - \lambda_c(b) -c \in \ker \lambda$.
	\end{enumerate}
\end{lemma}

\begin{proof}
	To prove (1) put $c_1=\lambda_\beta(c)\in C$ and $b_1=\lambda_{\gamma}(b)\in B$. As $B$ and $C$ are trivial braces, $\lambda_\beta(b+c) =\lambda_{\beta}(b)+\lambda_\beta(c)=b+c_1$ and similarly 
	$\lambda_\gamma(b+c)=b_1+c$.
	Then $\lambda_{\beta\circ\gamma}(b+c)=b_1+c_1=\lambda_{\gamma\circ\beta}(b+c)$. 
	
	Let us prove (2). As $B$ is a trivial brace, it follows from~\eqref{eq:formulas} that 
	\begin{align*}
		(c+b)\circ\beta-\beta &= (c\circ\lambda_{c'}(b))\circ\beta-\beta\\
		&=c\circ(\lambda_{c'}(b)\circ\beta)-\beta\\
		&=c\circ(\lambda_{c'}(b)+\beta)-\beta\\
		&=c\circ\lambda_{c'}(b)-c+c\circ\beta-\beta\\
		&=c+b+c*\beta.
	\end{align*}
	Part (3) follows from the following computation
	\begin{align*}
	    b \circ c \circ b' \circ c' &= (b\circ c) +\lambda_{b\circ c} (b' + \lambda_{b'}(c'))\\
	    &=b + \lambda_b(c) + \lambda_{b\circ c}(b') + \lambda_{b \circ c \circ b'}(c')\\
	     &= b+\lambda_{b}(c) + \lambda_{c}(b') + \lambda_{b\circ b'}(c')\\ 
	                                & =b + \lambda_b(c) +\lambda_c(-b) -c\\
	                                & =b \circ c - c \circ b.
	\end{align*}
	Moreover, by (1) it follows that $b \circ c \circ b' \circ c' \in \ker \lambda$.
\end{proof}

\begin{lemma}
\label{lem:hardworkfactoriz}
Let $A$ be a brace such that $A=B+C$ is a factorization through left ideals  $B$ and $C$. If $B$ and $C$ are trivial braces, then: 
\begin{enumerate}
    \item $B*C$ and $C*B$ are strong left ideals of $A$,
    \item $B*C$ and $C*B$ are trivial braces, and 
    \item $A^{(2)} = C*B+B*C=B*C+C*B$.
\end{enumerate}
\end{lemma}

\begin{proof}
Since $C$ is a left ideal, it follows that $B*C \subseteq C$. 
Let $b,\beta \in B$ and $c, \gamma \in C$. As $C$ is trivial, it follows that 
\begin{align*}
    \lambda_{b \circ c} (\beta * \gamma) & =\lambda_b(\beta * \gamma)\\
                                         & =\lambda_b \lambda_{\beta} (\gamma) - \lambda_b(\gamma)\\
                                         & =\lambda_{b \circ \beta \circ b'}\lambda_b(\gamma) -\lambda_b(\gamma)\\
                                         & =(b \circ \beta \circ b') * \lambda_b(\gamma) \in B*C. 
\end{align*}
Hence $B*C$ is a left ideal and trivial as a brace. 

Let $a \in A$, $b \in B$ and $c \in C$. Write $a=b_1 + c_1$, with $b_1 \in B$ and $c_1 \in C$. Then
\begin{equation}\label{eq:7}
    \begin{aligned}
        a+(b*c)-a & =a + \lambda_b(c) - c - a\\
                  & =-(b*a) + b*(a+c)\\
                  & =-(b*(b_1+c_1)) + b* (b_1 + c_1+c).
    \end{aligned}
\end{equation}
As $B+C=C+B$, it follows that for any $\beta \in B$ and $\gamma \in C$, there exist $\beta_1\in B$ and $\gamma_1\in C$ such that
$\beta + \gamma = \gamma_1 + \beta_1$. Hence, for any $b \in B$ it holds that
\[
b*(\beta + \gamma)=b*(\gamma_1 + \beta_1)= b*\gamma_1 + \gamma_1 + b*\beta_1 - \gamma_1=b*\gamma_1 ,
\]
as $B$ is trivial. Applying this on \eqref{eq:7} it follows that $B*C$ is a normal subgroup of $(A,+)$.
This proves (1) and (2) for $B*C$. The proof for $C*B$ is similar.

Now we show that $A^{(2)} \subseteq C*B + B*C$. Let $b,b_1 \in B$ and $c,c_1 \in C$. Then
\begin{align*}
    (b\circ c)*(b_1+c_1) & =(b\circ c)*b_1 + b_1 + (b \circ c)*c_1 -b_1\\
                         & =\lambda_{b\circ c}(b_1) - b_1 + b_1 + b*(c*c_1) + c*c_1 + b*c_1 -b_1\\
                         & =\lambda_c(b_1) - b_1 +b_1 +b*c_1-b_1\\
                         & =c*b_1 + b_1 + b*c_1 -b_1 \in C*B+B*C.
\end{align*}
Clearly $C*B + B*C \subseteq A^{(2)}$ and thus $A^{(2)} = C*B+B*C=B*C + C*B$. 
\end{proof}


\begin{theorem} 
\label{thm:Ito_braces}
Let $A$ be a brace. If $A=B+C$ is a factorization through strong left ideals $B$ and $C$ that are trivial as  braces,
then $A$ is right nilpotent of class at most three. In particular, $A$ is meta-trivial.
\end{theorem}

\begin{proof}
By Lemma \ref{lem:hardworkfactoriz}, $B*C$ and $C*B$ are strong left ideals of $A$, and both are trivial as  braces. Furthermore, 
\[
A^{(2)} = B*C + C*B = (B*C) \circ (C*B).
\]
It rests to show that $A^{(2)}$ acts trivially on $A$. We first show that $B*C$ acts trivially on $A$. For that purpose, 
let $b\in B$, $c\in C$ and $a\in A$. Write $a=\beta+\gamma$, where $\beta\in B$ and $\gamma\in C$. Then
\[
(b*c)*(\beta+\gamma)=(b*c)*\beta+\beta+(b*c)*\gamma-\beta=(b*c)*\beta, 
\]
as $C$ is a trivial brace. By Lemma \ref{lem:calculations}(3), 
\[
(b\circ c - c\circ b) + \beta = (b\circ c - c \circ b) \circ \beta = (b + \lambda_b(c) - \lambda_c(b) - c) \circ \beta.
\]
Since $(B,+)$ is a normal subgroup of $(A,+)$, 
\[
b\circ c-c\circ b=b+\lambda_b(c)-\lambda_c(b)-c=\lambda_b(c)-c+b_1
\]
for some $b_1\in B$. By Lemma~\ref{lem:calculations}(2), 
\begin{align*}
    (b\circ c - c\circ b) + \beta & =(\lambda_b(c) - c + b_1) \circ \beta\\
                                  & =\lambda_b(c) - c + b_1 + (b*c)*\beta + \beta
\end{align*}
and therefore $(b*c)*\beta = 0$. Thus $B*C$ acts trivially on $A$. As $(C,+)$ also is a normal subgroup of $(A,+)$,
it follows by symmetry that $C*B$ acts trivially on $A$. Hence $A^{(2)}$ acts trivially on $A$.
\end{proof}

\begin{corollary}
    Let $A$ be a brace. Assume that $A= B + C$, where $B$ and $C$ are (not necessarily strong) left ideals,
    which are trivial as braces. Then $A$ has a meta-trivial ideal $I$ such that $A/I$ is a trivial brace.
\end{corollary}

\begin{proof}
    By Lemma \ref{lem:hardworkfactoriz}, the ideal $A^{(2)}$ has a factorization through the strong left ideals $B*C$ and $C*B$,
    which are trivial braces. By Theorem~\ref{thm:Ito_braces}, $A^{(2)}$ is meta-trivial and hence the claim follows.
\end{proof}

Theorem \ref{thm:Ito_braces} has application to involutive solutions. 

\begin{theorem} 
\label{thm:MP}
Let $(X,r)$ be an involutive non-degenerate (not necessarily finite) solution of the Yang--Baxter equation with $|X|\geq 2$.
If the brace of abelian type $\mathcal{G}(X,r)$ admits a factorization through left ideals, 
which are trivial as left braces, then $(X,r)$ is a
multipermutation solution of level at most three.
\end{theorem}

\begin{proof}
Let $A=\mathcal{G}(X,r)$ and $G=G(X,r)$. Then Theorem~\ref{thm:Ito_braces} 
yields $A^{(m)}=0$ for some $m\leq 3$. Because $G/\Soc(G)\cong A$
as left braces, we get $G^{(m)}\subseteq\Soc(G)$, and thus $G^{(m+1)}=0$. Hence $G$ is a right nilpotent left brace of class at most four and,
by \cite[Proposition 6]{MR3574204}, $(G,r_G)$ is a multipermutation solution of  level at most three. Therefore, by~\cite[Theorem 5.15]{MR3861714}, 
$(X,r)$ is a multipermutation solution of level at most three.
\end{proof}

This shows that properties of the involutive non-degenerate set-theoretic solution $(X,r)$ are not completely determined by the
group theory of the additive and multiplicative groups of the left brace $\mathcal{G}(X,r)$. 

\begin{example}
\label{exa:B(8,27)}
Let $X=\{1,2,3,4\}$ and $r(x,y)=(\sigma_x(y),\tau_y(x))$ be the irretractable involutive non-degenerate solution given by 
\begin{alignat*}{4}
    \sigma_1 & =(34),\qquad & \sigma_2 & =(1324),\qquad & \sigma_3 & =(1423),\qquad & \sigma_4 & =(12),\\
    \tau_1 & =(24),\qquad & \tau_2 & =(1432),\qquad & \tau_3 & =(1234),\qquad & \tau_4 & =(13).
\end{alignat*}
The associated left brace $\mathcal{G}(X,r)$ has additive group $C_2^3$ and multiplicative group $D_8$. Furthermore,
$\mathcal{G}(X,r)$ is not right nilpotent. Hence it is impossible to decompose the left brace $\mathcal{G}(X,r)$ as in Theorem~\ref{thm:MP}. 
\end{example}

\begin{example}
The left brace $B(8,26)$ has the same additive and multiplicative groups as the brace $\mathcal{G}(X,r)$ 
of Example~\ref{exa:B(8,27)} but it has a factorization as in Theorem~\ref{thm:MP}. 
This shows that $B(8,26)$ is right nilpotent. 
\end{example}



\section*{Exercises}


\begin{prob}
\label{prob:decomposable}
Let $A$ be a brace. If there exists a proper strong left ideal $I$, 
then $(A,r_A)$ is decomposable as $A = I \cup A\setminus I$. 
\end{prob}

\begin{prob}
\label{prob:Ito_relaxed}
Prove that the assumptions of Theorem \ref{thm:Ito_braces} cannot be relaxed. 
\end{prob}

\begin{prob}
\label{prob:Ito_version2}
Let $A$ be a non-zero brace that has a factorization $A=B+C$ through left ideals $B$ and $C$, where both are trivial as braces. If $B$ is a strong left ideal of $A$, then $B$ or $C$ contains a non-zero ideal $I$ of $A$ that acts trivially on $A$.
\end{prob}



\begin{prob}
\label{prob:mul_abelian}
Let $A$ be a brace with abelian multiplicative group. Prove that the 
the additive group of $A$ is meta-abelian. 
\end{prob}

\begin{prob}
\label{prob:mul_cyclic}
Let $A$ be a finite brace with cyclic multiplicative group. Prove that the 
the additive group of $A$ is supersolvable. 
\end{prob}

\section*{Open problems}

\begin{problem}
\label{problem:Byott}
    Let $A$ be a brace with solvable additive group. Is the 
    multiplicative group of $A$ solvable?
\end{problem}

\section*{Notes}

Theorem \ref{thm:Ito_braces} was proved by Jespers, Kubat, Antwerpen and Vendramin in \cite{MR4023387}. 
Exercises \ref{prob:Ito_relaxed} 
and \ref{prob:Ito_version2} 
also appear in there. 
One cannot expect a naive result similar to
that of Kegel--Wielandt in the context of braces. 

Theorem~\ref{thm:mul_nilpotent} was proved by Tsang and Qin in~\cite{MR4077413}. 
Exercises~\ref{prob:mul_abelian} and~\ref{prob:mul_cyclic} also appear in~\cite{MR4077413}.

Problem~\ref{problem:Byott} was formulated by Byott in~\cite{MR3425626}.