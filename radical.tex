\chapter{Radical rings}
\label{radical}

\section*{A}

We will consider rings possibly without identity. Thus  
a \textbf{ring} is an abelian group $R$ with an associative multiplication 
$(x,y)\mapsto xy$ such that $(x+y)z=xz+yz$ and $x(y+z)=xy+xz$ for all $x,y,z\in
R$. If there is an element $1\in R$ such that $x1=1x=x$ for all $x\in R$, we 
say that $R$ is a ring (or a unitary ring).  A \textbf{subring} $S$ of $R$ is an additive
subgroup of $R$ closed under multiplication. 

\begin{example}
	$2\Z=\{2m:m\in\Z\}$ is a ring.  
\end{example}

A \textbf{left ideal} (resp. \textbf{right ideal}) is a subring $I$ of $R$ such that 
$rI\subseteq I$ (resp. $Ir\subseteq I$) for all $r\in R$. An \textbf{ideal}
(also two-sided ideal) of $R$ is a subring $I$ of $R$ that is both a left and a right ideal of $R$.

\begin{example}
	If $I$ and $J$ are both ideals of $R$, then the sum $I+J=\{x+y:x\in I,y\in J\}$ and
	the intersection $I\cap J$ are both ideals of $R$. The product $IJ$, defined as the additive
	subgroup of $R$ generated by $\{xy:x\in I,y\in J\}$, is also an ideal of $R$. 
\end{example}

\begin{example}
	If $R$ is a ring, the set $Ra =\{xa: x\in R\}$ is a left ideal
	of $R$. Similarly, the set $aR =\{ax: x\in R\}$ is a right ideal of $R$. The set $RaR$, which is
	defined as the additive subgroup of $R$ generated by $\{xay: x, y\in R\}$, is a
	ideal of $R$.
\end{example}

\begin{example}
	Ir $R$ is a unitary ring, then $Ra$ is the left ideal generated by $a$, $aR$ is
	the right ideal generated by $a$ and $RaR$ is the ideal generated by $a$. 
	If $R$ is not unitary, the left ideal generated by $a$ is $Ra+\Z a$,
	the right ideal generated by $a$ is $aR+\Z a$ and the ideal generated by 
	$a$ is $RaR+Ra+aR+\Z a$.
\end{example}

A ring $R$ is said to be \textbf{simple} if $R^2\ne\{0\}$ and the only ideals of 
$R$ are $0$ and $R$. The condition $R^2\ne\{0\}$ is trivially satisfied in the case of rings
with identity, as $1\in R^2$. 

\begin{example}
	Division rings are simple.
\end{example}

Let $S$ be a unitary ring. Recall that $M_n(S)$ is the ring of $n\times n$ square matrices 
with entries in $S$.  If $A=(a_{ij})\in M_n(S)$ y $E_{ij}$ is the matrix
such that $(E_{ij})_{kl}=\delta_{ik}\delta_{jl}$, then
\begin{equation}
	\label{eq:trick}
E_{ij}AE_{kl}=a_{jk}E_{il}
\end{equation}
for all $i,j,k,l\in\{1,\dots,n\}$. 

\begin{exercise}
	If $D$ is a division ring, then $M_n(D)$ is simple. 
\end{exercise}

Let $R$ be a ring. A left $R$-module (or module, for short)  
is an abelian group $M$ together with a map $R\times M\to M$, $(r,m)\mapsto rm$, such that
\begin{align*}
	&(r+s)m=rm+sm, &&
	r(m+n)=rm+rs, && r(sm)=(rs)m    
\end{align*}
for all $r,s\in R$, $m,n\in M$.  If $R$ has an identity 
$1$ and $1m=m$ holds for all $m\in M$, the module $M$ is said to be 
\textbf{unitary}.  If $M$ is a unitary module, then $M=RM\ne\{0\}$.


The module $M$ is said to be 
\textbf{simple} if $RM\ne\{0\}$ and the only submodules of $M$ are $0$ and $M$.
If $M$ is a simple module, then $M\ne\{0\}$.

%\begin{remark}
%	Si $R$ es unitario y $M$ es un módulo simple, entonces $M$ es unitario.
%\end{remark}

\begin{lemma}
	\label{lemma:simple}
	Let $M$ be a non-zero module. Then $M$ is simple if and only if $M=Rm$
	for all $0\ne m\in M$.
\end{lemma}

\begin{proof}
	Assume that $M$ is simple.  Let $m\ne 0$. Since $Rm$ is a submodule of the simple 
	module $M$, either $Rm=\{0\}$ or $Rm=M$.  Let $N=\{n\in M:Rn=\{0\}\}$. Since $N$ is a 
	submodule of $M$ and $RM\ne\{0\}$, $N=\{0\}$. Therefore $Rm=M$, as $m\ne0$.
	Now assume that $M=Rm$ for all $m\ne0$. Let $L$ be a non-zero submodule of 
	$M$ and let $0\ne x\in L$. Then $M=L$, as $M=Rx\subseteq L$. 
\end{proof} 

\begin{example}
	Let $D$ be a division ring and let $V$ be a non-zero vector space (over $D$). If 
	$R=\End_D(V)$, then $V$ is a simple $R$-módulo with $fv=f(v)$, $f\in R$.
	$v\in V$. 
% 	Para ver que $V$ es simple como $R$-módulo basta ver que $Rv=V$ para todo
% 	$v\ne0$. Sean $v,w\in V$, $v\ne0$.  Al completar $v\ne0$ a una base de $V$,
% 	vemos que existe $f\in R$ tal que $f(v)=w$. Luego $V$ es simple.
\end{example}

\begin{example}
	\label{exa:I_k}
	Let $n\geq2$.  If $D$ is a division ring and $R=M_n(D)$, then each 
	\[
	I_k=\{ (a_{ij})\in R:a_{ij}=0\text{ for $j\ne k$}\}
	\]
	is an $R$-module isomorphic to $D^n$. 
	Thus $M_{n}(D)$ is a simple ring that is not a simple $M_n(D)$-module.
\end{example}

A left ideal $L$ of a ring $R$ is said to be \textbf{minimal} if $L\ne\{0\}$ and 
$L$ does not strictly contain other left ideals of $R$. Similarly one defines
right minimal ideals and minimal ideals. 

\begin{example}
	Let $D$ be a division ring and let $R=M_n(D)$. Then $L=RE_{11}$ 
	is a minimal left ideal.
\end{example}

\begin{example}
	Let $L$ be a non-zero left ideal. If $RL\ne\{0\}$, then
	$L$ is minimal if and only if $L$ is a simple $R$-module. 
\end{example}

A left (resp. right) ideal $L$ of $R$ is said to be \textbf{regular} if
there exists $e\in R$ such that $r-re\in L$ (resp.  $r-er\in L$) for all $r\in R$. 
If $R$ is a ring with identity, every left (or right) ideal is regular. 
A left (resp. right) ideal $I$ of $R$ is said to be \textbf{maximal} if $I\ne M$ and $I$ is not properly contained 
in any other left (resp. right) ideal of $R$. 
A standard
application of Zorn's lemma proves that every unitary ring contains a maximal left (or right) ideal.  
Similarly one defines maximal ideals. 

% \begin{exercise}
% Prove that every ring with identity contains a maximal ideal.
% \end{exercise}

\begin{proposition}
	\label{proposition:R/I}
	Let $R$ be a ring and $M$ be a module. Then $M$ is simple if and only if
	$M\simeq R/I$ for some maximal regular left ideal $I$. 	
\end{proposition}

\begin{proof}
	Assume that $M$ is simple. Then $M=Rm$ for some $m\ne0$ by 
	Lemma~\ref{lemma:simple}. The map $\phi\colon R\to M$, $r\mapsto rm$, 
	is an epimorphism of $R$-modules, so the first isomorphism theorem implies that 
	$M\simeq R/\ker\phi$. 
	
	We claim that $I=\ker\phi$ is a maximal ideal. The correspondence theorem 
	and the simpllicity of $M$ imply that $I$ is a maximal ideal (because each left ideal $J$ such that 
	$I\subseteq J$ yields a submodule of $R/I$).

	We claim that $I$ is regular. Since $M=Rm$, there exists $e\in R$ such that $m=em$. If
	$r\in R$, then $r-re\in I$ since 
	$\phi(r-re)=\phi(r)-\phi(re)=rm-r(em)=0$.

	Supongamos ahora que $L$ es maximal y regular.  Por el teorema de la
	correspondencia, $R/L$ no tiene submódulos propios no nulos. Veamos
	entonces que $R(R/L)\ne0$. Si $R(R/L)=0$ y $r\in R$, entonces, como $L$ es
	regular, $r-re\in L$ y luego $r\in L$ pues 
	\[
	0=r(e+I)=re+I=r+I,
	\]
	una contradicción a la maximalidad de $L$.
\end{proof}


%\section{Nilpotencia}
%
%Recordemos que si $I$ es un ideal de un anillo $R$, se define $I^n$ como el
%subgrupo aditivo generado por el conjunto $\{y_1\dots y_n:y_j\in I\}$. 
%
%\begin{definition}
%	Un ideal $I$ de un anillo $R$ se dice \textbf{nilpotente} si $I^n=0$ para
%	algún $n\in\N$.
%\end{definition}
%
%Recordemos que un elemento $x$ de un anillo $R$ se dice \textbf{nilpotente} si
%existe $n\in\N$ tal que $x^n=0$. 
%
%\begin{definition}
%	Un ideal $I$ de un anillo $R$ se dice \textbf{nil} si todo elemento de $I$
%	es nilpotente.
%\end{definition}
%
%\begin{remark}
%	Un ideal nilpotente es nil. 
%\end{remark}
%
%\begin{example}
%	Sea $R=\C[x_1,x_2,\dots]/(x_1,x_2^2,x_3^3,\dots)$. El ideal
%	$I=(x_1,x_2,x_3,\dots)$ es nil en $R$ pues está generado por elementos
%	nilpotentes pero no es nilpotente. Si lo fuera, existiría $k\in\N$ tal que
%	$I^k=0$, y luego $x_i^k=0$ para todo $i$, una contradicción pues
%	$x_{k+1}^k\ne0$. 	
%\end{example}
%
%% example 2.7 del libro de springer
%% problema de kothe 
%
%\begin{lemma}
%	Si $I$ y $J$ son ideales nilpotentes, $I+J$ es nilpotente.
%\end{lemma}
%
%\begin{proof}
%	
%\end{proof}
%
%Un ideal $N$ de un anillo $R$ se dice \textbf{maximal-nilpotente} si $N$ es
%nilpotente y no está propiamente contenido en ningún ideal nilpotente de $R$.
%
%\begin{lemma}
%	Si el anillo $R$ contiene un ideal maximal-nilpotente $N$ entonces todo
%	ideal nilpotente está contenido en $N$.
%\end{lemma}
%
%\begin{proof}
%	
%\end{proof}
%
%\section{Anillos primos y semiprimos}
%
%\index{Dominio}
%Recordemos que un anillo $R$ se dice un \textbf{dominio} si para todo $a,b\in
%R$ tales que $ab=0$ se tiene $a=0$ o $b=0$.
%Una generalización al caso no conmutativo es la siguiente:
%
%\begin{definition}
%	\index{Anillo!primo}
%	Un anillo $R$ se dice \textbf{primo} si para todo $a,b\in R$ tales que
%	$aRb=0$ se tiene $a=0$ o $b=0$.
%\end{definition}
%
%\begin{lemma}
%	Sea $R$ un anillo. Las siguientes propiedades son equivalentes:
%	\begin{enumerate}
%		\item $R$ es primo.
%		\item Si $I,J\subseteq R$ son ideales a izquierda tales que $IJ=0$
%			entonces $I=0$ o $J=0$.
%%		\item Si $I,J\subseteq R$ son ideales a derecha tales que $IJ=0$ 
%%			entonces $I=0$ o $J=0$.
%		\item Si $I,J\subseteq R$ son ideales tales que $IJ=0$ entonces $I=0$ o
%			$J=0$.
%	\end{enumerate}
%\end{lemma}
%
%\begin{proof}
%	Vamos a demostrar que $(1)\implies(2)\implies(4)\implies(1)$. 
%	La implicación $(2)\implies(4)$ es trivial.
%
%	Veamos que $(4)\implies(1)$. Sean $a,b\in R$ tales que $aRb=0$.
%	Como entonces $(RaR)(RbR)=R(aRb)R=0$, $RaR=0$ o bien $RbR=0$. Supongamos
%	sin pérdida de generalidad que $RaR=0$. Entonces $Ra$ y $aR$ son ideales
%	biláteros tales que $(Ra)R=R(aR)=0$. Al aplicar la hipótesis, $Ra=aR=0$.
%	Como $\Z a$ es un ideal de $R$ tal que $(\Z a)R=0$, se concluye al aplicar
%	la hipótesis que $a=0$.
%
%	Veamos que $(1)\implies(2)$. Supongamos que $J\ne 0$, sea $y\in
%	J\setminus\{0\}$ y sea $x\in I$. Como 
%	$xRy\subseteq IRJ=I(RJ)\subseteq IJ=0$, 
%	se concluye, al usar la hipótesis, que $x=0$. 
%\end{proof}
%
%\begin{proposition}
%	Un anillo conmutativo es primo si y sólo si es un dominio. 
%\end{proposition}
%
%\begin{proof}
%	Supongamos que $R$ es un anillo primo. Si $a,b\in R$ son tales que $ab=0$
%	entonces $aRb=(ab)R=0$ y luego $a=0$ o bien $b=0$. Supongamos ahora que $R$
%	es un dominio. Si $a,b\in R$ son tales que $aRb=0$ entonces $(ab)R=0$ y
%	luego $a=0$ o bien $b=0$ pues $ab=0$. 
%\end{proof}
%
%\begin{definition}
%	\index{Anillo!semiprimo}
%	Un anillo $R$ se dice \textbf{semiprimo} si para todo $a\in R$ tal que
%	$aRa=0$ se tiene $a=0$.
%\end{definition}
%
%\begin{lemma}
%	Sea $R$ un anillo. Las siguientes aifrmaciones son equivalentes:
%	\begin{enumerate}
%		\item $R$ es semiprimo.
%		\item Si $I$ es un ideal a izquierda tal que $I^2=0$ entonces $I=0$.
%		\item Si $I$ es un ideal tal que $I^2=0$ entonces $I=0$.
%		\item $R$ no tiene ideales nilpotentes no nulos.
%	\end{enumerate}
%\end{lemma}
%
%\begin{proof}
%	Primero vamos a demostrar que $(1)\implies(2)\implies(3)\implies(1)$ 
%
%	La implicación $(5)\implies(4)$ es trivial.	
%	Demostremos entonces que
%	$(4)\implies(5)$. Sea $I$ un ideal nilpotente no nulo y sea $n\in\N$ el
%	mínimo tal que $I^n=0$. Como $(I^{n-1})^2=0$, por hipótesis se tiene que
%	$I^{n-1}=0$, una contradicción.
%\end{proof}
%

We will now discuss primitive rings. 

Let $R$ be a ring and $M$ be a left $R$-module. For a 
subset $N\subseteq M$
we define the \textbf{annihilator} of $N$ as the subset 
\[
\Ann_R(N)=\{r\in R:rn=0\;\forall n\in N\}.
\]

\begin{example}
	$\Ann_{\Z}(\Z/n)=n\Z$.
\end{example}

The following exercise is standard. 

\begin{exercise}
    Let $R$ be a ring and $M$ be a module. If $N\subseteq M$ is a subset, then 
	$\Ann_R(N)$ is a left ideal of $R$. If $N\subseteq M$ is a submodule of $R$, then 
	$\Ann_R(N)$ is an ideal of $R$. 
\end{exercise}

% \begin{lemma}
% 	\label{lemma:Ann}
% 	Let $R$ be a ring and $M$ be a module. If $N\subseteq M$ is a subset, then 
% 	$\Ann_R(N)$ is a left ideal of $R$. If $N\subseteq M$ is a submodule of $R$, then 
% 	$\Ann_R(N)$ is an ideal of $R$. 
% \end{lemma}

% \begin{proof}
% 	We left as an exercise to prove that $\Ann_R(N)$ is an additive subgroup of $R$. Then $\Ann_R(N)$
% 	is a left ideal, as $R\Ann_R(N)\subseteq\Ann_R(N)$. Indeed, if $r\in R$,
% 	$s\in\Ann_R(N)$ and $n\in N$, then $(rs)n=r(sn)=r0=0$. 
	
% 	If $N$ is a submodule, $\Ann_R(N)R\subseteq\Ann_R(N)$ since if 
% 	$s\in\Ann_R(N)$, $r\in R$ and $n\in N$, $rn\in\Ann_R(N)$, then
% 	$(sr)n=s(rn)=0$.
% \end{proof}


A module $M$ is said to be \textbf{faithful} if $\Ann_R(M)=\{0\}$. 

\begin{example}
	If $K$ is a field, then $K^n$ is a faithful unitary $M_n(K)$-module.
\end{example}

\begin{example}
	If $V$ is vector space over a field $K$, then $V$ is faithful unitary $\End_K(V)$-module.
\end{example}

\index{Ring!primitive}
A ring $R$ is said to be \textbf{primitive} if there exists a faithful simple $R$-módulo. Since 
we are considering left modules, our definition of primitive rings is that of left primitive rings. 
By convention, a primitive ring
will always mean a left primitive ring. 
The use 
of right modules yields to the notion of right primitive rings.  

\begin{proposition}
	\label{proposition:simple=>prim}
	If $R$ is a simple unitary ring, then $R$ is primitive. 
\end{proposition}

\begin{proof}
	Since $R$ is unitary, there exists a maximal left ideal $I$ and, moreover, $R$ is regular.
	By Proposition~\ref{proposition:R/I}, $R/I$ is a simple $R$-module. 
	Since $\Ann_R(R/I)$ is an ideal of $R$ and $R$ is simple, either $\Ann_R(R/I)\in\{0\}$ or 
	$\Ann_R(R/I)=R$. Moreover, since 
	$1\not\in\Ann(R/I)$, it follows that 
	$\Ann_R(R/I)=\{0\}$. 
\end{proof}

\begin{proposition}
	\label{proposition:prim+conm=cuerpo}
	If $R$ is a commutative ring, then $R$ is primitive if and only if $R$ is a field. 
\end{proposition}

\begin{proof}
	If $R$ is a field, then $R$ is primitive because it is a unitary simple ring, see  
	Proposition~\ref{proposition:simple=>prim}. If $R$ is a primitive commutative ring, Proposition~\ref{proposition:R/I} implies that there exists a maximal regular ideal $I$
	such that  
	$R/I$ is a faithful simple $R$-module. 
	Since $I\subseteq \Ann_R(R/I)=\{0\}$ and $I$ is regular, there exists $e\in R$ such that 
	$r=re=er$. Therefore $R$ is a unitary commutative ring. Since $I=\{0\}$ is a maximal ideal, 
	$R$ is a field. 
\end{proof}

\begin{example}
	The ring $\Z$ is not primitive. 
\end{example}

\index{Ideal!primitive}
An ideal $P$ of a ring $R$ is said to be \textbf{primitive} if $P=\Ann_R(M)$
for some simple $R$-module $M$. 

\begin{lemma}
	\label{lemma:primitivo}
	Let $R$ be a ring and $P$ be an ideal of $R$. Then $P$ is primitive if and only if 
	$R/P$ is a primitive ring.
\end{lemma}

\begin{proof}
	Assume that $P=\Ann_R(M)$ for some $R$-module $M$. Then $M$ is a simple 
	$R/P$-module with $(r+P)m=rm$, $r\in R$, $m\in M$. This is well-defined, as 
	$P=\Ann_R(M)$. Since $M$ is a simple $R$-module, it follows that $M$ is 
	a simple $R/P$-module. Moreover, $\Ann_{R/P}M=\{0\}$. Indeed, if 
	$(r+P)M=0$, then $r\in\Ann_RM=P$ and hence $r+P=P$.

	Assume now that $R/P$ is primitive. Let $M$ be a faithful simple $R/P$-module. 
	Then $rm=(r+P)m$, $r\in R$,
	$m\in M$, turns $M$ into an $R$-module. It follows that $M$ is simple and that $P=\Ann_R(M)$. 
\end{proof}

%\begin{example}
%	Si $I$ es un ideal maximal de un anillo unitario $R$, entonces $I$ es
%	primitivo. Como $I$ es ideal maximal y regular (pues $1\in R$), el cociente
%	$R/I$ es un anillo unitario simple y luego $R/I$ es primitivo por la
%	proposición~\ref{proposition:simple=>prim}. 
%\end{example}

%\begin{example}
%	Si $I$ es un ideal primitivo de un anillo conmutativo $R$, entonces $I$ es
%	maximal pues $R/I$ es un cuerpo (por ser primitivo y conmutativo), ver
%	proposición~\ref{proposition:prim+conm=cuerpo}.
%\end{example}

\begin{example}
	Let $R_1,\dots,R_n$ be primitive ring and $R=R_1\times\cdots\times
	R_n$. Then each $P_i=R_1\times\cdots\times R_{i-1}\times\{0\}\times
	R_{i+1}\times\cdots\times R_n$ is a primitive ideal of $R$ since 
	$R/P_i\simeq R_i$.
\end{example}

%Recordemos que un ideal a izquierda $L$ de $R$ se dice \textbf{minimal} si
%$L\ne0$ y $L$ no contiene propiamenete a otros ideales a izquierda no nulos de
%$R$.
%
%\begin{example}
%	Sea $L$ un ideal a izquierda de $R$ tal que $RL\ne0$. Entonoces $L$ es
%	simple si y sólo si $L$ es minimal.
%\end{example}

\begin{lemma}
	\label{lemma:maxprim}
	Let $R$ be a ring. Si $P$ es un ideal primitivo, existe un ideal a
	izquierda $L$ maximal tal que $P=\{x\in R:xR\subseteq L\}$.
	Recíprocamente, si $L$ es un ideal a izquierda maximal y regular, entonces
	$\{x\in R:xR\subseteq L\}$ es un ideal primitivo.
\end{lemma}

\begin{proof}
	Assume that $P=\Ann_R(M)$ for some simple $R$-module $M$. By
	Proposition~\ref{proposition:R/I}, there exists a regular maximal 
	left ideal 
	$L$ such that $M\simeq R/L$. Then $P=\Ann_R(R/L)=\{x\in
	R:xR\subseteq L\}$. 

	Conversely, let $L$ a regular maximal left ideal.By
	Proposition~\ref{proposition:R/I}, $R/L$ is a simple $R$-module simple. Then
	\[
	\Ann_R(R/L)=\{x\in R:xR\subseteq L\}
	\]
	if a primitive ideal.
\end{proof}

%\begin{remark}
%	Una consecuencia trivial del lema~\ref{lemma:maxprim} es la siguiente: en
%	un anillo unitario, todo ideal a izquierda maximal contiene un ideal
%	primitivo.
%\end{remark}

\begin{proposition}
    Maximal ideals of unitary rings are primitive.  
\end{proposition}

\begin{proof}
	Let $R$ be a ring with identity and $M$ be a maximal ideal of $R$. Then 
	$R/M$ is a simple unitary ring by 
	Proposition~\ref{proposition:R/I}. Then $R/M$ is primitive by
	Proposition~\ref{proposition:simple=>prim}. By lema~\ref{lemma:primitivo}, 
	$M$ is primitive. 
\end{proof}

\begin{exercise}
	Prove that every primitive ideal of a commutative ring is maximal.
\end{exercise}

\begin{exercise}
    Prove that $M_n(R)$ is primitive if and only if $R$ is primitive.
\end{exercise}

% Si $P$ es primitivo, entonces $R/P$ es un cuerpo(por ser primitivo y conmutativo) y luego $P$ es maximal

Let us discuss the Jacobson radical and radical rings. 

Let $R$ be a ring. The \textbf{Jacobson radical} $J(R)$
is the intersection of all the annihilators of simple left $R$-modules. If $R$ does not
have simple left $R$-modules, then $J(R)=R$. From the definition it follows
that $J(R)$ is an ideal. Moreover, 
	\[
		J(R)=\bigcap\{P:\text{$P$ left primitive ideal}\}.
	\]

If $I$ is an ideal of $R$ and $n\in\N$, $I^n$ is the additive subgroup of $R$ 
generated by the set $\{y_1\dots y_n:y_j\in I\}$. An ideal $I$ of $R$ is \textbf{nilpotent} 
if $I^n=\{0\}$ for some $n\in\N$. Similarly one defines right or left nil ideals. 
Note that an ideal $I$ is nilpotent if and only if there exists $n\in\N$ such that 
$x_1x_2\cdots x_n=0$ for all $x_1,\dots,x_n\in I$.  

An element $x$ of a ring is said to be \textbf{nil} (or nilpotent) if $x^n=0$ for some $n\in\N$. 
An ideal $I$ of a ring is said to be \text{nil} if every element of $I$ is nil. 
Every nilpotent ideal is nil, as $I^n=0$ implies $x^n=0$ for all 
$x\in I$.

\begin{example}
	Let $R=\C[x_1,x_2,\dots]/(x_1,x_2^2,x_3^3,\dots)$. The ideal 
	$I=(x_1,x_2,x_3,\dots)$ is nil in $R$, as it is generated by nilpotent element. However, it is not nilpotente. Indeed, if $I$ is nilpotent, then there exists $k\in\N$ such that 
	$I^k=0$ and hence $x_i^k=0$ for all $i$, a contradiction since 
	$x_{k+1}^k\ne0$. 	
\end{example}

\begin{proposition}
	\label{pro:nilJ}
	Let $R$ be a ring. Then every nil left ideal (resp. right ideal) is contained in $J(R)$.
\end{proposition}

\begin{proof}
	Assume that there is a nil left ideal (resp. right ideal) $I$ such that 
	$I\not\subseteq J(R)$. There exists a simple $R$-module $M$ such that 
	$n=xm\ne 0$ for some $x\in I$ and some $m\in M$. Since $M$ is simple,
	$Rn=M$ and hence there exists $r\in R$ such that 
	\[
	(rx)m=r(xm)=rn=m\quad\text{(resp.
	$(xr)n=x(rn)=xm=n$).}
	\]
	Thus $(rx)^km=m$ (resp. $(xr)^kn=n$) for all 
	$k\geq1$, a contradiction since $rx\in I$ (resp. $xr\in I$) is a nilpotent element. 
\end{proof}

Let $R$ be a ring. An element $a\in R$ is said to be 
\textbf{left quasi-regular} if there exists $r\in R$ such that $r+a+ra=0$. Similarly, 
$a$ is said to be \textbf{right quasi-regular} if there exists $r\in R$ such that $a+r+ar=0$. 

\begin{exercise}
	\label{exercise:circ}
	Let $R$ be a ring. Prove that $R\times R\to R$,
	$(r,s)\mapsto r\circ s=r+s+rs$, is an associative operation with neutral element $0$.
\end{exercise}

\begin{exercise}
	Let $R=\Z/3=\{0,1,2\}$. Compute the multiplication table with respect to the circle 
 	operation given by the previous exercise.  
 	%is then 
% 	\begin{table}[ht]
% 		\centering
% 		\begin{tabular}{c|ccc}
% 			$\circ$ & 0 & 1 & 2\tabularnewline
% 			\hline
% 			0 & 0 & 1 & 2\tabularnewline
% 			1 & 1 & 0 & 2\tabularnewline
% 			2 & 2 & 2 & 2\tabularnewline
% 		\end{tabular}
% 		\caption{The multiplication table of the radical ring $\Z/3$.}
% 	\end{table}
\end{exercise}

If $R$ is unitary, an element $x\in R$ is left quasi-regular (resp. right quasi-regular)
if and only if $1+x$ is left invertible (resp. right invertible). In fact, 
if $r\in R$ is such that $r+x+rx=0$, then $(1+r)(1+x)=1+r+x+rx=1$.
Conversely, if there exists $y\in R$ such that $y(1+x)=1$, then  
\[
(y-1)\circ x=y-1+x+(y-1)x=0.
\]

\begin{example}
	If $x\in R$ is a nilpotent element, then $y=\sum_{n\geq1}x^n\in R$ is quasi-regular. 
	En efecto, si existe $N$ tal que $x^N=0$, la suma que
	define al elemento $y$ es finita y cumple que $y+(-x)+y(-x)=0$.  
\end{example}

A left ideal $I$ of $R$ is said to be 
\textbf{left quasi-regular} (resp. right quasi-regular) if every element of $I$ is
left quasi-regular (resp. right quasi-regular). A left ideal 
is said to be \textbf{quasi-regular} if it is left and right quasi-regular. Similarly 
one defines right quasi-regular ideals and quasi-regular ideals. 

\begin{lemma}
	\label{lemma:casiregular}
	Let $I$ be a left ideal of $R$. If $I$ is left quasi-regular, then 
	$I$ is quasi-regular.
\end{lemma}

\begin{proof}
	Let $x\in I$. Let us prove that $x$ is right quasi-regular. Since $I$ is
	left quasi-regular, there exists $r\in R$ such that $r\circ x=r+x+rx=0$. Since 
	$r=-x-rx\in I$, there exists $s\in R$ tal que $s\circ
	r=s+r+sr=0$. Then $s$ is right quasi-regular and  
	\[
	x=0\circ x=(s\circ r)\circ x=s\circ (r\circ x)=s\circ 0=s.\qedhere
	\]
\end{proof}

\index{Lemma!Zorn}
Let $(A,\leq)$ be a partially order set, this means that $A$ is a set together with a 
reflexive, transitive and anti-symmetric binary relation
$R$ en $A\times A$, where $a\leq b$ if and only if $(a,b)\in R$. 
Recall that the relation is reflexive if $a\leq a$ for all $a\in A$, the relation is transitive if 
$a\leq b$ and $b\leq c$ imply that 
$a\leq c$ and the relation is anti-symmetric if $a\leq b$ and $b\leq a$ imply $a=b$.

The elements $a,b\in A$ are said to be \textbf{comparable} if $a\leq b$ or $b\leq
a$. An element $a\in A$ is said to be \textbf{maximal} if 
$c\leq a$ 
for all $c\in A$
that is comparable with $a$. 
An \textbf{upper bound} for a non-empty subset $B\subseteq A$ is an element $d\in
A$ such that $b\leq d$ for all $b\in B$. A \textbf{chain} in $A$ is a subset 
$B$ such that every pair of elements of $B$ are comparable. 
\textbf{Zorn's lemma} states the following property: 
\begin{quote}
If $A$ is a non-empty partially ordered set such that every chain in 
$A$ contains an upper bound in $A$, then $A$ contains a maximal element. 
\end{quote}

Our application of Zorn's lemma:

\begin{lemma}
	\label{lemma:maxreg}
	Let $R$ be a ring and $x\in R$ be an element that is not left quasi-regular Then there
	exists a maximal left ideal $M$ such that 
	$x\not\in M$. Moreover, $R/M$ is a simple $R$-module and  
	$x\not\in\Ann_R(R/M)$.
\end{lemma}

\begin{proof}
	Let $T=\{r+rx:r\in R\}$. A straightforward calculation shows that $T$ is a left ideal of 
	$R$ such that $x\not\in T$ (if $x\in T$, then $r+rx=-x$ for some 
	$r\in R$, a contradiction since $x$ is not left quasi-regular). 

	The only left ideal of $R$ contanining 
	$T\cup\{x\}$ is $R$. Indeed, if there exists a left ideal $U$ containing $T$, then 
    $x\not\in U$, since otherwise every $r\in R$ could be written as 
	$r=(r+rx)+r(-x)\in U$. 

	Let $\mathcal{S}$ be the set of proper left ideals of $R$ containing 
	$T$ partially ordered by inclusion. If $\{K_i:i\in I\}$ is a chain in 
	$\mathcal{S}$, then $K=\cup_{i\in I}K_i$ is an upper bound for the chain 
	($K$ is a proper, as $x\not\in K$). Zorn's lemma implies that 
	$\mathcal{S}$ admits a maximal element $M$. Thus $M$
	is a maximal left ideal such that $x\not\in M$. Moreover, $M$ is regular
	since $r+r(-x)\in T\subseteq M$ for all $r\in R$. Therefore $R/M$ is a simple 
	$R$-module by Proposition~\ref{proposition:R/I}. Since $x(x+M)\ne
	0$ (if $x^2\in M$, then  $x\in M$, as $x+x^2\in
	T\subseteq M$), it follows that $x\not\in\Ann_R(R/M)$.
\end{proof}

If $x\in R$ is not left quasi-regular, Lemma~\ref{lemma:maxreg} implies that there exists 
a simple $R$-module $M$ such $x\not\in\Ann_R(M)$. Thus 
$x\not\in J(R)$.

\begin{theorem}
	\label{thm:casireg_eq}
	Let $R$ be a ring and $x\in R$. The following statements are equivalent: 
	\begin{enumerate}
		\item The left ideal generated by $x$ is quasi-regular.
		\item $Rx$ is quasi-regular.
		\item $x\in J(R)$.
	\end{enumerate}
\end{theorem}

\begin{proof}
	The implication $(1)\implies(2)$ is trivial, as $Rx$ is included in the left ideal 
	generated by $x$.  
	
	We now prove $(2)\implies(3)$. If
	$x\not\in J(R)$, then Lemma~\ref{lemma:maxreg} implies that there exists a simple 
	$R$-module $M$ such that $xm\ne 0$ for some $m\in M$. The simplicity of $M$ implies
	that $R(xm)=M$. Thus there exists $r\in R$ such that $rxm=-m$. There is an element 
	$s\in R$ such that $s+rx+s(rx)=0$ and hence 
	\[
	-m=rxm=(-s-srx)m=-sm+sm=0,
	\]
	a contradiction. 
	
	Finally, to prove $(3)\implies(1)$ it is enough to note that 
	$x$ is left quasi-regular. Thus the left ideal generated by 
	$x$ is quasi-regular by Lemma~\ref{lemma:casiregular}.
\end{proof}

The theorem immediately implies the following corollary. 

\begin{corollary}
	If $R$ is a ring, then $J(R)$ if a quasi-regular ideal that contains every 
	left quasi-regular ideal. 
\end{corollary}

The following result is somewhat what we all had in mind. 

% \begin{proof}
% 	Es consecuencia inmediata del teorema~\ref{thm:casireg_eq}. 
%%	Como \[
%%	J(R)=\bigcap\{\Ann_R(R/I):I\text{ ideal a izquierda maximal y
%%	regular}\}
%%	\]
%%	por el teorema~\ref{thm:J(R)}, 
%%	todo $x\in J(R)$ es casi-regular gracias al
%%	lema~\ref{lemma:K}. Si $I$ es un ideal casi-regular a izquierda, entonces
%%	$I\subseteq J(R)$ por el lema~\ref{lemma:JsupsetCR}.
%\end{proof}

%\begin{lemma}
%%	\label{lemma:maxreg}
%	Sea $R$ un anillo y sea $I\ne R$ un ideal a izquierda regular. Entonces $I$
%	está contenido en algún ideal a izquierda maximal y regular.
%\end{lemma}
%
%\begin{proof}
%	
%\end{proof}

%\begin{lemma}
%	\label{lemma:K}
%	Sean $R$ un anillo y $K=\bigcap\{I:\text{$I$ ideal a izquierda maximal y
%	regular}\}$.  Entonces $K$ es un ideal a izquierda casi-regular.
%\end{lemma}
%
%\begin{proof}
%	Gracias al lema~\ref{lemma:casiregular}, basta ver que $K$ es casi-regular
%	a izquierda.  Sea $a\in K$ y sea $T=\{r+ra:r\in R\}$.  Es claro que $T$ es
%	un ideal a izquierda regular con $e=-a$. Como $a\in K$, existe $s\in R$ tal
%	que $s+a+sa=0$. Entonces, como $-a=s+sa$, 
%	\[
%	r+re=r+r(-a)=r+r(s+sa)=r+rs+rsa\in r+T
%	\]
%	para todo $r\in R$. 
%	Si $T\ne R$, por el lema~\ref{lemma:maxreg} existe un ideal a izquierda
%	$J$, maximal y regular.  Como $a\in K\subseteq J$, $J$ es ideal a izquierda
%	y $r+ra\in T\subseteq J$ para todo $r\in R$, se concluye que $R=J$, una
%	contradicción.  Luego $T=R$ y entonces existe $r\in R$ tal que $r+ra=-a$.
%	Esto implica que $a$ es casi-regular a izquierda. 
%\end{proof}

%\begin{lemma}
%	\label{lemma:JsupsetCR}
%	Sea $R$ un anillo que admite un $R$-módulo simple. Si $I$ es un ideal a
%	izquierda casi-regular a izquierda, $I\subseteq J(R)$.
%\end{lemma}
%
%\begin{proof}
%	Supongamos que $I\not\subseteq J(R)$. Existe entonces un $R$-módulo simple
%	$N$ tal que $IN\ne 0$. Entonces $In\ne 0$ para algún $0\ne n\in N$. Como
%	$I$ es un ideal a izquierda, $In\subseteq N$ es un submódulo no nulo del
%	simple $N$. Luego $In=N$. Existe entonces $x\in I$ tal que $xn=-n$. Como
%	$I$ es casi-regular a izquierda, existe $r\in R$ tal que $r+x+rx=0$.
%	Entonces
%	\[
%		0=0n=(r+x+rx)n=rn+xn+rxn=rn-n-rn=-n,
%	\]
%	una contradicción.
%\end{proof}

\begin{theorem}
	\label{thm:J(R)}
	Let $R$ be a ring such that $J(R)\ne R$. Then 
	\begin{align*}
		J(R)&=\bigcap\{I:\text{$I$ regular maximal left ideal of $R$}\}.
	\end{align*}
\end{theorem}

\begin{proof}
    We only prove the non-trivial inclusion. 
	Let 
	\[
	K=\bigcap\{I:\text{$I$ regular maximal left ideal of $R$}\}.
	\]
	By
	Proposition~\ref{proposition:R/I}, 
	\[
		J(R)=\bigcap\{\Ann_R(R/I):I\text{ regular maximal left ideal of $R$}\}.
	\]
	Let $I$ be a regular maximal left ideal. If $r\in J(R)\subseteq
	\Ann_R(R/I)$, then, since $I$ is regular, there exists $e\in R$ such that
	$r-re\in I$. Since 
	\[
	re+I=r(e+I)=0,
	\]
	$re\in I$ and hence $r\in I$. Thus $J(R)\subseteq K$. 
\end{proof}

\begin{example}
	Each maximal ideals of $\Z$ is of the form $p\Z=\{pm:m\in\Z\}$ for some prime number $p$. 
	Thus $J(\Z)=\cap_p p\Z=\{0\}$.
\end{example}

%\begin{example}
%	Sea $D$ un anillo de división y sea $R=D[x_1,\dots,x_n]$. Si $f\in J(R)$
%	entonces\dots Luego $J(R)=0$. 
%\end{example}

We now review some basic results useful to compute radicals. 

\begin{proposition}
	Let $\{R_i:i\in I\}$ be a family of rings. Then 
	\[
	J\left(\prod_{i\in I}R_i\right)=\prod_{i\in I}J(R_i).
	\]
\end{proposition}

\begin{proof}
	Let $R=\prod_{i\in I}R_i$ and $x=(x_i)_{i\in I}\in R$.  The left ideal 
    $Rx$ is quasi-regular if and only if each left ideal $R_ix_i$
	is quasi-regular in $R_i$, as $x$ is quasi-regular in $R$ if and only if each 
	$x_i$ is quasi-regular in $R_i$. Thus $x\in J(R)$ if and only if $x_i\in
	J(R_i)$ for all $i\in I$.
\end{proof}

For the next result we shall need a lemma.

\begin{lemma}
	\label{lemma:trickJ1}
	Let $R$ be a ring and $x\in R$. 
	If $-x^2$ is a left quasi-regular element, then $x$ también. 
\end{lemma}

\begin{proof}
	Sea $r\in R$ tal que $r+(-x^2)+r(-x^2)=0$ y sea $s=r-x-rx$. Entonces $x$ es
	casi-regular a izquierda pues 
	\begin{align*}
		s+x+sx&=(r-x-rx)+x+(r-x-rx)x\\
		&=r-x-rx+x+rx-x^2-rx^2=r-x^2-rx^2=0.\qedhere 
\end{align*}
\end{proof}

%\begin{lemma}
%	\label{lemma:trickJ2}
%	Sea $R$ un anillo. Entonces $x\in J(R)$ si y sólo si $Rx$ es un ideal a
%	izquierda casi-regular a izquierda.
%\end{lemma}
%
%\begin{proof}
%	Si $x\in J(R)$ entonces $Rx\subseteq J(R)$ y luego todo elemento de $Rx$ es
%	casi-regular a izquierda. Recíprocamente, si $Rx$ es casi-regular a
%	izquierda, $Rx+\Z x$ es un ideal a izquierda de $R$. Si $s=rx+nx\in Rx+\Z
%	x$, entonces $-s^2$ es casi-regular a izquierda (pues $-s^2\in Rx$). Por el
%	lema~\ref{lemma:trickJ1}, $s$ es casi-regular a izquierda; en particular,
%	$x$ es casi-regular a izquierda y luego $x\in J(R)$. 
%\end{proof}

\begin{proposition}
	\label{proposition:J(I)}
	If $I$ is an ideal of $R$, then $J(I)=I\cap J(R)$. 
\end{proposition}

\begin{proof}
	Since $I\cap J(R)$ if an ideal of $I$, if $x\in I\cap J(R)$, then $x$ is
	left quasi-regular in $R$. Let $r\in R$ be such that $r+x+rx=0$. 
	Since $r=-x-rx\in I$, $x$ is left quasi-regular 
	in $I$. Thus $I\cap J(R)\subseteq J(I)$. 

	Let $x\in J(I)$ and $r\in R$. Since $-(rx)^2=(-rxr)x\in
	I(J(I))\subseteq J(I)$, the element $-(rx)^2$ is left quasi-regular a izquierda
	en $I$. Thus $rx$ is left quasi-regular by
	Lemma~\ref{lemma:trickJ1}.
\end{proof}

\index{Ring!radical}
A ring $R$ is said to be \textbf{radical} if $J(R)=R$. 

\begin{example}
	If $R$ is a ring, then $J(R)$ is a radical ring, by Proposition~\ref{proposition:J(I)}.
\end{example}

\begin{example}
	The Jacobson radical of $\Z/8$ is $\{0,2,4,6\}$. 
\end{example}

There are several characterizations of radical rings. 

\begin{theorem}
	\label{theorem:anillo_radical}
	Let $R$ be ring. The following statements are equivalent: 
	\begin{enumerate}
		\item $R$ is radical.
		\item $R$ admits no simple $R$-modules. 
		\item $R$ no tiene ideales a izquierda maximales y regulares.
		\item $R$ no tiene ideales a izquierda primitivos.
		\item Every element of $R$ is quasi-regular. 
		\item $(R,\circ)$ is a group. 
	\end{enumerate}
\end{theorem}

\begin{proof}
	The equivalence $(1)\Longleftrightarrow(5)$ follows from 
	Theorem~\ref{thm:casireg_eq}. 
    
    The equivalence $(5)\Longleftrightarrow(6)$ is left as an exercise. 

	Let us prove that $(1)\implies(2)$. Assume that there exists a simple $R$-module $N$. Since 
	$R=J(R)\subseteq\Ann_R(N)$, $R=\Ann_S(N)$. 
	Hence $RN=\{0\}$, a contradiction to the simplicity of $N$.
	
	To prove $(2)\implies(3)$ we note that for each regular and maximal left ideal 
	$I$, the quotient $R/I$ is a simple $R$-module by
	Proposición~\ref{proposition:R/I}. 
	
	To prove $(3)\implies(4)$ assume that there is a primitive left ideal 
	$I=\Ann_R(M)$, where $M$ is some simple $R$-module. Since $R=J(R)\subseteq I$, it follows that  
    $I=R$, a contradiction to the simplicity of $M$.
    %$RM=\{0\}$.

	Finally we prove $(4)\implies(2)$. If $M$ is a simple $R$-module, then 
	$\Ann_R(M)$ is a primitive left ideal.
\end{proof}

\begin{example}
	Let 
	\[
	A=\left\{\frac{2x}{2y+1}:x,y\in\Z\right\}.
	\]
	Then $A$ is a radical ring, as the inverse of the element $\frac{2x}{2y+1}$
	with respect to the circle operation 
	$\circ$ is 
	\[
	\left(\frac{2x}{2y+1}\right)'=\frac{-2x}{2(x+y)+1}.
	\]
\end{example}

\index{Ring!nil}
A ring $R$ is said to be \textbf{nil} if for every $x\in R$ there
exists $n=n(x)$ such that $x^n=0$. 

\begin{exercise}
    Prove that a nil ring is a radical ring. 
\end{exercise}

\begin{exercise}
    Let $\R[X]$ be the ring of power series with real coefficients. Prove that the ideal 
    $X\R[X]$ consisting of power series with zero constant term is a radical ring
    that is not nil. 
\end{exercise}

The following problem is maybe the most important open 
problem in non-commutative ring theory. 


The conjecture is known to be true in several cases.
\framebox{Exercises?}

\begin{theorem}
	\label{thm:Jnilpotente}
	If $R$ is a left artinian ring, then $J(R)$ is nilpotent. 
\end{theorem}

\begin{proof}
	Let $J=J(R)$. Since $R$ is a left artinian ring, the sequence 
	$(J^m)_{m\in\N}$ of left ideals stabilizes. There exists 
	$k\in\N$ such that $J^k=J^l$ for all $l\geq k$. We claim that $J^k=\{0\}$. If
	$J^k\ne\{0\}$ let $\mathcal{S}$ the set of left ideals 
	$I$ such that $J^kI\ne\{0\}$. Since 
	\[
	J^kJ^k=J^{2k}=J^k\ne\{0\},
	\]
	the set $\mathcal{S}$ is non-empty. 
	Since $R$ is left artinian, $\mathcal{S}$ has a minimal element $I_0$. Since $J^kI_0\ne\{0\}$, let $x\in
	I_0\setminus\{0\}$ be such that $J^kx\ne\{0\}$. Moreover, $J^kx$ is a left ideal of $R$ 
	contained in $I_0$ and such that $J^kx\in\mathcal{S}$, as 
	$J^k(J^kx)=J^{2k}x=J^kx\ne\{0\}$. The minimality of $I_0$ implies that, $J^kx=I_0$. In particular, 
	there exists $r\in J^k\subseteq J(R)$ such that $rx=x$. Since $-r\in
	J(R)$ is left quasi-regular, there exists $s\in R$ such that $s-r-sr=0$.
	Thus 
	\[
		x=rx=(s-sr)x=sx-s(rx)=sx-sx=0,
	\]
	a contradiction.
\end{proof}

\begin{corollary}
	Let $R$ be a left artinian ring. Each nil left ideal is nilpotent and 
	$J(R)$ is the unique maximal nilpotent ideal of $R$. 
\end{corollary}

\begin{proof}
	Let $L$ be a nil left ideal of $R$. By Proposition~\ref{pro:nilJ}, $L$
	is contained in $J(R)$. Thus $L$ is nilpotent, as $J(R)$ 
	is nilpotent by Theorem~\ref{thm:Jnilpotente}. 
\end{proof}

\begin{theorem}
	Let $R$ be a ring and $n\in\N$. Then $J(M_n(R))=M_n(J(R))$. 
\end{theorem}

\begin{proof}
	We first prove that $J(M_n(R))\subseteq M_n(J(R))$. 
	If $J(R)=R$, the theorem is clear. Let us assume that $J(R)\ne R$ and let  
	$J=J(R)$. 
	If $M$ is a simple $R$-module, then $M^n$ is a simple $M_n(R)$-module with the usual multiplication. 
	Let $x=(x_{ij})\in J(M_n(R))$ and $m_1,\dots,m_n\in M$. Then
	\[
		x\colvec{3}{m_1}{\vdots}{m_n}=0.
	\]
	In particular, $x_{ij}\in\Ann_R(M)$ for all $i,j\in\{1,\dots,n\}$. Hence 
	$x\in M_n(J)$. 

	We now prove that $M_n(J)\subseteq J(M_n(R))$. Let 
	\[
		J_1=\begin{pmatrix}
			J & 0 & \cdots & 0\\
			J & 0 & \cdots & 0\\
			\vdots & \vdots & \ddots & \vdots\\
			J & 0 & \cdots & 0
		\end{pmatrix}
		\quad\text{and}\quad
		x=\begin{pmatrix}
			x_1 & 0 & \cdots & 0\\
			x_2 & 0 & \cdots & 0\\
			\vdots & \vdots & \ddots & \vdots\\
			x_n & 0 & \cdots & 0
		\end{pmatrix}\in J_1.
	\]
	Since $x_1$ es quasi-regular, there exists $y_1\in R$ such that $x_1+y_1+x_1y_1=0$.
	If
	\[
		y=\begin{pmatrix}
			y_1 & 0 & \cdots & 0\\
			0 & 0 & \cdots & 0\\
			\vdots & \vdots & \ddots & \vdots\\
			0 & 0 & \cdots & 0
		\end{pmatrix}, 
	\]
	then $u=x+y+xy$ is lower triangular, as  
	\[
		u=\begin{pmatrix}
			0 & 0 & \cdots & 0\\
			x_2y_1 & 0 & \cdots & 0\\
			x_3y_1 & 0 & \cdots & 0\\
			\vdots & \vdots & \ddots & \vdots\\
			x_ny_1 & 0 & \cdots & 0
		\end{pmatrix}.
	\]
	Since  
	$u^n=0$, the element
	\[
	v=-u+u^2-u^3+\cdots+(-1)^{n-1} u^{n-1}
	\]
	is such that 
	$u+v+uv=0$. Thus $x$ is right quasi-regular, as  
	\begin{align*}
		x+(y+v+yv)+x(y+v+yv)&=0,
	\end{align*}
	and therefore $J_1$ is right quasi-regular. Similarly one proves that 
	each $J_i$ is right quasi-regular and hence $J_i\subseteq J(M_n(R))$ for all 
	$i\in\{1,\dots,n\}$. In conclusion, 
	\[
	J_1+\cdots+J_n\subseteq J(M_n(R))
	\]
	and therefore $M_n(J)\subseteq J(M_n(R))$.
\end{proof}

For completeness we recall basic results on the Jacobson radical in the case
of unitary rings. 

\begin{exercise}
	Let $R$ be a unitary ring. Then  
	\[
	J(R)=\bigcap\{M:\text{$M$ is a left maximal ideal}\}.
	\]
\end{exercise}

\begin{exercise}
	Let $R$ be a unitary ring. The
	following statements are equivalent: 
	\begin{enumerate}
		\item $x\in J(R)$.
		\item $xM=0$ for all simple $R$-module $M$.
		\item $x\in P$ for all primitive left ideal $P$.
		\item $1+rx$ is invertible for all $r\in R$.
		\item $1+\sum_{i=1}^n r_ixs_i$ is invertible for all $n\in\N$ and all $r_i,s_i\in R$.
		\item $x$ belongs to every left maximal ideal maximal. 
	\end{enumerate}
\end{exercise}

\section*{B}

We now go back to study solutions to the YBE and discuss the intriguing interplay
between radical rings and involutive solutions. 

\begin{definition}
\index{Solution!involutive}
A solution $(X,r)$ is said to be \emph{involutive} if $r^2=\id$. 
\end{definition}

\index{Symmetric group}
For $n\geq2$, the \emph{symmetric group} $\Sym_n$ can be presented 
as the group with generators $\sigma_1,\dots,\sigma_{n-1}$ and relations
\begin{align*}
    &\sigma_i\sigma_{i+1}\sigma_i=\sigma_{i+1}\sigma_i\sigma_{i+1} && \text{if }1\leq i\leq n-2,\\
    &\sigma_i\sigma_j=\sigma_j\sigma_i && \text{if }|i-j|\geq 1,\\
    &\sigma_i^2=1 && \text{for all $i\in\{1,\dots,n-1\}$}.
\end{align*}
Let $(X,r)$ be an involutive solution. 
Then the map $\sigma_i\mapsto r_{i,i+1}=\id_{X^{i-1}}\times r\times\id_{X^{n-i-1}}$ extends 
to an action of $\Sym_n$ on $X^n$.

\begin{example}
Let $X$ be a non-empty set and $\sigma$ be a bijection on $X$. Then 
$(X,r)$, where $r(x,y)=(\sigma(y),\sigma^{-1}(x))$, is an involutive solution. 
\end{example}

\index{Jacobson!radical ring}
\index{Radical ring}
We now present a very important family of involutive solutions. 
These examples show an intriguing connection between the YBE and the 
theory of non-commutative rings. 


% In any ring $R$ the \emph{circle operation} 
% \[
% x\circ y=x+xy+y
% \]
% is always associative and such that $x\circ 0=0\circ x=x$ for all $x\in R$. 
% A non-unital ring (or \emph{ring}, for short) 
% $R$ is said to be a \emph{radical ring} if $(R,\circ)$ is a group. 
% In this case, following Jacobson's notation, the inverse of 
% an element $x$ with respect to the circle operation is denoted by $x'$. 


\begin{example}
Let $p$ be a prime and let $A=\Z/(p^2)$ be the cylic additive group of order $p^2$. 
The operation $x\circ y=x+y+pxy$ 
turns $A$ into a radical ring. 
\end{example}

\begin{example}
Let $A=\left\{\frac{2x}{2y+1}:x,y\in\Z\right\}$. The operation 
$a\circ b=a+b+ab$ turns $A$ into a radical ring. A straightforward computation shows that 
\[
\left(\frac{x}{2y+1}\right)'=\frac{2(-x)}{2(x+y)+1}.
\]
\end{example}

The following fundamental family of solutions appears in~\cite{MR2278047}. 
It turns out to be fundamental in the study of 
set-theoretic solutions to the YBE. 

\begin{proposition}
\label{pro:Rump}
Let $R$ be a radical ring. Then $(R,r)$, where 
\[
r(x,y)=( -x+x\circ y,(-x+x\circ y)'\circ x\circ y)
\]
is an involutive solution.
\end{proposition}

The proposition can demonstrated using Theorem~\ref{thm:LYZ}, 
see Exercise~\ref{prob:Rump}. 
We will prove a stronger result in Theorem~\ref{thm:YB}. 

\begin{proposition}
\label{pro:T}
Let $(X,r)$ be an involutive solution. 
Then the map $T\colon X\to X$, $x\mapsto\sigma_x^{-1}(x)$, is 
invertible with inverse $T^{-1}(y)=\tau^{-1}_y(y)$ and 
\[
T^{-1}\circ\sigma_x^{-1}\circ T=\tau_x
\]
for all $x\in X$. 
\end{proposition}

\begin{proof}
Let $U(x)=\tau_x^{-1}(x)$. Since $r$ is involutive, 
\[
(U(x),x)=r^2(U(x),x)=r(\sigma_{U(x)}(x),x)=(\sigma_{\sigma_{U(x)}(x)}(x),\tau_x\sigma_{U(x)}(x)).
\]
The second coordinate can be written as $U(x)=\sigma_{U(x)}(x)$. This 
implies that 
\[
T(U(x))=\sigma^{-1}_{U(x)}(U(x))=x.
\]
Similarly one obtains $U(T(x))=x$. 

Since $(X,r)$ is a solution, Lemma~\ref{lem:YB} implies that 
$\sigma_x\sigma_y=\sigma_{\sigma_x(y)}\sigma_{\tau_y(x)}$
%$\tau_{\tau_y(x)}\circ\tau_{\sigma_x(y)}=\tau_y\circ\tau_x$ 
holds for all $x,y\in X$. Then  
\[
\sigma_y^{-1}T(x)
=\sigma_y^{-1}\sigma_x^{-1}(x)
=\sigma^{-1}_{\tau_y(x)}\sigma^{-1}_{\sigma_x(y)}(x)
=\sigma^{-1}_{\tau_y(x)}\tau_y(x)
=T\tau_y(x)
\]
for all $y\in X$, by Equality~\eqref{eq:involutive}.
\end{proof}

Note that if $(X,r)$ is a non-degenerate involutive solution, then 
\[
(x,y)=r^2(x,y)=r(\sigma_x(y),\tau_y(x))=(\sigma_{\sigma_x(y)}\tau_y(x),\tau_{\tau_y(x)}\sigma_x(y)).
\]
Hence 
\begin{equation}
    \label{eq:involutive}
    \tau_y(x)=\sigma_{\sigma_x(y)}^{-1}(x),
    \quad
    \sigma_x(y)=\tau_{\tau_y(x)}^{-1}(y)
\end{equation}
for all $x,y\in X$. Thus for involutive solutions
it is enough to know $\{\sigma_x:x\in X\}$, as from this we obtain the
set $\{\tau_x:x\in X\}$. 


\begin{definition}
\index{Cycle set}
\index{Cycle set!non-degenerate}
A \emph{cycle set} is a pair $(X,\cdot)$, where $X$ is a non-empty 
set provided with a binary operation $X\times X\to X$, $(x,y)\mapsto x\cdot y$, 
such that 
\begin{equation}
    \label{eq:cycle_set}
    (x\cdot y)\cdot (x\cdot z)=(y\cdot x)\cdot (y\cdot z)
\end{equation}
holds for all $x,y,z\in X$ and each map $\varphi_x\colon X\to X$, $y\mapsto x\cdot y$, is bijective. 
A cycle set $(X,\cdot)$ is said to be \emph{non-degenerate} 
if the map $X\to X$, $x\mapsto x\cdot x$, is bijective. 
\end{definition}

\begin{definition}
\index{Homomorphism!of cycle sets}
Let $X$ and $Z$ be cycle sets. 
A \emph{homomorphism} between the cycle sets $X$ and $Z$ is a 
map $f\colon X\to Z$ such that $f(x\cdot y)=f(x)\cdot f(y)$ for all $x,y\in X$. An \emph{isomorphism} of cycle sets
is a bijective homomorphism of cycle sets. 
\end{definition}

Cycle sets and cycle set homomorphisms form a category. 
It is possible to prove that the category of 
solutions is equivalent to the category of cycle sets, 
see Exercise~\ref{prob:cycle_sets}. 

% \framebox{Exercise!}

% \begin{lemma}
% \label{lem:T_forCS}
% If $X$ is a cycle set, then $x\cdot (y\cdot y)=((y*x)\cdot y)\cdot ((y*x)\cdot y)$, where
% $y*x=z$ if and only if $y\cdot z=x$. 
% \end{lemma}

% \begin{proof}
%     Since the operation $x\mapsto y\cdot x$ is bijective, we can write $x=y\cdot (y*x)$. Then, using~\eqref{eq:cycle_set}, 
%     $x\cdot (y\cdot y)=(y\cdot (y*x))\cdot (y\cdot y)=((y*x)\cdot y)\cdot (y*x)\cdot y)$.
% \end{proof}

\begin{theorem}
\label{thm:CS}
There exists a bijective correspondence between non-isomorphic involutive solutions 
and non-isomorphic non-degenerate cycle sets. 
\end{theorem}



% \begin{proof}
%     Let us assume first that $r(x,y)=(\sigma_x(y),\tau_y(x))$ is an involutive solution. 
%     We want to prove that the operation
%     $x\cdot y=\sigma_x^{-1}(y)$ turns the set $X$ into a non-degenerate cycle set. 
%     It is clear that the maps $y\mapsto x\cdot y=\sigma_x^{-1}(y)$ are invertible. 
%     By Proposition~\ref{pro:T}, the operation $x\mapsto x\cdot x$ is bijective. \framebox{!}
%     Since $r(\tau^{-1}_y(x),y)=(\sigma_{\tau^{-1}_y(x)}(y),x)$, Lemma~\ref{lem:YB} implies that
%     \[
%     \tau_x\circ\tau_{\sigma_{\tau^{-1}_y(x)}(y)}=\tau_y\circ\tau_{\tau^{-1}_y(x)}.
%     \]
%     Moreover, since $r(\sigma_{\tau^{-1}_y(x)}(y),x)=(\tau^{-1}_y(x),y)$, it follows that
%     %$\tau_x\sigma_{\tau^{-1}_y(x)}(y)=y$, i.e.
%     \[
%     \sigma_{\tau^{-1}_y(x)}(y)=\tau_x^{-1}(y).
%     \]
%     % Since $r^2=\id_{X\times X}$, it follows that
%     % \[
%     % (\tau^{-1}_y(x),y)=r^2(\tau^{-1}_y(x),y)=r(\sigma_{\tau^{-1}_y(x)}(y),x)
%     % \]
%     % and hence $\sigma_{\tau^{-1}_y(x)}(y)=\tau^{-1}_x(y)$. 
%     This implies that
%     \[
%     \tau_x\circ\tau_{\tau^{-1}_x(y)}=\tau_x\circ\tau_{\sigma_{\tau^{-1}_y(x)}(y)}=\tau_y\circ\tau_{\tau^{-1}_y(x)},
%     \]
%     which turns out to be equivalent to
%     \[
%     \tau^{-1}_{\tau^{-1}_x(y)}\circ\tau^{-1}_x=\tau^{-1}_{\tau^{-1}_y(x)}\circ\tau^{-1}_y
%     \]
%     and hence equivalent to Equality~\eqref{eq:cycle_set}. 
    
%     Conversely, if $X$ is a non-degenerate cycle set, we want to prove that
%     \[
%     r(x,y)=((y*x)\cdot y,y*x),
%     \]
%     where $y*x=z$ if and only if $y\cdot z=x$, is an involutive solution. The definition of $r$ implies that
%     $r^2=\id_{X\times X}$. 
%     By Lemma~\ref{lem:YB}, we need to prove that
%     \begin{align*}
%         &\sigma_x\circ\sigma_y = \sigma_{\sigma_x(y)}\circ\sigma_{\tau_y(x)},&
%         &\sigma_{\tau_{\sigma_y(z)}(x)}\tau_z(y)=\tau_{\sigma_{\tau_y(x)}(z)}\sigma_x(y),&
%         &\tau_z\circ\tau_y=\tau_{\tau_z(y)}\circ\tau_{\sigma_y(z)}
%     \end{align*}
 
%     Write $\sigma_x(y)=(y*x)\cdot y$ and $\tau_y(x)=y*x$. 
%     We know that the third formula is equivalent to~\eqref{eq:cycle_set}. 
%     Let $T\colon X\to X$, $x\mapsto x\cdot x$. By assumption, $T$ is bijective and hence 
%     $T^{-1}\circ \tau^{-1}_x\circ T=\sigma_x$ for all $x\in X$, since by Lemma~\ref{lem:T_forCS}
%         \begin{align*}
%     \tau_x^{-1}T(y)&=\tau_x^{-1}(y\cdot y)=x\cdot (y\cdot y)\\
%     &=((y*x)\cdot y)\cdot ((y*x)\cdot y)=T((y*x)\cdot y)=T\sigma_x(y)
%     \end{align*}
%     for all $x,y\in X$. In particular, $(X,r)$ is non-degenerate. 
%     Moreover, 
%     \begin{align*}
%     \sigma_x\circ\sigma_y
%     &=(T^{-1}\circ\tau_x^{-1}\circ T)\circ (T^{-1}\circ\tau_y^{-1}\circ T)\\
%     &=T^{-1}\circ (\tau_y\circ\tau_x)^{-1}\circ T\\
%     &=T^{-1}\circ (\tau_{\tau_y(x)}\circ\tau_{\sigma_x(y)})^{-1}\circ T\\
%     &=(T^{-1}\circ\tau^{-1}_{\sigma_x(y)}\circ T)\circ (T^{-1}\circ\tau^{-1}_{\tau_y(x)}\circ T)\\
%     &=\sigma_{\sigma_x(y)}\circ\sigma_{\tau_y(x)}
% \end{align*}
% for all $x,y\in X$. Finally, Equality~\eqref{eq:cycle_set} can be written as
% \[
% \tau^{-1}_{\tau^{-1}_v(u)}\circ\tau^{-1}_v=\tau^{-1}_{\tau^{-1}_u(v)}\circ\tau^{-1}_u.
% \]
% With $u=\tau_z\tau_y(x)$ and $v=z$, it becomes
% \[
% \tau^{-1}_{\tau_y(x)}\circ\tau^{-1}_z
% =\tau^{-1}_{\tau^{-1}_{\tau_z\tau_y(x)}(z)}\circ\tau^{-1}_{\tau_z\tau_y(x)}
% =\tau^{-1}_{\sigma_{\tau_y(x)}(z)}\circ\tau^{-1}_{\tau_z\tau_y(x)},
% \]
% where the last equality holds $r^2=\id_{X\times X}$. 
% Using Lemma~\ref{lem:YB} one rewrites this formula as
% \[
% \tau_{\sigma_{\tau_y(x)}(z)}\circ\tau^{-1}_{\tau_y(x)}
% =\tau^{-1}_{\tau_z\tau_y(x)}\circ\tau_z
% =\tau^{-1}_{\tau_{\tau_z(y)}\tau_{\sigma_y(z)}(x)}\circ\tau_z.
% \]
% Evaluating this equality at $y$, 
% \[
% \tau_{\sigma_{\tau_y(x)}(z)}\sigma_x(y)=\tau_{\sigma_{\tau_y(x)}(z)}\tau^{-1}_{\tau_y(x)}(y)
% =\tau^{-1}_{\tau_{\tau_z(y)}\tau_{\sigma_y(z)}(x)}\tau_z(y)
% =\sigma_{\tau_{\sigma_y(z)}(x)}\tau_z(y).\qedhere
% \]
% \end{proof}

For the readers who are not familiar with the above-mentioned result, 
the bijective correspondence is given by 
\[
r(x,y)=(x*y,(x*y)\cdot x),
\]
where $x*y=z$ if and only if $x\cdot z=y$. We leave the proof for the reader, see 
Exercise~\ref{prob:CS}. However, we will prove a more 
general result in Theorem~\ref{thm:skewCS}. 

Theorem~\ref{thm:CS} can be used to construct and enumerate small 
involutive solutions~\cite{AMV}. Table~\ref{tab:IYB} shows the 
number of non-isomorphic involutive solutions of size $\leq10$. 
For size $\leq7$ the numbers of Table~\ref{tab:IYB} coincide with those in~\cite{MR1722951}
but differ by two for $n=8$, as two solutions of size eight 
are missing in~\cite{MR1722951}. 

\begin{table}[H]
\centering
\caption{Involutive solutions of size $\leq10$.}
\begin{tabular}{|r|ccccccccc|}
\hline
$n$ & 2 & 3 & 4 & 5 & 6 & 7 & 8 & 9 & 10\tabularnewline
\hline 
solutions & 2 & 5 & 23 & 88 & 595 & 3456 & 34530 & 321931 & 4895272\tabularnewline
% square-free & 1 & 2 & 5 & 17 & 68 & 336 & 2041 & \cellcolor{gray!30}{15534} & \cellcolor{gray!30}{150957}\tabularnewline
% indecomposable & 1 & 1 & 5 & 1 & 10 & 1 & \cellcolor{gray!30}{100} & \cellcolor{gray!30}{16} & \cellcolor{gray!30}{36}\tabularnewline
% multipermutation & 2 & 5 & 21 & 84 & 554 & 3295 & \cellcolor{gray!30}{32155} & \cellcolor{gray!30}{305916} & \cellcolor{gray!30}{4606440}\tabularnewline
% irretractable & 0 & 0 & 2 & 4 & 9 & 13 & 191 & \cellcolor{gray!30}{685} & \cellcolor{gray!30}{3590}\tabularnewline
\hline
\end{tabular}
\label{tab:IYB}
\end{table}



% \begin{definition}
% \index{Solution!permutation group}
% The \emph{permutation group} of a solution $(X,r)$ is the group
% \[
% \mathcal{G}(X,r)=\langle (\sigma_x,\widehat{\sigma_x}):x\in X\rangle\subseteq\Sym_X\times\Sym_X.
% \]
% \end{definition}

% \index{Solutions!linear involutive}
% Let us finish the chapter with the family of affine involutive solutions. Let $A$ be an abelian group and
% $a,b,c,d\in\colon A\to A$ be group homomorphisms. If 
% \[
% r(x,y)=(a(x)+b(y),c(x)+d(y))
% \]
% is bijective and $(A,r)$ is an solution involutive solution, then $(A,r)$ is an \emph{linear solution}.
% Note that $r$ satisfies the YBE if and only if
% \begin{subequations}
% \begin{gather}
% \label{eq:linear}
% (\id-a)\circ d\circ b=b\circ d,
% \quad
% d\circ(\id-d)=c\circ d\circ b,\\
% c\circ d\circ (\id-a)=d\circ c,
% \quad
% a\circ(\id-a)=b\circ a\circ c,
% \quad
% c\circ a=(\id-d)\circ a\circ c,\\
% a\circ b=b\circ a\circ (\id-d),
% \quad
% c\circ b-b\circ c=a\circ d\circ a-d\circ d.
% \end{gather}
% \end{subequations}
% Moreover, $r^2=\id$ if and only if
% \begin{align}
%     &a\circ a+b\circ c=\id,
%     &&
%     c\circ b+d\circ d=\id,
%     &&
%     a\circ b+b\circ d=0,
%     &&
%     c\circ a+d\circ c=0.
% \end{align}
% \end{equation}

\begin{prob}
\label{prob:cycle_sets}
Prove that the category of non-degenerate 
cycle sets and the category of solutions are equivalent. 
\end{prob}

\begin{prob}
\label{prob:Rump}
Prove Proposition~\ref{pro:Rump}. 
\end{prob}

\begin{prob}
If $X$ is a cycle set, then $x\cdot (y\cdot y)=((y*x)\cdot y)\cdot ((y*x)\cdot y)$, where
$y*x=z$ if and only if $y\cdot z=x$. 
\end{prob}

\begin{prob}
\label{prob:CS}
Prove Theorem~\ref{thm:CS}. 
\end{prob}

\section*{Open problems}


% \begin{conjecture}
% \label{conj:Koethe}
% If $R$ is a nil ring, then $R[X]$ is a radical ring. 
% \end{conjecture}

% \begin{conjecture}
% \label{conj:Koethe3}
% If $R$ is a nil ring, then $M_2(R)$ is a nil ring. 
% \end{conjecture}

% \begin{conjecture}
% \label{conj:Koethe4}
% Let $n\geq3$. If $R$ is a nil ring, then $M_n(R)$ is a nil ring. 
% \end{conjecture}

%Now we list some problems related to the solutions to the YBE. 

\begin{problem}[K\"othe]
\label{prob:Koethe}
Let $R$ be a ring. Is the sum 
of two arbitrary nil left ideals of $R$ is nil?
\end{problem}

\begin{problem}
Construct and enumerate involutive solutions of size $11$. 
\end{problem}

\begin{problem}
Estimate the number of solutions of size $n$ for $n\to\infty$. 
\end{problem}

\section*{Notes}

The material on non-commutative ring theory is standard, see for example~\cite{MR3308118}.
Radical rings were introduced by Jacobson in~\cite{MR12271}. Nil rings were
used by Zelmanov in his solution to Burnside's problem, see for example~\cite{MR1199575}. 

Open problem~\ref{prob:Koethe} is the well-known K\"othe's conjecture. 
The conjecture was first formulated in 1930, see \cite{MR1545158}. It is known to be true
in several cases. In full generality, the problem is still open. In~\cite{MR306251} 
Krempa proved that
the following statements are equivalent:
\begin{enumerate}
    \item K\"othe's conjecture is true.  
    \item If $R$ is a nil ring, then $R[X]$ is a radical ring. 
    \item If $R$ is a nil ring, then $M_2(R)$ is a nil ring. 
    \item Let $n\geq2$. If $R$ is a nil ring, then $M_n(R)$ is a nil ring. 
\end{enumerate}

In 1956 Amitsur formulated the following conjecture, see for example
\cite{MR0347873}: If $R$ is a nil ring, then $R[X]$ is a nil ring. In~\cite{MR1793911} 
Smoktunowicz found a counterexample to Amitsur's conjecture. 
This counterexample suggests that K\"othe's conjecture might be false. 
A simplification of Smoktunowicz's example
appears in~\cite{MR3169522}. See \cite{MR1879880,MR2275597} for more
information on K\"othe's conjecture and related topics. 


Rump introduced cycle sets in~\cite{MR2132760}. The bijective correspondence of 
Theorem~\ref{thm:CS} was 
also proved by Rump in~\cite{MR2132760}. A similar result can be 
found in~\cite[Proposition 2.2]{MR1722951}. 

The numbers of Table~\ref{tab:IYB} were computed in~\cite{AMV}
using a combination of~\cite{GAP4} and constraint programming techniques. 
The algorithm is based on an idea of Plemmons~\cite{MR0258994}, originally 
conceived to construct non-isomorphic semigroups.  

