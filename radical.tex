\chapter{Radical rings}

A \textbf{rng} (ring possibly without identity) 
is an abelian group $R$ with an associative multiplication 
$(x,y)\mapsto xy$ such that $(x+y)z=xz+yz$ and $x(y+z)=xy+xz$ for all $x,y,z\in
R$. If there is an element $1\in R$ such that $x1=1x=x$ for all $x\in R$, we 
say that $R$ is a ring (or a unitary ring).  A \textbf{subring} $S$ of $R$ is an additive
subgroup of $R$ closed under multiplication. 

\begin{example}
	$2\Z=\{2m:m\in\Z\}$ is a rng.  
\end{example}

A \textbf{left ideal} (resp. \textbf{right ideal}) is a subring $I$ of $R$ such that 
$rI\subseteq I$ (resp. $Ir\subseteq I$) for all $r\in R$. An \textbf{ideal}
(also two-sided ideal) of $R$ is a subring $I$ of $R$ that is both a left and a right ideal of $R$.

\begin{example}
If $I$ and $J$ are both ideals of $R$, then the sum $I+J=\{x+y:x\in I,y\in J\}$ and
the intersection $I\cap J$ are both ideals of $R$. The product $IJ$, defined as the additive
subgroup of $R$ generated by $\{xy:x\in I,y\in J\}$, is also an ideal of $R$. 
\end{example}

\begin{example}
If $R$ is a rng, the set $Ra =\{xa: x\in R\}$ is a left ideal
of $R$ and $aR =\{ax: x\in R\}$ is a right ideal of $R$. The set $RaR$, which is
defined as the additive subgroup of $R$ generated by $\{xay: x, y\in R\}$, is a
ideal of $R$.
\end{example}

\begin{example}
	Ir $R$ is a unitary ring, then $Ra$ is the left ideal generated by $a$, $aR$ is
	the right ideal generated by $a$ and $RaR$ is the ideal generated by $a$. 
	If $R$ is not unitary, the left ideal generated by $a$ is $Ra+\Z a$,
	the right ideal generated by $a$ is $aR+\Z a$ and the ideal generated by 
	$a$ is $RaR+Ra+aR+\Z a$.
\end{example}

A rng $R$ is said to be \textbf{simple} if $R^2\ne\{0\}$ and the only ideals of 
$R$ are $0$ and $R$. The condition $R^2\ne\{0\}$ is trivially satisfied in the case of rings
with identity, as $1\in R^2$. 

\begin{example}
	Division rings are simple.
\end{example}

Let $S$ be a unitary ring. Recall that $M_n(S)$ is the ring of $n\times n$ square matrices 
with entries in $S$.  If $A=(a_{ij})\in M_n(S)$ y $E_{ij}$ is the matrix
such that $(E_{ij})_{kl}=\delta_{ik}\delta_{jl}$, then
\begin{equation}
	\label{eq:trick}
E_{ij}AE_{kl}=a_{jk}E_{il}
\end{equation}
for all $i,j,k,l\in\{1,\dots,n\}$. 

\begin{exercise}
	If $D$ is a division ring, then $M_n(D)$ is simple. 
\end{exercise}

Let $R$ be a rng. A left $R$-module (or module, for short)  
is an abelian group $M$ together with a map $R\times M\to M$, $(r,m)\mapsto rm$, such that
$(r+s)m=rm+sm$,
$r(m+n)=rm+rs$, $r(sm)=(rs)m$ for all $r,s\in R$, $m,n\in M$.  If $R$ has an identity 
$1$ and $1m=m$ holds for all $m\in M$, the module $M$ is said to be 
\textbf{unitary}.  If $M$ is a unitary module, then $M=RM\ne\{0\}$.


The module $M$ is said to be 
\textbf{simple} if $RM\ne\{0\}$ and the only submodules of $M$ are $0$ and $M$.
If $M$ is a simple module, then $M\ne\{0\}$.

%\begin{remark}
%	Si $R$ es unitario y $M$ es un módulo simple, entonces $M$ es unitario.
%\end{remark}

\begin{lemma}
	\label{lemma:simple}
	Let $M$ be a non-zero module. Then $M$ is simple if and only if $M=Rm$
	for all $0\ne m\in M$.
\end{lemma}

\begin{proof}
	Assume that $M$ is simple.  Let $m\ne 0$. Since $Rm$ is a submodule of the simple 
	module $M$, either $Rm=0$ or $Rm=M$.  Let $N=\{n\in M:Rn=0\}$. Since $N$ is a 
	submodule of $M$ and $RM\ne\{0\}$, $N=\{0\}$. Therefore $Rm=M$s, as $m\ne0$.
	Now assume that $M=Rm$ for all $m\ne0$. Let $L$ be a non-zero submodule of 
	$M$ and let $0\ne x\in L$. Then $M=L$, as $M=Rx\subseteq L$. 
\end{proof} 

\begin{example}
	Let $D$ be a division ring and let $V$ be a non-zero vector space (over $D$). If 
	$R=\End_D(V)$, then $V$ is a simple $R$-módulo with $fv=f(v)$, $f\in R$.
	$v\in V$. 

	Para ver que $V$ es simple como $R$-módulo basta ver que $Rv=V$ para todo
	$v\ne0$. Sean $v,w\in V$, $v\ne0$.  Al completar $v\ne0$ a una base de $V$,
	vemos que existe $f\in R$ tal que $f(v)=w$. Luego $V$ es simple.
\end{example}

\begin{example}
	\label{exa:I_k}
	Sea $n\geq2$.  Si $D$ es un anillo de división y $R=M_n(D)$, entonces cada
	$I_k=\{ (a_{ij})\in R:a_{ij}=0\text{ para $j\ne k$}\}$ es un $R$-módulo (es
	más, es un $R$-módulo isomorfo a $D^n$).  Luego $M_{n}(D)$ es simple como
	anillo pero no como $M_n(D)$-módulo.
\end{example}

A left ideal $L$ of a rng $R$ is said to be \textbf{minimal} if $L\ne\{0\}$ and 
$L$ does not strictly contain other left ideals of $R$. Similarly one defines
right minimal ideals and minimal ideals. 

\begin{example}
	Sea $D$ un anillo de división y sea $R=M_n(D)$. Entonces $L=RE_{11}$ es un
	ideal a izquierda minimal.
\end{example}

\begin{example}
	Sea $L$ un ideal a izquierda no nulo. Si $RL\ne0$, entonces
	$L$ es minimal si y sólo si $L$ es simple como módulo.
\end{example}

Un ideal a izquierda $L$ de un anillo $R$ se dice \textbf{regular} si
existe $e\in R$ tal que $r-re\in L$ para todo $r\in R$. 
Similarmente, un ideal a derecha $I$ se dice \textbf{regular} si existe $e\in
R$ tal que $r-er\in I$ para todo $r\in R$.

\begin{remark}
	Si $R$ es un anillo unitario, todo ideal a izquierda (resp. a derecha) es
	regular.
\end{remark}

\begin{exercise}
	\index{Ideal!maximal}
	Un ideal $I$ de un anillo $R$ se dice maximal si $I\ne M$ e $I$ no está
	contenido propiamente en ningún ideal de $R$.  Demuestre que todo anillo
	unitario contiene un ideal maximal.
	% usar Zorn
\end{exercise}



\begin{proposition}
	\label{proposition:R/I}
	Sea $R$ un anillo y $M$ un $R$-módulo. Entonces $M$ es simple si y sólo si
	$M\simeq R/I$ para algún ideal a izquierda $I$ maximal y regular.		
\end{proposition}

\begin{proof}
	Supongamos que $M$ es simple, entonces $M=Rm$ para algún $m\ne0$ por el
	lema~\ref{lemma:simple}. Como $\phi\colon R\to M$, $r\mapsto rm$, es un
	epimorfismo de $R$-módulos, por el primer teorema de isomorfismos, 
	$M\simeq R/\ker\phi$. 
	
	Veamos que el ideal a izquierda $I=\ker\phi$ es maximal. Por el teorema de
	la correspondencia, como $M$ es simple, $I$ es maximal (pues todo ideal a
	izquierda $J$ tal que $I\subseteq J$ da un submódulo de $R/I$).

	Veamos que $I$ es regular. Como $M=Rm$, existe $e\in R$ tal que $m=em$. Si
	$r\in R$ entonces $r-re\in I$ pues
	$\phi(r-re)=\phi(r)-\phi(re)=rm-r(em)=0$.

	Supongamos ahora que $L$ es maximal y regular.  Por el teorema de la
	correspondencia, $R/L$ no tiene submódulos propios no nulos. Veamos
	entonces que $R(R/L)\ne0$. Si $R(R/L)=0$ y $r\in R$, entonces, como $L$ es
	regular, $r-re\in L$ y luego $r\in L$ pues 
	\[
	0=r(e+I)=re+I=r+I,
	\]
	una contradicción a la maximalidad de $L$.
\end{proof}


%\section{Nilpotencia}
%
%Recordemos que si $I$ es un ideal de un anillo $R$, se define $I^n$ como el
%subgrupo aditivo generado por el conjunto $\{y_1\dots y_n:y_j\in I\}$. 
%
%\begin{definition}
%	Un ideal $I$ de un anillo $R$ se dice \textbf{nilpotente} si $I^n=0$ para
%	algún $n\in\N$.
%\end{definition}
%
%Recordemos que un elemento $x$ de un anillo $R$ se dice \textbf{nilpotente} si
%existe $n\in\N$ tal que $x^n=0$. 
%
%\begin{definition}
%	Un ideal $I$ de un anillo $R$ se dice \textbf{nil} si todo elemento de $I$
%	es nilpotente.
%\end{definition}
%
%\begin{remark}
%	Un ideal nilpotente es nil. 
%\end{remark}
%
%\begin{example}
%	Sea $R=\C[x_1,x_2,\dots]/(x_1,x_2^2,x_3^3,\dots)$. El ideal
%	$I=(x_1,x_2,x_3,\dots)$ es nil en $R$ pues está generado por elementos
%	nilpotentes pero no es nilpotente. Si lo fuera, existiría $k\in\N$ tal que
%	$I^k=0$, y luego $x_i^k=0$ para todo $i$, una contradicción pues
%	$x_{k+1}^k\ne0$. 	
%\end{example}
%
%% example 2.7 del libro de springer
%% problema de kothe 
%
%\begin{lemma}
%	Si $I$ y $J$ son ideales nilpotentes, $I+J$ es nilpotente.
%\end{lemma}
%
%\begin{proof}
%	
%\end{proof}
%
%Un ideal $N$ de un anillo $R$ se dice \textbf{maximal-nilpotente} si $N$ es
%nilpotente y no está propiamente contenido en ningún ideal nilpotente de $R$.
%
%\begin{lemma}
%	Si el anillo $R$ contiene un ideal maximal-nilpotente $N$ entonces todo
%	ideal nilpotente está contenido en $N$.
%\end{lemma}
%
%\begin{proof}
%	
%\end{proof}
%
%\section{Anillos primos y semiprimos}
%
%\index{Dominio}
%Recordemos que un anillo $R$ se dice un \textbf{dominio} si para todo $a,b\in
%R$ tales que $ab=0$ se tiene $a=0$ o $b=0$.
%Una generalización al caso no conmutativo es la siguiente:
%
%\begin{definition}
%	\index{Anillo!primo}
%	Un anillo $R$ se dice \textbf{primo} si para todo $a,b\in R$ tales que
%	$aRb=0$ se tiene $a=0$ o $b=0$.
%\end{definition}
%
%\begin{lemma}
%	Sea $R$ un anillo. Las siguientes propiedades son equivalentes:
%	\begin{enumerate}
%		\item $R$ es primo.
%		\item Si $I,J\subseteq R$ son ideales a izquierda tales que $IJ=0$
%			entonces $I=0$ o $J=0$.
%%		\item Si $I,J\subseteq R$ son ideales a derecha tales que $IJ=0$ 
%%			entonces $I=0$ o $J=0$.
%		\item Si $I,J\subseteq R$ son ideales tales que $IJ=0$ entonces $I=0$ o
%			$J=0$.
%	\end{enumerate}
%\end{lemma}
%
%\begin{proof}
%	Vamos a demostrar que $(1)\implies(2)\implies(4)\implies(1)$. 
%	La implicación $(2)\implies(4)$ es trivial.
%
%	Veamos que $(4)\implies(1)$. Sean $a,b\in R$ tales que $aRb=0$.
%	Como entonces $(RaR)(RbR)=R(aRb)R=0$, $RaR=0$ o bien $RbR=0$. Supongamos
%	sin pérdida de generalidad que $RaR=0$. Entonces $Ra$ y $aR$ son ideales
%	biláteros tales que $(Ra)R=R(aR)=0$. Al aplicar la hipótesis, $Ra=aR=0$.
%	Como $\Z a$ es un ideal de $R$ tal que $(\Z a)R=0$, se concluye al aplicar
%	la hipótesis que $a=0$.
%
%	Veamos que $(1)\implies(2)$. Supongamos que $J\ne 0$, sea $y\in
%	J\setminus\{0\}$ y sea $x\in I$. Como 
%	$xRy\subseteq IRJ=I(RJ)\subseteq IJ=0$, 
%	se concluye, al usar la hipótesis, que $x=0$. 
%\end{proof}
%
%\begin{proposition}
%	Un anillo conmutativo es primo si y sólo si es un dominio. 
%\end{proposition}
%
%\begin{proof}
%	Supongamos que $R$ es un anillo primo. Si $a,b\in R$ son tales que $ab=0$
%	entonces $aRb=(ab)R=0$ y luego $a=0$ o bien $b=0$. Supongamos ahora que $R$
%	es un dominio. Si $a,b\in R$ son tales que $aRb=0$ entonces $(ab)R=0$ y
%	luego $a=0$ o bien $b=0$ pues $ab=0$. 
%\end{proof}
%
%\begin{definition}
%	\index{Anillo!semiprimo}
%	Un anillo $R$ se dice \textbf{semiprimo} si para todo $a\in R$ tal que
%	$aRa=0$ se tiene $a=0$.
%\end{definition}
%
%\begin{lemma}
%	Sea $R$ un anillo. Las siguientes aifrmaciones son equivalentes:
%	\begin{enumerate}
%		\item $R$ es semiprimo.
%		\item Si $I$ es un ideal a izquierda tal que $I^2=0$ entonces $I=0$.
%		\item Si $I$ es un ideal tal que $I^2=0$ entonces $I=0$.
%		\item $R$ no tiene ideales nilpotentes no nulos.
%	\end{enumerate}
%\end{lemma}
%
%\begin{proof}
%	Primero vamos a demostrar que $(1)\implies(2)\implies(3)\implies(1)$ 
%
%	La implicación $(5)\implies(4)$ es trivial.	
%	Demostremos entonces que
%	$(4)\implies(5)$. Sea $I$ un ideal nilpotente no nulo y sea $n\in\N$ el
%	mínimo tal que $I^n=0$. Como $(I^{n-1})^2=0$, por hipótesis se tiene que
%	$I^{n-1}=0$, una contradicción.
%\end{proof}
%

We will now discuss primitive rngs. Let $R$ be a rng and $M$ be a left $R$-module. For a 
subset $N\subseteq M$
we define the \textbf{annihilator} of $N$ as the subset 
\[
\Ann_R(N)=\{r\in R:rn=0\;\forall n\in N\}.
\]

\begin{example}
	$\Ann_{\Z}(\Z/n)=n\Z$.
\end{example}

\begin{lemma}
	\label{lemma:Ann}
	Sea $R$ un anillo y $M$ un $R$-módulo. Si $N\subseteq M$ es un subconjunto
	entonces $\Ann_R(N)$ es un ideal a izquierda de $R$. Si $N\subseteq M$ es un
	submódulo, $\Ann_R(N)$ es un ideal de $R$.
\end{lemma}

\begin{proof}
	Es fácil ver que $\Ann_R(N)$ es un subgrupo aditivo de $R$. Luego $\Ann_R(N)$
	es un ideal a izquierda pues $R\Ann_R(N)\subseteq\Ann_R(N)$: si $r\in R$,
	$s\in\Ann_R(N)$ y $n\in N$ entonces $(rs)n=r(sn)=r0=0$. 
	
	Si $N$ es un submódulo, $\Ann_R(N)R\subseteq\Ann_R(N)$ pues si
	$s\in\Ann_R(N)$, $r\in R$ y $n\in N$, $rn\in\Ann_R(N)$, entonces
	$(sr)n=s(rn)=0$.
\end{proof}

\begin{definition}
	\index{Módulo!fiel}
	Sea $R$ un anillo. Un $R$-módulo $M$ se dice \textbf{fiel} si $\Ann_R(M)=0$. 
\end{definition}

\begin{example}
	Si $K$ es un cuerpo, entonces $K^n$ es un $M_n(K)$-módulo unitario fiel.
\end{example}

\begin{example}
	Si $V$ es un $K$-espacio vectorial, entonces $V$ es un $\End_K(V)$-módulo
	unitario fiel.
\end{example}

\begin{definition}
	\index{Anillo!primitivo}
	Un anillo $R$ se dice \textbf{primitivo} si existe un $R$-módulo simple y
	fiel.
\end{definition}

Nuestra definición de anillo primitivo es, en realidad, la de anillo primitivo
a izquierda.  Si en lugar de usar módulos a izquierda se usan módulos a
derecha, se tiene la noción de anillo primitivo a derecha. 

\begin{proposition}
	\label{proposition:simple=>prim}
	Si $R$ es un anillo unitario simple, $R$ es primitivo.
\end{proposition}

\begin{proof}
	Existe un ideal a izquierda $I$ maximal. Además $I$ es regular pues $1\in R$.
	Por la proposición~\ref{proposition:R/I}, $R/I$ es un $R$-módulo
	simple. Como $\Ann_R(R/I)$ es un ideal de $R$ y $R$ es simple,
	$\Ann_R(R/I)\in\{0,R\}$. Luego $\Ann_R(R/I)=0$ pues $1\not\in\Ann(R/I)$.
\end{proof}

\begin{proposition}
	\label{proposition:prim+conm=cuerpo}
	Si $R$ es un anillo conmutativo, $R$ es primitivo si y sólo si $R$ es un
	cuerpo.
\end{proposition}

\begin{proof}
	Si $R$ es un cuerpo, $R$ es primitivo por ser un anillo unitario simple, ver 
	proposición~\ref{proposition:simple=>prim}. Si $R$ es un anillo
	conmutativo primitivo, la
	proposición~\ref{proposition:R/I} implica la existencia de un ideal $I$, 
	maximal y regular, tal que el $R$-módulo 
	$R/I$ es simple y fiel.
	Como $I\subseteq \Ann_R(R/I)=0$ y el ideal $I$ es regular, existe $e\in R$ tal
	que $r=re=er$. Luego $R$ es un anillo conmutativo unitario; es un cuerpo
	porque $I=0$ es un ideal maximal.
\end{proof}

\begin{example}
	$\Z$ no es primitivo pues no es un cuerpo.
\end{example}

\begin{definition}
	Un ideal $P$ de un anillo $R$ se dice \textbf{primitivo} si $P=\Ann_R(M)$
	para algún $R$-módulo $M$ simple.
\end{definition}

\begin{lemma}
	\label{lemma:primitivo}
	Sea $R$ un anillo y sea $P$ un ideal de $R$. Entonces $P$ es primitivo si y
	sólo si $R/P$ es un anillo primitivo.
\end{lemma}

\begin{proof}
	Supongamos que $P=\Ann_R(M)$ para algún $R$-módulo $M$. Entonces $M$ es un
	$R/P$-módulo simple con la acción $(r+P)m=rm$, $r\in R$, $m\in M$; la buena
	definición se obtiene al observar que $P=\Ann_R(M)$ y la simplicidad se
	obtiene de la simplicidad de $M$. Además $\Ann_{R/P}M=0$ pues  si
	$(r+P)M=0$ entonces $r\in\Ann_RM=P$ y luego $r+P=P$.

	Supongamos ahora que $R/P$ es primitivo. Sea $M$ un $R/P$-módulo fiel y
	simple.  Entonces $M$ es un $R$-módulo con la acción $rm=(r+P)m$, $r\in R$,
	$m\in M$. Es trivial verificar que $M$ es simple y que $P=\Ann_R(M)$. 
\end{proof}

%\begin{example}
%	Si $I$ es un ideal maximal de un anillo unitario $R$, entonces $I$ es
%	primitivo. Como $I$ es ideal maximal y regular (pues $1\in R$), el cociente
%	$R/I$ es un anillo unitario simple y luego $R/I$ es primitivo por la
%	proposición~\ref{proposition:simple=>prim}. 
%\end{example}

%\begin{example}
%	Si $I$ es un ideal primitivo de un anillo conmutativo $R$, entonces $I$ es
%	maximal pues $R/I$ es un cuerpo (por ser primitivo y conmutativo), ver
%	proposición~\ref{proposition:prim+conm=cuerpo}.
%\end{example}

\begin{example}
	Sean $R_1,\dots,R_n$ anillos primitivos y sea $R=R_1\times\cdots\times
	R_n$. Entonces cada $P_i=R_1\times\cdots\times R_{i-1}\times 0\times
	R_{i+1}\times\cdots\times R_n$ es un ideal primitivo de $R$ pues
	$R/P_i\simeq R_i$.
\end{example}

%Recordemos que un ideal a izquierda $L$ de $R$ se dice \textbf{minimal} si
%$L\ne0$ y $L$ no contiene propiamenete a otros ideales a izquierda no nulos de
%$R$.
%
%\begin{example}
%	Sea $L$ un ideal a izquierda de $R$ tal que $RL\ne0$. Entonoces $L$ es
%	simple si y sólo si $L$ es minimal.
%\end{example}

\begin{lemma}
	\label{lemma:maxprim}
	Sea $R$ un anillo. Si $P$ es un ideal primitivo, existe un ideal a
	izquierda $L$ maximal tal que $P=\{x\in R:xR\subseteq L\}$.
	Recíprocamente, si $L$ es un ideal a izquierda maximal y regular, entonces
	$\{x\in R:xR\subseteq L\}$ es un ideal primitivo.
\end{lemma}

\begin{proof}
	Supongamos que $P=\Ann_R(M)$, donde $M$ es $R$-módulo simple. Por la
	proposición~\ref{proposition:R/I} sabemos que existe un ideal a izquierda
	$L$, maximal y regular, tal que $M\simeq R/L$. Luego $P=\Ann_R(R/L)=\{x\in
	R:xR\subseteq L\}$. 

	Recíprocamente, sea $L$ un ideal a izquierda maximal y regular. Por la
	proposición~\ref{proposition:R/I}, $R/L$ es un $R$-módulo simple. Por
	definición, 
	\[
	\Ann_R(R/L)=\{x\in R:xR\subseteq L\}
	\]
	es un ideal primitivo.
\end{proof}

%\begin{remark}
%	Una consecuencia trivial del lema~\ref{lemma:maxprim} es la siguiente: en
%	un anillo unitario, todo ideal a izquierda maximal contiene un ideal
%	primitivo.
%\end{remark}

\begin{proposition}
	Todo ideal maximal en un anillo unitario es primitivo. 
\end{proposition}

\begin{proof}
	Si $R$ es un anillo unitario y $M$ es un ideal maximal de $R$, entonces
	$R/M$ es un anillo unitario simple por la
	proposición~\ref{proposition:R/I}. Luego $R/M$ es primitivo por la
	proposición~\ref{proposition:simple=>prim} y entonces $M$ es primitivo por
	el lema~\ref{lemma:primitivo}. 
\end{proof}

\begin{exercise}
	Demuestre que en un anillo conmutativo todo ideal primitivo es maximal.
\end{exercise}

\begin{exercise}
	Demuestre que $M_n(R)$ es primitivo si y sólo si $R$ es primitivo.
\end{exercise}

% Si $P$ es primitivo, entonces $R/P$ es un cuerpo(por ser primitivo y conmutativo) y luego $P$ es maximal

Let us discuss the Jacobson radical and radical rings. 

Let $R$ be a rng. The \textbf{Jacobson radical} $J(R)$
is the intersection of all the annihilators of simple left $R$-modules. If $R$ does not
have simple left $R$-modules, then $J(R)=R$. From the definition it follows
that $J(R)$ is an ideal. Moreover, 
	\[
		J(R)=\bigcap\{P:\text{$P$ left primitive ideal}\}.
	\]

Si $I$ es un ideal de un anillo $R$, se define $I^n$ como el subgrupo aditivo
generado por el conjunto $\{y_1\dots y_n:y_j\in I\}$. 

\begin{definition}
	\index{Ideal!nilpotente}
	\index{Elemento!nil}
	\index{Elemento!nilpotente}
	Un ideal $I$ de un anillo $R$ se dice \textbf{nilpotente} si $I^n=0$ para
	algún $n\in\N$.
\end{definition}

Análogamente pueden definirse ideales a izquierda (o a derecha) nil.

\begin{definition}
	\index{Elemento!nil}
	\index{Elemento!nilpotente}
	Un elemento $x$ de un anillo $R$ se dice \textbf{nil} (o nilpotente) si
	existe $n\in\N$ tal que $x^n=0$. 
\end{definition}

\begin{definition}
	\index{Ideal!nil}
	Un ideal $I$ de un anillo $R$ se dice \textbf{nil} si todo elemento de $I$
	es nil.
\end{definition}

Observemos que un ideal $I$ es nilpotente si y sólo si existe $n\in\N$ tal que
$x_1x_2\cdots x_n=0$ para todo $x_1,\dots,x_n\in I$.  Obviamente, todo ideal
nilpotente es nil pues si $I^n=0$ entonces, en particular, $x^n=0$ para todo
$x\in I$.

\begin{example}
	Sea $R=\C[x_1,x_2,\dots]/(x_1,x_2^2,x_3^3,\dots)$. El ideal
	$I=(x_1,x_2,x_3,\dots)$ es nil en $R$ pues está generado por elementos
	nilpotentes pero no es nilpotente. Si lo fuera, existiría $k\in\N$ tal que
	$I^k=0$, y luego $x_i^k=0$ para todo $i$, una contradicción pues
	$x_{k+1}^k\ne0$. 	
\end{example}

\begin{proposition}
	\label{pro:nilJ}
	Sea $R$ un anillo. Entonces $J(R)$ contiene a todo ideal a izquierda (resp.
	a derecha) nil.
\end{proposition}

\begin{proof}
	Supongamos que existe un ideal a izquierda (resp. a derecha) nil $I$ tal que
	$I\not\subseteq J(R)$. Entonces existe un $R$-módulo simple $M$ tal que
	$n=xm\ne 0$ para algún $x\in I$ y algún $m\in M$. Como $M$ es simple,
	$Rn=M$ y luego existe $r\in R$ tal que $(rx)m=r(xm)=rn=m$ (resp.
	$(xr)n=x(rn)=xm=n$). Esto implica que $(rx)^km=m$ (resp. $(xr)^kn=n$) para
	todo $k\geq1$, una contradicción pues $rx\in I$ (resp. $xr\in I$) es un
	elemento nilpotente. 
\end{proof}


\begin{definition}
	\index{Casi-regular a izquierda}
	\index{Casi-regular a derecha}
	Sea $R$ un anillo y sea $a\in R$. El elemento $a\in R$ se dice
	\textbf{casi-regular a izquierda} si existe $r\in R$ tal que $r+a+ra=0$; y
	se dice \textbf{casi-regular a derecha} si existe $r\in R$ tal que $a+r+ar=0$. 
\end{definition}

\begin{exercise}
	\label{exercise:circ}
	Sea $R$ un anillo. Demuestre que $R\times R\to R$,
	$(r,s)\mapsto r\circ s=r+s+rs$, es una operación asociativa con neutro $0$.
\end{exercise}

\begin{example}
	Sea $R=\Z/3=\{0,1,2\}$. Entonces
	\begin{table}
		\centering
		\begin{tabular}{c|ccc}
			$\circ$ & 0 & 1 & 2\tabularnewline
			\hline
			0 & 0 & 1 & 2\tabularnewline
			1 & 1 & 0 & 2\tabularnewline
			2 & 2 & 2 & 2\tabularnewline
		\end{tabular}
	\end{table}
\end{example}

\begin{remark}
	\label{remark:eq1inR}
	Si $R$ es unitario y $x\in R$ es casi-regular a izquierda (resp. a derecha)
	si y sólo si $1+x$ es inversible a izquierda (resp. a derecha). En efecto,
	si existe $r\in R$ tal que $r+x+rx=0$ entonces $(1+r)(1+x)=1+r+x+rx=1$.
	Recíprocamente, si existe $y\in R$ tal que $y(1+x)=1$ entonces 
	\[
	(y-1)\circ x=y-1+x+(y-1)x=0.
	\]
\end{remark}

\begin{example}
	Si $x\in R$ es un elemento nilpotente, entonces $y=\sum_{n\geq1}x^n\in R$
	es casi-regular. En efecto, si existe $N$ tal que $x^N=0$, la suma que
	define al elemento $y$ es finita y cumple que $y+(-x)+y(-x)=0$.  
\end{example}

Un ideal (a izquierda, a derecha o bilátero) $I$ de un anillo $R$ se dice
\textbf{casi-regular a izquierda} (resp. a derecha) si todo elemento de $I$ es
casi-regular a izquierda (resp. a derecha). Un ideal (a izquierda, a derecha o
bilátero) se dice \textbf{casi-regular} si es casi-regular a izquierda y a
derecha. 

\begin{lemma}
	\label{lemma:casiregular}
	Sea $I$ un ideal a izquierda de un anillo $R$. Si $I$ es casi-regular a
	izquierda, $I$ es casi-regular.
\end{lemma}

\begin{proof}
	Sea $x\in I$. Veamos que $x$ es casi-regular a derecha.  Como $I$ es
	casi-regular a izquierda, existe $r\in R$ tal que $r\circ x=r+x+rx=0$. Como
	$r=-x-rx\in I$, existe $s\in R$ tal que $s\circ
	r=s+r+sr=0$. Luego $s$ es casi-regular a derecha y 
	\[
	x=0\circ x=(s\circ r)\circ x=s\circ (r\circ x)=s\circ 0=s.
	\]
\end{proof}

\index{Lema de Zorn}
Sea $(A,\leq)$ un conjunto parcialmente ordenado, es decir $A$ es un conjunto
con una relación $R$ en $A\times A$ reflexiva, transitiva y antisimétrica.
(Reflexiva: $a\leq a$ para todo $a\in A$; transitiva: si $a\leq b$ y $b\leq c$
entonces $a\leq c$; y antisimétrica: si $a\leq b$ y $b\leq a$ entonces $a=b$.)
Dos elementos $a,b\in A$ se dicen \textbf{comparables} si $a\leq b$ o $b\leq
a$. Un elemento $a\in A$ se dice \textbf{maximal} si para todo $c\in A$
comparable con $a$, se tiene $c\leq a$. 
Una \textbf{cota superior} para un $B\subseteq A$ no vacío es un elemento $d\in
A$ tal que $b\leq d$ para todo $b\in B$.  Una \textbf{cadena} en $A$ es un
subconjunto $B$ tal que todo par de elementos de $B$ es comparable. 
El \textbf{lema de Zorn} afirma lo siguiente: 
\begin{quote}
Si $A$ es un conjunto no vacío parcialmente ordenado tal que toda cadena de
$A$ posee una cota superior en $A$, entonces $A$ posee elemento maximal.
\end{quote}
Usaremos el lema de Zorn para demostrar el siguiente lema:

\begin{lemma}
	\label{lemma:maxreg}
	Sea $R$ un anillo y sea $x\in R$ un elemento que no es casi-regular a
	izquierda. Entonces existe un ideal a izquierda $M$ maximal tal que
	$x\not\in M$. Más aún, $R/M$ es un $R$-módulo simple y
	$x\not\in\Ann_R(R/M)$.
\end{lemma}

\begin{proof}
	Sea $T=\{r+rx:r\in R\}$. Es fácil verificar que $T$ es un ideal a izquierda
	de $R$ tal que $x\not\in T$ (pues si $x\in T$ entonces $r+rx=-x$ para algún
	$r\in R$, una contradicción pues $x$ no es casi-regular a izquierda). 

	El único ideal a izquierda de $R$ que contiene
	a $T\cup\{x\}$ es $R$:  si existe un ideal a izquierda $U$ que contiene a $T$
	entonces $x\not\in U$ pues, de lo contrario, todo $r\in R$ podría
	escribirse como $r=(r+rx)+r(-x)\in U$. 

	Sea $\mathcal{S}$ el conjunto de ideales propios de $R$ que contienen a
	$T$, parcialmente ordenado por la inclusión. Si $\{K_i:i\in I\}$ es una
	cadena en $\mathcal{S}$, entonces $K=\cup_{i\in I}K_i$ es una cota superior
	para la cadena ($K$ es un ideal propio pues $x\not\in K$). Por el lema de
	Zorn, $\mathcal{S}$ admite un elemento maximal, digamos $M$. Entonces $M$
	es un ideal a izquierda maximal tal que $x\not\in M$. Además $M$ es regular
	pues $r+r(-x)\in T\subseteq M$ para todo $r\in R$. Luego $R/M$ es un
	$R$-módulo simple por la proposición~\ref{proposition:R/I}. Como $x(x+M)\ne
	0$ (pues si $x^2\in M$ entonces $x\in M$ al observar que $x+x^2\in
	T\subseteq M$), que $x\not\in\Ann_R(R/M)$.
\end{proof}

\begin{remark}
	\label{remark:xJ(R)}
	Si $x\in R$ no es casi-regular a izquierda, por el lema~\ref{lemma:maxreg}
	existe un $R$-módulo simple $M$ tal que $x\not\in\Ann_R(M)$. Luego
	$x\not\in J(R)$.
\end{remark}

\begin{theorem}
	\label{thm:casireg_eq}
	Sean $R$ un anillo y $x\in R$. Son
	equivalentes:
	\begin{enumerate}
		\item El ideal a izquierda generado por $x$ es casi-regular.
		\item $Rx$ es casi-regular.
		\item $x\in J(R)$.
	\end{enumerate}
\end{theorem}

\begin{proof}
	La implicación $(1)\implies(2)$ es trivial pues $Rx$ está contenido en el
	ideal a izquierda generado por $x$.  Demostremos que $(2)\implies(3)$. Si
	$x\not\in J(R)$ entonces por el lema~\ref{lemma:maxreg} existe un
	$R$-módulo simple $M$ tal que $xm\ne 0$ para algún $m\in M$. Por la
	simplicidad, $R(xm)=M$ y luego existe $r\in R$ tal que $rxm=-m$. Existe
	$s\in R$ tal que $s+rx+s(rx)=0$ y entonces
	\[
	-m=rxm=(-s-srx)m=-sm+sm=0,
	\]
	una contradicción. Por último, para demostrar que $(3)\implies(1)$ basta
	observar que $x$ es casi-regular a izquierda por la
	observación~\ref{remark:xJ(R)} y luego el ideal a izquierda generaod por
	$x$ es casi-regular por el lema~\ref{lemma:casiregular}.
\end{proof}

\begin{corollary}
	Si $R$ es un anillo, $J(R)$ es un ideal casi-regular que contiene a todo
	ideal casi-regular a izquierda.
\end{corollary}

\begin{proof}
	Es consecuencia inmediata del teorema~\ref{thm:casireg_eq}. 
%%	Como \[
%%	J(R)=\bigcap\{\Ann_R(R/I):I\text{ ideal a izquierda maximal y
%%	regular}\}
%%	\]
%%	por el teorema~\ref{thm:J(R)}, 
%%	todo $x\in J(R)$ es casi-regular gracias al
%%	lema~\ref{lemma:K}. Si $I$ es un ideal casi-regular a izquierda, entonces
%%	$I\subseteq J(R)$ por el lema~\ref{lemma:JsupsetCR}.
\end{proof}

%\begin{lemma}
%%	\label{lemma:maxreg}
%	Sea $R$ un anillo y sea $I\ne R$ un ideal a izquierda regular. Entonces $I$
%	está contenido en algún ideal a izquierda maximal y regular.
%\end{lemma}
%
%\begin{proof}
%	
%\end{proof}

%\begin{lemma}
%	\label{lemma:K}
%	Sean $R$ un anillo y $K=\bigcap\{I:\text{$I$ ideal a izquierda maximal y
%	regular}\}$.  Entonces $K$ es un ideal a izquierda casi-regular.
%\end{lemma}
%
%\begin{proof}
%	Gracias al lema~\ref{lemma:casiregular}, basta ver que $K$ es casi-regular
%	a izquierda.  Sea $a\in K$ y sea $T=\{r+ra:r\in R\}$.  Es claro que $T$ es
%	un ideal a izquierda regular con $e=-a$. Como $a\in K$, existe $s\in R$ tal
%	que $s+a+sa=0$. Entonces, como $-a=s+sa$, 
%	\[
%	r+re=r+r(-a)=r+r(s+sa)=r+rs+rsa\in r+T
%	\]
%	para todo $r\in R$. 
%	Si $T\ne R$, por el lema~\ref{lemma:maxreg} existe un ideal a izquierda
%	$J$, maximal y regular.  Como $a\in K\subseteq J$, $J$ es ideal a izquierda
%	y $r+ra\in T\subseteq J$ para todo $r\in R$, se concluye que $R=J$, una
%	contradicción.  Luego $T=R$ y entonces existe $r\in R$ tal que $r+ra=-a$.
%	Esto implica que $a$ es casi-regular a izquierda. 
%\end{proof}

%\begin{lemma}
%	\label{lemma:JsupsetCR}
%	Sea $R$ un anillo que admite un $R$-módulo simple. Si $I$ es un ideal a
%	izquierda casi-regular a izquierda, $I\subseteq J(R)$.
%\end{lemma}
%
%\begin{proof}
%	Supongamos que $I\not\subseteq J(R)$. Existe entonces un $R$-módulo simple
%	$N$ tal que $IN\ne 0$. Entonces $In\ne 0$ para algún $0\ne n\in N$. Como
%	$I$ es un ideal a izquierda, $In\subseteq N$ es un submódulo no nulo del
%	simple $N$. Luego $In=N$. Existe entonces $x\in I$ tal que $xn=-n$. Como
%	$I$ es casi-regular a izquierda, existe $r\in R$ tal que $r+x+rx=0$.
%	Entonces
%	\[
%		0=0n=(r+x+rx)n=rn+xn+rxn=rn-n-rn=-n,
%	\]
%	una contradicción.
%\end{proof}

\begin{theorem}
	\label{thm:J(R)}
	Sea $R$ un anillo tal que $J(R)\ne R$.  Entonces 
	\begin{align*}
		J(R)&=\bigcap\{I:\text{$I$ ideal a izquierda maximal y regular}\}.
	\end{align*}
\end{theorem}

\begin{proof}
	Sea $K=\bigcap\{I:\text{$I$ ideal a izquierda maximal y regular}\}$. Por la
	proposición~\ref{proposition:R/I} podemos escribir
	\[
		J(R)=\bigcap\{\Ann_R(R/I):I\text{ ideal a izquierda maximal y regular}\}.
	\]
	Sea $I$ un ideal a izquierda maximal y regular. Si $r\in J(R)\subseteq
	\Ann_R(R/I)$ entonces, como $I$ es regular, existe $e\in R$ tal que
	$r-re\in I$. Como
	\[
	re+I=r(e+I)=0,
	\]
	$re\in I$ y por lo tanto $r\in I$. Luego $J(R)\subseteq K$. 
	%Veamos que
	%$K\subseteq J(R)$: por el lema~\ref{lemma:K}, $K$ es casi-regular a
	%izquierda y luego $K\subseteq J(R)$.% por el lema~\ref{lemma:JsupsetCR}. 
	La otra implicación es trivial.
\end{proof}

\begin{example}
	Todo ideal maximal de $\Z$ es de la forma $p\Z=\{pm:m\in\Z\}$, donde $p$ es algún número
	primo. Entonces $J(\Z)=\cap_p p\Z=\{0\}$.
\end{example}

%\begin{example}
%	Sea $D$ un anillo de división y sea $R=D[x_1,\dots,x_n]$. Si $f\in J(R)$
%	entonces\dots Luego $J(R)=0$. 
%\end{example}

Veamos algunos resultados que nos permitirán calcular radicales. 

\begin{proposition}
	Sea $\{R_i:i\in I\}$ una familia de anillos. Entonces 
	\[
	J\left(\prod_{i\in I}R_i\right)=\prod_{i\in I}J(R_i).
	\]
\end{proposition}

\begin{proof}
	Sea $R=\prod_{i\in I}R_i$ y sea $x=(x_i)_{i\in I}\in R$.  El ideal a
	izquierda $Rx$ es casi-regular si y sólo si cada ideal a izquierda $R_ix_i$
	es casi-regular en $R_i$ (pues  $x$ casi-regular en $R$ si y sólo si cada
	$x_i$ es casi-regular en $R_i$.  Luego $x\in J(R)$ si y sólo si $x_i\in
	J(R_i)$ para todo $i\in I$.
\end{proof}

\begin{lemma}
	\label{lemma:trickJ1}
	Sea $R$ un anillo y sea $x\in R$.  Si $-x^2$ es un elemento casi-regular a
	izquierda, $x$ también. 
\end{lemma}

\begin{proof}
	Sea $r\in R$ tal que $r+(-x^2)+r(-x^2)=0$ y sea $s=r-x-rx$. Entonces $x$ es
	casi-regular a izquierda pues 
	\begin{align*}
		s+x+sx&=(r-x-rx)+x+(r-x-rx)x\\
		&=r-x-rx+x+rx-x^2-rx^2=r-x^2-rx^2=0.\qedhere 
\end{align*}
\end{proof}

%\begin{lemma}
%	\label{lemma:trickJ2}
%	Sea $R$ un anillo. Entonces $x\in J(R)$ si y sólo si $Rx$ es un ideal a
%	izquierda casi-regular a izquierda.
%\end{lemma}
%
%\begin{proof}
%	Si $x\in J(R)$ entonces $Rx\subseteq J(R)$ y luego todo elemento de $Rx$ es
%	casi-regular a izquierda. Recíprocamente, si $Rx$ es casi-regular a
%	izquierda, $Rx+\Z x$ es un ideal a izquierda de $R$. Si $s=rx+nx\in Rx+\Z
%	x$, entonces $-s^2$ es casi-regular a izquierda (pues $-s^2\in Rx$). Por el
%	lema~\ref{lemma:trickJ1}, $s$ es casi-regular a izquierda; en particular,
%	$x$ es casi-regular a izquierda y luego $x\in J(R)$. 
%\end{proof}

\begin{proposition}
	\label{proposition:J(I)}
	Si $I$ es un ideal de un anillo $R$, entonces $J(I)=I\cap J(R)$. 
\end{proposition}

\begin{proof}
	Como $I\cap J(R)$ es un ideal de $I$, si $x\in I\cap J(R)$ entonces $x$ es
	casi-regular a izquierda en $R$. Sea $r\in R$ tal que $r+x+rx=0$. 
	Como $r=-x-rx\in I$, el elemento $x$ es casi-regular a 
	izquierda en $I$. Luego $I\cap J(R)\subseteq J(I)$. 

	Sea ahora $x\in J(I)$ y sea $r\in R$. Como $-(rx)^2=(-rxr)x\in
	I(J(I))\subseteq J(I)$, el elemento $-(rx)^2$ es casi regular a izquierda
	en $I$. Luego $rx$ es casi regular a izquierda gracias al
	lema~\ref{lemma:trickJ1}.
\end{proof}

\begin{definition}
	\index{Anillo!radical}
	Un anillo $R$ se dice \textbf{radical} si $J(R)=R$.
\end{definition}

\begin{example}
	Si $R$ es un anillo, la proposición~\ref{proposition:J(I)} implica que
	$J(R)$ es un anillo radical.
\end{example}

\begin{theorem}
	\label{theorem:anillo_radical}
	Sea $R$ un anillo. Son equivalentes:
	\begin{enumerate}
		\item $R$ es radical.
		\item $R$ no admite $R$-módulos simples.
		\item $R$ no tiene ideales a izquierda maximales y regulares.
		\item $R$ no tiene ideales a izquierda primitivos.
		\item Todo elemento de $R$ es casi-regular.
		\item $(R,\circ)$ es un grupo.
	\end{enumerate}
\end{theorem}

\begin{proof}
	La equivalencia $(1)\Longleftrightarrow(5)$ es consecuencia del
	teorema~\ref{thm:casireg_eq}. La equivalencia $(5)\Longleftrightarrow(6)$
	es trivial por el ejercicio~\ref{exercise:circ}.

	Probemos que $(1)\implies(2)$. Supongamos que existe un $R$-módulo simple
	$N$.  Como $R=J(R)\subseteq\Ann_R(N)$, $R=\Ann_S(N)$ y luego $RN=0$, una
	contradicción pues $N$ es simple. 
	
	Para demostrar que $(2)\implies(3)$ basta observar que para todo ideal a
	izquierda $I$ maximal y regular, $R/I$ es un $R$-módulo simple por la
	proposición~\ref{proposition:R/I}. 
	
	Para ver que $(3)\implies(4)$ supongamos que existe un ideal a izquierda
	primitivo, digamos $I=\Ann_R(M)$, donde $M$ es algún $R$-módulo simple.
	Entonces $I=R$ pues $R=J(R)\subseteq I$, una contradicción a la simplicidad
	de $M$ pues entonces $RM=0$.

	La implicación $(4)\implies(2)$ es fácil: si $M$ es un $R$-módulo simple,
	entonces $\Ann_R(M)$ es un ideal a izquierda primitivo.
\end{proof}

\begin{example}
	Sea $A=\left\{\frac{2x}{2y+1}:x,y\in\Z\right\}$. Entonces $A$ es un anillo radical pues
	pues el inverso de un elemento $\frac{2x}{2y+1}$ con respecto a la operación $\circ$ es 
	\[
	\left(\frac{2x}{2y+1}\right)'=\frac{-2x}{2(x+y)+1}.
	\]
\end{example}

\begin{example}
	El radical de $\Z/8$ es $\{0,2,4,6\}$. 
\end{example}

\begin{theorem}
	\label{thm:Jnilpotente}
	Si $R$ es un anillo artiniano a izquierda, $J(R)$ es nilpotente. 
\end{theorem}

\begin{proof}
	Sea $J=J(R)$. Como $R$ es artiniano a izquierda, la sucesión
	$(J^m)_{m\in\N}$ de ideales a izquierda se estabiliza. Existe entonces
	$k\in\N$ tal que $J^k=J^l$ para todo $l\geq k$. Veamos que $J^k=0$. Si
	$J^k\ne0$ sea $\mathcal{S}$ el conjunto de todos los ideales a izquierda
	$I$ tales que $J^kI\ne0$. Como $J^kJ^k=J^{2k}=J^k\ne 0$, $\mathcal{S}$ es
	no vacío. Como $R$ es artiniano a izquierda, $\mathcal{S}$ posee un
	elemento minimal, digamos $I_0$. Como $J^kI_0\ne 0$, sea $x\in
	I_0\setminus\{0\}$ tal que $J^kx\ne 0$. Además $J^kx$ es un ideal a
	izquierda de $R$ contenido en $I_0$ tal que $J^kx\in\mathcal{S}$ (pues
	$J^k(J^kx)=J^{2k}x=J^kx\ne 0)$. Por la minimalidad de $I_0$, $J^kx=I_0$. En
	particular, existe $r\in J^k\subseteq J(R)$ tal que $rx=x$. Como $-r\in
	J(R)$ es casi-regular a izquierda, existe $s\in R$ tal que $s-r-sr=0$.
	Luego
	\[
		x=rx=(s-sr)x=sx-s(rx)=sx-sx=0,
	\]
	una contradicción.
\end{proof}

\begin{corollary}
	Sea $R$ un anillo artiniano a izquierda.  Todo ideal a izquierda (resp.
	derecha) nil es nilpotente y $J(R)$ es el único ideal maximal nilpotente de
	$R$.
\end{corollary}

\begin{proof}
	Sea $L$ un ideal a izquierda nil.  Por la proposición~\ref{pro:nilJ}, $L$
	está contenido en $J(R)$. Luego $L$ es nilpotente pues $J(R)$ es nilpotente
	por el teorema~\ref{thm:Jnilpotente}. 
\end{proof}

\begin{theorem}
	Sea $R$ un anillo y sea $n\in\N$. Entonces $J(M_n(R))=M_n(J(R))$. 
\end{theorem}

\begin{proof}
	Demostremos primero que $J(M_n(R))\subseteq M_n(J(R))$. 
	Si $J(R)=R$, no hay nada para demostrar. Supongamos entonces que $J(R)\ne R$ y sea 
	$J=J(R)$. 
	Si $M$ es un $R$-módulo simple, $M^n$ es un $M_n(R)$-módulo simple con la multiplicación usual.
	Sean $x=(x_{ij})\in J(M_n(R))$ y $m_1,\dots,m_n\in M$. Entonces
	\[
		x\colvec{3}{m_1}{\vdots}{m_n}=0.
	\]
	En particular, $x_{ij}\in\Ann_R(M)$ para todo $i,j\in\{1,\dots,n\}$ y luego
	$x\in M_n(J)$. 

	Veamos ahora que $M_n(J)\subseteq J(M_n(R))$. Sean 
	\[
		J_1=\begin{pmatrix}
			J & 0 & \cdots & 0\\
			J & 0 & \cdots & 0\\
			\vdots & \vdots & \ddots & \vdots\\
			J & 0 & \cdots & 0
		\end{pmatrix}
		\quad\text{y}\quad
		x=\begin{pmatrix}
			x_1 & 0 & \cdots & 0\\
			x_2 & 0 & \cdots & 0\\
			\vdots & \vdots & \ddots & \vdots\\
			x_n & 0 & \cdots & 0
		\end{pmatrix}\in J_1.
	\]
	Como $x_1$ es casi-regular, existe $y_1\in R$ tal que $x_1+y_1+x_1y_1=0$.
	Si
	\[
		y=\begin{pmatrix}
			y_1 & 0 & \cdots & 0\\
			0 & 0 & \cdots & 0\\
			\vdots & \vdots & \ddots & \vdots\\
			0 & 0 & \cdots & 0
		\end{pmatrix}, 
	\]
	entonces $u=x+y+xy$ es triangular inferior pues 
	\[
		u=\begin{pmatrix}
			0 & 0 & \cdots & 0\\
			x_2y_1 & 0 & \cdots & 0\\
			x_3y_1 & 0 & \cdots & 0\\
			\vdots & \vdots & \ddots & \vdots\\
			x_ny_1 & 0 & \cdots & 0
		\end{pmatrix}.
	\]
	Como 
	$u^n=0$, el elemento 
	\[
	v=-u+u^2-u^3+\cdots+(-1)^{n-1} u^{n-1}
	\]
	cumple que
	$u+v+uv=0$. Luego $x$ es casi-regular a derecha pues 
	\begin{align*}
		x+(y+v+yv)+x(y+v+yv)&=0,
	\end{align*}
	y entonces $J_1$ es casi-regular a derecha. Similarmente se demuestra que
	todo $J_i$ es casi-regular a derecha y luego $J_i\subseteq J(M_n(R))$ para
	todo $i\in\{1,\dots,n\}$. Se concluye entonces que $J_1+\cdots+J_n\subseteq
	J(M_n(R))$ y luego $M_n(J)\subseteq J(M_n(R))$ 
\end{proof}

For completeness we recall basic results on the Jacobson radical in the case
of unitary rings. 

\begin{exercise}
	Let $R$ be a unitary ring. Then  
	\[
	J(R)=\bigcap\{M:\text{$M$ is a left maximal ideal}\}.
	\]
\end{exercise}

\begin{exercise}
	Let $R$ be a unitary ring. The
	following statements are equivalent: 
	\begin{enumerate}
		\item $x\in J(R)$.
		\item $xM=0$ for all simple $R$-module $M$.
		\item $x\in P$ for all primitive left ideal $P$.
		\item $1+rx$ is invertible for all $r\in R$.
		\item $1+\sum_{i=1}^n r_ixs_i$ is invertible for all $n\in\N$ and all $r_i,s_i\in R$.
		\item $x$ belongs to every left maximal ideal maximal. 
	\end{enumerate}
\end{exercise}

Now...

\begin{definition}
\index{Solution!involutive}
A solution $(X,r)$ is said to be \emph{involutive} if $r^2=\id$. 
\end{definition}

\index{Symmetric group}
For $n\geq2$, the \emph{symmetric group} $\Sym_n$ can be presented 
as the group with generators $\sigma_1,\dots,\sigma_{n-1}$ and relations
\begin{align*}
    &\sigma_i\sigma_{i+1}\sigma_i=\sigma_{i+1}\sigma_i\sigma_{i+1} && \text{if }1\leq i\leq n-2,\\
    &\sigma_i\sigma_j=\sigma_j\sigma_i && \text{if }|i-j|\geq 1,\\
    &\sigma_i^2=1 && \text{for all $i\in\{1,\dots,n-1\}$}.
\end{align*}
Let $(X,r)$ be an involutive solution. 
Then the map $\sigma_i\mapsto r_{i,i+1}=\id_{X^{i-1}}\times r\times\id_{X^{n-i-1}}$ extends 
to an action of $\Sym_n$ on $X^n$.

\begin{example}
Let $X$ be a non-empty set and $\sigma$ be a bijection on $X$. Then 
$(X,r)$, where $r(x,y)=(\sigma(y),\sigma^{-1}(x))$, is an involutive solution. 
\end{example}

\index{Jacobson!radical ring}
\index{Radical ring}
We now present a very important family of involutive solutions. 
These examples show an intriguing connection between the YBE and the 
theory of non-commutative rings. 


In any ring $R$ the \emph{circle operation} 
\[
x\circ y=x+xy+y
\]
is always associative and such that $x\circ 0=0\circ x=x$ for all $x\in R$. 
A non-unital ring (or \emph{rng}, for short) 
$R$ is said to be a \emph{radical ring} if $(R,\circ)$ is a group. 
In this case, following Jacobson's notation, the inverse of 
an element $x$ with respect to the circle operation is denoted by $x'$. 

Radical rings were introduced by Jacobson in~\cite{MR12271}.
There are other characterizations of radical rings, see for example~\cite{MR3308118}.

\begin{example}
Let $p$ be a prime and let $A=\Z/(p^2)$ be the cylic additive group of order $p^2$. 
The operation $x\circ y=x+y+pxy$ 
turns $A$ into a radical ring. 
\end{example}

\begin{example}
Let $A=\left\{\frac{2x}{2y+1}:x,y\in\Z\right\}$. The operation 
$a\circ b=a+b+ab$ turns $A$ into a radical ring. A straightforward computation shows that 
\[
\left(\frac{x}{2y+1}\right)'=\frac{2(-x)}{2(x+y)+1}.
\]
\end{example}

The following fundamental family of solutions appears in~\cite{MR2278047}. 
It turns out to be fundamental in the study of 
set-theoretic solutions to the YBE. 

\begin{proposition}
\label{pro:Rump}
Let $R$ be a radical ring. Then $(R,r)$, where 
\[
r(x,y)=( -x+x\circ y,(-x+x\circ y)'\circ x\circ y)
\]
is an involutive solution.
\end{proposition}

The proposition can demonstrated using Theorem~\ref{thm:LYZ}, 
see Exercise~\ref{prob:Rump}. 
We will prove a stronger result in Theorem~\ref{thm:YB}. 

\begin{proposition}
\label{pro:T}
Let $(X,r)$ be an involutive solution. 
Then the map $T\colon X\to X$, $x\mapsto\sigma_x^{-1}(x)$, is 
invertible with inverse $T^{-1}(y)=\tau^{-1}_y(y)$ and 
\[
T^{-1}\circ\sigma_x^{-1}\circ T=\tau_x
\]
for all $x\in X$. 
\end{proposition}

\begin{proof}
Let $U(x)=\tau_x^{-1}(x)$. Since $r$ is involutive, 
\[
(U(x),x)=r^2(U(x),x)=r(\sigma_{U(x)}(x),x)=(\sigma_{\sigma_{U(x)}(x)}(x),\tau_x\sigma_{U(x)}(x)).
\]
The second coordinate can be written as $U(x)=\sigma_{U(x)}(x)$. This 
implies that 
\[
T(U(x))=\sigma^{-1}_{U(x)}(U(x))=x.
\]
Similarly one obtains $U(T(x))=x$. 

Since $(X,r)$ is a solution, Lemma~\ref{lem:YB} implies that 
$\sigma_x\sigma_y=\sigma_{\sigma_x(y)}\sigma_{\tau_y(x)}$
%$\tau_{\tau_y(x)}\circ\tau_{\sigma_x(y)}=\tau_y\circ\tau_x$ 
holds for all $x,y\in X$. Then  
\[
\sigma_y^{-1}T(x)
=\sigma_y^{-1}\sigma_x^{-1}(x)
=\sigma^{-1}_{\tau_y(x)}\sigma^{-1}_{\sigma_x(y)}(x)
=\sigma^{-1}_{\tau_y(x)}\tau_y(x)
=T\tau_y(x)
\]
for all $y\in X$, by Equality~\eqref{eq:involutive}.
\end{proof}

Note that if $(X,r)$ is a non-degenerate involutive solution, then 
\[
(x,y)=r^2(x,y)=r(\sigma_x(y),\tau_y(x))=(\sigma_{\sigma_x(y)}\tau_y(x),\tau_{\tau_y(x)}\sigma_x(y)).
\]
Hence 
\begin{equation}
    \label{eq:involutive}
    \tau_y(x)=\sigma_{\sigma_x(y)}^{-1}(x),
    \quad
    \sigma_x(y)=\tau_{\tau_y(x)}^{-1}(y)
\end{equation}
for all $x,y\in X$. Thus for involutive solutions
it is enough to know $\{\sigma_x:x\in X\}$, as from this we obtain the
set $\{\tau_x:x\in X\}$. 


\begin{definition}
\index{Cycle set}
\index{Cycle set!non-degenerate}
A \emph{cycle set} is a pair $(X,\cdot)$, where $X$ is a non-empty 
set provided with a binary operation $X\times X\to X$, $(x,y)\mapsto x\cdot y$, 
such that 
\begin{equation}
    \label{eq:cycle_set}
    (x\cdot y)\cdot (x\cdot z)=(y\cdot x)\cdot (y\cdot z)
\end{equation}
holds for all $x,y,z\in X$ and each map $\varphi_x\colon X\to X$, $y\mapsto x\cdot y$, is bijective. 
A cycle set $(X,\cdot)$ is said to be \emph{non-degenerate} 
if the map $X\to X$, $x\mapsto x\cdot x$, is bijective. 
\end{definition}

\begin{definition}
\index{Homomorphism!of cycle sets}
Let $X$ and $Z$ be cycle sets. 
A \emph{homomorphism} between the cycle sets $X$ and $Z$ is a 
map $f\colon X\to Z$ such that $f(x\cdot y)=f(x)\cdot f(y)$ for all $x,y\in X$. An \emph{isomorphism} of cycle sets
is a bijective homomorphism of cycle sets. 
\end{definition}

Cycle sets and cycle set homomorphisms form a category. 
It is possible to prove that the category of 
solutions is equivalent to the category of cycle sets, 
see Exercise~\ref{prob:cycle_sets}. 

% \framebox{Exercise!}

% \begin{lemma}
% \label{lem:T_forCS}
% If $X$ is a cycle set, then $x\cdot (y\cdot y)=((y*x)\cdot y)\cdot ((y*x)\cdot y)$, where
% $y*x=z$ if and only if $y\cdot z=x$. 
% \end{lemma}

% \begin{proof}
%     Since the operation $x\mapsto y\cdot x$ is bijective, we can write $x=y\cdot (y*x)$. Then, using~\eqref{eq:cycle_set}, 
%     $x\cdot (y\cdot y)=(y\cdot (y*x))\cdot (y\cdot y)=((y*x)\cdot y)\cdot (y*x)\cdot y)$.
% \end{proof}

\begin{theorem}
\label{thm:CS}
There exists a bijective correspondence between non-isomorphic involutive solutions 
and non-isomorphic non-degenerate cycle sets. 
\end{theorem}



% \begin{proof}
%     Let us assume first that $r(x,y)=(\sigma_x(y),\tau_y(x))$ is an involutive solution. 
%     We want to prove that the operation
%     $x\cdot y=\sigma_x^{-1}(y)$ turns the set $X$ into a non-degenerate cycle set. 
%     It is clear that the maps $y\mapsto x\cdot y=\sigma_x^{-1}(y)$ are invertible. 
%     By Proposition~\ref{pro:T}, the operation $x\mapsto x\cdot x$ is bijective. \framebox{!}
%     Since $r(\tau^{-1}_y(x),y)=(\sigma_{\tau^{-1}_y(x)}(y),x)$, Lemma~\ref{lem:YB} implies that
%     \[
%     \tau_x\circ\tau_{\sigma_{\tau^{-1}_y(x)}(y)}=\tau_y\circ\tau_{\tau^{-1}_y(x)}.
%     \]
%     Moreover, since $r(\sigma_{\tau^{-1}_y(x)}(y),x)=(\tau^{-1}_y(x),y)$, it follows that
%     %$\tau_x\sigma_{\tau^{-1}_y(x)}(y)=y$, i.e.
%     \[
%     \sigma_{\tau^{-1}_y(x)}(y)=\tau_x^{-1}(y).
%     \]
%     % Since $r^2=\id_{X\times X}$, it follows that
%     % \[
%     % (\tau^{-1}_y(x),y)=r^2(\tau^{-1}_y(x),y)=r(\sigma_{\tau^{-1}_y(x)}(y),x)
%     % \]
%     % and hence $\sigma_{\tau^{-1}_y(x)}(y)=\tau^{-1}_x(y)$. 
%     This implies that
%     \[
%     \tau_x\circ\tau_{\tau^{-1}_x(y)}=\tau_x\circ\tau_{\sigma_{\tau^{-1}_y(x)}(y)}=\tau_y\circ\tau_{\tau^{-1}_y(x)},
%     \]
%     which turns out to be equivalent to
%     \[
%     \tau^{-1}_{\tau^{-1}_x(y)}\circ\tau^{-1}_x=\tau^{-1}_{\tau^{-1}_y(x)}\circ\tau^{-1}_y
%     \]
%     and hence equivalent to Equality~\eqref{eq:cycle_set}. 
    
%     Conversely, if $X$ is a non-degenerate cycle set, we want to prove that
%     \[
%     r(x,y)=((y*x)\cdot y,y*x),
%     \]
%     where $y*x=z$ if and only if $y\cdot z=x$, is an involutive solution. The definition of $r$ implies that
%     $r^2=\id_{X\times X}$. 
%     By Lemma~\ref{lem:YB}, we need to prove that
%     \begin{align*}
%         &\sigma_x\circ\sigma_y = \sigma_{\sigma_x(y)}\circ\sigma_{\tau_y(x)},&
%         &\sigma_{\tau_{\sigma_y(z)}(x)}\tau_z(y)=\tau_{\sigma_{\tau_y(x)}(z)}\sigma_x(y),&
%         &\tau_z\circ\tau_y=\tau_{\tau_z(y)}\circ\tau_{\sigma_y(z)}
%     \end{align*}
 
%     Write $\sigma_x(y)=(y*x)\cdot y$ and $\tau_y(x)=y*x$. 
%     We know that the third formula is equivalent to~\eqref{eq:cycle_set}. 
%     Let $T\colon X\to X$, $x\mapsto x\cdot x$. By assumption, $T$ is bijective and hence 
%     $T^{-1}\circ \tau^{-1}_x\circ T=\sigma_x$ for all $x\in X$, since by Lemma~\ref{lem:T_forCS}
%         \begin{align*}
%     \tau_x^{-1}T(y)&=\tau_x^{-1}(y\cdot y)=x\cdot (y\cdot y)\\
%     &=((y*x)\cdot y)\cdot ((y*x)\cdot y)=T((y*x)\cdot y)=T\sigma_x(y)
%     \end{align*}
%     for all $x,y\in X$. In particular, $(X,r)$ is non-degenerate. 
%     Moreover, 
%     \begin{align*}
%     \sigma_x\circ\sigma_y
%     &=(T^{-1}\circ\tau_x^{-1}\circ T)\circ (T^{-1}\circ\tau_y^{-1}\circ T)\\
%     &=T^{-1}\circ (\tau_y\circ\tau_x)^{-1}\circ T\\
%     &=T^{-1}\circ (\tau_{\tau_y(x)}\circ\tau_{\sigma_x(y)})^{-1}\circ T\\
%     &=(T^{-1}\circ\tau^{-1}_{\sigma_x(y)}\circ T)\circ (T^{-1}\circ\tau^{-1}_{\tau_y(x)}\circ T)\\
%     &=\sigma_{\sigma_x(y)}\circ\sigma_{\tau_y(x)}
% \end{align*}
% for all $x,y\in X$. Finally, Equality~\eqref{eq:cycle_set} can be written as
% \[
% \tau^{-1}_{\tau^{-1}_v(u)}\circ\tau^{-1}_v=\tau^{-1}_{\tau^{-1}_u(v)}\circ\tau^{-1}_u.
% \]
% With $u=\tau_z\tau_y(x)$ and $v=z$, it becomes
% \[
% \tau^{-1}_{\tau_y(x)}\circ\tau^{-1}_z
% =\tau^{-1}_{\tau^{-1}_{\tau_z\tau_y(x)}(z)}\circ\tau^{-1}_{\tau_z\tau_y(x)}
% =\tau^{-1}_{\sigma_{\tau_y(x)}(z)}\circ\tau^{-1}_{\tau_z\tau_y(x)},
% \]
% where the last equality holds $r^2=\id_{X\times X}$. 
% Using Lemma~\ref{lem:YB} one rewrites this formula as
% \[
% \tau_{\sigma_{\tau_y(x)}(z)}\circ\tau^{-1}_{\tau_y(x)}
% =\tau^{-1}_{\tau_z\tau_y(x)}\circ\tau_z
% =\tau^{-1}_{\tau_{\tau_z(y)}\tau_{\sigma_y(z)}(x)}\circ\tau_z.
% \]
% Evaluating this equality at $y$, 
% \[
% \tau_{\sigma_{\tau_y(x)}(z)}\sigma_x(y)=\tau_{\sigma_{\tau_y(x)}(z)}\tau^{-1}_{\tau_y(x)}(y)
% =\tau^{-1}_{\tau_{\tau_z(y)}\tau_{\sigma_y(z)}(x)}\tau_z(y)
% =\sigma_{\tau_{\sigma_y(z)}(x)}\tau_z(y).\qedhere
% \]
% \end{proof}

For the readers who are not familiar with the above-mentioned result, 
the bijective correspondence is given by 
\[
r(x,y)=(x*y,(x*y)\cdot x),
\]
where $x*y=z$ if and only if $x\cdot z=y$. We leave the proof for the reader, see 
Exercise~\ref{prob:CS}. However, we will prove a more 
general result in Theorem~\ref{thm:skewCS}. 

Theorem~\ref{thm:CS} can be used to construct and enumerate small 
involutive solutions~\cite{AMV}. Table~\ref{tab:IYB} shows the 
number of non-isomorphic involutive solutions of size $\leq10$. 
For size $\leq7$ the numbers of Table~\ref{tab:IYB} coincide with those in~\cite{MR1722951}
but differ by two for $n=8$, as two solutions of size eight 
are missing in~\cite{MR1722951}. 

\begin{table}[H]
\centering
\caption{Involutive solutions of size $\leq10$.}
\begin{tabular}{|r|ccccccccc|}
\hline
$n$ & 2 & 3 & 4 & 5 & 6 & 7 & 8 & 9 & 10\tabularnewline
\hline 
solutions & 2 & 5 & 23 & 88 & 595 & 3456 & 34530 & 321931 & 4895272\tabularnewline
% square-free & 1 & 2 & 5 & 17 & 68 & 336 & 2041 & \cellcolor{gray!30}{15534} & \cellcolor{gray!30}{150957}\tabularnewline
% indecomposable & 1 & 1 & 5 & 1 & 10 & 1 & \cellcolor{gray!30}{100} & \cellcolor{gray!30}{16} & \cellcolor{gray!30}{36}\tabularnewline
% multipermutation & 2 & 5 & 21 & 84 & 554 & 3295 & \cellcolor{gray!30}{32155} & \cellcolor{gray!30}{305916} & \cellcolor{gray!30}{4606440}\tabularnewline
% irretractable & 0 & 0 & 2 & 4 & 9 & 13 & 191 & \cellcolor{gray!30}{685} & \cellcolor{gray!30}{3590}\tabularnewline
\hline
\end{tabular}
\label{tab:IYB}
\end{table}



% \begin{definition}
% \index{Solution!permutation group}
% The \emph{permutation group} of a solution $(X,r)$ is the group
% \[
% \mathcal{G}(X,r)=\langle (\sigma_x,\widehat{\sigma_x}):x\in X\rangle\subseteq\Sym_X\times\Sym_X.
% \]
% \end{definition}

% \index{Solutions!linear involutive}
% Let us finish the chapter with the family of affine involutive solutions. Let $A$ be an abelian group and
% $a,b,c,d\in\colon A\to A$ be group homomorphisms. If 
% \[
% r(x,y)=(a(x)+b(y),c(x)+d(y))
% \]
% is bijective and $(A,r)$ is an solution involutive solution, then $(A,r)$ is an \emph{linear solution}.
% Note that $r$ satisfies the YBE if and only if
% \begin{subequations}
% \begin{gather}
% \label{eq:linear}
% (\id-a)\circ d\circ b=b\circ d,
% \quad
% d\circ(\id-d)=c\circ d\circ b,\\
% c\circ d\circ (\id-a)=d\circ c,
% \quad
% a\circ(\id-a)=b\circ a\circ c,
% \quad
% c\circ a=(\id-d)\circ a\circ c,\\
% a\circ b=b\circ a\circ (\id-d),
% \quad
% c\circ b-b\circ c=a\circ d\circ a-d\circ d.
% \end{gather}
% \end{subequations}
% Moreover, $r^2=\id$ if and only if
% \begin{align}
%     &a\circ a+b\circ c=\id,
%     &&
%     c\circ b+d\circ d=\id,
%     &&
%     a\circ b+b\circ d=0,
%     &&
%     c\circ a+d\circ c=0.
% \end{align}
% \end{equation}
