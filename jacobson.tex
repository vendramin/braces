\chapter{The Jacobson radical}
\label{radical}

\section{Primitive rings}

Let $M$ be an $R$-module and $S$ a subset of $M$. The \emph{annihilator} of $S$ is
\[
\Ann_R(S)=\{ a\in R:as=0\;\text{for all } s\in S\} .
\]
If $S=\{m\}$, then we will write $\Ann_R(m)=\Ann_R(\{m\})$.

It is \textcolor{blue}{an exercise} to check that $\Ann_R(S)$ is a left ideal of $R$. Furthermore,
if $S$ is a submodule of $M$, then $\Ann_R(S)$ is an ideal of $R$.

Let $T$ be a subset of a right $R$-module, then the annihilator of $T$ is
\[
\Ann^r_R(T)=\{ a\in R: ta=0\;\text{for all }t\in T\} .
\]

\begin{definition}
    A \emph{faithful} $R$-module is an $R$-module $M$ such that $\Ann_R(M)=\{0\}$. 
\end{definition}

Similarly one defines faithful right $R$-modules.

\begin{example}
Let $K$ be a field and $R=M_2(K)$. Then
\[
I=\left(\begin{array}{cc} K&K\\
0&0\end{array}\right) =\left\{\left(\begin{array}{cc} a&b\\
0&0\end{array} \right): a,b\in K\right\} 
\]
is a right ideal of $R$ which is not a left ideal.
\textcolor{blue}{It is an exercise to show that} 
\[
\Ann_R(I)=\left(\begin{array}{cc} 0&K\\
0&K\end{array}\right)\quad\mbox{and}\quad \Ann^r_R(I)=\{0\}.
\]
Hence $I$ is a faithful right $R$-module.
\end{example}

\begin{lemma}\label{Lema 1.1.1}
Let $M$ be an $R$-module. Then $M$ is a faithful
$(R/\Ann_R(M))$-module.
\end{lemma}

\begin{proof}
We know that $\Ann_R(M)$ is an ideal of $R$, and thus $R/\Ann_R(M)$ is a ring. 
We define on $M$ the multiplication by element of $R/\Ann_R(M)$ by
\[
(a+\Ann_R(M))m=am
\]
for all $m\in M$ and $a\in R$. This multiplication is well-defined.
Let $m\in M$ and $a,a'\in R$ \textcolor{blue}{be} such that $a+\Ann_R(M)=a'+\Ann_R(M)$.
Since $a-a'\in \Ann_R(M)$, 
$(a-a')m=0$, that is $am=a'm'$. 
\textcolor{blue}{A straightforward calculation shows that} $M$ 
with the addition and this multiplication is a faithful $R/\Ann_R(M)$-module.
\end{proof}

\begin{definition}
\index{Module!simple}
A module $M$ is \emph{simple} if $M\neq \{0\}$ and $\{0\}$ and $M$ are the only submodules of $M$.
\end{definition}

\begin{lemma}
\label{Lema 1.1.2}
An $R$-module $M$ is simple if and only if there exists a maximal left ideal $I$ of $R$ such that
$M\cong R/I$.
\end{lemma}

\begin{proof}
Suppose first that $M$ is simple. Let  $m\in M\setminus\{ 0\}$. Then $M=Rm$.
Let $f\colon R\rightarrow M$ be the map defined by $f(a)=am$. 
It is clear that $f$ is an epimorphism of $R$-modules. 
By the first isomorphism theorem, 
$M\cong R/\ker(f)$. 
Furthermore, if $J$ is a left ideal of
$R$ such that $\ker(f)\subseteq J\subseteq R$, then
$J/\ker(f)$ is isomorphic to $\{0\}$ or $M$. Hence $J$ is equal to
$\ker(f)$ or $R$, and thus $\ker(f)$ is a maximal left ideal of $R$.

The converse is clear. 
\end{proof}

Note that Zorn's lemma implies that every \textcolor{blue}{unitary} ring has simple modules, see Exercise \ref{prob:Zorn_simple}. 

\begin{convention}
From now on, a ring will mean a non-zero ring, unless otherwise specified. 
\end{convention}

\begin{definition}
\index{Ring!left primitive}
\index{Ring!right primitive}
A ring $R$ is \emph{left primitive} if there exists a faithful simple $R$-module. 
\end{definition}

Similarly one defines \emph{right primitive} rings. 

\begin{definition}
\index{Ring!simple}
A ring $R$ is \emph{simple} if $\{0\}$ and $R$ are the only ideals of $R$.
\end{definition}

Note that every simple simple ring is (left and right) primitive.

\begin{example}
Let $K$ be a field. Then $M_n(K)$ is a simple ring for all positive integer $n$. Furthermore,
\[
\left(\begin{array}{cccc} K&K&\ldots&K\\
0&0&\ldots&0\\
\vdots &\vdots &&\vdots\\
0&0&\ldots&0\end{array}\right)
\]
is a minimal right ideal of $M_n(K)$ and thus it is a faithful simple
right $M_n(K)$-module. Similarly
\[
\left(\begin{array}{cccc} K&0&\ldots&0\\
	K&0&\ldots&0\\
	\vdots &\vdots &&\vdots\\
	K&0&\ldots&0\end{array}\right)
\]
is a minimal left ideal of $M_n(K)$ and thus it is a faithful simple
$M_n(K)$-module. 
\end{example}

\begin{proposition}
\label{Prop1.2.1}
Let $R$ be a commutative ring. Then the following statements are equivalent:
\begin{enumerate}
\item $R$ is primitive.
\item $R$ is simple.
\item $R$ is a field.
\end{enumerate}
\end{proposition}

\begin{proof}
    The implications $3)\implies 2)\implies 1)$ are trivial. Let us prove that $1)\implies 3)$. 
    Suppose that $R$ is primitive. Let $V$ be
a faithful simple $R$-module. By Lemma \ref{Lema 1.1.2}, $V\cong R/I$ for some maximal ideal $I$ of $R$. 
Now $\{0\}=\Ann_R(V)=\Ann_R(R/I)=I$, and thus $\{0\}$ is a maximal ideal of $R$. If $r\in R\setminus \{ 0\}$, then
$rR=Rr=R$. Hence $R$ is a field.
\end{proof}

The next result shows an example of a primitive ring which is not simple.

\begin{proposition}\label{Prop1.2.2}
Let $K$ be a field and $V$ an infinite dimensional $K$-vector space. Then
$\End_K(V)$ is a primitive ring which is not simple.
\end{proposition}

\begin{proof}
Let $R=\End_K(V)$. Define on $V$ a multiplication by elements of $R$ by the rule 
$f\cdot v=f(v)$ for $f\in R$ and $v\in V$.
\textcolor{blue}{Then} $V$ is an $R$-module.
It is clear that for every $v\in V\setminus \{ 0\}$ and $w\in V$ there exists
$f\in R$ such that $f(v)=w$. Hence $V$ is a simple $R$-module. Clearly, 
$\Ann_R(V)=\{0\}$ and thus $V$ is a faithful $R$-module. 
Hence $R$ is left primitive.

Let $V^*=\Hom_K(V,K)$ the dual space of $V$. Define on $V^*$ a multiplication by elements of $R$ by the rule  
$\omega\cdot f=\omega f$ for $f\in R$ and $\omega\in V^*$. Then $V^*$ is a
faithful simple right $R$-module. Hence $R$ is right primitive.

Note that $I=\{ f\in R\mid \dim_K(\im(f))<\infty\}$ 
is a non-zero proper ideal of $R$. 
Hence $R$ is not simple. 
\end{proof}

Bergman constructed in \cite{MR175940,MR167497} 
the first example of a right primitive ring which is not left primitive. 

\begin{proposition}
\label{Prop1.2.3}
Let $R$ be a left primitive ring. Then $M_n(R)$ is left primitive for all positive integer $n$.
\end{proposition}

\begin{proof}
Let $V$ be a faithful simple $R$-module. Then $M_{n\times 1}(V)$ is a faithful simple $M_n(R)$-module.
\end{proof}



Note that every division ring is a simple ring. The \textcolor{blue}{following result gives us a method to construct division rings.}

\begin{lemma}[Schur]
\index{Schur's lemma}
Let $V$ be a simple $R$-module. Then $\End_R(V)$ is a division ring.
\end{lemma}

\begin{proof}
Let $f\in \End_R(V)\setminus \{ 0\}$. Since $V$ is simple and $\im(f)\neq\{0\}$, \textcolor{blue}{it follows that} $\im(f)=V$. Since
$\ker(f)\neq V$, $\ker(f)=\{0\}$. Hence $f$ is an automorphism of $V$ and therefore $\End_R(V)$ is a division ring.
\end{proof}

\textcolor{blue}{The following theorem goes back to Chevalley and Jacobson.}

\begin{theorem}[Density theorem]
\index{Chevalley--Jacobson's theorem}
\index{Density theorem}
Let $V$ be a simple $R$-module and 
$D=\End_R(V)$. Then $V$ is a $D$-vector space. Furthermore, if $x_1,\dots,x_n\in V$ are $D$-linearly 
independent and $y_1,\dots,y_n\in V$, then there exists $r\in R$ such that $rx_i=y_i$ for all $i=1,\dots,n$.
\end{theorem}

\begin{proof}
By Schur's lemma, $D$ is a division ring.
We define on $V$ a multiplication by elements of $D$ by 
\[
f\cdot v=f(v)
\]
for $f\in D$ and $v\in V$. \textcolor{blue}{Then} $V$ with the addition and this multiplication is a $D$-vector space.

%Let $x_1,\dots,x_n\in V$ be $D$-linearly 
%independent and $x'_1,\dots,x'_n\in V$, 
\textcolor{blue}{We claim that there} exists $r\in R$ such that $rx_i=y_i$ for all $i=1,\dots,n$. \textcolor{blue}{We proceed by induction on $n$.}
For $n=1$, since $V$ is a simple $R$-module, $Rx_1=V$ and thus there exists $r\in R$ such that $rx_1=y_1$.

Suppose that $n>1$ and that the result is true for $n-1$.
By the induction hypothesis, $V^{n-1}=R(x_1,\dots,x_{n-1})$. We shall show that there exists $r_n\in R$ 
such that $r_nx_n\neq 0$ and $r_nx_i=0$ for all
$i=1,\dots,n-1$. Suppose that there is no such an element. In this case, 
the map 
\[
\varphi\colon R(x_1,\dots,x_{n-1})\rightarrow V,
\quad
r(x_1,\dots,x_{n-1})\mapsto rx_n, 
\]
is well-defined. Since $\Hom_R(V^{n-1}, V)\cong
(\End_R(V))^{n-1}=D^{n-1}$, we may identify
$\Hom_R(V^{n-1}, V)$ with $D^{n-1}$, and then there exist
$d_1,\dots,d_{n-1}\in D$ such that $\varphi =(d_1,\dots,d_{n-1})$. Thus
$x_n=\varphi (x_1,\dots,x_{n-1})=\sum_{i=1}^{n-1}d_ix_i$, a contradiction because $x_1,\dots,x_n$ are $D$-linearly independent. 
Hence there exists $r_n\in R$ such that $r_nx_n\neq 0$ and $r_nx_i=0$ for all
$i=1,\dots,n-1$. Similarly one can see that for every $j\in\{ 1,\dots,n\}$,
there exists $r_j\in R$ such that $r_jx_j\neq 0$ and $r_jx_i=0$ for all
$i\neq j$. Since  $V$ is a simple $R$-module, there exists $s_j\in R$
such that $s_jr_jx_j=y_j$. Let $r=\sum_{j=1}^ns_jr_j$. One checks that $rx_i=y_i$ for all $i=1,\dots,n$. Therefore the result follows by induction.
\end{proof}

\begin{theorem}\label{Teorema 1.2.4}
Let $R$ be a left primitive ring. Then one of the following conditions holds:
\begin{enumerate}
\item $R\cong M_n(D)$ for some division ring $D$ and some positive integer $n$.
\item There exists a division ring $D$ such that for every positive integer $m$ there exists a subring $S_m$ of $R$ and an surjective homomorphismm $\varphi _m\colon S_m\rightarrow M_m(D)$.
\end{enumerate}
\end{theorem}

\begin{proof}
Let $V$ be a faithful simple $R$-module \textcolor{blue}{and}
$D=\End_R(V)$. By Schur's lemma, $D$ is a division ring. 
\textcolor{blue}{Moreover,} 
$V$ is a $D$-vector space.
Let $\psi\colon R\rightarrow\End_D(V)$, $\psi
(r)(v)=rv$ for $r\in R$ and $v\in V$. It is clear that $\psi$ is a homomorphism of rings.

Suppose that $\dim_D(V)<\infty$ and let $v_1,\dots,v_n$ \textcolor{blue}{be} 
a $D$-basis of $V$. Let $f\in \End_D(V)$. By the density theorem, 
there exists $r\in R$ such that $f(v_i)=rv_i$ for all
$i=1,\dots,n$. Hence $\psi (r)=f$ and thus $\psi$ is surjective. Since
$V$ is a faithful $R$-module, $\psi$ is injective. Therefore $R\cong \End_D(V)\cong M_n(D)$. 

Suppose now that $V$ is an infinite dimensional $D$-vector space. Let $(
v_i) _{i\geq 1}$ be a family of $D$-linearly independent vectors of $V$. Let
$$V_m=Dv_1+\dots+Dv_m\quad\mbox{and}\quad S_m=\{ r\in R\mid rV_m\subseteq
V_m\} .$$
It is clear that $S_m$ is a subring of $R$. By the density theorem,
the restriction map 
$\psi |_{S_m}\colon S_m\rightarrow \End_D(V_m)$ is surjective. 
Hence there exists an surjective homomorphism from $S_m$ to $M_m(D)$ and the result follows.
\end{proof}

\section{Radicals}

\begin{definition}
\index{Jacobson radical}
Let $R$ be a ring. The \emph{Jacobson radical} of $R$ is
\[
J(R)=\bigcap_{V \mbox{ \scriptsize{simple} }R-\mbox{\scriptsize{module}}} \Ann_R(V).
\]
\end{definition}

Since each $\Ann_R(V)$ is an ideal of $R$, $J(R)$ also is an ideal of $R$.

\begin{theorem}[Jacobson]\label{Teorema 1.3.1}
Let $R$ be a ring and $a\in R$. Then the following conditions are equivalent.
\begin{itemize}\item[(i)] $a\in J(R)$.
\item[(ii)] $a\in  M$, for every maximal left ideal $M$ of $R$.
\item[(iii)] $R(1-xa)=R$ for all $x\in R$.
\item[(iv)] $1-xay$ is invertible for all $x,y\in R$.
\item[($i^*$)]-($iv^*$) the dual left-right conditions of $(i)-(iv)$.
\end{itemize}
\end{theorem}

\begin{proof}
$(i)\Rightarrow (ii)$. Let $M$ be a maximal left ideal of
$R$ and suppose that $a\in J(R)$. By Lemma \ref{Lema 1.1.2}, $V=R/M$ is a simple $R$-module. Thus $a\in \Ann_R(V)\subseteq M$.

$(ii)\Rightarrow (iii)$. Let $x\in R$ and suppose that $a\in  M$, for every maximal left ideal $M$ of $R$. Suppose that
$R(1-xa)\neq R$. By Zorn's lemma, there exists a maximal left ideal $M$ of $R$ such that
$R(1-xa)\subseteq M$. But $a\in M$, hence $1=1-xa+xa\in M$, a contradiction. Therefore $R(1-xa)=R$.

$(iii)\Rightarrow (iv)$. Let $x,y\in R$ and suppose that $R(1-xa)=R$ for all $x\in R$. Suppose that $R(1-xay)\neq
R$. By the implication $(i)\Rightarrow (iii)$, there exists a simple $R$-module $V$ such that $ayV\neq 0$. Hence $aV\neq 0$. 
Let $v\in V$ such that $av\neq
0$. Since $V$ is simple, there exists $z\in R$ such that $zav=v$. Thus
$(1-za)v=0$ and, since $v\neq 0$, we have that $R(1-za)\neq R$, a contradiction. Therefore $R(1-xay)=R$. Hence there exists $b\in R$
such that $(1-b)(1-xay)=1$. Thus $b=bxay-xay=(bx-x)ay$ and, then we get that $R(1-b)=R$. Hence $1-b$ is invertible 
and $(1-b)^{-1}=1-xay$.

$(iv)\Rightarrow (i)$. Let $V$ be a simple $R$-module and suppose that $1-xay$ is invertible for all $x,y\in R$. 
Suppose that $aV\neq 0$. Hence as above there exist $v\in V\setminus\{ 0\}$ and $z\in R$
such that $(1-za)v=0$. But $1-za$ is invertible, a contradiction, therefore $a\in \Ann_R(V)$ for all simple $R$-module $V$, that is 
$a\in J(R)$. 

Since condition $(iv)$ is left-right symmetric, the result follows.
\end{proof}

\begin{definition}
\index{Subdirect product} 
    Let $( R_i) _{i\in I}$ be a non-empty family of rings. Let
    \[
    \pi_j\colon\prod_{i\in I}R_i\rightarrow R_j\quad (j\in I),
    \]
    be the natural maps. A subring $S$ of $\prod_{i\in I}R_i$ is said to be a \emph{subdirect product} 
    of the rings $R_i$ if for every $j\in I$ the restiction $\pi_j|_S$ of $\pi_j$ to $S$ is surjective.
\end{definition}

Let $R$ be a ring and $I=\{ \Ann_R(V)\mid V$ is a simple $R$-module $\}$. Define
\[
\begin{array}{cccc}\varphi\colon &R&\rightarrow &\prod_{A\in I}R/A\\
&r&\mapsto &(r+A)_{A\in I}\end{array}
\]
It is clear that $\varphi$ is a homomorphism of rings and $\ker(\varphi
)=J(R)$. Hence, by the first isomorphism theorem there is an injective homomorphism
\[
\begin{array}{cccc}\overline{\varphi}\colon &R/J(R)&\rightarrow &\prod_{A\in I}R/A\\
&r+J(R)&\mapsto &(r+A)_{A\in I}\end{array}
\]
Thus $R/J(R)\cong \varphi (R)$, which is a subdirect product of the rings $R/A$, and these rings are left primitive rings by Lemma \ref{Lema 1.1.1}. 

\begin{definition}
\index{Ring!semiprimitive}
A ring is said to be \emph{semiprimitive} if it is isomorphic to a subdirect product of left primitive rings.
\end{definition}

\begin{proposition}\label{Prop1.3.2}
Let $R$ be a ring. Then the following conditions are equivalent.
\begin{itemize}\item[(i)] $R$ is semiprimitive.
\item[(ii)] $J(R)=0$.
\item[(iii)] $R$ is isomorphic to a subdirect product of right primitive rings.
\end{itemize}
\end{proposition}

\begin{proof}
Since $(ii)$ is left-right symmetric, it is enough to prove that  $(i)$ and
$(ii)$ are equivalent. Above we have seen that $(ii)$ implies $(i)$. 

We may assume that $R$ is a subdirect product of the rings $\{ R_i\} _{i\in I}$
and that $R_j$ is left primitive for all $j\in I$. Let $V_j$ be a faithful simple $R_j$-module. 
Let $\pi _j\colon\prod_{i\in I}R_i\rightarrow R_j$ be the natural map. Then $V_j$ is an $R$-module via $\pi _j$,
that is
$$r\cdot v:=\pi_j(r)v,$$
for all $v\in V_j$ and all $r\in R$.
It is clear that $V_j$ is a simple $R$-module. Since $V_j$ is a faithful $R_j$-module, we have that
$\Ann_R(V_j)=\ker(\pi_j |_R)$. Since $R$ is a subring of $\prod_{i\in I}R_i$, we have that $\bigcap_{j\in I}\Ann_R(V_j)=\{0\}$.
Since $J(R)\subseteq \bigcap_{j\in I}\Ann_R(V_j)$, we get that $J(R)=\{0\}$, and the result follows.
\end{proof}

Note that every left (or right) primitive ring is semiprimitive. A trivial 
consequence of Proposition \ref{Prop1.3.2} is the following result.

\begin{corollary}\label{Cor1.3.3}
    If $R$ is a ring, then $J(R/J(R))=\{0\}$. \qed
\end{corollary}


\begin{proposition}[Nakayama's lemma]
Let $R$ be a ring and $S\subseteq J(R)$. Let $M$ be a finitely generated $R$-module. If $SM=M$, then $M=0$.
\end{proposition}

\begin{proof}
Let $m_1=0,m_2,\dots,m_n\in M$ such that $M=Rm_1+Rm_2+\dots+Rm_n$.
We shall prove that $M=\{0\}$ by induction on $n$. For $n=1$, it is clear. Suppose that $n>1$ and the result is true for $n-1$. 
We know that there exist $s_1,s_2,\dots,s_n\in S$ such that
$$m_n=s_1m_1+s_2m_2+\dots+s_nm_n.$$
Hence $(1-s_n)m_n=s_1m_1+\dots+s_{n-1}m_{n-1}$ and, since $1-s_n$ is invertible, we get that
$m_n=(1-s_n)^{-1}s_1m_1+\dots+(1-s_n)^{-1}s_{n-1}m_{n-1}$. Hence $M=Rm_1+\dots+Rm_{n-1}$ and by the induction hypothesis, $M=\{0\}$.
Therefore the result follows by induction.
\end{proof}

Note that if $a$ is a nilpotent element of a ring $R$, then there exists a positive integer $n$ such that $a^n=0$, and then
$$(1-a)(1+a+\dots+a^{n-1})=1.$$

\begin{definition}
\index{Ideal!nil}
A \emph{nil left (right) ideal} of a ring $R$ is a left (right) ideal $I$ of $R$ such that every $a\in I$ is nilpotent. 
\end{definition}

By Theorem \ref{Teorema 1.3.1} and the above remark,
it is clear that every nil left (right) ideal of $R$ is contained in $J(R)$.

\begin{definition}
\index{Ideal!nilpotent}
A left (right) ideal $I$ of $R$ is \emph{nilpotent} if there exists a positive integer $n$ 
such that $I^n=0$. 
\end{definition}

\begin{definition}
\index{Ideal!nilradical}
An ideal $J$ of $R$ is said to be a \emph{nilradical} if it is a nil ideal and $R/J$ has no nonzero nilpotent ideal.
\end{definition}

\begin{proposition}\label{Prop1.3.4}
Let $R$ be a ring. Then there exists a unique maximal nil ideal $\mathcal{U}(R)$ of $R$. 
Furthermore, $\mathcal{U}(R)$ is a nilradical and it is called the upper nilradical of $R$.
\end{proposition}

\begin{proof}
First we shall see that the sum of two nil ideals is a nil ideal. Let $I_1,I_2$ be  nil ideals of $R$. 
Let $x\in I_1$ and $y\in I_2$. There exists a positive integer $n$ such that $x^n=0$. Hence $(x+y)^n\in
I_2$. Thus $x+y$ is nilpotent and $I_1+I_2$ is a nil ideal. Let $\mathcal{C}=\{ I\mid I\mbox{ is a nil ideal of }R\}$
Let $\mathcal{U}(R)=\sum_{I\in \mathcal{C}}I$. It is clear that  $\mathcal{U}(R)$ is the unique maximal nil ideal of $R$. 
Let $I$ be a nilpotent ideal of $R/\mathcal{U}(R)$ and let
$\pi\colon R\rightarrow R/\mathcal{U}(R)$ be the natural map. Then
there exists a positive integer $m$ such that $I^m=\{0\}$. Hence, $\pi
^{-1}(I)^m=\mathcal{U}(R)$. Thus $\pi ^{-1}(I)$ is a nil ideal and therefore
$\pi ^{-1}(I)=\mathcal{U}(R)$. Hence $I=\{0\}$, and the result follows.
\end{proof}

\begin{definition}
\index{Ring!prime}
A ring $R$ is said to be \emph{prime} if for ideals $I$ and $J$ of $R$, $IJ=\{0\}$ implies that
$I=\{0\}$ or $J=\{0\}$. 
\end{definition}

Note that the previous definition is equivalent to $aRb\neq\{0\}$ for all $a,b\in R\setminus\{0\}$.

\begin{example}
The commutative prime rings are precisely the commutative integral domains.
\end{example}

\begin{proposition}\label{Prop1.3.5}
Every left primitive ring is prime.
\end{proposition}

\begin{proof}
Let $V$ be a faithful simple $R$-module. Let $I,J$ be nonzero ideals of
$R$. We shall see that $IJ\neq \{0\}$. Since $V$ is faithful and simple,
$IV=V$ and $JV=V$. Hence $IJV=IV=V$ and thus $IJ\neq \{0\}$. 
\end{proof}

\begin{definition}
\index{Ideal!prime}
An ideal $P$ of a ring $R$ is said to be \emph{prime} if $R/P$ is a prime ring.
\end{definition}

Let $\mathcal{C}=\{ P\mid P$ prime ideal of $R\}$. Define
\[
\varphi\colon R\rightarrow \prod_{P\in\mathcal{C}}R/P
\]
by $\varphi (r)=(r+P)_{P\in\mathcal{C}}$ for all $r\in R$. It is clear thate $\varphi$ 
is a homomorphism of rings and $\ker(\varphi
)=\bigcap_{P\in \mathcal{C}}P$. The \emph{Baer-McCoy radical} or \emph{(Baer's) lower nilradical} of $R$ is
$$\mathcal{B}(R)=\bigcap_{P\in\mathcal{C}}P.$$
$\mathcal{B}(R)$ is also called the \emph{prime radical} of $R$.
A ring $R$ is said to be \emph{semiprime} if it is isomorphic to a subdirect product of prime rings.
Note that $\mathcal{B}(R)=\ker(\varphi)$. Hence, if $\mathcal{B}(R)=\{0\}$, then $R$ is semiprime. 
In fact we have the following result.

\begin{theorem}\label{Teorema 1.3.6}
Let $R$ be a ring. Then the following conditions are equivalent.
\begin{itemize}\item[(i)] $R$ is semiprime.
\item[(ii)] $\mathcal{B}(R)=\{0\}$.
\item[(iii)] For an ideal $I$ of $R$, $I^2=\{0\}$ implies $I=\{0\}$.
\end{itemize}
\end{theorem}

\begin{proof}
$(i)\Rightarrow (ii)$ We may assume that $R$ is a subdirect product of prime rings $(R_i)_{i\in I}$. Let $\pi_j\colon\prod_{i\in I}R_i\rightarrow R_j$ be the natural map for all $j\in I$. Then $P_j=\ker(\pi_j|_R)$ is a prime ideal of $R$ because $R/P_j\cong R_j$.
Hence $\bigcap_{i\in I}P_i=\{0\}$. Since $\mathcal{B}(R)\subseteq \bigcap_{i\in I}P_i$, we have that $\mathcal{B}(R)=\{0\}$.

$(ii)\Rightarrow (i)$ We have seen this above.


$(ii)\Rightarrow (iii)$ Suppose that $\mathcal{B}(R)=\{0\}$. Let $I$ be an ideal of $R$ such that $I^2=\{0\}$.
Let $P$ be a prime ideal of $R$. Since  $I^2\subseteq P$, we have that
$I\subseteq P$. Hence $I\subseteq \mathcal{B}(R)=\{0\}$.

$(iii)\Rightarrow (ii)$ Let $a\in R\setminus\{ 0\}$. We shall prove that there exists a prime ideal $P$ of $R$
such that $a\notin P$ and thus $\mathcal{B}(R)=\{0\}$.

By $(iii$), we have that $aRa\neq 0$. We shall construct inductively a sequence $a_0=a,a_1,\dots ,a_n,\dots$ of nonzero elements of $R$ such that
$a_n\in a_{n-1}Ra_{n-1}$ for all positive integer $n$. Suppose we have constructed
 $a_0,a_1,\dots,a_m$ for some $m\geq 0$. Then by $(iii)$,
$a_mRa_m\neq 0$. Let $a_{m+1}\in a_mRa_m\setminus\{ 0\}$. Hence we get $a_0,a_1,\dots,a_n,\dots$ nonzero elements in $R$
such that $a_n\in a_{n-1}Ra_{n-1}$. Let $S=\{ a_n\mid n\geq
0\}$. Let
$$\mathcal{C}=\{ I\mid I\;\mbox{ideal of}\; R\;\mbox{such that}\; I\cap
S=\emptyset \} .$$
It is clear that $\{0\}\in \mathcal{C}$ and that any chain in $\mathcal{C}$ (ordered by inclusion) has an upper bound in $\mathcal{C}$.
By Zorn's lemma, there exists a maximal element $P$ in $\mathcal{C}$. We shall see that $P$ is prime. Let $I,J$ be ideals
of $R$ such that $I\not\subseteq P$ and $J\not\subseteq P$. Since $P$
is maximal in $\mathcal{C}$, there exist non-negative integers $m,n$ such that
$a_m\in I+P$ and $a_n\in J+P$. We may assume that $m\leq n$. Hence
$a_n\in (I+P)\cap (J+P)$. Since $a_{n+1}\in a_nRa_n$, we have that
$a_{n+1}\in (I+P)(J+P)\subseteq IJ+P$. Since $P\cap S=\emptyset$,
$IJ\not\subseteq P$. Therefore the result follows.  
\end{proof}

\begin{proposition}\label{Prop1.3.7}
Let $R$ be a ring. Then $\mathcal{B}(R)$ is a nilradical of $R$, and
if $I$ is a nilradical of $R$, then $\mathcal{B}(R)\subseteq I$.
\end{proposition}

\begin{proof}
Let $I$ a nilradical of $R$. By definition $R/I$ has no nonzero nilpotent ideals. By Theorem \ref{Teorema 1.3.6}, $R/I$ is semiprime. Hence
$I$  is the intersection of the prime ideals of $R$ containing $I$, and thus $\mathcal{B}(R)\subseteq I$. By Proposition \ref{Prop1.3.4}, $R$ has nilradicals, and thus $\mathcal{B}(R)$ is a nil ideal of $R$. By Theorem \ref{Teorema 1.3.6}, $R/\mathcal{B}(R)$ has no nonzero nilpotent ideals. Hence $\mathcal{B}(R)$ is a nilradical of $R$, and the result follows.
\end{proof}

Let $R$ be a ring. Let $\mathcal{N}=\{ I\mid I$ nilpotent ideal of $R\}$. The \emph{nilpotent radical} of $R$ is
$N(R)=\sum_{I\in\mathcal{N}} I$. It is clear that $N(R)\subseteq \mathcal{B}(R)$, but in general this inclusion is not an equality, 
that is, in general, $N(R)$ is not a nilradical of $R$. Furthermore,
$N(R)$, in general, is not a nilpotent ideal.

Note that if $R$ is a commutative ring, then
$$N(R)=\mathcal{B}(R)=\mathcal{U}(R)=\{ x\in R\mid x\;\mbox{is nilpotent}\}$$
is the unique nilradical of $R$, and thus, in this case,  $N(R)$ is said to be the nilradical of $R$.

\section{Artinian rings}
An $R$-module $M$ is said to be \emph{Artinian} if every descending chain of submodules of $M$ is stationary, that is, if
$$M_1\supseteq M_2\supseteq\dots\supseteq M_n\supseteq\dots$$
is a descending chain of submodules of $M$, there exists a positive integer $n_0$ such that $M_n=M_{n_0}$ for all $n\geq n_0$.

The proof of the next result is easy and we left it to the reader.

\begin{proposition}\label{Prop1.4.1}
An $R$-module $M$ is Artinian if and only if every non-empty set of submodules of $M$ has a minimal element. \qed
\end{proposition}

A ring $R$ is said to be \emph{right Artinian} if it is Artinian as right $R$-module.
Similarly one defines left Artinian ring. We say that the ring $R$ is Artinian if it is right and left Artinian.

\begin{example}
Let $R$ be a ring. Suppose that $R$ has a subring which is a division ring $D$. Then every $R$-module also is a $D$-vector space. Hence every $R$-module finitely dimensional as $D$-vector space
is an Artinian $R$-module.
For example, if $D$ is a division ring, then $M_n(D)$ is an Artinian ring for all positive integer $n$.
\end{example}


Note that if $e$ is an idempotent element in a ring $R$, then
$eRe$ is a ring (possibly zero) and $e$ is its unit-element.

\begin{lemma}\label{Lema 1.4.2}
Let $R$ be a ring and $e\in R$ idempotent. Then $eRe\cong
\End_R(eR)$.
\end{lemma}

\begin{proof}
Consider the map $\varphi\colon eRe\rightarrow \End_R(eR)$ defined by  $\varphi
(ere)(es)=eres$ for all $r,s\in R$. It is easy to check that $\varphi$ is an injective homomorphism of rings.
Let $f\in \End_R(eR)$. Note that $f(e)=f(e^2)=f(e)e\in
eRe$. One can verify that $\varphi (f(e))=f$, and thus $\varphi$
is an isomorphism and the result follows.
\end{proof}

\begin{theorem}[Artin--Wedderburn]
\label{Teorema 1.4.3}
\index{Artin--Wedderburn's theorem}
The following conditions are equivalent.
\begin{itemize}\item[(i)] $R$ is an Artinian simple ring.
\item[(ii)] $R$ is a left Artinian simple ring.
\item[(iii)] $R$ is a left Artinian and a left primitive ring.
\item[(iv)] $R\cong M_n(D)$, for some division ring $D$ and some positive integer $n$.
\item[($i^*$)]-($iv^*$) the left-right dual condition of $(i)$-$(iv)$.
\end{itemize}
Furthermore, in (iv),  $n$ is uniquely determined by $R$ and $D$ is determined up to isomorphism.
\end{theorem}

\begin{proof}
It is clear that $(i)\Rightarrow (ii)$ and $(ii)\Rightarrow (iii)$.

$(iii)\Rightarrow (iv)$. Suppose that $R$ is a left Artinian and left primitive ring. Let $V$ be a faithful simple $R$-module
and $D=\End_R(V)$. By Schur's lemma, $D$ is a division ring.
We know that $V$ is a $D$-vector space.
Suppose that $V$ is an infinite dimensional $D$-vector space. Let $( v_n)_{n\geq 1}$ be a family of 
$D$-linearly independent vectors of $V$. Let $A_m=\Ann_R(\{
v_1,\dots,v_m\} )$. It is clear that
$$A_1\supseteq A_2\supseteq\dots\supseteq A_m\supseteq\dots$$
By the Density theorem these inclusions are strict, but this is a contradiction because $R$ is left Artinian. Hence $V$ 
is finite dimensional $D$-vector space. Let  $\dim_D(V)=n$. By the proof of Theorem \ref{Teorema 1.2.4}, $R\cong M_n(D)$.

$(iv)\Rightarrow (i)$ It is easy to check that $M_n(D)$ is an Artinian simple ring.

Let $D,D'$ be division rings and $n,m$ positive integers such that
$M_n(D)\cong M_m(D')$. We shall prove that $n=m$ and $D\cong D'$. Let
$\varphi\colon M_n(D)\rightarrow M_m(D')$ an isomorphism. Let
$E_{1,1}\in M_n(D)$ the matrix with $1$ in the $(1,1)$-entry and zero in the other entries. Note that $E_{1,1}$ is an idempotent of $M_n(D)$, and it is easy to see that $E_{1,1}M_n(D)$ is a simple right $M_n(D)$-module and $\dim_D(E_{1,1}M_n(D))=n$. By Lemma \ref{Lema 1.4.2}, we have:
\begin{align*} D\cong&
E_{1,1}M_n(D)E_{1,1}\cong\End_{M_n(D)}(E_{11}M_n(D))\\
\cong&
\End_{M_m(D')}(\varphi (E_{1,1})M_m(D'))\cong \varphi
(E_{1,1})M_m(D')\varphi (E_{1,1}).
\end{align*}
Since $\varphi (E_{1,1})$ is an idempotent, using an automorphism of $M_m(D')$, we may assume that
$$\varphi (E_{1,1})=\left(\begin{array}{cc} I_r&0\\
0&0\end{array}\right)\in M_m(D'),$$
where $I_r$ is the $r\times r$ identity  matrix. Since $\varphi
(E_{1,1})M_m(D')$ is a simple $M_m(D')$-module, it is clear that $r=1$. Hence $D\cong \varphi (E_{1,1})M_m(D')\varphi (E_{1,1})\cong D'$.
Furthermore,
$$n=\dim_D(E_{1,1}M_n(D))=\dim_{D'}(\varphi
(E_{1,1})M_m(D'))=m,$$
and the result follows.
\end{proof}

\begin{theorem}\label{Teorema 1.4.4}
Let $R$ be a right Artinian ring. Then $J(R)$ is nilpotent.
\end{theorem}

\begin{proof}
Let $J=J(R)$ and consider the chain
$$J\supseteq J^2\supseteq\dots\supseteq J^n\supseteq\dots$$
Since $R$ is right Artinian, there exists a positive integer $n$
such that $J^n=J^m$ for all $m\geq n$. Let $I=J^n$. Suppose that
$I\neq 0$. Since $R$ is right Artinian and $I=I^2$, there exists a right ideal $M$ which is minimal with the property
$MI\neq 0$. Since $MI^2=MI\neq 0$,
we have that $MI=M$. It is clear that $M$ is generated by one element, hence, by Nakayasma's lemma, $M=\{0\}$, in contradiction with $MI\neq 0$. Therefore,
$I=J^n=0$ and the result follows.
\end{proof}

\begin{theorem}[Wedderburn-Artin theorem]
Let $R$ be a ring. Then $R$ is right (left) Artinian and semiprimitive if and only if
$$R\cong M_{n_1}(D_1)\times \dots\times M_{n_r}(D_r),$$
where the $D_i$ are division rings. Furthermore, $r$ is determined by $R$, and the pairs $(D_i,n_i)$ are determined up to a permutation.
\end{theorem}

\begin{proof}
Let $\mathcal{C}=\{ A\mid A=Ann^r_R(S)$ where $S$ is a simple right $R$-module $\}$. Suppose that $R$ is right Artinian and semiprimitive.
By Theorem \ref{Teorema 1.3.1}, $J(R)=\bigcap_{A\in\mathcal{C}}A=0$. Let $\mathcal{A}=\{\bigcap_{A\in \mathcal{F}}A \mid \mathcal{F}$ is a finite subset of $\mathcal{C}\}$. Since $R$ is right Artinian, there exist $A_1,\dots,A_r\in \mathcal{C}$ such that
$\bigcap_{i=1}^rA_i$ is a minimal element of $\mathcal{A}$. Suppose that
$\bigcap_{i=1}^rA_i\neq 0$ and let $a\in\bigcap_{i=1}^rA_i\setminus\{ 0\}$.
Since $J(R)=0$, there exists $A\in \mathcal{C}$ such that $a\notin A$. Hence $A\cap (\bigcap_{i=1}^rA_i)$ is strictly contained in $\bigcap_{i=1}^rA_i$, in contradiction with the minimality of $\bigcap_{i=1}^rA_i$. Hence $\bigcap_{i=1}^rA_i=0$.

We may assume that for every $j\in\{ 1,\dots ,r\}$  
$$\bigcap_{1\leq i\leq r,\;
i\neq j}A_i\neq \{0\}.$$
Let $\varphi\colon R\rightarrow \prod_{i=1}^rR/A_i$ be the map defined by
$\varphi(x)=(x+A_1,\dots,x+A_r)$. It is clear that $\varphi$ is an injective homomorphism of rings. We shall see that it is surjective. By Theorem \ref{Teorema 1.4.3},
$R/A_i$ is simple. Hence $A_i$ is a maximal ideal of $R$. Thus
$A_1+\bigcap_{i=2}^rA_i=R$. Hence there exist $a_1\in A_1$ and
$b_1\in\bigcap_{i=2}^rA_i$ such that $1=a_1+b_1$, and thus $\varphi
(1-a_1)=(1+A_1,A_2,\dots ,A_r)$. Similarly one can see that there exists $a_j\in A_j$ such that
$\varphi(1-a_j)=(A_1,\dots ,A_{j-1},1+A_j,A_{j+1},\dots, A_r)$. Now it is easy to see that $\varphi$ is surjective.
Therefore $\varphi$ is an isomorphism. By Theorem \ref{Teorema 1.4.3},
$$R\cong M_{n_1}(D_1)\times\dots\times M_{n_r}(D_r),$$
where the $D_i$ are division rings.

The converse is a consequence of the fact that every finite product of right Artinian rings is right Artinian. 
Note that  $r$ is the number of minimal ideals of $R$ and that these minimal ideals are rings isomorphic to $M_{n_i}(D_i)$.
By Theorem \ref{Teorema 1.4.3}, the result follows.
\end{proof}

\section{Rings without unity}

Let $S$ be a ring without unity. Consider $S^1=\Z\times S$ with the addition defined componentwise and the multiplication defined by
the rule
$$(z_1,a)\cdot (z_2,b)=(z_1z_2,z_1b+z_2a+ab),
$$
for all $z_1,z_2\in\Z$ and all $a,b\in S$. It is easy to check that $S^1$ is a ring and that $(1,0)$ is its unit-element. Furthermore, 
$\{0\}\times S$ is an ideal of $S^1$, which is naturally isomorphic to $S$ (as rings without unity).

We say that a ring $S$ without unity is a \emph{(Jacobson) radical ring} if it is isomorphic to the Jacobson radical of a ring (with unity). Note that if $S$ is a radical ring and $R$ is a ring (with unity) such that $S\cong J(R)$, we have that for every $a\in S$ there exists a unique $b\in S$ such that $a+b+ab=a+b+ba=0$. This can be proved as follows. Let $\psi\colon S\rightarrow R$ be an injective homomorphism of rings without uniy such that $\im(\psi)=J(R)$, this exists because $S\cong J(R)$. Then, for $a\in S$,
$1+\psi(a)\in R$ is invertible. Thus there exists $c\in R$ such that $(1+\psi(a))(1+c)=(1+c)(1+\psi(a))=1$. Note that this implies that
$c=-\psi(a)c-\psi(a)\in J(R)$. Hence there exists $b\in S$ such that $\psi(b)=c$. Therefore 
$$a+b+ab=a+b+ba=0.$$ 
The uniqueness of $b$ is clear. 
We shall study the  maximal left ideals of $S^1=\Z\times S$. Let $M$ be a maximal left ideal of $S^1$. Since for all $a\in S$ there exists $b\in S$ such that $a+b+ab=a+b+ba=0$, we have that $(1,a)(1,b)=(1,b)(1,a)=1$.
Note that if $(0,a)\notin M$, then there exist $(z_1,c)\in S^1$ and $(z_2,d)\in M$ such that
$$(z_1,c)(0,a)+(z_2,d)=(1,0),$$
that is $z_2=1$ and $z_1a+ca+d=0$, thus $(1,d)\in M$, a contradiction because $(1,d)$ is invertible in $S^1$. Hence $\{0\}\times S\subseteq M$. Now it is easy to see that $M=I\times S$, where $I$ is a maximal ideal of $\Z$. Therefore
$J(S^1)=\{0\}\times S\cong S$. 

Thus we have proved the following result.
\begin{proposition}\label{Prop5.1}
	Let $S$ be a ring without unity. Then the following conditions are equivalent.
	\begin{itemize}
		\item[(i)] $S$ is a radical ring.
		\item[(ii)] For all $a\in S$ there exists a unique $b\in S$ such that $a+b+ab=a+b+ba=0$.
		\item[(iii)] $S\cong J(S^1)$. \qed 
	\end{itemize}
	\end{proposition}  

Let $S$ be a ring without unity (or with unity).
Define on $S$ the operation $\circ$ by
$$a\circ b=a+b+ab,$$
for all $a,b\in S$. It is easy to see that $(S,\circ)$ is a monoid with neutral element $0$.
Note that $S$ is a radical ring if and only if $(S,\circ)$ is a group. If $a\in S$ is invertible in the monoid $(S,\circ)$, we shall denote by $a'$ its inverse.

A \emph{nil ring} is a ring $S$ without unity such that every element of $S$ is nilpotent. Note that every nil ring is a radical ring.

\begin{example} 
Let $K$ be a field. Then $XK[\![X]\!]$ is a radical ring 
and it is not a nil ring.
\end{example}


\section{Exercises}

\begin{prob}
    \label{prob:Zorn_simple}
    \textcolor{blue}{Prove that unitary rings have simple modules.} 
\end{prob}

\begin{prob}
	\label{prob:Rump}
	Prove Proposition~\ref{pro:Rump}. 
\end{prob}

\begin{prob}
	If $X$ is a cycle set, then $x\cdot (y\cdot y)=((y*x)\cdot y)\cdot ((y*x)\cdot y)$, where
	$y*x=z$ if and only if $y\cdot z=x$. 
\end{prob}

\begin{prob}
	\label{prob:CS}
	Prove Theorem~\ref{thm:CS}. 
\end{prob}

\begin{prob}
\label{prob:Herstein}
Let $D$ be the ring of rationals with odd denominators. Let
$R=\begin{pmatrix}
    D & \Q\\
    0 & \Q
\end{pmatrix}$. Prove that $R$ is right noetherian and 
$J(R)=\begin{pmatrix}
J(D) & \Q\\
0 & 0
\end{pmatrix}$. Prove that 
$J(R)^n\supseteq\begin{pmatrix}0&\Q\\0&0\end{pmatrix}$ and hence $\bigcap_nJ(R)^n$ is non-zero. 
\end{prob}

\section{Open problems}


% \begin{conjecture}
% \label{conj:Koethe}
% If $R$ is a nil ring, then $R[X]$ is a radical ring. 
% \end{conjecture}

% \begin{conjecture}
% \label{conj:Koethe3}
% If $R$ is a nil ring, then $M_2(R)$ is a nil ring. 
% \end{conjecture}

% \begin{conjecture}
% \label{conj:Koethe4}
% Let $n\geq3$. If $R$ is a nil ring, then $M_n(R)$ is a nil ring. 
% \end{conjecture}

%Now we list some problems related to the solutions to the YBE. 

\begin{problem}[Jacobson]
\label{prob:Jacobson}
\index{Jacobson conjecture}
\index{Jacobson--Herstein conjecture}
Let $R$ be a noetherian ring. Is then 
\[
\bigcap_{n\geq1}J(R)^n=\{0\}?
\]
\end{problem}

\begin{problem}[K\"othe]
	\label{prob:Koethe}
	Let $R$ be a ring. Is the sum 
	of two arbitrary nil left ideals of $R$ is nil?
\end{problem}

\begin{problem}
	Construct and enumerate involutive solutions of size $11$. 
\end{problem}

\begin{problem}
	Estimate the number of solutions of size $n$ for $n\to\infty$. 
\end{problem}

\section*{Notes}

The material on non-commutative ring theory is standard, see for example~\cite{MR3308118} {\bf Ferran: quiz\'as deber\'{i}amos citar el libro Algebra, vol 2, de P. M. Cohn}.
Radical rings were introduced by Jacobson in~\cite{MR12271}. Nil rings were
used by Zelmanov in his solution to Burnside's problem, see for example~\cite{MR1199575}. 

Open problem \ref{prob:Jacobson} was originally formulated by Jacobson in 1956 \cite{MR0222106} 
for one-sided noetherian rings. In 1965 Herstein \cite{MR188253} found a counterexample
in the case of one-sided noetherian rings (see Exercise \ref{prob:Herstein})
and reformulated the conjecture as it appears here. 

Open problem~\ref{prob:Koethe} is the well-known K\"othe's conjecture. 
The conjecture was first formulated in 1930, see \cite{MR1545158}. It is known to be true
in several cases. In full generality, the problem is still open. In~\cite{MR306251} 
Krempa proved that
the following statements are equivalent:
\begin{enumerate}
	\item K\"othe's conjecture is true.  
	\item If $R$ is a nil ring, then $R[X]$ is a radical ring. 
	\item If $R$ is a nil ring, then $M_2(R)$ is a nil ring. 
	\item Let $n\geq2$. If $R$ is a nil ring, then $M_n(R)$ is a nil ring. 
\end{enumerate}

In 1956 Amitsur formulated the following conjecture, see for example
\cite{MR0347873}: If $R$ is a nil ring, then $R[X]$ is a nil ring. In~\cite{MR1793911} 
Smoktunowicz found a counterexample to Amitsur's conjecture. 
This counterexample suggests that K\"othe's conjecture might be false. 
A simplification of Smoktunowicz's example
appears in~\cite{MR3169522}. See \cite{MR1879880,MR2275597} for more
information on K\"othe's conjecture and related topics. 


Rump introduced cycle sets in~\cite{MR2132760}. The bijective correspondence of 
Theorem~\ref{thm:CS} was 
also proved by Rump in~\cite{MR2132760}. A similar result can be 
found in~\cite[Proposition 2.2]{MR1722951}. 

The numbers of Table~\ref{tab:IYB} were computed in~\cite{MR4405502}
using a combination of~\cite{GAP4} and constraint programming techniques. 
The algorithm is based on an idea of Plemmons~\cite{MR0258994}, originally 
conceived to construct non-isomorphic semigroups.  

