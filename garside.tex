\chapter{Garside groups}
\label{Garside}


\section{A}

In this chapter prove that the structure group of a finite involutive solution
is a Garside group. 

\index{Monoid}
A \emph{monoid} is a non-empty set $M$ provided 
with an associative binary operation $M\times M\to M$, $(x,y)\mapsto xy$, 
and an identity element. A monoid $M$ is said to be \emph{cancellative} 
if
\[
xy=xz\implies y=z
\quad
\text{and}
\quad 
xy=zy\implies x=z
\]
for all $x,y,z\in M$. 

\begin{definition}
\index{Garside!monoid}
\index{Monoid!Garside}
A \emph{Garside monoid} is a pair $(M,\Delta)$, where $M$ is a cancellative monoid such that
\begin{enumerate}
    \item There exists a map $d\colon M\to\N$ such that $d(xy)\geq d(x)+d(y)$ and $d(x)\ne0$ if $x\ne 1$. 
    \item 
    \item $\Delta$ is a Garside element of $M$...
    \item The family of all divisors of $\Delta$ is finite. 
\end{enumerate}
\end{definition}

\begin{definition}
A group $G$ is said to be a \emph{Garside group} if...
\end{definition}

Structure groups of involutive solutions are Garside groups. 

\begin{theorem}
\label{thm:Chouraqui}
Let $(X,r)$ be an involutive solution. Then $G(X,r)$ is a Garside group.
\end{theorem}

\begin{proof}

\end{proof}

At this point it is easy to prove the following important result 
of Gateva--Ivanova and Van den Bergh. 

\begin{theorem}
\label{thm:torsion_free}
Let $(X,r)$ be an involutive solution. Then $G(X,r)$ has no torsion. In particular, $G(X,r)$ is a Bieberbach group.
\end{theorem}

\begin{proof}

\end{proof}

As a consequence we obtain the following result on linear 
representations of the structure group of an involutive solution.

\begin{theorem}
\label{thm:ESS}
\end{theorem}

\begin{proof}

\end{proof}

\begin{theorem}
\label{thm:D}
\end{theorem}

\begin{proof}

\end{proof}

\begin{definition} A Garside monoid is a pair $(M,\Delta)$ where $M$ is a
cancellative  monoid such that:
\begin{itemize}
\item[(1)] there exists $d: M \longrightarrow\mathbb{N}$ satisfying
$d(ab)\geq d(a) + d(b)$ and $a\neq 1$ implies $d(a)\neq 0$,
\item[(2)] any two elements of $M$ have a left- and a right-lcm and a left- and
a right-gcd,
\item[(3)]  $\Delta$ is a Garside element of M, this meaning that
the left- and right-divisors of $\Delta$ coincide and generate $M$,
\item[(4)] the family of all divisors of $\Delta$ in $M$ is finite.
\end{itemize}
\end{definition}

Recall that a non-invertible element of a monoid is an atom if it is
not a product of two non-invertible elements. A monoid $S$ is atomic
if every non-invertible element of $S$ is a product of atoms.


Note that if $(M,\Delta)$ is a Garside monoid, then condition (1) in
the definition implies that $d(1)=0$ and $M$ is conical, that is
$ab=1$ implies $a=b=1$. Furthermore, $M$ is atomic and the set of
atoms of $M$ is finite by (4). Moreover, for every $a\in M$, the
supremum of the number of atoms in the factorization of $a$ as a
product of atoms is finite. Note that if $a\in M$ is a divisor of
$\Delta$, then there exist $a',a''\in M$ such that
$aa'=a'a''=\Delta$ and $a\Delta=\Delta a''$. Now it is easy to see
that $M$ satisfies the left and right Ore conditions. Thus it has
left and right group of fractions $G=M^{-1}M=MM^{-1}$.

\begin{definition} A group $G$ is said to be a Garside group if it is the
(left) group of fractions of a submonoid $M$ and there exists
$\Delta\in M$ such that $(M,\Delta)$ is a Garside monoid.
\end{definition}

In \cite[Theorem~3.3]{MR2764830}, Chouraqui proved that the structure
group of a finite solution of the YBE is a Garside group. We shall
prove this result using the natural structure of left brace of the
structure group of a solution of the YBE.

\begin{lemma}\label{gar1} Let $(X,r)$ be a finite solution of the YBE. Let
$n=[G(X,r):\Soc(G(X,r))]$. Let $z$ be an integer. Let
$\Delta_z=\sum_{x\in X}zx$. Then $g\Delta_z=g+\Delta_z$, for all
$g\in G(X,r)$. In particular, $\Delta_z^m=m\Delta_z$, for all
integer $m$, and $n\Delta_z$ is a central element of the structure
group $G(X,r)$.
\end{lemma}

\begin{proof} Let $g\in G(X,r)$. We have
$$g\Delta_z=\lambda_g(\Delta_z)+g=\lambda_g(\sum_{x\in X}zx)+g=\sum_{x\in X}z\lambda_g(x)+g=\Delta_z+g.$$
The second part is a consequence of the first and the fact that
$n\Delta_z\in \Soc(G(X,r))$.
\end{proof}

Let $(X,r)$ be a finite solution of the YBE. Let $M(X,r)$ be the
submonoid of $G(X,r)$ generated by $X$.

\begin{lemma}\label{gar2} $M(X,r)=\{ \sum_{x\in X}z_xx\mid z_x\in
\mathbb{N}\}$ and $M(X,r)^{-1}=-M(X,r)$.
\end{lemma}

\begin{proof} Let $M=\{ \sum_{x\in X}z_xx\mid z_x\in \mathbb{N}\}$.
Note that $\lambda_g(a)\in M$, for all $g\in G(X,r)$ and all $a\in
M$.

Every element of $M(X,r)$ is of the form $x_1\cdots x_m$, with $x_1,
\dots ,x_m\in X$. We shall prove that $x_1\cdots x_m\in M$ by
induction on $m$. For $m=1$ it is clear. Suppose that $m>1$ and that
$y_1\cdots y_{m-1}\in M$ for all $y_1,\dots,y_{m-1}\in X$. By the
induction hypothesis, $x_2\cdots x_m\in M$, and thus
$$x_1\cdots x_m=x_1+\lambda_{x_1}(x_2\cdots x_m)\in M.$$
Hence $M(X,r)\subseteq M$. Let $\sum_{x\in X}z_xx\in M$. We shall
prove that $\sum_{x\in X}z_xx\in M(X,r)$ by induction on
$t=\sum_{x\in X}z_x$. For $t=1$ it is clear. Suppose that $t>1$ and
that $\sum_{x\in X}a_xx\in M(X,r)$ for all $\sum_{x\in X}a_xx\in M$
with $\sum_{x\in X}a_x<t$. Since $t>1$ there exists $x_0\in X$ such
that $z_{x_0}>0$. Let $a=-\lambda^{-1}_{x_0}(x_0)+\sum_{x\in
X}z_x\lambda^{-1}_{x_0}(x)$. Now we have that $a\in M$ and, by the
induction hypothesis, $a\in M(X,r)$. Thus
$$\sum_{x\in X}z_xx=x_0-x_0+\sum_{x\in X}z'_xx=x_0\lambda^{-1}_{x_0}(-x_0+\sum_{x\in X}z'_xx)=x_0a\in M(X,r).$$
Hence $M=M(X,r)$.

Note that $g^{-1}=-\lambda_{g^{-1}}(g)$ and
$-g=\lambda_{(-g)^{-1}}(g)^{-1}$ for all $g\in G(X,r)$. Therefore
$M^{-1}=-M$, and the result follows.
\end{proof}

Note that $M(X,r)$ has a degree function
$$\deg\colon M(X,r)\longrightarrow \mathbb{N}$$
defined by $\deg(x_1\cdots x_m)=m$, for all $x_1,\dots ,x_m\in X$.

\begin{lemma}\label{degree} Let $a\in M(X,r)$. By Lemma~\ref{gar2},
$a=\sum_{x\in X}a_xx$, for some $a_x\in \mathbb{N}$. Then
$\deg(a)=\sum_{x\in X}a_x$.
\end{lemma}

\begin{proof} We will prove the result by induction on $\deg(a)$.
If $\deg(a)=1$, then $a=x$ for some $x\in X$, and the result is
clear. Suppose that $\deg(a)=m>1$ and that the result is true for
all $b\in M(X,r)$ with $\deg(b)<m$. We have that $a=x_1\cdots x_m$,
for some $x_1,\dots ,x_m\in X$. By the induction hypothesis,
$$x_2\cdots x_m=\sum_{x\in X}b_xx,$$
for some $b_x\in \mathbb{N}$ such that $\sum_{x\in X}b_x=m-1$. We
have
\begin{eqnarray*}
a&=&x_1+\lambda_{x_1}(x_2\cdots x_m)\\
&=&x_1+\lambda_{x_1}(\sum_{x\in X}b_xx)\\
&=&x_1+\sum_{x\in X}b_x\lambda_{x_1}(x).
\end{eqnarray*}
Since $\lambda_{x_1}(x)\in X$, for all $x_1,x\in X$, the result
follows.
\end{proof}


\begin{lemma}\label{gar3} Let $a,b\in M(X,r)$ and $c,d\in M(X,r)^{-1}$. By
Lemma~\ref{gar2},
$$a=\sum_{x\in X}a_xx,\; b=\sum_{x\in X}b_xx,\; c=\sum_{x\in X}c_xx\quad\mbox{and}\quad d=\sum_{x\in X}d_xx,$$
for some $a_x,b_x,-c_x,-d_x\in \mathbb{N}$. Then
\begin{itemize}
\item[(i)] $b^{-1}a\in M(X,r)$ if and only if $b_x\leq a_x$ for all
$x\in X$.
\item[(i)] $c^{-1}d\in M(X,r)^{-1}$ if and only if $c_x\geq d_x$ for all
$x\in X$.
\end{itemize}
\end{lemma}

\begin{proof} $(i)$ Note that
$$b^{-1}a=b^{-1}+\lambda_{b^{-1}}(a)=-\lambda_{b^{-1}}(b)+\lambda_{b^{-1}}(a)=\lambda_{b^{-1}}(a-b).$$
Therefore $b^{-1}a\in M(X,r)$ if and only if $a-b\in M(X,r)$ and the
result follows easily by Lemma~\ref{gar2}.

$(ii)$ As above $c^{-1}d=\lambda_{c^{-1}}(d-c)$. Therefore
$c^{-1}d\in M(X,r)^{-1}$ if and only if $d-c\in M(X,r)^{-1}$ and the
result follows easily by Lemma~\ref{gar2}.
\end{proof}

\begin{lemma}\label{gar4} Any two elements of $M(X,r)$ have a left- and a
right-lcm and a left- and a right-gcd.
\end{lemma}

\begin{proof} Let $a,b\in M(X,r)$. By Lemma~\ref{gar2}
$$a=\sum_{x\in X}a_xx\quad\mbox{and}\quad b=\sum_{x\in X}b_xx,$$
for some $a_x,b_x\in \mathbb{N}$. By Lemma~\ref{gar3} it is clear
that the left-gcd of $a$ and $b$ is
$$\lgcd(a,b)=\sum_{x\in X}\min(a_x,b_x)x,$$
and that its right-lcm is
$$\rlcm(a,b)=\sum_{x\in X}\max(a_x,b_x)x.$$
Now consider the monoid $M(X,r)^{-1}$. Since $a^{-1},b^{-1}\in
M(X,r)^{-1}$, by Lem\-ma~\ref{gar2}
$$a^{-1}=\sum_{x\in X}a'_xx\quad\mbox{and}\quad b^{-1}=\sum_{x\in X}b'_xx,$$
for some non-positive integers $a'_x,b'_x$. By Lemma~\ref{gar3} it
is clear that the left-gcd of $a^{-1}$ and $b^{-1}$ in $M(X,r)^{-1}$
is
$$\lgcd(a^{-1},b^{-1})=\sum_{x\in X}\max(a'_x,b'_x)x,$$
and that its right-lcm is
$$\rlcm(a^{-1},b^{-1})=\sum_{x\in X}\min(a'_x,b'_x)x.$$
Note that if $d^{-1}=\lgcd(a^{-1},b^{-1})$, then $d$ is the
right-gcd of $a$ and $b$ in $M(X,r)$, and if
$m^{-1}=\rlcm(a^{-1},b^{-1})$, then $m$ is the left-lcm of $a$ and
$b$ in $M(X,r)$. Thus the result follows.
\end{proof}

\begin{theorem}\label{chouraqui} (\cite[Theorem~3.3]{MR2764830}) Let $(X,r)$ be a
finite solution of the YBE. Let $m$ be a positive integer. Then
$(M(X,r),\Delta_m)$ is a Garside monoid and thus $G(X,r)$ is a
Garside group.
\end{theorem}

\begin{proof} It is clear that the submonoid $M(X,r)$ of $G(X,r)$
is cancellative. We have seen that $M(X,r)$ has a degree function
$\deg$. In particular condition (1) in the definition of Garside
monoid is satisfied.

By Lemma~\ref{gar4}, condition (2) in the definition of Garside
monoid is satisfied by $M(X,r)$.

By Lemma~\ref{gar2}, $\Delta_m\in M(X,r)$ and, by Lemma~\ref{gar1},
$\Delta_m^{-1}=-\Delta_m$. By Lemma~\ref{gar3}, the set of
left-divisors of $\Delta_m$ in $M(X,r)$ is
$$D_1=\{\sum_{x\in X}z_xx\mid z_x\in \mathbb{Z},\; 0\leq z_x\leq m, \mbox{ for all }x\in X\},$$
and the set of left-divisors of $\Delta_m^{-1}$ in the monoid
$M(X,r)^{-1}$ is
$$D_2=\{-\sum_{x\in X}z_xx\mid z_x\in \mathbb{Z},\; 0\leq z_x\leq m, \mbox{ for all }x\in X\}.$$
Note that $$(\sum_{x\in X}z_xx)^{-1}=-\lambda_{(\sum_{x\in
X}z_xx)^{-1}}(\sum_{x\in X}z_xx)=-\sum_{x\in
X}z_x\lambda_{(\sum_{x\in X}z_xx)^{-1}}(x),$$ for all
$z_x\in\mathbb{Z}$. Hence $D_2^{-1}=D_1$. Thus the set of
right-divisors of $\Delta_m$ in $M(X,r)$ also is $D_1$, which is
finite and contains $X$. Therefore $\Delta_m$ is a Garside element
of $M(X,r)$. Hence $(M(X,r),\Delta_m)$ is a Garside monoid, and
clearly $G(X,r)$ is its group of left (right) group of fractions.
\end{proof}

In \cite{MR3374524} Dehornoy develops a sort of right-cyclic calculus to
give another proof of Theorem~\ref{chouraqui}. He also describes
finite quotients of $G(X,r)$ that play a role similar to the role
that Coxeter groups play for Artin-Tits groups. We shall see this
using the left brace structure of $G(X,r)$.

\begin{definition} (\cite[Definition~5.1]{MR3374524}) If $(M,\Delta)$ is a Garside
monoid and $G$ is its group of fractions, a surjective homomorphism
$\pi\colon G\longrightarrow \overline{G}$ is said to provide a
Garside germ for $(G,M,\Delta)$ if there exists a map
$\sigma\colon\overline{G}\longrightarrow M$ such that
$\pi\circ\sigma$ is the identity, the image of $\sigma$ is the
family of all divisors of $\Delta$ in $M$, and $M$ admits a
presentation
$$\langle \sigma(\overline{G})\mid \{ \sigma(f)\sigma(g)=\sigma(fg)\mid f,g\in \overline{G}\mbox{ and }\| f\|_{\overline{S}}+\| g\|_{\overline{S}}=\| fg\|_{\overline{S}} \}\rangle,$$
where $\overline{S}$ is the image under $\pi$ of the atom set of
$M$, and $\| f\|_{\overline{S}}$ is the length of a shortest
decomposition of $f$ as a product of elements of $\overline{S}$.
\end{definition}

\begin{proposition} (\cite[Proposition~5.2]{MR3374524}) Let $(X,r)$ be a finite
solution of the YBE. Let $|X|=n$, $G=G(X,r)$, $M=M(X,r)$ and $d=[G
:\Soc(G)]$. Let $k$ be a positive integer such that $kd-1\geq 1$ and
$\Delta=\Delta_{kd-1}$. Then $N=\{kdg\mid g\in G\}$ is a normal
subgroup of $G$ and the natural map $\pi\colon G\longrightarrow G/N$
provides a Garside germ for $(G,M,\Delta)$. The group $G/N$ has
order $(kd)^{n}$ and the kernel of $\pi$ is isomorphic to
$\mathbb{Z}^{n}$.
\end{proposition}

\begin{proof} By Theorem~\ref{chouraqui}, $(M,\Delta)$ is a Garside
monoid. We also know that the set of left-divisors of $\Delta$ is
$$D_1=\{\sum_{x\in X}z_xx\mid z_x\in \mathbb{Z},\; 0\leq z_x\leq kd-1 \}.$$
It is easy to see that $N$ is a left ideal of the left brace $G$.
Let $g,h\in G$. Then, since $d=[G:\Soc(G)]$, $dg\in Soc(G)$ and
\begin{align*}h^{-1}(kdg)h=&h^{-1}(kdg+h)=\lambda_{h^{-1}}(kdg+h)+h^{-1}\\
=&kd\lambda_{h^{-1}}(g)+\lambda_{h^{-1}}(h)+h^{-1}=kd\lambda_{h^{-1}}(g)\in
N.\end{align*}
Hence $N$ is an ideal of the left brace $G$. In
particular, the multiplicative group of $N$ is a normal subgroup of
the structure group $G$. The natural homomorphism $\pi\colon
G\longrightarrow G/N$ is in fact a homomorphism of left braces.
Since the additive group of the left brace $G$ is free abelian with
basis $X$, the restriction of $\pi$ to $D_1$ is a bijection
$\pi|_{D_1}\colon D_1\longrightarrow G/N$. Let
$\sigma=(\pi|_{D_1})^{-1}$. We have that $\pi(\sigma(f))=f$, for all
$f\in G/N$. We know by \cite{MR1637256} that $M$ admits a presentation
as a monoid
$$\langle X\mid xy=zt\mbox{ whenever }r(x,y)=(z,t)\rangle.$$
The set of atoms of $M$ is $X$. Let $\overline{X}=\pi(X)$. Let $\|
a\|_{\overline{X}}$ denote the length of a shortest decomposition of
$a\in G/N$ as a product of elements of $\overline{X}$. Let $a\in
G/N$ and
$$\sigma(a)=\sum_{x\in X}a_xx,$$
for some $a_x\in \mathbb{N}$. By Lemma~\ref{degree}, $\|
a\|_{\overline{X}}=\sum_{x\in X}a_x$. Hence, for $f,g\in G/N$,
$$\| f\|_{\overline{X}}+\| g\|_{\overline{X}}=\|
fg\|_{\overline{X}}$$ if and only if
$$\sigma(f)\sigma(g)=\sigma(fg).$$
Since $kd>1$, for $x,y\in X$, $xy\notin D_1$ if and only if $kd=2$
and $xy=x+\lambda_x(y)=2x$, that is if and only if $kd=2$ and
$\lambda_x(y)=x$. In this case,
$r(x,y)=(\lambda_x(y),\lambda^{-1}_{\lambda_x(y)}(x))=(x,y)$. Now it
is easy to see that $M$ admits a presentation
$$\langle \sigma(G/N)\mid \{ \sigma(f)\sigma(g)=\sigma(fg)\mid f,g\in G/N\mbox{ and }\| f\|_{\overline{X}}+\| g\|_{\overline{X}}=\| fg\|_{\overline{X}} \}\rangle.$$
Therefore the result follows.
\end{proof}





\section*{Exercises}

\section*{Open problems}

\section*{Notes}

\index{Chouraqui, F.}
\index{Dehornoy, P.}
\index{Ced\'o, F.}
Theorem~\ref{thm:Chouraqui} was proved by Chouraqui in~\cite{MR2764830}. Our proof is based on the work 
of Dehornoy~\cite{MR3374524} and the presentation of Cedó's survey~\cite{MR3824447}. 

\index{Gateva--Ivanova, T.}
\index{Van den Bergh, M.}
Theorem~\ref{thm:torsion_free} was proved by Gateva--Ivanova and Van den Bergh in~\cite{MR1637256} using somewhat 
different methods. 