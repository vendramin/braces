\chapter{Garside structure and involutive solutions}

In this chapter prove that the structure group of a finite involutive solution
is a Garside group. 

\index{Monoid}
A \emph{monoid} is a non-empty set $M$ provided 
with an associative binary operation $M\times M\to M$, $(x,y)\mapsto xy$, 
and an identity element. A monoid $M$ is said to be \emph{cancellative} 
if
\[
xy=xz\implies y=z
\quad
\text{and}
\quad 
xy=zy\implies x=z
\]
for all $x,y,z\in M$. 

\begin{definition}
\index{Garside!monoid}
\index{Monoid!Garside}
A \emph{Garside monoid} is a pair $(M,\Delta)$, where $M$ is a cancellative monoid such that
\begin{enumerate}
    \item There exists a map $d\colon M\to\N$ such that $d(xy)\geq d(x)+d(y)$ and $d(x)\ne0$ if $x\ne 1$. 
    \item 
    \item $\Delta$ is a Garside element of $M$...
    \item The family of all divisors of $\Delta$ is finite. 
\end{enumerate}
\end{definition}

\begin{definition}
A group $G$ is said to be a \emph{Garside group} if...
\end{definition}

Structure groups of involutive solutions are Garside groups. 

\begin{theorem}
\label{thm:Chouraqui}
Let $(X,r)$ be an involutive solution. Then $G(X,r)$ is a Garside group.
\end{theorem}

\begin{proof}

\end{proof}

At this point it is easy to prove the following important result 
of Gateva--Ivanova and Van den Bergh. 

\begin{theorem}
\label{thm:torsion_free}
Let $(X,r)$ be an involutive solution. Then $G(X,r)$ has no torsion. In particular, $G(X,r)$ is a Bieberbach group.
\end{theorem}

\begin{proof}

\end{proof}

As a consequence we obtain the following result on linear 
representations of the structure group of an involutive solution.

\begin{theorem}
\label{thm:ESS}
\end{theorem}

\begin{proof}

\end{proof}

\begin{theorem}
\label{thm:D}
\end{theorem}

\begin{proof}

\end{proof}

\section*{Exercises}

\section*{Open problems}

\section*{Notes}

\index{Chouraqui, F.}
\index{Dehornoy, P.}
\index{Ced\'o, F.}
Theorem~\ref{thm:Chouraqui} was proved by Chouraqui in~\cite{MR2764830}. Our proof is based on the work 
of Dehornoy~\cite{MR3374524} and the presentation of Cedó's survey~\cite{MR3824447}. 

\index{Gateva--Ivanova, T.}
\index{Van den Bergh, M.}
Theorem~\ref{thm:torsion_free} was proved by Gateva--Ivanova and Van den Bergh in~\cite{MR1637256} using somewhat 
different methods. 