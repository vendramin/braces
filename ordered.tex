\chapter{Ordered groups}
\label{ordered}

% The \textbf{braid group} $\B_n$ is the group with
% generators $\sigma_1,\dots,\sigma_{n-1}$ and relations
% \begin{align*}
%     \sigma_i\sigma_j &= \sigma_j\sigma_i && \text{if $|i-j|>1$},\\
%     \sigma_i\sigma_{i+1}\sigma_i &= \sigma_{i+1}\sigma_i\sigma_{i+1} && \text{if $|i-j|=1$}.
% \end{align*}

% nociones básicas
% grupo de trenzas
% difusos, upp, Kaplansky
% BCV

\section*{A}

\index{Group!left-orderable}
A group $G$ is \textbf{left-orderable} is there is a total ordering $<$ on $G$ 
such that $x<y$ implies $zx<zy$ for all $x,y,z\in G$. Similarly one
defines right ordered groups. 

\begin{example}
The group $\Z$ is left-orderable. 
\end{example}

\begin{example}
	If $G$ is a left-orderable group and $H$ is a subgroup of $G$, then $H$ is left-orderable.
\end{example}

\begin{example}
	Let $G=\Z^2$ and $v\in\R^2$ with irrational slope. Then...
\end{example}

\begin{proposition}
Let 
\[
\begin{tikzcd}
	1 & K & G & Q & 1
	\arrow[from=1-1, to=1-2]
	\arrow["\iota", from=1-2, to=1-3]
	\arrow["p", from=1-3, to=1-4]
	\arrow[from=1-4, to=1-5]
\end{tikzcd}
\]
be a short exact sequence of groups. 
If both $K$ and $Q$ are left-orderable, then $G$ is left-orderable. 
\end{proposition}

\begin{proof}
	We define 
	\[
	x<y\Longleftrightarrow
	\begin{cases}
		1<_Kx^{-1}y & \text{if $p(x)=p(y)$},\\
		p(x)<_Qp(y) & \text{otherwise}.		
	\end{cases}
	\]
	A straightforward computation shows that then $G$ is left-orderble. 
\end{proof}

\begin{example}
	Let us show that $G=\langle x,y:xyx^{-1}=y^{-1}\rangle$ is left-orderable. Let 
	$f\colon G\to\Z$ be given by $x\mapsto 1$ and $y\mapsto 0$. Then $\ker f=\langle y\rangle$ 
	and the sequence 
	\[
	\begin{tikzcd}
	1 & \langle y\rangle & G & \Z & 1
	\arrow[from=1-1, to=1-2]
	\arrow[from=1-2, to=1-3]
	\arrow[from=1-3, to=1-4]
	\arrow[from=1-4, to=1-5]
\end{tikzcd}
\]
is exact. The map
\[
G\to\GL_2(\C),
\quad 
x\mapsto\begin{pmatrix}
-1&0\\
0&1	
\end{pmatrix},
\quad
y\mapsto\begin{pmatrix}
1&1\\
0&1	
\end{pmatrix},
\]
is a group homomorphism. 
In particular, $y$ has infinite order and hence $\langle y\rangle\simeq\Z$ 
is left orderable. The previous proposition implies then that
$G$ is left-orderable. 
\end{example}

\index{Baumslag--Solitar's group}
The previous example is the Baumslag--Solitar group $B(1,-1)$. Recall that for $n,m\in\Z$
the Baumslag--Solitar's group is defined as the 
group \[
B(m,n)=\langle a,b:ba^mb^{-1}=a^n\rangle.
\]
The map 
\[
B(m,n)\to\GL_2(\C),\quad
a\mapsto\begin{pmatrix}
\frac{m}{n}&1\\
0&1	
\end{pmatrix}\quad\text{ and }\quad  
b\mapsto\begin{pmatrix}
-1&0\\
0&1	
\end{pmatrix}
\] 
is a group homomorphism.

% G no es bi-orderable
A group $G$ is said to be \textbf{poly-$\Z$} if...

\begin{exercise}
Prove that poly-$\Z$ groups are left-orderable. 	
\end{exercise}

\index{Group!indicable}
\index{Group!locally indicable}
A group $G$ is said to be \textbf{indicable} if there exists a non-trivial 
group homomorphism $G\to\Z$, and $G$ is said to be \textbf{locally indicable} 
if every finitely generated subgroup of $G$ is indicable.  

\framebox{FIXME}
% hay que completar todo esto! 

\begin{theorem}[Burns--Hale]
\index{Burns, ?}
\index{Hale, ?}
\index{Burns--Hale's theorem}
Let $G$ be a group. Then $G$ is left-orderable if and only if 
for each finitely generated non-trivial subgroup $H$ of $G$ there exists a left-ordered group $L$ 
and a non-trivial group homomorphism $H\to L$.  
\end{theorem}

\begin{proof}
	If $G$ is left-orderable, take $L=H$. 
	
	Let $L$ be a left-orderable group. We claim that for all $\{x_1,\dots,x_n\}\subseteq G\setminus\{1\}$ 
	there exist $\epsilon_1,\dots,\epsilon_n\in\{-1,1\}$ such that 
	\[
	1\not\in S(x_1^{\epsilon_1},\dots,x_n^{\epsilon_n}),
	\]
	where $S(x_1^{\epsilon_1},\dots,x_n^{\epsilon_n})$ denotes the semigroup generated by 
	the set $\{x_1^{\epsilon_1},\dots,x_n^{\epsilon_n}\}$. 
	We proceed by induction on $n$. If $n=1$, then $x_1\in G\setminus\{1\}$. Let $\epsilon_1=1$. If
	$1\in S(x_1)$, then $x_1$ is an element of finite order and hence $\langle x_1\rangle\to L$ is the trivial
	homomorphism. Now assume that the claim holds for some $n\geq 1$. Let $\{x_1,\dots,x_n\}\subseteq G\setminus\{1\}$. 
	By assumption, there exists a non-trivial group homomorphism 
	$h\colon\langle x_1,\dots,x_n\rangle\to L$. In particular, $h(x_i)\ne 1$ for some $i\in\{1,\dots,n\}$. Without loss
	of generality we may assume that $h(x_j)\ne 1$ for all $j\in\{1,\dots,k-1\}$ and 
	$h(j)=1$ for all $j>k$. Since $L$ is left-orderable and $h(x_j)\ne 1$ for all $j\leq k$, there
	are elements $\epsilon_j\in\{-1,1\}$ such that $h(x_j^{\epsilon_j})>1$ for all $j\leq k$. By the inductive hypothesis, 
	there are elements $\epsilon_{k+1},\dots,\epsilon_n\in\{-1,1\}$ such that 
	$1\not\in S(x_{k+1}^{\epsilon_{k+1}},\dots,x_n^{\epsilon_n})$. If $1\in S(x_1^{\epsilon_1},\dots,x_n^{\epsilon_n})$, then
	$1=h(1)>1$, a contradiction.       
\end{proof}

An inmediate corollary:

\begin{corollary}
	Locally indicable groups are left-orderable. 
\end{corollary}

Another consequence of Burns--Hale's Theorem:

\begin{exercise}
	Let $G$ be a group and $\{N_\alpha:\alpha\}$ be a collection of normal subgroups of $G$ such that
	$\cap_{\alpha}N_\alpha=\{1\}$. If $G/N_{\alpha}$ is left-ordererable for all $\alpha$, then 
	$G$ is left-orderable. 
\end{exercise}

\index{Group!bi-orderable}
A group $G$ is \textbf{bi-orderable} if there exists a total ordering $<$ in $G$ 
such that $x<y$ implies $xz<yz$ and $zx<zy$ for all $z\in G$. 

\index{Group!with the unique product property}
\index{Unique product property}
A group $G$ satisfies the \textbf{unique product property} if there are non-empty
finite subsets $A$ and $B$ such that $|gA\cap B|=1$ for some $g\in G$. Thus $G$ 
satisfies the unique product property if and only if for all finite non-empty subsets
$A$ and $B$ there exists $g\in G$ such that $g=ab$ for unique elements 
$a\in A$ and $b\in B$.  

\begin{proposition}
	A group with the unique product property is torsion-free.	
\end{proposition}

\begin{proof}
	Assume that $G$ has torsion and let $g\in G$ be an element of order $n\geq2$. 
	Let $A=B=\{1,g,g^2,\dots,g^{n-1}\}$. Then $g$ admits more than one representation
	of the form $g=ab$ for $a\in A$ and $b\in B$, so $G$ cannot have the unique product property. 
\end{proof}

\begin{proposition}
	A left-orderable group satisfies the unique product property.
\end{proposition}

\begin{proof}
	Let $G$ be a group and 
    $A=\{a_1,\dots,a_n\}$ and $B=\{b_1,\dots,b_m\}$ be non-empty 
    subsets of $G$. We may assume that $a_1<a_2<\cdots<a_n$ and
    $b_1<b_2<\cdots<b_m$. Let $g\in G$ be such that $ga_n=b_1$. Then
    $ga_1<ga_2<\cdots<ga_n=b_1<b_2<\cdots<b_m$. 
\end{proof}

A group $G$ satisfies the \textbf{double unique product property} if for any two given finite non-empty 
subsets $A$ and $B$ of $G$ such that $|A|+|B|>2$ there exist at least two unique products in $AB$. 

\begin{theorem}[Strojnowski]
	\label{thm:Strojnowski}
	\index{Strojonowski's!Theorem}
	Sea $G$ un grupo. Las siguientes afirmaciones son equivalentes:
	\begin{enumerate}
		\item $G$ tiene la propiedad del doble producto único.
		\item Para todo subconjunto $A\subseteq G$ finito y no vacío, existe al
			menos un producto único en $AA=\{a_1a_2:a_1,a_2\in A\}$.
		\item $G$ tiene la propiedad del producto único.
	\end{enumerate}
\end{theorem}

\begin{proof}
	La implicación $(1)\implies(2)$ es trivial.  Demostremos que vale
	$(2)\implies(3)$. Si $G$ no tiene la propiedad del producto único, existen
	subconjuntos $A,B\subseteq G$ finitos y no vacíos tales que todo elemento
	de $AB$ admite al menos dos representaciones. Sea $C=AB$. Todo $c\in C$ es
	de la forma $c=(a_1b_1)(a_2b_2)$ con $a_1,a_2\in A$ y $b_1,b_2\in B$. Como
	$a_2^{-1}b_1^{-1}\in AB$, existen $a_3\in A\setminus\{a_2\}$ y $b_3\in B\setminus\{b_1\}$ tales que
	$a_2^{-1}b_1^{-1}=a_3^{-1}b_3^{-1}$. Luego $b_1a_2=b_3a_3$ y entonces
	\[
	c=(a_1b_1)(a_2b_2)=(a_1b_3)(a_3b_2)
	\]
	son dos representaciones distintas de $c$ en $AB$.
	pues $a_2\ne a_3$ y $b_1\ne b_3$.

	Demostremos ahora que $(3)\implies(1)$. Si $G$ tiene la propiedad del
	producto único pero no tiene la propiedad del doble producto único, existen
	subconjuntos $A,B\subseteq G$ finitos y no vacíos con $|A|+|B|>2$ tales que
	en $AB$ existe un único elemento $ab$ con una única representación en $AB$.
	Sean $C=a^{-1}A$ y $D=Bb^{-1}$. Entonces $1\in C\cap D$ y el elemento
	neutro $1$ admite una única representación en $CD$ (pues si $1=cd$ con
	$c=a^{-1}a_1\ne 1$ y $d=b_1b^{-1}\ne 1$, entonces $ab=a_1b_1$ con $a\ne
	a_1$ y $b\ne b_1)$. Sean $E=D^{-1}C$ y $F=DC^{-1}$. Todo elemento de $EF$
	se escribe como $(d_1^{-1}c_1)(d_2c_2^{-1})$. Si $c_1\ne 1$ o $d_2\ne 1$
	entonces $c_1d_2=c_3d_3$ para algún $c_3\in C\setminus\{c_1\}$ y algún
	$d_3\in D\setminus\{d_2\}$. Entonces
	$(d_1^{-1}c_1)(d_2c_2^{-1})=(d_1^{-1}c_3)(d_3c_2^{-1})$ son dos
	representaciones distintas para $(d_1^{-1}c_1)(d_2c_2^{-1})$. Si $c_2\ne 1$
	o $d_1\ne 1$ entonces $c_2d_1=c_4d_4$ para algún $d_4\in D\setminus\{d_1\}$
	y algún $c_4\in C\setminus\{c_2\}$ y entonces, como
	$d_1^{-1}c_2^{-1}=d_4^{-1}c_4^{-1}$,
	$(d_1^{-1}1)(1c_2^{-1})=(d_4^{-1}1)(1c_4^{-1})$.  Como $|C|+|D|>2$, $C$ o
	$D$ contienen algún $c\ne1$, y entonces $(1\cdot 1)(1\cdot 1)=(1\cdot
	c)(1\cdot c^{-1})$. Demostramos entonces que todo elemento de $EF$ tiene al
	menos dos representaciones. 
\end{proof}

% passman lema 1.9 pag 589
\begin{exercise}
	Demuestre que si $G$ es un grupo que satisface la propiedad del producto
	único, entonces $K[G]$ tiene solamente unidades triviales.
\end{exercise}

En general es muy difícil verificar si un grupo posee la propiedad del producto
único. Una propiedad similar es la de ser un grupo difuso. Si $G$ es un grupo
libre de torsión y $A\subseteq G$ es un subconjunto, diremos que $A$ es
antisimétrico si $A\cap A^{-1}\subseteq\{1\}$, donde $A^{-1}=\{a^{-1}:a\in
A\}$. El conjunto de \textbf{elementos extremales} de $A$ se define como
$\Delta(A)=\{a\in A:Aa^{-1}\text{ es antisimétrico}\}$. Luego
\[
	a\in A\setminus\Delta(A)
	\Longleftrightarrow
	\text{existe $g\in G\setminus\{1\}$ tal que $ga\in A$ y $g^{-1}a\in A$}.
\]

\begin{definition}
	\index{Grupo!difuso}
	Un grupo $G$ se dice \textbf{difuso} si para todo subconjunto $A\subseteq
	G$ tal que $2\leq |A|<\infty$ se tiene $|\Delta(A)|\geq2$.
\end{definition}

\begin{lemma}
	Si $G$ es ordenable a derecha, entonces $G$ es difuso.	
\end{lemma}

\begin{proof}
	Supongamos que $A=\{a_1,\dots,a_n\}$ y $a_1<a_2<\cdots<a_n$. Vamos a
	demostrar que $\{a_1,a_n\}\subseteq\Delta(A)$. Si $a_1\in
	A\setminus\Delta(A)$, existe $g\in G\setminus\{1\}$ tal que $ga_1\in A$ y
	$g^{-1}a_1\in A$. Esto implica que $a_1\leq ga_1$ y $a_1\leq g^{-1}a_1$, de
	donde se concluye que $1\leq g$ y $1\leq g^{-1}$, una contradicción. De la
	misma forma se demuestra que $a_n\in \Delta(A)$.
\end{proof}

\begin{lemma}
	\label{lemma:difuso=>2up}
	Si $G$ es difuso, entonces $G$ tiene la propiedad del doble producto único.	
\end{lemma}

\begin{proof}
	Supongamos que $G$ no tiene la propiedad del doble producto único. Existen
	entonces subconjuntos finitos $A,B\subseteq G$ con $|A|+|B|>2$ tales que
	$C=AB$ tiene a lo sumo un producto único. Luego $|C|\geq2$. Como $G$ es
	difuso, $|\Delta(C)|\geq2$. Si $c\in\Delta(C)$, entonces $c$ tiene una
	única expresión como $c=ab$ con $a\in A$ y $b\in B$ (de lo contrario, si
	$c=a_0b_0=a_1b_1$ con $a_0\ne a_1$ y $b_0\ne b_1$. Si $g=a_0a_1^{-1}$,
	entonces $g\ne 1$, $gc=a_0a_1^{-1}a_1b_1=a_0b_1\in C$ y además
	$g^{-1}c=a_1a_0^{-1}a_0b_0=a_1b_0\in C$. Luego $c\not\in\Delta(c)$, una
	contradicción.
\end{proof}



%Un grupo $G$ se dice \textbf{débilmente difuso} si para todo subconjunto
%finito $A\subseteq G$ no vacío se tiene $\Delta(A)\ne\emptyset$. La técnica
%usada para demostrar el lema~\ref{lemma:difuso=>2up} puede usarse para
%demostrar que un grupo débilmente difuso posee la propiedad del producto
%único. El teorema~\ref{theorem:Strojnowski} sugiere entonces la siguiente
%pregunta: 
%
%\begin{problem}
%	¿Existe un grupo débilmente difuso que no sea difuso?
%\end{problem}
%
%\section{El grupo de Promislow}
%
%Veremos un ejemplo concreto de un grupo sin torsión que no es ordenable, no es
%difuso y no tiene la propiedad del producto único.
%
%\begin{exercise}
%	\label{exercise:Dinfty}
%	Demuestre que $G=\langle x,y:x^2=y^2=1\rangle$ es isomorfo al grupo diedral infinito.
%\end{exercise}
%
%\begin{definition}
%	Se define el grupo de Promislow como 
%	\[
%		G=\langle x,y:x^{-1}y^2x=y^{-2},\,y^{-1}x^2y=x^{-2}\rangle.
%	\]
%\end{definition}
%
%\begin{proposition}
%	\label{proposition:Promislow}
%	El grupo de Promislow es libre de torsión y no satisface la propiedad del
%	producto único. 
%\end{proposition}
%
%\begin{proof}
%	
%\end{proof}



\section*{B}

The braid group. 

\section*{C}

Kaplanski's problems. 

\section*{D}

The following result proves that the structure groups of finite solutions
are left-orederable if and only if they are locally indicable. 

\begin{theorem}[Chiswell--Kropholler]
\index{Chiswell, I.}
\index{Kropholler, P.}
\index{Chiswell--Kropholler's theorem}
A left-orderable solvable group is locally indicable.  
\end{theorem}

See~\cite{MR1205890} for the proof. 

Since structure groups of finite involutive solutions are
solvable, the following corollary follows: 

\begin{corollary}
\label{cor:LO<=>LI}
	Let $(X,r)$ be a finite involutive solution. Then $G(X,r)$ is left-ordered if and only if $G(X,r)$ is locally indicable. 
\end{corollary}

Our aim is to characterize involutive multipermutation solutions
in terms of the left-orderability of the structure group. Let us start with an example. 

\begin{example}
\label{exa:Navas}
	Let $X=\{1,2,3,4\}$ and $r\colon X\times X\to X\times X$, $r(x,y)=(\sigma_x(y),\tau_y(x))$, 
	where
	\begin{align*}
	\sigma_1&=(12), &\sigma_2&=(1324), &\sigma_3&=(34), &\sigma_4&=(1423),\\
	\tau_1&=(14),&\tau_2&=(1243),&\tau_3&=(23),&\tau_4&=(1342).	
	\end{align*}
	The group 
	\[
	G(X,r)=\langle a,b,c,d:a^2=bd,ac=ca,ad=dc,ba=cb,b^2=d^2,c^2=db\rangle
	\]
	is not left-orderable. Let $\deg\colon G(X,r)\to\Z$ be the degree of $G(X,r)$ and	
	\[
	K=\ker(\deg)=\langle a^{-1}b,a^{-1}c,a^{-1}d,b^{-1}c,b^{-1}d,c^{-1}d\rangle.
	\]
	Write $p=a^{-1}b$, $q=a^{-1}c$, $r=a^{-1}d$, $s=b^{-1}c$, $t=b^{-1}d$ and $u=c^{-1}d$. Then 
	\[
	ps=q,\quad
	qu=r,\quad
	su=t,\quad
	pr=q^{-1},\quad
	qs=u,\quad
	st=p.
	\]
	To prove that $G(x,r)$ is not left-orderable it is enough to prove that
	$G(X,r)$ is not locally indicable. Suppose on the contrary that
	there exists a group homomorphism $\phi\colon K\to\R$. Since 
	\[
	\det\begin{pmatrix}
		1 & -1 & 0 & 1 & 0 & 0\\
		0 & 1 & -1 & 0 & 0 & 1\\
		0 & 0 & 0 & 1 & -1 & 1\\
		1 & 1 & 1 & 0 & 0 & 0\\
		0 & 1 & 0 & 1 & 0 & 1\\
		1 & 0 & 0 & -1 & -1 & 0	
	\end{pmatrix}
		=16\ne0,
	\]
 	the linear system 
	\begin{align*}
		0 &= \phi(p)-\phi(q)+\phi(s),\\
		0 &= \phi(q)-\phi(r)+\phi(u),\\
		0 &= \phi(s)-\phi(t)+\phi(u),\\
		0 &= \phi(p)+\phi(q)+\phi(r),\\
		0 &= \phi(q)-\phi(s)-\phi(u),\\
		0 &= \phi(p)-\phi(s)-\phi(t),
	\end{align*}
	has only the trivial solution. Thus 
	$\phi$ is the zero homomorphism and hence  
	$G(X,r)$ 
	is not left-orderable by Corollary~\ref{cor:LO<=>LI}. 
\end{example}

\begin{theorem}
\label{thm:MP}
	Let $(X,r)$ be a finite involutive solution. The following statements are equivalent:
	\begin{enumerate}
		\item $(X,r)$ is multipermutation. 
		\item $G(X,r)$ is poly-$\Z$.
		\item $G(X,r)$ is left-orderable
		\item $G(X,r)$ is diffuse.
	\end{enumerate}	
\end{theorem}

\begin{proof}
Since $(X,r)$ is not retractable, there exist $x,y\in X$ such that $x\ne y$ and 
$\sigma_x=\sigma_y$. Let $K=\langle x,y\rangle\cap\ker\deg$. The sequence
\[
	\begin{tikzcd}
	1 & K & G(X,r) & \Z & 1
	\arrow[from=1-1, to=1-2]
	\arrow[from=1-2, to=1-3]
	\arrow["\deg", from=1-3, to=1-4]
	\arrow[from=1-4, to=1-5]
\end{tikzcd}
\]
is exact. Since $K$ is isomorphic to a subgroup of $\Z^{X}$, $K$ is left-ordetable. Thus $G(X,r)$ 
is left-orderable by Proposition~\ref{pro:4-19}.
\end{proof} 

\section*{Exercises}




\section*{Open problems}

\begin{problem}[?]
	Is there a non-diffuse group with the unique product property? 	
\end{problem}

\section*{Notes}

\index{Navas, A.}
Example~\ref{exa:Navas} goes back to Navas. 

\index{Jespers, E.}
\index{Okni\'nski, J.}
\index{Chouraqui, F.}
\index{Bachiller, D.}
\index{Ced\'o, F.}
\index{Lebed, V.}
\index{Vendramin, L.}
Theorem~\ref{thm:MP} is the joint work of different people. 
The implication $1)\implies 2)$ was first proved by Jespers and Okni\'nski in~\cite{}, see also
~\cite{MR2301033}. Chouraqui independently proved
that $1)\implies 3)$ in ~\cite{MR3572046}. In~\cite{MR3815290} Bachiller, Ced\'o and Vendramin proved that $3)\implies 1)$. The equivalence between
$1)$ and $4)$ was proved by Lebed and Vendramin in~\cite{MR3974961}.


