\chapter{Simple braces}
\label{Sbraces}

\section*{A}

As we have seen in Chapter \ref{Class}, the classification of finite involutive solutions is reduced to the classification of the finite  braces of abelian type. A first step in the classification of the finite braces of abelian type is the classification of the finite simple braces of abelian type.

\begin{definition}
\index{Brace!simple}
A non-zero brace $B$ is \emph{simple} if $\{0\}$ and $B$ are the only ideals of $B$.
\end{definition}


\begin{example}
\index{Brace!simple trivial}
Let $G$ be a simple additive group. Then the trivial brace $(G,+, \circ )$, is a simple brace. These are the simple trivial braces.
\end{example}

Thus the classification of the finite simple trivial braces is equivalent to the classification of the finite simple groups.
In particular, the finite simple trivial braces of abelian type are the trivial braces of prime cardinality. 

By Theorem \ref{thm:add_nilpotent}, the multiplicative group of a finite brace of nilpotent type is solvable. Let $B$ be a non-zero finite brace of nilpotent type. Let $p_1,\dots ,p_k$ be the distinct prime divisors of $|B|$. Let $B_{p_i}$ be the Sylow $p_i$-subgroup of the additive group of $B$, for $i=1,\dots ,k$. Note that $B_{p_i}$ is a left ideal of $B$, and the multiplicative groups of these left ideals $B_{p_i}$ form a Sylow system of the multiplicative group of $B$. Thus
$$B=\prod_{i=1}^kB_{p_i}.$$
Note that for $a\in B_{p_i}$, $b\in B_{p_j}$, where $i\neq j$, there exist unique $c\in B_{p_j}$ and $d\in B_{p_i}$ such that $a\circ b=c\circ d$. On the other hand, 
$$a\circ b=\lambda_a(b)\circ \lambda_a(b)'\circ a\circ b=\lambda_a(b)\circ (b'+\lambda_{b'}(a')-b')'.$$
Hence $c=\lambda_a(b)$ and $d=(b'+\lambda_{b'}(a')-b')'$.
    


\begin{lemma}
    \label{lem:nilpotentsimple}
    Let $B$ be a finite simple brace of nilpotent type with nilpotent multiplicative group. Then $B$ is a trivial brace of prime order. 
\end{lemma}

\begin{proof}
    Let $p$ be a prime divisor of $|B|$. Let $B_p$ be the Sylow $p$-subgroup of the additive group of $B$. We know that $B_p$ is a left ideal of $B$. In particular, $B_p$ also is a Sylow $p$-subgroup of the multiplicative group of $B$. Since the multiplicative group of $B$ is nilpotent, we have that $B_p$ is an ideal of $B$. Since $B$ is simple, $B=B_p$. Let $G$ be the multiplicative group of $B$ and let $A$ be the additive group of $B$. Then the semidirect product $H=A\rtimes G$ via the lambda map is a $p$-group, and thus it is nilpotent. Let $a\in A$ and $g\in G$. We have that 
    $$(a,0)(0,g)(-a,0)(0,g')=(a,g)(-a,g')=(a-\lambda_g(a),0)=(-(g*a),0).$$
    Therefore $[\{0\}\times G,A\times\{0\}]=B*B\times \{0\}$. Since $H$ is nilpotent, it is clear that $B*B\neq B$. Since $B*B$ is an ideal of $B$ and $B$ is simple, it follows that $B*B=\{0\}$. Hence $B$ is a trivial brace. Since $B$ is simple, $|B|=p$.   
\end{proof}

The above result shows that if $B$ is a finite non-trivial simple brace of abelian type, 
then there exist two distinct prime divisors of $|B|$. To understand the structure of $B$ we need to study the structure of the left ideals $B_p$, where $B_p$ is the Sypow $p$-subgroup of $(B,+)$, and the restriction of the lambda map to $B_p$ acting on $B_q$, where $p,q$ are distinct prime divisors of $|B|$.


Let $B$ be a finite simple brace of abelian type of order $p^nq^m$ for distinct primes $p,q$ and positive integers $n,m$. Let $B_p$ and $B_q$ be the Sylow $p$-subgroup and the Sylow $q$-subgroup of the additive group of $B$ respectively. Let $\alpha\colon B_q\rightarrow \Aut(B_p,+)$ and $\beta\colon B_p\rightarrow \Aut(B_q,+)$ be restictions of the lambda map of $B$. Let $a\in B_p$ and $b\in B_q$ and suppose that $(a,b)\neq (0,0)$. Since $B$ is simple, we have that the ideal generated by $a+b$ is $B$. In particular,  if $a\neq 0$, then the ideal generated by $a$ is $B$. Let $I_0$ be a minimal nonzero ideal of $B_p$. Clearly $I_0$ is not an ideal of $B$. This means that there exist $d\in B_q$ and $c\in I_0$ such that $c*d=(\lambda_c-\id)(d)=(\beta_c-id)(d)\neq 0$. Let 
$J_0$ be the ideal of $B_q$ generated by $\{ (\beta_c-id)(d)\mid c\in I_0, \; d\in B_q\}$. Let $I_1$ be the ideal of $B_p$ generated by 
$$I_0\cup \{(\alpha_d-id)(c)\mid d\in J_0,\; c\in B_p\}.$$ 
Let $J_1$ be the ideal of $B_q$ generated by 
$$J_0\cup \{(\beta_c-id)(d)\mid c\in I_1,\; d\in B_q\}.$$
For $k>1$ let  $I_k$ be the ideal of $B_p$ generated by 
$$I_{k-1}\cup \{(\alpha_d-id)(c)\mid d\in J_{k-1},\; c\in B_p\},$$ and let $J_K$ be the ideal of $B_q$ generated by 
$$J_{k-1}\cup \{(\beta_c-id)(d)\mid c\in I_k,\; d\in B_q\}.$$
Then there exists a positive integer $k$ such that $I_k=B_p$ and $J_k=B_q$. 

We shall study the main constructions of the finite simple braces 
of abelian type of order $p^nq^m$ for distinct primes $p,q$ and positive integers $n,m$.
  
\section*{Matched product of braces} 

Let $B$ be a left brace. Suppose that $(B,+)$ is the direct product of two left ideals $B_1$ and $B_2$ of $B$.
Note that for $a,x\in B_1$ and $b,y\in B_2$, we have 
\begin{eqnarray*}
	\lambda_{a+b}(x+y)&=&\lambda_{a+b}(x)+\lambda_{a+b}(y)\\
	&=&\lambda_{a}\lambda_{\lambda^{-1}_{a}(b)}(x)+\lambda_{a}\lambda_{\lambda^{-1}_{a}(b)}(y)\\
	&=&\lambda_{b}\lambda_{\lambda^{-1}_{b}(a)}(x)+\lambda_{b}\lambda_{\lambda^{-1}_{b}(a)}(y).
\end{eqnarray*}	
Let $\alpha\colon (B_2,\circ)\rightarrow\Aut(B_1,+)$ and $\beta\colon (B_1,\circ)\rightarrow\Aut(B_2,+)$ be the restrictions of the lambda map, i.e. 
$$\alpha_b(x)=\lambda_b(x)\mbox{ and }\beta_a(y)=\lambda_a(y),$$
where $\alpha_b=\alpha(b)$ and $\beta_a=\beta(a)$, for all $a,x\in B_1$ and $b,y\in B_2$. Then
$$\alpha_b\lambda^{(1)}_{\alpha^{-1}_b(a)}=\lambda^{(1)}_a\alpha_{\beta^{-1}_a(b)}$$
and	
$$\beta_a\lambda^{(2)}_{\beta^{-1}_a(b)}=\lambda^{(2)}_b\beta_{\alpha^{-1}_b(a)},$$
for all $a\in B_1$ and $b\in B_2$, where $\lambda^{(1)}$ is the lambda map of $B_1$ and $\lambda^{(2)}$ is the lambda map of $B_2$.

\begin{definition}\label{def:matchedpair}
	\index{matched pair}
	Let $B_1$ and $B_2$ be two left braces. Let $\alpha\colon (B_2,\circ)\rightarrow \Aut(B_1,+)$ and $\beta\colon (B_1,\circ)\rightarrow \Aut(B_2,+)$ be group homomorphisms. We say that $(B_1,B_2,\alpha,\beta)$ is a \emph{matched pair} of left braces if
	$$\alpha_b\lambda^{(1)}_{\alpha^{-1}_b(a)}=\lambda^{(1)}_a\alpha_{\beta^{-1}_a(b)}$$
	and	
	$$\beta_a\lambda^{(2)}_{\beta^{-1}_a(b)}=\lambda^{(2)}_b\beta_{\alpha^{-1}_b(a)},$$
	where $\alpha_b=\alpha(b)$ and $\beta_a=\beta(a)$, for all $a\in B_1$ and $b\in B_2$, and where $\lambda^{(1)}$ is the lambda map of $B_1$ and $\lambda^{(2)}$ is the lambda map of $B_2$.	
\end{definition}

\begin{lemma} \label{lemma:lambda}
	Let $(B,+)$ be a group (not necessarily abelian). Let
	$$\lambda\colon B\rightarrow \Aut(B.+),\quad a\mapsto \lambda_a,$$ 
	be a map such that
	$\lambda_a\circ\lambda_b=\lambda_{a+\lambda_a(b)}$, for all $a,b\in B$. Then $A$ with the sum $+$ and the product defined by $a\circ b:=a+\lambda_a(b)$, for all $a,b\in B$, is a left brace.	
\end{lemma}	

\begin{proof}
	Note that, if $0$ is the neutral element of $(B,+)$, then $\lambda_0\circ\lambda_0=\lambda_{\lambda_0(0)}=\lambda_0$. Hence
	$\lambda_0=\id$. Since $a\circ 0=a+\lambda_a(0)=a+0=a$ and $0\circ a=0+\lambda_0(a)=a$, we have that $0$ is the neutral element of $(B,\circ)$. Since $\lambda_a\circ\lambda_{\lambda^{-1}_{a}(-a)}=\lambda_{a-a}=\lambda_0=\id$, we get that $\lambda^{-1}_a=\lambda_{\lambda^{-1}_a(-a)}$. Now $a\circ \lambda^{-1}_{a}(-a)=a-a=0$ and $\lambda^{-1}_a(-a)\circ a=\lambda^{-1}_a(-a)+\lambda_{\lambda^{-1}_a(-a)}(a)=\lambda^{-1}_a(-a)+\lambda^{-1}_{a}(a)=\lambda^{-1}_a(-a+a)=\lambda^{-1}_a(0)=0$. Hence $\lambda^{-1}_a(-a)$ is the simmetric element of $a$ in $(B,\circ)$. We shall check the associativity property:
	\begin{eqnarray*}
		a\circ (b\circ c)&=&a+\lambda_a(b+\lambda_b(c))=a+\lambda_a(b)+\lambda_a(\lambda_b(c))\\
		&=&a+\lambda_a(b)+\lambda_{a+\lambda_a(b)}(c)=(a\circ b)+\lambda_{a\circ b}(c)\\
		&=&(a\circ b)\circ c.
	\end{eqnarray*}
Hence $(B,\circ)$ is a group. Note that
\begin{eqnarray*}
	a\circ (b+c)&=&a+\lambda_a(b+c)=a+\lambda_a(b)+\lambda_a(c)\\
	&=&a+\lambda_a(b)-a+a+\lambda_a(c)=a\circ b-a+a\circ c.
\end{eqnarray*}
Therefore $(B,+,\circ)$ is a left brace, and the result follows.
\end{proof}

\begin{theorem}\label{thm:matchedproduct}
	Let $(B_1,B_2,\alpha,\beta)$ be a matched pair of left braces. Then $B_1\times B_2$ is a left brace with sum given by
	$$(a,b)+(a',b')=(a+a'.b+b')$$
and lambda map given by
$$\lambda_{(a,b)}(a',b'):=\left( \lambda^{(1)}_{a}\alpha_{\beta^{-1}_a(b)}(a'), \lambda^{(2)}_{b}\beta_{\alpha^{-1}_b(a)}(b')\right).$$
Moreover $B_1\times \{0\}$ and $\{0\}\times B_2$ are left ideals isomorphic to $B_1$ and $B_2$ as left braces respectively.
This left brace is \index{matched product} the \emph{matched product} of $B_1$ and $B_2$, and it is denoted by $B_1\bowtie B_2$.

Conversely, if $B$ is a left brace such that $(B,+)$ is the direct product of two left ideals $L_1$ and $L_2$ of $B$, then $(L_1,L_2,\alpha,\beta)$ is a matched pair of left braces, where $\alpha$ and $\beta$ are the restrictions of the lambda map of $B$, and $B$  is isomorphic to the corresponding matched product $L_1\bowtie L_2$.   	
\end{theorem}	

\begin{proof}
	To prove the first part, by Lemma \ref{lemma:lambda}, it is enough to check that
	$$\lambda_{(a,b)}\lambda_{(a',b')}=\lambda_{(a,b)+\lambda_{(a,b)}(a',b')},$$
	for all $a,a'\in B_1$ and $b,b'\in B_2$. Let $x\in B_1$ and $y\in B_2$. We have that
	\begin{eqnarray*}
		\lambda_{(a,b)}\lambda_{(a',b')}(x,y)&=&\lambda_{(a,b)}\left( \lambda^{(1)}_{a'}\alpha_{\beta^{-1}_{a'}(b')}(x), \lambda^{(2)}_{b'}\beta_{\alpha^{-1}_{b'}(a')}(y)\right)\\
		&=&\left( \lambda^{(1)}_{a}\alpha_{\beta^{-1}_{a}(b)}\lambda^{(1)}_{a'}\alpha_{\beta^{-1}_{a'}(b')}(x), \lambda^{(2)}_{b}\beta_{\alpha^{-1}_{b}(a)}\lambda^{(2)}_{b'}\beta_{\alpha^{-1}_{b'}(a')}(y)\right)
	\end{eqnarray*}	
and
	\begin{eqnarray*}
	\lambda_{(a,b)+\lambda_{(a,b)}(a',b')}(x,y)&=&\lambda_{(a+\lambda^{(1)}_{a}\alpha_{\beta^{-1}_{a}(b)}(a'),b+\lambda^{(2)}_{b}\beta_{\alpha^{-1}_{b}(a)}(b'))}(x,y)\\
	&=&\left( \lambda^{(1)}_{a+\lambda^{(1)}_{a}\alpha_{\beta^{-1}_{a}(b)}(a')}\alpha_{\beta^{-1}_{a+\lambda^{(1)}_{a}\alpha_{\beta^{-1}_{a}(b)}(a')}(b+\lambda^{(2)}_{b}\beta_{\alpha^{-1}_{b}(a)}(b'))}(x),\right.\\
	&&\quad\left. \lambda^{(2)}_{b+\lambda^{(2)}_{b}\beta_{\alpha^{-1}_{b}(a)}(b')}\beta_{\alpha^{-1}_{b+\lambda^{(2)}_{b}\beta_{\alpha^{-1}_{b}(a)}(b')}(a+\lambda^{(1)}_{a}\alpha_{\beta^{-1}_{a}(b)}(a'))}(y)\right).
\end{eqnarray*}	
Thus, it is enough to check the following equalities:
\begin{equation}\label{eq:matched1}
	\lambda^{(1)}_{a}\alpha_{\beta^{-1}_{a}(b)}\lambda^{(1)}_{a'}\alpha_{\beta^{-1}_{a'}(b')}=\lambda^{(1)}_{a+\lambda^{(1)}_{a}\alpha_{\beta^{-1}_{a}(b)}(a')}\alpha_{\beta^{-1}_{a+\lambda^{(1)}_{a}\alpha_{\beta^{-1}_{a}(b)}(a')}(b+\lambda^{(2)}_{b}\beta_{\alpha^{-1}_{b}(a)}(b'))}
\end{equation}
and
\begin{equation}\label{eq:matched2}
	\lambda^{(2)}_{b}\alpha_{\alpha^{-1}_{b}(a)}\lambda^{(2)}_{b'}\beta_{\alpha^{-1}_{b'}(a')}=\lambda^{(2)}_{b+\lambda^{(2)}_{b}\beta_{\alpha^{-1}_{b}(a)}(b')}\beta_{\alpha^{-1}_{b+\lambda^{(2)}_{b}\beta_{\alpha^{-1}_{b}(a)}(b')}(a+\lambda^{(1)}_{a}\alpha_{\beta^{-1}_{a}(b)}(a'))}.
\end{equation}
We shall prove (\ref{eq:matched1}). The equality (\ref{eq:matched2}) is proved similarly. 
We have
\begin{eqnarray*}
	\lefteqn{\lambda^{(1)}_{a+\lambda^{(1)}_{a}\alpha_{\beta^{-1}_{a}(b)}(a')}\alpha_{\beta^{-1}_{a+\lambda^{(1)}_{a}\alpha_{\beta^{-1}_{a}(b)}(a')}(b+\lambda^{(2)}_{b}\beta_{\alpha^{-1}_{b}(a)}(b'))}}\\
	&=& \lambda^{(1)}_{a\circ \alpha_{\beta^{-1}_{a}(b)}(a')}\alpha_{\beta^{-1}_{a\circ \alpha_{\beta^{-1}_{a}(b)}(a')}(b+\lambda^{(2)}_{b}\beta_{\alpha^{-1}_{b}(a)}(b'))}\\
	&=& \lambda^{(1)}_{a\circ \alpha_{\beta^{-1}_{a}(b)}(a')}\alpha_{\beta^{-1}_{a\circ \alpha_{\beta^{-1}_{a}(b)}(a')}(b)+\beta^{-1}_{a\circ \alpha_{\beta^{-1}_{a}(b)}(a')}(\lambda^{(2)}_{b}\beta_{\alpha^{-1}_{b}(a)}(b'))}\\
	&=& \lambda^{(1)}_{a\circ \alpha_{\beta^{-1}_{a}(b)}(a')}\alpha_{\beta^{-1}_{a\circ \alpha_{\beta^{-1}_{a}(b)}(a')}(b)+\lambda^{(2)}_{\beta^{-1}_{a\circ \alpha_{\beta^{-1}_{a}(b)}(a')}}\beta^{-1}_{\alpha^{-1}_b(a\circ \alpha_{\beta^{-1}_a(b)}(a'))}\beta_{\alpha^{-1}_{b}(a)}(b')}\\
	&=& \lambda^{(1)}_{a\circ \alpha_{\beta^{-1}_{a}(b)}(a')}\alpha_{\beta^{-1}_{a\circ \alpha_{\beta^{-1}_{a}(b)}(a')}(b)\circ \beta^{-1}_{\alpha^{-1}_b(a\circ \alpha_{\beta^{-1}_a(b)}(a'))}\beta_{\alpha^{-1}_{b}(a)}(b')}\\
	&=& \lambda^{(1)}_{a\circ \alpha_{\beta^{-1}_{a}(b)}(a')}\alpha_{\beta^{-1}_{a\circ \alpha_{\beta^{-1}_{a}(b)}(a')}(b)\circ \beta^{-1}_{\alpha^{-1}_b(a+\lambda^{(1)}_a\alpha_{\beta^{-1}_a(b)}(a'))}\beta_{\alpha^{-1}_{b}(a)}(b')}\\
	&=& \lambda^{(1)}_{a\circ \alpha_{\beta^{-1}_{a}(b)}(a')}\alpha_{\beta^{-1}_{a\circ \alpha_{\beta^{-1}_{a}(b)}(a')}(b)\circ \beta^{-1}_{\alpha^{-1}_b(a+\alpha_b\lambda^{(1)}_{\alpha^{-1}_b(a)}(a'))}\beta_{\alpha^{-1}_{b}(a)}(b')}\\
	&=& \lambda^{(1)}_{a\circ \alpha_{\beta^{-1}_{a}(b)}(a')}\alpha_{\beta^{-1}_{a\circ \alpha_{\beta^{-1}_{a}(b)}(a')}(b)\circ \beta^{-1}_{\alpha^{-1}_b(a)+\lambda^{(1)}_{\alpha^{-1}_b(a)}(a')}\beta_{\alpha^{-1}_{b}(a)}(b')}\\
	&=& \lambda^{(1)}_{a\circ \alpha_{\beta^{-1}_{a}(b)}(a')}\alpha_{\beta^{-1}_{a\circ \alpha_{\beta^{-1}_{a}(b)}(a')}(b)\circ \beta^{-1}_{\alpha^{-1}_b(a)\circ a'}\beta_{\alpha^{-1}_{b}(a)}(b')}\\
	&=& \lambda^{(1)}_{a\circ \alpha_{\beta^{-1}_{a}(b)}(a')}\alpha_{\beta^{-1}_{a\circ \alpha_{\beta^{-1}_{a}(b)}(a')}(b)\circ \beta^{-1}_{a'}(b')}\\
	&=& \lambda^{(1)}_{a}\lambda^{(1)}_{\alpha_{\beta^{-1}_{a}(b)}(a')}\alpha_{\beta^{-1}_{\alpha_{\beta^{-1}_{a}(b)}(a')}\beta^{-1}_a(b)}\alpha_{\beta^{-1}_{a'}(b')}\\
	&=& \lambda^{(1)}_{a}\alpha_{\beta^{-1}_a(b)}\lambda^{(1)}_{a'}\alpha_{\beta^{-1}_{a'}(b')}.
\end{eqnarray*}
This finishes the proof of the first part.

The second part follows easily by the argument above Definition \ref{def:matchedpair}.
\end{proof}
The matched product of two braces can be generalized to the mastched product of more than two braces following the case of braces of abelian type (see \cite{MR3812099} for the matched product of several braces of abelian type). 

Let $B$ be a finite left brace of nilpoent type. Let $B_1,\dots ,B_n$ be the Sylow subgroups of $(B,+)$. Then $(B.+)$ is the direct product of these Sylow subgroups and, moreover, $B_1,\dots ,B_n$ are left ideals of $B$. Thus $B$ is the matched product of $B_1,\dots ,B_n$. In particular, every simple non-trivial left braces of abelian type is the matched product of two or more left braces of abelian type of order a power of a prime. But, in fact all the known simple non-trivial left braces of abelian type are constructed using another kind of product, the so called asymmetric product.

\section*{Asymmetric product of braces of abelian type}

Let $S$ and $T$ be two (additive) abelian groups. Recall that a (normalized)
\emph{symmetric $2$-cocyle} \index{symmetric $2$-cocyle} on $T$ with values in $S$ is a map $b\colon
T\times T\longrightarrow S$ such that
\begin{enumerate}
	\item[(i)]\label{cond0} $b(0,0)=0$;
	\item[(ii)]\label{cond1} $b(t_1,t_2)=b(t_2,t_1)$;
	\item[(iii)]\label{cond2} $b(t_1+t_2,t_3)+b(t_1,t_2)=b(t_1,t_2+t_3)+b(t_2,t_3)$, for all $t_1,t_2,t_3\in T$.
\end{enumerate}
As a consequence, we get that $b(t,0)=b(0,t)=0$, for all $t\in T$.

\begin{theorem}\label{ccs}
	Let $T$ and $S$ be two left braces of abelian type. Let $b \colon T\times T
	\longrightarrow S$ be a symmetric $2$-cocycle on $(T, +)$ with
	values in $(S, +)$, and let $\alpha \colon (S,\circ)\longrightarrow
	\Aut(T,+,\circ)$ be a homomorphism of groups such that
	\begin{eqnarray}\label{cond3} && s\cdot b(t_2, t_3) + b(t_1\circ \alpha_s(t_2 + t_3),t_1)=
		b(t_1\circ\alpha_s(t_2), t_1\circ\alpha_s(t_3))+ s,\end{eqnarray}
	where $\alpha_s=\alpha(s)$, for all $s\in S$ and $t_1, t_2, t_3\in T$. Then the addition and
	multiplication over $T\times S$ given by
	$$
	(t_1,s_1)+(t_2,s_2)=(t_1+t_2,~s_1+s_2+b(t_1,t_2)),
	$$
	$$
	(t_1,s_1)\circ (t_2,s_2)=(t_1\circ \alpha_{s_1}(t_2),~s_1\circ s_2),
	$$
	define a structure of left brace on $T\times S$. We call this left
	brace the \emph{asymmetric product} \index{asymmetric product} of $T$ by $S$ (via $b$ and $\alpha$)
	and denote it by  $T\rtimes_\circ S$.
\end{theorem}

\begin{proof}
	It is easy to check that $(T\times S,+)$ is an abelian group and that $(T\times S,\circ)$ is a group. In fact, 
	$(T\times S,+)$ is the extension of the abelian groups $S$ and $T$ via the $2$-cocycle $b$, and $(T\times S, \circ)$ is the semidirect product $T\rtimes_{\alpha}S$ of $T$ by $S$ via $\alpha$. Thus it is enough to check that
	$$(t_1,s_1)\circ ((t_2,s_2)+(t_3,s_3))=(t_1,s_1)\circ (t_2,s_2)-(t_1,s_1)+(t_1,s_1)\circ (t_3,s_3).$$
	We have that
	\begin{eqnarray*}
		\lefteqn{(t_1,s_1)\circ (t_2,s_2)-(t_1,s_1)+(t_1,s_1)\circ (t_3,s_3)}\\
		&=&(t_1\circ\alpha_{s_1}(t_2),s_1\circ s_2)-(t_1,s_1)+(t_1\circ\alpha_{s_1}(t_3),s_1\circ s_3)\\
		&=&(t_1\circ\alpha_{s_1}(t_2)-t_1,s_1\circ s_2-s_1-b(t_1\circ\alpha_{s_1}(t_2)-t_1,t_1))+(t_1\circ\alpha_{s_1}(t_3),s_1\circ s_3)\\
		&=&(t_1\circ\alpha_{s_1}(t_2)-t_1+t_1\circ\alpha_{s_1}(t_3),s_1\circ s_2-s_1-b(t_1\circ\alpha_{s_1}(t_2)-t_1,t_1)+s_1\circ s_3\\
		&&\qquad+b(t_1\circ\alpha_{s_1}(t_2)-t_1,t_1\circ\alpha_{s_1}(t_3)))\\
        &=&(t_1\circ(\alpha_{s_1}(t_2)+\alpha_{s_1}(t_3)),s_1\circ (s_2+s_3)-b(t_1\circ\alpha_{s_1}(t_2)-t_1,t_1)\\
        &&\qquad+b(t_1\circ\alpha_{s_1}(t_2)-t_1,t_1\circ\alpha_{s_1}(t_3)))\\
        &=&(t_1\circ(\alpha_{s_1}(t_2+t_3)),s_1\circ (s_2+s_3)+b(t_1\circ\alpha_{s_1}(t_2),t_1\circ\alpha_{s_1}(t_3))\\
        &&\qquad-b(t_1\circ\alpha_{s_1}(t_2)-t_1+t_1\circ\alpha_{s_1}(t_3),t_1))\\
        &=&(t_1\circ(\alpha_{s_1}(t_2+t_3)),s_1\circ (s_2+s_3)+b(t_1\circ\alpha_{s_1}(t_2),t_1\circ\alpha_{s_1}(t_3))\\
        &&\qquad-b(t_1\circ(\alpha_{s_1}(t_2)+\alpha_{s_1}(t_3)),t_1))\\
        &=&(t_1\circ(\alpha_{s_1}(t_2+t_3)),s_1\circ (s_2+s_3)+b(t_1\circ\alpha_{s_1}(t_2),t_1\circ\alpha_{s_1}(t_3))\\
        &&\qquad-b(t_1\circ\alpha_{s_1}(t_2+t_3)),t_1))\\
        &=&(t_1\circ(\alpha_{s_1}(t_2+t_3)),s_1\circ (s_2+s_3)-s_1+s_1\circ b(t_2,t_3))\\
        &=&(t_1\circ(\alpha_{s_1}(t_2+t_3)),s_1\circ (s_2+s_3+b(t_2,t_3))\\
        &=&(t_1,s_1)\circ (t_2+t_3,s_2+s_3+b(t_2,t_3))\\
        &=&(t_1,s_1)\circ ((t_2,s_2)+(t_3,s_3)).
	\end{eqnarray*}	
Therefore the result follows.
\end{proof}

Note that the lambda map of $T\rtimes_\circ S$ is defined by
\begin{eqnarray} \label{deflambda}
	\lambda_{(t_1,s_1)}(t_2,s_2)&=&
	\left(
	\lambda_{t_1}\alpha_{s_1}(t_2),~
	\lambda_{s_1}(s_2)-b(\lambda_{t_1}\alpha_{s_1}(t_2),t_1)
	\right),
\end{eqnarray}
and its socle is
$$
\Soc(T\rtimes_\circ S)=\{(t,s)\mid \lambda_s=\id_S,\;\lambda_t\circ
\alpha_s=\id_T,\; b(t,t')=0\mbox{ for all }t'\in T\}.
$$
Moreover, the subset $T\times\{0\}$ is a normal subgroup of $(T\rtimes_\circ S,\circ)$, and $\{0\}\times S$ is a left
ideal of $T\rtimes_\circ S$.

A particular case of this theorem is when we assume that $b$ is a symmetric bilinear form. In this case, conditions
(\ref{cond0})-(\ref{cond2}) are automatic, and
condition (\ref{cond3}) becomes
$$
\lambda_s(b(t_2,~t_3))=b(\lambda_{t_1}\alpha_{s}(t_2),~\lambda_{t_1}\alpha_{s}(t_3)),
$$
which is equivalent to the two conditions:
\begin{align}
	\lambda_s(b(t_2,~t_3))&=b(\alpha_{s}(t_2),~\alpha_{s}(t_3)),\label{condAlpha} \\
	b(t_2,~t_3)&=b(\lambda_{t_1}(t_2),~\lambda_{t_1}(t_3)),\label{condLambda}
\end{align}
for all $s\in S$ and $t_1,t_2,t_3\in T$.
Furthermore, if $T$ and $S$ are trivial left braces of abelian type, then these two conditions (\ref{condAlpha}) and (\ref{condLambda}) reduce to 
$$b(t_2,~t_3)=b(\alpha_{s}(t_2),~\alpha_{s}(t_3)),$$
for all $s\in S$ and $t_1,t_2,t_3\in T$, i.e. $\alpha_s$ is in the orthogonal group of the bilinear form $b$, for all $s\in S$.
   
\section*{Construction of simple braces of abelian type}

Let $p,q$ be two distinct primes. Consider the trivial left braces over the abelian groups $T=(\Z/(p))^{n_1}\times (\Z/(q))^{n_2}$ and $S=\Z/(p)\times \Z/(q)$. Let $b_1\colon (\Z/(p))^{n_1}\times (\Z/(p))^{n_1}\rightarrow \Z/(p)$ and $b_2\colon (\Z/(q))^{n_2}\times (\Z/(q))^{n_2}\rightarrow \Z/(q)$ be two non-degenerate symmetric bilinear forms. Let $c_1$ be an element of order $q$ in the orthogonal group of $b_1$ and let $c_2$ be an element of order $p$ in the orthogonal group of $b_2$. Let $\alpha\colon S\rightarrow \Aut(T)$ be the map defined by $\alpha(i,j)=\alpha_{(i,j)}$ and $\alpha_{(i,j)}(u,v)=(c_1^{j}(u),c_2^{i}(v))$, for all $(i,j)\in S$ and $(u,v)\in T$. Let $b\colon T\times T\rightarrow S$ be the map defined by $b((u_1,v_1),(u_2,v_2))=(b_1(u_1,u_2),b_2(v_1,v_2))$, for all $(u_1,v_1),(u_2,v_2)\in T$. Then
\begin{eqnarray*}b(\alpha_{(i,j)}(u_1,v_1),\alpha_{(i,j)}(u_2,v_2))&=&(b_1(c_1^{j}(u_1),c_1^{j}(u_2)),b_2(c_2^{i}(v_1),c_2^{i}(v_2)))\\
	&=&(b_1(u_1,u_2),b_2(v_1,v_2)).
\end{eqnarray*}
Hence we can consider the asymmetric product $T\rtimes_{\circ}S$ of $T$ by $S$ (via $b$ and $\alpha$). 

\begin{theorem}\label{thm:simplebrace}
	With the above notation, the left brace $T\rtimes_{\circ}S$ is simple if and only if $c_1-\id$ and $c_2-\id$ are invertible. 
\end{theorem}  

\begin{proof}
	Let $J=\{((u,v),(i,j))\mid u\in\im(c_1-\id),\; v\in \im(c_2-\id), \; (i,j)\in S\}$. We shall prove that $J$ is an ideal of $T\rtimes_{\circ}S$. Clearly $J$ is a subgroup of $(T\rtimes_{\circ}S, +)$. Let $((u,v),(i,j))\in J$ and $((u',v'),(k,l))\in T\rtimes_{\circ}S$. We have that
	\begin{eqnarray*}
		\lefteqn{\lambda_{((u',v'),(k,l))}((u,v),(i,j))}\\
		&=&-((u',v'),(k,l))+((u',v'),(k,l))\circ ((u,v),(i,j))\\
		&=&((c_1^{l}(u),c_2^{k}(v)),(i-b_1(c_1^{l}(u),u'),j-b_2(c_2^{k}(v),v')))\in J.
	\end{eqnarray*}
Hence $J$ is a left ideal of $T\rtimes_{\circ}S$. Furthermore
\begin{eqnarray*}
	\lefteqn{((u,v),(i,j))*((u',v'),(k,l))}\\
	&=&-((u,v),(i,j))+((u,v),(i,j))\circ ((u',v'),(k,l))-((u',v'),(k,l))\\
	&=&((c_1^{j}(u'),c_2^{i}(v')),(k-b_1(c_1^{j}(u'),u),l-b_2(c_2^{ik}(v'),v)))-((u',v'),(k,l))\\
	&=&((c_1^{j}(u')-u',c_2^{i}(v')-v'),(-b_1(c_1^{j}(u'),u)-b_1(c_1^{j}(u')-u',u'),-b_2(c_2^{ik}(v'),v)\\
	&&\qquad\qquad -b_2(c_2^{i}(v')-v',v')))\in J.
\end{eqnarray*}	
Thus $J$ is an ideal of $T\rtimes_{\circ}S$. Therefore if $T\rtimes_{\circ}S$ is simple, then $c_1-\id$ and $c_2-\id$ are invertible.

Conversely, suppose that $c_1-\id$ and $c_2-\id$ are invertible. Let $I$ be a nonzero ideal of $T\rtimes_{\circ}S$.
Let $((u,v),(i,j))\in I$ be a nonzero element. Suppose first that $(u,v)=(0,0)$. In this case, $(i,j)\neq (0,0)$. We may assume that $i\neq 0$. Note that $q((0,0),(i,j))=((0,0),(qi,0))\in I$. Hence we may assume that $j=0$. Let $i'\in\Z/(p)$ the inverse of $i$. Then
$i'((0,0),(i,0))=((0,0),(1,0))\in I$. Now,  
\begin{eqnarray*}\lefteqn{((0,0),(1,0))*((u',v'),(k,l))}\\
	&=&((0,(c_2-\id)(v')),(b_1(u',u'),-b_2((c_2-\id)(v'),v')))\in I,
\end{eqnarray*}
for all $((u',v'),(k,l))\in T\rtimes_{\circ}S$. Thus for every $w\in (\Z/(q))^{n_2}$ there exists $l_w\in \Z/(q)$ such that
$((0,w),(0,l_w))\in I$. Let $l\in \Z/(q)$. Note that if $w\neq 0$, since $b_2$ is non-degenerate, 
there exists $w'\in (\Z/(q))^{n_2}$ such that $b_2(w,w')=l_w-l$. Hence $\lambda_{((0,w'),(0,0))}((0,w),(0,l))=((0,w),(0,l-b_2(w,w')))=((0,w),(0,l))\in I$, for all $w\in (\Z/(q))^{n_2}\setminus\{0\}$ and $l\in \Z/(q)$. Thus, if $w\neq 0$, then $((0,w),(0,l))-((0,w),(0,0))=((0,0),(0,l))\in I$, for all $l\in\Z/(q)$. In particular, $((0,0),(0,1))\in I$. Now by a similar argument one can see that $((u',0),(k,0))\in I$, for all $u'\in (\Z/(p))^{n_1}$ and $k\in\Z/(p)$. Now
$((u',0),(k,0))+((0,w),(0,l))=((u',w),(k,l))\in I$, for all $u'\in (\Z/(p))^{n_1}$, $w\in (\Z/(q))^{n_2}$, $k\in\Z/(p)$ and $l\in\Z/(q)$.
Thus $I=T\rtimes_{\circ}S$ in this case. 

Suppose now that $(u,v)\neq (0,0)$. In this case we may assume that $u\neq 0$. Then $q((u,v),(i,j))=((qu,0),(i',j'))\in I$, for some $(i',j')\in S$. Furthermore 
$$q((qu,0),(i',j'))=((q^2u,0),(i'',0))\in I,$$ 
for some $i''\in\Z/(p)$. Since $q^2u\neq 0$ and $b_1$ is non-degenerate, there exists $u'\in (\Z/(p))^{n_1}$ such that $b_1(q^2u,u')=i''-k$. Hence
$\lambda_{((u',0),(0,0))}((q^2u,0),(i'',0))=((q^2u,0),(i''-b_1(q^2u,u')))=((q^2u,0),(k,0))\in I$, for all $k\in\Z/(p)$, and then
$((q^2u,0),(1,0))-((q^1u,0),(0,0))=((0,0),(1,0))\in I$. Now, as above, one can prove that $I=T\rtimes_{\circ}S$, and the result follows.
\end{proof}

We shall give now some concrete constructions of simple left braces of abelian type using Theorem \ref{thm:simplebrace}. Let $p,q$ be two distinct primes. Let $n,m$ be two positive integers. Let
$$D_1=\left(\begin{array}{rrrrr}
0&0&\ldots &0&-1\\
\vspace{-5pt}1&0&\ldots &0&-1\\
\vspace{-5pt}0&\ddots &\ddots &\vdots&\vdots\\
\vdots &\ddots &\ddots &0&-1\\
0&\ldots&0&1&-1\end{array}\right),\; E_1=\left(\begin{array}{cccc}
\vspace{-5pt}1-q&1&\ldots &1\\
\vspace{-5pt}1&\ddots &\ddots&\vdots\\
\vdots &\ddots &\ddots &1\\
1&\ldots&1&1-q\end{array}\right)\in M_{q-1}(\Z/(p))$$
 and  
 $$D_2=\left(\begin{array}{rrrrr}
0&0&\ldots&0&-1\\
\vspace{-5pt}1&0&\ldots&0&-1\\
\vspace{-5pt}0&\ddots&\ddots&\vdots&\vdots\\
\vdots&\ddots&\ddots&0&-1\\
0&\ldots&0&1&-1\end{array}\right),\; E_2=\left(\begin{array}{cccc}
\vspace{-5pt}1-p&1&\ldots &1\\
\vspace{-5pt}1&\ddots &\ddots&\vdots\\
\vdots &\ddots &\ddots &1\\
1&\ldots&1&1-p\end{array}\right)\in M_{p-1}(\Z/(q)).$$ 
Note that $D_1$ is the companion matrix of the polynomial $1+x+\dots +x^{q-1}$ and thus it has order $q$. Similarly $D_2$ has order $p$. 
It is easy to check that $D_1^tE_1D_1=E_1$ and $D_2^tE_2D_2=E_2$. Consider in $M_{n_1(q-1)}(\Z/(p))$ the block diagonal matrices with $n_1$ blocks of degree $q-1$:
$$C_1=\left(\begin{array}{cccc}
	D_1&0&\ldots &0\\
	0&\ddots &\ddots&\vdots\\
	\vdots &\ddots &\ddots &0\\
	0&\ldots&0&D_1\end{array}\right)\mbox{ and } B_1=\left(\begin{array}{cccc}
	E_1&0&\ldots &0\\
	0&\ddots &\ddots&\vdots\\
	\vdots &\ddots &\ddots &0\\
	0&\ldots&0&E_1\end{array}\right).$$
Similarly consider in $M_{n_2(p-1)}(\Z/(q))$ the block diagonal matrices with $n_2$ blocks of degree $p-1$:
$$C_2=\left(\begin{array}{cccc}
	D_2&0&\ldots &0\\
	0&\ddots &\ddots&\vdots\\
	\vdots &\ddots &\ddots &0\\
	0&\ldots&0&D_2\end{array}\right)\mbox{ and } B_2=\left(\begin{array}{cccc}
	E_2&0&\ldots &0\\
	0&\ddots &\ddots&\vdots\\
	\vdots &\ddots &\ddots &0\\
	0&\ldots&0&E_2\end{array}\right).$$
Let $T=(\Z/(p))^{n_1(q-1)}\times (\Z/(q))^{n_2(p-1)}$ and $S=\Z/(p)\times\Z/(q)$ be the trivial left braces over these abelian groups.
Let $b\colon T\times T\rightarrow S$ be the symmetric linear map defined by
$$b((u_1,v_1),(u_2,v_2))=(u_1B_1u_2^t,v_1B_2v_2^t),$$
for all $u_1,u_2\in (\Z/(p))^{n_1(q-1)}$ and $v_1,v_2\in (\Z/(q))^{n_2(p-1)}$. Note that $\det(B_1)=(-q)^{q-2}\neq 0$ and $\det(B_2)=(-p)^{p-2}\neq 0$, thus the symmetric biliear forms associated to these matrices asre non-degenerate.
Let $\alpha\colon S\rightarrow \Aut(T)$ be the map defined by $\alpha(i,j)=\alpha_{(i,j)}$ and $\alpha_{(i,j)}(u,v)=(u(C_1^{t})^j,v(C_2^t)^{i})$, for all $(i,j)\in S$ and $(u,v)\in T$. Since $C_k^tB_kC_k=B_k$, for $k=1,2$, we can consider the asymmetric product $T\rtimes_{\circ}S$ of $T$ by $S$ (via $b$ and $\alpha$). Since $C_k-I$, where $I$ is the identity matrix, is invertible, by Theorem \ref{thm:simplebrace}, $T\rtimes_{\circ}S$ is a simple left brace of abelian type. 

Another construction is based in the following fact. Let $D$ an invertible matrix in $M_n(K)$, where $K$ is a field. Consider the block matrices
$$C=\left(\begin{array}{cc}
D^t&0\\
0&D^{-1}\end{array}\right)\in M_{2n}(K)\quad\mbox{and}\quad B=\left(\begin{array}{cc}
0&I_n\\
I_n&0\end{array}\right)\in M_{2n}(K).$$
Then one can check that $C^tBC=B$.     
Thus we can take any invertible matrix $D_1\in M_{m_1}(\Z/(p))$ of order $q$ and any matrix $D_2\in M_{m_2}(\Z/(q))$ of order $p$, such that $D_k-I$ also is invertible for $k=1,2$. 
Consider in $M_{2n_1m_1}(\Z/(p))$ the block square matrices with $2n_1$ blocks of degree $m_1$ in the diagonal:
$$C_1=\left(\begin{array}{cccccc}
	D_1^t&0&0&\ldots &0&0\\
	0&D_1^{-1}&0&\ldots &0&0\\
	0&0&\ddots &\ddots &\vdots&\vdots \\
	\vdots&\vdots&\ddots&\ddots&0&0\\
	0&0&\ldots&0&D_1^t&0\\
    0&0&\ldots&0&0&D_1^{-1}\end{array}\right)\mbox{ and } B_1=\left(\begin{array}{cccccc}
	0&I_{m_1}&0&\ldots &0&0\\
	I_{m_1}&0&0&\ldots &0&0\\
	0&0&\ddots&\ddots&\vdots&\vdots\\
	\vdots&\vdots &\ddots &\ddots &0&0\\
	0&0&\ldots&0&0&I_{m_1}\\
	0&0&\ldots&0&I_{m_1}&0
\end{array}\right).$$
Similarly consider in $M_{2n_2m_2}(\Z/(q))$ the block square matrices with $2n_2$ blocks of degree $m_2$ in the diagonal:
$$C_2=\left(\begin{array}{cccccc}
	D_2^t&0&0&\ldots &0&0\\
	0&D_2^{-1}&0&\ldots &0&0\\
	0&0&\ddots &\ddots &\vdots&\vdots \\
	\vdots&\vdots&\ddots&\ddots&0&0\\
	0&0&\ldots&0&D_2^t&0\\
	0&0&\ldots&0&0&D_2^{-1}\end{array}\right)\mbox{ and } B_1=\left(\begin{array}{cccccc}
	0&I_{m_2}&0&\ldots &0&0\\
	I_{m_2}&0&0&\ldots &0&0\\
	0&0&\ddots&\ddots&\vdots&\vdots\\
	\vdots&\vdots &\ddots &\ddots &0&0\\
	0&0&\ldots&0&0&I_{m_2}\\
	0&0&\ldots&0&I_{m_2}&0
\end{array}\right).$$
Let $T=(\Z/(p))^{2n_1m_1}\times (\Z/(q))^{2n_2m_2}$ and $S=\Z/(p)\times\Z/(q)$ be the trivial left braces over these abelian groups.
Let $b\colon T\times T\rightarrow S$ be the symmetric linear map defined by
$$b((u_1,v_1),(u_2,v_2))=(u_1B_1u_2^t,v_1B_2v_2^t),$$
for all $u_1,u_2\in (\Z/(p))^{2n_1m_1}$ and $v_1,v_2\in (\Z/(q))^{2n_2m_2}$.
Let $\alpha\colon S\rightarrow \Aut(T)$ be the map defined by $\alpha(i,j)=\alpha_{(i,j)}$ and $\alpha_{(i,j)}(u,v)=(u(C_1^{t})^j,v(C_2^t)^{i})$, for all $(i,j)\in S$ and $(u,v)\in T$. Since $C_k^tB_kC_k=B_k$, for $k=1,2$, we can consider the asymmetric product $T\rtimes_{\circ}S$ of $T$ by $S$ (via $b$ and $\alpha$). Since $C_k-I$, where $I$ is the identity matrix, is invertible, by Theorem \ref{thm:simplebrace}, $T\rtimes_{\circ}S$ is a simple left brace of abelian type. 

For example, for $p=2$ and $q=31$, with the first construction we obtain simple left braces of abelian type of order $2^{30n_1}31^{n_2}$ for all positive integers $n_1,n_2$. With the second construction, taking
$$D_1=\left(\begin{array}{ccccc}
0&0&0&0&1\\
1&0&0&0&0\\
0&1&0&0&1\\
0&0&1&0&0\\
0&0&0&1&0\end{array}\right)\in M_5(\Z/(2)),$$
we obtain simple left braces of abelian type of order $2^{10n_1}31^{2n_2}$ for all positive integers $n_1,n_2$. Using the two methods one can easily construct simple left braces of abelian type of order $2^{10n_1}31^{n_2}$ for all positive integers $n_1,n_2$.



\section*{Exercises}

\begin{prob}
\end{prob}

\begin{prob}
\end{prob}


\begin{prob}
\end{prob}

\begin{prob}
\end{prob}


\section*{Notes}

\index{Bachiller, D.}
\index{Ced\'o, F}
\index{Jespers, E.}
\index{Okni\'{n}ski, J.}
The first example of a simple non-trivial brace of abelian type is due to Bachiller \cite{MR3763276}. 
In the same papar Bachiller introduced the matched product of two left braces of abelian type. One can see that the definition of matched product of two left braces is the same that in the case of braces of abelian type. 
In \cite{MR3812099}, Bachiller, Ced\'o, Jespers and Okni\'{n}ski introduced the matched product of more than two left braces of abelian type (this can be generaliced to arbitrari left braces without changes). In the same paper the authors construct several families of  simple non-trivial left braces of abelian type using matched products.

\index{Catino, F.}
\index{Colazzo, I.}
\index{Stefanelli, P.}
The asymmetric product of left braces of abelian type was intruduced by Catino, Colazzo and Stefanelli in \cite{MR3478858}, and Theorem \ref{ccs} is due to them (\cite[Theorem 3]{MR3478858}).  In \cite{MR4020748} Bachiller, Ced\'o, Jespers and Okni\'{n}ski used the asymmetric product to construct to costruct new families of simple non-trivial left braces of abelian type. In fact, every known simple non-trivial left brace is an asymmetric product (see \cite{MR4020748, MR4161288, MR4122077}).
Theorem \ref{thm:simplebrace} is based on \cite[Theorem 3.6]{MR3812099} and \cite[Theorem 6.2]{MR4020748}. The two concrete constructions of simple non-trivial simple braces of abelian type appear in \cite{MR3812099,  MR4161288}.
