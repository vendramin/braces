\chapter{The Yang--Baxter equation}

In \cite{MR1183474}, Drinfeld briefly discuss set-theoretic solutions to the Yang--Baxter equation. 
He observed that
it makes sense to consider the Yang--Baxter equation in the category of sets and that 
"maybe it would be interesting to study set-theoretical solutions". 

\begin{definition}
\index{Solution}
\index{Solution!finite}
A \emph{set-theoretic solution} to the Yang--Baxter equation (YBE) is a pair $(X,r)$, 
where $X$ is a set and $r\colon X\times X\to X\times X$ is a bijective map that satisfies 
\[
(r\times\id)(\id\times r)(r\times\id)=
(\id\times r)(r\times\id)(\id\times r),
\]
where, if $r(x,y)=(\sigma_x(y),\tau_y(x))$, then 
\begin{align*}
& r\times\id\colon X\times X\times X\to X\times X\times X, &&(r\times\id)(x,y,z)=(\sigma_x(y),\tau_y(x),z),\\
& \id\times r\colon X\times X\times X\to X\times X\times X, &&(\id\times r)(x,y,z)=(x,\sigma_y(z),\tau_z(y)).
\end{align*}
The solution $(X,r)$ is said to be \emph{finite} if $X$ is a finite set. 
\end{definition}

\begin{figure}
\begin{tikzpicture}
\pic[
  braid/.cd,
  number of strands=3,
  height=.5cm,
  width=.5cm,
  ultra thick,
  gap=0.1,
  name prefix=braid,
] {braid={a_{1}a_2a_1}};
\end{tikzpicture}
\hspace{1cm}
\begin{tikzpicture}
\pic[
  braid/.cd,
  number of strands=3,
  height=.5cm,
  width=.5cm,
  ultra thick,
  gap=0.1,
  name prefix=braid,
] {braid={a_{2}a_1a_2}};
\end{tikzpicture}
\caption{The Yang--Baxter equation.}
\label{fig:braid}
\end{figure}

\index{Braid group}
For $n\geq2$, the \emph{braid group} $\B_n$ is defined as the group with generators $\sigma_1,\dots,\sigma_{n-1}$ and relations
\begin{align*}
    &\sigma_i\sigma_{i+1}\sigma_i=\sigma_{i+1}\sigma_i\sigma_{i+1} && \text{if }1\leq i\leq n-2,\\
    &\sigma_i\sigma_j=\sigma_j\sigma_i && \text{if }|i-j|\geq 1.
\end{align*}
Let $(X,r)$ be a set-theoretic solution to the YBE. Write $X^n=X\times\cdots\times X$ ($n$-times).  
For $i<n$ let $r_{i,i+1}=\id_{X^{i-1}}\times r\times\id_{X^{n-i-1}}\colon X^n\to X^n$. Then the map $\sigma_i\mapsto r_{i,i+1}$ extends 
to an action of $\B_n$ on $X^n$.

\begin{example}
\index{Solution!trivial}
Let $X$ be a set. Then $(X,r)$, where $r(x,y)=(y,x)$, is a solution to the YBE. This solution 
is known as the \emph{trivial solution} over the set $X$. 
\end{example}

By convention, we write
\[
r(x,y)=(\sigma_x(y),\tau_y(x)).
\]

\begin{lemma}
    \label{lem:YB}
    Let $X$ be a non-empty set and $r\colon X\times X\to X\times X$ be a bijective map.
    Then $(X,r)$ is a set-theoretic solution to the YBE if and only if 
    \begin{align*}
        &\sigma_x\circ\sigma_y = \sigma_{\sigma_x(y)}\circ\sigma_{\tau_y(x)},&
        &\sigma_{\tau_{\sigma_y(z)}(x)}\tau_z(y)=\tau_{\sigma_{\tau_y(x)}(z)}\sigma_x(y),&
        &\tau_z\circ\tau_y=\tau_{\tau_z(y)}\circ\tau_{\sigma_y(z)}
    \end{align*}
    for all $x,y,z\in X$. 
\end{lemma}

\begin{proof}
    We write $r_1=r\times\id$ and $r_2=\id\times r$. We first compute
    \begin{align*}
        r_1r_2r_1(x,y,z)&=r_1r_2(\sigma_x(y),\tau_y(x),z)
        =r_1(\sigma_x(y),\sigma_{\tau_y(x)}(z),\tau_z\sigma_x(y))\\
        &=\left(\sigma_{\sigma_x(y)}\sigma_{\tau_y(x)}(z),\tau_{\sigma_{\tau_y(x)}(z)}\sigma_x(y),\tau_z\tau_y(x)\right).
    \end{align*}
    Then we compute
    \begin{align*}
        r_2r_1r_2(x,y,z)&=r_2r_1(x,\sigma_y(z),\tau_z(y))
        =r_2(\sigma_x\sigma_y(z),\tau_{\sigma_y(z)}(x),\tau_z(y))\\
        &=\left(\sigma_x\sigma_y(z),\sigma_{\tau_{\sigma_y(z)}(x)}\tau_z(y),\tau_{\tau_z(y)}\tau_{\sigma_y(z)}(x)\right)
    \end{align*}
    and the claim follows.    
\end{proof}

If $(X,r)$ is a solution, by definition the map $r\colon X\times X\to X\times X$ is 
invertible. By convention, we write 
 \[
 r^{-1}(x,y)=(\widehat{\sigma}_x(y),\widehat{\tau}_y(x)).
 \]
 Note that this implies that
 \[
 x=\widehat{\sigma}_{\sigma_x(y)}\tau_y(x),\quad
 y=\widehat{\tau}_{\tau_y(x)}\sigma_x(y).
 \]
 Since $(X,r^{-1})$ is a solution, Lemma~\ref{lem:YB} implies that 
 the following formulas hold:
 \[
 \widehat{\tau}_y\circ\widehat{\tau_x}=\widehat{\tau}_{\tau_y(x)}\circ\widehat{\tau}_{\sigma_x(y)},
 \quad
 \widehat{\sigma}_x\circ\widehat{\sigma_y}=\widehat{\sigma}_{\sigma_x(y)}\circ\widehat{\sigma}_{\tau_y(x)}.
 \]
% Since $r(\tau^{-1}_y(x),y)=(\sigma_{\tau^{-1}_y(x)}(y),x)$, 
% it follows that 
% \[
% \widehat{\tau}_x\sigma_{\tau^{-1}_y(x)}(y)=y.
% \]
% for all $x,y\in X$. Moreover, 
% \[
% x\triangleright y=\tau_x\sigma_{\tau^{-1}_y(x)}(y)=\tau_x\widehat{\tau}^{-1}_x(y)
% \]
% for all $x,y\in X$. 

\begin{definition}
A \emph{homomorphism} between the set-theoretic solutions $(X,r)$ and
$(Y,s)$ is a map $f\colon X\to Y$ such that the diagram 
\[\begin{tikzcd}
	{X\times X} & {X\times X} \\
	{Y\times Y} & {Y\times Y}
	\arrow["r", from=1-1, to=1-2]
	\arrow["{f\times f}"', from=1-1, to=2-1]
	\arrow["{f\times f}", from=1-2, to=2-2]
	\arrow["s"', from=2-1, to=2-2]
\end{tikzcd}
\]
is commutative. An \emph{isomorphism} of solutions is a bijective
homomorphism of solutions.
\end{definition}

Since we are interested in studying the combinatorics behind set-theoretic solutions to the YBE,
it makes sense to study the following family of solutions. 

\begin{definition}
\index{Solution!non-degenerate}
We say that a solution $(X,r)$ to the YBE 
is \emph{non-degenerate} if the maps $\sigma_x$ and $\tau_x$ are 
permutations of $X$. 
\end{definition}

By convention, a \emph{solution} we will mean a non-degenerate solution to the YBE.

\begin{lemma}
\label{lem:LYZ}
Let $(X,r)$ be a solution. 
\begin{enumerate}
    \item Given $x,u\in X$, there exist unique $y,v\in X$ such that $r(x,y)=(u,v)$. 
    \item Given $y,v\in X$, there exist unique $x,u\in X$ such that $r(x,y)=(u,v)$. 
\end{enumerate}
\end{lemma}

\begin{proof}
    For the first claim take $y=\sigma_x^{-1}(u)$ and $v=\tau_y(x)$. 
    For the second, $x=\tau_y^{-1}(v)$ and $u=\sigma_x(y)$. 
\end{proof}

The bijectivity of $r$ means that any row determines the whole square. Lemma~\ref{lem:LYZ}
means that any column also determines the whole square, see Figure~\ref{fig:braid}.

\begin{figure}
\begin{tikzpicture}
\pic[
  braid/.cd,
  number of strands=2,
  ultra thick,
  gap=0.1,
  name prefix=braid,
] {braid={a_{1}^{-1}}};
\node[] at (-.25,-.12) {$x$};
\node[] at (1.25,-.12) {$y$};
\node[] at (-.25,-1.4) {$u$};
\node[] at (1.25,-1.4) {$v$};
\node[] at (-.25,-.75) {$r$};
\end{tikzpicture}
\begin{tikzpicture}
\pic[
  braid/.cd,
  number of strands=2,
  ultra thick,
  gap=0.1,
  name prefix=braid,
] {braid={a_{1}}};
\node[] at (-.25,-.12) {$x$};
\node[] at (1.25,-.12) {$y$};
\node[] at (-.25,-1.4) {$u$};
\node[] at (1.25,-1.4) {$v$};
\node[] at (-.25,-.75) {$r^{-1}$};
\end{tikzpicture}
\caption{Any row or column determines the whole square.}
\label{fig:braid}
\end{figure}

\begin{example}
If the map $(x,y)\mapsto(\sigma_x(y),\tau_y(x))$ satisfies the Yang--Baxter equation, then 
so does $(x,y)\mapsto (\tau_x(y),\sigma_y(x))$. 
\end{example}

\begin{example}
\label{exa:Lyubashenko}
Let $X$ be a non-empty set and $\sigma$ and $\tau$ be 
bijections on $X$ such that $\sigma\circ\tau=\tau\circ\sigma$. Then 
$(X,r)$, where $r(x,y)=(\sigma(y),\tau(x))$, is a non-degenerate solution. 
This is known as the \emph{permutation solution} associated
with permutations $\sigma$ and $\tau$. 
%The solution $(X,r)$ is involutive 
%if and only if $\tau^{-1}=\sigma$. 
\end{example}

\begin{example}
\label{exa:Wada}
Let $G$ be a group. Then $(G,r)$, where $r(x,y)=(xy^{-1}x^{-1},xy^2)$, is a solution. 
\end{example}

\begin{example}
\label{exa:Venkov}
Let $G$ be a group. Then $(G,r)$, where $r(x,y)=(xyx^{-1},x)$, is a solution. 
%It is involutive if and only if $G$ is abelian. 
\end{example}

\begin{theorem}
\label{thm:LYZ}
Let $G$ be a group and let $\xi\colon G\times G\to G$, $\xi(x,y)=x\rhd y$,
be a left action of $G$ on itself, and 
let $\eta\colon G\times G\to G$, $\eta(x,y)=x\lhd y$, 
be a right action of $G$ on itself. If the compatibility condition
\[
uv=(u\rhd v)(u\lhd v)
\]
holds for all $u,v\in G$, then the pair $(G,r)$, where 
\[
r\colon G\times G\to G\times G,\quad
r(u,v)=(u\rhd v,u\lhd v)
\]
is a bijective solution. 
\end{theorem}

\begin{proof}
We write $r_1=r\times\id$ and $r_2=\id\times r$. Let
\[
r_1r_2r_1(u,v,w)=(u_1,v_1,w_1),\quad
r_2r_1r_2(u,v,w)=(u_2,v_2,w_2).
\]
The compatibility condition implies that $u_1v_1w_1=u_2v_2w_2$. 
So we need to prove that $u_1=u_2$ and $v_1=v_2$. From Lemma~\ref{lem:YB}
we note that
\begin{align*}
&u_1=(u\rhd v)\rhd ( (u\lhd v)\rhd w),
&&v_1=(u\lhd v)\lhd w,\\
&u_2=u\rhd (v\rhd w),
&&v_2=(u\lhd (v\rhd w))\lhd (v\lhd w).
\end{align*}
Using the compatibility condition and the fact that $\xi$ is a left action, 
\begin{align*}
    &u_1=((u\rhd v)(u\lhd v))\rhd w=(uv)\rhd w=u\rhd (v\rhd w)=u_2.
\end{align*}
Similarly, since $\eta$ is a right action, 
\[
v_2=u\lhd ((v\rhd w)(v\lhd w))=u\lhd (vw)=(u\lhd v)\lhd w=v_1.
\]

To prove that $r$ is invertible we proceed as follows. Write $r(u,v)=(x,y)$, thus $u\rhd v=x$, $u\lhd v=y$ and $uv=xy$. Since 
\begin{align*}
& (y\rhd v^{-1})u=(y\rhd v^{-1})(y\lhd v^{-1})=yv^{-1}=x^{-1}u,
\end{align*}
it follows that $y\rhd v^{-1}=x^{-1}$, i.e. $v^{-1}=y^{-1}\rhd x^{-1}$. Similarly, 
\[
v(u^{-1}\lhd x)=(u^{-1}\rhd x)(u^{-1}\lhd x)=u^{-1}x=vy^{-1}
\]
implies that $u^{-1}=y^{-1}\lhd x^{-1}$. Clearly 
$r^{-1}=\zeta\circ (i\times i)\circ r\circ (i\times i)\circ \zeta$,
is the inverse of $r$, where $\zeta(x,y)=(y,x)$ and $i(x)=x^{-1}$. 
\end{proof}

\begin{proposition}
Under the assumptions of Theorem~\ref{thm:LYZ}, 
if $r(x,y)=(u,v)$, then 
\[
r(x^{-1},y^{-1})=(u^{-1},v^{-1}),
\quad
r(x^{-1},u)=(y,v^{-1}),
\quad
r(v,y^{-1})=(u^{-1},x).
\]
\end{proposition}

\begin{proof}
In the proof of Theorem~\ref{thm:LYZ} we found that 
the inverse of $r$ is given by $r^{-1}=\zeta\circ (i\times i)\circ r\circ (i\times i)\circ \zeta$,
where $\zeta(x,y)=(y,x)$ and $i(x)=x^{-1}$, it follows that $r(x^{-1},y^{-1})=(u^{-1},v^{-1})$.  
To prove the equality $r(x^{-1},u)=(y,v^{-1})$ we proceed as follows. Since $r(x,y)=(u,v)$, it 
follows that $x\triangleright y=u$. Then $x^{-1}\triangleright u=y$ and
hence $r(x^{-1},u)=(y,z)$ for some $z\in G$. 
Since $xy=uv$ and $x^{-1}u=yz$, it follows that $yt=yv^{-1}$. Then 
$z=v^{-1}$. Similarly one proves $r(v,y^{-1})=(u^{-1},x).$
\end{proof}

The rest of the chapter will be devoted to involutive solutions. 

\begin{definition}
\index{Solution!involutive}
A solution $(X,r)$ is said to be \emph{involutive} if $r^2=\id$. 
\end{definition}

\index{Symmetric group}
For $n\geq2$, the \emph{symmetric group} $\Sym_n$ can be presented 
as the group with generators $\sigma_1,\dots,\sigma_{n-1}$ and relations
\begin{align*}
    &\sigma_i\sigma_{i+1}\sigma_i=\sigma_{i+1}\sigma_i\sigma_{i+1} && \text{if }1\leq i\leq n-2,\\
    &\sigma_i\sigma_j=\sigma_j\sigma_i && \text{if }|i-j|\geq 1,\\
    &\sigma_i^2=1 && \text{for all $i\in\{1,\dots,n-1\}$}.
\end{align*}
Let $(X,r)$ be an involutive solution. 
Then the map $\sigma_i\mapsto r_{i,i+1}=\id_{X^{i-1}}\times r\times\id_{X^{n-i-1}}$ extends 
to an action of $\Sym_n$ on $X^n$.

\begin{example}
Let $X$ be a non-empty set and $\sigma$ be a bijection on $X$. Then 
$(X,r)$, where $r(x,y)=(\sigma(y),\sigma^{-1}(x))$, is an involutive solution. 
\end{example}

\index{Jacobson!radical ring}
\index{Radical ring}
We now present a very important family of involutive solutions. These examples show an intriguing connection between the YBE and the 
theory of non-commutative rings. 

In any ring $R$ the \emph{circle operation} 
\[
x\circ y=x+xy+y
\]
is always associative and such that $x\circ 0=0\circ x=x$ for all $x\in R$. 
A non-unital ring (or \emph{rng}, for short) 
$R$ is said to be a \emph{radical ring} if $(R,\circ)$ is a group. 
In this case, following Jacobson's notation, the inverse of 
an element $x$ with respect to the circle operation is denoted by $x'$. 

Radical rings were introduced by Jacobson in~\cite{MR12271}.
There are other characterizations of radical rings, see for example~\cite{MR3308118}.

\begin{example}
Let $p$ be a prime and let $A=\Z/(p^2)$ be the cylic additive group of order $p^2$. 
The operation $x\circ y=x+y+pxy$ 
turns $A$ into a radical ring. 
\end{example}

\begin{example}
Let $A=\left\{\frac{2x}{2y+1}:x,y\in\Z\right\}$. The operation 
$a\circ b=a+b+ab$ turns $A$ into a radical ring. A straightforward computation shows that 
\[
\left(\frac{x}{2y+1}\right)'=\frac{2(-x)}{2(x+y)+1}.
\]
\end{example}

The following fundamental family of solutions appears in~\cite{MR2278047}. 
It turns out to be fundamental in the study of 
set-theoretic solutions to the YBE. 

\begin{proposition}
\label{pro:Rump}
Let $R$ be a radical ring. Then $(R,r)$, where 
\[
r(x,y)=( -x+x\circ y,(-x+x\circ y)'\circ x\circ y)
\]
is an involutive solution.
\end{proposition}

The proposition can demonstrated using Theorem~\ref{thm:LYZ}, 
see Exercise~\ref{prob:Rump}. 
We will prove a stronger result in Theorem~\ref{thm:YB}. 

\begin{proposition}
\label{pro:T}
Let $(X,r)$ be an involutive solution. 
Then the map $T\colon X\to X$, $x\mapsto\sigma_x^{-1}(x)$, is 
invertible with inverse $T^{-1}(y)=\tau^{-1}_y(y)$ and 
\[
T^{-1}\circ\sigma_x^{-1}\circ T=\tau_x
\]
for all $x\in X$. 
\end{proposition}

\begin{proof}
Let $U(x)=\tau_x^{-1}(x)$. Since $r$ is involutive, 
\[
(U(x),x)=r^2(U(x),x)=r(\sigma_{U(x)}(x),x)=(\sigma_{\sigma_{U(x)}(x)}(x),\tau_x\sigma_{U(x)}(x)).
\]
The second coordinate can be written as $U(x)=\sigma_{U(x)}(x)$. This 
implies that 
\[
T(U(x))=\sigma^{-1}_{U(x)}(U(x))=x.
\]
Similarly one obtains $U(T(x))=x$. 

Since $(X,r)$ is a solution, Lemma~\ref{lem:YB} implies that 
$\sigma_x\sigma_y=\sigma_{\sigma_x(y)}\sigma_{\tau_y(x)}$
%$\tau_{\tau_y(x)}\circ\tau_{\sigma_x(y)}=\tau_y\circ\tau_x$ 
holds for all $x,y\in X$. Then  
\[
\sigma_y^{-1}T(x)
=\sigma_y^{-1}\sigma_x^{-1}(x)
=\sigma^{-1}_{\tau_y(x)}\sigma^{-1}_{\sigma_x(y)}(x)
=\sigma^{-1}_{\tau_y(x)}\tau_y(x)
=T\tau_y(x)
\]
for all $y\in X$, by Equality~\eqref{eq:involutive}.
\end{proof}

Note that if $(X,r)$ is a non-degenerate involutive solution, then 
\[
(x,y)=r^2(x,y)=r(\sigma_x(y),\tau_y(x))=(\sigma_{\sigma_x(y)}\tau_y(x),\tau_{\tau_y(x)}\sigma_x(y)).
\]
Hence 
\begin{equation}
    \label{eq:involutive}
    \tau_y(x)=\sigma_{\sigma_x(y)}^{-1}(x),
    \quad
    \sigma_x(y)=\tau_{\tau_y(x)}^{-1}(y)
\end{equation}
for all $x,y\in X$. Thus for involutive solutions
it is enough to know $\{\sigma_x:x\in X\}$, as from this we obtain the
set $\{\tau_x:x\in X\}$. 


\begin{definition}
\index{Cycle set}
\index{Cycle set!non-degenerate}
A \emph{cycle set} is a pair $(X,\cdot)$, where $X$ is a non-empty 
set provided with a binary operation $X\times X\to X$, $(x,y)\mapsto x\cdot y$, 
such that 
\begin{equation}
    \label{eq:cycle_set}
    (x\cdot y)\cdot (x\cdot z)=(y\cdot x)\cdot (y\cdot z)
\end{equation}
holds for all $x,y,z\in X$ and each map $\varphi_x\colon X\to X$, $y\mapsto x\cdot y$, is bijective. 
A cycle set $(X,\cdot)$ is said to be \emph{non-degenerate} 
if the map $X\to X$, $x\mapsto x\cdot x$, is bijective. 
\end{definition}

\begin{definition}
\index{Homomorphism!of cycle sets}
Let $X$ and $Z$ be cycle sets. 
A \emph{homomorphism} between the cycle sets $X$ and $Z$ is a 
map $f\colon X\to Z$ such that $f(x\cdot y)=f(x)\cdot f(y)$ for all $x,y\in X$. An \emph{isomorphism} of cycle sets
is a bijective homomorphism of cycle sets. 
\end{definition}

Cycle sets and cycle set homomorphisms form a category. 
It is possible to prove that the category of 
solutions is equivalent to the category of cycle sets, 
see Exercise~\ref{prob:cycle_sets}. 

% \framebox{Exercise!}

% \begin{lemma}
% \label{lem:T_forCS}
% If $X$ is a cycle set, then $x\cdot (y\cdot y)=((y*x)\cdot y)\cdot ((y*x)\cdot y)$, where
% $y*x=z$ if and only if $y\cdot z=x$. 
% \end{lemma}

% \begin{proof}
%     Since the operation $x\mapsto y\cdot x$ is bijective, we can write $x=y\cdot (y*x)$. Then, using~\eqref{eq:cycle_set}, 
%     $x\cdot (y\cdot y)=(y\cdot (y*x))\cdot (y\cdot y)=((y*x)\cdot y)\cdot (y*x)\cdot y)$.
% \end{proof}

\begin{theorem}
\label{thm:CS}
There exists a bijective correspondence between non-isomorphic involutive solutions 
and non-isomorphic non-degenerate cycle sets. 
\end{theorem}



% \begin{proof}
%     Let us assume first that $r(x,y)=(\sigma_x(y),\tau_y(x))$ is an involutive solution. 
%     We want to prove that the operation
%     $x\cdot y=\sigma_x^{-1}(y)$ turns the set $X$ into a non-degenerate cycle set. 
%     It is clear that the maps $y\mapsto x\cdot y=\sigma_x^{-1}(y)$ are invertible. 
%     By Proposition~\ref{pro:T}, the operation $x\mapsto x\cdot x$ is bijective. \framebox{!}
%     Since $r(\tau^{-1}_y(x),y)=(\sigma_{\tau^{-1}_y(x)}(y),x)$, Lemma~\ref{lem:YB} implies that
%     \[
%     \tau_x\circ\tau_{\sigma_{\tau^{-1}_y(x)}(y)}=\tau_y\circ\tau_{\tau^{-1}_y(x)}.
%     \]
%     Moreover, since $r(\sigma_{\tau^{-1}_y(x)}(y),x)=(\tau^{-1}_y(x),y)$, it follows that
%     %$\tau_x\sigma_{\tau^{-1}_y(x)}(y)=y$, i.e.
%     \[
%     \sigma_{\tau^{-1}_y(x)}(y)=\tau_x^{-1}(y).
%     \]
%     % Since $r^2=\id_{X\times X}$, it follows that
%     % \[
%     % (\tau^{-1}_y(x),y)=r^2(\tau^{-1}_y(x),y)=r(\sigma_{\tau^{-1}_y(x)}(y),x)
%     % \]
%     % and hence $\sigma_{\tau^{-1}_y(x)}(y)=\tau^{-1}_x(y)$. 
%     This implies that
%     \[
%     \tau_x\circ\tau_{\tau^{-1}_x(y)}=\tau_x\circ\tau_{\sigma_{\tau^{-1}_y(x)}(y)}=\tau_y\circ\tau_{\tau^{-1}_y(x)},
%     \]
%     which turns out to be equivalent to
%     \[
%     \tau^{-1}_{\tau^{-1}_x(y)}\circ\tau^{-1}_x=\tau^{-1}_{\tau^{-1}_y(x)}\circ\tau^{-1}_y
%     \]
%     and hence equivalent to Equality~\eqref{eq:cycle_set}. 
    
%     Conversely, if $X$ is a non-degenerate cycle set, we want to prove that
%     \[
%     r(x,y)=((y*x)\cdot y,y*x),
%     \]
%     where $y*x=z$ if and only if $y\cdot z=x$, is an involutive solution. The definition of $r$ implies that
%     $r^2=\id_{X\times X}$. 
%     By Lemma~\ref{lem:YB}, we need to prove that
%     \begin{align*}
%         &\sigma_x\circ\sigma_y = \sigma_{\sigma_x(y)}\circ\sigma_{\tau_y(x)},&
%         &\sigma_{\tau_{\sigma_y(z)}(x)}\tau_z(y)=\tau_{\sigma_{\tau_y(x)}(z)}\sigma_x(y),&
%         &\tau_z\circ\tau_y=\tau_{\tau_z(y)}\circ\tau_{\sigma_y(z)}
%     \end{align*}
 
%     Write $\sigma_x(y)=(y*x)\cdot y$ and $\tau_y(x)=y*x$. 
%     We know that the third formula is equivalent to~\eqref{eq:cycle_set}. 
%     Let $T\colon X\to X$, $x\mapsto x\cdot x$. By assumption, $T$ is bijective and hence 
%     $T^{-1}\circ \tau^{-1}_x\circ T=\sigma_x$ for all $x\in X$, since by Lemma~\ref{lem:T_forCS}
%         \begin{align*}
%     \tau_x^{-1}T(y)&=\tau_x^{-1}(y\cdot y)=x\cdot (y\cdot y)\\
%     &=((y*x)\cdot y)\cdot ((y*x)\cdot y)=T((y*x)\cdot y)=T\sigma_x(y)
%     \end{align*}
%     for all $x,y\in X$. In particular, $(X,r)$ is non-degenerate. 
%     Moreover, 
%     \begin{align*}
%     \sigma_x\circ\sigma_y
%     &=(T^{-1}\circ\tau_x^{-1}\circ T)\circ (T^{-1}\circ\tau_y^{-1}\circ T)\\
%     &=T^{-1}\circ (\tau_y\circ\tau_x)^{-1}\circ T\\
%     &=T^{-1}\circ (\tau_{\tau_y(x)}\circ\tau_{\sigma_x(y)})^{-1}\circ T\\
%     &=(T^{-1}\circ\tau^{-1}_{\sigma_x(y)}\circ T)\circ (T^{-1}\circ\tau^{-1}_{\tau_y(x)}\circ T)\\
%     &=\sigma_{\sigma_x(y)}\circ\sigma_{\tau_y(x)}
% \end{align*}
% for all $x,y\in X$. Finally, Equality~\eqref{eq:cycle_set} can be written as
% \[
% \tau^{-1}_{\tau^{-1}_v(u)}\circ\tau^{-1}_v=\tau^{-1}_{\tau^{-1}_u(v)}\circ\tau^{-1}_u.
% \]
% With $u=\tau_z\tau_y(x)$ and $v=z$, it becomes
% \[
% \tau^{-1}_{\tau_y(x)}\circ\tau^{-1}_z
% =\tau^{-1}_{\tau^{-1}_{\tau_z\tau_y(x)}(z)}\circ\tau^{-1}_{\tau_z\tau_y(x)}
% =\tau^{-1}_{\sigma_{\tau_y(x)}(z)}\circ\tau^{-1}_{\tau_z\tau_y(x)},
% \]
% where the last equality holds $r^2=\id_{X\times X}$. 
% Using Lemma~\ref{lem:YB} one rewrites this formula as
% \[
% \tau_{\sigma_{\tau_y(x)}(z)}\circ\tau^{-1}_{\tau_y(x)}
% =\tau^{-1}_{\tau_z\tau_y(x)}\circ\tau_z
% =\tau^{-1}_{\tau_{\tau_z(y)}\tau_{\sigma_y(z)}(x)}\circ\tau_z.
% \]
% Evaluating this equality at $y$, 
% \[
% \tau_{\sigma_{\tau_y(x)}(z)}\sigma_x(y)=\tau_{\sigma_{\tau_y(x)}(z)}\tau^{-1}_{\tau_y(x)}(y)
% =\tau^{-1}_{\tau_{\tau_z(y)}\tau_{\sigma_y(z)}(x)}\tau_z(y)
% =\sigma_{\tau_{\sigma_y(z)}(x)}\tau_z(y).\qedhere
% \]
% \end{proof}

For the readers who are not familiar with the above-mentioned result, 
the bijective correspondence is given by 
\[
r(x,y)=(x*y,(x*y)\cdot x),
\]
where $x*y=z$ if and only if $x\cdot z=y$. We leave the proof for the reader, see 
Exercise~\ref{prob:CS}. However, we will prove a more 
general result in Theorem~\ref{thm:skewCS}. 

Theorem~\ref{thm:CS} can be used to construct and enumerate small 
involutive solutions~\cite{AMV}. Table~\ref{tab:IYB} shows the 
number of non-isomorphic involutive solutions of size $\leq10$. 
For size $\leq7$ the numbers of Table~\ref{tab:IYB} coincide with those in~\cite{MR1722951}
but differ by two for $n=8$, as two solutions of size eight 
are missing in~\cite{MR1722951}. 

\begin{table}[H]
%\centering
\caption{Involutive solutions of size $\leq10$.}
\begin{tabular}{|r|ccccccccc|}
\hline
$n$ & 2 & 3 & 4 & 5 & 6 & 7 & 8 & 9 & 10\tabularnewline
\hline 
solutions & 2 & 5 & 23 & 88 & 595 & 3456 & 34530 & 321931 & 4895272\tabularnewline
% square-free & 1 & 2 & 5 & 17 & 68 & 336 & 2041 & \cellcolor{gray!30}{15534} & \cellcolor{gray!30}{150957}\tabularnewline
% indecomposable & 1 & 1 & 5 & 1 & 10 & 1 & \cellcolor{gray!30}{100} & \cellcolor{gray!30}{16} & \cellcolor{gray!30}{36}\tabularnewline
% multipermutation & 2 & 5 & 21 & 84 & 554 & 3295 & \cellcolor{gray!30}{32155} & \cellcolor{gray!30}{305916} & \cellcolor{gray!30}{4606440}\tabularnewline
% irretractable & 0 & 0 & 2 & 4 & 9 & 13 & 191 & \cellcolor{gray!30}{685} & \cellcolor{gray!30}{3590}\tabularnewline
\hline
\end{tabular}
\label{tab:IYB}
\end{table}



% \begin{definition}
% \index{Solution!permutation group}
% The \emph{permutation group} of a solution $(X,r)$ is the group
% \[
% \mathcal{G}(X,r)=\langle (\sigma_x,\widehat{\sigma_x}):x\in X\rangle\subseteq\Sym_X\times\Sym_X.
% \]
% \end{definition}

% \index{Solutions!linear involutive}
% Let us finish the chapter with the family of affine involutive solutions. Let $A$ be an abelian group and
% $a,b,c,d\in\colon A\to A$ be group homomorphisms. If 
% \[
% r(x,y)=(a(x)+b(y),c(x)+d(y))
% \]
% is bijective and $(A,r)$ is an solution involutive solution, then $(A,r)$ is an \emph{linear solution}.
% Note that $r$ satisfies the YBE if and only if
% \begin{subequations}
% \begin{gather}
% \label{eq:linear}
% (\id-a)\circ d\circ b=b\circ d,
% \quad
% d\circ(\id-d)=c\circ d\circ b,\\
% c\circ d\circ (\id-a)=d\circ c,
% \quad
% a\circ(\id-a)=b\circ a\circ c,
% \quad
% c\circ a=(\id-d)\circ a\circ c,\\
% a\circ b=b\circ a\circ (\id-d),
% \quad
% c\circ b-b\circ c=a\circ d\circ a-d\circ d.
% \end{gather}
% \end{subequations}
% Moreover, $r^2=\id$ if and only if
% \begin{align}
%     &a\circ a+b\circ c=\id,
%     &&
%     c\circ b+d\circ d=\id,
%     &&
%     a\circ b+b\circ d=0,
%     &&
%     c\circ a+d\circ c=0.
% \end{align}
% \end{equation}

\section*{Exercises}

\begin{prob}
\label{prob:Wada}
Let $G$ be a group. Prove that $r(x,y)=(xy^{-1}x^{-1},xy^2)$ 
is involutive if and only if $x^2=1$ for all $x\in G$. 
\end{prob}

\begin{prob}
Prove that if $(X,r)$ be a solution...? \framebox{FIXME}
\end{prob}

\begin{prob}
\label{prob:cycle_sets}
Prove that the category of non-degenerate 
cycle sets and the category of solutions are equivalent. 
\end{prob}

\begin{prob}
\label{prob:Rump}
Prove Proposition~\ref{pro:Rump}. 
\end{prob}

\begin{prob}
If $X$ is a cycle set, then $x\cdot (y\cdot y)=((y*x)\cdot y)\cdot ((y*x)\cdot y)$, where
$y*x=z$ if and only if $y\cdot z=x$. 
\end{prob}

\begin{prob}
\label{prob:CS}
Prove Theorem~\ref{thm:CS}. 
\end{prob}

\begin{prob}
\label{prob:perm_group}
Let $(X,r)$ be a solution. Prove that $\mathcal{G}(X,r)\simeq\langle (\sigma_x,\tau^{-1}_x):x\in X\rangle$. 
\end{prob}

\section*{Open problems}

\begin{problem}
Construct and enumerate involutive solutions of size $11$. 
\end{problem}

\begin{problem}
Estimate the number of solutions of size $n$ for $n\to\infty$. 
\end{problem}

\section*{Notes}

\index{Gateva--Ivanova, T.}
\index{Van den Bergh, M.}
The first papers where set-theoretic solutions are studied are those of Etingof, Schedler and Soloviev~\cite{MR1722951} 
and Gateva--Ivanova and Van den Bergh~\cite{MR1637256}. Both papers deal with non-degenerate involutive solutions. 

\index{Rump, W.}
\index{Drinfeld, V.}
\index{Wada, M.}
\index{Lyubashenko, V.}
In~\cite{MR1183474}, Drinfeld attributes Example~\ref{exa:Lyubashenko} to 
Lyubashenko. Example~\ref{exa:Wada} 
appears in the work of Wada~\cite{MR1167178}. Proposition~\ref{pro:Rump} 
was proved by Rump~\cite{MR2278047}. 
%a similar example appears in the appendix of~\cite{MR1722951}. 

\index{Lu, J--H.}
\index{Yan, M.}
\index{Zhu, Y--C.}
\index{Etingof, P.}
\index{Schedler, T.}
\index{Soloviev, A.}
Theorem~\ref{thm:LYZ} goes back to Lu, Yan and Zhu, see~\cite{MR1769723}.
Similar results can be found in the work of Etingof, Schedler and Soloviev~\cite{MR1722951} for involutive solutions and 
in Soloviev's paper~\cite{MR1809284}. 

Rump introduced cycle sets in~\cite{MR2132760}. The bijective correspondence of 
Theorem~\ref{thm:CS} was 
also proved by Rump in~\cite{MR2132760}. A similar result can be 
found in~\cite[Proposition 2.2]{MR1722951}. 

The numbers of Table~\ref{tab:IYB} were computed in~\cite{AMV}
using a combination of~\cite{GAP4} and constraint programming techniques. 
The algorithm is based on an idea of Plemmons~\cite{MR0258994}, originally 
conceived to construct non-isomorphic semigroups.  
