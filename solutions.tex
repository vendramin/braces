\chapter{The Yang--Baxter equation}
\label{YB}

%The quantum Yang--Baxter equation is one of the important equations of mathematical physics first appearing in the papers of Yang \cite{MR261870} and Baxter \cite{MR290733}. A solution to the Yang--Baxter equation is a bijective linear map $R\colon V\otimes V\to V\otimes V$, where $V$ is a vector space, such that
%\[R_{12}R_{13}R_{23}=R_{23}R_{13}R_{12},\]
%where $R_{ij}$ denotes the map $V\otimes V\otimes V\to V\otimes V\otimes V$ acting as $R$ on the $(i,j)$ tensor factor and as the identity on the remaining factor. Let $\varphi\colon V\otimes V\to V\otimes V$ be the linear map such that $\varphi(u\otimes v)=v\otimes u$ for all $u,v\in V$. Then one can check that $R\colon V\otimes V\to V\otimes V$ is a solution to the quantum Yang--Baxter equation if and only if $\overline{R}=\varphi R$ satisfies 
%\[ \overline{R}_{12}\overline{R}_{23}\overline{R}_{12}=\overline{R}_{23}\overline{R}_{12}\overline{R}_{23}.\]
%In this case, we say that $\overline{R}$ is a solution to the Yang--Baxter equation. 

%Note that if $X$ is a basis of $V$ and $r\colon X\times X\to X\times X$ is a bijective map such that
%\[(r\times \id_X)(\id_X\times r)(r\times \id_X)=(\id_X\times r)(r\times \id_X)(\id_X\times r),\]
% then $r$ induces a solution to the Yang--Baxter equation. In this case, we say that $r$ is a set-theoretic solution to the Yang--Baxter equation.

%In this chapter we study a particular important class of set-theoretic solutions to the Yang--Baxter equation.

\section{Set-theoretic solutions}

In \cite{MR1183474}, Drinfeld briefly discuss set-theoretic solutions to the YBE. 
He not only observed that
it makes sense to consider the YBE in the category of sets but also that 
"maybe it would be interesting to study set-theoretical solutions". 

\begin{definition}
\index{Solution}
\index{Solution!finite}
A \emph{set-theoretic solution} to the Yang--Baxter equation (YBE) is a pair $(X,r)$, 
where $X$ is a non-empty set and $r\colon X\times X\to X\times X$ is a bijective map that satisfies 
\[
(r\times\id_X)(\id_X\times r)(r\times\id_X)=
(\id_X\times r)(r\times\id_X)(\id_X\times r),
\]
where, if $r(x,y)=(\sigma_x(y),\tau_y(x))$, then 
\begin{align*}
& r\times\id_X\colon X\times X\times X\to X\times X\times X, &&(r\times\id_X)(x,y,z)=(\sigma_x(y),\tau_y(x),z),\\
& \id_X\times r\colon X\times X\times X\to X\times X\times X, &&(\id_X\times r)(x,y,z)=(x,\sigma_y(z),\tau_z(y)).
\end{align*}
The solution $(X,r)$ is said to be \emph{finite} if $X$ is a finite set. 
\end{definition}

\begin{figure}
\centering
\begin{tikzpicture}
\pic[
  braid/.cd,
  number of strands=3,
  height=.5cm,
  width=.5cm,
  ultra thick,
  gap=0.1,
  name prefix=braid,
] {braid={a_{1}a_2a_1}};
\end{tikzpicture}
\hspace{1cm}
\begin{tikzpicture}
\pic[
  braid/.cd,
  number of strands=3,
  height=.5cm,
  width=.5cm,
  ultra thick,
  gap=0.1,
  name prefix=braid,
] {braid={a_{2}a_1a_2}};
\end{tikzpicture}
\caption{The Yang--Baxter equation.}
\label{fig:braid}
\end{figure}

\begin{example}
    Let $X$ be a non-empty set. Then $(X,\id_{X^2})$ is a set-theoretic 
    solution to the YBE. 	
\end{example}

\begin{example}
    \index{Solution!trivial}
    Let $X$ be a non-empty set. Then $(X,r)$, 
    where $r(x,y)=(y,x)$, is a  set-theoretic solution to the YBE. This solution 
    is known as the \emph{trivial solution} over the set $X$. 
\end{example}

\begin{convention}
If $(X,r)$ is a set-theoretic solution to the YBE, we always write
\[
r(x,y)=(\sigma_x(y),\tau_y(x)).
\]
By definition, the map $r\colon X\times X\to X\times X$ is 
invertible. By convention, we always write 
\[
 r^{-1}(x,y)=(\widehat{\sigma}_x(y),\widehat{\tau}_y(x)).
\]
It follows that
\[
x=\widehat{\sigma}_{\sigma_x(y)}\tau_y(x),\quad
y=\widehat{\tau}_{\tau_y(x)}\sigma_x(y).
\]
\end{convention}



\begin{lemma}
    \label{lem:YB}
    Let $X$ be a non-empty set and $r\colon X\times X\to X\times X$ be a bijective map.
    Then $(X,r)$ is a set-theoretic solution to the YBE if and only if 
    \begin{align*}
        &\sigma_x\sigma_y = \sigma_{\sigma_x(y)}\sigma_{\tau_y(x)},&
        &\sigma_{\tau_{\sigma_y(z)}(x)}\tau_z(y)=\tau_{\sigma_{\tau_y(x)}(z)}\sigma_x(y),&
        &\tau_z\tau_y=\tau_{\tau_z(y)}\tau_{\sigma_y(z)}
    \end{align*}
    for all $x,y,z\in X$. 
\end{lemma}

\begin{proof}
    We write $r_1=r\times\id_X$ and $r_2=\id_X\times r$. We first compute
    \begin{align*}
        r_1r_2r_1(x,y,z)&=r_1r_2(\sigma_x(y),\tau_y(x),z)
        =r_1(\sigma_x(y),\sigma_{\tau_y(x)}(z),\tau_z\tau_y(x))\\
        &=\left(\sigma_{\sigma_x(y)}\sigma_{\tau_y(x)}(z),\tau_{\sigma_{\tau_y(x)}(z)}\sigma_x(y),\tau_z\tau_y(x)\right).
    \end{align*}
    Then we compute
    \begin{align*}
        r_2r_1r_2(x,y,z)&=r_2r_1(x,\sigma_y(z),\tau_z(y))
        =r_2(\sigma_x\sigma_y(z),\tau_{\sigma_y(z)}(x),\tau_z(y))\\
        &=\left(\sigma_x\sigma_y(z),\sigma_{\tau_{\sigma_y(z)}(x)}\tau_z(y),\tau_{\tau_z(y)}\tau_{\sigma_y(z)}(x)\right)
    \end{align*}
    and the \textcolor{red}{result} follows. 
\end{proof}

\begin{example}
If $(X,r)$ is a set-theoretic solution to the YBE, then
so is $(X,r^{-1})$. By Lemma~\ref{lem:YB},
the following formulas hold:
\[
\widehat{\tau}_y\widehat{\tau_x}=\widehat{\tau}_{\tau_y(x)}\widehat{\tau}_{\sigma_x(y)},
\quad
\widehat{\sigma}_x\widehat{\sigma_y}=\widehat{\sigma}_{\sigma_x(y)}\widehat{\sigma}_{\tau_y(x)}.
\]
\end{example}

\begin{example}
Let $X=\{1,2,3,4\}$ and $r(x,y)=(\sigma_x(y),\tau_y(x))$, where
\begin{align*}
&\sigma_1=(132),&&
\sigma_2=(124),&&
\sigma_3=(143),&&
\sigma_4=(234),\\
&\tau_1=(12)(34),&&
\tau_2=(12)(34),&&
\tau_3=(12)(34),&&
\tau_4=(12)(34).
\end{align*}
Then $r$ is invertible with $r^{-1}(x,y)=(\widehat{\sigma}_x(y),\widehat{\tau}_y(x))$ given by
\begin{align*}
&\widehat{\sigma}_1=(12)(34), &&
\widehat{\sigma}_2=(12)(34), &&
\widehat{\sigma}_3=(12)(34), &&
\widehat{\sigma}_4=(12)(34),\\
&\widehat{\tau}_1=(142),&&
\widehat{\tau}_2=(123),&&
\widehat{\tau}_3=(243),&&
\widehat{\tau}_4=(134).
\end{align*}
\end{example}

\begin{definition}
\index{Homomorphism!of set-theoretic solutions}
\index{Isomorphism!of set-theoretic solutions}
A \emph{homomorphism} between the set-theoretic solutions $(X,r)$ and
$(Y,s)$ is a map $f\colon X\to Y$ such that the diagram 
\[\begin{tikzcd}
	{X\times X} & {X\times X} \\
	{Y\times Y} & {Y\times Y}
	\arrow["r", from=1-1, to=1-2]
	\arrow["{f\times f}"', from=1-1, to=2-1]
	\arrow["{f\times f}", from=1-2, to=2-2]
	\arrow["s"', from=2-1, to=2-2]
\end{tikzcd}
\]
is commutative, that is $s (f\times f)=(f\times f) r$. An \emph{isomorphism} of solutions is a bijective
homomorphism of solutions.
\end{definition}

Since we are interested in studying the combinatorics behind set-theoretic solutions to the YBE,
it makes sense to study the following family of solutions. 

\begin{definition}
\index{Solution!non-degenerate}
We say that a set-theoretic solution $(X,r)$ to the YBE 
is \emph{non-degenerate} if all 
the maps $\sigma_x$ and $\tau_x$ are 
permutations of $X$. 
\end{definition}

\begin{convention}
A \emph{solution} \textcolor{red}{to the YBE} we will always mean a non-degenerate set-theoretic 
solution to the YBE.
\end{convention}

\begin{lemma}
\label{lem:LYZ}
Let $(X,r)$ be a solution \textcolor{red}{to the YBE}. 
\begin{enumerate}
    \item Given $x,u\in X$, there exist unique $y,v\in X$ such that $r(x,y)=(u,v)$. 
    \item Given $y,v\in X$, there exist unique $x,u\in X$ such that $r(x,y)=(u,v)$. 
\end{enumerate}
\end{lemma}

\begin{proof}
    For the first claim take $y=\sigma_x^{-1}(u)$ and $v=\tau_y(x)$. 
    For the second, $x=\tau_y^{-1}(v)$ and $u=\sigma_x(y)$. 
\end{proof}

The bijectivity of $r$ means that any row determines the whole square. Lemma~\ref{lem:LYZ}
means that any column also determines the whole square, see Figure~\ref{fig:braid}.

\begin{figure}
\centering
\begin{tikzpicture}
\pic[
  braid/.cd,
  number of strands=2,
  ultra thick,
  gap=0.1,
  name prefix=braid,
] {braid={a_{1}^{-1}}};
\node[] at (-.25,-.12) {$x$};
\node[] at (1.25,-.12) {$y$};
\node[] at (-.25,-1.4) {$u$};
\node[] at (1.25,-1.4) {$v$};
\node[] at (-.25,-.75) {$r$};
\end{tikzpicture}
\begin{tikzpicture}
\pic[
  braid/.cd,
  number of strands=2,
  ultra thick,
  gap=0.1,
  name prefix=braid,
] {braid={a_{1}}};
\node[] at (-.25,-.12) {$x$};
\node[] at (1.25,-.12) {$y$};
\node[] at (-.25,-1.4) {$u$};
\node[] at (1.25,-1.4) {$v$};
\node[] at (-.25,-.75) {$r^{-1}$};
\end{tikzpicture}
\caption{Any row or column determines the whole square.}
\label{fig:braid}
\end{figure}

\begin{example}
    If the map $(x,y)\mapsto(\sigma_x(y),\tau_y(x))$ satisfies the YBE, then 
    so does $(x,y)\mapsto (\tau_x(y),\sigma_y(x))$. 
\end{example}

\begin{example}
\label{exa:Lyubashenko}
\index{Solution!permutation}
\index{Solution!Lyubashenko}
Let $X$ be a non-empty set and $\sigma$ and $\tau$ be 
bijections on $X$ such that $\sigma\tau=\tau\sigma$. Then 
$(X,r)$, where $r(x,y)=(\sigma(y),\tau(x))$, is a solution \textcolor{red}{to the YBE}. 
This is known as the \emph{permutation solution} associated
with permutations $\sigma$ and $\tau$. 
\end{example}

%
%\begin{example}
%\label{exa:Wada}
%Let $G$ be a group. Then $(G,r)$, where $r(x,y)=(xy^{-1}x^{-1},xy^2)$, is a solution. 
%\end{example}

\begin{example}
\label{exa:Venkov}
Let $G$ be a group. Then $(G,r)$, where $r(x,y)=(xyx^{-1},x)$, is a solution \textcolor{red}{to the YBE}. 
%It is involutive if and only if $G$ is abelian. 
\end{example}

Now we will prove the main theorem of this chapter. The result
shows an intriguing connection 
between group actions and non-degenerate solutions. It 
was proved by Lu, Yan and Zhu. 

\begin{theorem}
\label{thm:LYZ}
Let $G$ be a group and let $\xi\colon G\times G\to G$, $\xi(x,y)=x\rhd y$,
be a left action of the group $G$ on itself as a set, and 
let $\eta\colon G\times G\to G$, $\eta(x,y)=x\lhd y$, 
be a right action of the group $G$ on itself as a set. If the compatibility condition
\[
uv=(u\rhd v)(u\lhd v)
\]
holds for all $u,v\in G$, then the pair $(G,r)$, where 
\[
r\colon G\times G\to G\times G,\quad
r(u,v)=(u\rhd v,u\lhd v)
\]
is a solution. 
%Moreover, 
%if $r(x,y)=(u,v)$, then 
%\[
%r(x^{-1},y^{-1})=(u^{-1},v^{-1}),
%\quad
%r(x^{-1},u)=(y,v^{-1}),
%\quad
%r(v,y^{-1})=(u^{-1},x).
%\]
\end{theorem}

\begin{proof}
We write $r_1=r\times\id_G$ and $r_2=\id_G\times r$. Let
\[
r_1r_2r_1(u,v,w)=(u_1,v_1,w_1),\quad
r_2r_1r_2(u,v,w)=(u_2,v_2,w_2).
\]
The compatibility condition implies that $u_1v_1w_1=u_2v_2w_2$. 
So we need to prove that $u_1=u_2$ and $w_1=w_2$. We note that
\begin{align*}
&u_1=(u\rhd v)\rhd ( (u\lhd v)\rhd w),
&&w_1=(u\lhd v)\lhd w,\\
&u_2=u\rhd (v\rhd w),
&&w_2=(u\lhd (v\rhd w))\lhd (v\lhd w).
\end{align*}
Using the compatibility condition and the fact that $\xi$ is a left action, 
\begin{align*}
    &u_1=((u\rhd v)(u\lhd v))\rhd w=(uv)\rhd w=u\rhd (v\rhd w)=u_2.
\end{align*}
Similarly, since $\eta$ is a right action, 
\[
w_2=u\lhd ((v\rhd w)(v\lhd w))=u\lhd (vw)=(u\lhd v)\lhd w=w_1.
\]

To prove that $r$ is invertible we proceed as follows. 
Write $r(u,v)=(x,y)$, thus $u\rhd v=x$, $u\lhd v=y$ and $uv=xy$. Since 
\begin{align*}
& (y\rhd v^{-1})u=(y\rhd v^{-1})(y\lhd v^{-1})=yv^{-1}=x^{-1}u,
\end{align*}
it follows that $y\rhd v^{-1}=x^{-1}$, i.e. $v^{-1}=y^{-1}\rhd x^{-1}$. Similarly, 
\[
v(u^{-1}\lhd x)=(u^{-1}\rhd x)(u^{-1}\lhd x)=u^{-1}x=vy^{-1}
\]
implies that $u^{-1}=y^{-1}\lhd x^{-1}$. Clearly 
$r^{-1}=\zeta (i\times i) r (i\times i) \zeta$,
is the inverse of $r$, where $\zeta(x,y)=(y,x)$ and $i(x)=x^{-1}$. 
\end{proof}

\begin{proposition}\label{prop:LYZ}
Under the assumptions of Theorem~\ref{thm:LYZ}, 
if $r(x,y)=(u,v)$, then 
\[
r(v^{-1},u^{-1})=(y^{-1},x^{-1}),
\quad
r(x^{-1},u)=(y,v^{-1}),
\quad
r(v,y^{-1})=(u^{-1},x).
\]
\end{proposition}

\begin{proof}
In the proof of Theorem~\ref{thm:LYZ} we found that 
the inverse of the map $r$ is given by $r^{-1}=\zeta (i\times i) r (i\times i) \zeta$,
where $\zeta(x,y)=(y,x)$ and $i(x)=x^{-1}$. Hence 
\[
r^{-1}(y^{-1},x^{-1})=\zeta (i\times i) r (i\times i) \zeta(y^{-1},x^{-1})=\zeta (i\times i) r (x,y)=(v^{-1},u^{-1}).
\]
It follows that $r(v^{-1},u^{-1})=(y^{-1},x^{-1})$.  
To prove the equality $r(x^{-1},u)=(y,v^{-1})$ we proceed as follows. Since $r(x,y)=(u,v)$, it 
follows that $x\rhd y=u$. Then $x^{-1}\rhd u=y$ and
hence $r(x^{-1},u)=(y,z)$ for some $z\in G$. 
Since $xy=uv$ and $x^{-1}u=yz$, it immediately follows that $yz=yv^{-1}$. Then 
$z=v^{-1}$. Similarly one proves $r(v,y^{-1})=(u^{-1},x)$.
\end{proof}
%
%\begin{proposition}
%Under the assumptions of Theorem~\ref{thm:LYZ}, 
%if $r(x,y)=(u,v)$, then 
%\[
%r(x^{-1},y^{-1})=(u^{-1},v^{-1}),
%\quad
%r(x^{-1},u)=(y,v^{-1}),
%\quad
%r(v,y^{-1})=(u^{-1},x).
%\]
%\end{proposition}
%
%\begin{proof}
%In the proof of Theorem~\ref{thm:LYZ} we found that 
%the inverse of the map $r$ is given by $r^{-1}=\zeta (i\times i) r (i\times i) \zeta$,
%where $\zeta(x,y)=(y,x)$ and $i(x)=x^{-1}$. It follows that $r(x^{-1},y^{-1})=(u^{-1},v^{-1})$.  
%To prove the equality $r(x^{-1},u)=(y,v^{-1})$ we proceed as follows. Since $r(x,y)=(u,v)$, it 
%follows that $x\triangleright y=u$. Then $x^{-1}\triangleright u=y$ and
%hence $r(x^{-1},u)=(y,z)$ for some $z\in G$. 
%Since $xy=uv$ and $x^{-1}u=yz$, it follows that $yt=yv^{-1}$. Then 
%$z=v^{-1}$. Similarly one proves $r(v,y^{-1})=(u^{-1},x).$
%\end{proof}


\section{Involutive solutions}

\textcolor{red}{In this section we study solutions to the YBE with an intriguing interplay
with radical rings.} 

\begin{definition}
	\index{Solution!involutive}
	A solution $(X,r)$ \textcolor{red}{to the YBE} is said to be \emph{involutive} if $r^2=\id$. 
\end{definition}

\index{Symmetric group}
For $n\geq2$, the \emph{symmetric group} $\Sym_n$ can be presented 
as the group with generators $\sigma_1,\dots,\sigma_{n-1}$ and relations
\begin{align*}
	&\sigma_i\sigma_{i+1}\sigma_i=\sigma_{i+1}\sigma_i\sigma_{i+1} && \text{if }1\leq i\leq n-2,\\
	&\sigma_i\sigma_j=\sigma_j\sigma_i && \text{if }|i-j|> 1,\\
	&\sigma_i^2=1 && \text{for all $i\in\{1,\dots,n-1\}$}.
\end{align*}
Let $(X,r)$ be an involutive solution \textcolor{red}{to the YBE}. 
Then the map $\sigma_i\mapsto r_{i,i+1}=\id_{X^{i-1}}\times r\times\id_{X^{n-i-1}}$ extends 
to an action of $\Sym_n$ on $X^n$.

\begin{example}
	Let $X$ be a non-empty set and $\sigma$ be a bijection on $X$. Then 
	$(X,r)$, where $r(x,y)=(\sigma(y),\sigma^{-1}(x))$, is an involutive solution \textcolor{red}{to the YBE}. 
\end{example}

\index{Jacobson!radical ring}
\index{Radical ring}
We now present a very important family of involutive solutions \textcolor{red}{to the YBE}. 
These examples show an intriguing connection between the YBE and the 
theory of non-commutative rings. 


% In any ring $R$ the \emph{circle operation} 
% \[
% x\circ y=x+xy+y
% \]
% is always associative and such that $x\circ 0=0\circ x=x$ for all $x\in R$. 
% A non-unital ring (or \emph{ring}, for short) 
% $R$ is said to be a \emph{radical ring} if $(R,\circ)$ is a group. 
% In this case, following Jacobson's notation, the inverse of 
% an element $x$ with respect to the circle operation is denoted by $x'$. 


\begin{example}
	Let $p$ be a prime and let $A=\Z/(p^2)=\Z/p^2\Z$ be the cylic additive group of order $p^2$. 
	Then $A$ with a new multiplication $*$ defined by $a*b=pab$ is a radical ring. In this case, $x\circ y=x+y+pxy$, and $x'=-x+px^2$. 
\end{example}

\begin{example}
	Let $n$ be an integer such that $n>1$. Let $A=\left\{\frac{nx}{ny+1}:x,y\in\Z\right\}\subseteq \Q$. Note that A is a subring without unity of the field $\Q$. In fact $A$ is a radical ring. A straightforward computation shows that 
	\[
	\left(\frac{nx}{ny+1}\right)'=\frac{-nx}{n(x+y)+1}.
	\]
\end{example}

The following fundamental family of solutions appears in~\cite{MR2278047}. 
It turns out to be fundamental in the study of 
set-theoretic solutions to the YBE. 

\begin{proposition}
	\label{pro:Rump}
	Let $R$ be a radical ring. Then $(R,r)$, where 
	\[
	r(x,y)=( -x+x\circ y,(-x+x\circ y)'\circ x\circ y)
	\]
	is an involutive solution \textcolor{red}{to the YBE}.
\end{proposition}

The proposition can be demonstrated using Theorem~\ref{thm:LYZ}, 
see Exercise~\ref{prob:Rump}. 
We will prove a stronger result in Theorem~\ref{thm:YB}. 

Note that if $(X,r)$ is an  involutive solution \textcolor{red}{to the YBE}, then 
\[
(x,y)=r^2(x,y)=r(\sigma_x(y),\tau_y(x))=(\sigma_{\sigma_x(y)}\tau_y(x),\tau_{\tau_y(x)}\sigma_x(y)).
\]
Hence 
\begin{equation}
	\label{eq:involutive}
	\tau_y(x)=\sigma_{\sigma_x(y)}^{-1}(x),
	\quad
	\sigma_x(y)=\tau_{\tau_y(x)}^{-1}(y)
\end{equation}
for all $x,y\in X$. Thus for involutive solutions
it is enough to know $\{\sigma_x:x\in X\}$, as from this we obtain the
set $\{\tau_x:x\in X\}$. 


\begin{proposition}
	\label{pro:T}
	Let $(X,r)$ be an involutive solution \textcolor{red}{to the YBE}. 
	Then the map $T\colon X\to X$, $x\mapsto\sigma_x^{-1}(x)$, is 
	invertible with inverse $T^{-1}(y)=\tau^{-1}_y(y)$ and 
	\[
	T^{-1}\sigma_x^{-1} T=\tau_x
	\]
	for all $x\in X$. 
\end{proposition}

\begin{proof}
	Let $U(x)=\tau_x^{-1}(x)$. Since $r$ is involutive, 
	\[
	(U(x),x)=r^2(U(x),x)=r(\sigma_{U(x)}(x),x)=(\sigma_{\sigma_{U(x)}(x)}(x),\tau_x\sigma_{U(x)}(x)).
	\]
	In particular, 
	$x=\tau_x\sigma_{U(x)}(x)$ and hence 
	$U(x)=\sigma_{U(x)}(x)$. This 
	implies that 
	\[
	T(U(x))=\sigma^{-1}_{U(x)}(U(x))=x.
	\]
	Similarly one obtains $U(T(x))=x$. 
	
	Since $(X,r)$ is a solution \textcolor{red}{to the YBE}, Lemma~\ref{lem:YB} implies that 
	$\sigma_x\sigma_y=\sigma_{\sigma_x(y)}\sigma_{\tau_y(x)}$
	%$\tau_{\tau_y(x)}\circ\tau_{\sigma_x(y)}=\tau_y\circ\tau_x$ 
	holds for all $x,y\in X$. Then  
	\[
	\sigma_y^{-1}T(x)
	=\sigma_y^{-1}\sigma_x^{-1}(x)
	=\sigma^{-1}_{\tau_y(x)}\sigma^{-1}_{\sigma_x(y)}(x)
	=\sigma^{-1}_{\tau_y(x)}\tau_y(x)
	=T\tau_y(x)
	\]
	for all $y\in X$, \textcolor{red}{by \eqref{eq:involutive}}.
\end{proof}



\begin{definition}
	\index{Cycle set}
	\index{Cycle set!non-degenerate}
	A \emph{cycle set} is a pair $(X,\cdot)$, where $X$ is a non-empty 
	set provided with a binary operation $X\times X\to X$, $(x,y)\mapsto x\cdot y$, 
	such that 
	\begin{equation}
		\label{eq:cycle_set}
		(x\cdot y)\cdot (x\cdot z)=(y\cdot x)\cdot (y\cdot z)
	\end{equation}
	holds for all $x,y,z\in X$ and each map $\varphi_x\colon X\to X$, $y\mapsto x\cdot y$, is bijective. 
	A cycle set $(X,\cdot)$ is said to be \emph{non-degenerate} 
	if the map $X\to X$, $x\mapsto x\cdot x$, is bijective. 
\end{definition}

\begin{definition}
	\index{Homomorphism!of cycle sets}
	\index{Isomorphism!of cycle sets}
	Let $X$ and $Z$ be cycle sets. 
	A \emph{homomorphism} between the cycle sets $X$ and $Z$ is a 
	map $f\colon X\to Z$ such that $f(x\cdot y)=f(x)\cdot f(y)$ for all $x,y\in X$. An \emph{isomorphism} of cycle sets
	is a bijective homomorphism of cycle sets. 
\end{definition}

%Cycle sets and cycle set homomorphisms form a category. 
%It is possible to prove that the category of 
%solutions is equivalent to the category of cycle sets, 
%see Exercise~\ref{prob:cycle_sets}. 

% \framebox{Exercise!}

% \begin{lemma}
% \label{lem:T_forCS}
% If $X$ is a cycle set, then $x\cdot (y\cdot y)=((y*x)\cdot y)\cdot ((y*x)\cdot y)$, where
% $y*x=z$ if and only if $y\cdot z=x$. 
% \end{lemma}

% \begin{proof}
%     Since the operation $x\mapsto y\cdot x$ is bijective, we can write $x=y\cdot (y*x)$. Then, using~\eqref{eq:cycle_set}, 
%     $x\cdot (y\cdot y)=(y\cdot (y*x))\cdot (y\cdot y)=((y*x)\cdot y)\cdot (y*x)\cdot y)$.
% \end{proof}

\begin{theorem}
	\label{thm:CS}
	There exists a bijective correspondence between non-isomorphic involutive solutions \textcolor{red}{to the YBE} 
	and non-isomorphic non-degenerate cycle sets. 
\end{theorem}



% \begin{proof}
%     Let us assume first that $r(x,y)=(\sigma_x(y),\tau_y(x))$ is an involutive solution. 
%     We want to prove that the operation
%     $x\cdot y=\sigma_x^{-1}(y)$ turns the set $X$ into a non-degenerate cycle set. 
%     It is clear that the maps $y\mapsto x\cdot y=\sigma_x^{-1}(y)$ are invertible. 
%     By Proposition~\ref{pro:T}, the operation $x\mapsto x\cdot x$ is bijective. \framebox{!}
%     Since $r(\tau^{-1}_y(x),y)=(\sigma_{\tau^{-1}_y(x)}(y),x)$, Lemma~\ref{lem:YB} implies that
%     \[
%     \tau_x\circ\tau_{\sigma_{\tau^{-1}_y(x)}(y)}=\tau_y\circ\tau_{\tau^{-1}_y(x)}.
%     \]
%     Moreover, since $r(\sigma_{\tau^{-1}_y(x)}(y),x)=(\tau^{-1}_y(x),y)$, it follows that
%     %$\tau_x\sigma_{\tau^{-1}_y(x)}(y)=y$, i.e.
%     \[
%     \sigma_{\tau^{-1}_y(x)}(y)=\tau_x^{-1}(y).
%     \]
%     % Since $r^2=\id_{X\times X}$, it follows that
%     % \[
%     % (\tau^{-1}_y(x),y)=r^2(\tau^{-1}_y(x),y)=r(\sigma_{\tau^{-1}_y(x)}(y),x)
%     % \]
%     % and hence $\sigma_{\tau^{-1}_y(x)}(y)=\tau^{-1}_x(y)$. 
%     This implies that
%     \[
%     \tau_x\circ\tau_{\tau^{-1}_x(y)}=\tau_x\circ\tau_{\sigma_{\tau^{-1}_y(x)}(y)}=\tau_y\circ\tau_{\tau^{-1}_y(x)},
%     \]
%     which turns out to be equivalent to
%     \[
%     \tau^{-1}_{\tau^{-1}_x(y)}\circ\tau^{-1}_x=\tau^{-1}_{\tau^{-1}_y(x)}\circ\tau^{-1}_y
%     \]
%     and hence equivalent to Equality~\eqref{eq:cycle_set}. 

%     Conversely, if $X$ is a non-degenerate cycle set, we want to prove that
%     \[
%     r(x,y)=((y*x)\cdot y,y*x),
%     \]
%     where $y*x=z$ if and only if $y\cdot z=x$, is an involutive solution. The definition of $r$ implies that
%     $r^2=\id_{X\times X}$. 
%     By Lemma~\ref{lem:YB}, we need to prove that
%     \begin{align*}
%         &\sigma_x\circ\sigma_y = \sigma_{\sigma_x(y)}\circ\sigma_{\tau_y(x)},&
%         &\sigma_{\tau_{\sigma_y(z)}(x)}\tau_z(y)=\tau_{\sigma_{\tau_y(x)}(z)}\sigma_x(y),&
%         &\tau_z\circ\tau_y=\tau_{\tau_z(y)}\circ\tau_{\sigma_y(z)}
%     \end{align*}

%     Write $\sigma_x(y)=(y*x)\cdot y$ and $\tau_y(x)=y*x$. 
%     We know that the third formula is equivalent to~\eqref{eq:cycle_set}. 
%     Let $T\colon X\to X$, $x\mapsto x\cdot x$. By assumption, $T$ is bijective and hence 
%     $T^{-1}\circ \tau^{-1}_x\circ T=\sigma_x$ for all $x\in X$, since by Lemma~\ref{lem:T_forCS}
%         \begin{align*}
%     \tau_x^{-1}T(y)&=\tau_x^{-1}(y\cdot y)=x\cdot (y\cdot y)\\
%     &=((y*x)\cdot y)\cdot ((y*x)\cdot y)=T((y*x)\cdot y)=T\sigma_x(y)
%     \end{align*}
%     for all $x,y\in X$. In particular, $(X,r)$ is non-degenerate. 
%     Moreover, 
%     \begin{align*}
%     \sigma_x\circ\sigma_y
%     &=(T^{-1}\circ\tau_x^{-1}\circ T)\circ (T^{-1}\circ\tau_y^{-1}\circ T)\\
%     &=T^{-1}\circ (\tau_y\circ\tau_x)^{-1}\circ T\\
%     &=T^{-1}\circ (\tau_{\tau_y(x)}\circ\tau_{\sigma_x(y)})^{-1}\circ T\\
%     &=(T^{-1}\circ\tau^{-1}_{\sigma_x(y)}\circ T)\circ (T^{-1}\circ\tau^{-1}_{\tau_y(x)}\circ T)\\
%     &=\sigma_{\sigma_x(y)}\circ\sigma_{\tau_y(x)}
% \end{align*}
% for all $x,y\in X$. Finally, Equality~\eqref{eq:cycle_set} can be written as
% \[
% \tau^{-1}_{\tau^{-1}_v(u)}\circ\tau^{-1}_v=\tau^{-1}_{\tau^{-1}_u(v)}\circ\tau^{-1}_u.
% \]
% With $u=\tau_z\tau_y(x)$ and $v=z$, it becomes
% \[
% \tau^{-1}_{\tau_y(x)}\circ\tau^{-1}_z
% =\tau^{-1}_{\tau^{-1}_{\tau_z\tau_y(x)}(z)}\circ\tau^{-1}_{\tau_z\tau_y(x)}
% =\tau^{-1}_{\sigma_{\tau_y(x)}(z)}\circ\tau^{-1}_{\tau_z\tau_y(x)},
% \]
% where the last equality holds $r^2=\id_{X\times X}$. 
% Using Lemma~\ref{lem:YB} one rewrites this formula as
% \[
% \tau_{\sigma_{\tau_y(x)}(z)}\circ\tau^{-1}_{\tau_y(x)}
% =\tau^{-1}_{\tau_z\tau_y(x)}\circ\tau_z
% =\tau^{-1}_{\tau_{\tau_z(y)}\tau_{\sigma_y(z)}(x)}\circ\tau_z.
% \]
% Evaluating this equality at $y$, 
% \[
% \tau_{\sigma_{\tau_y(x)}(z)}\sigma_x(y)=\tau_{\sigma_{\tau_y(x)}(z)}\tau^{-1}_{\tau_y(x)}(y)
% =\tau^{-1}_{\tau_{\tau_z(y)}\tau_{\sigma_y(z)}(x)}\tau_z(y)
% =\sigma_{\tau_{\sigma_y(z)}(x)}\tau_z(y).\qedhere
% \]
% \end{proof}

For the readers who are not familiar with the above-mentioned result, 
the bijective correspondence is given by 
\[
r(x,y)=(x*y,(x*y)\cdot x),
\]
\textcolor{red}{where} $x*y=\varphi_x^{-1}(y)$. We leave the proof for the reader, see 
Exercise~\ref{prob:CS}. However, we will prove a more 
general result in Theorem~\ref{thm:skewCS}. 

Theorem~\ref{thm:CS} can be used to construct and enumerate small 
involutive solutions \textcolor{red}{to the YBE}~\cite{MR4405502}. Table~\ref{tab:IYB} shows the 
number of non-isomorphic involutive solutions of size $\leq10$. 
For size $\leq7$ the numbers of Table~\ref{tab:IYB} coincide with those in~\cite{MR1722951}
but differ by two for $n=8$, as two solutions of size eight 
are missing in~\cite{MR1722951}. 

\begin{table}[H]
	\centering
	\caption{Involutive solutions of size $\leq10$.}
	\begin{tabular}{|r|ccccccccc|}
		\hline
		$n$ & 2 & 3 & 4 & 5 & 6 & 7 & 8 & 9 & 10\tabularnewline
		\hline 
		solutions & 2 & 5 & 23 & 88 & 595 & 3456 & 34530 & 321931 & 4895272\tabularnewline
		% square-free & 1 & 2 & 5 & 17 & 68 & 336 & 2041 & \cellcolor{gray!30}{15534} & \cellcolor{gray!30}{150957}\tabularnewline
		% indecomposable & 1 & 1 & 5 & 1 & 10 & 1 & \cellcolor{gray!30}{100} & \cellcolor{gray!30}{16} & \cellcolor{gray!30}{36}\tabularnewline
		% multipermutation & 2 & 5 & 21 & 84 & 554 & 3295 & \cellcolor{gray!30}{32155} & \cellcolor{gray!30}{305916} & \cellcolor{gray!30}{4606440}\tabularnewline
		% irretractable & 0 & 0 & 2 & 4 & 9 & 13 & 191 & \cellcolor{gray!30}{685} & \cellcolor{gray!30}{3590}\tabularnewline
		\hline
	\end{tabular}
	\label{tab:IYB}
\end{table}



% \begin{definition}
% \index{Solution!permutation group}
% The \emph{permutation group} of a solution $(X,r)$ is the group
% \[
% \mathcal{G}(X,r)=\langle (\sigma_x,\widehat{\sigma_x}):x\in X\rangle\subseteq\Sym_X\times\Sym_X.
% \]
% \end{definition}

% \index{Solutions!linear involutive}
% Let us finish the chapter with the family of affine involutive solutions. Let $A$ be an abelian group and
% $a,b,c,d\in\colon A\to A$ be group homomorphisms. If 
% \[
% r(x,y)=(a(x)+b(y),c(x)+d(y))
% \]
% is bijective and $(A,r)$ is an solution involutive solution, then $(A,r)$ is an \emph{linear solution}.
% Note that $r$ satisfies the YBE if and only if
% \begin{subequations}
% \begin{gather}
% \label{eq:linear}
% (\id-a)\circ d\circ b=b\circ d,
% \quad
% d\circ(\id-d)=c\circ d\circ b,\\
% c\circ d\circ (\id-a)=d\circ c,
% \quad
% a\circ(\id-a)=b\circ a\circ c,
% \quad
% c\circ a=(\id-d)\circ a\circ c,\\
% a\circ b=b\circ a\circ (\id-d),
% \quad
% c\circ b-b\circ c=a\circ d\circ a-d\circ d.
% \end{gather}
% \end{subequations}
% Moreover, $r^2=\id$ if and only if
% \begin{align}
%     &a\circ a+b\circ c=\id,
%     &&
%     c\circ b+d\circ d=\id,
%     &&
%     a\circ b+b\circ d=0,
%     &&
%     c\circ a+d\circ c=0.
% \end{align}
% \end{equation}

%\begin{prob}
%	\label{prob:cycle_sets}
%	Prove that the category of non-degenerate 
%	cycle sets and the category of solutions are equivalent. 
%\end{prob}

\section{Racks}

\begin{definition}
\label{defn:rack}
\index{Rack}
A \emph{rack} is a pair $(X,\triangleleft)$, 
where $X$ is a non-empty set and 
$X\times X\to X$, $(x,y)\mapsto x\triangleleft y$, is a binary operation on $X$ such that
the maps $\rho_y\colon X\to X$, $x\mapsto x\triangleleft y$, are bijective for all $y\in X$, and 
\begin{equation}
\label{eq:rack}
(x\triangleleft y)\triangleleft z=(x\triangleleft z)\triangleleft (y\triangleleft z)
\end{equation}
for all $x,y,z\in X$.
\end{definition}

Racks are used in low-dimensional topology~\cite{MR3379534}, singularities~\cite{MR975077} 
and in the classification of finite-dimensional pointed Hopf algebras~\cite{MR1994219}.

\begin{example}
\index{Rack!trivial}
    Let $X$ be a set. Then $x\triangleleft y=x$ turns $X$ into a rack. 
    This is the \emph{trivial rack} on $X$. 
\end{example}

\begin{example}
    \index{Rack!dihedral}
    Let $X=\Z/n$. Then $x\triangleleft y=2y-x$ turns $X$ into a rack. This is 
    the \emph{dihedral rack} of size $n$. 
\end{example}

\begin{example}
    \index{Rack!Alexander}
    Let $A$ be an abelian group and $f\in\Aut(A)$. Then 
    \[
    x\triangleleft y=(\id-f)(y)+f(x)
    \]
    turns $A$ into a rack. These racks 
    are known as the \emph{Alexander racks}.
\end{example}

\begin{definition}
    \index{Rack!homomorphism}
    \index{Rack!isomorphism}
    \index{Homomorphism!of racks}
    Let $X$ and $Z$ be racks. 
    A \emph{rack homomorphism} between the racks $X$ and $Z$ is a map $f\colon X\to Z$ such that 
    $f(x\triangleleft y)=f(x)\triangleleft f(y)$ for all $x,y\in X$. 
    An \emph{isomorphism} of racks is a bijective rack homomorphism. 
\end{definition}

For $n\in\N$, let $r(n)$ be the number of isomorphism classes of racks of size
$n$. Some values of $r(n)$ appear 
in Table~\ref{tab:racks}, see for example~\cite{MR3957904}.  

\begin{table}[H]
\centering
\caption{Enumeration of non-isomorphic racks.}
\begin{tabular}{|c|cccccccccccc|}
\hline
$n$ & 2 & 3 & 4 & 5 & 6 & 7 & 8 & 9 & 10 & 11 & 12 & 13\tabularnewline
\hline
$r(n)$ & 2 & 6 & 19 &74&353 & 2080 & 16023 & 159526 & 2093244 & 36265070 & 836395102 & 25794670618\tabularnewline
\hline
\end{tabular}
\label{tab:racks}
\end{table}

\begin{proposition}
\label{pro:Venkov}
Let $X$ be a non-empty set and $X\times X\to X$, $(x,y)\mapsto x\triangleleft y$ be a binary operation on $X$. Then
$r(x,y)=(y,x\triangleleft y)$ is a solution \textcolor{red}{to the YBE} if and only if $(X,\triangleleft)$ is a rack. 
\end{proposition}

\begin{proof}
The map $r$ satisfies $(r\times\id_X)(\id_X\times r)(r\times\id_X)=(\id_X\times r)(r\times\id_X)(\id_X\times r)$ if and only
if~\eqref{eq:rack} holds for all $x,y,z\in X$. \textcolor{red}{The map $r$ is bijective if and only if} 
the maps $X\to X$, $x\mapsto x\triangleleft y$, are bijective. \textcolor{red}{Hence the result follows.} 
\end{proof}

The connection between racks and solutions goes deeper than 
the phenomenon appearing in Proposition~\ref{pro:Venkov}. 


\begin{proposition}
    \label{pro:derived}
    Let $(X,r)$ be a solution \textcolor{red}{to the YBE}. Then 
    \begin{equation}
    \label{eq:derived}
    x\triangleleft y=\sigma_y\tau_{\sigma_x^{-1}(y)}(x)=\sigma_y\widehat{\sigma}^{-1}_y(x)
    \end{equation}
    turns $X$ into a rack and each $\sigma_x$ is a rack homomorphism. 
    Moreover, $(X,r)$ is involutive if and only if the rack $(X,\triangleleft)$ is trivial. 
\end{proposition}

\begin{proof}
    Since $r(x,\sigma_x^{-1}(y))=(y,\tau_{\sigma_x^{-1}(y)}(x))$, 
    it follows that 
    $\widehat{\sigma}_y^{-1}(x)=\tau_{\sigma_x^{-1}(y)}(x)$ 
    for all $x,y\in X$. Hence the second equality of~\eqref{eq:derived} holds 
    and the maps $X\to X,\, x\mapsto x\triangleleft y$ are bijective.
    
    Now we show that
    \begin{equation}\label{eq:rackhom}
    \sigma_x(y)\triangleleft \sigma_x\sigma_y(z)=\sigma_x(y\triangleleft\sigma_y(z))
    \end{equation}
    for all $x,y\in X$. Write $r(x,y)=(u,v)$. On the one hand, by 
    Lemma~\ref{lem:YB}, 
    \begin{align*}
        \sigma_x(y)\triangleleft \sigma_x\sigma_y(z)
        &=u\triangleleft \sigma_u\sigma_v(z)\\
        &=\sigma_{\sigma_u\sigma_v(z)}\tau_{\sigma_{\tau_y(x)}(z)}\sigma_x(y)
        =\sigma_{\sigma_x\sigma_y(z)}\sigma_{\tau_{\sigma_y(z)}(x)}\tau_z(y).
    \end{align*}
    On the other hand, 
    \begin{align*}
        \sigma_x(y\triangleleft\sigma_y(z)) 
        &=\sigma_x\sigma_{\sigma_y(z)}\tau_z(y)
        =\sigma_{\sigma_x\sigma_y(z)}\sigma_{\tau_{\sigma_y(z)}(x)}\tau_z(y).
    \end{align*}
    Therefore \eqref{eq:rackhom} follows.
    
    By Proposition~\ref{pro:Venkov}, 
    to prove that $(X,\triangleleft)$ is a rack it is enough to show that
    $s(x,y)=(y,x\triangleleft y)$ satisfies the YBE. For that purpose, we demonstrate that 
    the map $J\colon X^3\to X^3$, $J(x,y,z)=(x,\sigma_x(y),\sigma_x\sigma_y(z))$ 
    is invertible and 
    satisfies 
    \[
    (\id_X\times s) J=J(\id_X\times r),
    \quad
    (s\times\id_X) J=J(r\times\id_X).
    \]
    One checks that the map 
    $(x,y,z)\mapsto (x,\sigma_x^{-1}(y),\sigma_{\sigma_x^{-1}(y)}^{-1}\sigma_x^{-1}(z))$ 
    is the inverse of $J$. 

    By \eqref{eq:rackhom}, 
    \begin{align*}
    \sigma_x(y)\triangleleft \sigma_x\sigma_y(z)
    &=\sigma_x(y\triangleleft \sigma_y(z))
    =\sigma_x\sigma_{\sigma_y(z)}\tau_{\sigma_y^{-1}\sigma_y(z)}(y)
    =\sigma_x\sigma_{\sigma_y(z)}\tau_z(y)
    \end{align*}
    Then it follows that 
    \begin{align*}
        (\id_X\times s)J(x,y,z)
        &=(\id_X\times s)(x,\sigma_x(y),\sigma_x\sigma_y(z))\\
        &=(x,\sigma_x\sigma_y(z),\sigma_x(y)\triangleleft \sigma_x\sigma_y(z))\\
        &=(x,\sigma_x\sigma_y(z),\sigma_x\sigma_{\sigma_y(z)}\tau_z(y))\\
        &=J(x,\sigma_y(z),\tau_z(y))\\
        &=J(\id_X\times r)(x,y,z).
    \end{align*}
    Similarly one proves that $(s\times\id_X)J=J(r\times\id_X)$.  
    This implies that $(X,s)$ is a solution \textcolor{red}{to the YBE} and 
    hence $(X,\triangleleft)$ is a rack by Proposition~\ref{pro:Venkov}. 

    If $(X,r)$ is involutive, 
    then $x\triangleleft\sigma_x(y)=\sigma_{\sigma_x(y)}\tau_y(x)=x$ by~\eqref{eq:involutive}. 
    Conversely, if $x\triangleleft y=x$ for all $x,y\in X$,
    then 
    \textcolor{red}{$\sigma^{-1}_y(x)=\widehat{\sigma}^{-1}_y(x)$ for all $x,y\in X$ and}
    $r$ is involutive, as 
    \[
    r^2(x,\sigma_x^{-1}(y))=r(y,\sigma_y^{-1}(x))=(x,\sigma_x^{-1}(y)).\qedhere
    \]
\end{proof}

\begin{definition}
\index{Solution!derived rack}
\index{Solution!derived}
The rack constructed in Proposition~\ref{pro:derived} is 
known as the \emph{derived rack} of $(X,r)$. 
\textcolor{red}{The solution $(X,s)$ to the YBE, defined by $s(x,y)=(y,\sigma_y\widehat{\sigma}^{-1}_y(x))$ for all $x,y\in X$, is called the {\em derived solution} of $(X,r)$.}  
\end{definition}

There is a dual version of the derived rack:

\begin{proposition}
    \label{pro:derived_dual}
    Let $(X,r)$ be a solution \textcolor{red}{to the YBE}. Then 
    \[
    x\blacktriangleleft y=\tau_y\sigma_{\tau_x^{-1}(y)}(x)=\tau_y\widehat{\tau}^{-1}_y(x)
    \]
    turns $X$ into a rack and each $\tau_x$ is a rack homomorphism. 
\end{proposition}

\begin{proof}
    Since $(X,r)$ is a solution, then so is $(X,r_0)$, where 
    $r_0(x,y)=(\tau_x(y),\sigma_y(x))$. Then the claim follows 
    from Proposition~\ref{pro:derived} applied to 
    the solution $(X,r_0)$.
\end{proof}

\begin{definition}
\index{Solution!dual derived rack}
\index{Solution!dual derived}
The rack constructed in Proposition~\ref{pro:derived_dual} is 
known as the \emph{dual derived rack} of $(X,r)$. \textcolor{red}{The solution $(X,s')$ to the YBE, defined by $s'(x,y)=(\tau_x\widehat{\tau}^{-1}_x(y),x)$ for all $x,y\in X$, is called the {\em dual derived solution} of $(X,r)$.}  

\end{definition}

In general, the racks constructed in Propositions~\ref{pro:derived} 
and~\ref{pro:derived_dual} are different:

\begin{example}
Let $X=\{1,\dots,5\}$ and $(X,r)$ be the solution \textcolor{red}{to the YBE} given by
\begin{align*}
&\sigma_1=\id, && \sigma_2=\id, && \sigma_3=\id, && \sigma_4=(13)(45), &&\sigma_5=(12)(45),\\
&\tau_1=\id, && \tau_2=\id, && \tau_3=\id, && \tau_4=(23)(45), &&\tau_5=(23)(45).
\end{align*}
On the one hand the derived rack of $(X,r)$ is given by the permutations 
\begin{align*}
&\sigma_1\widehat{\sigma}_1^{-1}=\sigma_2\widehat{\sigma}_2^{-1}=\sigma_3\widehat{\sigma}_3^{-1}=\id,
&&\sigma_4\widehat{\sigma}_4^{-1}=(132),
&&\sigma_5\widehat{\sigma}_5^{-1}=(123).
\end{align*}
On the other hand, the dual derived rack by 
\begin{align*}
&\tau_1\widehat{\tau}_1^{-1}=\tau_2\widehat{\tau}_2^{-1}=\tau_3\widehat{\tau}_3^{-1}=\id,
&&\tau_4\widehat{\tau}_4^{-1}=(123),
&&\tau_5\widehat{\tau}_5^{-1}=(132).
\end{align*}
% [ [ (), (), (), (1,3)(4,5), (1,2)(4,5) ], [ (), (), (), (2,3)(4,5), (2,3)(4,5) ] ]
%[ [ (), (), (), (2,3)(4,5), (2,3)(4,5) ], [ (), (), (), (1,3)(4,5), (1,2)(4,5) ] ]
% [ (), (), (), (1,3,2), (1,2,3) ]
% [ (), (), (), (1,2,3), (1,3,2) ]
\end{example}

We now prove that the racks 
of Propositions~\ref{pro:derived} 
and~\ref{pro:derived_dual} are isomorphic. 
We shall need a lemma. 



\begin{lemma}
\label{lem:T_invertible}
Let $(X,r)$ be a solution \textcolor{red}{to the YBE}. 
The map $T\colon X\to X$, $x\mapsto\sigma_x^{-1}(x)$, is invertible with
inverse $U\colon X\to X$, $x\mapsto\tau^{-1}_x(x\blacktriangleleft x)$. 
\end{lemma}

\begin{proof}
Let $x\in X$ and $y=U(x)=\tau^{-1}_x(x\blacktriangleleft x)$. 
Then $\tau_x(y)=x\blacktriangleleft x=\tau_x\widehat{\tau}^{-1}_x(x)$ and hence
$y=\widehat{\tau}^{-1}_x(x)$. Then $\widehat{\tau}_x(y)=x$ and 
\[
r^{-1}(y,x)=(\widehat{\sigma}_y(x),x)=(z,x),
\]
where $z\in X$ is such that $\sigma_z(x)=y$. By Lemma~\ref{lem:YB}, $\sigma_y=\sigma_z$. Then 
it follows that $x=\sigma^{-1}_y(y)=T(y)$. Therefore $y=U(x)=U(T(y))$.

To prove that $T(U(x))=x$, first note that 
\[
r(\tau^{-1}_x(x),x)=(\sigma_{\tau^{-1}_x(x)}(x),x)
\]
and Lemma~\ref{lem:YB} imply that $\sigma_{\tau^{-1}_x(x)}=\sigma_{\sigma_{\tau^{-1}_x(x)}(x)}$. Now
\begin{align*}
T(U(x))&=T(\tau^{-1}_x(x\blacktriangleleft x))=T(\sigma_{\tau^{-1}_x(x)}(x))\\
&=\sigma^{-1}_{\sigma_{\tau^{-1}_x(x)}(x)}\sigma_{\tau^{-1}_x(x)}(x)
=\sigma^{-1}_{\tau^{-1}_x(x)}\sigma_{\tau^{-1}_x(x)}(x)=x.\qedhere
\end{align*}
\end{proof}

There is a version of Proposition~\ref{pro:T} for arbitrary solutions. 
A similar result appears in Exercise~\ref{prob:variationT}.

\begin{proposition}
    Let $(X,r)$ be a solution \textcolor{red}{to the YBE}. Then $T\colon X\to X$, $x\mapsto\sigma_x^{-1}(x)$, is a bijective
    map such that 
    \[
      T\tau_y=\widehat{\sigma}^{-1}_y T,
      \quad
      T \widehat{\tau_y}=\sigma^{-1}_{y} T
    \]
    and $T(x\blacktriangleleft y)=T(x)\triangleleft T(y)$ for all $x,y\in X$. 
\end{proposition}

\begin{proof}
    Lemma~\ref{lem:T_invertible} proves that $T$ is bijective. 
    We now compute 
    \begin{align*}
        T\tau_y(x) &= 
        \sigma^{-1}_{\tau_y(x)}\tau_y(x)
        =\sigma^{-1}_{\tau_y(x)}\sigma^{-1}_{\sigma_x(y)}\sigma_{\sigma_x(y)}\tau_y(x)\\
        &=\sigma^{-1}_y\sigma^{-1}_x\sigma_{\sigma_x(y)}\tau_y(x)
        =\sigma^{-1}_y\sigma_x^{-1}(x\triangleleft\sigma_x(y))
        =\sigma^{-1}_y(T(x)\triangleleft y)
        =\widehat{\sigma}^{-1}_yT(x).
    \end{align*}
    Since $\widehat{\tau}_y(x)=\sigma^{-1}_{\widehat{\sigma}_x(y)}(x)$, Lemma~\ref{lem:YB} implies that 
    \begin{align*}
        T\widehat{\tau_y}(x) 
        &=\sigma^{-1}_{\widehat{\tau}_y(x)}\widehat{\tau}_y(x)
        =\sigma^{-1}_{\widehat{\tau}_y(x)}\sigma^{-1}_{\widehat{\sigma}_x(y)}(x)
        =\sigma^{-1}_{y}\sigma^{-1}_{x}(x)
        =\sigma^{-1}_{y}T(x).
    \end{align*}
    These formulas imply that
    \begin{equation}
        \label{eq:T_rack}
        T\tau_y\widehat{\tau}_y^{-1}
        =\widehat{\sigma}^{-1}_y T \widehat{\tau}^{-1}_y
        =\widehat{\sigma}^{-1}_y \sigma_y T.
    \end{equation}
    We evaluate Equality~\eqref{eq:T_rack} on $X$. On the one hand, 
    $T(x\blacktriangleleft y)=T\tau_y\widehat{\tau_y}^{-1}(x)$.
    On the other hand,
    \begin{align*}
        \widehat{\sigma}_y^{-1}\sigma_yT(x)
        &=\sigma_y^{-1}\sigma_y\widehat{\sigma}_y^{-1}\sigma_yT(x)
        =\sigma_y^{-1}(\sigma_yT(x)\triangleleft y)=T(x)\triangleleft T(y).\qedhere
    \end{align*}
\end{proof}


%If $(X,r)$ is a solution, the map $r\colon X\times X\to X\times X$ is invertible. Write 
% \[
% r^{-1}(x,y)=(\widehat{\sigma}_x(y),\widehat{\tau}_y(x)).
% \]
% Since $(X,r^{-1})$ is a solution, Lemma~\ref{lem:YB} implies that 
% the following formulas hold:
% \[
% \widehat{\tau}_y\widehat{\tau_x}=\widehat{\tau}_{\tau_y(x)}\widehat{\tau}_{\sigma_x(y)},
% \quad
% \widehat{\sigma}_x\widehat{\sigma_y}=\widehat{\sigma}_{\sigma_x(y)}\widehat{\sigma}_{\tau_y(x)}.
% \]
% Since $r(\tau^{-1}_y(x),y)=(\sigma_{\tau^{-1}_y(x)}(y),x)$, 
% it follows that 
% \[
% \widehat{\tau}_x\sigma_{\tau^{-1}_y(x)}(y)=y.
% \]
% for all $x,y\in X$. Moreover, 
% \[
% x\triangleright y=\tau_x\sigma_{\tau^{-1}_y(x)}(y)=\tau_x\widehat{\tau}^{-1}_x(y)
% \]
% for all $x,y\in X$. 

%We first need a lemma. 

% \begin{proposition}
%     Let $(X,r)$ be a solution and let $(X,\triangleright)$ be its derived rack. 
%     Then 
%     \[
%     T\circ \sigma_x=\tau_x^{-1}\circ\rho_x\circ T
%     \]
%     for all $x\in X$, where $T\colon X\to X$, $T(y)=\tau_y^{-1}(y)$, and 
%     $\rho_x\colon X\to X$, $\rho_x(y)=x\triangleright y$.
% \end{proposition}

% \begin{proof}
% By using Lemma~\ref{lem:YB} we compute 
% \begin{align*}
%     T\sigma_x(y) &= \tau^{-1}_{\sigma_x(y)}\sigma_x(y)
%     =\tau^{-1}_{\sigma_x(y)}(\tau^{-1}_{\tau_y(x)}(\tau_y(x)\triangleright y))\\
%     &=\tau^{-1}_{\sigma_x(y)}\tau^{-1}_{\tau_y(x)}(\tau_y(x)\triangleright y))\\
%     &=\tau^{-1}_{x}\tau^{-1}_{y}(\tau_y(x)\triangleright y))\\
%     &=\tau^{-1}_{x}(x\triangleright T(y)).\qedhere 
% \end{align*}
% \end{proof}

% \begin{proposition}
% \label{pro:T_general}
% Let $(X,r)$ be a solution. 
% The map $T\colon X\to X$, $x\mapsto\sigma_{\tau_x^{-1}(x)}(x)$, is invertible with
% inverse $U\colon X\to X$, $x\mapsto\sigma^{-1}_x(x)$. 
% \end{proposition}

% \begin{proof}
% Let $(X,\triangleright)$ be the derived rack of $(X,r)$. Then 
% $T(x)=\tau_x^{-1}(x\triangleright x)$. Since 
% $r(\tau_z^{-1}(z),z)=(\sigma_{\tau^{-1}_z(z)}(z),z)$, 
% \begin{equation}
%     \label{eq:Ttrick}
%     \sigma_{\tau^{-1}_z(z)}\circ\sigma_z=\sigma_{\sigma_{\tau^{-1}_z(z)}(z)}\circ\sigma_z
% \end{equation}
% holds for all $z\in X$ by 
% Lemma~\ref{lem:YB}. On the one hand, 
% \begin{align*}
%     U(T(x)) 
%     &= U(\sigma_{\tau^{-1}_x(x)}(x))
%     =\sigma^{-1}_{\sigma_{\tau^{-1}_x(x)}(x)}\sigma_{\tau^{-1}_x(x)}(x)=x
% \end{align*}
% because $\sigma_{\tau^{-1}_x(x)}=\sigma_{\sigma_{\tau^{-1}_x(x)}(x)}$ by~\eqref{eq:Ttrick}. 

% Now let $x,y\in X$ be such that $U(x)=y$. Since $r^{-1}(\sigma_x(y),\tau_y(x))=(x,y)$, 
% it follows that $\widehat{\tau}_{\tau_y(x)}\sigma_x(y)=y$ and that 
% $\widehat{\tau}_{\tau_y(x)}=\widehat{\tau_y}$. Then
% \[
% \widehat{\tau}_y(x)=\widehat{\tau}_{\tau_y(x)}(x)=\widehat{\tau}_{\tau_y(x)}\sigma_x(y)=y
% \]
% and hence $\widehat{\tau}^{-1}_y(y)=x$. Now $y\triangleright y=\tau_y\widehat{\tau}^{-1}_y(y)=\tau_y(x)$ and therefore
% \[
% T(y)=\tau_y^{-1}(y\triangleright y)=x.\qedhere
% \]
% \end{proof}

As it happens in the involutive case, there is a nice combinatorial structure that describes  
a solution \textcolor{red}{to the YBE}. 

\begin{definition}
\index{Skew cycle set}
\index{Skew cycle set!non-degenerate}
\label{defn:skewCS}
A \emph{skew cycle set} is a triple $(X,\triangleleft,\cdot)$, 
where $X$ is a non-empty set, 
$(X,\triangleleft)$ is a rack and
$X\times X\to X$, $(x,y)\mapsto x\cdot y$, is a binary operation on $X$ such that the maps
$X\to X$, $y\mapsto x\cdot y$, are rack isomorphisms, and 
\begin{equation}
\label{eq:skew_CS}
(x\cdot y)\cdot (x\cdot z)=(y\cdot (x\triangleleft y))\cdot (y\cdot z)
\end{equation}
for all $x,y,z\in X$. A skew cycle set $(X,\triangleleft,\cdot)$ is said to be 
\emph{non-degenerate} if the map $X\to X$, $x\mapsto x\cdot x$, is bijective.
\end{definition}

If $(X,\triangleleft,\cdot)$ is a skew cycle set 
and $(X,\triangleleft)$ is a trivial rack, then $(X,\cdot)$ is a cycle set.

\begin{definition}
\index{Homomorphism!of skew cycle sets}
\index{Isomorphism!of skew cycle sets}
Let $X$ and $Z$ be skew cycle sets. 
A \emph{homomorphism} between the skew cycle sets $X$ and $Z$ is a 
map $f\colon X\to Z$ such that $f(x\cdot y)=f(x)\cdot f(y)$ and 
$f(x\triangleleft y)=f(x)\triangleleft f(y)$ for all $x,y\in X$. 
An \emph{isomorphism} of skew cycle sets
is a bijective homomorphism of skew cycle sets. 
\end{definition}

%Cycle sets and cycle set homomorphisms form a category. 
%It is possible to prove that the category of 
%solutions is equivalent to the category of cycle sets, 
%see Exercise~\ref{prob:cycle_sets}. 


Theorem~\ref{thm:CS} can be generalized to arbitrary solutions.

\begin{theorem}
\label{thm:skewCS}
There exists a bijective correspondence between solutions \textcolor{red}{to the YBE}
and non-degenerate skew cycle sets. 
\end{theorem}

\begin{proof}
Let $(X,r)$ be a solution \textcolor{red}{to the YBE} and $(X,\triangleleft)$ be its derived rack. We will prove that 
the operation $x\cdot y=\sigma_x^{-1}(y)$ turns $(X,\triangleleft)$ into a skew cycle set.
By Proposition~\ref{pro:derived}, 
the 
maps $X\to X$, $y\mapsto x\cdot y$, are bijective rack homomorphisms. 

On the one hand, since $r(x,\sigma_x^{-1}(y))=(y,\tau_{\sigma_x^{-1}(y)}(x))$, 
\begin{align*}
(x\cdot y)\cdot (x\cdot z)
&=\sigma_x^{-1}(y)\cdot\sigma_x^{-1}(z)
=\sigma_{\sigma_x^{-1}(y)}^{-1}\sigma_x^{-1}(z)\\
&=\left(\sigma_x\sigma_{\sigma_x^{-1}(y)}\right)^{-1}(z)
=\left(\sigma_y\sigma_{\tau_{\sigma_x^{-1}(y)}(x)}\right)^{-1}(z).
\end{align*}
On the other hand, 
\begin{align*}
(y\cdot (x\triangleleft y))\cdot (y\cdot z)
&=\sigma_y^{-1}(\sigma_y\tau_{\sigma_x^{-1}(y)}(x))\cdot\sigma_y^{-1}(z)\\
&=\sigma^{-1}_{\tau_{\sigma_x^{-1}(y)}(x)}\sigma_y^{-1}(z)
=\left(\sigma_y\sigma_{\tau_{\sigma^{-1}_x(y)}(x)}\right)^{-1}(z).
\end{align*}
Therefore $(X,\triangleleft ,\cdot)$ is a skew cycle set. 
Furthermore, by Lemma \ref{lem:T_invertible}, this skew cycle set is non-degenerate.

Now we prove the converse statement. 
Let $(X,\triangleleft,\cdot)$ be a non-degenerate skew cycle set.
For $x,y\in X$ let 
\[
\sigma_x(y)=x*y,
\quad
\tau_y(x)=\sigma_{\sigma_x(y)}^{-1}(x\triangleleft\sigma_x(y)), 
\]
where $x*y=z$ if and only if $x\cdot z=y$. 
Since $X$ is a skew cycle set, each $\sigma_x$ is bijective. Let us prove that the $\tau_x$ are bijective. 
Equality~\eqref{eq:skew_CS} with $y=\sigma_x(z)$ implies that
\begin{equation}
\begin{aligned}
\label{eq:solskew}
\sigma_z^{-1}\sigma_x^{-1}
&=\sigma^{-1}_{\sigma_x^{-1}(y)}\sigma_x^{-1}\\
&=\sigma^{-1}_{\sigma_y^{-1}(x\triangleleft y)}\sigma_y^{-1}
=\sigma^{-1}_{\sigma^{-1}_{\sigma_x(z)}(x\triangleleft\sigma_x(z))}\sigma^{-1}_{\sigma_x(z)}
=\sigma^{-1}_{\tau_z(x)}\sigma^{-1}_{\sigma_x(z)}
\end{aligned}    
\end{equation}
for all $x,z\in X$. 
Since each $\sigma_x$ is a rack homomorphism
and $\sigma^{-1}_{\sigma_x(y)}\sigma_x=\sigma_{\tau_y(x)}\sigma_y^{-1}$ holds for all $x,y\in X$, it follows that
\[
\tau_y(x)=\sigma^{-1}_{\sigma_x(y)}(x\triangleleft\sigma_x(y))
=\sigma^{-1}_{\sigma_x(y)}\sigma_x(\sigma^{-1}_x(x)\triangleleft y)
=\sigma_{\tau_y(x)}\sigma_y^{-1}(\sigma_x^{-1}(x)\triangleleft y).
\]
holds for all $x,y\in X$. 
Therefore $T\tau_y=\sigma_y^{-1}\rho_y T$, where $T\colon X\to X$, $T(x)=x\cdot x$ 
and $\rho_y\colon X\to X$, $\rho_y(x)=x\triangleleft y$ are bijective maps. In particular, 
$\tau_y$ is bijective for all $y\in X$. 

Let $r\colon X\times X\to X\times X$ be the map defined by $r(x,y)=(\sigma_x(y),\tau_y(x))$, for all $x,y\in X$. 
Now we prove that $(X,r)$ is a solution \textcolor{red}{to the YBE}. Let $s\colon X\times X\to X\times X$ be the map defined by $s(x,y)=(y,x\triangleleft y)$. By Proposition \ref{pro:Venkov}, $(X,s)$ is a solution \textcolor{red}{to the YBE}. As in the proof of Proposition \ref{pro:derived}, The map $J\colon X^3\to X^3$, $J(x,y,z)=(x,\sigma_x(y),\sigma_x\sigma_y(z))$ is invertible and satisfies that 
$$(\id\times s) J=J (\id\times r),$$
because the $\sigma_x$ are rack homomorphisms. Furthermore, by \eqref{eq:solskew} we have that
\begin{align*}
	(s\times \id)J(x,y,z)&=(s\times \id)(x,\sigma_x(y),\sigma_x\sigma_y(z))\\
	&=(\sigma_x(y),x\triangleleft \sigma_x(y),\sigma_x\sigma_y(z))\\
	&=(\sigma_x(y), \sigma_{\sigma_x(y)}\tau_y(x),\sigma_{\sigma_x(y)}\sigma_{\tau_y(x)}(z))\\
	&=J(\sigma_x(y),\tau_y(x),z)\\
	&=J(r\times \id)(x,y,z).
\end{align*}
Therefore $(X,r)$ is a solution \textcolor{red}{to the YBE}.

Let $(X,\triangleleft,\cdot)$ be a non-degenerate skew cycle set. 
Let $G(X,\triangleleft,\cdot)=(X,r)$, where
$r(x,y)=(x*y,(x*y)\cdot(x\triangleleft (x*y)))$, where $x*y=z$ 
if and only if $y=x\cdot z$. We have seen that $(X,r)$ is a solution \textcolor{red}{to the YBE}.

For every solution $(X,r)$ \textcolor{red}{to the YBE} we define $F(X,r)=(X,\triangleleft,\cdot)$, 
where $x\triangleleft y=\sigma_y\tau_{\sigma_x^{-1}(y)}(x)$ 
and $x\cdot y=\sigma_x^{-1}(y)$. 
We have seen that $(X,\triangleleft,\cdot)$ is a non-degenerate skew cycle set.

A direct calculation shows that $F(G(X,\triangleleft,\cdot))=(X,\triangleleft,\cdot)$ for every non-degenerate skew cycle set $(X,\triangleleft,\cdot)$ and $G(F(X,r))=(X,r)$ for every solution $(X,r)$.
\end{proof}

Theorem~\ref{thm:skewCS} can be used to construct small solutions, see Table~\ref{tab:non_involutive}.

\begin{table}[H]
\centering
\caption{Enumeration of non-involutive solutions.}
\begin{tabular}{|c|ccccccc|}
\hline
$n$ & 2 & 3 & 4 & 5 & 6 & 7 & 8\tabularnewline
\hline
$s(n)$ & 2 & 21 & 253 & 3519 & 100071 & 4602720 & 422449480\tabularnewline
\hline
\end{tabular}
\label{tab:non_involutive}
\end{table}

%\section*{B}
%\index{Quandle}
%An interesting family of racks is that that of quandles. A \textbf{quandle} is a rack $(X,\triangleleft)$ 
%such that $x\triangleleft x=x$ for all $x\in X$.
%{\bf (Ferran: ¿Tienes alguna idea de c\'omo desarrollar el tema?)} 



% \begin{prob}
%     \label{prob:Venkov}
%     Let $X$ be a set and $X\times X\to X$, $(x,y)\to x\triangleright y$. Prove that 
%     the pair $(X,s)$, where
%     $s(x,y)=(x\triangleright y,x)$, is a solution if and only if $(X,\triangleright)$ is a rack. 
%     Prove that
% \end{prob}


\section{Exercises}

\begin{prob}
\index{Braid group}
For $n\geq2$, the \emph{braid group} $\B_n$ is defined as the group with generators $\sigma_1,\dots,\sigma_{n-1}$ and relations
\begin{align*}
    &\sigma_i\sigma_{i+1}\sigma_i=\sigma_{i+1}\sigma_i\sigma_{i+1} && \text{if }1\leq i\leq n-2,\\
    &\sigma_i\sigma_j=\sigma_j\sigma_i && \text{if }|i-j|> 1.
\end{align*}
Let $(X,r)$ be a set-theoretic solution to the YBE. Write $X^n=X\times\cdots\times X$ ($n$-times).  
For $i<n$ let 
$r_{i,i+1}=\id_{X^{i-1}}\times r\times\id_{X^{n-i-1}}\colon X^n\to X^n$.
Then the map $\sigma_i\mapsto r_{i,i+1}$ extends 
to an action of $\B_n$ on $X^n$.
\end{prob}

\begin{prob}
\label{prob:Wada}
Let $G$ be a group. Prove that the following maps satisfy the set-theoretic YBE:
\begin{enumerate}[label=\alph*)]
\item $r(x,y)=(y,x^{-1})$.
\item $r(x,y)=(y^{-1},x^{-1})$.
\item $r(x,y)=(xyx,x^{-1})$.
\item $r(x,y)=(x^2y,y^{-1}x^{-1}y)$.	
\end{enumerate}	
\end{prob}

\begin{prob}
\label{prob:Wada_racks}
	Let $G$ be a group. Prove that the following maps satisfy the set-theoreic YBE:
	\begin{enumerate}[label=\alph*)]
	\item $r(x,y)=(x^myx^{-m},x)$ for every integer $m$.
	\item $r(x,y)=(xy^{-1}x,x)$.
\end{enumerate}	
\end{prob}

\begin{prob}
\label{prob:D_n}
Let $n\geq2$ and $X=\Z/(n)$ be the ring of integers modulo $n$. Prove that
the map $r(x,y)=(2x-y,x)$ satisfies the the set-theoretic YBE.  	
\end{prob}


\begin{prob}
Let $G$ be a group and $f\in\Aut(G)$. Prove that 
the map 
\[
r(x,y)=(f(y),f(y)^{-1}xy)
\]
satisfies the set-theoretic YBE.	
\end{prob}

\begin{prob}
    \label{prob:xx}
    Let $(X,r)$ be a solution. Let $(X,\triangleleft)$ and $(X,\blacktriangleleft)$ be the derived rack and the dual derived rack of $(X,r)$, respectively. Prove that $x\triangleleft x=x\blacktriangleleft x$ for all $x\in X$. 
\end{prob}

\begin{prob}
    \label{prob:tau_hat}
    Let $(X,r)$ be a solution and $(X,\triangleleft)$ be its derived rack.
    Prove that 
    \[
    \widehat{\tau}_x(y\triangleleft z)=\widehat{\tau}_x(y)\triangleleft \widehat{\tau}_x(z)
    \]
    for all $x,y,z\in X$. 
\end{prob}

\begin{prob}
    \label{prob:variationT}
    Let $(X,r)$ be a solution and let $(X,\triangleright)$ be its derived rack. 
    Prove that 
    \[
    T\sigma_x(y)=\tau_x^{-1}(x\triangleright T(y))
    \]
    for all $x\in X$, where $T\colon X\to X$, $T(y)=\tau_y^{-1}(y)$. 
\end{prob}

\begin{prob}
\label{prob:guitar}
Let $(X,r)$ be a solution  and $(X,s)$ its derived solution, \textcolor{red}{i. e. $s(x,y)=(y,\sigma_y\widehat{\sigma}^{-1}_y(x))$ for all $x,y\in X$}. Let $J_2(x,y)=(x,\sigma_x(y))$ and
$J_{n+1}=Q_{n+1}\circ (\id\times J_n)$ for $n\geq 2$, where 
\[
Q_{n+1}(x_1,\dots,x_{n+1})=(x_1,\sigma_{x_{1}}(x_2),\dots,\sigma_{x_{1}}(x_{n+1})).
\]
Prove that $J_n\circ r_{i,i+1}=s_{i,i+1}\circ J_n$ for all $n\geq2$ and $i\in\{1,\dots,n-1\}$, where $r_{i,i+1}=\id^{\times i-1}\times r\times \id^{\times n-i-1}$ and $s_{i,i+1}=\id^{\times i-1}\times s\times \id^{\times n-i-1}$. 
\end{prob}

\begin{prob}
	\label{prob:Rump}
	Prove Proposition~\ref{pro:Rump}. 
\end{prob}

\begin{prob}
	If $X$ is a cycle set, then $x\cdot (y\cdot y)=((y*x)\cdot y)\cdot ((y*x)\cdot y)$, where
	$y*x=z$ if and only if $y\cdot z=x$. 
\end{prob}

\begin{prob}
	\label{prob:CS}
	Prove Theorem~\ref{thm:CS}. 
\end{prob}



%\begin{prob}
%Let $X$ be a \framebox{finite} non-empty set and $r\colon X\times X\to X\times X$, $(x,y)\mapsto (\sigma_x(y),\tau_y(x))$, be a map.
%Prove that $(X,r)$ is a solution if and only if 
%the maps $\sigma_x\colon X\to X$ are bijective for all $x\in X$,
%$r^2=\id_{X\times X}$ and 
%\[
%\sigma_x\circ\sigma_{\sigma^{-1}_x(y)}=\sigma_y\circ\sigma_{\sigma^{-1}_y(x)}
%\]
%for all $x,y\in X$. 
%\end{prob}
%
%
%\begin{prob}
%Prove that if $(X,r)$ be a solution...? \framebox{FIXME}
%\end{prob}
%
%\begin{prob}
%\label{prob:perm_group}
%Let $(X,r)$ be a solution. Prove that $\mathcal{G}(X,r)\simeq\langle (\sigma_x,\tau^{-1}_x):x\in X\rangle$. 
%\end{prob}

\section{Open problems}


\begin{problem}
\label{problem:racks14}
Enumerate isomorphism classes of racks of size 14. 
\end{problem}

\begin{problem}
\label{problem:non_involutive9}
Enumerate non-involutive solutions of size $\geq9$. 
\end{problem}

\begin{problem}
\label{problem:involutive11}
Enumerate involutive solutions of size $\geq11$. 
\end{problem}


\begin{problem}
	Estimate the number of solutions of size $n$ for $n\to\infty$. 
\end{problem}

\section{Notes}

\index{Gateva--Ivanova, T.}
\index{Van den Bergh, M.}
The first papers where set-theoretic solutions are studied are those of Etingof, Schedler and Soloviev~\cite{MR1722951} 
and Gateva--Ivanova and Van den Bergh~\cite{MR1637256}. 
Both papers deal with non-degenerate involutive solutions, i.e. solutions
$(X,r)$ where $r^2=\id$.  

%\index{Rump, W.}
\index{Drinfeld, V.}
\index{Wada, M.}
\index{Lyubashenko, V.}
In~\cite{MR1183474}, Drinfeld attributes Example~\ref{exa:Lyubashenko} to 
Lyubashenko. 

\index{Lu, J--H.}
\index{Yan, M.}
\index{Zhu, Y--C.}
\index{Etingof, P.}
\index{Schedler, T.}
\index{Soloviev, A.}
Theorem~\ref{thm:LYZ} goes back to Lu, Yan and Zhu, see~\cite{MR1769723}.
Similar results can be found in the work of Etingof, Schedler and Soloviev~\cite{MR1722951} 
for involutive solutions 
and in Soloviev's paper~\cite{MR1809284}.

\textcolor{red}{Rump introduced cycle sets in~\cite{MR2132760}. The bijective correspondence of 
Theorem~\ref{thm:CS} was 
also proved by Rump in~\cite{MR2132760}. A similar result can be 
found in~\cite[Proposition 2.2]{MR1722951}.} 

\textcolor{red}{The numbers of Table~\ref{tab:IYB} were computed in~\cite{MR4405502}
using a combination of~\cite{GAP4} and constraint programming techniques. 
The algorithm is based on an idea of Plemmons~\cite{MR0258994}, originally 
conceived to construct non-isomorphic semigroups. }


%The solutions of Exercises~\ref{prob:Wada_racks} and~\ref{prob:D_n} 
%are particular cases of a more general type of set-theoretic solutions that we will study %in Chapter~\ref{racks}.  
%Proposition~\ref{pro:Rump} was proved by Rump~\cite{MR2278047}. 
%a similar example appears in the appendix of~\cite{MR1722951}. 
 
In~\cite{MR1183474}, Drinfeld attributes Proposition~\ref{pro:Venkov} to Venkov. 

% Definition~\ref{defn:rack} is that of a \emph{left rack}. 
% A \emph{right rack} is defined as a pair $(X,\triangleleft)$, where $X$ is a non-empty set, 
% $X\times X\to X$, $(x,y)\mapsto x\triangleleft y$, is 
% a binary operation on $X$ such that the maps $x\mapsto x\triangleleft y$, are bijective and 
% \[
% (x\triangleleft y)\triangleleft z=(x\triangleleft z)\triangleleft (y\triangleleft z)
% \]
% for all $x,y,z\in X$. 
% As we did in Proposition~\ref{pro:derived}, one proves that $x\triangleleft y=...$ turns $X$ into a 
% right rack. This leads to the right derived solution of $(X,r)$. 
%According to Drinfeld, Exercise~\label{prob:Venkov} 
A particular family of racks turns out to be useful in combinatorial  knot theory. A quandle
is a rack $(X,\triangleleft)$ such that $x\triangleleft x=x$ for all $x\in X$. 

There are several papers on the enumeration of isomorphic classes of finite racks~\cite{MR3665565,MR3118951,MR3904151}. 
Estimations on the number of finite
racks of size $n$ appear in~\cite{MR3118951}. 

The numbers of Table~\ref{tab:non_involutive} were computed using 
Theorem~\ref{thm:skewCS} essentially with the same technique used to construct involutive solutions~\cite{MR4405502}. 
The construction of non-involutive solutions of size 9 seems to be feasible with these methods. 
However, it should be noted that a huge number of solutions is expected. 

The derived rack of a solution was first defined in the work of Soloviev~\cite{MR1809284}. Most of the properties
of the derived racks mentioned in this chapter were proved in~\cite{MR3974961}.


Exercises~\ref{prob:Wada} and~\ref{prob:Wada_racks} 
appear in the work of Wada~\cite{MR1167178} on representations of braid groups. 

Exercises~\ref{prob:xx} and~\ref{prob:tau_hat} appear in~\cite{MR3974961}. 

The map $J_n$ of Exercise~\ref{prob:guitar} is known as the \emph{guitar map}. 
It was first considered by Etingof, Schedler and
Soloviev in~\cite{MR1722951} for involutive solutions. The construction was extended to non-involutive solutions
by Soloviev in~\cite{MR1809284} and Lu, Yan and Zhu in~\cite{MR1769723}. In~\cite{MR3374524} Dehornoy
used the inverse of the guitar map to develop his right-cyclic calculus and to
obtain short proofs for results on the structure group of involutive solutions. 
In~\cite{MR1994219} Andruskiewitsch and Graña use the guitar map to study certain isomorphisms of Nichols algebras. 
A particular case of the guitar map also appears in the work of Przytycki~\cite{MR2906433}. 


Problem~\ref{problem:racks14} appears in~\cite{MR3957904}. 

Problems~\ref{problem:non_involutive9} and \ref{problem:involutive11}
appear in \cite{MR4405502}.



