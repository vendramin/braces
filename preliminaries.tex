\chapter{Preliminaries}
\label{preliminaries}

%\section*{A}

In this chapter we recall the basic concepts and results without proofs in group and ring theory used in this book.

A {\em semigroup} is a set $S$ joint with an associative operation
\[ 
\begin{array}{cccc} *\colon &S\times S&\longrightarrow S\\
&(x,y)&\mapsto&x*y \end{array}
\]
that is $(x*y)*z=x*(y*z)$, for all $x,y,z\in S$.

If there exists an element $e\in S$ such that $e*x=x*e=x$, for all $x\in S$, then the semigroup $(S,*)$ is called a {\em monoid}, and $e$ is the {\em neutral element} for $*$. Note that the neutral element in a monoid $(S,*)$ is unique.

A {\em group} is a monoid $(G,*)$ such that for every $x\in G$ there exists $x'\in G$ satisfying
\[ 
x*x'=x'*x=e,\]
where $e\in G$ is the neutral element for $*$. The element $x'$ is called the {\em symmetric element} of $x$ for $*$, and it is unique.

The standard notation for the operation of a general semigroup is the multiplicative terminology. Thus, if $(S,\cdot)$ is a semigroup, then we write the multiplication  $x\cdot y=xy$, and we say that $xy$ is the product of $x$ and $y$. If $(S,\cdot)$  is a monoid, then $1$ denotes its neutral element and it is called the {\em unit-element} of $S$. If $(S,\cdot)$ is a group, then the symmetric element $x'$ of $x\in S$ is called the {\em inverse} of $x$ and it is denoted by $x^{-1}$.

In some chapters of this book, we also will use the additive terminology for general semigroups, monoids and groups. Thus, if $(S,+)$ is a semigroup the addition of elements $x+y$ is called the sum of $x$ and $y$. If $(S,+)$  is a monoid, then $0$ denotes its neutral element and it is called the {\em zero} of $S$. If $(S,\cdot)$ is a group, then the symmetric element $x'$ of $x\in S$ is called the {\em opposite} of $x$ and it is denoted by $-x$.

In the remainder of this chapter we shall use the multiplicative terminology for general semigroups, monoids and groups.

A semigroup $S$ is said to be {\em commutative} if $xy=yx$, for all $x,y\in S$. A commutative group is also called an {\em abelian group}.

\begin{example}
 $(\Z, +)$,  $(\Q, +)$, $(\R, +)$, $(\C, +)$, $(\Q\setminus\{0\}, \cdot)$, $(\R\setminus\{0\}, \cdot)$, 
	$(\C\setminus\{0\}, \cdot)$ are abelian groups.
\end{example}

\begin{example}
	The {\em symmetric group} on a set $X$ is the set 
$$\Sym_X=\{ f\colon X\rightarrow X\mid f \mbox{ is bijective}\}$$
with the composition of maps. For every positive integer $n$ the symmetric group of degree $n$ is
$\Sym_n=\Sym_{\{ 1,2,\dots ,n\}}$. Note that $\Sym_n$ is not abelian for $n>2$.
\end{example}    



A {\em subgroup} of a group $G$ is a non-empty subset $H$ of $G$ such that $xy^{-1}\in H$, for all $x,y\in H$. Note that every subgroup $H$ of $G$ also is a group with the restriction to $H$ of the operation of $G$. The notation $H\leq G$ will mean that $H$ is a subgroup of $G$.

\begin{proposition}\label{intersection}
	Let $(H_i)_{i\in I}$ be a non-empty family of subgroups of a group $G$. Then 
	\[ \bigcap_{i\in I}H_i \]
    is a subgroup of $G$. \qed
	\end{proposition}   

Let $S$ be a subset of a group $G$. The subgroup of $G$ generated by $S$ is
$$\langle S\rangle=\bigcap_{S\subseteq H\leq G}H.$$
Note that if $S$ is a non-empty subset of $G$, then
$$\langle S\rangle=\{ x_1^{\varepsilon_1}\cdots x_n^{\varepsilon_n}\mid \varepsilon_i=\pm 1,\; x_i\in S,\; 1\leq i\leq n \}.$$
We say that $G$ is {\em finitely generated} if there exists a finite subset $F=\{ x_1,\dots ,x_n\}$ of $G$ such that $G=\langle F\rangle$. In this case we also write $G=\langle x_1,\dots ,x_n\rangle$. We say that $G$ is {\em cyclic} if there exists $x\in G$ such that $G=\langle x\rangle$. 

Let $H$ be a subgroup of a group $G$. The {\em left cosets} of $H$ in $G$ are of the form
$$xH=\{ xh\mid h\in H\},$$
for $x\in G$. A subset of $G$ containing just one element from each left coset of $H$ in $G$ is called a {\em left transversal} of $H$ in $G$. Right cosets and right transversals are defined similarly.
We say that $H$ is a {\em normal subgroup} of $G$ if $xH=Hx$, for all $x\in G$. In this case the set
$G/H=\{xH\mid x\in G\}$, with the operation defined by the rule $(xH)\cdot (yH)=(xy)H$, for all $x,y$, is a group called the {\em quotient group} of $G$ by the normal subgroup $H$. The notation $H\lhd G$ will mean that $H$ is a normal subgroup of $G$. Note that every subgroup of an abelian group is normal.

\begin{example}
	The subgroups of the group $(\Z,+)$ are of the form
	$$n\Z=\{ nz\mid z\in\Z\},$$
	for a non-negative integer $n$. Note that $\Z$ and $\Z/n\Z$ are cyclic groups.  
	\end{example} 

\begin{example} Let $G$ be a group and let $H$ be a subgroup of $G$. The normalizer of $H$ in $G$ is the set
	$$N_G(H)=\{ x\in G\mid xH=Hx\}.$$
	Note that $H\lhd N_G(H)\leq G$.
	\end{example}

A {\em homomorphism of groups} is a map $f\colon G_1\rightarrow G_2$, where $G_1,G_2$ are groups, and such that $f(xy)=f(x)f(y)$, for all $x,y\in G_1$. A bijective homomorphism is called an {\em isomorphism}. Two groups $G_1,G_2$ are isomorphic if there exists an isomorphism $f\colon G_1\rightarrow G_2$. The notation $G_1\cong G_2$ will mean that $G_1$ and $G_2$ are isomorphic. An {\em automorphism} of a group $G$ is an isomorphism from $G$ to itself. The set of all automorphisms of a group $G$ is a group with the composition of maps, called the automorphism group of $G$, and denoted by $\Aut(G)$.

\begin{example} Let $G$ be a group and let $H$ be a subgroup of $G$. The inclusion mapping $\iota\colon H\rightarrow G$, $h\mapsto h$, for all $h\in H$, is an injective homomorphism of groups. Let $N$ be a normal subgroup of $G$. The natural map $\pi\colon G\rightarrow G/N$, $x\mapsto xN$, is a surjective homomorphism of groups.  
	\end{example}

\begin{example}
	Let $G$ be a group. For every $x\in G$, the map $\varphi_x\colon G\rightarrow G$ defined by $\varphi_x(y)=xyx^{-1}$, for all $y\in G$, is an automorphism of $G$. The map $\varphi\colon G\rightarrow \Aut(G)$ defined by $\varphi(x)=\varphi_x$, for all $x\in G$, is a homomorphism of groups. The image of $\varphi$ is the subgroup of {\em inner automorphisms} of $G$, denoted by $\Inn(G)$.
	\end{example}
 
The {\em kernel} of a homomorphism of groups $f\colon G_1\rightarrow G_2$ is the set
$$\ker (f)=\{ x\in G_1\mid f(x)=1\}.$$
It is easy to show that $\ker(f)$ is a normal subgroup of $G_1$. The image of $f$, $\im(f)$, is a subgroup of $G_2$.

\begin{theorem}[First isomorphism theorem]
	For any homomorphism of groups $f\colon G_1\rightarrow G_2$ there exists a unique isomorphism $\tilde f\colon G_1/\ker(f)\rightarrow\im(f)$ such that the diagram
	\[\begin{tikzcd}
		{G_1} & {G_2} \\
		{G_1/\ker(f)} & {\im(f)}
		\arrow["f", from=1-1, to=1-2]
		\arrow["{\pi}"', from=1-1, to=2-1]
		\arrow["{\iota}", from=2-2, to=1-2]
		\arrow["{\tilde f}", from=2-1, to=2-2]
	\end{tikzcd}
	\]
	is commutative, that is $f=\iota\circ \tilde f\circ\pi$, where $\iota$ is the inclusion mapping and $\pi$ is the natural homomorphism. \qed	
	\end{theorem} 
 
Let $H,K$ be subgroups of a group $G$. Note that if $H\leq N_G(K)$, then $KH=\{xy\mid x\in K,\; y\in H\}$ is a subgroup of $G$. Furthermore $K\lhd KH$.

\begin{theorem}[Second isomorphism theorem]
	Let $G$ be a group and $H,K$ subgroups of $G$, such that $H\leq N_G(K)$. Then $K\cap H\lhd H$ and
	$$H/(K\cap H)\cong (KH)/K. \; \qed$$  
	\end{theorem}

\begin{theorem}[Third isomorphism theorem] Given a group $G$ and $N\lhd G$, there is a natural bijection between the subgroups of $G$ containing $N$ and the subgroups of $G/N$: $H\leftrightarrow H/N$. Furthermore, if $N\leq H\lhd G$, then
	$$(G/N)/(H/N)\cong G/H.\; \qed$$ 
	\end{theorem}


A {\em ring} is a set $R$ joint with two operations
$$+,\cdot\colon R\times R\rightarrow R$$
satisfying the following properties:
\begin{itemize}
	\item[1.] $(R,+)$ is an abelian group.
	\item[2.] $(ab)c=a(bc)\quad\forall a,b,c\in R$.
    \item[3.] $a(b+c)=ab+ac,\; (a+b)c=ac+bc\quad\forall a,b,c\in R$.
\end{itemize}
If $(R,\cdot)$ has unit-element, then this is denoted by $1$ and it is said that $R$ is a ring with unity.
\bigskip

{\bf Convention.} From now on, a ring will mean a ring with unity, unless otherwise specified. 
\bigskip

A {\em commutative ring} is a ring $R$ such that
$$ab=ba\quad\forall a,b\in R.$$

Let $R$ be a ring. We say that $a\in R$ is a {\em zero-divisor} if there exists $b\in R\setminus\{0\}$ such that $ab=0$ or $ba=0$. We say that $a\in R$ is an {\em idempotent} if $a^2=a$. We say that $a\in R$ is {\em nilpotent} if there exists a positive integer $n$ such that $a^n=0$.  

An {\em integral domain} is a ring $R$ with $0\neq 1$ and without non-zero zero-divisors.
A {\em skew field or division ring} is a ring $R$ such that $(R\setminus \{0\},\cdot)$ is a group. 
A {\em field} is a commutative division ring. Note that any division ring is an integral domain.

\begin{example} The zero ring is $\{0\}$, with $0+0=0=0\cdot 0$. This is the only ring such that $0=1$.
\end{example}

\begin{example}
 $\Z$ is a commutative integral domain which is not a field.
 $\Q,\R,\C$ are fields.
 \end{example}

%\demo{Exemple}

%Considerem a $\Bbb R^4$ el seg{\"u}ent producte: $(a,b,c,d)(x,y,z,t)=$
%$$=(ax-by-cz-dt,ay+bx+ct-dz,az-bt+cx+dy,at+bz-cy+dx).$$
%Es deixa com a exercici comprovar que l'$\Bbb R$-e.v. $\Bbb R^4$ junt
%amb aquest producte {\'e}s una $\Bbb R$-{\`a}lgebra i $(1,0,0,0)$ {\'e}s la seva
%unitat. Aquesta {\`a}lgebra es denota per $\Bbb H$ i es diu {\`a}lgebra dels
%quaternions de Hamilton. Siguin
%$$\bold 1=(1,0,0,0),\; i=(0,1,0,0),\; j=(0,0,1,0),\; k=(0,0,0,1).$$
%Si identifiquem $\Bbb R$ amb $\Bbb R\bold 1$ per $a=a\bold 1$, tenim
%$$\Bbb H=\{ a+bi+cj+dk\mid a,b,c,d\in \Bbb R\}$$
%i $i^2=j^2=k^2=-1$, $ij=k=-ji$, $jk=i=-kj$, $ki=j=-ik$, a m{\'e}s
%$$(a+bi+cj+dk)(a-bi-cj-dk)=a^2+b^2+c^2+d^2,$$
%per tant tot element de $\Bbb H$ no nul t{\'e} invers, i aix{\'\i} $\Bbb H$ {\'e}s
%un {\`a}lgebra de divisi{\'o} sobre $\Bbb R$.
%De forma natural es veu que $\Bbb C$ {\'e}s una sub{\`a}lgebra de $\Bbb H$.
%\enddemo
%
%\medpagebreak


\begin{example}
If $\{ R_i\} _{i\in I}$ is a non-empty family of rings, then
$\prod_{i\in I}R_i$ with the addition and the multiplication defined componentwise is a ring. If $|I|\geq 2$ and the rings $R_i$ are non-zero, then $\prod_{i\in I}R_i$ has non-trivial idempotents, that is different from $0$ and $1$. For example:
$$e_i=(\delta _{ij})_{j\in I}\quad\mbox{where}\quad\delta_{ij}=\left\{\begin{array}{ll}
0&\quad\mbox{if}\; j\neq i\\
1&\quad\mbox{if}\; j=i\end{array}\right. $$
\end{example}

\begin{example}
	Let $R$ be a ring. The {\em power series ring} over $R$ is the set
	$$R[[x]]=\left\{ \sum_{i=0}^{\infty}a_ix^{i}\mid a_{i}\in R\right\},
	$$
	with the addition and the multiplication defined by the rules
	$$\sum_{i=0}^{\infty}a_ix^{i} +\sum_{i=0}^{\infty}b_ix^{i}=\sum_{i=0}^{\infty}(a_i+b_i)x^{i}$$
	and
	$$\left(\sum_{i=0}^{\infty}a_ix^{i}\right)\cdot\left( \sum_{i=0}^{\infty}b_ix^{i}\right)=\sum_{i=0}^{\infty}c_ix^{i},$$
	where $c_i=\sum_{j=0}^{i}a_jb_{i-j}$.
\end{example}
	
\begin{example}	
Let $R$ be a ring and let $n$ be a positive integer. The full $n\times n$ {\em matrix ring} is the set
$$M_n(R)=\left\{\left(\begin{array}{cccc}
	a_{1,1}&a_{1,2}&\ldots&a_{1,n}\\
	a_{2,1}&a_{2,2}&\ldots&a_{2,n}\\
	\vdots&\vdots&\ddots&\vdots\\
	a_{n,1}&a_{n,2}&\ldots&a_{n,n}
	\end{array}\right)\mid a_{i,j}\in R\right\},$$
with the addition definend componentwise and the multiplication defined by the rule
$$\left(\begin{array}{cccc}
	a_{1,1}&a_{1,2}&\ldots&a_{1,n}\\
	a_{2,1}&a_{2,2}&\ldots&a_{2,n}\\
	\vdots&\vdots&\ddots&\vdots\\
	a_{n,1}&a_{n,2}&\ldots&a_{n,n}
\end{array}\right)\cdot\left(\begin{array}{cccc}
b_{1,1}&b_{1,2}&\ldots&b_{1,n}\\
b_{2,1}&b_{2,2}&\ldots&b_{2,n}\\
\vdots&\vdots&\ddots&\vdots\\
b_{n,1}&b_{n,2}&\ldots&b_{n,n}
\end{array}\right)=\left(\begin{array}{cccc}
	c_{1,1}&c_{1,2}&\ldots&c_{1,n}\\
	c_{2,1}&c_{2,2}&\ldots&c_{2,n}\\
	\vdots&\vdots&\ddots&\vdots\\
	c_{n,1}&c_{n,2}&\ldots&c_{n,n}
\end{array}\right),$$
where $c_{i,j}=\sum_{k=1}^na_{i,k}b_{k,j}$.
\end{example}

A {\em subring} of a ring $R$ is a subset $S$ of $R$ such that
\begin{itemize}
\item[(i)] $a-b\in S\quad\forall a,b\in S$.
\item[(ii)] $ab\in S\quad\forall a,b\in S$.
\item[(iii)] $1\in S$.
\end{itemize}

\begin{example}
	$\Z$ is a subring of $\Q$, which is a subring of $\R$, which is a subring of $\C$.
	\end{example}

\begin{example}
	Let $R$ be a ring. The {\em polynomial ring} over $R$ is the following subring of the power series ring over $R$:
	$$R[x]=\left\{ \sum_{i=0}^{\infty}a_ix^{i}\in R[[x]] \mid a_i\neq 0 \mbox{ for finitely many non-negative integers } i \right\}.$$
	The elements of $R[x]$ are usualy written as finite sums of the form $\sum_{i=0}^na_ix^{i}$ and are called polynomials. Note also that $R$ is a subring of $R[x]$.
	\end{example}

An {\em ideal} of a ring $R$ is a non-empty subset $I$ of $R$ such that
\begin{itemize}
	\item[(i)] $a+b\in I\quad\forall a,b\in I$.
\item[(ii)] $ar,ra\in I\quad\forall a\in I,\;\forall r\in R$.
\end{itemize}
Any ideal of $R$ differen from $R$ is said to be {\em proper}.


\begin{example}
Let $R$ be a ring. Then $\{ 0\}$ and $R$ are ideals of $R$. The ideal $\{ 0\}$
is called the zero ideal and sometimes it is writen as $0$.
\end{example}

\begin{example}
The ideals of the ring $\Z$ are its additive subgroups, that is $n\Z$, for all non-negative integers $n$.
\end{example}


\begin{example} Let $(I_j)_{j\in J}$ be a non-empty family of ideals of a ring $R$. Then $\cap_{j\in J}I_j$ is an ideal of $R$.
\end{example}

If $S$ is a subset of a ring $R$, then the ideal of $R$ generated by $S$ 
is the intersection of all ideal $I$ of $R$ containing $S$. If
$(I_j)_{j\in J}$ is a non-empty family of ideals of a ring $R$, then its sum $\sum_{j\in J}I_j$ is the ideal of $R$ generated by
$\cup_{j\in  J}I_j$.

Let $I$ be an ideal of a ring $R$. The quotient ring of $R$ by the ideal $I$ is the additive quotient group
$R/I$ with the multiplication of left cosets defined by $(a+I)\cdot (b+I)=(ab)+I$, for all $a,b\in R$. It is easy to check that the multiplication of left cosets is well-defined and that $R/I$ with the addition and the multiplication of left cosets is a ring.



Let $R$ and $S$ be rings. A map $f\colon R\rightarrow S$
is said to be a {\em homomorphism of rings} if it satisfies the following properties:
\begin{itemize}
	\item[(i)] $f(a+b)=f(a)+f(b)\quad\forall a,b\in R$.
    \item[(ii)] $f(ab)=f(a)f(b)\quad\forall a,b\in R$.
    \item[(iii)] $f(1)=1$.
\end{itemize}




It is easy to check that $\im(f)$ is a subring of $S$ and
$\ker(f)=\{ r\in R\mid f(r)=0\}$ is an ideal of $R$.

\begin{example}
Let  $R$ be a ring. The map $f\colon \Z\rightarrow R$ defined by
$$f(z)=\left\{\begin{array}{ll}
1+\dots +1\;\mbox{($z$ times)}&\mbox{ if } z>0\\
0&\mbox{ if } z=0\\
(-1)+\dots+(-1)\;\mbox{($-z$ times)}&\mbox{ if } z<0\end{array}\right.
$$
is a homomorphism of rings. There exists a unique non-negative integer $n$ such that
$\ker(f)=n\Z$. This integer $n$ is said to be the {\em characteristic} of $R$.
\end{example}

\begin{example} Let $R$ be a ring and let $S$ be a subring of $R$. The inclusion mapping $\iota\colon S\rightarrow R$, $a\mapsto a$, for all $a\in S$, is an injective homomorphism of rings. Let $I$ be an ideal of $R$. The natural map $\pi\colon R\rightarrow R/I$, $a\mapsto a+I$, is a surjective homomorphism of rings.  
\end{example}


An {\em isomorphism} of rings is a bijective homomorphism. Two rings $R,S$ are isomorphic if there exists an isomorphism $f\colon R\rightarrow S$. The notation $R\cong S$ will mean that $R$ and $S$ are isomorphic. An {\em automorphism} of a ring $R$ is an isomorphism from $R$ to itself. 

\begin{theorem}[First isomorphism theorem]
	For any homomorphism of rings $f\colon R\rightarrow S$ there exists a unique isomorphism 
	$\tilde f\colon R/\ker(f)\rightarrow\im(f)$ such that the diagram
	\[\begin{tikzcd}
		{R} & {S} \\
		{R/\ker(f)} & {\im(f)}
		\arrow["f", from=1-1, to=1-2]
		\arrow["{\pi}"', from=1-1, to=2-1]
		\arrow["{\iota}", from=2-2, to=1-2]
		\arrow["{\tilde f}", from=2-1, to=2-2]
	\end{tikzcd}
	\]
	is commutative, that is $f=\iota\circ \tilde f\circ\pi$, where $\iota$ is the inclusion mapping and $\pi$ is the natural homomorphism. \qed	
\end{theorem} 

\begin{theorem}[Second isomorphism theorem]
	Let $R$ be a ring,  $S$ a subring and $I$ an ideal of $R$. Then $I\cap S$ is an ideal of $S$ and
	$$S/(I\cap S)\cong (S+I)/I. \; \qed$$  
\end{theorem}

\begin{theorem}[Third isomorphism theorem] Let $R$ be a ring and $I$ an ideal of $R$. Then there is a natural bijection between the subrings (respectively ideals) of $R$ containing $I$ and the subrings (respectively ideals) of $R/I$: $S\leftrightarrow S/I$. Furthermore, if $J$ is an ideal of $R$ containing $I$, then
	$$(R/I)/(J/I)\cong R/J.\; \qed$$ 
\end{theorem}

\index{Lemma!Zorn}
Let $(A,\leq)$ be a {\em partially order set}, this means that $A$ is a set together with a 
reflexive, transitive and anti-symmetric binary relation
$R$ on $A\times A$, where $a\leq b$ if and only if $(a,b)\in R$. 
Recall that the relation is reflexive if $a\leq a$ for all $a\in A$, the relation is transitive if 
$a\leq b$ and $b\leq c$ imply that 
$a\leq c$ and the relation is anti-symmetric if $a\leq b$ and $b\leq a$ imply $a=b$.

The elements $a,b\in A$ are said to be {\em comparable} if $a\leq b$ or $b\leq
a$. An element $a\in A$ is said to be {\em maximal} if 
$c\leq a$ 
for all $c\in A$
that is comparable with $a$. 
An {\em upper bound} for a non-empty subset $B\subseteq A$ is an element $d\in
A$ such that $b\leq d$ for all $b\in B$. A {\em chain} in $A$ is a subset 
$B$ such that every pair of elements of $B$ are comparable. 
{\em Zorn's lemma} states the following property: 
\begin{quote}
	If $A$ is a non-empty partially ordered set such that every chain in 
	$A$ has an upper bound in $A$, then $A$ contains a maximal element. 
\end{quote}


\section*{Exercises}

\begin{prob} Prove all the results stated in this chapter.
\end{prob}



\section*{Notes}
The material in this chaspter is basic and standard, see for example \cite{Cohn}.
