\chapter{Preliminaries}
\label{preliminaries}

%In this chapter we recall the basic concepts and results without proofs in group and ring theory used in this book.

\section{Groups}

\index{Semigroup}
A {\em semigroup} is a set $S$ with an associative operation
\[ 
*\colon S\times \longrightarrow S, 
\quad
(x,y)\mapsto x*y,
\]
that is $(x*y)*z=x*(y*z)$ for all $x,y,z\in S$.

If there exists an element $e\in S$ such that $e*x=x*e=x$ for all $x\in S$, then 
the semigroup $(S,*)$ is called a {\em monoid} and $e$ is the {\em neutral element} for $*$. 
The neutral element in a monoid $(S,*)$ is unique.

\index{Group}
A {\em group} is a monoid $(G,*)$ such that for every $x\in G$ there exists $x'\in G$ satisfying
\[ 
x*x'=x'*x=e,
\]
where $e\in G$ is the neutral element for $*$. The element $x'$ is called 
the {\em symmetric element} of $x$ for $*$ and it is unique.

The standard notation for the operation of a general semigroup is the multiplicative terminology. 
Thus if $(S,\cdot)$ is a semigroup, then we write the multiplication  $x\cdot y=xy$  
and we say that $xy$ is the product of $x$ and $y$. If $(S,\cdot)$  is a monoid, 
then $1$ denotes its neutral element and it is called the {\em unit-element} of $S$. 
If $(S,\cdot)$ is a group, then the symmetric element $x'$ of $x\in S$ is called the {\em inverse} 
of $x$ and it is denoted by $x^{-1}$.

\index{Semigroup!additive}
In some chapters, we also will use the additive terminology for general semigroups, monoids and groups. 
Thus if $(S,+)$ is a semigroup the addition of elements $x+y$ is called the sum of $x$ and $y$. If $(S,+)$ 
is a monoid, then $0$ denotes its neutral element and it is called the {\em zero} of $S$. 
If $(S,\cdot)$ is a group, then the symmetric element $x'$ of $x\in S$ is called the {\em opposite} of $x$ and it is denoted by $-x$.

In the remainder of this chapter we shall use the multiplicative terminology for general semigroups, monoids and groups.

\index{Semigroup!commutative}
\index{Group!abelian}
A semigroup $S$ is said to be {\em commutative} if $xy=yx$ for all $x,y\in S$. A commutative group is also called an {\em abelian group}.

\begin{example}
 $(\Z, +)$,  $(\Q, +)$, $(\R, +)$, $(\C, +)$, $(\Q\setminus\{0\}, \cdot)$, $(\R\setminus\{0\}, \cdot)$, 
	$(\C\setminus\{0\}, \cdot)$ are abelian groups.
\end{example}

\begin{example}
\index{Symmetric group}
The {\em symmetric group} on a set $X$ is the set 
\[
\Sym_X=\{ f\colon X\rightarrow X: f \mbox{ is bijective}\}
\]
with the composition of maps. For every positive integer $n$ the symmetric group of degree $n$ is
$\Sym_n=\Sym_{\{ 1,2,\dots ,n\}}$. Note that $\Sym_n$ is not abelian for $n>2$.
\end{example}    

\index{Subgroup}
A {\em subgroup} of a group $G$ is a non-empty subset $H$ of $G$ such that $xy^{-1}\in H$ for all $x,y\in H$. 
Note that every subgroup $H$ of $G$ also is a group with the operation of $G$ restricted to $H$.
The notation $H\leq G$ will mean that $H$ is a subgroup of $G$.

\begin{proposition}
\label{intersection}
	Let $(H_i)_{i\in I}$ be a non-empty family of subgroups of a group $G$. Then 
	\[ 
	\bigcap_{i\in I}H_i 
	\]
    is a subgroup of $G$. \qed
\end{proposition}   

\index{Subgroup!generated by a subset}
\index{Subgroup!finitely generated}
Let $S$ be a subset of a group $G$. The \emph{subgroup of $G$ generated by} 
$S$ is
\[
\langle S\rangle=\bigcap_{S\subseteq H\leq G}H.
\]
Note that if $S$ is a non-empty subset of $G$, then
\[
\langle S\rangle=\{ x_1^{\varepsilon_1}\cdots x_n^{\varepsilon_n}: n\geq0,\; \varepsilon_i=\pm 1,\; x_i\in S,\; 1\leq i\leq n \}.
\]
We say that $G$ is {\em finitely generated} if there exists a finite subset $F=\{ x_1,\dots ,x_n\}$ of $G$ such that $G=\langle F\rangle$. In this case we also write $G=\langle x_1,\dots ,x_n\rangle$. We say that $G$ is {\em cyclic} if there exists $x\in G$ such that $G=\langle x\rangle$. 

\index{Left coset}
\index{Right coset}
\index{Subgroup!normal}
Let $H$ be a subgroup of a group $G$. The {\em left cosets} of $H$ in $G$ are 
the subsets of $G$ of the form
\[
xH=\{ xh: h\in H\},
\]
for $x\in G$. A subset of $G$ containing just one element from each left coset of $H$ in $G$ is called a {\em left transversal} of $H$ in $G$. 
\textcolor{blue}{Note that $xH=yH$ if and only if $x^{-1}y\in H$.}
Right cosets and right transversals are defined similarly.
We say that $H$ is a {\em normal subgroup} of $G$ if $xH=Hx$ for all $x\in G$. In this case the set
$G/H=\{xH: x\in G\}$, with the operation defined by the rule $(xH)\cdot (yH)=(xy)H$ for all $x,y\in G$, 
is a group called the {\em quotient group} of $G$ by the normal subgroup $H$. 
\textcolor{blue}{The notation $H\unlhd G$ will mean that $H$ is a normal subgroup of $G$.}
Note that every subgroup of an abelian group is normal.

\begin{example}
	The subgroups of the group $(\Z,+)$ are of the form
	\[
	n\Z=\{ nz: z\in\Z\},
	\]
	for a non-negative integer $n$. Note that $\Z$ is a cyclic group.  
\end{example} 

\begin{example}
\textcolor{blue}{The group $\Z/n\Z=\{0,1,\dots,n-1\}$ of integers modulo $n$ is a cyclic group.}
\end{example}

\begin{example} 
\index{Normalizer of a subgroup}
Let $G$ be a group and let $H$ be a subgroup of $G$. 
The \emph{normalizer} of $H$ in $G$ is the set
	\[
	N_G(H)=\{ x\in G: xH=Hx\}.
	\]
	Note that $H\unlhd N_G(H)\leq G$.
\end{example}

\index{Order!of a group}
\index{Order!of an element}
\index{Index!of a subgroup}
\textcolor{blue}{Let $G$ be a finite group. The {\em order} $|G|$ of $G$ is the number of elements of $G$. Let $H$ be a subgroup of $G$. The {\em index} 
$(G:H)$ of $H$ in $G$ is the number of left cosets of $H$ in $G$.}
The index $(G:H)$ also coincides with the number of right cosets of $H$ in $G$.

\begin{theorem}[Lagrange's theorem]
\index{Lagrange's theorem}
Let $G$ be a finite group and let $H$ be a subgroup of $G$. 
Then $|G|=(G:H)|H|$.\qed
\end{theorem}

\index{Homomorphism!of groups}
\index{Isomorphism!of groups}
\index{Automorphism group}
A {\em homomorphism of groups} is a map $f\colon G_1\rightarrow G_2$, where $G_1$ \textcolor{blue}{and} $G_2$ 
are groups, and such that $f(xy)=f(x)f(y)$ for all $x,y\in G_1$. A bijective homomorphism is called an {\em isomorphism}. Two groups $G_1$ \textcolor{blue}{and} $G_2$ are isomorphic if there exists an isomorphism $f\colon G_1\rightarrow G_2$. The notation $G_1\cong G_2$ will mean that $G_1$ and $G_2$ are isomorphic. 
An {\em automorphism} of a group $G$ is an isomorphism $G\to G$. \textcolor{blue}{The set $\Aut(G)$ of all automorphisms of a group $G$ is a group with the usual 
composition of maps; it is called the {\em automorphism group} of $G$.}

\begin{example} 
    Let $G$ be a group and let $H$ be a subgroup of $G$. 
    The inclusion mapping $\iota\colon H\rightarrow G$, $h\mapsto h$, is an injective homomorphism of groups. 
\end{example}

\begin{example}
\index{Inner automorphism group}
	Let $G$ be a group. For every $x\in G$, the map $\varphi_x\colon G\rightarrow G$ defined by $\varphi_x(y)=xyx^{-1}$, for all $y\in G$, is an automorphism of $G$. The map $\varphi\colon G\rightarrow \Aut(G)$ defined by $\varphi(x)=\varphi_x$, for all $x\in G$, is a homomorphism of groups. 
    \textcolor{blue}{The group $\Inn(G)$ of {\em inner automorphisms} of $G$ is defined as the 
	image of $\varphi$.}
\end{example}

\index{Kernel!of a group homomorphism}
\index{Image!of a group homomorphism}
The {\em kernel} of a homomorphism of groups $f\colon G_1\rightarrow G_2$ is the set
\[
\ker (f)=\{ x\in G_1: f(x)=1\}.
\]
\textcolor{blue}{The kernel $\ker(f)$ is a normal 
subgroup of $G_1$. The image $\im(f)=\{f(g):g\in G_1\}$ 
of $f$ is a subgroup of $G_2$.}

\begin{example}
\index{Quotient group}
\index{Quotient group!canonical map}
\textcolor{blue}{
Let $G$ be a group and $N$ be a normal subgroup of $G$. 
The natural map $\pi\colon G\rightarrow G/N$, $x\mapsto xN$, is a surjective homomorphism of groups with kernel equal to $N$.}
\end{example}

\begin{theorem}[First isomorphism theorem]
\index{First isomorphism theorem!for groups}
	For any homomorphism of groups $f\colon G_1\rightarrow G_2$ there exists a unique isomorphism $\tilde f\colon G_1/\ker(f)\rightarrow\im(f)$ such that the diagram
	\[\begin{tikzcd}
		{G_1} & {G_2} \\
		{G_1/\ker(f)} & {\im(f)}
		\arrow["f", from=1-1, to=1-2]
		\arrow["{\pi}"', from=1-1, to=2-1]
		\arrow["{\iota}", from=2-2, to=1-2]
		\arrow["{\tilde f}", from=2-1, to=2-2]
	\end{tikzcd}
	\]
	is commutative, that is \textcolor{blue}{$f=\iota\tilde f\pi$}, 
	where $\iota$ is the inclusion mapping and $\pi$ is the natural homomorphism. \qed	
\end{theorem} 
 
\textcolor{blue}{
Let $H$ and $K$ be subgroups of a group $G$. If $H\leq N_G(K)$, 
then \[
KH=\{xy: x\in K,\; y\in H\}
\]
is a subgroup of $G$. Furthermore $K\unlhd KH$.}

\begin{theorem}[Second isomorphism theorem]
\index{Second isomorphism theorem!for groups}
	Let $G$ be a group and $H$ and $K$ be subgroups of $G$ 
	uch that $H\leq N_G(K)$. Then $K\cap H\unlhd H$ and
	\[
	H/(K\cap H)\cong (KH)/K.\;\qed
	\]
\end{theorem}

\begin{theorem}[Third isomorphism theorem] 
\index{Third isomorphism theorem!for groups}
Given a group $G$ and $N\unlhd G$, 
\textcolor{blue}{the map $H\mapsto H/N$ yields} 
a bijective correspondence between \textcolor{blue}{(normal)} 
subgroups of $G$ containing $N$ and \textcolor{blue}{(normal)} subgroups of $G/N$. 
Furthermore, if $N\leq H\unlhd G$, then
\[
(G/N)/(H/N)\cong G/H.\;\qed
\]
\end{theorem}

Let $n$ be a positive integer. The elements of $\Sym_n$ are called permutations. For $\sigma\in\Sym_n$ and $\tau\in\Sym_n$ 
we will write $\sigma\tau$ to denote the composition $\sigma\circ\tau$. 
We denote each element $\sigma\in\Sym_n$ by
\[ 
\sigma=\left(\begin{array}{cccc}
1&2&\ldots&n\\
\sigma(1)&\sigma(2)&\ldots&\sigma(n)\end{array}\right).
\]

\textcolor{blue}{
A permutation $\sigma\in \Sym_n$ is a {\em cycle} of length $r$ 
if there exist $r$ distinct elements 
$a_1,\dots ,a_r\in\{ 1,\dots, n\}$ such that $\sigma(a_i)=a_{i+1}$ for $1\leq i\leq r-1$, $\sigma(a_n)=a_1$ and $\sigma(b)=b$ 
for all $b\in \{1,\dots, n\}\setminus\{ a_1,\dots ,a_r\}$. In this case, we will denote $\sigma$ by
\[ 
\sigma=(a_1\,\dots\, a_r).
\]
}
\textcolor{blue}{A {\em transposition} is a cycle of length two. 
We say that the cycles
$(a_1\,\dots\, a_r)$ and $(b_1\,\dots\, b_s)$ are disjoint if 
\[
\{ a_1,\dots, a_r\}\cap \{ b_1,\dots, b_s\}=\emptyset.
\]
Note that if $\sigma$ and $\tau$ are disjoint cycles, then $\sigma\tau=\tau\sigma$.}

\begin{theorem}
    \textcolor{blue}{
    Every permutation $\sigma\in\Sym_n$ can be written as a product of pairwise disjoint cycles.
    % \[ 
    % \sigma=(a_1\,\dots\, a_{r_1})(a_{r_1+1}\,\dots\, a_{r_2})\cdots (a_{r_s+1}\,\dots\, a_n),
    % \]
    % with $1\leq r_1<r_2<\dots<r_s\leq n$. 
    This factorization is unique except for the order in which the factors occur.}
\end{theorem}

\begin{theorem}
    \textcolor{blue}{
    Let $n>1$ be an integer. There exists a unique surjective group homomorphism 
    $\sgn\colon \Sym_n\rightarrow \{ 1,-1\}$
    such that $\sgn(\tau)=-1$ for every transposition $\tau\in\Sym_n$.}
\end{theorem}

\textcolor{blue}{
\index{Sign!of a permutation}
\index{Alternating group}
The group homomorphism $\sgn$ is known as the {\em sign homomorphism} of $\Sym_n$. 
The kernel of $\sgn$ is the {\em alternating group} $\Alt_n$.
We say that the elements of $\Alt_n$ are the {\em even} permutations of $\Sym_n$ 
and the elements of $\Sym_n\setminus\Alt_n$
are the {\em odd} permutations of $\Sym_n$.}

\begin{theorem}
\textcolor{blue}{
Let $n>1$ be an integer. Then the following statements hold:  
\begin{enumerate}
    %\item $\Sym_n$ is generated by $(1\, 2),(1\, 3),\dots (1\, n)$,
    %\item $\Alt_n$ is generated by $(1\, 2\, 3),(1\, 2\, 4),\dots (1\, 2\, n)$.
    \item $\Sym_n=\langle (ij):1\leq i<j\leq n\rangle$. 
	\item $\Sym_n=\langle (12),(13),\dots,(1n)\rangle$.
	\item $\Sym_n=\langle (12),(23),\dots,(n-1\,n)\rangle$.
	\item $\Sym_n=\langle (12),(12\cdots n)\rangle$.
	\item $\Alt_n=\langle (ijk):1\leq i<j<k\leq n\rangle$.
	\item $\Alt_n=\langle (12k):3\leq k\leq n\rangle$.
    \end{enumerate}}
\end{theorem}

\begin{theorem}[Cayley's theorem]
    Let $G$ be a finite group of order $n$. Then $G$ is isomorphic to a subgroup of $\Sym_n$.
\end{theorem}

\index{$p$-group}
\index{$p$-Sylow subgroup}
Let $p$ be a prime. A {\em $p$-group} is a group of order a power of $p$. Let $G$ be a finite group of order
$n=p^\alpha m$, where $\gcd(p,m)=1$. 
Any subgroup of $G$ of order $p^\alpha$ is called a {\em Sylow $p$-subgroup} of $G$. 

\textcolor{blue}{Let 
$\Syl_p(G)$ be the set of Sylow $p$-subgroups of $G$. }

\begin{theorem}[Sylow]
\index{Sylow's theorems}
\textcolor{blue}{
    Let $G$ be a finite group and $p$ a prime. The following statements hold:
    \begin{enumerate}
        \item $\Syl_p(G)\ne\emptyset$.
        \item 
        %$G$ has Sylow $p$-subgroups and 
        Every $p$-subgroup of $G$ is contained in a Sylow $p$-subgroup of $G$.
        \item 
        %All Sylow $p$-subgroups of $G$ are conjugate, i.e. 
        If $P_1,P_2\in\Syl_p(G)$, then  
        %are Sylow $p$-subgroups of $G$, 
        there exists $x\in G$ such that $xP_1x^{-1}=P_2$. 
        \item %The number of Sylow $p$-subgroups is congruent to 
        $|\Syl_p(G)|\equiv 1\bmod p$.
    \end{enumerate}}
\end{theorem}

\begin{theorem}
    Let $p$ be a prime. Any \textcolor{blue}{finite} non-trivial $p$-group has a non-trivial center.
\end{theorem}

\index{Free group}
Let $F$ be a group and $X$ a set. It is said that $F$ is a \emph{free group} 
on $X$ if there exists a function
$\sigma\colon X\rightarrow F$ such that for every function $\alpha\colon X\rightarrow G$, where $G$ is a group, there exists 
a unique group homomorphism $f\colon F\rightarrow G$ such that $f\sigma=\alpha$. Note that, in this case, 
$\sigma$ is injective and $F$ also is a free group on $\sigma(X)$ with respect the inclusion map $\sigma(X)\rightarrow F$.
A group is free if it is free on some set.

\begin{theorem}
    Let $X$ be a set. Then there exists a free group $F$ on $X$.
\end{theorem}

\begin{proposition}
    Let $G$ be a group and $X$ a subset of $G$. If every element $g\in G$ can be written uniquely in the form
    \begin{equation}\label{1normalform}
    g=x_1^{n_1}\cdots x_s^{n_s},
    \end{equation}
    \textcolor{blue}{
    for some $s\geq 0$, 
    where $x_1,\dots,x_s\in X$ are all different and $n_1,\dots,n_s\in \mathbb{Z}\setminus \{0\}$, 
    then $G$ is free on $X$. }
\end{proposition}

If $G$ is a free group on \textcolor{blue}{a subset} $X\subseteq G$, 
then we say that $X$ is a {\em basis} of $G$, and the 
right-hand side of the equality (\ref{1normalform}) is called the {\em normal form} 
of $g\in G$ with respect the basis $X$. All the \textcolor{blue}{bases} 
of $G$ have the same cardinal,
which is then called the {\em rank} of the free group $G$.

A {\em free presentation} of a group $G$ is a surjective homomorphism $\pi \colon F\rightarrow G$, where $F$ is a free group. The kernel of $\pi$ is the subgroup of the {\em relators} of the presentation. Let $Y$ be a basis of $F$ and let $S$ be a subset of $\ker(\pi)$ such that $\ker(\pi)$ is the smallest normal subgroup of $F$ containing $S$, i.e. $\ker(\pi)$ is the intersection of all normal subgroups containing $S$. \textcolor{blue}{Every $s\in S$ has a normal form $w(s)$ with respect to $Y$.}
Then we write
\[
G=\gr(Y: w(s)=1\mbox{ for all }s\in S)
\]
and say that the right-hand side of this equality is a presentation of $G$ with set of generators $Y$ and defining relations $w(s)=1$ 
for all $s\in S$. If $Y=\{ y_1,\dots, y_n\}$, then we also write
\[
G=\gr(y_1,\dots, y_n: w(s)=1\mbox{ for all }s\in S).
\]

\begin{example}
    Let $C_n$ be a cyclic group of order $n$. 
    Then $C_n=\gr(x: x^n=1)$ is a presentation of $C_n$.
\end{example}

\section{Rings}

\index{Ring}
A {\em ring} is a set $R$ with two operations
\[
R\times R\rightarrow R,
\quad
(x,y)\mapsto xy,
\qquad
R\times R\rightarrow R,
\quad
(x,y)\mapsto x+y,
\]
satisfying the following properties:
\begin{enumerate}
	\item $(R,+)$ is an abelian group.
	\item $(ab)c=a(bc)$ for all $a,b,c\in R$.
    \item $a(b+c)=ab+ac$ and $(a+b)c=ac+bc$ for all $a,b,c\in R$.
\end{enumerate}
If $(R,\cdot)$ has an identity element, then this is denoted by $1$ 
and it is said that $R$ is a \textcolor{blue}{unitary ring}. 

\begin{convention}
    From now on, a ring will mean a \textcolor{blue}{unitary ring}, unless otherwise specified. 
\end{convention}

\index{Ring!commutative}
A {\em commutative ring} is a ring $R$ such that
$ab=ba$ for all $a,b\in R$.

\index{Zero divisor}
\index{Idempotent}
\index{Nilpotent}
Let $R$ be a ring. We say that $a\in R$ is a {\em zero-divisor} if there exists $b\in R\setminus\{0\}$ 
such that $ab=0$ or $ba=0$. We say that $a\in R$ is an {\em idempotent} if $a^2=a$. 
We say that $a\in R$ is {\em nilpotent} if there exists a positive integer $n$ such that $a^n=0$.  
\textcolor{blue}{An idempotent $a\in R$ is {\em non-trivial} if $a\not\in\{0,1\}$.}

\index{Integral domain}
\index{Divison ring}
An {\em integral domain} is a ring $R$ with $0\neq 1$ and without non-zero zero-divisors.
A {\em division ring} is a ring $R$ such that $(R\setminus \{0\},\cdot)$ is a group. 
A {\em field} is a commutative division ring. Any division ring is an integral domain.

\begin{example} 
    The zero ring is $\{0\}$ with $0+0=0=0\cdot 0$. This is the only ring such that $0=1$.
\end{example}

\begin{example}
    On the one hand, $\Z$ is a commutative integral domain which is not a field. On the other hand, 
    $\Q$, $\R$ and $\C$ are fields.
\end{example}

%\demo{Exemple}

%Considerem a $\Bbb R^4$ el seg{\"u}ent producte: $(a,b,c,d)(x,y,z,t)=$
%$$=(ax-by-cz-dt,ay+bx+ct-dz,az-bt+cx+dy,at+bz-cy+dx).$$
%Es deixa com a exercici comprovar que l'$\Bbb R$-e.v. $\Bbb R^4$ junt
%amb aquest producte {\'e}s una $\Bbb R$-{\`a}lgebra i $(1,0,0,0)$ {\'e}s la seva
%unitat. Aquesta {\`a}lgebra es denota per $\Bbb H$ i es diu {\`a}lgebra dels
%quaternions de Hamilton. Siguin
%$$\bold 1=(1,0,0,0),\; i=(0,1,0,0),\; j=(0,0,1,0),\; k=(0,0,0,1).$$
%Si identifiquem $\Bbb R$ amb $\Bbb R\bold 1$ per $a=a\bold 1$, tenim
%$$\Bbb H=\{ a+bi+cj+dk\mid a,b,c,d\in \Bbb R\}$$
%i $i^2=j^2=k^2=-1$, $ij=k=-ji$, $jk=i=-kj$, $ki=j=-ik$, a m{\'e}s
%$$(a+bi+cj+dk)(a-bi-cj-dk)=a^2+b^2+c^2+d^2,$$
%per tant tot element de $\Bbb H$ no nul t{\'e} invers, i aix{\'\i} $\Bbb H$ {\'e}s
%un {\`a}lgebra de divisi{\'o} sobre $\Bbb R$.
%De forma natural es veu que $\Bbb C$ {\'e}s una sub{\`a}lgebra de $\Bbb H$.
%\enddemo
%
%\medpagebreak

\begin{example}
    If $\{ R_i\} _{i\in I}$ is a non-empty family of rings, then
    $\prod_{i\in I}R_i$, with the addition and the multiplication defined component-wise, is a ring. 
    If $|I|\geq 2$ and the rings $R_i$ are non-zero, then $\prod_{i\in I}R_i$ has non-trivial idempotents. For example,
    \[
    e_i=(\delta _{ij})_{j\in I}\quad\mbox{where}\quad\delta_{ij}=\begin{cases}
        0&\mbox{if}\; j\neq i,\\
        1&\mbox{if}\; j=i.
    \end{cases}
    \]
\end{example}

\begin{example}
	Let $R$ be a ring. The {\em power series ring} over $R$ is the set
	\[
	R[\![X]\!]=\left\{ \sum_{i=0}^{\infty}a_iX^{i}: a_{i}\in R\right\}
	\]
	with the addition and the multiplication defined by the rules
	\begin{align*}
	    &\sum_{i=0}^{\infty}a_iX^{i} +\sum_{i=0}^{\infty}b_iX^{i}=\sum_{i=0}^{\infty}(a_i+b_i)X^{i},
	\shortintertext{and}
	    &\left(\sum_{i=0}^{\infty}a_iX^{i}\right)\left( \sum_{i=0}^{\infty}b_iX^{i}\right)=\sum_{i=0}^{\infty}c_iX^{i},    
	\end{align*}
	where $c_i=\sum_{j=0}^{i}a_jb_{i-j}$.
\end{example}
	
\begin{example}	
Let $R$ be a ring and let $n$ be a positive integer. 
The full $n\times n$ {\em matrix ring} is the set
$$M_n(R)=\left\{\left(\begin{array}{cccc}
	a_{1,1}&a_{1,2}&\ldots&a_{1,n}\\
	a_{2,1}&a_{2,2}&\ldots&a_{2,n}\\
	\vdots&\vdots&\ddots&\vdots\\
	a_{n,1}&a_{n,2}&\ldots&a_{n,n}
	\end{array}\right): a_{i,j}\in R\right\}$$
with the addition defined component-wise and the multiplication defined by the rule
$$\left(\begin{array}{cccc}
	a_{1,1}&a_{1,2}&\ldots&a_{1,n}\\
	a_{2,1}&a_{2,2}&\ldots&a_{2,n}\\
	\vdots&\vdots&\ddots&\vdots\\
	a_{n,1}&a_{n,2}&\ldots&a_{n,n}
\end{array}\right)\left(\begin{array}{cccc}
b_{1,1}&b_{1,2}&\ldots&b_{1,n}\\
b_{2,1}&b_{2,2}&\ldots&b_{2,n}\\
\vdots&\vdots&\ddots&\vdots\\
b_{n,1}&b_{n,2}&\ldots&b_{n,n}
\end{array}\right)=\left(\begin{array}{cccc}
	c_{1,1}&c_{1,2}&\ldots&c_{1,n}\\
	c_{2,1}&c_{2,2}&\ldots&c_{2,n}\\
	\vdots&\vdots&\ddots&\vdots\\
	c_{n,1}&c_{n,2}&\ldots&c_{n,n}
\end{array}\right),$$
where $c_{i,j}=\sum_{k=1}^na_{i,k}b_{k,j}$.
\end{example}

\index{Subring}
\textcolor{blue}{A {\em subring} of a ring $R$ is a subset $S$ of $R$ such that
$a-b\in S$ and $ab\in S$ for all $a,b\in S$ 
and $1\in S$.}

% \begin{itemize}
% \item[(i)] $a-b\in S\quad\forall a,b\in S$.
% \item[(ii)] $ab\in S\quad\forall a,b\in S$.
% \item[(iii)] $1\in S$.
% \end{itemize}

\begin{example}
    \textcolor{blue}{$\Z\subseteq\Q\subseteq\R\subseteq\C$ is a chain of subrings.}
	%$\Z$ is a subring of $\Q$, which is a subring of $\R$, which is a subring of $\C$.
\end{example}

\begin{example}
	Let $R$ be a ring. The {\em polynomial ring} over $R$ is the following subring of the power series ring over $R$:
	\[
	R[X]=\left\{ \sum_{i=0}^{\infty}a_iX^{i}\in R[\![X]\!] : a_i\neq 0 \mbox{ for finitely many non-negative integers } i \right\}.
	\]
	The elements of $R[X]$ are usually written as finite sums of the 
	form $\sum_{i=0}^na_iX^{i}$ and are called polynomials. Note also that $R$ is a subring of $R[X]$.
\end{example}

\textcolor{blue}{
An {\em ideal} of a ring $R$ is a non-empty subset $I$ of $R$ such that
$a+b\in I$ for $a,b\in I$, 
$ar\in I$ and $ra\in I$ for all $a\in I$ and $r\in R$. }
% \quad\forall a\in I,\;\forall r\in R$.
% \begin{itemize}
% 	\item[(i)] $a+b\in I\quad\forall a,b\in I$.
% \item[(ii)] $ar,ra\in I\quad\forall a\in I,\;\forall r\in R$.
% \end{itemize}

\begin{example}
    Let $R$ be a ring. Then $\{ 0\}$ and $R$ are ideals of $R$. The ideal $\{ 0\}$
    is called the zero ideal and sometimes it is written as $0$.
\end{example}

Any ideal of $R$ different from $R$ is said to be {\em proper}.

\begin{example}
    \textcolor{blue}{The ideals of the ring $\Z$ are of the form $n\Z$ for some integer $n\geq0$.}
\end{example}

\begin{example} 
    Let $\{I_j\}_{j\in J}$ be a non-empty family of ideals of a ring $R$. Then $\cap_{j\in J}I_j$ is an ideal of $R$.
\end{example}

If $S$ is a subset of a ring $R$, then the ideal of $R$ generated by $S$ 
is the intersection of all ideal $I$ of $R$ containing $S$. If
$\{I_j\}_{j\in J}$ is a non-empty family of ideals of a ring $R$, then its sum $\sum_{j\in J}I_j$ is the ideal of $R$ generated by
$\cup_{j\in  J}I_j$.

Let $I$ be an ideal of a ring $R$. The quotient ring of $R$ by the ideal $I$ is the additive quotient group
$R/I$ with the multiplication of left cosets defined by 
\[
(a+I)(b+I)=(ab)+I
\]
for $a,b\in R$. \textcolor{blue}{Since $I$ is an ideal of $R$, the multiplication is well-defined. It follows that 
$R/I$ with the addition and the multiplication of left cosets is a ring.}

\textcolor{blue}{
Let $R$ and $S$ be rings. A map $f\colon R\rightarrow S$
is said to be a {\em homomorphism of rings} if 
$f(a+b)=f(a)+f(b)$ and $f(ab)=f(a)f(b)$ for all $a,b\in R$ and $f(1)=1$.
% it satisfies the following properties:
% \begin{enumerate}
% 	\item $f(a+b)=f(a)+f(b)$ for all $a,b\in R$.
%     \item $f(ab)=f(a)f(b)$ for all $a,b\in R$.
%     \item $f(1)=1$.
% \end{enumerate}
The image $\im(f)$ of $f$ is a subring of $S$ and the kernel 
$\ker(f)=\{ r\in R: f(r)=0\}$ of $f$ is an ideal of $R$.}

\begin{example}
Let  $R$ be a ring. The map $f\colon \Z\rightarrow R$ defined by
\[
f(z)=\begin{cases}
    1+\dots +1\mbox{ ($z$ times)}&\mbox{ if } z>0,\\
    0 &\mbox{ if } z=0,\\
    (-1)+\dots+(-1) \mbox{ ($-z$ times)}&\mbox{ if } z<0,
    \end{cases}
\]
is a homomorphism of rings. There exists a unique non-negative integer $n$ such that
$\ker(f)=n\Z$. This integer $n$ is said to be the {\em characteristic} of $R$.
\end{example}

\begin{example} 
\textcolor{blue}{Let $R$ be a ring and let $S$ be a subring of $R$. The inclusion $\iota\colon S\rightarrow R$, $a\mapsto a$, is an injective homomorphism of rings. Let $I$ be an ideal of $R$. The natural map $\pi\colon R\rightarrow R/I$, $a\mapsto a+I$, is a surjective homomorphism of rings with kernel equal to $I$.}  
\end{example}


An {\em isomorphism} of rings is a bijective homomorphism. Two rings $R,S$ are isomorphic if there exists an isomorphism $f\colon R\rightarrow S$. The notation $R\cong S$ will mean that $R$ and $S$ are isomorphic. An {\em automorphism} of a ring $R$ is an isomorphism from $R$ to itself. 

\begin{theorem}[First isomorphism theorem]
	For any homomorphism of rings $f\colon R\rightarrow S$ there exists a unique isomorphism 
	$\tilde f\colon R/\ker(f)\rightarrow\im(f)$ such that the diagram
	\[\begin{tikzcd}
		{R} & {S} \\
		{R/\ker(f)} & {\im(f)}
		\arrow["f", from=1-1, to=1-2]
		\arrow["{\pi}"', from=1-1, to=2-1]
		\arrow["{\iota}", from=2-2, to=1-2]
		\arrow["{\tilde f}", from=2-1, to=2-2]
	\end{tikzcd}
	\]
	is commutative, that is $f=\iota \tilde f\pi$, where $\iota$ 
	is the inclusion mapping and $\pi$ is the natural homomorphism. \qed	
\end{theorem} 

\begin{theorem}[Second isomorphism theorem]
	Let $R$ be a ring,  $S$ a subring and $I$ an ideal of $R$. Then $I\cap S$ is an ideal of $S$ and
	$$S/(I\cap S)\cong (S+I)/I. \; \qed$$  
\end{theorem}

\begin{theorem}[Third isomorphism theorem] 
Let $R$ be a ring and $I$ an ideal of $R$. Then there is a natural bijection between the subrings (respectively ideals) of $R$ containing $I$ and the subrings (respectively ideals) of $R/I$. 
%$S\leftrightarrow S/I$. 
Furthermore, if $J$ is an ideal of $R$ containing $I$, then
\[
(R/I)/(J/I)\cong R/J.\; \qed
\]
\end{theorem}

\index{Lemma!Zorn}
Let $(A,\leq)$ be a {\em partially order set}, this means that $A$ is a set together with a 
reflexive, transitive and anti-symmetric binary relation
$R$ on $A\times A$, where $a\leq b$ if and only if $(a,b)\in R$. 
Recall that the relation is reflexive if $a\leq a$ for all $a\in A$, the relation is transitive if 
$a\leq b$ and $b\leq c$ imply that 
$a\leq c$ and the relation is anti-symmetric if $a\leq b$ and $b\leq a$ imply $a=b$.

The elements $a,b\in A$ are said to be {\em comparable} if $a\leq b$ or $b\leq
a$. An element $a\in A$ is said to be {\em maximal} if 
$c\leq a$ 
for all $c\in A$
that is comparable with $a$. 
An {\em upper bound} for a non-empty subset $B\subseteq A$ is an element $d\in
A$ such that $b\leq d$ for all $b\in B$. A {\em chain} in $A$ is a subset 
$B$ such that every pair of elements of $B$ are comparable. 
{\em Zorn's lemma} states the following property: 
\begin{quote}
	If $A$ is a non-empty partially ordered set such that every chain in 
	$A$ has an upper bound in $A$, then $A$ contains a maximal element. 
\end{quote}


\section{Exercises}

\begin{prob} 
    Prove all the results stated in this chapter.
\end{prob}

\section{Notes}

The material in this chapter is basic and standard, see for example \cite{Cohn}.
