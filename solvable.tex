\chapter{Solvable groups}

\section*{A}

\index{Derived!series}
For a group $G$ we define 
\[
		G^{(0)}=G,\quad
		G^{(i+1)}=[G^{(i)},G^{(i)}]\quad i\geq0.
\]
The \textbf{derived series} of $G$ is the sequence 
\[
G=G^{(0)}\supseteq G^{(1)}\supseteq G^{(2)}\supseteq\cdots
\]
Each $G^{(i)}$ is a characteristic subgroup of $G$. We say that 
$G$ is \textbf{solvable} if $G^{(n)}=\{1\}$ for some $n\in\N$. Clearly every abelian group
is solvable. A non-abelian simple group cannot be solvable. Nilpotent groups
are solvable.

\begin{exercise}
	The group $\SL_2(3)$ is solvable. 
\end{exercise}

\index{Subgroup!minimal normal}
Let $p$ be a prime number. An \textbf{elementary abelian} $p$-group is a group 
$P$ such that $x^p=1$ for all $x\in P$.
A subgroup $M$ of a group $G$ is said to be \textbf{minimal normal} if $M\ne\{1\}$,
$M$ is normal in $G$ and the unique normal subgroup of $G$ strictly contained in $M$ is
the trivial subgroup. Every finite group contains a minimal normal subgroup.  

\begin{example}
	If a normal subgroup $M$ is minimal (with respect to the inclusion), then it is
	minimal normal. The converse statement is not true. The subgroup of 
	$\Alt_4$ generated by $(12)(34)$, $(13)(24)$ and $(14)(23)$ is minimal normal in 
	$\Alt_4$ but it is not minimal. 
\end{example}

\begin{example}
	Let $G=\D_{6}=\langle r,s:r^6=s^2=1,\,srs=r^{-1}\rangle$ be the dihedral group
	of size twelve. The subgroups $S=\langle r^2\rangle$ 
	and $T=\langle r^3\rangle$ are minimal normal subgroups.  
\end{example}

\begin{exercise}
	Let $G=\SL_2(3)$. The unique minimal normal subgroup of $G$ is
	$Z(\SL_2(3))\simeq C_2$:
\end{exercise}

\index{Subgroup!characteristic}
A subgroup $H$ of a group $G$ is said to be \textbf{characteristic} if 
$f(H)\subseteq H$ for all $f\in\Aut(G)$. 
The center $Z(G)$ and the commutator subgroup $[G,G]$ of $G$ are both 
characteristic subgroups of $G$. Every characteristic subgroup of $G$ is normal in $G$. 
If $H$ is a characteristic subgroup of $K$ and $K$ is normal in $G$, 
then $H$ is normal in $G$. 

\begin{lemma}
	\label{lemma:minimal_normal}
	Let $M$ be a minimal normal subgroup of $G$. If $M$ is solvable and finite, then
	$M$ is an elementary abelian $p$-group for some prime number $p$. 
\end{lemma}

\begin{proof}
	Since $M$ is solvable, $[M,M]\subsetneq M$. Moreover, $[M,M]$ es normal en $G$, as 
    $[M,M]$ is characteristic in $M$ and $M$ is normal in $G$. Since $M$ is minimal normal, 
    it follows that $[M,M]=\{1\}$ and hence $M$ is abelian. 
	Now if $M$ is finite, there exists a prime number $p$ such that 
	$P=\{x\in M:x^p=1\}$ is a non-trivial subgroup of $M$.  
	Since $P$ is characteristic in $M$, the subgroup
	$P$ is normal in $G$. Thus $P=M$. 
\end{proof}

\begin{theorem}
	\label{theorem:resoluble}
	Let $G$ be a group. 
	\begin{enumerate}
		\item Each subgroup $H$ of $G$ is solvable. 
		\item Let $K$ be a normal subgroup of $G$. Then $G$ is solvable
			if and only if $K$ and $G/K$ are both solvable.
	\end{enumerate}
\end{theorem}

\begin{proof}
    By induction, $H^{(i)}\subseteq G^{(i)}$ for all 
    $i\geq0$. Let us prove the second claim. Let $Q=G/K$ and $\pi\colon G\to Q$ be the canonical map. 
    By induction we prove that $\pi(G^{(i)})=Q^{(i)}$ for all 
	$i\geq0$. The case $i=0$ is trivial, as $\pi$ is surjective. Now assume that
	the result holds for some $i\geq0$. Then 
	\[
		\pi(G^{(i+1)})=\pi([G^{(i)},G^{(i)}])=[\pi(G^{(i)}),\pi(G^{(i)})]=[Q^{(i)},Q^{(i)}]=Q^{(i+1)}.
	\]

	Assume that $Q$ and $K$ are both solvable. Since $Q$ is solvable, 
	there exists $n$ such that $Q^{(n)}=\{1\}$.
	Since $\pi(G^{(n)})=Q^{(n)}=\{1\}$, it follows that $G^{(n)}\subseteq K$. Since $K$
	is solvable, there exists $m$ such that 
	\[
		G^{(n+m)}\subseteq (G^{(n)})^{(m)}\subseteq K^{(m)}=\{1\},
	\]
	and hence $G$ is solvable.  

	Let us now assume that $G$ is solvable. There exists $n\in\N$ such that $G^{(n)}=\{1\}$.
	Thus $Q$ is solvable, as $Q^{n}=f(G^{(n)})=f(\{1\})=\{1\}$. The group $K$ is also 
	solvable, as it is a subgroup of $G$. 
\end{proof}

\begin{example}
	Let $n\geq5$. The group $\Sym_n$ is  not solvable.
\end{example}

% \begin{example}
% 	Si $H$ y $K$ son grupos resolubles entonces $H\times K$ es resoluble.
% \end{example}

\begin{exercise}
	Let $p$ be a prime number and $G$ be a finite $p$-group. Prove that 
	$G$ is solvable. 
\end{exercise}

% \begin{proof}
% 	Procederemos por inducción en $|G|$. Supongamos que el resultado es válido
% 	para todos los $p$-grupos de orden $<|G|$. Como $Z(G)\ne1$, por hipótesis
% 	inductvia $G/Z(G)$ es un $p$-grupo resoluble.  Como $Z(G)$ es resoluble por
% 	ser un grupo abeliano, $G$ es resoluble por el
% 	teorema~\ref{theorem:resoluble}. 
% \end{proof}

%\begin{exercise}
%	\label{exercise:resoluble:eq}
%	Sea $G$ un grupo. Demuestre que las siguientes afirmaciones son equivalentes:
%	\begin{enumerate}
%		\item $G$ es resoluble.
%		\item $G$ admite una sucesión de subgrupos $G=G_0\supseteq
%			G_1\supseteq\cdots\supseteq G_n=1$ tal que cada $G_i$ es normal en
%			$G$ y cada $G_{i-1}/G_i$ es abeliano.
%		\item $G$ admite una sucesión de subgrupos $G=G_0\supseteq
%			G_1\supseteq\cdots\supseteq G_n=1$ tal que cada $G_i$ es normal en
%			$G_{i-1}$ y cada $G_{i-1}/G_i$ es abeliano.
%	\end{enumerate}
%\end{exercise}
%
%\begin{svgraybox}
%	Para demostrar $(1)\implies(2)$ basta considerar la serie derivada. La
%	implicación $(2)\implies(3)$ es trivial. Para demostrar $(3)\implies(1)$
%	hay que observar que, como $G^{(i)}=[G^{(i-1)},G^{(i-1)}]\subseteq G_i$
%	pues $G_{i-1}/G_i$ es abeliano, $G^{(n)}=1$.
%\end{svgraybox}

To prove Wielandt's theorem on solvable groups 
we need the following lemma.  

\begin{lemma}
	\label{lemma:4Wielandt}
	Let $G$ be a finite group. If $H$ and $K$ are subgroups of $G$ of coprime indices, then 
    $G=HK$ and $(H:H\cap K)=(G:K)$.
\end{lemma}

\begin{proof}
	Let $D=H\cap K$. Since
	\[
	(G:D)=\frac{|G|}{|H\cap K|}=(G:H)(H:H\cap K),
	\]
	$(G:H)$ divides $(G:D)$. Similarly, $(G:K)$ divides 
	$(G:D)$. Since $(G:H)$ and $(G:K)$ are coprime, $(G:H)(G:K)$
	divides $(G:D)$. In particular, 
	\[
	\frac{|G|}{|H|}\frac{|G|}{|K|}=(G:H)(G:K)\leq (G:D)=\frac{|G|}{|H\cap K|}
	\]
	and hence $|G|=|HK|$. Since 
	\[
	|G|=|HK|=|H||K|/|H\cap K|, 
	\]
	it follows that 
	$(G:K)=(H:H\cap K)$.
\end{proof}

\index{Clausura normal}
The 
\textbf{normal closure} $H^G$ of a subgroup $H$ of $G$ is the subgroup
\[
	H^G=\langle xHx^{-1}:x\in G\rangle
\]
generated by all conjugates of $H$. The 
subgroup $H^G$ is the smallest normal subgroup of $G$ containing $H$. 

\begin{example}
	Let $G=\Alt_4$ and $H=\{\id,(12)(34)\}$. Then 
	\[
	H^G=\{\id,(12)(34),(13)(24),(14)(23)\}\simeq C_2\times C_2.
	\]
\end{example}


\begin{theorem}[Wielandt]
	\label{theorem:Wielandt:solvable}
	Let $G$ be a finite group and $H$, $K$ and $L$ be subgroups of $G$ 
	with pair-wise coprime indices. If $H$, $K$ and $L$ are solvable, 
	then $G$ is solvable. 
\end{theorem}

\begin{proof}
	Assume the theorem is not valid and let $G$ be a minimal counterexample. 
	Then $G$ is not trivial.  
	Let $N$ be a minimal-normal subgroup of 
	$G$ and $\pi\colon G\to G/N$, $g\mapsto gN$, be the canonical map. Since 
	by definition $N$ is non-trivial, 
	it follows that $|G/N|<1$. 
	The subgroups 
	$\pi(H)=\pi(HN)$, $\pi(K)=\pi(KN)$ and $\pi(L)=\pi(LN)$ of $\pi(G)=G/N$ are solvable. 
	The correspondence theorem implies that 
	the indices of $\pi(H)$, $\pi(K)$ and $\pi(L)$ in $\pi(G)$ are pair-wise coprime. By the 
	minimality of $G$, the group 
	$\pi(G)$ is solvable. If $H=\{1\}$, then 
	$|G|=(G:H)$ is coprime with $(G:K)$ and hence $G=K$ is solvable. So we may assume that  
	$H\ne \{1\}$. Let $M$ be a minimal normal subgroup of $H$. By
	Lemma~\ref{lemma:minimal_normal}, $M$ is a $p$-group for some prime number $p$. We may
	assume that $p$ does not divide 
	$(G:K)$ (if $p$ divides $(G:K)$, then $p$ does no divide $(G:L)$ and 
	hence it is enough to replace $K$ by $L$). There exists 
	$P\in\Syl_p(G)$ such that $P\subseteq K$. By Sylow's theorem, 
	there exists  $g\in G$ such that $M\subseteq
	gKg^{-1}$. Since $(G:gKg^{-1})=(G:K)$ and $(G:H)$ are coprime, 
	Lemma~\ref{lemma:4Wielandt} implies that $G=(gKg^{-1})H$. 
	
	We claim that all conjugate of $M$ are included in $gKg^{-1}$. 
	If $x\in G$, then $x=uv$ for some $u\in 
	gKg^{-1}$ and $v\in H$. Since $M$ is normal in $H$, 
	\[
	xMx^{-1}=(uv)M(uv)^{-1}=uMu^{-1}\subseteq gKg^{-1}.
	\]
	In particular, $\{1\}\ne M^G\subseteq gKg^{-1}$ is solvable, as $gKg^{-1}$ is
	solvable. The minimality of $G$ implies that $G/M^G$ is solvable. Hence 
	$G$ is solvable by Theorem~\ref{theorem:resoluble}.
\end{proof}


\index{$p$-complemento}
Let $G$ be a finite group of order $p^{\alpha}m$ with $p$ a prime number coprime with $m$. 
A subgroup $H$ of $G$ is a \textbf{$p$-complement} if $|H|=m$. 

\begin{example}
	Sea $G=\Sym_3$. Then $H=\langle (123)\rangle$ is a $2$-complement 
	and $K=\langle (12)\rangle$ is a $3$-complement.
\end{example}

A famous theorem of Burnside states that finite groups whose order are divisible 
by exactly two primes are solvable. 

\begin{theorem}[Burnside]
\index{Burnside!Theorem}
Let $p$ and $q$ be prime numbers and let $G$ be a group of order $p^\alpha q^\beta$. Then
$G$ is solvable. 
\end{theorem}

There is a quite easy proof that uses basic character theory, see for example.
A non-character-theoretic proof is known but it is harder, see~\cite{MR2426855}.

\begin{theorem}[Hall]
	\label{theorem:Hall:solvable}
	Let $G$ be a finite group that admits a $p$-complement for all primes
	$p$ dividing the order of $G$. Then $G$ is solvable. 
\end{theorem}

\begin{proof}
	Let $|G|=p_1^{\alpha_1}\cdots
	p_k^{\alpha_k}$ with the $p_j$ being distinct primes. We proceed by induction on 
	$k$. If $k=1$, then $G$ is a $p$-group and the result is clear. If $k=2$, then 
	Burnside's theorem implies the claim. Assume now that $k\geq3$. For each 
	$j\in\{1,2,3\}$ let $H_j$ be $p_j$-complement in
	$G$. Since $|H_j|=|G|/p_j^{\alpha_j}$, the subgroups $H_j$ have coprime indices. 
	
	We claim that $H_1$ is solvable. Note that $|H_1|=p_2^{\alpha_2}\cdots
	p_k^{\alpha_k}$. Let $p$ be a prime number that divides $|H_1|$ and let $Q$ be a 
	$p$-complement in $G$. 
	Since $(G:H_1)$ and $(G:Q)$ are coprime, Lemma~\ref{lemma:4Wielandt} implies that 
	\[
	(H_1:H_1\cap Q)=(G:Q).
	\]
	Thus $H_1\cap Q$ is a $p$-complement in $H_1$. Hence $H_1$ is
	solvable by the inductive hypothesis. Similarly, $H_2$ and 
	$H_3$ are both solvable. 
	
	Since $H_1$, $H_2$ and $H_3$ are solvable and have coprime indices, Wielandt's theorem
	implies the claim. 
\end{proof}

\section*{B}

We now use Hall's theorem to obtain information related to the
structure of finite braces. 

\begin{theorem}
\label{thm:add_nilpotent}
Let $A$ be a finite brace of nilpotent type. Then 
the multiplicative group of $A$ is solvable.
\end{theorem}

\begin{proof}
    Let $K$ be the additive group of $A$ and $G$ be the multiplicative group of $A$. Assume
    that $|A|=p_1^{\alpha_1}\cdots p_n^{\alpha}$ for different primes numbers $p_1,\dots,p_n$. 
    Since $K$ is nilpotent, each $K_i\in\Syl_{p_j}(K)$ is normal in $K$, so 
    each $K_i$ is a left ideal of $A$. It follows that for each $i\in\{1,\dots,n\}$ both $K_i$ and 
    $\prod_{j\ne i}K_j$ are braces of coprime order. In particular, for 
    each $i\in\{1,\dots,n\}$ there exists a subgroup of $G$ of order coprime with $p$. 
    Then $G$ is solvable by Hall's theorem. 
\end{proof}


% The previous theorem is not true in the case of finite braces of non-nilpotent type. 

% \begin{theorem}
% \label{thm:Smoktunowicz}
%     Let $A$ be a finite skew left brace of nilpotent type. Then $A$ is
%     left nilpotent if and only if the multiplicative group of $A$ is nilpotent.
% \end{theorem}

% \begin{proof}
%     Proposition~\ref{pro:left_p} and Theorem~\ref{thm:left_p} prove the theorem.
% \end{proof}

%The following theorem was found by Smoktunowicz in the case of braces of abelian type. 

\section*{Exercises}

\begin{prob}
\label{prob:G(X,r)solvable}
Let $(X,r)$ be a finite involutive solution. Prove that $G(X,r)$ is solvable. 
\end{prob}

% \begin{prob}
% Is the result of Exercise~\ref{prob:G(X,r)solvable} true in the case of non-involutive finite solutions?
% \end{prob}

\section*{Notes}

Solvable groups...

In~\cite{MR1722951}, Etingof, Schedler and Soloviev proved that the structure group of a finite involutive
solution is always solvable. 
% The proof can translated into the language of braces 
% to obtain Theorem~\ref{thm:add_nilpotent}.

