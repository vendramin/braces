\chapter{Solvable groups}
\label{solvable}

%In this chapter we develop the basic definitions and results on solvable groups to prove the classical results of P. Hall. 


\section{Derived series}

\index{Derived!series}
\index{Group!solvable}
\index{Group!simple}
For a group $G$ we define 
\[
		G^{(0)}=G,\quad
		G^{(i+1)}=[G^{(i)},G^{(i)}]\quad i\geq0.
\]
The {\em derived series} of $G$ is the sequence 
\[
G=G^{(0)}\supseteq G^{(1)}\supseteq G^{(2)}\supseteq\cdots
\]
Each $G^{(i)}$ is a characteristic subgroup of $G$. We say that 
$G$ is {\em solvable} if $G^{(n)}=\{1\}$ for some  $n\in\N$. Clearly every abelian group
is solvable. A non-trivial group $G$ is said to be {\em simple} if $\{ 1\}$ and $G$ are the only normal subgroups of $G$. Note that a non-abelian simple group cannot be solvable. Nilpotent groups
are solvable.


\index{Elementary abelian $p$-group}
Let $p$ be a prime number. An {\em elementary abelian} $p$-group is \textcolor{red}{an abelian} $p$-group 
$P$ such that $x^p=1$ for all $x\in P$.
%A subgroup $M$ of a group $G$ is said to be {\em minimal normal} if $M\ne\{1\}$,
%$M$ is normal in $G$ and the unique normal subgroup of $G$ strictly contained in $M$ is
%the trivial subgroup. Every finite group contains a minimal normal subgroup.  

%\begin{example}
%	If a {\bf non-trivial} normal subgroup $M$ is minimal (with respect to the inclusion), then it is
%	minimal normal. The converse statement is not true. The subgroup of 
%	$\Alt_4$ generated by $(12)(34)$, $(13)(24)$ and $(14)(23)$ is minimal normal in 
%	$\Alt_4$ but it is not minimal {\bf non-trivial}. 
%\end{example}

%\begin{example}
%	Let $G=\D_{6}=\langle r,s:r^6=s^2=1,\,srs=r^{-1}\rangle$ be the dihedral group
%	of size twelve. The subgroups $S=\langle r^2\rangle$ 
%	and $T=\langle r^3\rangle$ are minimal normal subgroups. 
%\end{example}



\begin{lemma}
	\label{lem:minimal_normal}
	Let $G$ be a non-trivial group.
	Let $M$ be a minimal normal subgroup of $G$. If $M$ is solvable and finite, then
	$M$ is an elementary abelian $p$-group for some prime number $p$. 
\end{lemma}

\begin{proof}
	Since $M$ is solvable, $[M,M]\subsetneq M$. Moreover, $[M,M]$ is normal in $G$, as 
    $[M,M]$ is characteristic in $M$ and $M$ is normal in $G$. Since $M$ is minimal normal, 
    it follows that $[M,M]=\{1\}$ and hence $M$ is abelian. 
	Now,  since $M$ is finite, there exists a prime number $p$ such that 
	$P=\{x\in M:x^p=1\}$ is a non-trivial subgroup of $M$.  
	Since $P$ is characteristic in $M$, the subgroup
	$P$ is normal in $G$. Thus $P=M$. 
\end{proof}



\begin{theorem}
	\label{theorem:resoluble}
	Let $G$ be a group. 
	\begin{enumerate}
		\item  If $G$ is solvable, then each subgroup $H$ of $G$ is solvable. 
		\item Let $K$ be a normal subgroup of $G$. Then $G$ is solvable
			if and only if $K$ and $G/K$ are both solvable.
	\end{enumerate}
\end{theorem}

\begin{proof}
    By induction one proves that $H^{(i)}\subseteq G^{(i)}$ for all 
    $i\geq0$. Let us prove the second claim. Let $Q=G/K$ and let $\pi\colon G\to Q$ be the canonical map. 
    By induction we prove that $\pi(G^{(i)})=Q^{(i)}$ for all 
	$i\geq0$. The case $i=0$ is trivial, as $\pi$ is surjective. Now assume that
	the result holds for some $i\geq0$. Then 
	\[
		\pi(G^{(i+1)})=\pi([G^{(i)},G^{(i)}])=[\pi(G^{(i)}),\pi(G^{(i)})]=[Q^{(i)},Q^{(i)}]=Q^{(i+1)}.
	\]

	Assume that $Q$ and $K$ are both solvable. Since $Q$ is solvable, 
	there exists $n$ such that $Q^{(n)}=\{1\}$.
	Since $\pi(G^{(n)})=Q^{(n)}=\{1\}$, it follows that $G^{(n)}\subseteq K$. Since $K$
	is solvable, there exists $m$ such that 
	\[
		G^{(n+m)}= (G^{(n)})^{(m)}\subseteq K^{(m)}=\{1\},
	\]
	and hence $G$ is solvable.  

	Let us now assume that $G$ is solvable. There exists $n\in\N$ such that $G^{(n)}=\{1\}$.
	Thus $Q$ is solvable, as $\textcolor{red}{Q^{(n)}}=\pi (G^{(n)})=\pi(\{1\})=\{1\}$. The group $K$ also is 
	solvable, as it is a subgroup of $G$. 
\end{proof}

\begin{theorem}
	\label{theorem:F(G)centraliza}
	Let $G$ be a finite non-trivial solvable group and let $N$ be a minimal normal subgroup of $G$. Then
    $N\subseteq Z(F(G))$ and thus $F(G)\subseteq C_G(N)$.
\end{theorem}

\begin{proof}
	Let $N$ be a minimal normal subgroup of $G$. By Theorem \ref{theorem:resoluble}, $N$ is solvable. 
	By Lemma \ref{lem:minimal_normal}, $N$ is an elementary abelian $p$-group for some prime number $p$.
	In particular, $N$ is a normal nilpotent subgroup of $G$. By Fitting's theorem, $N\subseteq F(G)$ and $F(G)$ is a normal nilpotent subgroup of $G$. By Hirsch's theorem $N\cap Z(F(G))\neq\{ 1\}$. Since $Z(F(G))$ is a characteristic subgroup of $F(G)$ and $F(G)$ is normal in $G$, we have that $Z(F(G))$ is a normal subgroup of $G$.
	Hence $N\cap Z(F(G))$ also is a normal subgroup of $G$.
	Since $N$ is a minimal normal subgroup of $G$ and $N\cap Z(F(G))$ is non-trivial, we have that $N=N\cap Z(F(G))$.
	Therefore the result follows.
\end{proof}



%\begin{example}
%	Let $n\geq5$. The group $\Sym_n$ is  not solvable.
%\end{example}

% \begin{example}
% 	Si $H$ y $K$ son grupos resolubles entonces $H\times K$ es resoluble.
% \end{example}

%\begin{exercise}
%	Let $p$ be a prime number and $G$ be a finite $p$-group. Prove that 
%	$G$ is solvable. 
%\end{exercise}

% \begin{proof}
% 	Procederemos por inducción en $|G|$. Supongamos que el resultado es válido
% 	para todos los $p$-grupos de orden $<|G|$. Como $Z(G)\ne1$, por hipótesis
% 	inductvia $G/Z(G)$ es un $p$-grupo resoluble.  Como $Z(G)$ es resoluble por
% 	ser un grupo abeliano, $G$ es resoluble por el
% 	teorema~\ref{theorem:resoluble}. 
% \end{proof}

%\begin{exercise}
%	\label{exercise:resoluble:eq}
%	Sea $G$ un grupo. Demuestre que las siguientes afirmaciones son equivalentes:
%	\begin{enumerate}
%		\item $G$ es resoluble.
%		\item $G$ admite una sucesión de subgrupos $G=G_0\supseteq
%			G_1\supseteq\cdots\supseteq G_n=1$ tal que cada $G_i$ es normal en
%			$G$ y cada $G_{i-1}/G_i$ es abeliano.
%		\item $G$ admite una sucesión de subgrupos $G=G_0\supseteq
%			G_1\supseteq\cdots\supseteq G_n=1$ tal que cada $G_i$ es normal en
%			$G_{i-1}$ y cada $G_{i-1}/G_i$ es abeliano.
%	\end{enumerate}
%\end{exercise}
%
%\begin{svgraybox}
%	Para demostrar $(1)\implies(2)$ basta considerar la serie derivada. La
%	implicación $(2)\implies(3)$ es trivial. Para demostrar $(3)\implies(1)$
%	hay que observar que, como $G^{(i)}=[G^{(i-1)},G^{(i-1)}]\subseteq G_i$
%	pues $G_{i-1}/G_i$ es abeliano, $G^{(n)}=1$.
%\end{svgraybox}

\section{Burnside's theorem}

In this section we shall prove an important theorem of Burnside, which says that every finite group of order $p^nq^m$, for some primes $p,q$ and positive integers $m,n$, is solvable. For its proof, we shall introduce some results of representation theory and integral extensions. 

\index{Algebra}
Let $K$ be a field. A {\em $K$-algebra} is a $K$-vector space $R$ joint with a multiplication on $R$ such that
$(R,+,\cdot)$ is a ring and $\alpha(a\cdot b)=(\alpha a)\cdot b=a\cdot (\alpha b)$, for all $\alpha\in K$ and $a,b\in R$.

\index{Group algebra}
Let $G$ be a group and $K$ a field. The {\em group algebra} $K[G]$ of the group $G$ over the field $K$ is a $K$-vector space with basis $G$
with a multiplication on $K[G]$ defined by
\[ \left(\sum_{i=1}^na_ix_i\right)\left(\sum_{j=1}^mb_jy_j\right)=\sum_{i=1}^n\sum_{j=1}^m(a_ib_j)(x_iy_j),\]
for all $a_1,\dots, a_n,b_1,\dots ,b_m\in K$ and $x_1,\dots ,x_n,y_1,\dots ,y_m\in G$.
We will write the elements of $a\in K[G]$ in the form
\[ a=\sum_{x\in G}a_xx,\]
with $a_x\in K$, assuming that $\{ x\in G\mid a_x\neq 0\}$ is finite. With this notation, we have that
\[\left(\sum_{x\in G}a_xx\right)\left(\sum_{x\in G}b_xx\right)=\sum_{x\in G}c_xx,\]
where
\[ c_x=\sum_{y\in G}a_yb_{y^{-1}x},\]
for all $x\in G$. One can easily check that the $K$-vector space $K[G]$ with this multiplication is a $K$-algebra. Note that the unit-element $1_G$ of $G$ is the unit-element of the ring $(K[G],+,\cdot)$. The map $K\to K[G]:\alpha\mapsto \alpha 1_G$ is an injective ring homomorphism, and we will identify the elements $\alpha\in K$ with $\alpha 1_G\in K[G]$, thus $K\subseteq K[G]$. We will denote the unit-element of $K[G]$ by $1$.

\begin{definition}\index{Representation}
\index{Representation!faithful}
A {\em (linear) representation} of a group $G$ over a field $K$ is a group homomorphism 
\[\rho\colon G\rightarrow \Aut(V),\]
Where $V$ is a $K$-vector space. If $\dim(V)=n<\infty$, then we say that $\rho$ has degree $n$. It is said that $\rho$ is {\em faithful} if $\ker(\rho)=\{ 1\}$.
\end{definition}

Let $\rho\colon G\rightarrow\Aut(V)$ be a representation of a group $G$ over a field $K$ of finite degree $n$. 
Then, fixing a basis $\mathcal{B}$ of $V$, we have a group isomorphism $\psi_{\mathcal{B}}\colon \Aut(V)\rightarrow \GL_n(K)$, where $\GL_n(K)=\{ A\in M_n(K)\mid A\mbox{ is invertible}\}$, defined by $\psi_{\mathcal{B}}(f)=M(f,\mathcal{B})$, the matrix
of $f$ with respect the basis $\mathcal{B}$, for all $f\in \Aut(V)$. We say that $\psi_{\mathcal{B}}\circ \rho\colon G\rightarrow \mathbf{GL}_n(K)$ is a matricial representation of $G$ over $K$. The map $\chi\colon G\rightarrow K$ defined by 
$\chi(x)=\mathrm{tr}(\psi_{\mathcal{B}}( \rho(x)))$, the trace of the matrix $\psi_{\mathcal{B}}( \rho(x))$, for all $x\in G$, is called \index{Character}
the {\em character} of $\rho$. It is easy to see that $\chi$ is independent of the choice of the basis $\mathcal{B}$ of $V$.

\begin{example}
Let $G$ be a finite group of order $n$. Let $K$ be a field. Then $K[G]$ is a $K$-vector space of dimension $n$. The map
\[ \rho\colon G\rightarrow \Aut_K(K[G]),\]
defined by $\rho(x)(a)=xa$, for all $x\in G$ and $a\in K[G]$, is a representation of $G$ over $K$ of degree $n$, and it is called the regular representation of $G$ over $K$. If $\chi$ is the character of $\rho$, then $\chi(1)=n$ and $\chi(x)=0$, for all $x\in G\setminus\{ 1\}$. 
\end{example}

Given a representation $\rho\colon G\rightarrow \Aut(V)$ of a group $G$ over a field $K$, we define 
\[\widehat{\rho}\colon K[G]\rightarrow \End_K(V),\]
by $\widehat{\rho}\left(\sum_{x\in G}a_xx\right)=\sum_{x\in G}a_x\rho(x)$, for all $\sum_{x\in G}a_xx\in K[G]$. Note that
$\widehat{\rho}$ is a ring homomorphism. We define on $V$ a multiplication by elements of $K[G]$ by
\[ \left(\sum_{x\in G}a_xx\right)\cdot v:=\sum_{x\in G}a_x\rho(x)(v)\]
for all $\sum_{x\in G}a_xx\in K[G]$ and $v\in V$. One can check that $V$ with the sum and this product by elements of $K[G]$ is a $K[G]$-module. This is the $K[G]$-module corresponding to the representation $\rho$.

Conversely, if $W$ is a $K[G]$-module, then $W$ also is a $K$-vector space, and the map $\rho\colon G\rightarrow\Aut_K(W)$, 
defined by $\rho(x)(w)=xw$, for all $x\in G$ and $w\in W$, is a representation of $G$ over $K$. We say that $\rho$ is the representation of $G$ over $K$ corresponding to the $K[G]$-module $W$.

One can easily check that this gives a bijective correspondence between the representations of $G$ over $K$ and the $K[G]$-modules.

\begin{lemma}\label{equivrep}
Let $G$ be a group and $K$ a field. Let $M_1,M_2$ be isomorphic $K[G]$-modules of finite dimension over $K$. Let $\rho_1,\rho_2$ be the representations over $K$ corresponding to $M_1,M_2$ respectively. Let $\chi_1,\chi_2$ be the characters of $\rho_1,\rho_2$ respectively. Then $\chi_1(g)=\chi_2(g)$, for all $g\in G$.
\end{lemma}

\begin{proof}
    Let $f: M_1\rightarrow M_2$ be an isomorphism of $K[G]$-modules. Then
    \[ \rho_2(g)(f(m))=gf(m)=f(gm)=f(\rho_1(g)(m)),\]
    for all $g\in G$ and $m\in M_1$. Hence
    \[ \rho_2(g)=\textcolor{red}{f \rho_1(g) f^{-1}}.\]
    Therefore $\chi_2(g)=\chi_1(g)$, and the result follows.
\end{proof}
 
\index{Representation!irreducible} 
\begin{definition}
A representation of a group $G$ over a field $K$ is said to be {\em irreducible} if its corresponding $K[G]$-module is simple.
\end{definition}

%\index{Module!semisimple}
%\begin{definition}
%Let $R$ be a ring. An $R$-module is {\em semisimple} if it is a direct sum of simple submodules. We say that $R$ is (left) semisimple is $R$ is semisimple as (left) $R$-module, that is if $R$ is the direct sum of minimal nonzero left ideals. 
%\end{definition}

%\begin{example}
%Let $D$ be a division ring. For every positive integer $n$, $M_n(D)$ is a semisimple ring. In fact, if $e_{i,j}\in M_n(D)$ is the matrix with $1$ in the $(i,j)$-entry and $0$ in the other entries, then $M_n(D)e_{i,j}$ is a minimal nonzero left ideal of $M_n(D)$, and
%\[ M_n(D)=\bigoplus_{j=1}^nM_n(D)e_{j,j}.\]
%\end{example}

%\begin{example}\label{semisimple}
%Let $R$ be an Artinian semiprimitive ring. The by Wedderburn-Artin theorem,
%there exist positive integers $n_1,\dots ,n_r$ and division rings $D_1,\dots ,D_r$ such that
%\[ R\cong M_{n_1}(D_1)\times\dots\times M_{n_r}(D_r).\]
%Since each ring $M_{n_i}(D_i)$ is semisimple, it is easy to see that $M_{n_1}(D_1)\times\dots\times M_{n_r}(D_r)$ also is semisimple, 
%and thus $R$ is semisimple.
%\end{example}

\begin{theorem}[Maschke's theorem]
\index{Maschke}\index{Maschke's theorem}
Let $G$ be a finite group. Let $K$ be a field of characteristic $p\geq 0$. If $p$ is not a divisor of $|G|$, then $J(K[G])=\{0\}$ and therefore $K[G]$ is semisimple.
\end{theorem}

\begin{proof}
    Since $K[G]$ is finite dimensional, it is Artinian. By \textcolor{red}{Wedderburn-Artin's theorem}, it is enough to show that $J(K[G])=\{ 0\}$. Let $W$ be a vector subspace of $K[G]$ such that
    $K[G]=J(K[G])\oplus W$. Let $\pi\colon K[G]\rightarrow K[G]$ be the map defined by $\pi(a+b)=a$, for all $a\in J(K[G])$ and $b\in W$.
    It is clear that $\pi$ is a $K$-linear map. We define $\pi^*\colon K[G]\rightarrow K[G]$ by
    \[\pi^*(c)=\frac{1}{|G|}\sum_{x\in G}x^{-1}\pi(xc),\]
    for all $c\in K[G]$. Note that $\pi^*$ is a $K$-linear map. Let $c\in K[G]$ and $y\in G$. We have that
    \begin{align*}
        \pi^*(yc)=&\frac{1}{|G|}\sum_{x\in G}x^{-1}\pi(xyc)\\
        =&\frac{y}{|G|}\sum_{x\in G}y^{-1}x^{-1}\pi(xyc)\\
        =& y\pi^*(c).
    \end{align*}
    Hence $\pi^*$ is a homomorphism of $K[G]$-modules. Furthermore, $\pi^*(K[G])\subseteq J(K[G])$. Let $a\in J(K[G])$. Since
    $\pi(xa)=xa$, for all $x\in G$, we have that $\pi^*(a)=a$. Hence $\pi^*(K[G])=J(K[G])$ and $(\pi^*)^2=\pi^*$. Thus $\ker(\pi^*)=\im(\id-\pi^*)$ and $K[G]=J(K[G])\oplus \ker(\pi^*)$. Now there exist $a\in J(K[G])$ and $b\in \ker(\pi^*)$ such that
    $1=a+b$. Hence $a=a^2+ab$. Since $ab\in J(K[G])\cap\ker(\pi^*)=\{ 0\}$, we get that $a=a^2$. Hence $(1-a)a=0$. Since $1-a$ is invertible, we have that $a=0$. Hence $1=b\in \ker(\pi^*)$, and thus $\ker(\pi^*)=K[G]$. Therefore $J(K[G])=\{ 0\}$, and the result follows. 
\end{proof}

\index{Ring!center}
The {\em center} of a ring $R$ is $Z(R)=\{ a\in R\mid ab=ba, \mbox{ for all }b\in R\}$.

\index{Conjugate class}
Let $G$ be a group. The {\em conjugate class} of an element $x\in G$ is $C_x=\{ yxy^{-1}\mid y\in G\}$. Note that if $C_x$ is finite, then $|C_x|=(G:C_G(x))$. In fact, if $T$ is a left transversal of $C_G(x)$ in $G$, then $C_x=\{ yxy^{-1}\mid y\in T\}$, and if $yxy^{-1}=zxz^{-1}$, for $y,z\in T$, then $z^{-1}y\in C_G(x)$, and thus $y=z$. 

\begin{lemma}\label{basiscenter}
    Let $G$ be a finite group. Let $K$ be a field. Let $C_1,\dots, C_r$ be the distinct conjugacy classes of $G$. Let
    \[ \alpha_i=\sum_{x\in C_i}x.\]    
    Then $\alpha_1,\dots ,\alpha_r$ is a $K$-basis of $Z(K[G])$.
\end{lemma}

\begin{proof}
    It is clear that $\alpha_1,\dots, \alpha_r$ are $K$-linearly independent. Since
    \[ y\alpha_i y^{-1}=\sum_{x\in C_i}yxy^{-1}=\sum_{x\in C_i}x=\alpha_i\]
    for all $y\in G$, we have that $\alpha_i\in Z(K[G])$. Let $b=\sum_{x\in G}b_xx\in Z(K[G])$.
    Then, for every $y\in G$,
    \[ b=yby^{-1}=\sum_{x\in X}b_xyxy^{-1}.\]
    Hence $b_x=b_{yxy^{-1}}$ for all $x,y\in G$. For every $i=1,\dots ,r$, choose an element $x_i\in C_i$. Now we have that
    \[ b=\sum_{i=1}^rb_{x_i}\alpha_i,\]
    and the result follows.
\end{proof}

\begin{definition}\index{Integral element}
Let $R$ be a subring of a commutative ring $S$. An element $a\in S$ is said to be {\em integral} over $R$ if 
it is a root of a monic polynomial
\[ a_0+a_1x+\dots +a_{n-1}x^{n-1}+x^n\in R[x],\]
for some positive integer $n$.
\end{definition}

\begin{proposition}\label{integralelement}
Let $S$ be a commutative integral domain. Let $R$ be a subring of $S$. For any $a\in S$, the following conditions are equivalent.
\begin{enumerate}
    \item $a$ is integral over $R$.
    \item $R[a]$ is a finitelly generated $R$-module.
    \item There is a non-zero finitely generated $R$-submodule $M$ of $S$ such that $aM\subseteq M$.
\end{enumerate}
\end{proposition}

\begin{proof}
    $1)\implies 2).$ Suppose that $a$ is integral over $R$. Then there exist a positive integer $n$ and $a_0,\dots ,a_{n-1}\in R$ 
    such that
    \[ a_0+a_1a+\dots +a_{n-1}a^{n-1}+a^n=0.\]
    We shall prove by induction on $m$ that $a^m\in R+Ra+\dots +a^{n-1}R$. If $m<n$, then it is clear that 
    $a^m\in R+Ra+\dots +a^{n-1}R$. Assume that $m\geq n$ and that $a^k\in R+Ra+\dots +a^{n-1}R$, for all $k<m$. We have that
    \[ a^m=-a_0a^{m-n}-a_1a^{m-n+1}-\dots -a_{n-1}a^{m-1}.\]
    Hence, by the inductive hypothesis, $a^m\in R+Ra+\dots +a^{n-1}R$. Thus, by induction, $R[a]=R+Ra+\dots +Ra^{n-1}$.
    
    $2)\implies 3)$. Suppose that $R[a]$ is a finitely generated $R$-module. Note that $aR[a]\subseteq R[a]$. Thus $(3)$ follows.
    
    $3)\implies 1)$. Suppose that there exists a non-zero finitely generated $R$-submodule $M$ of $S$ such that $aM\subseteq M$. 
    Let $s_1,\dots ,s_m\in S$ be elements such that
    $M=Rs_1+\dots +Rs_m$. Hence, there exist elements $a_{i,j}\in R$ such that
    \[ as_i=\sum_{j=1}^ma_{i,j}s_j,\]
    that is
    \[\sum_{j=1}^m(a\delta_{i,j}-a_{i,j})s_j=0\]
    for all $i=1,\dots ,m$. Since $M\neq\{0\}$, some $s_j\neq 0$. Therefore
    \[\det\left(\begin{array}{cccc}
    a-a_{1,1}&-a_{1,2}&\ldots&-a_{1,m}\\
    -a_{2,1}&a-a_{2,2}&\ddots&\vdots\\
    \vdots&\ddots&\ddots&-a_{m-1,m}\\
    -a_{m,1}&\ldots&-a_{m,m-1}&a-a_{m,m}\end{array}\right)=0.\]
    Therefore $a$ is integral over $R$.
\end{proof}

\begin{corollary}\label{integralclousure}
    Let $R$ be a subring of a commutative integral domain $S$. Then 
    \[ \{ a\in S\mid a \mbox{ is integral over }R\}\]
    is a subring of $S$ containing $R$.
\end{corollary}

\begin{proof}
    Let $T=\{ a\in S\mid a \mbox{ is integral over }R\}$. It is clear that $R\subseteq T$. Let $a,b\in T$. 
    Since $a$ is integral over $R$,  by Proposition~\ref{integralelement}, $R[a]$ is a finitely generated $R$-submodule of $S$.
    Since $b$ is integral over $R$, $b$ also is integral over $R[a]$. Hence, by Proposition~\ref{integralelement}, 
    $R[a,b]$ is a finitely generated $R[a]$-submodule of $S$. Since $R[a]$ is finitely generated as $R$-module, we have that
    $R[a,b]$ also is finitely generated as $R$-module. Since $(a-b)R[a,b]\subseteq R[a,b]$ and $abR[a,b]\subseteq R[a,b]$, 
    by Proposition~\ref{integralelement}, $a-b,ab\in T$. Therefore, the result follows. 
\end{proof}

\begin{example}\label{Zintclosed}
We shall see that $\Z=\{ a\in \Q\mid a \mbox{ is integral over }\Z\}$. Let $a\in \Q$ be integral over $\Z$. We shall see that $a\in\Z$.
We may assume that $a=b/c$, where $b,c$ are nonzero coprime integers. Since $a$ is integer over $\Z$, there exist a positive integer $n$ and $a_0,\dots ,a_{n-1}\in \Z$ such that
\[ a_0+a_1\frac{b}{c}+\dots +a_{n-1}\frac{b^{n-1}}{c^{n-1}}+\frac{b^{n}}{c^{n}}=0.\]
Hence
\[-b^n=c(a_0c^{n-1}+\dots +a_{n-1}b^{n-1}).\]
Since $b,c$ are coprime, we get that $c=\pm 1$, and thus $a\in \Z$.
\end{example}

\index{Field!algebraically closed}
A field $K$ is said to be {\em algebraically closed} if every polynomial $p(x)\in K[x]$ of positive degree has a root in $K$.

\begin{proposition}\label{algclosedfield}
Let $K$ be an algebraically closed field. Let $R$ be a semiprimitive finite dimensional $K$-algebra.
Then
\[ R\cong M_{n_1}(K)\times\dots\times M_{n_r}(K),\]
for some positive integers $n_1,\dots ,n_r$.
\end{proposition}

\begin{proof}
    Since $\dim_K(R)<\infty$, $R$ is Artinian. By Wedderburn-Artin's theorem there exist positive integers $n_1,\dots ,n_r$
    and division rings $D_1,\dots ,D_r$ such that
    \[ R\cong M_{n_1}(D_1)\times\dots\times M_{n_r}(D_r).\]
    Furthermore, each $D_i$ is a finite dimensional $K$-algebra. We may assume that $K\subseteq D_i$. Let $a\in D_i$. Since
    $\dim_K(D_i)<\infty$, $a$ is algebraic over $K$. Hence $K[a]$ is an algebraic field extension of $K$. Since $K$ is algebraically closed, $K[a]=K$. Therefore $D_i=K$, and the result follows.
\end{proof}

\begin{proposition}\label{integraloverZ}
Let $K$ be an algebraically closed field of characteristic zero. Let $G$ be a finite group. Let $\rho$ be an 
irreducible representation of $G$ over $K$ of finite degree $n$. Let $\chi$ be the character of $\rho$. If $g\in G$ has $l$ conjugate elements
in $G$, then 
\[\frac{l\chi(g)}{n}\]
is integral over $\Z$.
\end{proposition}

\begin{proof}
    Let $C_1,\dots ,C_r$ be the distinct conjugacy classes of $G$. We may assume that $C_1=\{ 1\}$. 
    Let $\alpha_i=\sum_{x\in C_i}x$. By Lemma~\ref{basiscenter}, $\alpha_1,\dots ,\alpha_r$ is a $K$-basis of $Z(K[G])$.
    For each $i$, fix an element $x_i\in C_i$. Since $\alpha_i\alpha_j\in Z(K[G])$, 
    \begin{equation}\label{charintegral} \alpha_i\alpha_j=\sum_{k=1}^rm_{i,j,k}\alpha_k,\end{equation}
    where 
    \[ m_{i,j,k}=|\{ (x,y)\in C_i\times C_j\mid xy=x_k\}|.\]
    Let $M$ be the $K[G]$-module corresponding to $\rho$. Since $\rho$ is irreducible, $M$ is simple. By Schur's lemma,
    $\End_{K[G]}(M)$ is a division ring. Since $\dim_K(M)=n$, we have that $\dim_K(\End_{K[G]}(M))\leq n^2$.
    Thus $\End_{K[G]}(M)$ is a finite dimensional $K$-algebra. As we have seen in the proof of Proposition~\ref{algclosedfield},
    we have that $\End_{K[G]}(M)\cong K$. Since $\alpha_i\in Z(K[G])$,
    \[ \widehat{\rho}(\alpha_i)=\sum_{x\in C_i}\rho(x)\in\End_{K[G]}(M).\]
    Hence there exists $\lambda_i\in K$ such that
    \[\widehat{\rho}(\alpha_i)(m)=\lambda_im,\]
    for all $m\in M$. Now we have that
    \[ |C_i|\chi(x_i)=\sum_{x\in C_i}\chi(x)=\mathrm{tr} (\widehat{\rho}(\alpha_i))=n\lambda_i.\]
    Hence 
    \[\lambda_i=\frac{|C_i|\chi(x_i)}{n}.\]
    Applying $\widehat{\rho}$ to (\ref{charintegral}), we get
    $\lambda_i\lambda_j=\sum_{k=1}^rm_{i,j,k}\lambda_k$. Hence
    \[ \sum_{k=1}^r(\lambda_i\delta_{i,j}-m_{i,j,k})\lambda_k=0,\]
    for all $j=1,\dots, r$. Since $\alpha_1=1$, we have that $\lambda_1=1\neq 0$. Hence
    \[\det\left(\begin{array}{cccc}
    \lambda_i-m_{i,1,1}&-m_{i,1,2}&\ldots &-m_{i,1,r}\\
    -m_{i,2,1}&\lambda_i-m_{i,2,2}&\ddots&\vdots\\
    \vdots&\ddots&\ddots&-m_{i,r-1,r}\\
    -m_{i,r,1}&\ldots&-m_{i,r,r-1}&\lambda_i-m_{i,r,r}
    \end{array}\right)=0.\]
    Therefore $\lambda_i$ is integral over $\Z$. Note that if $g\in C_i$, then there exists $y\in G$ such that $g=yx_iy^{-1}$, and thus
    \[\lambda_i=\frac{|C_i|\chi(x_i)}{n}=\frac{l\chi(g)}{n}.\]
    Therefore, the result follows.
\end{proof}

Note that if $X$ is a non-empty set and $V$ is a vector space over a field $F$, then the set 
$\mathrm{Map}(X,V)$ of all maps $f\colon X\rightarrow V$ with the sum and the product by elements of $F$ defined by
\[(f+g)(x)=f(x)+g(x),\quad (\lambda f)(x)=\lambda f(x)\]
for all $f,g\in\mathrm{Map}(X,V)$, $\lambda\in F$ and $x\in X$, is an $F$-vector space.

\begin{lemma}[Dedekind's lemma] \index{Dedekind's lemma}
Any family of distinct homomorphisms of a field $K$ into another field $F$ is $F$-linearly independent.
\end{lemma}

\begin{proof}
    Suppose that the result is not true. Then there exist a smallest integer $n>1$, distinct homomorphisms $f_1,\dots, f_n$ of $K$ into $F$ and nonzero $a_1,\dots ,a_n\in F$, such that
    \[ a_1f_1+\dots +a_nf_n=0.\]
    Hence
    \begin{equation}\label{Dedekind1}
    a_1f_1(x)+\dots +a_nf_n(x)=0,
    \end{equation}
    for all $x\in K$. Let $x,y\in K$. We have
    \begin{align*}
        0=&a_1f_1(yx)+\dots +a_nf_n(yx)\\
        =&a_1f_1(y)f_1(x)+\dots +a_nf_n(y)f_n(x).
    \end{align*}
    Multiplying (\ref{Dedekind1}) by $f_1(y)$, we get
    \[0=a_1f_1(y)f_1(x)+\dots +a_nf_1(y)f_n(x).\]
    Hence, substracting the last two equalities, we get
    \[0=a_2(f_2(y)-f_1(y))f_2(x)+\dots +a_n(f_n(y)-f_1(y))f_n(x).\]
    Therefore
    \[a_2(f_2(y)-f_1(y))f_2+\dots +a_n(f_n(y)-f_1(y))f_n=0,\]
    for all $y\in K$. By the minimality of $n$, we have that $f_2,\dots, f_n$ are $F$-linearly independent. Hence
    $a_2(f_2(y)-f_1(y))=0$ and, since $a_2\neq 0$, we get that $f_2(y)=f_1(y)$, for all $y\in K$, a contradiction, 
    because $f_1\neq f_2$. Therefore the result follows.  
\end{proof}

\begin{lemma}\label{fixedfield}
Let $\xi_1,\dots,\xi_n\in \C$ be roots of unity. Let $F=\Q(\xi_1,\dots ,\xi_n)$ be the subfield of $\C$ generated 
by $\xi_1,\dots ,\xi_n$. Let $G$ be the group of automorphisms of $F$ as $\Q$-algebra. Then
\[\Q=\{ a\in F\mid \alpha(a)=a \mbox{ for all }\alpha\in G\}.\]
\end{lemma}

\begin{proof}
    Note that the multiplicative subgroup of $\C\setminus\{0\}$ generated by $\xi_1,\dots,\xi_n$ is a finite cyclic group generated by some root of unity $\xi$. Hence $F=\Q[\xi]$. Let $p(x)\in \Q[x]$ be a monic irreducible polynomial such that $p(\xi)=0$.
    Let $r$ be the degree of $p(x)$. Note that if the multiplicative order of $\xi$ is $m$, then $p(x)$ is a divisor of $x^m-1$. Hence
    all roots of $p(x)$ are roots of $x^m-1$, and thus they belong to $F$. Since $p(x)$ is irreducible over $\Q$, all its roots are simple. Let $\eta_1,\dots \eta_r\in F$ be the distinct roots of $p(x)$. Every automorphism $\alpha\in G$ is determined by $\alpha(\xi)$. Note that $\alpha(\xi)\in \{ \eta_1,\dots ,\eta_r\}$. Furthermore, for $i=1,\dots ,r$, the map $\alpha_i\colon F\rightarrow F$, defined by $\alpha_i(q(\xi))=q(\eta_i)$, for all $q(x)\in\Q[x]$ of degree $<r$, is an automorphism of $F\cong \Q[x]/(p(x))$. Hence $|G|=r$. Let $L=\{ a\in F\mid \alpha(a)=a \mbox{ for all }\alpha\in G\}$. It is easy to see that $L$ is a subfield of $F$ and $\Q\subseteq L\subseteq F$. We have that $\dim_{\Q}F=r$. Suppose that $\dim_LF=s<r$. 
    Let $a_1,\dots a_s\in F$ be an $L$-basis of $F$. Since $s<r$, the system of $s$ equations in $r$ unknowns $x_i$:
    \[\sum_{i=1}^r\alpha_i(a_j)x_i=0\quad (j=1,\dots ,s),\]
    has a nonzero solution $x_i=b_i$ in $F$. Let $a\in F$. There exist $c_1,\dots, c_s\in L$ such that
    $a=\sum_{j=1}^sc_ja_j$. Hence
    \[ \sum_{i=1}^r\alpha_i(a)b_i=\sum_{i=1}^r\sum_{j=1}^sc_j\alpha_i(a_j)b_i=\sum_{j=1}^sc_j\sum_{i=1}^r\alpha_i(a_j)b_i=0.\]
    Thus
    \[\sum_{i=1}^rb_i\alpha_i=0,\]
    in contradiction with Dedekind's lemma. Therefore $\dim_LF=r$ and thus $L=\Q$.
\end{proof}

\begin{lemma}\label{Burnsidekey}
    Let $\rho$ be an irreducible representation of a finite group $G$ over $\C$ of finite degree $n$. Let $\chi$ be the character of $\rho$. Assume that $g\in G$ has $l$ conjugate elements in $G$ and that $l$ and $n$ are coprime. Then, either $\chi(g)=0$ or $\rho(g)$ is a scalar.
\end{lemma}

\begin{proof}
By Proposition~\ref{integraloverZ}, $\frac{l\chi(g)}{n}$ is integral over $\Z$. Since $l$ and $n$ are coprime, there exist integers
$r,s$ such that $1=rl+sn$. Hence
\[ \frac{\chi(g)}{n}=\frac{rl\chi(g)}{n}+s\chi(g).\]
Since $G$ is finite, there exists a positive integer $m$ such that $g^m=1$. Hence $\chi(g)$ is a sum of roots of unity in $\C$.
Hence, by Corollary~\ref{integralclousure}, $\frac{\chi(g)}{n}$ is integral over $\Z$. Let $\xi_1,\dots ,\xi_n\in\C$ roots of unity
such that $\chi(g)=\sum_{i=1}^n\xi_i$. Hence
\[\left|\frac{\chi(g)}{n}\right|=\frac{\left|\sum_{i=1}^n\xi_i\right|}{n}\leq 1.\]
\textcolor{red}{Let $M$ be the $\C[G]$-module corresponding to $\rho$.}
If $\xi_1=\dots =\xi_n$, then we may assume that there exists a $\C$-basis $\mathcal{B}$ of $M$ such that the matrix of 
$\rho(g)$ with respect to $\mathcal{B}$ is
\[ A=\left(\begin{array}{ccccc}
\xi_1&0&\ldots&\ldots&0\\
\varepsilon_1&\xi_1&\ddots&&\vdots\\
0&\varepsilon_2&\ddots&\ddots&\vdots\\
\vdots&\ddots&\ddots&\ddots&0\\
0&\ldots&0&\varepsilon_{n-1}&\xi_1\end{array}\right),\]
where $\varepsilon_i\in\{ 0,1\}$. Since 
$g^m=1$, we have that $\rho(g)^m=\id$. Let $I\in M_n(\C)$ be the identity matrix and $N=A-\xi_1I$. 
Note that $N$ is nilpotent, in fact $N^n=0$. We have
\[ I=A^m=(\xi_1I+N)^m=\xi_1^mI+\sum_{i=1}^m\xi_1^{m-i}N^{i}.
\]
Hence $\xi^m=1$ and
\begin{align*}
    0=&\sum_{i=1}^m\xi_1^{m-i}N^{i}=\xi_1^{m-1}N\left(\sum_{i=1}^m\xi_1^{1-i}N^{i-1}\right)\\
    =&\xi_1^{m-1}N\left( I+\sum_{i=2}^m\xi_1^{1-i}N^{i-1}\right).
\end{align*}
Since $ I+\sum_{i=2}^m\xi_1^{1-i}N^{i-1}$ is invertible, we get that $N=0$. Therefore $A=\xi_1I$, and thus $\rho(g)$ is a scalar in this case.

Suppose that $\xi_1,\dots ,\xi_n$ are not equal. In this case
\[\frac{|\chi(g)|}{n}<1.\]
Let $F=\Q(\xi_1,\dots ,\xi_n)$ be the subfield of $\C$ generated by $\xi_1,\dots ,\xi_n$. Let $H$ be the group of automorphisms of $F$.
Then, for every $\alpha\in H$, $\alpha(\xi_i)$ is a root of unity. Hence
\[\frac{|\alpha(\chi(g))|}{n}=\frac{|\alpha(\xi_1)+\dots +\alpha(\xi_n)|}{n}<1.\]
By Lemma \ref{fixedfield}, 
\[\prod_{\alpha\in H}\alpha\left(\frac{\chi(g)}{n}\right)\in \Q.\]
Since $\frac{\chi(g)}{n}$ is integral over $\Z$, $\alpha\left(\frac{\chi(g)}{n}\right)$ also is integral over $\Z$. By Corollary~\ref{integralclousure},  
\[\prod_{\alpha\in H}\alpha\left(\frac{\chi(g)}{n}\right)\]
is integral over $\Z$. By Example \ref{Zintclosed}, 
\[\prod_{\alpha\in H}\alpha\left(\frac{\chi(g)}{n}\right)\in\Z.\]
Since
\[\left|\prod_{\alpha\in H}\alpha\left(\frac{\chi(g)}{n}\right)\right|<1,\]
we have that $\chi(g)=0$ and the result follows.
\end{proof}

\begin{theorem}[Burnside]\label{Burnside}
\index{Burnside's!theorem}
Let $G$ be a finite group with a conjugacy class with $p^m$ elements, for some prime $p$ and positive integer $m$. 
Then $G$ is not simple.
\end{theorem}

\begin{proof}
    Suppose that $G$ is simple. It is clear that $G$ is not abelian. Let $g\in G$ be an element with $p^m$ conjugate elements in $G$.
    Let $\rho$ be an irreducible non-trivial representation of $G$ over $\C$ of degree $n$. Let $\chi$ be the character of $\rho$.
    Suppose that $\chi(g)\neq 0$ and that $p$ is not a divisor of $n$. By Lemma~\ref{Burnsidekey}, $\rho(g)$ is scalar and thus it is central in $\rho(G)$. Since $G$ is simple and $\rho$ is non-trivial, we have that $\ker(\rho)=\{ 1\}$ and $\rho(G)\cong G$ is simple.
    Hence $\rho(g)=\id$, and thus $g=1$, in contradiction with the fact that $g$ has $p^m$ conjugate elements. Therefore, if $p$ is not a divisor of $n$, then $\chi(g)=0$.
    
    Let $\sigma$ be the regular representation of $G$ over $\C$, that is, $\sigma\colon G\rightarrow \Aut_{\C}(\C[G])$ is defined by
    $\sigma(h)(a)=ha$, for all $h\in G$ and $a\in\C[G]$.
    Let $\psi$ be the character of $\sigma$. By Maschke's theorem $J(\C[G])=\{0\}$. Since $\C[G]$ is Artinian, by Proposition~\ref{algclosedfield}, there exist positive integers $n_1,\dots,n_s$ such that $\C[G]\cong M_{n_1}(\C)\times\dots\times M_{n_s}(\C)$.
    Let $f\colon \C[G]\rightarrow M_{n_1}(\C)\times\dots\times M_{n_s}(\C)$ be an isomorphism. Let
    \[I_{i,j}=f^{-1}(\{0\}\times\dots\times\{0\}\times M_{n_i}(\C)e_{j,j}\times\{0\}\times\dots\times\{0\}),\]
    where $e_{j,j}\in M_{n_i}(\C)$ is the matrix with $1$ in the $(j,j)$ entry and $0$ elsewhere. Note that  $I_{i,1},\dots ,I_{i,n_i}$ are  minimal isomorphic left ideals of $\C[G]$ of dimension $n_i$ over $\C$. Furthermore
    \[\C[G]=\bigoplus_{i=1}^s\bigoplus_{j=1}^{n_i}I_{i,j}.\]
    Let $\sigma_{i,j}\colon G\rightarrow \Aut_{\C}(I_{i,j})$ be the map defined by
    $\sigma_{i,j}(h)(a)=ha$, for all $h\in G$ and $a\in I_{i,j}$. Let $\psi_{i,j}$ be the character of the representation $\sigma_{i,j}$. Then
    it is easy to see that $\psi(h)=\sum_{i=1}^s\sum_{j=1}^{n_i}\psi_{i,j}(h)$, for all $h\in G$. By Lemma \ref{equivrep}, $\psi_{i,1}(h)=\psi_{i,j}(h)$, for all $i$ and all $j=1,\dots ,n_i$. Hence
    \[\psi(h)=\sum_{i=1}^sn_i\psi_{i,1}(h),\]
    for all $h\in G$. We may assume that $n_1=1$ and 
    \[I_{1,1}=\C[G]\frac{1}{|G|}\sum_{x\in G}x=\C\frac{1}{|G|}\sum_{x\in G}x,\]
    which corresponds to the trivial representation $\sigma_{1,1}$ of degree $1$. Note that $\sigma_{i,1}$ is irreducible  and non-trivial, for all $i>1$. Since $n_i$ is the degree of $\sigma_{i,1}$ and $p^m$ is the number of conjugate elements of $g$, we have seen that if $p$ is not a divisor of $n_i$ and $i>1$, then  $\psi_{i,1}(g)=0$.
    Hence
    \[0=\psi(g)=\psi_{1,1}(g)+\sum_{i=2}^sn_i\psi_{i,1}(g)=1+p\sum_{i=2}^sr_i\psi_{i,1}(g),\]
    where 
    \[r_i=\left\{\begin{array}{ll}
    \frac{n_i}{p}&\mbox{ if $p$ is a divisor of $n_i$}\\
    0&\mbox{otherwise}\end{array}\right.\]
    Hence, by Corollary~\ref{integralclousure} 
    \[-\frac{1}{p}=\sum_{i=2}^sr_i\psi_{i,1}(g)\]
    is integral over $\Z$, because $\psi_{i,1}$ is a sum of roots of unity, in contradiction with Example \ref{Zintclosed}.
    Therefore the result follows.
\end{proof}

\begin{theorem}[Burnside's $p$-$q$-theorem]
\index{Burnside's $p$-$q$-theorem}
Let $G$ be a finite group of order $p^nq^m$, for some prime numbers $p,q$ and non-negative integers $m,n$. Then $G$ is solvable.
\end{theorem}

\begin{proof}
    Let $G$ be a counterexample of minimal order, that is $G$ is a finite non-solvable group, $|G|=p^nq^m$ and every group $H$ of order
    $|H|=p_1^{r}q_1^{s}<|G|$, for some prime numbers $p_1,q_1$ and non-negative integers $r,s$, is solvable.  Let $S$ be a Sylow $q$-subgroup of $G$. Since finite $p$-groups are nilpotent, and thus solvable, we have that $S$ is non-trivial. Since $S$ is a non-trivial $q$-subgroup, we know that $Z(S)$ is non-trivial. Let $g\in Z(S)\setminus\{ 1\}$. Then the number of conjugate elements of $g$ in $G$ is $(G:C_G(g))$. Since $S\subseteq C_G(g)$ and $p^n=(G:S)=(G:C_G(g))(C_g(g):S)$, we have that either $g\in Z(G)$ or $g$ has
    $p^r>1$ conjugate elements. By Theorem~\ref{Burnside}, in both cases $G$ is not simple. Let $N$ be a non-trivial proper normal subgroup of $G$. Since $|N|,|G/N|<|G|$ and $|N|,|G/N|$ are divisors of $|G|$, we have that $N$ and $G/N$ are solvable. By Theorem~\ref{theorem:resoluble}, $G$ is solvable, a contradiction. Therefore the result follows.
\end{proof}

A non-character-theoretic proof of this theorem is known but it is harder, see~\cite{MR2426855}.

Burnside's theorem has several generalizations.
\begin{theorem}[Kegel--Wielandt]
\index{Kegel--Wielandt's theorem}
\label{thm:KegelWielandt}
Let $G$ be a finite group such that 
$G=AB$ for nilpotent subgroups $A$ and $B$ of $G$. Then $G$ is solvable.
\end{theorem}


For the proof we refer to~\cite[Theorem 2.4.3]{MR1211633}.


\begin{theorem}[Feit--Thompson]
\index{Feit--Thompson's theorem}
\label{thm:FeitThompson}
Every finite group of odd order is solvable.
\end{theorem}


The proof of the theorem is extremely hard. It occupies a full volume of 
\emph{Pacific Journal of Mathematics}, see~\cite{MR166261}.


\section{Hall subgroups}

In this section we shall prove a characterization of finite solvable groups due to P. Hall. For this we will use Burnside's $p$-$q$-theorem and Wielandt's theorem.
To prove Wielandt's theorem on solvable groups 
we need the following lemma.  

\begin{lemma}
	\label{lemma:4Wielandt}
	Let $G$ be a finite group. If $H$ and $K$ are subgroups of $G$ of coprime indices, then 
    $G=HK$ and $(H:H\cap K)=(G:K)$.
\end{lemma}

\begin{proof}
	Let $D=H\cap K$. Since
	\[
	(G:D)=\frac{|G|}{|H\cap K|}=(G:H)(H:H\cap K),
	\]
	$(G:H)$ divides $(G:D)$. Similarly, $(G:K)$ divides 
	$(G:D)$. Since $(G:H)$ and $(G:K)$ are coprime, $(G:H)(G:K)$
	divides $(G:D)$. In particular, 
	\[
	\frac{|G|}{|H|}\frac{|G|}{|K|}=(G:H)(G:K)\leq (G:D)=\frac{|G|}{|H\cap K|}
	\]
	and hence $|G|=|HK|$. Since 
	\[
	|G|=|HK|=|H||K|/|H\cap K|, 
	\]
	it follows that 
	$(G:K)=(H:H\cap K)$.
\end{proof}

\index{Normal clousure}
The 
{\em normal closure} $H^G$ of a subgroup $H$ of $G$ is the subgroup
\[
	H^G=\langle xHx^{-1}:x\in G\rangle
\]
generated by all conjugates of $H$. The 
subgroup $H^G$ is the smallest normal subgroup of $G$ containing $H$. 

\begin{example}
	Let $G=\Alt_4$ and $H=\{\id,(12)(34)\}$. Then 
	\[
	H^G=\{\id,(12)(34),(13)(24),(14)(23)\}\simeq C_2\times C_2.
	\]
\end{example}


\begin{theorem}[Wielandt]
	\label{theorem:Wielandt:solvable}\index{Wielandt's theorem}
	Let $G$ be a finite group and $H$, $K$ and $L$ be subgroups of $G$ 
	with pair-wise coprime indices. If $H$, $K$ and $L$ are solvable, 
	then $G$ is solvable. 
\end{theorem}

\begin{proof}
	Assume the theorem is not valid and let $G$ be a minimal counterexample. 
	Then $G$ is not trivial.  
	Let $N$ be a minimal normal subgroup of 
	$G$ and $\pi\colon G\to G/N$, $g\mapsto gN$, be the canonical map. Since 
	by definition $N$ is non-trivial, 
	it follows that $|G/N|<|G|$. 
	The subgroups 
	$\pi(H)=\pi(HN)$, $\pi(K)=\pi(KN)$ and $\pi(L)=\pi(LN)$ of $\pi(G)=G/N$ are solvable. 
	The correspondence theorem implies that 
	the indices of $\pi(H)$, $\pi(K)$ and $\pi(L)$ in $\pi(G)$ are pair-wise coprime. By the 
	minimality of $G$, the group 
	$\pi(G)$ is solvable. If $H=\{1\}$, then 
	$|G|=(G:H)$ is coprime with $(G:K)$ and hence $G=K$ is solvable. So we may assume that  
	$H\ne \{1\}$. Let $M$ be a minimal normal subgroup of $H$. By
	Lemma~\ref{lem:minimal_normal}, $M$ is a $p$-group for some prime number $p$. We may
	assume that $p$ does not divide 
	$(G:K)$ (if $p$ divides $(G:K)$, then $p$ does no divide $(G:L)$ and 
	hence it is enough to replace $K$ by $L$). There exists 
	$P\in\Syl_p(G)$ such that $P\subseteq K$. By Sylow's theorem, 
	there exists  $g\in G$ such that $M\subseteq
	gKg^{-1}$. Since $(G:gKg^{-1})=(G:K)$ and $(G:H)$ are coprime, 
	Lemma~\ref{lemma:4Wielandt} implies that $G=(gKg^{-1})H$. 
	
	We claim that all  conjugates of $M$ are included in $gKg^{-1}$. 
	If $x\in G$, then $x=uv$ for some $u\in 
	gKg^{-1}$ and $v\in H$. Since $M$ is normal in $H$, 
	\[
	xMx^{-1}=(uv)M(uv)^{-1}=uMu^{-1}\subseteq gKg^{-1}.
	\]
	In particular, $\{1\}\ne M^G\subseteq gKg^{-1}$ is solvable, as $gKg^{-1}$ is
	solvable. The minimality of $G$ implies that $G/M^G$ is solvable. Hence 
	$G$ is solvable by Theorem~\ref{theorem:resoluble}.
\end{proof}


\index{$p$-complement}
Let $G$ be a finite group of order $p^{\alpha}m$ with $p$ a prime number coprime with $m$. 
A subgroup $H$ of $G$ is a {\em $p$-complement} if $|H|=m$. 

\begin{example}
	Let $G=\Sym_3$. Then $H=\langle (123)\rangle$ is a $2$-complement 
	and $K=\langle (12)\rangle$ is a $3$-complement.
\end{example}

%A famous theorem of Burnside states that finite groups whose order are divisible 
%by exactly two primes are solvable. 

%\begin{theorem}[Burnside]
%\index{Burnside!Theorem}
%Let $p$ and $q$ be prime numbers and let $G$ be a group of order $p^\alpha q^\beta$. Then
%$G$ is solvable. 
%\end{theorem}
%
%There is a quite easy proof that uses basic character theory, see for example.
%A non-character-theoretic proof is known but it is harder, see~\cite{MR2426855}.

\begin{theorem}[Hall]
	\label{theorem:Hall:solvable}\index{Hall's theorem}
	Let $G$ be a finite group that admits a $p$-complement for all primes
	$p$ dividing the order of $G$. Then $G$ is solvable. 
\end{theorem}

\begin{proof}
	Let $|G|=p_1^{\alpha_1}\cdots
	p_k^{\alpha_k}$ with the $p_j$ being distinct primes. We proceed by induction on 
	$k$. If $k=1$, then $G$ is a $p_1$-group and the result is clear. If $k=2$, then 
	Burnside's $p$-$q$-theorem implies the claim. Assume now that $k\geq3$. For each 
	$j\in\{1,2,3\}$ let $H_j$ be $p_j$-complement in
	$G$. Since $|H_j|=|G|/p_j^{\alpha_j}$, the subgroups $H_j$ have pair-wise coprime indices. 
	
	We claim that $H_1$ is solvable. Note that $|H_1|=p_2^{\alpha_2}\cdots
	p_k^{\alpha_k}$. Let $p$ be a prime number that divides $|H_1|$ and let $Q$ be a 
	$p$-complement in $G$. 
	Since $(G:H_1)$ and $(G:Q)$ are coprime, Lemma~\ref{lemma:4Wielandt} implies that 
	\[
	(H_1:H_1\cap Q)=(G:Q).
	\]
	Thus $H_1\cap Q$ is a $p$-complement in $H_1$. Hence $H_1$ is
	solvable by the inductive hypothesis. Similarly, $H_2$ and 
	$H_3$ are both solvable. 
	
	Since $H_1$, $H_2$ and $H_3$ are solvable and have  pair-wise coprime indices, by Wielandt's theorem, $G$ also is solvable.
	Therefore the result follows by induction. 
\end{proof}

\index{$\pi$-group}
\begin{definition}
Let $\pi$ be a non-empty set of primes. A finite group $G$ is a {\em $\pi$-group} if every prime divisor $p$ of $|G|$ belongs to $\pi$.  
\end{definition}

\index{Hall $\pi$-subgroup}
\begin{definition}
Let $\pi$ be a non-empty set of primes. A $\pi$-subgroup $H$ of a finite group $G$ is a {\em Hall $\pi$-subgroup} of $G$ if $(G:H)$ and $p$ are coprime, for all $p\in \pi$.
\end{definition}

\begin{theorem}[Hall]\label{thm:Hall_pi_subgroup} \index{Hall's!theorem}
Let $G$ be a finite solvable group. Then the following conditions hold.
\begin{enumerate}
    \item For every non-empty set $\pi$ of primes, there exists a Hall $\pi$-subgroup of $G$.
    \item Let $\pi$ be a non-empty set of primes. Let $H$ be a Hall $\pi$-subgroup of $G$. Then for every
    $\pi$-subgroup $L$ of $G$ there exists $g\in G$ such that $L\subseteq gHg^{-1}$.
\end{enumerate}
\end{theorem}

\begin{proof}
    Let $G$ be a counterexample to $1)$ of smallest order. Hence $G$ is a finite solvable group and
    there exists a non-empty set of prime divisors of $|G|$ such that $G$ has no Hall $\pi$-subgroups. Furthermore, every finite solvable group $K$ such that $|K|<|G|$ has Hall $\tau$-subgroups for all non-empty set of primes $\tau$. Let $N$ be a minimal normal subgroup of $G$. By Theorem \ref{theorem:resoluble}, $N$ and $G/N$ are solvable. By Lemma \ref{lem:minimal_normal}, $N$ is an elementary abelian $p$-subgroup of $G$, for some prime $p$.
    Suppose that $p\in\pi$. In this case, since $|G/N|<|G|$, there exists a subgroup $H$ of $G$ such that $N\subseteq H$ and $H/N$ is a Hall $\pi$-subgroup
    of $G/N$. Since $|H|=|H/N|\cdot |N|$, we have that $H$ is a $\pi$-subgroup of $G$. Since $(G:H)=(G/N:H/N)$, we get that $H$ is a Hall $\pi$-subgroup of $G$, a contradiction. Therefore, $p\notin \pi$. Since $|G/N|<|G|$, there exists a subgroup $H$ of $G$ such that $N\subseteq H$ and $H/N$ is a Hall $\pi$-subgroup
    of $G/N$. Since $|H|=|H/N|\cdot |N|$, we have that $H$ is a $\textcolor{red}{\pi\cup\{ p\}}$-subgroup of $G$. Note that every Hall $\pi$-subgroup of $H$ also is a Hall $\pi$-subgroup of $G$. Hence $H=G$. Note that $N$ is a Sylow $p$-subgroup of $G$. Let $M$ be a normal subgroup of $G$ such that $N\subseteq M$ and $M/N$ is a minimal normal subgroup of $G/N$. Since $M/N$ is solvable, by Lemma \ref{lem:minimal_normal}, $M/N$ is an elementary abelian $q$-group, for some prime $q$. Since $N$ is a Sylow $p$-subgroup of $G$, we have that $q\neq p$. Hence $q\in \pi$. Let $S$ be a Sylow $q$-subgroup of $M$. By Frattini's argument, $G=MN_G(S)=NSN_G(S)=NN_G(S)$. Note that, by the above argument, $S$ is not normal in $G$. Hence $|N_G(S)|<|G|$. Since $N_G(S)$ is solvable, there exists a Hall $\pi$-subgroup $H_1$ of $N_G(S)$. Since $(G:H_1)=(G:N_G(S))(N_G(S):H_1)$ is a divisor of $|N|(N_G(S):H_1)$, we have that $H_1$ also is a Hall $\pi$-subgroup of $G$, a contradiction. Therefore $1)$ is proved.
    
    Assume now that 2) does not hold. Let $G$ be a counterexample to 2) of smallest order. Hence $G$ is a finite solvable group and there exist a non-empty set of primes $\pi$, a Hall $\pi$-subgroup $H$ of $G$ and a $\pi$-subgroup $L$ of $G$ such that, for every $g\in G$, $L\not\subseteq gHg^{-1}$. Furthermore, if $K$ is a finite solvable group such that $|K|<|G|$, then for every non-empty set of primes $\tau$, every Hall $\tau$-subgroup $U$ and every $\tau$-subgroup $V$ of $K$, there exists $x\in K$ such that $V\subseteq xUx^{-1}$. Let $N$ be a minimal normal subgroup of $G$. As above, we know that $N$ is an elementary abelian $p$-group. 
    Then $NH/N$ and $NL/N$ are $\pi$-subgroups of $G/N$.  Since $H$ is a Hall $\pi$-subgroup of $G$, we have that $NH/N$ is a Hall $\pi$-subgroup of $G/N$.
    Suppose that $p\in \pi$. In this case, $NH$ is a $\pi$-subgroup. Since $H$ is a Hall $\pi$-subgroup of $G$, we have that $H=NH$. Since $|G/N|<|G|$, there exists $x\in G$ such that $NL\subseteq (xN)NH(x^{-1}N)=xNHx^{-1}=xHx^{-1}$, and thus
    $L\subseteq xHx^{-1}$, a contradiction. Hence $p\notin \pi$, and $L\subseteq xNHx^{-1}$. If $|xNHx^{-1}|<|G|$, then, since $xHx^{-1}$ is a Hall $\pi$-subgroup of $xNHx^{-1}$, there exists $y\in xNHx^{-1}$ such that $L\subseteq yxHx^{-1}y^{-1}$, a contradiction. Hence $|NH|=|xNHx^{-1}|=|G|$, and thus $G=NH$. 
    Suppose that $|L|<|H|$. In this case, $|NL|=|N|\cdot |L|<|N|\cdot |H|=|G|$. Note that $L$ is a Hall $\pi$-subgroup of $NL$ and
    \[ |(NL)\cap H|=\frac{|NL|\cdot |H|}{|(NL)H|}=\frac{|N|\cdot |L|\cdot |H|}{|NH|}=\frac{|N|\cdot |L|\cdot |H|}{|N|\cdot |H|}=|L|.\]
    Hence there exists $z\in NL$ such that $L\subseteq z((NL)\cap H)z^{-1}\subseteq zHz^{-1}$, a contradiction. Hence $|L|=|H|$. 
    Let $M$ be a normal subgroup of $G$ such that $N\subseteq M$ and $M/N$ is a minimal normal subgroup of $G/N$. As above, we know that $M/N$ is an elementary abelian $q$-subgroup, for some prime $q\in\pi$. Note that 
    \[|H\cap M|=\frac{|H|\cdot |M|}{|MH|}=\frac{|H|\cdot |M|}{|NH|}=\frac{|M|}{|N|}.\]
    Hence $H\cap M$ is a Sylow $q$-subgroup of $M$. Since $L\cap M$ is a $q$-subgroup of $M$, by Sylow's theorem, there exists $z\in M$ such that $L\cap M\subseteq z(H\cap M)z^{-1}$. Since $|L|=|H|$, we have that $G=NH=NL=MH=ML$ and
    \[ |L\cap M|=\frac{|L|\cdot |M|}{|ML|}=\frac{|H|\cdot |M|}{|MH|}=|H\cap M|.\]
    Hence $L\cap M=z(H\cap M)z^{-1}$ and thus 
    \[ L\subseteq N_G(L\cap M)=N_G(z(H\cap M)z^{-1})=zN_G(H\cap M)z^{-1}.\]
    By the above argument, we have that $G\neq N_G(H\cap M)$. Hence $|zN_G(H\cap M)z^{-1}|=|N_G(H\cap M)|<|G|$. Since $zHz^{-1}$ is a Hall $\pi$-subgroup of $zN_G(H\cap M)z^{-1}$, there exists $t\in zN_G(H\cap M)z^{-1}$ such that $L\subseteq tzHz^{-1}t^{-1}$, a contradiction. Therefore $2)$ follows.
\end{proof}

\begin{definition}
\index{Sylow system} Let $G$ be a finite group. Let $p_1,\dots ,p_r$ be the distinct prime divisors of $|G|$. A {\em Sylow system} of $G$ is a set $\{ S_1,\dots ,S_r\}$, where $S_i$ is a Sylow $p_i$-subgroup of $G$, such that $S_iS_j=S_jS_i$, for all $i,j$. 
\end{definition}

\begin{theorem}[Hall]
Let $G$ be a finite group. Then $G$ is solvable if and only if $G$ has a Sylow system.
\end{theorem}

\begin{proof}
    Suppose that $G$ has a Sylow system $\{ S_1,\dots, S_r\}$, where $S_i$ is a Sylow $p_i$-subgroup of $G$. Since $S_iS_j=S_jS_i$, for all $i,j$, $C_i=S_1\cdots S_{i-1}S_{i+1}\cdots S_r$ is a subgroup of $G$. Furthermore 
    \[ |C_i|=|S_1|\cdots |S_{i-1}|\cdot |S_{i+1}|\cdots |S_r|\]
    and$(G:C_i)=|S_i|$. Hence $C_i$ is a $p_i$-complement of $G$. By Theorem \ref{theorem:Hall:solvable}, $G$ is solvable.
    
    Conversely, suppose that $G$ is solvable. By Theorem \ref{thm:Hall_pi_subgroup}, for every prime divisor $p_i$ of $|G|$, $G$ has a $p_i$-complement $H_i$. Let $p_1,\dots ,p_r$ be the distinct prime divisors of $|G|$. Clearly, to prove that $G$ has a Sylow system, we may assume that $r>2$. Let $I$ be a non-empty subset of $\{ 1,\dots ,r\}$. Let $H_I=\bigcap_{i\in I}H_i$. We shall show that
    \begin{equation}\label{eq:indexH_I}
    (G:H_I)=\prod_{i\in I}(G:H_i)
    \end{equation}
    by induction on $|I|$. For $|I|=1$, it is clear. Assume that $|I|>1$ and that (\ref{eq:indexH_I}) holds for every non-empty subset $J$ of $\{ 1,\dots ,r\}$ such that $|J|<|I|$. Let $i\in I$ and $I_1=I\setminus \{ i\}$. By the inductive hypothesis, we have that
    \[ (G:H_{I_1})=\prod_{k\in I_1}(G:H_k).\]
    Hence $(G:H_i)$ and $(G:H_{I_1})$ are coprime. By Lemma \ref{lemma:4Wielandt}, $G=H_iH_{I_1}$. Now we have that
    \begin{align*}\prod_{k\in I}(G:H_k)=&(G:H_i)(G:H_{I_1})=\frac{|G|^2}{|H_i|\cdot |H_{I_1}|}\\
    =&\frac{|G|\cdot |H_iH_{I_1}|}{|H_i|\cdot |H_{I_1}|}=
    \frac{|G|}{|H_i\cap H_{I_1}|}\\
    =&(G:H_i\cap H_{I_1})=(G:H_I)\end{align*}
    Hence (\ref{eq:indexH_I}) follows by induction.
    Let $I,J$ be non-empty subsets of $\{1,\dots ,r\}$.  We shall prove that \begin{equation}\label{eq:H_IH_J}
        H_IH_J=H_JH_I
    \end{equation} 
    by induction on $|I|\cdot |J|$. For $|I|=|J|=1$, (\ref{eq:H_IH_J}) follows from Lemma \ref{lemma:4Wielandt}. Suppose that $|I|\cdot |J|>1$ and that (\ref{eq:H_IH_J}) holds for all non-empty subsets $I_1,J_1$ of $\{ 1,\dots r\}$, such that $|I_1|\cdot|J_1|<|I|\cdot |J|$. If $I\subseteq J$, then $H_J\subseteq H_I$ and $H_IH_J=H_I=H_JH_I$. Hence we may assume that there exist $i\in I\setminus J$. Let $I_1=I\setminus\{ i\}$. By the inductive hypothesis, $H_{I_1}H_J=H_JH_{I_1}$ is a subgroup of $G$. Note that $H_I,H_J$ are subgroups of the group $H_{I_1}H_J$. Since $H_I=H_{I_1}\cap H_i$, 
    \[ |H_IH_J|=\frac{|H_I|\cdot |H_J|}{|H_I\cap H_J|}=\frac{|H_{I_1}|\cdot |H_i|\cdot|H_J|}{|H_{I_1}H_i|\cdot |H_{I_1}\cap H_i\cap H_J|}=\frac{|H_{I_1}H_J|\cdot |H_i|\cdot|H_{I_1}\cap H_J|}{|H_{I_1}H_i|\cdot |H_{I_1}\cap H_i\cap H_J|}.\]
    Note that $(G:H_i)$ and $(G:H_{I_1})$ are coprime. Similarly, $(G:H_i)$ and $(G:H_J)$ are coprime. By Lemma \ref{lemma:4Wielandt}, $G=H_iH_{I_1}=H_iH_J=H_i(H_{I_1}\cap H_J)$. Hence
    \[ |H_IH_J|=\frac{|H_{I_1}H_J|\cdot |H_i|\cdot|H_{I_1}\cap H_J|}{|H_iH_{I_1}|\cdot |H_{I_1}\cap H_i\cap H_J|}=\frac{|H_{I_1}H_J|\cdot |H_i(H_{I-1}\cap H_J)|}{|H_iH_{I_1}|}=|H_{I_1}H_J|.\]
    Hence $H_IH_J=H_{I_1}H_J=H_JH_I$. Therefore (\ref{eq:H_IH_J}) follows by induction. 
    Let $i\in\{ 1,\dots , r\}$, $J_i=\{1,\dots ,r\}\setminus \{ i\}$ and $S_i=H_{J_i}$. By (\ref{eq:indexH_I}), $S_i$ is a Sylow $p_i$-subgroup of $G$. By (\ref{eq:H_IH_J}), $S_iS_j=S_jS_i$. Hence $\{ S_1,\dots ,S_r\}$ is a Sylow system of $G$, and the result follows. 
\end{proof}

%Veamos una aplicación a grupos finitos resolubles. 
%
%\begin{theorem}
%	Sea $G$ un grupo finito no trivial y resoluble. Todo subgrupo normal $N$ no
%	trivial contiene un subgrupo normal abeliano no trivial y este subgrupo está en realidad 
%	contenido en $F(N)$. 
%\end{theorem}
%
%\begin{proof}
%	Sabemos que $N\cap G^{(0)}=N\ne\{1\}$. Como $G$ es un grupo resoluble resoluble, 
%	existe $m\in\N$ tal que $N\cap
%	G^{(m)}=\{1\}$. Sea $n\in\N$ maximal tal que $N\cap G^{(n)}$ es no trivial. 
	%\ne \{1\}$. 
%	Como $[N,N]\subseteq N$ y $[G^{(n)},G^{(n)}]=G^{(n+1)}$, 
%	\[
%	[N\cap G^{(n)},N\cap G^{(n)}]\subseteq N\cap G^{(n+1)}=\{1\}.
%	\]
%	Luego $N\cap G^{(n)}$ es un subgrupo abeliano de $G$. Como además es normal
%	y nilpotente, $N\cap G^{(n)}\subseteq N\cap F(G)=F(N)$.
%\end{proof}
%
%\begin{theorem}
%	\label{theorem:F(G)centraliza}
%	Si $G$ es un grupo finito y $N$ es un subgrupo minimal-normal entonces
%	entonces $F(G)\subseteq C_G(N)$.
%\end{theorem}
%
%\begin{proof}
%	Por el teorema de Fitting, $F(G)$ es un subgrupo normal y nilpotente. 
%	Sea $N$ un subgrupo minimal-normal de $G$. 
%	El subgrupo $N\cap F(G)$
%	es normal en $G$.  Además $[F(G),N]\subseteq N\cap F(G)$. Si $N\cap F(G)=\{1\}$ entonces
%	$[F(G),N]=\{1\}$. Si no, $N=N\cap F(G)\subseteq F(G)$ por la minimalidad de $N$. Como
%	$F(G)$ es nilpotente, $N\cap Z(F(G))\ne \{1\}$ por el
%	teorema de Hirsch. 
	%~\ref{theorem:Z(nilpotent)}. 
%	Como $Z(F(G))$ es característico en $F(G)$ y
%	$F(G)$ es normal en $G$, $Z(F(G))$ es normal en $G$. Como $\{1\}\ne N\cap Z(F(G))$ es
%	normal en $G$, la minimalidad de $N$ implica que $N=N\cap Z(F(G))\subseteq
%	Z(F(G))$ y luego $[F(G),N]=\{1\}$. 
%\end{proof}
%
%\begin{corollary}
%	Sea $G$ un grupo finito y resoluble. 
%	\begin{enumerate}
%		\item Si $N$ es un subgrupo minimal-normal entonces $N\subseteq
%			Z(F(G))$. 
%		\item Si $H$ es un subgrupo normal entonces $H\cap F(G)\ne\{1\}$.
%	\end{enumerate}
%\end{corollary}
%
%\begin{proof}
%	Demostremos la primera afirmación. Como $N$ es un $p$-grupo por el
%	lema~\ref{lemma:minimal_normal}, $N$ es nilpotente y luego $N\subseteq
%	F(G)$. 	Además $F(G)\subseteq C_G(N)$ por el
%	teorema anterior.  Luego $N\subseteq Z(F(G))$. 
%
%	Demostremos ahora la segunda afirmación. El subgrupo $H$ contiene un
%	subgrupo minimal-normal $N$ y $N\subseteq F(G)$. Luego $H\cap F(G)\ne\{1\}$. 
%\end{proof}

\section{Perfect groups}
\index{Group!perfect}
A group $G$ is said to be {\em perfect} if $[G,G]=G$. Note that
$G$ is perfect if and only if $G/[G,G]$ is trivial. Every non-abelian simple group is perfect.

Let $K$ be a field. \index{Special linear group} the {\em special linear} group of degree $n$ over $K$ is 
\[ \SL_n(K)=\{ A\in\GL_n(K)\mid \det(A)=1\}.\]
If $K=\F_q$ is the finite field of $q$ elements, then we will write $\SL_n(q)=\SL_n(\F_q)$.

\begin{example}
The groups $\SL_2(2)$ and $\SL_2(3)$ are not perfect. 	
\end{example}

Let $p$ be a prime number and $q=p^m$, for some positive integer $m$.  
The groups $\SL_n(q)$ are perfect except the cases $\SL_2(2)$ and $\SL_2(3)$. As an example,
let us prove that $\SL_2(q)$ is perfect if $q>3$. We first 
claim that $\SL_2(q)$ is generated by matrices
$X_{ij}(\lambda)=I+\lambda E_{ij}$, where $I$ denotes the identity matrix, 
$E_{ij}$ is the matrix with a one at position $(i,j)$ and zero in all other entries, 
$i,j\in\{1,2\}$ are distinct and $\lambda\in\F_q\setminus\{0\}$. First note that
a matrix $\begin{pmatrix}1&b\\c&d\end{pmatrix}\in\SL_2(q)$ is 
a product of some of the $X_{ij}(\lambda)$, as 
\[
\begin{pmatrix}
1&b\\
c&d	
\end{pmatrix}
=\begin{pmatrix}
1&0\\
c&1
\end{pmatrix}
\begin{pmatrix}
1&b\\
0&1	
\end{pmatrix}
=X_{21}(c)X_{12}(b).
\]
This implies that a matrix $\begin{pmatrix}a&b\\c&d\end{pmatrix}\in\SL_2(q)$, 
with $c\ne 0$, also is a product
of some $X_{ij}(\lambda)$. Indeed, if
$\lambda$ is such that $a=1-\lambda c$, then
\begin{align*}
\begin{pmatrix}
a&b\\
c&d	
\end{pmatrix}
&=\begin{pmatrix}
1&-\lambda\\
0&1	
\end{pmatrix}
\begin{pmatrix}
1&b+\lambda d\\
c&d	
\end{pmatrix}
%&=
%\begin{pmatrix}
%1&-\lambda\\
%0&1	
%\end{pmatrix}
%\begin{pmatrix}
%1&0\\
%c&1	
%\end{pmatrix}
%\begin{pmatrix}
%1&b+\lambda d\\
%0&1	
%\end{pmatrix}
%=X_{12}(-\lambda)X_{21}(c)X_{12}(b+\lambda d.
%\shortintertext{and hence}
%\begin{pmatrix}
%a&b\\
%c&d	
%\end{pmatrix}
=X_{12}(-\lambda)X_{21}(c)X_{12}(b+\lambda d).
\end{align*}
Finally, 
\[
\begin{pmatrix}
	a&b\\
	0&a^{-1}
\end{pmatrix}
=\begin{pmatrix}
	1&0\\
	-1&1
\end{pmatrix}
\begin{pmatrix}
	a&b\\
	a&b+a^{-1}
\end{pmatrix}
\]
and therefore $\begin{pmatrix}
	a&b\\
	0&a^{-1}
\end{pmatrix}$ is a product of some $X_{ij}(\lambda)$ since
$\begin{pmatrix}
a&b\\
a&b+a^{-1}	
\end{pmatrix}$ 
is a product of some $X_{ij}(\lambda)$. 
To prove that $[\SL_2(q),\SL_2(q)]=\SL_2(q)$ we first 
note that
\[
\left[\begin{pmatrix}a&0\\0&a^{-1}\end{pmatrix},\begin{pmatrix}1&b\\0&1\end{pmatrix}\right]
=\begin{pmatrix}
1 & (a^2-1)b\\
0 & 1
\end{pmatrix}.
\]	

Since $q> 3$, given $\lambda\in\F_q$ and $a\in\F_q\setminus\{-1,0,1\}$, there exists
$b\in\F_q$ such that $\lambda=(a^2-1)b$. This implies that each $X_{ij}(\lambda)$ belongs to the
commutator subgroup of $\SL_2(q)$. 



\begin{theorem}[Gr\"un]
	\label{theorem:Grun}
	\index{Gr\"un!theorem}
	If $G$ is a perfect group, then $Z(G/Z(G))=\{1\}$. 
\end{theorem}

\begin{proof}
	The three subgroups lemma with $X=Y=G$, $Z=\zeta_2(G)$ and $N=\{1\}$ yields  
	\[
	\{1\}=[\zeta_2(G),G,G]=[\zeta_2(G),[G,G]]=[\zeta_2(G),G].
	\]
	Thus 	
	$\zeta_2(G)\subseteq Z(G)$. Now we prove that $Z(G/Z(G))$ is trivial. Let $\pi\colon G\to G/Z(G)$ be the canonical map
	and $x\in G$ be such that 
	$\pi(x)$ is a central element. Then 
	\[
	[\pi(x),\pi(y)]=\pi([x,y])=1
	\]
	for all $y\in G$. In particular, 
	$[x,y]\in Z(G)=\zeta_1(G)$ for all $y\in G$. This means that $x\in\zeta_2(G)\subseteq Z(G)$. 
\end{proof}





% The previous theorem is not true in the case of finite braces of non-nilpotent type. 

% \begin{theorem}
% \label{thm:Smoktunowicz}
%     Let $A$ be a finite skew left brace of nilpotent type. Then $A$ is
%     left nilpotent if and only if the multiplicative group of $A$ is nilpotent.
% \end{theorem}

% \begin{proof}
%     Proposition~\ref{pro:left_p} and Theorem~\ref{thm:left_p} prove the theorem.
% \end{proof}

%The following theorem was found by Smoktunowicz in the case of braces of abelian type. 

\section{Exercises}
\begin{prob}
	Prove that the group $\SL_2(3)$ is solvable. 
\end{prob}

\begin{prob}
	Compute the Frattini subgroup of $G=\SL_2(3)$.  
\end{prob}

\begin{prob}
	Let $p$ be a prime.  Prove that if $G$ is an elementary abelian $p$-group, then 
%	Sea $p$ un número primo. Sea $G$ un $p$-grupo elemental abeliano, es decir
%	$G\simeq C_p^m$ para algún $m\in\N$.  Supongamos además que
%	$G=\langle x_1\rangle\times\cdots\times\langle x_m\rangle$ con $\langle x_j\rangle\simeq C_p$.  
%	Veamos que $\Phi(G)$ es trivial. 
%	Sea $j\in\{1,\dots,m\}$ y sea $n_j\in\{1,\dots,p-1\}$. Como el conjunto
%	\[
%	\{x_1,\dots,x_{j-1},x_j^{n_j},x_{j+1},\dots,x_m\}
%	\]
%	genera al grupo $G$ y $\{x_1,\dots,x_{j-1},x_{j+1},\dots,x_m\}$ no lo hace,
%	entonces $x_j^{n_j}\not\in\Phi(G)$ por el lema de los no-generadores. 
	$\Phi(G)=\{1\}$.
\end{prob}

\begin{prob}
	Let $G=\SL_2(3)$. The unique minimal normal subgroup of $G$ is
	$Z(\SL_2(3))\simeq C_2$:
\end{prob}



\begin{prob}
Let $q\geq5$. 
Prove that $\SL_n(q)$ is perfect.  	
\end{prob}


\begin{prob}
Let $G$ be a perfect group and $N$ be a normal subgroup of $G$. Then $G/N$ is perfect.  	
\end{prob}

% \begin{prob}
% Is the result of Exercise~\ref{prob:G(X,r)solvable} true in the case of non-involutive finite solutions?
% \end{prob}

\section{Notes}

Solvable groups...

%In~\cite{MR1722951}, Etingof, Schedler and Soloviev proved that the structure group of a finite involutive
%solution is always solvable. 
% The proof can translated into the language of braces 
% to obtain Theorem~\ref{thm:add_nilpotent}.

