\chapter{The transfer map}

% nociones básicas, Schur
% Schur para braces y torsión del grupo de estructura

\section*{A}

\index{Transfer map}
Let $G$ be a group and $H$ be a finite index subgroup. We will define
a group homomorphism $G\to H/[H,H]$, known as the \textbf{transfer map} of $G$
on $H$. Fix a \textbf{left transversal} $T$ of $H$ in $G$.

\begin{lemma}
	\label{lem:sigma}
	Let $G$ be a group and $H$ be a subgroup of finite index $n=(G:H)$. Let
	$S=\{s_1,\dots,s_n\}$ and $T=\{t_1,\dots,t_n\}$ be transversals of $H$ in $G$.
	If $g\in G$, there exist unique $h_1,\dots,h_n\in H$ and a permutation 
	$\sigma\in\Sym_n$  
	such that
		\[
		gt_i=s_{\sigma(i)}h_i,\quad
		i\in\{1,\dots,n\}.
	\]
\end{lemma}

\begin{proof}
	If $i\in\{1,\dots,n\}$, then there exists a unique $j\in\{1,\dots,n\}$ such that $gt_i\in
	s_jH$. Thus there exists a unique $h_i\in H$ such that $gt_i=s_jh_i$. Take 
	$\sigma(i)=j$ and thus there is a well-defined map 
	$\sigma\colon\{1,\dots,n\}\to\{1,\dots,n\}$.  To prove that 
	$\sigma\in\Sym_n$ it is enough to check that $\sigma$ is injective. If
	$\sigma(i)=\sigma(k)=j$, since $gt_i=s_jh_i$ and $gt_k=s_jh_k$, it follows that 
	$t_i^{-1}t_k=h_i^{-1}h_k\in H$. Hence $i=k$, as $t_iH=t_kH$.
\end{proof}

%\begin{exercise} 
%	Demuestre que la acción de $G$ en el conjunto de coclases $H\backslash G$
%	dada por $(Hx)\cdot g=H(xg)$ induce una acción a derecha de $G$ en $T$. 
%\end{exercise}
%
%\begin{svgraybox}
%	Si $t\in T$ y $g\in G$ existe un único $t\cdot g\in T$ tal que $(Ht)\cdot
%	g=H(t\cdot g)$. Como $G$ actúa en $H\backslash G$ por multiplicación a
%	derecha, $H(t\cdot (g_1g_2))=H((t\cdot g_1)\cdot g_2)$ para todo $t\in T$,
%	$g_1,g_2\in G$.
%\end{svgraybox}
%
%\begin{exercise} 
%	Demuestre que si $t\in T$ y $g\in G$ entonces $tg(t\cdot g)^{-1}\in H$.
%\end{exercise}
%
%\begin{svgraybox}
%	Sean $t\in T$ y $g\in G$. Entonces $t\cdot g\in H$ es el único elemento de
%	$H$ tal que $H(tg)=H(t\cdot g)$. Luego $(tg)(t\cdot g)^{-1}\in H$.
%\end{svgraybox}


Let $G$ be a group and $H$ be a subgroup of $G$ of finite index $n$. If
$T=\{t_1,\dots,t_n\}$ is a transversal of $H$ in $G$, we define the map 
	\[
		\nu_T\colon G\to H/[H,H],\quad
		\nu_T(g)=\prod_{i=1}^n h_i
	\]
where $gt_i=t_jh_i$. Note that the product is well-defined since $H/[H,H]$ is an abelian group. 
We now prove that the map does not depend on the 
transversal. 

\begin{lemma}
	\label{lem:nu_T}
	Let $G$ be a group and $H$ be a subgroup of $G$ of finite index. if $T$ 
	and $S$ are transversal of $H$ in $G$, then $\nu_T=\nu_S$.
\end{lemma}

\begin{proof}
	Assume that $gs_i=s_{\sigma(i)}h_i$ for all $i$. Write 
	$s_i=t_ik_i$, $k_i\in H$. If $l_i=k_{\sigma(i)}h_ik_i^{-1}$, then 
	\[
	gt_i=gs_ik_i^{-1}=s_{\sigma(i)}h_ik_i^{-1}=t_{\sigma(i)}k_{\sigma(i)}h_ik_i^{-1}=t_{\sigma(i)}l_i
	\]
	for all $i\in\{1,\dots,n\}$. Moreover,  
	\[
			s_{\sigma(i)}^{-1}gs_i=k_{\sigma(i)}^{-1}t_{\sigma(i)}^{-1}gt_ik_i.
	\]
	Since $H/[H,H]$ is abelian, 
	\begin{align*}
		\nu_S(g)
		&=\prod_{i=1}^n s_{\sigma(i)}^{-1}gs_i
		=\prod_{i=1}^n k_{\sigma(i)}^{-1}t_{\sigma(i)}^{-1}gt_ik_i\\
		&=\prod_{i=1}^n k_{\sigma(i)}^{-1}\prod_{i=1}^n k_i\prod_{i=1}^n t_{\sigma(i)}^{-1}gt_i
		=\prod_{i=1}^n t_{\sigma(i)}^{-1}gt_i
		=\nu_T(g).\qedhere
	\end{align*}
\end{proof}

By Lemma~\ref{lem:nu_T}, if $H$ is a finite-index subgroup of $G$, the map 
\[
\nu\colon G\to H/[H,H],
\quad
\nu(g)=\nu_T(g),
\]
where $T$ is some transversal of $H$ in $G$, is well-defined. 

\begin{theorem}
	\label{theorem:transfer}
	Let $G$ be a group and $H$ be a finite-index subgroup of $G$. Then $\nu(xy)=\nu(v)\nu(y)$ 
	for all $x,y\in G$.
\end{theorem}

\begin{proof}
	Let $T=\{t_1,\dots,t_n\}$ be a transversal of $H$ in $G$. Let $x,y\in G$. By 
	Lemma~\ref{lem:sigma}, there exist unique elements $h_1,\dots,h_n,k_1,\dots,k_n\in H$ and 
	there are permutations $\sigma,\tau\in\Sym_n$ such that $xt_i=t_{\sigma(i)}h_i$ and 
	$yt_i=t_{\tau(i)}k_i$. Since 
	\[
	xyt_i=xt_{\tau(i)}k_i=t_{\sigma\tau(i)}h_{\tau(i)}k_i
	\]
	and $H/[H,H]$ is abelian, 
	\[
		\nu(xy)=\prod_{i=1}^n h_{\tau(i)}k_i=\prod_{i=1}^n h_{\tau(i)}\prod_{i=1}^n k_i=\nu(x)\nu(y).\qedhere
	\]
\end{proof}

If $G$ is a group and $H$ is a finite-index subgroup of $G$, the 
\textbf{transfer homomorphism} is the group homomorphism $\nu\colon G\to H/[H,H]$,
$\nu(g)=\nu_T(g)$, for some transversal $T$ of $H$ in $G$.
	
%\begin{theorem}
%	\label{theorem:P_noabeliano}
%	Sea $G$ un grupo finito. Sea $p$ un primo que divide al orden de $[G,G]\cap
%	Z(G)$. Si $P\in\Syl_p(G)$ entonces $P$ es no abeliano.
%\end{theorem}
%
%\begin{proof}
%	Supongamos que $P$ es abeliano y sea $T=\{t_1,\dots,t_n\}$ un transversal
%	de $P$ en $G$. Como $[G,G]\cap Z(G)$ es un subgrupo normal de $G$, podemos
%	suponer que $P\cap [G,G]\cap Z(G)\ne1$. Sea $z\in P\cap [G,G]\cap Z(G)$ tal
%	que $z\ne1$. 
%
%	Sea $\nu\colon G\to P$ el morfismo de transferencia. Vamos a calcular
%	$\nu(z)$ con el lema~\ref{lemma:sigma}. Para cada $i\in\{1,\dots,n\}$ sean
%	$x_1,\dots,x_n\in P$ y sea $\sigma\in\Sym_n$ tales que
%	$zt_i=t_{\sigma(i)}x_i$. Como $z\in Z(G)$, se tiene
%	$t_i=t_{\sigma(i)}x_iz^{-1}$ y luego la unicidad del lema~\ref{lemma:sigma}
%	implica que $\sigma=\id$ y $x_i=z$ para todo $i$. Luego 
%	\[
%	\nu(z)=z^{|T|}=z^{(G:P)}. 
%	\]
%
%	Como $P$ es abeliano, $[G,G]\subseteq\ker\nu$. Luego $\nu(z)=1$. Esto es
%	una contradicción pues $1\ne z\in P$ y $z^{(G:P)}=1$ implica que $z$ tiene
%	orden no divisible por $p$. 
%\end{proof}

% rotman 7.47, 7.48

\begin{lemma}
	\label{lem:evaluation}
	Let $G$ be a group and $H$ be a subgroup of $G$ with $(G:H)=n$. Let 
	$T=\{t_1,\dots,t_n\}$ be a transversal of $H$ in $G$.  For each $g\in G$
	there are elements $s_{1},\dots,s_{m}\in T$ and positive integers $n_1,\dots,n_m$
	(depending on $g$) such that 
	\[
	s_i^{-1}g^{n_i}s_i\in H,
	\quad
	n_1+\cdots+n_m=n\quad\text{and}\quad   
	\nu(g)=\prod_{i=1}^m s_i^{-1}g^{n_i}s_i.
	\]
\end{lemma}

% TODO: explicar mejor!

\begin{proof}
	For each $i$ there exist $h_1,\dots,h_n\in H$ and $\sigma\in\Sym_n$ such that 
	$gt_i=t_{\sigma(i)}h_i$. Write $\sigma$ as a product 
	\[
		\sigma=\alpha_1\cdots\alpha_m
	\]
	of disjoint cycles. 

	For each $i\in\{1,\dots,n\}$, write  
	$\alpha_i=(j_{1}\cdots j_{n_i})$. Since  
	\[
		g t_{j_k}=t_{\sigma(j_k)}h_{j_k}=\begin{cases}
			t_{j_1}h_{n_i} & \text{if $i=n_i$},\\
			t_{j_{k+1}}h_{k} & \text{otherwise},
		\end{cases}
	\]
	it follows that 
	\[
	t_{j_1}^{-1}g^{n_i}t_{j_1}
	=t_{j_1}^{-1}gg^{n_i-1}t_{j_1}
	=t_{j_1}^{-1}gt_{j_r}h_{j_{r-1}}\cdots h_{j_1}
	=h_{j_r}\cdots h_{j_1}\in H,
	\]
	and we let $s_i=t_{j_1}$. Now the claim follows, since $\nu(g)=h_1\cdots h_{n}$.
\end{proof}

\begin{proposition}
	\label{prop:v(g)=g^n}
	Let $G$ be a group and $H$ be a central subgroup of index $n$. Then 
	$\nu(g)=g^n$ for all $g\in G$.
\end{proposition}

\begin{proof}
	Let $g\in G$. By Lemma~\ref{lem:evaluation}, there exist $s_1,\dots,s_m\in
	H$ such that $s_i^{-1}g^{n_i}s_i\in H$ and $\nu(g)=\prod_{i=1}^m
	s_i^{-1}g^{n_i}s_i$.  Since $H$ is central in $G$, then it is normal in $G$. Thus 
	\[
	g^{n_i}=s_i(s_i^{-1}g^{n_i}s_i)s_i^{-1}\in H\subseteq Z(G)
	\]
	and hence  
	\[
		\nu(g)
		=\prod_{i=1}^m s_i^{-1}g^{n_i}s_i
		=\prod_{i=1}^m g^{n_i}
		=g^{\sum_{i=1}^m n_i}
		=g^n.\qedhere
	\]
\end{proof}

\begin{exercise}
	Let $G$ be a group with a central subgroup $H$ of index $n$. Then 
	$g\mapsto g^n$ is a group homomorphism. 
\end{exercise}

\begin{exercise}
	\label{corollary:[x,y]^n=1}
	Let $G$ be a group such that $(G:Z(G))=n$. If $x,y\in G$, then $[x,y]^n=1$. 
\end{exercise}

Another application. 

\begin{proposition}
	\label{prop:semidirecto}
	Sea $G$ un grupo finito y sea $H$ un subgrupo abeliano de
	índice $n$, donde $n$ es coprimo con $|H|$.  Sea
	$N=\ker(\nu\colon G\to H)$. Entonces $G\simeq N\rtimes H$.
\end{proposition}

\begin{proof}
	Since $H$ is abelian, $H=H/[H,H]$ and the transfer map is 
	$\nu\colon G\to H$. By Lemma~\ref{lem:evaluation}, 
	\[
		\nu(h)
		=\prod_{i=1}^m s_i^{-1}h^{n_i}s_i
		=\prod_{i=1}^m h^{n_i}
		=h^{\sum_{i=1}^m n_i}=h^n.
	\]
	The composition $H\hookrightarrow G\xrightarrow{\nu} H$ is a group homomorphism. 
	
	We claim that it is an isomorphism. It is injective: If $h^n=1$, then 
	$|h|$ divides $|H|$ and divides $n$. Since $n$ and $|H|$ are
	coprime, $h=1$. It is surjectice: Since $n$ and $|H|$ are coprime, there exists 
	$m\in\Z$ such that $nm\equiv 1\bmod |H|$. If $h\in H$, then $h^m\in
	H$ and $\nu(h^m)=h^{nm}=h$. 

	Therefore $G\simeq N\rtimes H$, as $N$ is normal in $G$, $N\cap
	H=\{1\}$ and $G=NH$ (because $|NH|=|N||H|$ and $G/N\simeq H$).
\end{proof}

\begin{exercise}
	Let $H$ be a central subgroup of a finite group $G$. If $|H|$
	and $|G/H|$ are coprime, then $G\simeq H\times G/H$.
\end{exercise}

%\begin{proof}
%	Es consecuencia inmediata del corolario~\ref{corollary:semidirecto} pues
%	$H$ es normal por ser un subgrupo central.
%\end{proof}

% TODO: Transitivity of the transfer

\section*{B}

As an application of the transfer map we will prove several theorems
about the commutator subgroup. We start
with the following result which of course it is of independ interest. 

\begin{theorem}[Dietzmann]
	\index{Dietzmann!theorem}
	\label{theorem:Dietzmann} 
	Let $G$ be a group and $X\subseteq G$ be a finite subset of 
	$G$ closed under conjugation. If there exists $n\in\N$ such that 
	$x^n=1$ for all $x\in X$, then $\langle X\rangle$ is a finite subgroup of 
	$G$.
\end{theorem}

\begin{proof}
	Let $S=\langle X\rangle$ be the subgroup generated by $S$. Since $x^{-1}=x^{n-1}$, every element 
	of $S$ can be written as a finite product of elements of $X$. 
	
	Fix $x\in X$. We claim that if $x\in X$ appears $k\geq 1$ times in the representation of $s$, then 
	$s$ is a product of $m$
	elements of $X$ where the first $k$ elements are equal to $x$. Assume that 
	\[
	s=x_1x_2\cdots x_{t-1}xx_{t+1}\cdots x_m,
	\]
	where each $x_j\ne x$ for all $j\in\{1,\dots,t-1\}$. Then 
	\[
		s=x(x^{-1}x_1x)(x^{-1}x_2x)\cdots (x^{-1}x_{t-1}x)x_{t+1}\cdots x_m
	\]
	is a product of $m$ elements of $X$ since $X$ is closed under conjugation and
	the first element is our $x$. The same argument implies that $s$
	can be written as 
	\[
		s=x^ky_{k+1}\cdots y_m,
	\]
	where the $y_j$ are elements of $X\setminus\{x\}$.

	Let $s\in S$. Write $s$ as a product of $m$ elements of $X$,
	where $m$ is minimal. To see that $S$ is finite it is enough to show that  
	$m\leq (n-1)|X|$. 
	
	If $m>(n-1)|X|$, then 
	at least one $x\in X$ appears $n$ times in the representation of $s$. Without loss of generality, write
	\[
		s=x^nx_{n+1}\cdots x_m=x_{n+1}\cdots x_m,
	\]
	a contradiction to the minimality of $m$. 
\end{proof}

We prove Schur's theorem we need a lemma.

\begin{lemma}
	\label{lemma:[s,t]} 
	Let $G$ be a group and $T$ be a transversal of $Z(G)$ in
	$G$. Then each commutator of $G$ is of the form $[s,t]$ for $s,t\in T$. In particular, 
	$G$ has a finite number of commutators if $Z(G)$ is of finite index. 
\end{lemma}

\begin{proof}
	Every element of $G$ is of the form $sx$ for $s\in T$ and $x\in
	Z(G)$. To prove the first claim note that 
	$[sx,ty]=[s,t]$, 
	as $x,y\in Z(G)$. The second claim now follows from	$|T|=(G:Z(G))$.
\end{proof}

We now prove Schur's theorem. 

\begin{theorem}[Schur]
	\label{theorem:Schur_commutador}
	\index{Schur!theorem}
	If $Z(G)$ has finite index in $G$, then $[G,G]$ is finite. 
\end{theorem}

\begin{proof}
	Let $X=\{[x,y]:x,y\in G\}$. By Lemma~\ref{lemma:[s,t]}), $X$ is finite.
	Moreover, $X$ is closed under conjugation, 
	\[
		g[x,y]g^{-1}=[gxg^{-1},gyg^{-1}]
	\]
	for all $g,x,y\in G$. If $n=(G:Z(G))$, then $x^n=1$ for all $x\in X$
	by Corollary~\ref{corollary:[x,y]^n=1}. Thus the claim follows from 
	Dietzmann's theorem (Theorem~\ref{theorem:Dietzmann}.
\end{proof}

\begin{corollary}[Sury]
	If the set of commutators of a group $G$ is finite, then 
	$[G,G]$ is finite.
\end{corollary}

\begin{proof}
	Let $C$ be the set of commutators of $G$ and let $H$ be the subgroup of $G$ 
	generated by $C$. The group $H$ is finitely generated, say by the elements 
	$h_1,\dots,h_n$. Since $h\in Z(H)$ if and only if $h\in C_H(H_i)$ for all 
	$i\in\{1,\dots,n\}$, we conclude that $Z(H)=\cap_{i=1}^n C_H(h_i)$. Moreover, if 
	$h\in H$, then $hh_ih^{-1}=ch_i$ for some $c\in C$. Thus the conjugacy class of each 
	$h_i$ contains at most as many elements as $C$. This implies that 
	\[
		|H/Z(H)|=|H/\cap_{i=1}^n C_H(H_i)|\leq\prod_{i=1}^n (H:C_H(h_i))\leq |C|^n.
	\]
	Since $H/Z(H)$ is finite, $[H,H]$ is finite. Hence  
	$[G,G]=\langle C\rangle\subseteq [H,H]$ 
	is a finite group.
\end{proof}

The corollary can be used to prove another proof of
the following result. 

\begin{theorem}[Hilton--Niroomand]
	\index{Teorema de!Hilton--Niroomand}	
	Let $G$ be a finitely generated group. If $[G,G]$ is finite and $G/Z(G)$ is generated by
	$n$ elements, then  
	\[
	|G/Z(G)|\leq |[G,G]|^n. 
	\]
\end{theorem}

\begin{proof}
	Assume that $G/Z(G)=\langle x_1Z(G),\dots,x_nZ(G)\rangle$. Let 
	\[
		f\colon G/Z(G)\to [G,G]\times\cdots\times [G,G],
		\quad
		y\mapsto ([x_1,y],\dots,[x_n,y]).
	\]
	Note that $f$ is well-defined: If $y\in G$ y $z\in Z(G)$, then 
	\[
		f(yz)=[x_i,yz]=[x_i,y]=f(y). 
	\]
	We claim that $f$ is injective. Assume that $f(xZ(G))=f(yZ(G))$. Then 
	$[x_i,x]=[x_i,y]$ for all $i\in\{1,\dots,n\}$. For each $i$ we compute  
	\begin{align*}
		[x^{-1}y,x_i] &= x^{-1}[y,x_i]x[x^{-1},x_i]\\
		&=x^{-1}[y,x_i][x_i,x]x=x^{-1}[x_i,y]^{-1}[x_i,x]x=x^{-1}[x_i,y]^{-1}[x_i,y]x=1.
	\end{align*}
	This implies that $x^{-1}y\in Z(G)$. Indeed, since  
	every $g\in G$ can be written as $g=x_kz$ for some $k\in\{1,\dots,n\}$ and some $z\in Z(G)$, 
	it follows that 
    \[
    [x^{-1}y,g]=[x^{-1}y,x_kz]=[x^{-1}y,x_k]=1.
    \]
    Thus $f$ is injective and hence
	$|G/Z(G)|\leq |[G,G]|^n$. 
\end{proof}

% serre, 7.12
An application to infinite groups. 

\begin{theorem}
	Let $G$ be a torsion-free group that contains a finite-index subgroup isomorphic to  
	$\Z$. Then $G\simeq\Z$.
\end{theorem}

\begin{proof}
	We may assume that $G$ contains a finite-index normal subgroup isomorphic to $\Z$. Indeed, 
	if $H$ is a finite-index subgroup of $G$ such that $H\simeq\Z$, then 
	$K=\cap_{x\in G}xHx^{-1}$ is a non-trivial normal subgroup of $G$ (because $K=\Core_G(H)$ and 
	$G$ has no torsion) and hence $K\simeq\Z$ (because  
	$K\subseteq H$) and $(G:K)=(G:H)(H:K)$ is finite.
	The action of $G$ on $K$ by conjugation induces a group homomorphism  
	$\epsilon\colon G\to\Aut(K)$. Since $\Aut(K)\simeq\Aut(\Z)=\{-1,1\}$, 
	there are two cases to consider.
	
	Assume first that $\epsilon=\id$. Since $K\subseteq Z(G)$, let
	$\nu\colon G\to K$ be the transfer homomorphism. By
	Proposition~\ref{prop:v(g)=g^n}, $\nu(g)=g^n$, where $n=(G:K)$. Since
	$G$ has no torsion, $\nu$ is injective. Thus
	$G\simeq\Z$ because it is isomorphic to a subgroup of $K$.

	Assume now that $\epsilon\ne\id$. Let $N=\ker\epsilon\ne G$. Since
	$K\simeq\Z$ is abelian, $K\subseteq N$. The result proved in the previous paragraph 
	applied to $\epsilon|_N=1$ implies that $N\simeq\Z$, as 
	$N$ contains a finite-index subgroup isomorphic to $\Z$. Let $g\in G\setminus N$. 
	Since $N$ is normal in $G$, $G$ acts by conjugation on $N$ and hence 
	there exists a group homomorphism $c_g\in\Aut(N)\simeq\{-1,1\}$. Since
	$K\subseteq N$ y $g$ acts non-trivially on $K$, 
	\[
	c_g(n)=gng^{-1}=n^{-1}
	\]
	for all $n\in N$.  Since 
	$g^2\in N$, 
	\[
		g^2=gg^2g^{-1}=g^{-2}.
	\]
	Therefore $g^4=1$, a contradiction since $g\ne1$ and $G$ has no torsion.\end{proof}

% TODO: Alperin--Kuo?


\section*{C}

\begin{lemma}
	\label{lem:normal_complement}
	Let $G$ be a finite group and $p$ be a prime number dividing the order of $G$. Let
	$P\in\Syl_p(G)$. If $g,h\in C_G(P)$ are conjugate in $G$, then they are
	conjugate in $N_G(P)$.
\end{lemma}

\begin{proof}
	Let $x\in G$ be such that $g=xhx^{-1}$. Then $g\in C_G(xPx^{-1})$ and hence $P$
	and $xPx^{-1}$ are Sylow subgroups of $C_G(g)$. By the second Sylow's theorem, 
	there exists 
	$c\in C_G(g)$ such that $P=cxP(cx)^{-1}$. Theredfore $cx\in N_G(P)$ and
	\[
	(cx)h(cx)^{-1}=c(xhx^{-1})c^{-1}=cgc^{-1}=g.\qedhere
	\]
\end{proof}

\index{Normal complement}
\index{Group!$p$-nilpotente}
Let $G$ be a finite group and $p$ a prime number dividing the order of $G$. A 
\textbf{normal $p$-complement} is a normal subgroup $N$ of $G$ of
order not divisible by $p$ and such that $(G:N)$ is a power of $p$.
We say that $G$ is \textbf{$p$-nilpotent} if $G$ contains a normal $p$-complement.

\begin{proposition}
	If the group $G$ admits a normal $p$-complement $N$, then $N$ is a characteristic subgroup of $G$.
\end{proposition}

\begin{proof}
	Assume that $|G|=p^\alpha n$, where $\gcd(n,p)=1$. Let 
	$\pi\colon G\to G/N$ be the canonical map. Then $|N|=n$. We claim that 
	$N$ is the unique subgroup of $G$ of order $n$. If
	$K$ is a subgroup of $G$ of order $n$, then $\pi(K)\simeq K/K\cap N$
	and hence $|\pi(K)|$ divides $n$. Since $\pi(K)$ is a subgroup of $G/N$, 
	it follows that 
	$|\pi(K)|$ divides $p$. Thus $\pi(K)$ is trivial and hence 
	$K=N$ and $G$ contains a unique subgroup of order $n$. In particular, $N$ is characteristic in $G$. 
\end{proof}

\begin{theorem}[Burnside]
	\index{Burnside's!theorem}
	\label{thm:Burnside:normal_complement}
	Let $G$ be a finite group and $p$ be a prime number dividing the order of $G$. Let
	$P\in\Syl_p(G)$ be such that $P\subseteq Z(N_G(P))$. Then $G$ is
	$p$-nilpotent.
\end{theorem}

\begin{proof}
	Since $P$ is abelian, let $\nu\colon G\to P$ be the transfer map. 
	Let $g\in P$.  By Lemma~\ref{lem:evaluation}, there exist $s_1,\dots,s_m\in
	G$ and $n_1,\dots,n_m$ such that $n_1+\cdots+n_m=n$,
	$s_i^{-1}g^{n_i}s_i\in P$ and  
	\[
		v(g)=\prod_{i=1}^m s_i^{-1}g^{n_i}s_i.
	\]
	Since $P$ is abelian, $P\subseteq C_G(P)$. By Lemma~\ref{lem:normal_complement}, 
	there exist elements $c_i\in N_G(P)$ such that 
	\[
	s_i^{-1}g^{n_i}s_i=c_i^{-1}g^{n_i}c_i.
	\]
	Thus $s_i^{-1}g^{n_i}s_i=g_i^{n_i}$, as $P\subseteq Z(N_G(P))$. Therefore 
	$\nu(g)=g^n$, where $n=(G:P)$. Since $n$ and $|P|$ are coprime, 
	there exist $r,s\in\Z$ such that $rn+s|P|=1$. Thus the restriction 
	$\nu|_P$ is surjective, as 
	\[
	g=(g^r)^n=\nu(g^r).
	\]
	By the isomorphism theorem, $P/\ker\nu\cap P\simeq\nu(P)=P$. 
	Thus $\ker\nu\cap P=\{1\}$. Moreover, $\nu(G)=\nu(P)$, as  
	$P\supseteq \nu(G)\supseteq \nu(P)=P$.
	
	We now claim that $\ker\nu$ is a normal $p$-complement in $G$. Indeed, $\ker\nu$ is normal in $G$. 
	Since $(G:\ker\nu)=|\nu(G)|=|P|$ and $P$ is a Sylow $p$-subgroup, 
	$\ker\nu$ has order coprime with $p$.
%	Sea $K=\ker\nu$. Por el primer teorema de isomorfismo, $G/K\simeq P$. El
%	subgrupo $K$ es un complemento normal de $P$ en $G$ pues $G=KP$ y $K\cap
%	P=1$ (pues $|K|=n$ y $|P|$ son coprimos).
\end{proof}

% $P\subseteq Z(N_G(P))$ si y sólo si $N_G(P)=C_G(P)$.


\begin{lemma}
	\label{lemma:NC}
	Sea $G$ un grupo y $H$ un subgrupo de $G$. Entonces $C_G(H)$ es un subgrupo
	normal de $N_G(H)$ y $N_G(H)/C_G(H)$ es isomorfo a un subgrupo de
	$\Aut(H)$.
\end{lemma}

\begin{proof}
	Sea $\phi\colon N_G(H)\to\Aut(H)$,  $\phi(g)=c_g|H$, donde
	$c_g(h)=ghg^{-1}$.  La función $\phi$ está bien definida (pues su dominio
	es $N_G(H)$) y es morfismo de grupos. Como $\ker\phi=C_G(H)$, se tiene que
	$C_G(H)$ es normal en $N_G(H)$. Por el primer teorema de isomorfismo,
	$N_G(H)/C_G(H)\simeq\phi(N_G(H))\leq\Aut(H)$.
\end{proof}

\begin{corollary}
	\label{corollary:Sylow_ciclico}
	Sea $G$ un grupo finito y sea $p$ el menor primo que divide a $|G|$. Si
	algún $P\in\Syl_p(G)$ es cíclico, $G$ es $p$-nilpotente.
\end{corollary}

\begin{proof}
	Supongamos que $|P|=p^m$.  Por el lema~\ref{lemma:NC}, $N_G(P)/C_G(P)$ es
	isomorfo a un subgrupo de $\Aut(P)$. Como $P$ es cíclico, $|N_G(P)/C_G(P)|$ divide a 
	\[
		|\Aut(P)|=\phi(|P|)=p^{m-1}(p-1).
	\]
	Como $P\subseteq C_G(P)$ por ser $P$ abeliano, $p$ es coprimo con
	$|N_G(P)/C_G(P)|$.  Luego $|N_G(P)/C_G(P)|$ divide a $p-1$. Pero $p-1$ y
	$|G|$ son coprimos, pues $p$ es el menor primo que divide a $|G|$.  Como
	además $|N_G(P)/C_G(P)|$ divide al orden de $G$, se concluye que
	$|N_G(P)/C_G(P)|=1$, es decir: $N_G(P)=C_G(P)$. 

	Como $P$ es abeliano, $P\subseteq Z(C_G(P))=Z(N_G(P))$. El teorema de
	Burnside~\ref{theorem:Burnside:normal_complement} implica entonces que $G$
	es $p$-nilpotente. 
\end{proof}

\begin{exercise}
	Sea $G$ un grupo finito tal que todos sus subgrupos de Sylow son cíclicos.
	Entonces $G$ es resoluble.
\end{exercise}

Vamos a demostrar algo más fuerte:

\begin{corollary}
	\label{corollary:Sylow_ciclicos:resoluble}
	Sea $G$ un grupo finito tal que todos sus subgrupos de Sylow son cíclicos.
	Entonces $G$ es superresoluble.
\end{corollary}

%\begin{proof}
%	Supongamos que $G$ es no trivial y hagamos inducción en $|G|$. Si $p$ es el
%	menor primo que divide a $|G|$, por el
%	corolario~\ref{corollary:Sylow_ciclico} el grupo $G$ tiene un
%	$p$-complemento normal $N$. Por hipótesis inductiva, $N$ es super
%	resoluble, y entonces existe una sucesión
%	\[
%		1=N_0\subseteq N_1\subseteq\cdots\subseteq N_k=N
%	\]
%	de subgrupos normales de $N$ tal que cada cociente $N_{i+1}/N_i$ es cíclico
%	de orden primo.  Como $K$ es un complemento normal, es un subgrupo
%	característico de $G$ y luego cada $N_i$ es normal en $G$. Sea $P\in\Syl_p(G)$ 
%	y supongamos que $|G|=p^{\alpha}m$ con $m$ no divisible por $p$. 
%	Sea $\pi\colon G\to G/K\simeq P$ el morfismo canónico. 
%	Sabemos que para cada $j\in\{1,\dots,\alpha-1\}$, existe un subgrupo $H_h$ tal que $K\subseteq H_j$ y $|\pi(H_j)|=p^j$. 
%	Como $P$ es cíclico, todo subgrupo es característico en $P$ y luego 
%\end{proof}
%\begin{example}
%	Si los Sylows de un grupo finito $G$ son cíclicos y $p$ es el menor primo que
% 	divide a $|G|$, sabemos que existe un $p$-complemento. Este subgrupo podría no ser cíclico!
%\end{example}

\begin{proof}
	Supongamos que $G$ es no trivial y hagamos inducción en $|G|$. Si $p$ es el
	menor primo que divide a $|G|$, por el
	corolario~\ref{corollary:Sylow_ciclico} el grupo $G$ tiene un
	$p$-complemento normal $N$. Por hipótesis inductiva, $N$ es resoluble. Como
	$G/N$ es un $p$-grupo, es resoluble. Luego $G$ es resoluble.
\end{proof}

\begin{corollary}
	Sea $G$ un grupo finito cuyo orden es libre de cuadrados. Entonces $G$ es
	resoluble.
\end{corollary}

\begin{proof}
	Es consecuencia del corolario~\ref{corollary:Sylow_ciclicos:resoluble} pues
	en este caso todo subgrupo de Sylow es cíclico.  
\end{proof}

\begin{corollary}
	Sea $G$ un grupo finito simple no abeliano y sea $p$ el menor primo que
	divide a $|G|$. Entonces $p^3$ divide a $|G|$ o bien $12$ divide a $|G|$.
\end{corollary}

\begin{proof}
	Sea $P\in\Syl_p(G)$. Por el corolario~\ref{corollary:Sylow_ciclico}, $P$ no
	es cíclico, y entonces $|P|\geq p^2$. Si $p^3$ no divide a $|G|$, $P\simeq
	C_p\times C_p$ es un $\F_p$-espacio vectorial de dimensión dos. Como
	$|N_G(P)/C_G(P)|$ divide al orden de $G$, los divisores primos de
	$|N_G(P)/C_G(P)|$ son $\geq p$. Además, como $N_G(P)/C_G(P)$ es isomorfo a
	un subgrupo de $\Aut(P)$ por el lema~\ref{lemma:NC} y
	$\Aut(P)\simeq\GL_2(p)$ tiene orden $(p^2-1)(p^2-p)=p(p+1)(p-1)^2$,
	$|N_G(P)/C_G(P)|$ divide a $p(p+1)(p-1)^2$.  Como $P$ es abeliano,
	$P\subseteq C_G(P)$. Entonces $|N_G(P)/C_G(P)|$ es coprimo con $p$ y luego
	$|N_G(P)/C_G(P)|$ divide a $(p+1)(p-1)^2$. Como $p$ es el menor primo que
	divide a $|G|$, los números $p-1$ y $|N_G(P)/C_G(P)|$ son coprimos, y
	entonces $|N_G(P)/C_G(P)|$ divide a $p+1$.  Además, por el teorema de
	Burnside~\ref{theorem:Burnside:normal_complement}, $|N_G(P)/C_G(P)|\ne1$.
	Esto implica que $p=2$ pues si $p$ es impar el menor primo que divide a
	$|N_G(P)/C_G(P)|$ es $\geq p+2$.  Como entonces $p=2$, se concluye que
	$|N_G(P)/C_G(P)|=3$ y luego $|G|$ es divisible por $12=2^23$. 
\end{proof}

\begin{theorem}
	\label{theorem:[GG]PZNG(P)=1}
	Sea $G$ un grupo finito y sea $P$ un subgrupo de Sylow abeliano. Entonces 
	$[G,G]\cap P\cap Z(N_G(P))=1$.
\end{theorem}

\begin{proof}
	Sea $x\in [G,G]\cap P\cap Z(N_G(P))$ y sea $\nu\colon G\to P$ el morfismo
	de transferencia.  Por el lema~\ref{lemma:evaluation} existen
	$s_1,\dots,s_m\in G$ y existen $n_1,\dots,n_m$ tales que
	$n_1+\cdots+n_m=(G:P)$, $s_i^{-1}g^{n_i}s_i\in P$ y 
	\[
		v(x)=\prod_{i=1}^m s_i^{-1}x^{n_i}s_i.
	\]
	Como $P$ es abeliano, $P\subseteq C_G(P)$. Entonces $x^{n_i}$ y
	$s_i^{-1}x^{n_i}s_i$ son conjugados en $N_G(P)$ por el
	lema~\ref{lemma:normal_complement}. Como $x^{n_i}$ es central en $N_G(P)$ y
	$[G,G]\subseteq\ker\nu$, se concluye que $x=1$ pues $1=\nu(x)=x^{(G:P)}$ y
	$x\in P$.
\end{proof}

\begin{corollary}
	Sea $G$ un grupo finito no abeliano y sea $P\in\Syl_2(G)$ tal que $P\simeq
	C_{a_1}\times\cdots\times C_{a_k}$ con $a_1>a_2\geq a_3\geq\cdots\geq
	a_k\geq 2$.  Entonces $G$ no es simple. 
\end{corollary}

\begin{proof}
	Sea $S=\{x^{n/2}:x\in P\}$. Es fácil ver que $S$ es un subgrupo de $P$ y
	que $S$ es característico en $P$, es decir: $f(S)\subseteq S$ para todo
	$f\in\Aut(P)$. Como $S\simeq C_2$, podemos escribir $S=\{1,s\}$. Entonces
	$s\in Z(N_G(P))$ pues $gsg^{-1}\in S$ para todo $g\in N_G(P)$. Por el
	teorema~\ref{theorem:[GG]PZNG(P)=1}, $s\not\in[G,G]$ y luego $[G,G]\ne G$.
	Si $G$ fuera simple, $G$ sería abeliano pues $[G,G]=1$.
\end{proof}

Vimos en el corolario~\ref{corollary:Sylow_ciclicos:resoluble} que todo grupo
tal que todos sus subgrupos de Sylow son cíclicos es resoluble.

\begin{definition}
	\index{Z-grupo}
	Un Z-grupo es un grupo finito $G$ tal que 
	odos sus subgrupos de Sylow son
	cíclicos.
\end{definition}

\index{Grupo!meta-cíclico}
Un grupo $G$ se dice \emph{meta-cíclico} si $G$ tiene un subgrupo normal $N$
cíclico tal que $G/N$ es cíclico.

\begin{lemma}
	Si $G$ es un grupo resoluble, entonces $C_G(F(G))=F(G)$.
%	https://groupprops.subwiki.org/wiki/Solvable_implies_Fitting_subgroup_is_self-centralizing		
\end{lemma}

\begin{proof}
	
\end{proof}

\begin{theorem}
	\label{theorem:Z=>metacyclic}
	Todo Z-grupo es meta-cíclico.
\end{theorem}

\begin{proof}
	Sea $G$ un Z-grupo.	Por el
	corolario~\ref{corollary:Sylow_ciclicos:resoluble}, $G$ es resoluble y
	entonces, por el lema, el subgrupo de Fitting $F(G)$ satisface
	$C_G(F(G))\subseteq F(G)$. 
	
	Demostremos que $F(G)$ es cíclico. En efecto, como $F(G)$ es nilpotente,
	$F(G)$ es producto directo de sus subgrupos de Sylow. Como todo subgrupo de
	Sylow de $F(G)$ es un $p$-subgrupo de $G$, todo Sylow de $F(G)$ es cíclico
	(por estar contenido en algún subgrupo de Sylow de $G$). 

	Como $F(G)$ es cíclico, $F(G)$ es en particular abeliano y luego
	$F(G)\subseteq C_G(F(G))$. Si $G$ actúa en $F(G)$ por conjugación, se tiene
	un morfismo $\gamma\colon G\to\Aut(F(G))$ tal que
	$\ker\gamma=C_G(F(G))=F(G)$ (pues $\gamma_g(x)=gxg^{-1}$). En particular,
	$G/F(G)$ es abeliano por ser isomorfo a un subgrupo del grup abeliano
	$\Aut(F(G))$. Como además los subgrupos de Sylow de $G/F(G)$ son cíclicos (pues
	son cocientes de los subgrupos de Sylow de $G$), $G/F(G)$ es cíclico.
\end{proof}

\section*{D}

\section*{Exercises}

\section*{Open problems}

\section*{Notes}


