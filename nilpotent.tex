\chapter{Nilpotent groups}

\section*{A}

If $G$ is a group and $x,y,z\in G$, the conjugation (as a left action) will be denoted
by $\prescript{x}{}y=xyx^{-1}$. The commutator between $x$ and $y$ is then 
\[
[x,y]=xyx^{-1}y^{-1}=(\prescript{x}{}y)y^{-1}.
\]
We also write 
$[x,y,z]=[x,[y,z]]$. 
If $X$, $Y$ and $Z$ are subgroups of $G$, we write 
\[
[X,Y]=\langle [x,y]:x\in X,\,y\in Y\rangle
\]
and $[X,Y,Z]=\left[ X,[Y,Z] \right]$. Note that $[X,Y]=[Y,X]$. 

\begin{exercise}[Hall--Witt]
	\label{xca:HallWitt}
	\index{Hall--Witt!identity}
	\index{Hall, P.}
	\index{Witt, E.}
	Let $G$ be a group and $x,y,z\in G$. Then 
	\begin{equation}
		\label{eq:HallWitt}
	\left(\prescript{y}{}[x,y^{-1},z]\right)\left(\prescript{z}{}[y,z^{-1},x]\right)\left(\prescript{x}{}[z,x^{-1},y]\right)=1.
	\end{equation}
\end{exercise}

\index{Jacobi, G.}
\index{Jacobi!identity}
Note that if $G$ is such that $[G,G]$ is central, then Hall--Witt's identity 
turns out to be Jacobi's identity. 

%\begin{lemma}[three subgroups lemma]
%	\label{lemma:3subgrupos}
%	\index{Lemma!three subgroups}
%	Let $X$, $Y$ and $Z$ be subgroups of $G$ such that $[X,Y,Z]=[Y,Z,X]=\{1\}$.
%	Then $[Z,X,Y]=\{1\}$.
%\end{lemma}
%
%\begin{proof}
%	Since $[x,y]\in C_G(z)$ implies that 
%	$[X,Y]\subseteq C_G(Z)$, it
%	is enough to prove that $[z,x^{-1},y]=1$ for all $x\in X$, $y\in Y$ and $z\in Z$, 
%	Since $[y^{-1},z]\in [Y,Z]$, it follows that 
%	$[x,y^{-1},z]\in [X,Y,Z]=\{1\}$. Thus $\prescript{y}{}[x,y^{-1},z]=1$.
%	Similarly, $\prescript{z}{}[y,z^{-1},x]=1$. Hence the 
%	Hall--Witt identity yields 
%	$[z,x^{-1},y]=1$.
%\end{proof}

\begin{lemma}[three subgroups lemma]
	\label{lemma:3subgrupos_general}
	Let $N$ be a normal subgroup of $G$ and let $X$, $Y$ and $Z$
	be subgroups of $G$. If $[X,Y,Z]\subseteq N$ and $[Y,Z,X]\subseteq N$, then 
	$[Z,X,Y]\subseteq N$.
\end{lemma}

\begin{proof}
	We first prove the lemma in the case where $N=\{1\}$. 	Since $[x,y]\in C_G(z)$ implies that 
	$[X,Y]\subseteq C_G(Z)$, it
	is enough to prove that $[z,x^{-1},y]=1$ for all $x\in X$, $y\in Y$ and $z\in Z$, 
	Since $[y^{-1},z]\in [Y,Z]$, it follows that 
	$[x,y^{-1},z]\in [X,Y,Z]=\{1\}$. Thus $\prescript{y}{}[x,y^{-1},z]=1$.
	Similarly, $\prescript{z}{}[y,z^{-1},x]=1$. Hence the 
	Hall--Witt identity yields 
	$[z,x^{-1},y]=1$.

	We now demonstrate the general case. Let $N$ be a normal subgroup of $G$ 
	and $\pi\colon G\to G/N$ be the canonical map. Since $[X,Y,Z]\subseteq N$, 
	\begin{align*}
		\{1\}&=\pi([X,Y,Z])=\pi([ X,[Y,Z]])\\
		&=[\pi(X),\pi([Y,Z])]=[ \pi(X),[\pi(Y),\pi(Z)]]=[\pi(X),\pi(Y),\pi(Z)]. 
	\end{align*}
	Similarly one proves that $[\pi(Y),\pi(Z),\pi(X)]=\{1\}$. By the previous paragraph,  
	$[\pi(Z),\pi(X),\pi(Y)]=\{1\}$, so $[Z,X,Y]\subseteq N$.
\end{proof}

\index{Lower central series}
The \textbf{lower central series} of a group $G$ is 
the sequence $\gamma_k(G)$, $k\in\N$, 
defined recursively as 
\[
		\gamma_1(G)=G,\quad
		\gamma_{i+1}(G)=[G,\gamma_i(G)]\quad i\geq 1.
\]
\index{Group!nilpotent}
\index{Nilpotency index}
A group $G$ is said to be \textbf{nilpotent} if there exists positive integer $c$ such that 
$\gamma_{c+1}(G)=\{1\}$. The smallest $c$ such that $\gamma_{c+1}(G)=\{1\}$ is
the \textbf{nilpotency index} (or \textbf{nilpotency class}) of $G$. 

% \begin{exercise}
% \label{xca:nilpotente=>resoluble}
% 	Demuestre que todo grupo nilpotente es resoluble.
% \end{exercise}

\begin{example}
	A group is nilpotent of class one if and only if it is abelian.  
\end{example}

% \begin{example}
% 	$\Sym_3$ es resoluble pues $\Sym_3\supseteq \Alt_3\supseteq\{1\}$ es una serie
% 	de composición con factores abelianos pero $\Sym_3$ no es nilpotente pues
% 	\[
% 		\gamma_1(\Sym_3)=\Alt_3,\quad
% 		\gamma_2(\Sym_3)=[\Alt_3,\Sym_3]=\Alt_3.
% 	\]
% 	Luego $\gamma_i(\Sym_3)\ne1$ para todo $i\geq1$. 
% \end{example}

\begin{example}
    The group $G=\Alt_4$ is not nilpotent, as 
	\[
		\gamma_1(G)=G,\quad
		\gamma_j(G)=\{\id,(12)(34),(13)(24),(14)(23)\}\simeq C_2\times C_2
	\]
	for all $j\geq2$. 
\end{example}

\begin{exercise}
	\label{xca:gamma}
	Let $G$ be group. Prove the following statements:
	\begin{enumerate}
		\item Each $\gamma_i(G)$ is a characteristic group of $G$.
		\item $\gamma_i(G)\supseteq\gamma_{i+1}(G)$ for all $i\geq1$.
		\item If $f\colon G\to H$ is a surjective group homomorphism, then 
			$f(\gamma_i(G))=\gamma_i(H)$ for all $i\geq1$.
	\end{enumerate}
\end{exercise}

% \begin{exercise}
% 	\label{xca:HxK_nilpotente}
% 	Demuestre que si $H$ y $K$ son nilpotentes entonces $H\times K$ es
% 	nilpotente.
% \end{exercise}

\begin{theorem}
	\label{theorem:nilpotent}
	Let $G$ be a nilpotent group. 
	\begin{enumerate}
		\item If $H$ is a subgroup of $G$, then $H$ is nilpotent. 
		\item If $f\colon G\to H$ is a surjective group homomorphism, then $H$ is nilpotent. 
	\end{enumerate}
\end{theorem}

\begin{proof}
	For the first statement note that $\gamma_i(H)\subseteq\gamma_i(G)$ for all 
	$i\geq1$. Let us prove the second claim, if there exists $c$ such that $\gamma_{c+1}(G)=\{1\}$, 
	then 
	\[
	\gamma_{c+1}(H)=f(\gamma_{c+1}(G))=f(\{1\})=\{1\}.\qedhere
	\]
\end{proof}

\begin{example}
    The group $\SL_2(3)$ is not nilpotent, as $\Alt_4$ 
	is a quotient of $\SL_2(3)$. 
\end{example}

There exist a non-nilpotent group $G$ with a normal subgroup $K$ 
such that $K$ and $G/K$ are both nilpotent. For example, take $G=\Sym_3$ and $K=\Alt_3$. 

\begin{exercise}
	Let $p$ be a prime number. Prove that finite $p$-groups are nilpotent. 
\end{exercise}

\begin{theorem}
	\label{theorem:gamma}
	Let $G$ be a group. Then $[\gamma_i(G),\gamma_j(G)]\subseteq
	\gamma_{i+j}(G)$ for all $i,j\geq1$.	
\end{theorem}

\begin{proof}
	We proceed by induction on $j$. The case where $j=1$ is trivial, as by definition one has 
	$[G,\gamma_j(G)]=\gamma_{j+1}(G)$. Assume now that the result holds for 
	some $j\geq1$ and all $i\geq1$. 
	We first note that 	
	\begin{equation*}
		[G,\gamma_i(G),\gamma_j(G)]=[\gamma_i(G),G,\gamma_j(G)]=[\gamma_{i+1}(G),\gamma_j(G)]\subseteq \gamma_{i+j+1}(G)
	\end{equation*}
	by the inductive hypothesis. Moreover, again using the inductive hypothesis,  
	\begin{equation*}
	[\gamma_i(G),\gamma_j(G),G]\subseteq [\gamma_{i+j}(G),G]=\gamma_{i+j+1}(G).
	\end{equation*}
	Lemma~\ref{lemma:3subgrupos_general} implies that 
	$[\gamma_j(G),G,\gamma_i(G)]\subseteq \gamma_{i+j+1}(G)$. Thus  
	\[
	[\gamma_i(G),\gamma_{j+1}(G)]=[\gamma_{j+1}(G),\gamma_i(G)]=[\gamma_j(G),G,\gamma_i(G)]\subseteq \gamma_{i+j+1}(G).\qedhere
	\]
\end{proof}

Certainly we can consider other type of arbitrary commutators, say for example 
$[[G,G],G]$ and $[G,G,G]=[G,[G,G]]$. This naturally suggest the notion of the
weight of a commutator. For example, $[[G,G],G]$ and $[G,G,G]=[G,[G,G]]$ are
both commutators of weight three. 

\begin{corollary}
	Every commutator of weight $n$ is contained in 
	$\gamma_n(G)$.
\end{corollary}

\begin{proof}
	We proceed by induction on $n$. The case $n=1$ is trivial, so assume that
	the result holds for some $n\geq1$.  Let  
	$[A,B]$ be a commutator, where $A$ is a commutator of weight $k$,
	$B$ is a commutator of weight $l$ and $n+1=k+l$. Since $k<n$ and $l<n$, the inductive 
	hypothesis implies that $A\subseteq \gamma_k(G)$ y $B\subseteq
	\gamma_l(G)$. Thus 
	\[
	[A,B]\subseteq [\gamma_k(G),\gamma_l(G)]\subseteq
	\gamma_{k+l}(G)
	\]
	by the previous theorem. 
\end{proof}

\index{Normalizer condition}
Nilpotent groups satisfy the normalizer condition. A group 
$G$ satisfies the \textbf{normalizer condition} if each proper subset is smaller 
than its normalizer. 

\begin{lemma}[normalizer condition]
	\label{lem:normalizer}
	Let $G$ be a nilpotent group. If $H$ is a proper subgroup of $G$, then 
	$H\subsetneq N_G(H)$.
\end{lemma}

\begin{proof}
	Since $G$ is nilpotent, there a positive integer $c$ such that 
	\[
	G=\gamma_1(G)\supseteq\cdots\supseteq\gamma_{c+1}(G)=\{1\}.
	\]
	Since 
	$\{1\}=\gamma_{c+1}(G)\subseteq H$ and $\gamma_1(G)\not\subseteq H$, 
	let $k$ be the smallest positive integer such that $\gamma_k(G)\subseteq H$. Since 
	\[
		[\gamma_{k-1}(G),H]\subseteq [\gamma_{k-1}(G),G]=\gamma_k(G)\subseteq H,
	\]
	it follows that 
	$xHx^{-1}\subseteq H$ for all $x\in\gamma_{k-1}(G)$,
	so  $\gamma_{k-1}(G)\subseteq N_G(H)$. If $N_G(H)=H$, then 
	$\gamma_{k-1}(G)\subseteq H$, a contradiction to the minimality of $k$. 
\end{proof}

%\begin{example}
%	Un grupo $G$ es nilpotente de clase dos si y sólo $\gamma_2(G)

%\end{example}

For a group $G$ we define the sequence $\zeta_0(G),\zeta_1(G),\dots$
recursively as 
\[
	\zeta_0(G)=\{1\},\quad
	\zeta_{i+1}(G)=\{g\in G:[g,x]\in\zeta_{i}(G)\text{ para todo $x\in G$}\},\quad i\geq 0.
\]
In particular, $\zeta_1(G)=Z(G)$.

\begin{lemma}
	\label{lem:central_ascendente}
	Let $G$ be a group. Each $\zeta_i(G)$ is a normal
	subgroup of $G$. 
\end{lemma}

\begin{proof}
	We proceed by induction on $i$. The case $i=0$ is trivial, as 
	$\zeta_0(G)=\{1\}$.  Assume that the result holds for some $i\geq0$.
	We claim that $\zeta_{i+1}(G)$ is a subgroup of $G$. 
	Let $g,h\in \zeta_{i+1}(G)$ and $x\in G$. The inductive hypothesis implies that 
	\begin{align*}
	&[g^{-1},x]=(xg^{-1})[g,x^{-1}](xg^{-1})^{-1}\in (xg^{-1})\zeta_i(G)(xg^{-1})^{-1}=\zeta_i(G),\\
	&[gh,x]=[g,hxh^{-1}][h,x]\in \zeta_{i}(G).
	\end{align*}
	Since $1\in\zeta_{i+1}(G)$, the sets $\zeta_i(G)$ are subgroups of $G$. 
	
	To prove 
	that each subgroup is normal we also proceed by induction on $i$. 
	If $g\in\zeta_{i+1}(G)$ and $x\in G$, then $xgx^{-1}\in\zeta_{i+1}(G)$. Indeed,  
	\[
	[xgx^{-1},y]=x[g,x^{-1}yx]x^{-1}\in\zeta_{i}(G)
	\]
	for all $y\in G$.
\end{proof}


\index{Ascending!central series}
For a group $G$ the \textbf{ascending central series} of $G$ is the sequence 
\[
	\{1\}=\zeta_0(G)\subseteq\zeta_1(G)\subseteq\zeta_2(G)\subseteq\cdots
\]

%\index{Grupo!perfecto}
A group $G$ is said to be \textbf{perfect} if $[G,G]=G$. Note that
$G$ is perfect if and only if $G/[G,G]$ is trivial. The alternating 
simple group $\Alt_5$ is the smallest non-trivial perfect group. 

\begin{example}
The groups $\SL_2(2)$ and $\SL_2(3)$ are not perfect. 	
\end{example}

Let $p$ be a prime number and $q=p^m$ for some $m\in\N$.  
The groups $\SL_n(q)$ are perfect except the cases $\SL_2(2)$ and $\SL_2(3)$. As an example,
let us prove that $\SL_2(p)$ is perfect if $p\geq5$ is a prime number. We first 
claim that $\SL_2(q)$ is generated by matrices
$X_{ij}(\lambda)=I+\lambda*E_{ij}$, where $I$ denotes the identity, 
$E_{ij}$ is the matriz with a one at position $(i,j)$ and zero in all other entries, 
$i,j\in\{1,2\}$ and $\lambda\in\F_q$.  

\begin{exercise}
Let $q\geq5$. 
Prove that $\SL_n(q)$ is perfect.  	
\end{exercise}


\begin{exercise}
Let $G$ be a perfect group and $N$ be a normal subgroup of $G$. Then $G/N$ is perfect.  	
\end{exercise}


%A $G$ se dice \textbf{perfecto} si $[G,G]=G$.

\begin{theorem}[Gr\"un]
	\label{theorem:Grun}
	\index{Teorema!de Gr\"un}
	Si $G$ es un grupo perfecto, entonces $Z(G/Z(G))=\{1\}$. 
\end{theorem}

\begin{proof}
	The three subgroups lemma with $X=Y=G$ and $Z=\zeta_2(G)$ yields  
	\[
	\{1\}=[\zeta_2(G),G,G]=[\zeta_2(G),[G,G]]=[\zeta_2(G),G].
	\]
	Thus 	
	$\zeta_2(G)\subseteq Z(G)$ and hence 
	$\zeta_2(G)=Z(G/Z(G))=\{1\}$. 
\end{proof}

	\index{Normalizador}
    Sea $G$ un grupo. Se dice que un subgrupo $K$ de $G$ \textbf{normaliza} a
	$H$ si $K\subseteq N_G(H)$.
	\index{Centralizador}
	Se dice que un subgrupo $K$ de $G$ \textbf{centraliza} a
	$H$ si $K\subseteq C_G(H)$, es decir si y sólo si $[H,K]=\{1\}$.

\begin{exercise}
	Sean $K$ y $H$ subgrupos de $G$ con $K\subseteq H$ y $K$ normal en $G$.
	Demuestre que $[H,G]\subseteq K$ si y sólo si $H/K\subseteq Z(G/K)$. 
\end{exercise}
\begin{lemma}
	\label{lemma:gamma_zeta}
	Sea $G$ un grupo. Existe $c$ tal que $\zeta_c(G)=G$ si y sólo si 
	$\gamma_{c+1}(G)=\{1\}$. Más aún, en ese caso 
	\[
	\gamma_{i+1}(G)\subseteq\zeta_{c-i}(G)
	\]
	para todo $i\in\{0,1,\dots,c\}$. 
\end{lemma}

\begin{proof}
	Supongamos primero que $\zeta_c(G)=G$. Por inducción vamos a demostrar
	que $\gamma_{i+1}(G)\subseteq\zeta_{c-i}(G)$.  Como el caso $i=0$ es
	trivial, supongamos que el resultado es válido para un cierto $i\geq0$. Si
	$g\in\gamma_{i+2}(G)=[\gamma_{i+1}(G),G]$, podemos escribir 
	\[
	g=\prod_{k=1}^N [g_k,x_k],
	\]
	donde $g_1,\dots,g_N\in\gamma_{i+1}(G)$ y $x_1,\dots,x_N\in G$. Por
	hipótesis inductiva 
	\[
	g_j\in\gamma_j(G)\subseteq\zeta_{c-j}(G)
	\]
	y entonces $[g_j,x_j]\in\zeta_{c-i-1}(G)$ para todo $j$. Luego
	$g\in\zeta_{c-(i+1)}(G)$. La implicación que queremos queda demostrada
	al tomar $i=c$. 

	Supongamos ahora que $\gamma_{c+1}(G)=1$. Demostremos por inducción en $c-i$ que
	$\gamma_{c+1-i}(G)\subseteq\zeta_{c-i}(G)$. El caso $c-i=0$ es trivial. Si el
	resultado es válido para algún $c-i\geq0$, sea
	$g\in\gamma_{c+2-i}(G)=[\gamma_{c+1-i}(G),G]$. Escribimos
	\[
		g=\prod_{k=1}^N[g_k,x_k]
	\]
	con $g_1,\dots,g_N\in\gamma_{c+1-i}(G)\subseteq\zeta_{i}(G)$ por
	hipótesis inductiva. Luego $g\in \zeta_{c-(i+1)}(G)$ pues cada
	$[g_j,x_j]\in\zeta_{i-1}(G)$. Al tomar $i=0$ se obtiene la implicación
	buscada.
\end{proof}

\begin{example}
	Sea $G=\Sym_3$. Entonces $\zeta_j(G)=\{1\}$ para todo $j\geq 0$: 
\end{example}

\begin{definition}
	\index{Serie!central}
	Sea $G$ un grupo. Una \textbf{serie central} para $G$ es una sucesión 
	\[
		G=G_0\supseteq G_1\supseteq\cdots\supseteq G_n=\{1\}
	\]
	de subgrupos normales de $G$ tal que para cada $i\in\{1,\dots,n\}$, 
	$\pi_i(G_{i-1})$ es un subgrupo de $Z(G/G_i)$, donde $\pi_i\colon G\to
	G/G_i$ es el morfismo canónico.
\end{definition}

\begin{lemma}
	\label{lemma:serie_central}
	Sea $G$ un grupo y sea $G=G_0\supseteq G_1\supseteq\cdots\supseteq G_n=\{1\}$
	una serie central para $G$. Entonces $\gamma_{i+1}(G)\subseteq G_i$ para
	todo $i$.
\end{lemma}

\begin{proof}
	Procederemos por inducción en $i$. El caso $i=0$ es trivial.  Supongamos
	que el resultado es válido para algún $i\geq0$. Entonces 
	\[
	\gamma_{i+1}(G)=[G,\gamma_i(G)]\subseteq [G,G_{i-1}]\subseteq G_i
	\]
	pues, como $\pi_i(G_{i-1})\subseteq Z(G/G_{i})$, entonces
	$\pi([G,G_{i-1}])=[\pi(G),\pi(G_{i-1})]=\{1\}$ y luego $[G,G_{i-1}]\subseteq
	G_i$.
\end{proof}

% extender el lema para ver qué pasa con zeta_i

\begin{theorem}
	Un grupo es nilpotente si y sólo si admite una serie central.
\end{theorem}

\begin{proof}
	Si el grupo $G$ es nilpotente, entonces los $\gamma_j(G)$ forman una serie
	central para $G$. Recíprocamente, si $G=G_0\supseteq
	G_1\supseteq\cdots\supseteq G_n=\{1\}$ es una serie central para $G$, entonces,
	por el lema anterior, $G$ es nilpotente pues
	\[
	\gamma_{n+1}(G)\subseteq G_n=\{1\}.\qedhere
	\]
\end{proof}

\begin{exercise}
\label{xca:nilpotente_central}
	Sea $G$ un grupo. Demuestre que si $K$ es un subgrupo de $Z(G)$ tal que
	$G/K$ es nilpotente, entonces $G$ es nilpotente.
\end{exercise}

\begin{theorem}[Hirsch]
	\label{theorem:Z(nilpotent)}
	\index{Teorema!de Hirsch}
	Sea $G$ un grupo nilpotente. Si $H$ es un subgrupo normal no trivial de $G$
	entonces $H\cap Z(G)\ne\{1\}$. En particular, $Z(G)\ne\{1\}$. 
%	Si $G$ es un grupo nilpotente entonces $Z(G)\ne1$. 
\end{theorem}

\begin{proof}
	Como $\zeta_0(G)=\{1\}$ y existe $c$ tal que $\zeta_c(G)=G$, existe 
	\[
	m=\min\{k:H\cap\zeta_k(G)\ne\{1\}\}.
	\]
	Como $H$ es normal, 
	\[
	[H\cap\zeta_m(G),G]\subseteq H\cap[\zeta_m(G),G]\subseteq H\cap\zeta_{m-1}(G)=\{1\}.
	\]
	Luego $\{1\}\ne H\cap\zeta_m(G)\subseteq H\cap Z(G)$. Si $H=G$ entonces $Z(G)\ne\{1\}$. 
%	Sea $c$ tal que $\gamma_{c+1}(G)=1$ y $\gamma_{c}(G)\ne1$. 
%	Por el lema~\ref{lemma:gamma_zeta}, 
%	\[
%	1\ne\gamma_{c}(G)\subseteq \zeta_{1}(G)=Z(G).
%	\]
\end{proof}

\begin{exercise}
\label{xca:nilpotente_minimalnormal}
	Sea $G$ un grupo nilpotente y sea $M$ un subgrupo minimal-normal de $G$.
	Demuestre que $M\subseteq Z(G)$.
\end{exercise}

% \begin{svgraybox}
% 	Como $M\cap Z(G)$ es normal en $G$, la minimalidad de $M$ implica que hay
% 	dos posibilidades: $M\cap Z(G)$ es trivial o bien $M=M\cap Z(G)\subseteq Z(G)$.
% 	Por el teorema~\ref{theorem:Z(nilpotent)}, $M\cap Z(G)\ne 1$.
% \end{svgraybox}

\begin{corollary}
	Sea $G$ un grupo nilpotente no abeliano y sea $A$ un subgrupo
	maximal-normal y abeliano de $G$. Entonces $A=C_G(A)$.
\end{corollary}

\begin{proof}
	Como $A$ es abeliano, $A\subseteq C_G(A)$. Supongamos que $A\ne C_G(A)$.
	El centralizador $C_G(A)$ es normal en $G$ pues, como $A$ es normal en $G$, 
	\[
		gC_G(A)g^{-1}=C_G(gAg^{-1})=C_G(A).
	\]
	para todo $g\in G$.  Sea $\pi\colon G\to G/A$ el morfismo canónico.
	Entonces $\pi(C_G(A))$ es un subgrupo normal no trivial de $\pi(G)$. Como
	$G$ es nilpotente, $\pi(G)$ es nilpotente y, por el
	teorema de Hirsch, $\pi(C_G(A))\cap Z(\pi(G))\ne\{1\}$. Sea
	$x\in C_G(A)\setminus A$ tal que $\pi(x)$ es central en $\pi(G)$.  El grupo
	$\langle A,x\rangle$ es abeliano pues $x\in C_G(A)$. Además $\langle
	A,x\rangle$ es normal en $G$ pues $A$ es normal en $G$ y  si $g\in G$
	entonces $gxg^{-1}x^{-1}\in A$ porque $\pi(x)$ es central y luego
	$gxg^{-1}\in \langle A,x\rangle$. Luego $A\subsetneq \langle
	A,x\rangle\subsetneq G$, una contradicción.
\end{proof}

\begin{theorem}
	Sea $G$ un grupo nilpotente. Valen las siguientes afirmaciones:
	\begin{enumerate}
		\item Todo subgrupo minimal-normal tiene orden primo y es central.
		\item Todo subgrupo maximal es normal de índice primo y contiene a
			$[G,G]$. 
	\end{enumerate}
\end{theorem}

\begin{proof}
	Demostremos la primera afirmación. Sea $N$ un subgrupo minimal-normal. Como
	$N\cap Z(G)\ne\{1\}$ por el teorema el Hirsch, $N\cap Z(G)$ es
	un subgrupo normal de $G$ contenido en $N$. Luego $N=N\cap Z(G)\subseteq
	Z(G)$ por la minimalidad de $N$. En particular, $N$ es abeliano. Además,
	como todo subgrupo de $N$ es normal en $G$, $N$ es simple y luego $N\simeq
	C_p$ para algún primo $p$.

	Demostremos ahora la segunda afirmación. Si $M$ es un subgrupo maximal, $M$
	es normal en $G$ gracias a la condidición normalizadora. La
	maximalidad de $M$ implica que $G/M$ no contiene subgrupos propios no
	triviales. Luego $G/M\simeq C_p$ para algún primo $p$. Como en particular
	$G/M$ es abeliano, $[G,G]\subseteq M$. 
\end{proof}

Es importante remarcar que el teorema anterior no garantiza, por ejemplo, la existencia de subgrupos maximales.
Recordemos que $\Q$ es un grupo abeliano (por lo tanto, nilpotente) que no tiene subgrupos maximales. 

\begin{proposition}
	\label{proposition:g^n}
	Sea $G$ un grupo nilpotente y sea $H$ un subgrupo de $G$ de índice $n$. Si
	$g\in G$ entonces $g^n\in H$.
\end{proposition}

%%% TODO: hacer inducción en el índice para escribirlo mejor

\begin{proof}
	El resultado es obvio en el caso en que $H$ sea un subgrupo normal.  Sea
	$H_0=H$ y $H_{i+1}=N_G(H_i)$ para $i\geq0$. Por definición, $H_{i}$ es
	normal en $H_{i+1}$ y además, como $G$ es nilpotente, si $H_i\ne G$
	entonces $H_i\subsetneq H_{i+1}$ por la condición normalizadora. 
	Como $(G:H)$ es finito, existe $k$ tal que $H_k=G$. Veamos que 
	\[
		g^{(G:H)}=g^{(H_k:H_{k-1})(H_{k-1}:H_{k-2})\cdots (H_1:H_0)}\in H.
	\]
	Observemos que $g^{(H_k:H_{k-1})}\in H_{k-1}$ pues $H_{k-1}$ es normal en $H_k=G$, y que, como 
	$g^{(H_k:H_{k-1})}\in H_k$, entonces 
	\[
	g^{(H_k:H_{k-2})}=g^{(H_k:H_{k-1})(H_{k-1}:H_{k-2})}=\left(g^{(H_k:H_{k-1})}\right)^{(H_{k-1}:H_{k-2})}\in H_{k-2}
	\]
	pues $H_{k-2}$ es normal en $H_{k-1}$. Al repetir este argumento, $g^{(G:H)}\in H$. 
\end{proof}

\begin{example}
	La proposición anterior no vale si el grupo $G$ no es
	nilpotente. Sea $G=\Sym_3$ y sea $H=\{\id,(12)\}$ de índice tres.  Si
	$g=(13)$ entonces $g^{3}=(13)\not\in H$.
\end{example}

El lema que daremos a continuación es una herramienta útil para hacer demostraciones por inducción
en grupos nilpotentes. 

\begin{lemma}
	\label{lemma:a[GG]}
	Sea $G$ un grupo nilpotente de clase $c\geq2$. Si $x\in G$ entonces el subgrupo 
	$\langle x,[G,G]\rangle$ es nilpotente de clase $<c$.
\end{lemma}

\begin{proof}
	Sea $H=\langle x,[G,G]\rangle$.  Si $x\in [G,G]$, el resultado es cierto.
	Supongamos entonces que $x\not\in [G,G]$. Observemos que 
	\[
		H=\{x^nc:n\in\Z,c\in [G,G]\},
	\]
	pues $[G,G]$ es normal en $G$. Para demostrar el lema  
	alcanza entonces con probar que $[H,H]\subseteq\gamma_3(G)$. Sean $h=x^nc,k=x^md\in H$
	con $c,d\in [G,G]$. 
	Como 
	\[
	[h,x^m]=[x^n,[c,x^m]][c,x^m]\in\gamma_4(G)\gamma_3(G)\subseteq\gamma_3(G),
	\]
	entonces 
	\begin{align*}
		[h,k]&=[h,x^m][x^m,[h,d]][h,d]\\
			&=[x^n,[c,x^m]][c,x^m][x^m,[h,d]][h,d]\in\gamma_3(G).\qedhere
	\end{align*}
%	Como
%	$[G,G]=\gamma_2(G)\subseteq\zeta_{c-1}(G)$, al usar la definición de $H$ se
%	obtiene que $[G,G]\subseteq \zeta_{c-1}(G)\cap H\subseteq\zeta_{c-1}(H)$.
%	Luego $H/\zeta_{c-1}(H)$ es cíclico generado por $a\zeta_{c-1}(H)$. Si
%	$\zeta_{c-1}(H)=H$, no hay nada para demostrar. En caso contrario,
%	$Z(H)\subseteq\zeta_{c-1}(H)$ implicaría que $H$ es abeliano y luego $H$ es
%	nilpotente de clase uno.
\end{proof}

Veamos un ejemplo, está hecho con la computadora. 

\begin{example}
	Sea $G=\D_{8}=\langle r,s:r^{8}=s^2=1,srs=r^{-1}\rangle$ el grupo diedral
	de orden 16. El grupo $G$ es nilpotente de clase tres
	y $[G,G]=\{1,r^2,r^4,r^6\}\simeq C_4$. El subgrupo $\langle
	s,[G,G]\rangle\simeq\D_4$ es nilpotente de clase dos.
\end{example}

Ahora una aplicación del lema. 

\begin{theorem}
	\label{theorem:T(nilpotent)}
	Si $G$ es un grupo nilpotente, entonces
	\[
	T(G)=\{g\in G:g^n=1\text{ para algún $n\in\N$}\}
	\]
	es un subgrupo de $G$. 
\end{theorem}

\begin{proof}
	Sean $a,b\in T(G)$ y sean
	\[
		A=\langle a,[G,G]\rangle,\quad
		B=\langle b,[G,G]\rangle.
	\]
	Como $A$ y $B$ son nilpotentes por el lema anterior, por hipótesis
	inductiva, $T(A)$ es un subgrupo de $A$ y $T(B)$ es un subgrupo de $B$.
	Como $T(A)$ es característico en $A$ y $A$ es normal en $G$, $T(A)$ es
	normal en $G$. Similarmente se demuestra que $T(B)$ es normal en $B$.  
	Veamos ahora que todo elemento de $T(A)T(B)$ tiene orden finito: si
	$x\in T(A)T(B)$, digamos $x=a_1b_1$ con
	$a_1$ de orden $m$, entonces $x$ tiene orden finito pues 
	\begin{align*}
	x^m=(a_1b_1)^m&=%(a_1b_1a_1^{-1})(a_1^2b_1a_1^{-2})\cdots (a_1^{m-1} b_1 a_1^{-m+1}b_1)\\
	(a_1b_1a_1^{-1})(a_1^2b_1a_1^{-2})\cdots (a^{m-1} b_1 a^{-m+1})b_1\in T(B).
	\end{align*}
	
	Para entender mejor el truco hagamos un caso concreto, digamos $m=3$. La fórmula en este caso queda así:
	\begin{align*}
	(a_1b_1)^3&=(a_1b_1)(a_1b_1)(a_1b_1)\\
	&=(a_1b_1a_1^{-1})(a_1^2b_1a_1^{-2})a_1^3b_1
	=(a_1b_1a_1^{-1})(a_1^2b_1a_1^{-2})b_1,
	\end{align*}
	pues $a_1^3=1$.
	
	El truco nos permite entonces demostrar que $ab$ y $a^{-1}$ tienen ambos 
	orden finito. Luego $T(G)$ es un
	subgrupo de $G$.
\end{proof}

Otra aplicación.

\begin{theorem}
	\label{theorem:a=b}
	Sea $G$ un grupo nilpotente y sin torsión y sean $a,b\in G$. Si existe
	$n\ne 0$ tal que $a^n=b^n$ entonces $a=b$.
\end{theorem}

\begin{proof}
	Procederemos por inducción en el orden de nilpotencia $c$ de $G$. El
	resultado es trivialmente cierto si $G$ es abeliano. Supongamos entonces
	que $G$ es nilpotente de índice $c\geq1$. Como $\langle a,[G,G]\rangle$ es un
	subgrupo de $G$ nilpotente de índice $<c$, y $bab^{-1}=[b,a]a\in \langle
	a,[G,G]\rangle$, por hipótesis inductiva, $ba=ab$ pues 
	\[
		a^n=(bab^{-1})^n=b^n.
	\]
	Luego $(ab^{-1})^n=a^nb^{-n}=1$ y por lo tanto, como $G$ no tiene torsión,
	se concluye que $a=b$.
\end{proof}

\begin{corollary}
	Sea $G$ un grupo nilpotente sin torsión. Sean $x,y\in G$ tales que
	$x^ny^m=y^mx^n$ para algún $n,m\ne 0$, entonces $xy=yx$.
\end{corollary}

\begin{proof}
	Sean $a=x$ y $b=y^nxy^{-n}$. Como $a^m=b^m$, el teorema anterior
	implica que $a=b$ y luego $xy^n=y^nx$. Al usar nuevamente ese 
	teorema, esta vez con con $a=y$ y $b=xyx^{-1}$, se
	concluye que $xy=yx$. 
\end{proof}

\begin{lemma}
	\label{lemma:fg}
	Sea $G$ un grupo finitamente generado y sea $H$ un subgrupo de índice
	finito. Entonces $H$ es finitamente generado.
\end{lemma}

\begin{proof}
	Supongamos que $G$ está generado por $\{g_1,\dots,g_m\}$. Podemos suponer, sin pérdida de generalidad, que
	para cada $i$ existe $k$ tal que $g_i^{-1}=g_k$. 
	
	Sea $\{1=t_1,\dots,t_n\}$ un transversal de $H$ en $G$, es decir un 
	conjunto de representantes de $G/H$. Para $i\in\{1,\dots,n\}$,
	$j\in\{1,\dots,m\}$, escribimos
	\[
		t_ig_j=h(i,j)t_{k(i,j)}.
	\]
	Vamos a demostrar que $H$ está generado por los $h(i,j)$. Sea $x\in H$.
	Escribamos 
	\begin{align*}
	x &=g_{i_1}\cdots g_{i_s}\\
	&= (t_1g_{i_1})g_{i_2}\cdots g_{i_s}\\
	&= h(1,i_1)t_{k_1}g_{i_2}\cdots g_{i_s}\\
	&= h(1,i_1)h(k_1,i_2)t_{k_2}g_{i_3}\cdots g_{i_s}\\
	&= h(1,i_1)h(k_1,i_2)\cdots h(k_{s-1},i_s)t_{k_s},
	\end{align*}
	donde $k_1,\dots,k_{s-1}\in\{1,\dots,n\}$. Como $t_{k_s}\in H$ pues $x\in H$, entonces
	$t_{k_s}=1\in H$ y luego $x$ está generado por los $h(i,j)$.
\end{proof}

\begin{theorem}
	\label{theorem:T(G)finito}
	Sea $G$ un grupo finitamente generado, de torsión y nilpotente. Entonces
	$G$ es finito. 	
\end{theorem}

\begin{proof}
	Procederemos por inducción en la clase de nilpotencia $c$. El caso $c=1$ es
	verdadero pues $G$ es abeliano. Supongamos entonces que el resultado es
	válido para $c\geq1$.  Como $[G,G]$ y $G/[G,G]$ son nilpotentes de clase
	$<c$, finitamente generados por el lema anterior y de torsión, por
	hipótesis inductiva se tiene que $[G,G]$ y $G/[G,G]$ son finitos. Luego $G$
	es también finito, de hecho puede probarse que $G$ es de orden $|[G,G]|(G:[G,G])$.
\end{proof}

%\begin{lemma}
%	\label{lemma:kgenerators}
%	Sea $G$ un grupo y sea $G=G_0\supseteq G_1\supseteq\cdots\supseteq G_k=1$
%	una sucesión de subgrupos de $G$ tal que cada $G_{i+1}$ es normal en $G_i$
%	y cada $G_{i}/G_{i+1}$ es cíclico. Todo subgrupo de $G$ es finitamente
%	generado por $k$ elementos.
%\end{lemma}
%
%\begin{proof}
%	Procedemos por inducción en $k$. Supongamos primero que $k=1$. Entonces
%	$G\simeq G_0/G_1$ es cíclico y luego todo subgrupo de $G$ está generado por
%	un elemento. Supongamos ahora que el resultado es válido para $k\geq1$. Sea
%	$H$ un subgrupo de $G$, sea $N=G_{1}$ y sea $\pi\colon G\to G/N$ el
%	morfismo canónico. El grupo 
%	\[
%		\pi(H)\simeq H/H\cap N
%	\]
%	es cíclico pues un un subgrupo del grupo cíclico $G_k/G_{k-1}=G/N$. Como
%	existe $h\in H$ tal que $\pi(H)$ está generado por $\pi(h)$, se concluye que 
%	$H=\langle \pi(h),H\cap N\rangle$. Por hipótesis
%	inductiva, $H\cap N$ está generado por $k-1$ elementos y luego $H$ está
%	generado por $k$ elementos.
%\end{proof}
%
%\begin{theorem}
%	Sea $G$ un grupo nilpotente y finitamente generado. Entonces $T(G)$ es
%	finito.
%\end{theorem}
%
%%%% aca hay que hacer producto tensorial para construir una serie con factores cíclicos
%%%% ver libro de Khukhro
%%%% Nilpotent Groups and Their Automorphisms
%\begin{proof}
%	Sabemos por el teorema~\ref{theorem:} que existe una sucesión
%	$G=G_0\supseteq G_1\supseteq\cdots\supseteq G_k=G$ de subgrupos normales de
%	$G$ con factores cíclicos. 
%\end{proof}




%\section{Grupos finitos nilpotentes}

El siguiente lema también resultará muy útil, especialmente en caso de trabajar con 
grupos finitos nilpotentes. 

\begin{lemma}
	\label{lemma:normalizador}
	Sean $G$ un grupo finito, $p$ un primo que divide a $|G|$ y
	$P\in\Syl_p(G)$. Entonces
	\[
	N_G(N_G(P))=N_G(P). 
	\]
\end{lemma}

\begin{proof}
	Sea $H=N_G(P)$. Como $P$ es normal en $H$, $P$ es el único $p$-subgrupo de
	Sylow de $H$. Para ver que $N_G(H)=H$ basta demostrar que $N_G(H)\subseteq
	H$. Sea $g\in N_G(H)$. Como 
	\[
	gPg^{-1}\subseteq gHg^{-1}=H,
	\]
	$gPg^{-1}\in\Syl_p(H)$ y $H$ tiene un único $p$-subgrupo de Sylow, 
	$P=gPg^{-1}$.  Luego $g\in N_G(P)=H$. 
\end{proof}

\begin{theorem}
	\label{theorem:nilpotente:eq}
	Sea $G$ un grupo finito. Son equivalentes:
	\begin{enumerate}
		\item $G$ es nilpotente.
		\item Todo subgrupo de Sylow de $G$ es normal.
		\item $G$ es producto directo de sus subgrupos de Sylow.
	\end{enumerate}
\end{theorem}

\begin{proof}
	Veamos que $(1)\implies(2)$. Sea $P\in\Syl_p(G)$. Queremos ver que $P$ es
	normal en $G$, es decir $N_G(P)=G$.  Por el lema anterior, 
	$N_G(N_G(P))=N_G(P)$. La condición normalizadora implica entonces que $N_G(P)=G$.

	Veamos ahora que $(2)\implies(3)$. Sean $p_1,\cdots,p_k$ los factores
	primos de $|G|$ y para cada $i\in\{1,\dots,k\}$ sea $P_i\in\Syl_{p_i}(G)$.
	Por hipótesis, cada $P_j$ es normal en $G$.

	Vamos a demostrar que $P_1\cdots P_j\simeq P_1\times\cdots\times P_j$ para todo $j$.
	El caso $j=1$ es trivial. Supongamos entonces que el resultado vale para
	algún $j\geq 1$. Como 
	\[
	N=P_1\cdots P_j\simeq P_1\times\cdots\times P_j
	\]
	es normal en $G$ y tiene orden coprimo con $|P_{j+1}|$, $N\cap
	P_{j+1}=\{1\}$. Luego
	\[
		NP_{j+1}\simeq N\times P_{j+1}\simeq P_1\times\cdots\times P_j\times P_{j+1}
	\]
	pues $P_{j+1}$ es también normal en $G$. 

	Ahora que sabemos que $P_1\cdots P_k\simeq P_1\times\cdots\times P_k$ es un
	subgrupo de orden $|G|$, se concluye que $G=P_1\times\cdots\times P_k$.

	Para ver que $(3)\implies(1)$ simplemente hace falta observar que todo
	$p$-grupo es nilpotente (proposición~\ref{proposition:pgrupo_nilpotente}) y
	que el producto directo de finitos nilpotentes es nilpotente.
	%(ejercicio~\ref{exercise:HxK_nilpotente}).
\end{proof}

\begin{exercise}
	\label{xca:truco}
	Sea $G$ un grupo finito. Demuestre que si $P\in\Syl_p(G)$ y $M$ es un subgrupo de $G$ tal que
	$N_G(P)\subseteq M$ entonces $M=N_G(M)$. 
\end{exercise}

% \begin{svgraybox}
% 	Sea $x\in N_G(M)$. Como $P\subseteq M$ y $M$ es normal en $N_G(M)$,
% 	$xPx^{-1}\subseteq M$.  Como $P$ y $xPx^{-1}$ son $p$-subgrupos de Sylow de
% 	$M$, existe $m\in M$ tal que 
% 	\[
% 	mPm^{-1}=xPx^{-1}.
% 	\]
% 	Luego $x\in M$ pues
% 	$m^{-1}x\in N_G(P)\subseteq M$. 
% \end{svgraybox}

\begin{exercise}
	\label{xca:normalizadora}
	Sea $G$ un grupo finito. Son equivalentes:
	\begin{enumerate}
		\item $G$ es nilpotente.
		\item Si $H\subsetneq G$ es un subgrupo entonces $H\subsetneq N_G(H)$.
		\item Todo subgrupo maximal de $G$ es normal en $G$.
	\end{enumerate}
\end{exercise}

% \begin{svgraybox}
% 	Para demostrar que $(1)\implies(2)$ simplemente usamos el
% 	lema~\ref{lemma:normalizadora}. Para demostrar que $(2)\implies(3)$ hacemos
% 	lo siguiente: si $M$ es un subgrupo maximal, como $M\subsetneq N_G(M)$ por
% 	hipótesis, $N_G(M)=G$ por maximalidad. Finalmente demostremos que
% 	$(3)\implies(1)$.  Sea $P\in\Syl_p(G)$. Si $P$ no es normal en $G$,
% 	$N_G(P)\ne G$ y entonces existe un subgrupo maximal $M$ tal que
% 	$N_G(P)\subseteq M$. Como $M$ es normal en $G$, el
% 	ejercicio~\ref{exercise:truco} implica que $M=N_G(M)=G$, una contradicción.
% 	Luego $P$ es normal en $G$ y entonces $G$ es nilpotente por el
% 	teorema~\ref{theorem:nilpotente:eq}.
% \end{svgraybox}

% ejercicio: G finito. Es nilpotente si y solo si dos elementos de ordenes coprimos conmnutan
% 5.41 rotman

\begin{theorem}
	Sea $G$ un grupo finito nilpotente. Si $p$ es un primo que divide al orden
	de $G$, existe un subgrupo minimal-normal de orden $p$ y existe un subgrupo
	maximal de índice $p$.
\end{theorem}

\begin{proof}
	Supongamos que $|G|=p^{\alpha}m$, donde $p$ es un primo coprimo con $m$.
	Escribamos $G=P\times H$, donde $P\in\Syl_p(G)$.  Como $Z(P)$ es un
	subgrupo normal no trivial de $P$, todo subgrupo 
	de $Z(P)$ minimal-normal en $G$ tiene orden $p$ (y esos subgrupos 
	existen porque $G$ es un grupo finito). Por otro lado, como $P$ contiene un subgrupo
	de índice $p$, que resulta maximal. Luego $P\times H$ también contiene un
	subgrupo maximal de índice $p$.
\end{proof}

\begin{exercise}
	\label{xca:pgrupos}
	Sea $p$ un primo y sea $G$ un grupo no trivial de orden $p^n$.
	Demuestre las siguientes afirmaciones:
	\begin{enumerate}
		\item $G$ tiene un subgrupo normal de orden $p$.
		\item Para todo $j\in\{0,\dots,n\}$ existe un subgrupo normal de
			$G$ de orden $p^j$. 
	\end{enumerate}
\end{exercise}

% \begin{svgraybox}
% 	\begin{enumerate}
% 		\item Sabemos que $Z(G)\ne1$. Sea $g\in Z(G)$ tal que $g\ne 1$.
% 			Supongamos que el orden de $g$ es $p^k$ para algún $k\geq1$.
% 			Entonces $g^{p^{k-1}}$ tiene orden $p$ y luego genera un subgrupo
% 			central de orden $p$. 
% 		\item Procederemos por inducción en $n$. Si $n=1$ el resultado es
% 			trivial.  Supongamos entonces que el resultado vale para un cierto
% 			$n\geq2$. Por el punto anterior, $G$ posee un subgrupo normal $N$
% 			de orden $p$. Luego $G/N$ tiene orden $p^{n-1}$. Sea $\pi\colon G\to G/N$ el morfismo canónico. 
% 			Por hipótesis
% 			inductiva, para cada $j\in\{0,\dots,n-1\}$. Por el teorema de la
% 			correspondecia, cada subgrupo normal $S_j$ de $G/N$ de orden $p^j$ se
% 			corresponde con un subgrupo $\pi^{-1}(S_j)$ de $G$ de orden $p^{j+1}$ pues, como
% 			$\pi$ es sobreyectiva, se tiene $\pi(\pi^{-1}(S_j))=S_j$, y luego
% 			\[
% 			p^j=|S_j|=|\pi(\pi^{-1}(S_j))|=\frac{|\pi^{-1}(S_j)|}{|\pi^{-1}(S_j)\cap N|}=\frac{|\pi^{-1}(S_j)|}{|N|}=\frac{|\pi^{-1}(S_j)|}{p}.
% 			\]
% 	\end{enumerate}
% \end{svgraybox}

\begin{exercise}
\label{xca:nilpotente_equivalencia}
	Sea $G$ un grupo finito. Demuestre que las siguientes afirmaciones son
	equivalentes: 
	\begin{enumerate}
		\item $G$ es nilpotente.
		\item Cualesquiera dos elementos de órdenes coprimos conmutan. 
		\item Todo cociente no trivial de $G$ tiene centro no trivial.
		\item Si $d$ divide al orden de $G$, existe un subgrupo normal de $G$
			de orden $d$.
	\end{enumerate}
\end{exercise}

% \begin{svgraybox}
% 	Veamos que $(1)\implies(2)$. Sabemos que $G$ es producto directo de sus
% 	subgrupos de Sylow, digamos $G=\prod_{i=1}^k S_i$, donde los $S_i$ son los
% 	distintos subgrupos de Sylow de $G$.  Sean
% 	$x=(x_1,\dots,x_k),y=(y_1,\dots,y_k)\in G$. Como $|x|$ y $|y|$ son
% 	coprimos, para cada $i\in\{1,\dots,k\}$ se tiene $x_i=1$ o $y_i=1$. Luego
% 	\[
% 		[x,y]=([x_1,y_1],[x_2,y_2],\dots,[x_k,y_k])=1. 
% 	\]
% 	Demostremos ahora que $(2)\implies(1)$. Supongamos que
% 	$|G|=p_1^{\alpha_1}\cdots p_k^{\alpha_k}$, donde los $p_j$ son primos
% 	distintos y para cada $j$ sea $P_j\in\Syl_{p_j}(G)$. Como elementos de
% 	órdenes coprimos conmutan, la función $P_1\times\cdots\times P_k\to G$,
% 	$(x_1,\dots,x_k)\mapsto x_1\cdots x_k$, es un morfismo inyectivo de grupos.
% 	Como entonces $G\simeq P_1\times\cdots P_k$, y cada $P_j$ es nilpotente,
% 	$G$ es nilpotente. 

% 	Para demostrar que $(1)\implies(3)$ simplemente hay que observar que todo
% 	cociente de $G$ es nilpotente y luego utilizar el
% 	teorema~\ref{theorem:Z(nilpotent)}. Demostremos que $(3)\implies(1)$. Como
% 	todo cociente no trivial de $G$ tiene centro no trivial, en particular
% 	$Z_1=Z(G)$ es no trivial. Si $Z_1=G$ entonces $G$ es abeliano y no hay nada
% 	para demostrar. Si $Z_1\ne G$ entonces $G/Z_1\ne 1$; luego $Z(G/Z_1)\ne 1$.
% 	Si $\pi_1\colon G\to G/Z_1$ es el morfismo canónico,
% 	$Z_2=\pi_1^{-1}(Z(G/Z_1))$. Inductivamente, si tenemos construido el
% 	subgrupo $Z_i$, $Z_i\ne G$ y  $\pi_i\colon G\to G/Z_{i}$ es el morfismo
% 	canónico, se define el subgrupo $Z_{i+1}=\pi_i^{-1}(Z(G/Z_i))$. Por
% 	construcción, $Z_i\subseteq Z_{i+1}$ para todo $i$. Como $G$ es finito,
% 	existe $k$ tal que $Z_k=G$ y luego $G$ es nilpotente.

% 	Demostremos que $(1)\implies(4)$. Esta implicación es consecuencia
% 	inmediata del ejercicio~\ref{exercise:pgrupos}. 
% 	Como $G$ es nilpotente, $G$ producto
% 	directo de sus $p$-grupos de Sylow. Si $d=p_1^{\alpha_1}\cdots
% 	p_k^{\alpha_k}$ es un divisor del orden de $G$, basta tomar
% 	$H=H_1\times\cdots\times H_k$, 
% 	donde cada $H_j$ es un subgrupo normal del $p_j$-subgrupo de Sylow de $G$
% 	de orden $p_j^{\alpha_j}$. Para demostrar que $(4)\implies(1)$ simplemente
% 	se aplica la hipótesis a cada $p$-subgrupo de $G$ de orden maximal.
% \end{svgraybox}

El siguiente resultado, que puede demostrarse en forma
completamente elemental, fue descubierto en 2014. 

\begin{theorem}[Baumslag--Wiegold]
	\index{Teorema de!Baumslag--Wiegold}
	Sea $G$ un grupo finito tal que $|xy|=|x||y|$ si $x$ e $y$ son elementos de
	órdenes coprimos. Entonces $G$ es nilpotente.
\end{theorem}

\begin{proof}
	Sean $p_1,\dots,p_n$ los distintos primos que dividen al orden de $G$. Para cada
	$i\in\{1,\dots,n\}$ sea $P_i\in\Syl_{p_i}(G)$.  Primero vamos a demostrar
	que $G=P_1\cdots P_n$. La inclusión no trivial equivale a 
	demostrar que la función 
	\[
		\psi\colon P_1\times\cdots\times P_n\to G,\quad
		(x_1,\dots,x_n)\mapsto x_1\cdots x_n
	\]
	es sobreyectiva. Procederemos de la siguiente forma. Primero vemos que  
	la función $\psi$ es inyectiva. En efecrto, si
	$\psi(x_1,\dots,x_n)=\psi(y_1,\dots,y_n)$, entonces 
	\[
		x_1\cdots
	x_n=y_1\cdots y_n. 
	\]
	Si $y_n\ne x_n$, entonces $x_1\cdots x_{n-1}=(y_1\cdots
	y_{n-1})y_nx_n^{-1}$. Pero $x_1\cdots x_{n-1}$ es un elemento de orden
	coprimo con $p_n$ y $y_1\cdots y_{n-1}y_nx_n^{-1}$ es un elemento de orden
	múltiplo de $p_n$, una contradicción. Entonces $x_n=y_n$ y luego, el mismo
	argumento, prueba que $\psi$ es inyectiva. Como $|P_1\times\cdots\times
	P_n|=|G|$, se concluye que $\psi$ es biyectiva. En particular, $\psi$ es sobreyectiva
	y luego $G=P_1\cdots P_n$.

	Veamos ahora que cada $P_j$ es normal en $G$.  Sea $j\in\{1,\dots,n\}$ y
	sea $x_j\in P_j$. Sea $g\in G$ y sea $y_j=gx_jg^{-1}$.  Como $y_j\in G$,
	podemos escribir $y_j=z_1\cdots z_n$ con $z_k\in P_k$ para todo $k$.  Como
	el orden de $y_j$ es una potencia del primo $p_j$, el elemento $z_1\cdots
	z_n$ tiene orden una potencia de $p_j$ y luego $z_k=1$ para todo $k\ne j$ y
	además $y_j=z_j\in P_j$. Como cada subgrupo de Sylow es normal en $G$, se
	concluye que $G$ es nilpotente.
\end{proof}


%\section{Grupos nilpotentes de clase dos}

\begin{lemma}
	\label{lemma:commutador}
	Si $x,y\in G$ son tales que $[x,y]\in C_G(x)\cap C_G(y)$, entonces
	\[
	[x,y]^n=[x^n,y]=[x,y^n]
	\]
	para todo $n\in\Z$.
\end{lemma}

\begin{proof}
	Procederemos por inducción en $n\geq0$. El caso $n=0$ es trivial. Supongamos entonces
	que el resultado vale para algún $n\geq0$. Entonces, como $[x,y]\in C_G(x)$, 
	\begin{align*}
		[x,y]^{n+1}&=[x,y]^n[x,y]
		=[x^n,y][x,y]=[x^n,y]xyx^{-1}y^{-1}=x[x^n,y]yx^{-1}y^{-1}=[x^{n+1},y].
	\end{align*}
	Para demostrar el lema en el caso $n<0$ basta observar que, como $[x,y]^{-1}=[x^{-1},y]$, 
	$[x,y]^{-n}=[x^{-1},y]^n=[x^{-n},y]$.
\end{proof}

\begin{lemma}[Hall]
    \index{Lema!de Hall}
	\label{lemma:Hall}
	Sea $G$ un grupo nilpotente de clase dos y $x,y\in G$. Entonces
	\[
		(xy)^n=[y,x]^{n(n-1)/2}x^ny^n
	\]
	para todo $n\in\N$.
\end{lemma}

\begin{proof}
	Procederemos por inducción en $n$. Como el caso $n=1$ es trivial,
	supongamos que el resultado es válido para algún $n\geq1$. Entonces,
	gracias al lema anterior, 
	\begin{align*}
		(xy)^{n+1} &= (xy)^n(xy)=[y,x]^{n(n-1)/2}x^ny^{n-1}(yx)y\\
		&=[y,x]^{n(n-1)/2}x^{n}[y^n,x]xy^{n+1}=[y,x]^{n(n-1)/2}[y,x]^nx^{n+1}y^{n+1}.\qedhere 
	\end{align*}
\end{proof}

\begin{lemma}
	\label{lemma:class2}
	Sea $p>2$ un número primo y sea 
	$P$ un $p$-grupo de clase de nilpotencia $\leq2$. 
	Si $[y,x]^p=1$ para todo $x,y\in P$ entonces $P\to [P,P]$,
	$x\mapsto x^p$, es un morfismo de grupos.
\end{lemma}

\begin{proof}
	Por lema de Hall,
	$(xy)^p=[y,x]^{p(p-1)/2}x^py^p=x^py^p$. 	
\end{proof}

\begin{theorem}
	\label{thm:class2}
	Sea $p>2$ un número primo y sea 
	$P$ un $p$-grupo de clase de nilpotencia $\leq2$. 
	Entonces $\{x\in P:x^p=1\}$ es un subgrupo de $P$.
\end{theorem}

\begin{proof}
	Como $P$ tiene clase de nilpotencia dos, los conmutadores son centrales.
	Para cada $x\in G$, la función $g\mapsto [g,x]$ es un morfismo de grupos
	pues
	\[
		[gh,x]=ghxh^{-1}g^{-1}x^{-1}=g[h,x]xg^{-1}x^{-1}=[g,x][h,x].
	\]
	En particular, si $x,y\in P$ con $x^p=y^p=1$, entonces
	\[
		[x,y]^p=[x^p,y]=[1,y]=1.
	\]
	Luego, al usar el lema de Hall, se concluye que
	$(xy)^p=[y,x]^{p(p-1)/2}x^py^p=1$.
\end{proof}

