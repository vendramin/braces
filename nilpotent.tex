\chapter{Nilpotent groups}
\label{nilpotent}

\section*{Central series}

\index{Group!central series}\index{Group!center}
Let $G$ be a group. The {\em center} of $G$ is 
\[ Z(G)=\{ x\in G\mid xy=yx,\; \mbox{ for all } y\in G\}.\]
A subgroup $H$ of $G$ is {\em central} in $G$ if $H\subseteq Z(G)$. \index{Subgroup!central}

\index{Subgroup!characteristic}
A subgroup $H$ of a group $G$ is said to be {\em characteristic} if 
$f(H)= H$ for all $f\in\Aut(G)$. The center $Z(G)$ of $G$ is a characteristic subgroup of $G$. Every characteristic subgroup of $G$ is normal in $G$. 
If $H$ is a characteristic subgroup of $K$ and $K$ is normal in $G$, 
then $H$ is normal in $G$. 


\index{Centralizer} Let $X$ be a non-empty subset of $G$. The {\em centralizer} of $X$ in $G$ is
\[ C_G(X)=\{ y\in G\mid xy=yx,\;\mbox{ for all }x\in X\}.
\]
Note that $C_G(X)$ is a subgroup of $G$. In particular, $C_G(G)=Z(G)$.

Let $H$ be a normal subgroup of $G$. Let $\varphi\colon G\rightarrow \Aut(H)$ be the map defined by $\varphi(x)=\varphi_x$ and
$\varphi_x(h)=xhx^{-1}$, for all $x\in G$ and $h\in H$. It is easy to check that $\varphi$ is a homomorphism of groups and $\ker(\varphi)=C_G(H)$. Hence $C_G(H)$ is a normal subgroup of $G$. We also say that $\varphi$ is a left action of $G$ on $H$ by conjugation.


If $G$ is a group and $x,y,z\in G$, then the conjugation (as a left action) will be denoted
by $\prescript{x}{}y=xyx^{-1}$. \index{Commutator} The {\em commutator} between $x$ and $y$ is then 
\[
[x,y]=xyx^{-1}y^{-1}=(\prescript{x}{}y)y^{-1}.
\]
We also write 
$[x,y,z]=[x,[y,z]]$. 
If $X$, $Y$ and $Z$ are subgroups of $G$, we write 
\[
[X,Y]=\langle [x,y]:x\in X,\,y\in Y\rangle
\]
and $[X,Y,Z]=\left[ X,[Y,Z] \right]$. Since $[X,Y]=[Y,X]$, it follows that  
$[X,Y,Z]=[X,Z,Y]$. 

\index{Subgroup!commutator} Note that the {\em commutator subgroup} $[G,G]$ of $G$ is a characteristic subgroup of $G$.

The proof of the following lemma is an easy exercise.

\begin{lemma}[Hall--Witt's identity]
	\label{xca:HallWitt}
	\index{Hall--Witt!identity}
	\index{Hall, P.}
	\index{Witt, E.}
	Let $G$ be a group and $x,y,z\in G$. Then 
	\begin{equation}
		\label{eq:HallWitt}
	\left(\prescript{y}{}[x,y^{-1},z]\right)\left(\prescript{z}{}[y,z^{-1},x]\right)\left(\prescript{x}{}[z,x^{-1},y]\right)=1.
	\end{equation}
\end{lemma}

\index{Jacobi, G.}
\index{Jacobi!identity}
Note that if $G$ is such that $[G,G]$ is central, then Hall--Witt's identity 
turns out to be Jacobi's identity, i.e. 
\[ [x,y,z][y,z,x][z,x,y]=1.\]  

%\begin{lemma}[three subgroups lemma]
%	\label{lemma:3subgrupos}
%	\index{Lemma!three subgroups}
%	Let $X$, $Y$ and $Z$ be subgroups of $G$ such that $[X,Y,Z]=[Y,Z,X]=\{1\}$.
%	Then $[Z,X,Y]=\{1\}$.
%\end{lemma}
%
%\begin{proof}
%	Since $[x,y]\in C_G(z)$ implies that 
%	$[X,Y]\subseteq C_G(Z)$, it
%	is enough to prove that $[z,x^{-1},y]=1$ for all $x\in X$, $y\in Y$ and $z\in Z$, 
%	Since $[y^{-1},z]\in [Y,Z]$, it follows that 
%	$[x,y^{-1},z]\in [X,Y,Z]=\{1\}$. Thus $\prescript{y}{}[x,y^{-1},z]=1$.
%	Similarly, $\prescript{z}{}[y,z^{-1},x]=1$. Hence the 
%	Hall--Witt identity yields 
%	$[z,x^{-1},y]=1$.
%\end{proof}

\index{Three subgroup lemma}
\begin{lemma}[three subgroups lemma]
	\label{lemma:3subgrupos_general}
	Let $N$ be a normal subgroup of $G$ and let $X$, $Y$ and $Z$
	be subgroups of $G$. If $[X,Y,Z]\subseteq N$ and $[Y,Z,X]\subseteq N$, then 
	$[Z,X,Y]\subseteq N$.
\end{lemma}

\begin{proof}
	We first prove the lemma in the case where $N=\{1\}$. 	In this case,  
	$[Y,Z]\subseteq C_G(X)$ and $[Z,X]\subseteq C_G(Y)$. It
	is enough to prove that $[z,x^{-1},y]=1$, for all $x\in X$, $y\in Y$ and $z\in Z$. 
	Let $x\in X$, $y\in Y$ and $z\in Z$. Since $[x,y^{-1},z]=1$, it follows that 
	$\prescript{y}{}[x,y^{-1},z]=1$.
	Similarly, $\prescript{z}{}[y,z^{-1},x]=1$. Hence the 
	Hall--Witt identity yields 
	$[z,x^{-1},y]=1$.

	We now demonstrate the general case. Let $N$ be a normal subgroup of $G$ 
	and $\pi\colon G\to G/N$ be the canonical map. Since $[X,Y,Z]\subseteq N$, 
	\begin{align*}
		\{1\}&=\pi([X,Y,Z])=\pi([ X,[Y,Z]])\\
		&=[\pi(X),\pi([Y,Z])]=[ \pi(X),[\pi(Y),\pi(Z)]]=[\pi(X),\pi(Y),\pi(Z)]. 
	\end{align*}
	Similarly one proves that $[\pi(Y),\pi(Z),\pi(X)]=\{1\}$. By the previous paragraph,  
	$[\pi(Z),\pi(X),\pi(Y)]=\{1\}$, so $[Z,X,Y]\subseteq N$.
\end{proof}

\index{Lower central series}
The {\em lower central series} of a group $G$ is 
the sequence $\gamma_k(G)$, for positive integers $k$, 
defined recursively as 
\[
		\gamma_1(G)=G,\quad
		\gamma_{i+1}(G)=[G,\gamma_i(G)]\quad i\geq 1.
\]

\index{Group!nilpotent}
\index{Nilpotency index}
\index{Nilpotency class}
A group $G$ is said to be {\em nilpotent} if there exists a non-negative integer $c$ such that 
$\gamma_{c+1}(G)=\{1\}$. The smallest $c$ such that $\gamma_{c+1}(G)=\{1\}$ is
the {\em nilpotency index} (or {\em nilpotency class}) of $G$. 

% \begin{exercise}
% \label{xca:nilpotente=>resoluble}
% 	Demuestre que todo grupo nilpotente es resoluble.
% \end{exercise}

\begin{example}
	A group is nilpotent of class one if and only if it is  non-trivial and abelian.  
\end{example}

% \begin{example}
% 	$\Sym_3$ es resoluble pues $\Sym_3\supseteq \Alt_3\supseteq\{1\}$ es una serie
% 	de composición con factores abelianos pero $\Sym_3$ no es nilpotente pues
% 	\[
% 		\gamma_1(\Sym_3)=\Alt_3,\quad
% 		\gamma_2(\Sym_3)=[\Alt_3,\Sym_3]=\Alt_3.
% 	\]
% 	Luego $\gamma_i(\Sym_3)\ne1$ para todo $i\geq1$. 
% \end{example}

\begin{example}
    The group $G=\Alt_4$ is not nilpotent, as 
	\[
		\gamma_1(G)=G,\quad
		\gamma_j(G)=\{\id,(12)(34),(13)(24),(14)(23)\}\simeq C_2\times C_2
	\]
	for all $j\geq2$. 
\end{example}

\begin{lemma}
	\label{xca:gamma}
	Let $G$ be group. Then the following conditions hold. %Prove the following statements:
	\begin{enumerate}
		\item Each $\gamma_i(G)$ is a characteristic subgroup of $G$.
		\item $\gamma_i(G)\supseteq\gamma_{i+1}(G)$ for all $i\geq1$.
		\item If $f\colon G\to H$ is a surjective group homomorphism, then 
			$f(\gamma_i(G))=\gamma_i(H)$ for all $i\geq 1$.
	\end{enumerate}
\end{lemma}

\begin{proof}
    We shall prove the first statement by induction on $i$. Let $h\in \Aut(G)$. For $i=1$, the result is clear. Suppose that the result holds for some $i\geq 1$. By the inductive hypothesis,
    \[h(\gamma_{i+1}(G))=h([G,\gamma_i(G)])=[h(G),h(\gamma_i(G))]=[G,\gamma_i(G)]=\gamma_{i+1}(G).\]
    Therefore, the first statement follows by induction.
    
    The second and the third statements can be easily proved by induction on $i$.
\end{proof}

% \begin{exercise}
% 	\label{xca:HxK_nilpotente}
% 	Demuestre que si $H$ y $K$ son nilpotentes entonces $H\times K$ es
% 	nilpotente.
% \end{exercise}

\begin{theorem}
	\label{theorem:nilpotent}
	Let $G$ be a nilpotent group. 
	\begin{enumerate}
		\item If $H$ is a subgroup of $G$, then $H$ is nilpotent. 
		\item If $f\colon G\to H$ is a surjective group homomorphism, then $H$ is nilpotent. 
	\end{enumerate}
\end{theorem}

\begin{proof}
	For the first statement note that $\gamma_i(H)\subseteq\gamma_i(G)$ for all 
	$i\geq 1$. Let us prove the second claim, if there exists $c$ such that $\gamma_{c+1}(G)=\{1\}$, 
	then 
	\[
	\gamma_{c+1}(H)=f(\gamma_{c+1}(G))=f(\{1\})=\{1\}.\qedhere
	\]
\end{proof}

%\begin{example}
%    The group $\SL_2(3)$ is not nilpotent, as $\Alt_4$ 
%	is a quotient of $\SL_2(3)$. 
%\end{example}

There exist a non-nilpotent group $G$ with a normal subgroup $K$ 
such that $K$ and $G/K$ are both nilpotent. For example, take $G=\Sym_3$ and $K=\Alt_3$. 


\begin{theorem}
	\label{theorem:gamma}
	Let $G$ be a group. Then $[\gamma_i(G),\gamma_j(G)]\subseteq
	\gamma_{i+j}(G)$ for all $i,j\geq1$.	
\end{theorem}

\begin{proof}
	We proceed by induction on $i$. The case where $i=1$ is trivial, as by definition one has 
	$[G,\gamma_j(G)]=\gamma_{j+1}(G)$. Assume now that the result holds for 
	some $i\geq1$ and all $j\geq1$. 
	We first note that 	
	\begin{equation*}
		[G,\gamma_i(G),\gamma_j(G)]\subseteq [G,\gamma_{i+j}(G)]= \gamma_{i+j+1}(G)
	\end{equation*}
	by the inductive hypothesis. Moreover, again using the inductive hypothesis,  
	\begin{equation*}
	[\gamma_i(G),\gamma_j(G),G]=[\gamma_{i}(G),G,\gamma_{j}(G)]=[\gamma_i(G),\gamma_{j+1}(G)]\subseteq \gamma_{i+j+1}(G).
	\end{equation*}
	Lemma~\ref{lemma:3subgrupos_general} implies that 
	$[\gamma_j(G),G,\gamma_i(G)]\subseteq \gamma_{i+j+1}(G)$. Thus  
	\[
	[\gamma_{i+1}(G),\gamma_{j}(G)]=[[G,\gamma_{i}(G)],\gamma_j(G)]=[\gamma_j(G),G,\gamma_i(G)]
	\subseteq \gamma_{i+j+1}(G).\qedhere
	\]
\end{proof}

Certainly we can consider other type of arbitrary commutators, say for example 
$[[G,G],G]$ and $[G,G,G]=[G,[G,G]]$. This naturally suggest the notion of the
weight of a commutator. For example, $[[G,G],G]$ and $[G,G,G]=[G,[G,G]]$ are
both commutators of weight three. 

\begin{corollary}
	Every commutator of weight $n$ is contained in 
	$\gamma_n(G)$.
\end{corollary}

\begin{proof}
	We proceed by induction on $n$. The case $n=1$ is trivial, so assume that $n\geq 1$ and
	the result holds for all $m$ such that  $n\geq m\geq 1$.  Let  
	$[A,B]$ be a commutator, where $A$ is a commutator of weight $k$,
	$B$ is a commutator of weight $l$ and $n+1=k+l$. Since $k\leq n$ and $l\leq n$, the inductive 
	hypothesis implies that $A\subseteq \gamma_k(G)$ y $B\subseteq
	\gamma_l(G)$. Thus 
	\[
	[A,B]\subseteq [\gamma_k(G),\gamma_l(G)]\subseteq
	\gamma_{k+l}(G)
	\]
	by the previous theorem. Therefore, the result follows by induction.
\end{proof}

\index{Normalizer condition}
A group 
$G$ satisfies the {\em normalizer condition} if each  proper subgroup $H$ is properly contained in its normalizer $N_G(H)$, that is $H\subsetneq N_G(H)$. 

\begin{lemma}[normalizer condition]
	\label{lem:normalizer}
	Let $G$ be a nilpotent group. If $H$ is a proper subgroup of $G$, then 
	$H\subsetneq N_G(H)$.
\end{lemma}

\begin{proof}
	Since $G$ is nilpotent, there a positive integer $c$ such that 
	\[
	G=\gamma_1(G)\supseteq\cdots\supseteq\gamma_{c+1}(G)=\{1\}.
	\]
	Since 
	$\{1\}=\gamma_{c+1}(G)\subseteq H$ and $\gamma_1(G)\not\subseteq H$, 
	let $k$ be the smallest positive integer such that $\gamma_k(G)\subseteq H$. Since 
	\[
		[H,\gamma_{k-1}(G)]\subseteq [G,\gamma_{k-1}(G)]=\gamma_k(G)\subseteq H,
	\]
	it follows that 
	$xHx^{-1}\subseteq H$ for all $x\in\gamma_{k-1}(G)$,
	so  $\gamma_{k-1}(G)\subseteq N_G(H)$. If $N_G(H)=H$, then 
	$\gamma_{k-1}(G)\subseteq H$, in contradiction with the minimality of $k$. Therefore
	$H$ is a proper subgroup of $N_G(H)$.
\end{proof}

%\begin{example}
%	Un grupo $G$ es nilpotente de clase dos si y sólo $\gamma_2(G)

%\end{example}

For a group $G$ we define the sequence $\zeta_0(G),\zeta_1(G),\dots$
recursively as 
\[
	\zeta_0(G)=\{1\},\quad
	\zeta_{i+1}(G)=\{g\in G:[x,g]\in\zeta_{i}(G)\text{ for all $x\in G$}\},\quad i\geq 0.
\]
In particular, $\zeta_1(G)=Z(G)$.

\begin{lemma}
	\label{lem:central_ascendente}
	Let $G$ be a group. Each $\zeta_i(G)$ is a normal
	subgroup of $G$. 
\end{lemma}

\begin{proof}
	We proceed by induction on $i$. The case $i=0$ is trivial, as 
	$\zeta_0(G)=\{1\}$.  Assume that the result holds for some $i\geq0$. Thus $\zeta_i(G)$ is a normal subgroup of $G$.
	Let $\pi\colon G\rightarrow G/\zeta_i(G)$ be the canonical map. Note that
	\[ \zeta_{i+1}(G)=\pi^{-1}(Z(G/\zeta_i(g))).\]
	Therefore $\zeta_{i+1}(G)$ is a normal subgroup of $G$, and the result follows by induction.
%	
%	We claim that $\zeta_{i+1}(G)$ is a subgroup of $G$. 
%	Let $g,h\in \zeta_{i+1}(G)$ and $x\in G$. The inductive hypothesis implies that 
%	\begin{align*}
%	&[g^{-1},x]=(xg^{-1})[g,x^{-1}](xg^{-1})^{-1}\in (xg^{-1})\zeta_i(G)(xg^{-1})^{-1}=\zeta_i(G),\\
%	&[gh,x]=[g,hxh^{-1}][h,x]\in \zeta_{i}(G).
%	\end{align*}
%	Since $1\in\zeta_{i+1}(G)$, the sets $\zeta_i(G)$ are subgroups of $G$. 
%	
%	To prove 
%	that each subgroup is normal we also proceed by induction on $i$. 
%	If $g\in\zeta_{i+1}(G)$ and $x\in G$, then $xgx^{-1}\in\zeta_{i+1}(G)$. Indeed,  
%	\[
%	[xgx^{-1},y]=x[g,x^{-1}yx]x^{-1}\in\zeta_{i}(G)
%	\]
%	for all $y\in G$.
\end{proof}


\index{Upper central series}
For a group $G$ the \textbf{\em upper central series} of $G$ is the sequence 
\[
	\{1\}=\zeta_0(G)\subseteq\zeta_1(G)\subseteq\zeta_2(G)\subseteq\cdots
\]


\index{Normalizer}
\index{Centralizer}
A subgroup $K$ of $G$ {\em normalizes} a subgroup 
$H$ if $K\subseteq N_G(H)$.
A subgroup $K$ of $G$ {\em centralizes} a subgroup 
$H$ if $K\subseteq C_G(H)$, that is if and only if $[H,K]=\{1\}$.

%\begin{exercise}
%	Sean $K$ y $H$ subgrupos de $G$ con $K\subseteq H$ y $K$ normal en $G$.
%	Demuestre que $[H,G]\subseteq K$ si y sólo si $H/K\subseteq Z(G/K)$. 
%\end{exercise}
\begin{lemma}
	\label{lem:gamma_zeta}
	Let $G$ be a group. There exists $c$ such that $\zeta_c(G)=G$ if and only if 
	$\gamma_{c+1}(G)=\{1\}$. In this case,  
	\[
	\gamma_{i+1}(G)\subseteq\zeta_{c-i}(G)
	\]
	for all $i\in\{0,1,\dots,c\}$. 
\end{lemma}

\begin{proof}
    For $c=0$, the result is trivial. Suppose that $c\geq 1$.
	Assume first that $\zeta_c(G)=G$. To prove that 
    $\gamma_{i+1}(G)\subseteq\zeta_{c-i}(G)$ we proceed by induction. The case where $i=0$ is
	trivial, so assume the result holds for some $i\geq0$. 
	Let $x\in G$ and $g\in \gamma_{i+1}(G)$. By the inductive hypothesis $g\in \zeta_{c-i}(G)$. 
	Hence $[x,g]\in \zeta_{c-i-1}(G)$. Therefore, $\gamma_{i+2}(G)\subseteq \zeta_{c-i-1}$.
%	If
%	$g\in\gamma_{i+2}(G)=[G,\gamma_{i+1}(G)]$, then   
%	\[
%	g=\prod_{k=1}^N [x_k,g_k],
%	\]
%	for some $g_1,\dots,g_N\in\gamma_{i+1}(G)$ and $x_1,\dots,x_N\in G$. By the inductice hypothesis, 
%	\[
%	g_k\in\gamma_i(G)\subseteq\zeta_{c-i}(G)
%	\]
%	for all $k$ and thus $[x_k,g_k]\in\zeta_{c-i-1}(G)$ for all $k$. Hence 
%	$g\in\zeta_{c-(i+1)}(G)$.
	By induction, this claim follows. In particular, $\gamma_{c+1}(G)\subseteq \zeta_{0}(G)=\{ 1\}$.
	
	Assume now that $\gamma_{c+1}(G)=\{1\}$. We prove that 
	$\gamma_{i+1}(G)\subseteq\zeta_{c-i}(G)$ for all $i$. We proceed by backward induction on $i$. 
	The case $i=c$ is trivial, so assume the
	result holds for some $i+1\leq c$. Let $g\in\gamma_{i}(G)$. By the inductive hypothesis, 
	\[
	[x,g]\in [G,\gamma_i(G)]=\gamma_{i+1}(G)\subseteq\zeta_{c-i}(G),
	\]
	for all $x\in G$.
	Thus $g\in\zeta_{c-i+1}(G)$ by definition. Hence $\gamma_{i}(G)\subseteq\zeta_{c-i+1}(G)$, and the result follows
	by induction. 
\end{proof}

\begin{example}
	If $G=\Sym_3$, then $\zeta_j(G)=\{1\}$ for all $j\geq 0$.
	\end{example}

\index{Central!series}
A {\em central series} of a group $G$ is a sequence
\[
	G=G_0\supseteq G_1\supseteq\cdots\supseteq G_n=\{1\}
\]
of normal subgroups of $G$ such that for each $i\in\{1,\dots,n\}$, 
$\pi_i(G_{i-1})$ is a subgroup of $Z(G/G_i)$, where $\pi_i\colon G\to
G/G_i$ is the canonical map. 

\begin{proposition}
	\label{pro:serie_central}
	Let $G$ be a group and let $G=G_0\supseteq G_1\supseteq\cdots\supseteq G_n=\{1\}$
	be a central series of $G$. Then $\gamma_{i+1}(G)\subseteq G_i\subseteq \zeta_{n-i}(G)$, for all $i$.
\end{proposition}

\begin{proof}
    We shall prove that $\gamma_{i+1}(G)\subseteq G_i$ by induction on $i$. 
	The case $i=0$ is trivial, so we assume that 
	the result holds for some 
	$i\geq 0$. 
	Let $\pi_{i+1}\colon G\to G/G_{i+1}$ is the canonical map. 
	Since $\pi_{i+1}(G_{i})\subseteq Z(G/G_{i+1})$, it follows that
	\[
	\pi_{i+1}([G,G_{i}])=[\pi_{i+1}(G),\pi_{i+1}(G_{i})]=\{1\}.
	\]
	This implies that $[G,G_{i}]\subseteq\ker\pi_{i+1}=G_{i+1}$. Hence, by the inductive hypothesis,
	\[
	\gamma_{i+2}(G)=[G,\gamma_{i+1}(G)]\subseteq [G,G_{i}]\subseteq G_{i+1}. 
	\] Thus the first inclusion follows by induction.
	
	We shall prove that $G_i\subseteq \zeta_{n-i}(G)$, by induction on $n-i$. For $n-i=0$, $G_n=\{ 1\}=\zeta_0(G)$. Suppose that the
	inclusion holds for some $n-i\geq 0$. Since $G_i\subseteq \zeta_{n-i}(G)$, there exists a surjective group homomorphism
	$f\colon G/G_i\rightarrow G/\zeta_{n-i}(G)$, such that $f(gG_i)=g\zeta_{n-i}(G)$, for all $g\in G$. 
	Hence $f(Z(G/G_i))\subseteq Z(G/\zeta_{n-i}(G))$. Since  $f\circ\pi_i\colon G\rightarrow G/\zeta_{n-i}(G)$ is the canonical map and
	$f(\pi_i(G_{i-1}))\subseteq f(Z(G/G_i))\subseteq Z(G/\zeta_{n-i}(G))$, we have that 
	$G_{i-1}\subseteq (f\circ \pi_i)^{-1}(Z(G/\zeta_{n-i}(G)))=\zeta_{n-i+1}(G)$. Therefore, the result follows by induction.
\end{proof}

% extender el lema para ver qué pasa con zeta_i


%\begin{proof}
%	If $G$ is nilpotent, then the $\gamma_j(G)$ form a central series for $G$. Conversely, if 
%	$G=G_0\supseteq
%	G_1\supseteq\cdots\supseteq G_n=\{1\}$ is a central series of $G$, then 
%	$G$ is nilpotent by Lemma~\ref{lem:serie_central}. 
%\end{proof}



\begin{theorem}[Hirsch]
	\label{thm:Z(nilpotent)}
	\index{Hirsch's Theorem}
	Let $G$ be a non-trivial nilpotent group. If $H$ is a non-trivial normal subgroup of $G$, then 
	$H\cap Z(G)\ne\{1\}$. In particular, $Z(G)\ne\{1\}$. 
\end{theorem}

\begin{proof}
	Since $\zeta_0(G)=\{1\}$ and there exists $c\geq 1$ such that $\zeta_c(G)=G$, there exists  
	\[
	m=\min\{k:H\cap\zeta_k(G)\ne\{1\}\}.
	\]
	Since $H$ is normal, 
	\[
	[H\cap\zeta_m(G),G]\subseteq H\cap[\zeta_m(G),G]\subseteq H\cap\zeta_{m-1}(G)=\{1\}.
	\]
	Thus $\{1\}\ne H\cap\zeta_m(G)\subseteq H\cap Z(G)$. If $H=G$, then $Z(G)\ne\{1\}$. 
\end{proof}


% \begin{svgraybox}
% 	Como $M\cap Z(G)$ es normal en $G$, la minimalidad de $M$ implica que hay
% 	dos posibilidades: $M\cap Z(G)$ es trivial o bien $M=M\cap Z(G)\subseteq Z(G)$.
% 	Por el teorema~\ref{theorem:Z(nilpotent)}, $M\cap Z(G)\ne 1$.
% \end{svgraybox}

\index{Subgroup!maximal normal}
A subgroup $M$ of $G$ is {\em maximal normal} if it is maximal
among all normal proper subgroups of $G$.  

\begin{corollary}
	Let $G$ be a non-abelian nilpotent group and $A$ be an abelian 
	maximal normal subgroup of $G$. Then $A=C_G(A)$.
\end{corollary}

\begin{proof}
	Since $A$ is abelian, $A\subseteq C_G(A)$. Assume that $A\ne C_G(A)$.
	We know that the centralizer of a normal subgroup is normal. Thus the centralizer $C_G(A)$ is normal in $G$. 
%	In fact, since $A$ is normal in $G$, 
%	\[
%		gC_G(A)g^{-1}=C_G(gAg^{-1})=C_G(A)
%	\]
%	for all $g\in G$.  
	Let $\pi\colon G\to G/A$ be the canonical map.
	Then $\pi(C_G(A))$ is a non-trivial normal subgroup of $\pi(G)$. Since 
	$G$ is nilpotent, $\pi(G)$ is nilpotent. By Hirsch's theorem, 
	$\pi(C_G(A))\cap Z(\pi(G))\ne\{1\}$. Let
	$x\in C_G(A)\setminus A$ be such that $\pi(x)$ is central in $\pi(G)$. 
	Note that if $g\in G$, then $gxg^{-1}\in xA$. 
	Hence 
	$\langle A,x\rangle$ is an abelian normal subgroup of $G$ such that   
	$A\subsetneq \langle
	A,x\rangle\subsetneq G$, a contradiction. Therefore $A=C_G(A)$. 
\end{proof}

\index{Subgroup!minimal normal}
A subgroup $M$ of a group $G$ is said to be {\em minimal normal} if $M\ne\{1\}$,
$M$ is normal in $G$ and the unique normal subgroup of $G$ strictly contained in $M$ is
the trivial subgroup. Every finite group contains a minimal normal subgroup.  

\begin{example}
	If a non-trivial normal subgroup $M$ is minimal (with respect to the inclusion), then it is
	minimal normal. The converse statement is not true. The subgroup of 
	$\Alt_4$ generated by $(12)(34)$, $(13)(24)$ and $(14)(23)$ is minimal normal in 
	$\Alt_4$ but it is not minimal {\bf non-trivial}. 
\end{example}

\begin{example}
	Let $G=\D_{6}=\langle r,s:r^6=s^2=1,\,srs=r^{-1}\rangle$ be the dihedral group
	of size twelve. The subgroups $S=\langle r^2\rangle$ 
	and $T=\langle r^3\rangle$ are minimal normal subgroups.
	\end{example}


\begin{theorem}\label{theorem:minmax_nilpotent}
	Let $G$ be a non-trivial nilpotent group. The following statements hold:
	\begin{enumerate}
		\item Every minimal normal subgroup of $G$ has prime order and it is central. 
		\item Every maximal subgroup of $G$ is normal, has prime index and contains $[G,G]$. 
	\end{enumerate}
\end{theorem}

\begin{proof}
	We first prove (1). Let $N$ be a minimal normal subgroup of $G$. Since $G$ is nilpotent, 
	$N\cap Z(G)\ne\{1\}$ by Hirsch's theorem. It follows that $N\cap Z(G)$ is a normal subgroup of $G$ contained in $N$. 
	Thus $N=N\cap Z(G)\subseteq
	Z(G)$ by the minimality of $N$. In particular, $N$ is abelian. Since 
	every subgroup of $N$ is normal in $G$, $N$ should be cyclic of prime order.

	We now prove (2). If $M$ is a maximal subgroup, then $M$
	is normal en $G$ by the normalizer condition. The maximality of $M$ implies that 
	$G/M$ contains no non-trivial proper subgroups.  Since $G/M$ is nilpotent and $Z(G/M)$ is non-trivial, we have that $G/M$ is abelian. Thus 
	$G/M$ should be cyclic of prime order and thus $[G,G]\subseteq M$. 
\end{proof}

The theorem does not prove the existence of maximal subgroups, see for example what happens with
the additive group of rational numbers. 

\begin{proposition}
	\label{pro:g^n}
	Let $G$ be a nilpotent group and $H$ be a subgroup of $G$ of index $n$. If 
	$g\in G$, then $g^n\in H$.
\end{proposition}

\begin{proof}
	We proceed by induction on $n$. The case $n=1$ is trivial. Assume  that $n>1$ and the result holds 
	for all subgroups of index $<n$. If $H$ is  a normal subgroup of index $n$, then clearly $g^n\in H$, for all $g\in G$. 
	Hence we may assume that $H$ is a non-normal subgroup of $G$ of index $n$. By the normalizer condition, $H\subsetneq N_G(H)$. Hence
	$(G:N_G(H)),(N_G(H):H)<n$, and thus, by the inductive hypothesis, $g^{(G:N_G(H))}\in N_G(H)$ and
	\[
		g^n=g^{(G:H)}=g^{(G:N_G(H))(N_G(H):H)}\in H, 
	\]
	for all $g\in G$. Therefore, the result follows by induction.
\end{proof}

\begin{example}
The nilpotency of $G$ is needed in the previous proposition. 
If $G=\Sym_3$ and $H=\{\id,(12)\}$, then $(G:H)=3$. If 
$g=(13)$, then $g^{3}=(13)\not\in H$. 	
\end{example}

The following tool is useful to perform induction 
on nilpotent groups.  

\begin{lemma}
	\label{lem:a[GG]}
	Let $G$ be a nilpotent group of class $c\geq2$. If $x\in G$, then the subgroup 
	$\langle x,[G,G]\rangle$ is nilpotent of class $<c$.
\end{lemma}

\begin{proof}
	Let $H=\langle x,[G,G]\rangle$.   Note that  
	\[
		H=\{x^nc:n\in\Z,c\in [G,G]\},
	\]
	as $[G,G]$ is normal in $G$. It is enough to show that   
	$[H,H]\subseteq\gamma_3(G)$. Let $h=x^nc,k=x^md\in H$
	where $c,d\in [G,G]$. 
	Since 
	\[
	[h,x^m]=[x^n,[c,x^m]][c,x^m]\in\gamma_4(G)\gamma_3(G)\subseteq\gamma_3(G),
	\]
	it follows that  
	\begin{align*}
		[h,k]&=[h,x^m][x^m,[h,d]][h,d]\\
			&=[x^n,[c,x^m]][c,x^m][x^m,[h,d]][h,d]\in\gamma_3(G).\qedhere
	\end{align*}
\end{proof}

\begin{example}
	Let $G=\D_{8}=\langle r,s:r^{8}=s^2=1,srs=r^{-1}\rangle$ be the dihedral group of order 
	16. Then $G$ is nilpotent of class three and 
	$[G,G]=\{1,r^2,r^4,r^6\}\simeq C_4$. The subgroup $\langle
	s,[G,G]\rangle\simeq\D_4$ is nilpotent of class two.
\end{example}

Now an application of Lemma~\ref{lem:a[GG]}. 

\begin{theorem}
	\label{thm:T(nilpotent)}
	If $G$ is a nilpotent group, then 
	\[
	T(G)=\{g\in G:g^n=1\text{ for some positive integer }n\}
	\]
	is a subgroup of $G$. 
\end{theorem}

\begin{proof} 
    We shall show the result by induction on the nilpotency class $c$ of $G$.
    If $G$ is abelian, then the result is clear. Suppose that $c\geq 2$ and that the result holds for all nilpotent 
    groups of class $<c$. 
	Let $a,b\in T(G)$ and let $A=\langle a,[G,G]\rangle$ and $B=\langle b,[G,G]\rangle$. 
	Since $A$ and $B$ are both nilpotent of class $<2$ by the previous lemma, the inductive hypothesis implies that
	$T(A)$ is a subgroup of $A$ and $T(B)$ is a subgroup of $B$. 
	Since $T(A)$ is characteristic in $A$ and $A$ is normal in $G$, it follows that 
	$T(A)$ is
	normal in $G$. Similarly, $T(B)$ is normal in $B$.  
	We claim that every element of $T(A)T(B)$ has finite order. Indeed, if 
	$x\in T(A)T(B)$, say $x=a_1b_1$ with 
	$a_1\in T(A)$ of order $m$ and $b_1\in T(B)$, then $x$ has finite order since  
	\begin{align*}
	x^m=(a_1b_1)^m&=
		(a_1b_1a_1^{-1})(a_1^2b_1a_1^{-2})\cdots (a_1^{m-1} b_1 a_1^{-m+1})b_1\in T(B).
	\end{align*}
	This trick allows us to prove that both $ab$ and $a^{-1}$ have finite order. 
	Hence $T(G)$ is a subgroup of $G$.  Therefore, the result follows by induction. 
\end{proof}

Another application of Lemma \ref{lem:a[GG]}.

\begin{theorem}
	\label{thm:a=b}
	Let $G$ be a torsion-free nilpotent group and let $a,b\in G$. If $a^n=b^n$ for some 
	$n\ne 0$, then $a=b$.
\end{theorem}

\begin{proof}
	We proceed by induction on the nilpotency class $c$ of $G$. The claim holds if $G$ is abelian. Assume 
	that $G$ is nilpotent of class $c\geq2$. Since $\langle a,[G,G]\rangle$ is a nilpotent subgroup of
	$G$ of nilpotency class $<c$ and $bab^{-1}=[b,a]a\in \langle
	a,[G,G]\rangle$, the inductive hypothesis implies that $ba=ab$, as  
	$a^n=(bab^{-1})^n=b^n$. Thus $(ab^{-1})^n=a^nb^{-n}=1$. Since $G$ is torsion-free, it follows that $a=b$.
\end{proof}

\begin{corollary}
	Let $G$ be a torsion-free nilpotent group. If $x,y\in G$ are such that 
	$x^ny^m=y^mx^n$ for some $n,m\ne 0$, then $xy=yx$.
\end{corollary}

\begin{proof}
	Let $a=x$ and $b=y^mxy^{-m}$. Since $a^n=b^n$, Theorem~\ref{thm:a=b} implies that $a=b$. Thus $xy^m=y^mx$. 
	By using Theorem~\ref{thm:a=b} now with $a=y$ and $b=xyx^{-1}$, we conclude that 
	$xy=yx$. 
\end{proof}

The following lemma is well-known. We include the proof for completeness. 

\begin{lemma}
	\label{lem:fg}
	Let $G$ be a finitely generated group and $H$ be a finite-index subgroup of $G$. 
	Then $H$ is finitely generated. 
\end{lemma}

\begin{proof}
	Assume that $G=\langle g_1,\dots,g_m\rangle$. Without loss of generality we may assume that 
	for each $i$ there exists $k$ such that $g_i^{-1}=g_k$. 
	Let $\{1=t_1,\dots,t_n\}$ be a right transversal of $H$ in $G$. For 
	$i\in\{1,\dots,n\}$ and 
	$j\in\{1,\dots,m\}$ we write 
	\[
		t_ig_j=h(i,j)t_{k(i,j)},
	\]
	where $h(i,j)\in H$.
	We claim that $H$ is generated by the $h(i,j)$. If $x\in H$, then 
	\begin{align*}
	x &=g_{i_1}\cdots g_{i_s}\\
	&= (t_1g_{i_1})g_{i_2}\cdots g_{i_s}\\
	&= h(1,i_1)t_{k_1}g_{i_2}\cdots g_{i_s}\\
	&= h(1,i_1)h(k_1,i_2)t_{k_2}g_{i_3}\cdots g_{i_s}\\
	&= h(1,i_1)h(k_1,i_2)\cdots h(k_{s-1},i_s)t_{k_s},
	\end{align*}
	where $k_1,\dots,k_{s-1}\in\{1,\dots,n\}$. Thus $t_{k_s}\in H$ and hence 
	$t_{k_s}=1$, which implies the claim.  
\end{proof}

\index{Order!of an element}
Let $G$ be a group. The {\em order} of an element $x\in G$ is the order of the group $\langle x\rangle$, and it is denoted by $|x|$. \index{Group!periodic}\index{Group!torsion} The group $G$ is said to be {\em torsion} or {\em periodic} if every element of $G$ has finite order.


\begin{theorem}
	\label{thm:T(G)finito}
	Let $G$ be a finitely generated torsion nilpotent group. Then
	$G$ is finite.  	
\end{theorem}

\begin{proof}
	We proceed by induction on the nilpotency class $c$. The case $c=1$ is true since $G$ is abelian. Assume 
	that $c\geq 2$ and the result holds groups of nilpotency class $<c$.  Since $G/[G,G]$ is an abelian finitely generated torsion group, it is finite. By Lemma \ref{lem:fg}, $[G,G]$ also is finitely generated. Since $[G,G]$ is a finitely generated torsion nilpotent group of class $<c$, the inductibe hypothesis implies that $[G,G]$ is finite. Thus $G$
	is finite. Therefore the result follows by induction.
\end{proof}

%\begin{exercise}
%Prove that in the previous theorem 
%, de hecho puede probarse que $G$ es de orden $|[G,G]|(G:[G,G])$.	
%\end{exercise}
%

%\begin{lemma}
%	\label{lemma:kgenerators}
%	Sea $G$ un grupo y sea $G=G_0\supseteq G_1\supseteq\cdots\supseteq G_k=1$
%	una sucesión de subgrupos de $G$ tal que cada $G_{i+1}$ es normal en $G_i$
%	y cada $G_{i}/G_{i+1}$ es cíclico. Todo subgrupo de $G$ es finitamente
%	generado por $k$ elementos.
%\end{lemma}
%
%\begin{proof}
%	Procedemos por inducción en $k$. Supongamos primero que $k=1$. Entonces
%	$G\simeq G_0/G_1$ es cíclico y luego todo subgrupo de $G$ está generado por
%	un elemento. Supongamos ahora que el resultado es válido para $k\geq1$. Sea
%	$H$ un subgrupo de $G$, sea $N=G_{1}$ y sea $\pi\colon G\to G/N$ el
%	morfismo canónico. El grupo 
%	\[
%		\pi(H)\simeq H/H\cap N
%	\]
%	es cíclico pues un un subgrupo del grupo cíclico $G_k/G_{k-1}=G/N$. Como
%	existe $h\in H$ tal que $\pi(H)$ está generado por $\pi(h)$, se concluye que 
%	$H=\langle \pi(h),H\cap N\rangle$. Por hipótesis
%	inductiva, $H\cap N$ está generado por $k-1$ elementos y luego $H$ está
%	generado por $k$ elementos.
%\end{proof}
%
%\begin{theorem}
%	Sea $G$ un grupo nilpotente y finitamente generado. Entonces $T(G)$ es
%	finito.
%\end{theorem}
%
%%%% aca hay que hacer producto tensorial para construir una serie con factores cíclicos
%%%% ver libro de Khukhro
%%%% Nilpotent Groups and Their Automorphisms
%\begin{proof}
%	Sabemos por el teorema~\ref{theorem:} que existe una sucesión
%	$G=G_0\supseteq G_1\supseteq\cdots\supseteq G_k=G$ de subgrupos normales de
%	$G$ con factores cíclicos. 
%\end{proof}

\section*{Finite nilpotent groups}

%We now turn out attention to finite nilpotent groups. 
Let $G$ be a finite group and $p$ a prime. We denote by $\Syl_p(G)$ the set of the Sylow $p$-subgroups of $G$.

\begin{lemma}
	\label{lemma:N_G(H)=H}
	Let $G$ be a finite group and let $P\in\Syl_p(G)$. If $H$ is a subgroup of $G$
	such that $N_G(P)\subseteq H$, then $N_G(H)=H$.
\end{lemma}

\begin{proof}
	Let $x\in N_G(H)$. Since $P\in\Syl_p(H)$ and $xPx^{-1}\subseteq xHx^{-1}=H$, we have that there exists
	$h\in H$ such that $(hx)P(hx)^{-1}=P$. Hence $hx\in
	N_G(P)\subseteq H$, and thus $x\in H$. This proves the result. 
\end{proof}

%\begin{lemma}
%	\label{lem:normalizador}
%	Let $G$ be a finite group, $p$ a prime divisor of $|G|$ and 
%	$P\in\Syl_p(G)$. Then
%	$N_G(N_G(P))=N_G(P)$. 
%\end{lemma}
%
%\begin{proof}
%	Let $H=N_G(P)$. Since $P$ is normal in $H$, $P$ is the unique Sylow 
%	$p$-subgroup of $H$. To prove that $N_G(H)=H$ it is enough to show that $N_G(H)\subseteq
%	H$. Let $g\in N_G(H)$. Since  
%	$gPg^{-1}\subseteq gHg^{-1}=H$, 
%	both 
%	$P, gPg^{-1}\in\Syl_p(H)$ and $H$ have a unique Sylow $p$-subgroup, so   
%	$P=gPg^{-1}$. Thus $g\in N_G(P)=H$. 
%\end{proof}

\begin{lemma}\label{pgroup}
	Let $p$ be a prime number. Then $p$-groups are nilpotent.
\end{lemma}

\begin{proof}
    We know that every non-trivial $p$-group has non-trivial center. Let $G$ be a $p$-group. Suppose that $|G|=p^n$. We shall prove the result by induction on $n$. For $n=1$, $G$ is cyclic of order $p$ and the result is trivial. Suppose that $n>1$ and that the result holds for $p$-groups of order $<p^n$. We may assume that $G$ is non-abelian. Since $Z(G)$ is a non-trivial proper subgroup of $G$, by the inductive hypothesis, $Z(G)$ and $G/Z(G)$ are nilpotent. Let $\pi\colon G\rightarrow G/Z(G)$ be the canonical map. By Lemma \ref{lem:gamma_zeta}, there exists $c$ such that $\zeta_c(G/Z(G))=G/Z(G)$. Note that
    \[ \pi^{-1}(\zeta_i(G/Z(G)))=\zeta_{i+1}(G).\]
    Hence $\zeta_{c+1}(G)=G$ is nilpotent by Lemma \ref{lem:gamma_zeta}. Therefore the result follows by induction. 
\end{proof}

\index{Direct product of subgroups}
We say that a group $G$ is the {\em direct product} of its subgroups $H_1,\dots ,H_n$ if every subgroup $H_i$ is normal in $G$, $G=H_1\cdots H_n$ and
$H_{k+1}\cap H_1\cdots H_{k}=\{ 1\}$, for all $k=1,\dots, n-1$. Note that, for $i<j$, $h_i\in H_i$ and $h_j\in H_j$, we have that $[h_i,h_j]\in H_i\cap H_j=\{ 1\}$. Note that, in this case the map $f\colon H_1\times\dots\times H_n\rightarrow G$, defined by $f(h_1,\dots ,h_n)=h_1\cdots h_n$, for all $h_i\in H_i$, is an isomorphism of groups. It is clear that $f$ is surjective. Let $(a_1,\dots,a_n),(b_1,\dots,b_n)\in H_1\times\dots\times H_n$. We have that
\begin{align*}
f((a_1,\dots,a_n)(b_1,\dots,b_n))&=f(a_1b_1,\dots,a_nb_n)\\
&=a_1b_1\cdots a_nb_n\\
&=a_1\cdots a_nb_1\cdots b_n\\
&=f(a_1,\dots ,a_n)f(b_1,\dots ,b_n).
\end{align*}
Thus $f$ is a surjective homomorphism of groups. Now it is easy to see that $\ker(f)=\{ 1\}$. 


\begin{theorem}
	\label{thm:nilpotente:eq}
	Let $G$ be a finite group. The following conditions are equivalent: 
	\begin{enumerate}
		\item $G$ is nilpotent.
		\item Every Sylow subgroup of $G$ is normal.
		\item $G$ is the direct product of its Sylow subgroups. 
	\end{enumerate}
\end{theorem}

\begin{proof}
	We prove that $1)\implies 2)$. Suppose that $G$ is nilpotent. Let $P\in\Syl_p(G)$.  By Lemma~\ref{lemma:N_G(H)=H},  
	$N_G(N_G(P))=N_G(P)$. Now  $N_G(P)=G$ by the normalizer condition. Hence $P$ is normal in $G$.

	Now we prove that $2)\implies 3)$. Let $p_1,\dots,p_n$ be the distinct prime divisors of $|G|$. For every $i\in\{1,\dots,n\}$, let $P_i\in\Syl_{p_i}(G)$.
	By hypothesis, each $P_j$ is normal in $G$. Since $|P_1\cdots P_n|=|G|$, we have that $G=P_1\cdots P_n$. Clearly
	$|P_{k+1}|, |P_1\cdots P_k|$ are coprime for all $k=1,\dots, n-1$. Hence $P_{k+1}\cap P_1\cdots P_k=\{ 1\}$, for all $k=1,\dots, n-1$. Thus $G$ is the direct product of its Sylow subgroups. 

	Finally, to prove that $3)\implies 1)$ we note that every 
	$p$-group is nilpotent and that a finite direct product of nilpotent groups is nilpotent. 
\end{proof}


\begin{theorem}[Baumslag--Wiegold]
	\index{Baumslag--Wiegold's theorem}
	Let $G$ be a finite group such that $|xy|=|x||y|$ whenever $x,y\in G$ have coprime order. Then
	$G$ is nilpotent. 
\end{theorem}

\begin{proof}
	Let $p_1,\dots,p_n$ be the distinct prime divisors of the order of $G$. For each 
	$i\in\{1,\dots,n\}$ let $P_i\in\Syl_{p_i}(G)$. We claim that 
	$G=P_1\cdots P_n$. The non-trivial inclusion is equivalent to show that the map
	\[
		\psi\colon P_1\times\cdots\times P_n\to G,\quad
		(x_1,\dots,x_n)\mapsto x_1\cdots x_n
	\]
	is surjective. We first show that $\psi$ is injective. If 
	$\psi(x_1,\dots,x_n)=\psi(y_1,\dots,y_n)$, then  
	\[
		x_1\cdots
	x_n=y_1\cdots y_n. 
	\]
	If $y_n\ne x_n$, then $x_1\cdots x_{n-1}=(y_1\cdots
	y_{n-1})y_nx_n^{-1}$. But $x_1\cdots x_{n-1}$ has order coprime with 
	$p_n$ and $y_1\cdots y_{n-1}y_nx_n^{-1}$ has order divisible by $p_n$, a contradiction. 
	Thus $x_n=y_n$ and hence the same argument proves that $\psi$ is injective. 
	Since $|P_1\times\cdots\times
	P_n|=|G|$, we conclude that $\psi$ is bijective. In particular, $\psi$ is surjective and hence 
	$G=P_1\cdots P_n$.

	We now show that each $P_j$ is normal in $G$.  Let $j\in\{1,\dots,n\}$ and 
	$x_j\in P_j$. Let $g\in G$ and $y_j=gx_jg^{-1}$.  Since $y_j\in G$,
	we write $y_j=z_1\cdots z_n$ with $z_k\in P_k$ for all $k$. Since the order of 
	$y_j$ is a power of $p_j$, it follows that $z_1\cdots
	z_n$ has order a power of $p_j$. Thus $z_k=1$ for all $k\ne j$ and 
	$y_j=z_j\in P_j$. Since each Sylow subgroup is normal, $G$ is nilpotent, by Theorem \ref{thm:nilpotente:eq}. 
\end{proof}


%\section{Grupos nilpotentes de clase dos}
%We conclude the section with an exercise
%on nilpotent groups of class two. 





%
%\begin{lemma}
%	\label{lemma:commutador}
%	Si $x,y\in G$ son tales que $[x,y]\in C_G(x)\cap C_G(y)$, entonces
%	\[
%	[x,y]^n=[x^n,y]=[x,y^n]
%	\]
%	para todo $n\in\Z$.
%\end{lemma}
%
%\begin{proof}
%	Procederemos por inducción en $n\geq0$. El caso $n=0$ es trivial. Supongamos entonces
%	que el resultado vale para algún $n\geq0$. Entonces, como $[x,y]\in C_G(x)$, 
%	\begin{align*}
%		[x,y]^{n+1}&=[x,y]^n[x,y]
%		=[x^n,y][x,y]=[x^n,y]xyx^{-1}y^{-1}=x[x^n,y]yx^{-1}y^{-1}=[x^{n+1},y].
%	\end{align*}
%	Para demostrar el lema en el caso $n<0$ basta observar que, como $[x,y]^{-1}=[x^{-1},y]$, 
%	$[x,y]^{-n}=[x^{-1},y]^n=[x^{-n},y]$.
%\end{proof}

%\begin{lemma}[Hall]
%    \index{Lema!de Hall}
%	\label{lemma:Hall}
%	Sea $G$ un grupo nilpotente de clase dos y $x,y\in G$. Entonces
%	\[
%		(xy)^n=[y,x]^{n(n-1)/2}x^ny^n
%	\]
%	para todo $n\in\N$.
%\end{lemma}
%
%\begin{proof}
%	Procederemos por inducción en $n$. Como el caso $n=1$ es trivial,
%	supongamos que el resultado es válido para algún $n\geq1$. Entonces,
%	gracias al lema anterior, 
%	\begin{align*}
%		(xy)^{n+1} &= (xy)^n(xy)=[y,x]^{n(n-1)/2}x^ny^{n-1}(yx)y\\
%		&=[y,x]^{n(n-1)/2}x^{n}[y^n,x]xy^{n+1}=[y,x]^{n(n-1)/2}[y,x]^nx^{n+1}y^{n+1}.\qedhere 
%	\end{align*}
%\end{proof}

%\begin{lemma}
%	\label{lemma:class2}
%	Sea $p>2$ un número primo y sea 
%	$P$ un $p$-grupo de clase de nilpotencia $\leq2$. 
%	Si $[y,x]^p=1$ para todo $x,y\in P$ entonces $P\to [P,P]$,
%	$x\mapsto x^p$, es un morfismo de grupos.
%\end{lemma}
%
%\begin{proof}
%	Por lema de Hall,
%	$(xy)^p=[y,x]^{p(p-1)/2}x^py^p=x^py^p$. 	
%\end{proof}
%
%\begin{theorem}
%	\label{thm:class2}
%	Sea $p>2$ un número primo y sea 
%	$P$ un $p$-grupo de clase de nilpotencia $\leq2$. 
%	Entonces $\{x\in P:x^p=1\}$ es un subgrupo de $P$.
%\end{theorem}
%
%\begin{proof}
%	Como $P$ tiene clase de nilpotencia dos, los conmutadores son centrales.
%	Para cada $x\in G$, la función $g\mapsto [g,x]$ es un morfismo de grupos
%	pues
%	\[
%		[gh,x]=ghxh^{-1}g^{-1}x^{-1}=g[h,x]xg^{-1}x^{-1}=[g,x][h,x].
%	\]
%	En particular, si $x,y\in P$ con $x^p=y^p=1$, entonces
%	\[
%		[x,y]^p=[x^p,y]=[1,y]=1.
%	\]
%	Luego, al usar el lema de Hall, se concluye que
%	$(xy)^p=[y,x]^{p(p-1)/2}x^py^p=1$.
%\end{proof}

\section*{Fratini subgroup}

\index{Frattini!subgroup}
Let $G$ be a group. If $G$ has maximal subgroups, the {\em Frattini subgroup}
$\Phi(G)$ of $G$ is defined as the intersection of all maximal subgroups of $G$. 
Otherwise, $\Phi(G)=G$. 

Note that $\Phi(G)$ is a characteristic subgroup of $G$.


\begin{example}
	Let $G=\Sym_3$. The maximal subgroups of $G$ are 
	\[
	M_1=\langle (123)\rangle,
	\quad
	M_2=\langle (12)\rangle,
	\quad
	M_3=\langle (23)\rangle,
	\quad
	M_4=\langle (13)\rangle.
	\]
	Thus 
	$\Phi(G)=\{1\}$. 
\end{example}

\begin{example}
	Let $G=\langle g\rangle\simeq C_{12}$. The subgroups of $G$ are  
	\[
	\{1\},\quad
	\langle g^6\rangle\simeq C_2,\quad
	\langle g^4\rangle\simeq C_3,\quad
	\langle g^3\rangle\simeq C_4,\quad
	\langle g^2\rangle\simeq C_6,\quad
	G.
	\]
	The maximal subgroups are $\langle g^3\rangle\simeq C_4$ and $\langle
	g^2\rangle\simeq C_6$. Hence \[
	\Phi(G)=\langle g^3\rangle\cap \langle
	g^2\rangle=\langle g^6\rangle\simeq C_2.
	\] 
\[\begin{tikzcd}
	& {C_{12}} \\
	{C_4} && {C_6} \\
	& {C_2} && {C_3} \\
	&& {\{1\}}
	\arrow[no head, from=4-3, to=3-2]
	\arrow[no head, from=4-3, to=3-4]
	\arrow[no head, from=3-4, to=2-3]
	\arrow[no head, from=2-3, to=1-2]
	\arrow[no head, from=3-2, to=2-1]
	\arrow[no head, from=2-1, to=1-2]
	\arrow[no head, from=3-2, to=2-3]
\end{tikzcd}\]
\end{example}


\begin{lemma}[Dedekind]
	\label{lem:Dedekind}
	\index{Dedekind!lemma}
	Let $H$, $K$ and $L$ be subgroups of $G$ such that $H\subseteq L\subseteq G$. Then
	$HK\cap L=H(K\cap L)$.
\end{lemma}

\begin{proof}
	We only need to prove that $HK\cap L\subseteq H(K\cap L)$, 
	as the other inclusion is trivial. If 
	$x=hk\in HK\cap L$, where $x\in L$, $h\in H$ and $k\in K$,
	then $k=h^{-1}x\in L\cap K$ since $H\subseteq L$. Thus $x=hk\in H(L\cap
	K)$.
\end{proof}

\begin{lemma}
	\label{lem:G=HPhi(G)}
	Let $G$ be a finite group. If $H$ is a subgroup of $G$ such that $G=H\Phi(G)$,
	then $H=G$.
\end{lemma}

\begin{proof}
	Assume that $H\ne G$. Let $M$ be a maximal subgroup of $G$ such that
	$H\subseteq M$. Since $\Phi(G)\subseteq M$, it follows that $G=H\Phi(G)\subseteq M$, a contradiction
\end{proof}

\begin{proposition}
	\label{pro:phi(N)phi(G)}
	Let $G$ be a finite group and $N$ be a normal subgroup of $G$. Then $\Phi(N)\subseteq\Phi(G)$.
\end{proposition}

\begin{proof}
	Since $\Phi(N)$ is characteristic in $N$ and $N$ is normal in $G$, it follows that 
	$\Phi(N)$
	is normal in $G$.  Let $M$ be a maximal subgroup of $G$ such that 
	$\Phi(N)\not\subseteq M$.  Then $\Phi(N)M=G$, as 
	$M=\Phi(N)M\supseteq\Phi(N)$ otherwise. By applying Lemma \ref{lem:Dedekind} with 
	$H=\Phi(N)$, $K=M$ and $L=N$,   
	\[
		N=G\cap N=(\Phi(N)M)\cap N=\Phi(N)(M\cap N).
	\]
	By Lemma \ref{lem:G=HPhi(G)} $N=M\cap N\subseteq M$, a contradiction. Hence every maximal subgroup of $G$ 
	contains $\Phi(N)$ and therefore $\Phi(G)\supseteq\Phi(N)$. 
\end{proof}

The Frattini subgroup can be characterized in terms of non-generators.  

\begin{lemma}[non-generators]
	\label{lemma:nongenerators}
	\index{Lema!de los no-generadores}
	Let $G$ be a finite group. Then 
	\[
	\Phi(G)=\{x\in G:\text{if $G=\langle x,Y\rangle$ for some $Y\subseteq G$, then $G=\langle Y\rangle$}\}.
	\]
\end{lemma}

\begin{proof}
	Let $x\in\Phi(G)$ be such that $G=\langle
	x,Y\rangle$ for some subset $Y$ of $G$. If $G\ne \langle Y\rangle$, then 
	there exists a maximal subgroup $M$ such that $\langle Y\rangle\subseteq M$. Since 
	$x\in M$, $G=\langle x,Y\rangle\subseteq M$, a contradiction.
	
	Conversely, 
	let $x\in G$ and $M$ be a maximal subgroup of $G$. If $x\not\in M$, then, since $G=\langle
	x,M\rangle$, it follows that $G=\langle M\rangle=M$, a contradiction. Thus $x\in M$ for all
	maximal subgroup $M$. Hence $x\in \Phi(G)$. 
\end{proof}

\begin{lemma}[Frattini's argument]
\index{Frattini's argument}
Let $G$ be a finite group, $H$ a normal subgroup of $G$ and $p$ a prime. Then, for every $P\in \Syl_p(H)$, $G=HN_G(P)$.
\end{lemma}

\begin{proof}
    Let $P\in\Syl_p(H)$ and $g\in G$. We have that $gPg^{-1}\subseteq gHg^{-1}=H$. Hence $gPg^{-1}\in\Syl_p(H)$.
    Therefore, there exists $h\in H$ such that $hPh^{-1}=gPg^{-1}$. Thus $h^{-1}g\in N_G(P)$. Now we have that
    $g=hh^{-1}g\in HN_G(P)$, and the result follows.
\end{proof}

\begin{theorem}[Frattini]
	\label{theorem:Frattini}
	\index{Frattini's theorem}
	If $G$ is a finite group, then $\Phi(G)$ is nilpotent. 
\end{theorem}

\begin{proof}
	Let $P\in\Syl_p(\Phi(G))$ for some prime $p$. Since $\Phi(G)$ is normal in
	$G$, Frattini's argument implies that 
	% (que vimos en el lema~\ref{lemma:Frattini_argument}) 
	$G=\Phi(G)N_G(P)$. By Lemma~\ref{lem:G=HPhi(G)},
	$G=N_G(P)$. Since Sylow subgroups of $\Phi(G)$ are normal in $G$,
	$\Phi(G)$ is nilpotent, by Theorem \ref{thm:nilpotente:eq}.
\end{proof}

%\begin{exercise}
%	\label{exercise:G/M}
%	Let $G$ be a group and $M$ be a normal subgroup of $G$ that is maximal. Then
%	$G/M$ is cyclic of prime order.
%\end{exercise}

% \begin{svgraybox}
% 	Por el teorema de la correspondencia, $G/M$ no tiene subgrupos no trivales.
% 	Luego $G/M\simeq C_p$ para algún primo $p$.
% \end{svgraybox}

\begin{theorem}[Gasch\"utz]
	\label{thm:Gaschutz}
	\index{Gasch\"utz's theorem}
	If $G$ is a finite group, then $[G,G]\cap Z(G)\subseteq\Phi(G)$.
\end{theorem}

\begin{proof}
	Let $D=[G,G]\cap Z(G)$. Assume that $D$ is not contained in $\Phi(G)$.
	Hence there exists a maximal subgroup $M$ 
	such that $D$ is not contained in $M$. Thus $G=MD$. Since $D\subseteq Z(G)$, it is clear that $M$ is normal in $G$.
%	, since if  
%	$g=md\in G=MD$, then 
%	\[
%		gMg^{-1}=(md)Md^{-1}m^{-1}=mMm^{-1}=M.
%	\]
	Since $G/M\cong D/(D\cap M)$ is abelian,  $[G,G]\subseteq M$ and hence 
	$D\subseteq [G,G]\subseteq M$, a contradiction. Therefore, the result follows.
\end{proof}


\begin{theorem}[Wielandt]
	\label{theorem:Wielandt}
	\index{Wielandt's!theorem}
	Let $G$ be a finite group. Then $G$ is nilpotent if and only if
	$[G,G]\subseteq\Phi(G)$.
\end{theorem}

\begin{proof}
	Suppose that $[G,G]\subseteq\Phi(G)$. Let $P\in\Syl_p(G)$. If $N_G(P)\ne
	G$, then $N_G(P)\subseteq M$ for some maximal subgroup $M$ of $G$. Let
	$g\in G$ and $m\in M$. Since 
	\[
		gmg^{-1}m^{-1}=[g,m]\in [G,G]\subseteq\Phi(G)\subseteq M,
	\]
	we get that $gmg^{-1}\in M$. Hence $M$ is normal in $G$, that is $G=N_G(M)$. But, since $N_G(P)\subseteq M$, by
	lemma~\ref{lemma:N_G(H)=H}, $N_G(M)=M$, a contradiction. Therefore $P$ is normal in $G$, and thus $G$ is nilpotent, by Theorem \ref{thm:nilpotente:eq}.
	
	The converse follows by Theorem \ref{theorem:minmax_nilpotent}.
\end{proof}

\begin{theorem}
	\label{theorem:G/phi(G)}
	Let $G$ be a finite group. Then $G$ is nilpotent if and only if	$G/\Phi(G)$ is nilpotent.
\end{theorem}

%%% TODO: la demostración no está bien explicada!
\begin{proof}
	If $G$ is nilpotent, then clearly $G/\Phi(G)$ also is nilpotent. Suppose that $G/\Phi(G)$ is
	nilpotent. Let $P\in\Syl_p(G)$. Since
	$\Phi(G)P/\Phi(G)\in\Syl_p(G/\Phi(G))$ and $G/\Phi(G)$ is nilpotent,
	$\Phi(G)P/\Phi(G)$ is a normal subgroup of $G/\Phi(G)$. Hence,  $\Phi(G)P$ is  normal in $G$.
	By Frattini's argument,  
	\[
		G=\Phi(G)PN_G(P)=\Phi(G)N_G(P).
	\]
	By Lemma \ref{lem:G=HPhi(G)}),  $G=N_G(P)$, and thus $P$ is normal in $G$. By Theorem \ref{thm:nilpotente:eq}, $G$ is nilpotent.
\end{proof}

\begin{theorem}[Hall]
	\index{Hall's!theorem}
	\label{theorem:Hall_nilpotente}
	Let $G$ be a finite group and $N$ a normal subgroup of $G$. If $N$ and
	$G/[N,N]$ are nilpotent, then $G$ is nilpotent.
\end{theorem}

\begin{proof}
	Since $N$ is nilpotent, by Theorem~\ref{theorem:Wielandt}, $[N,N]\subseteq\Phi(N)$.
	By Proposition~\ref{pro:phi(N)phi(G)},
	$[N,N]\subseteq\Phi(N)\subseteq\Phi(G)$. 
	Hence, there exists a surjective homomorphism
	$G/[N,N]\to G/\Phi(G)$. Since $G/[N,N]$ is nilpotent, we get that $G/\Phi(G)$ is nilpotent.
%    \[
%    \begin{tikzcd}
%	G & {G/\Phi(G)} \\
%	{G/[N,N]}
%	\arrow[from=1-1, to=1-2]
%	\arrow[from=1-1, to=2-1]
%	\arrow[dashed, from=2-1, to=1-2]
%   \end{tikzcd}
%    \]
%     \[
% 	\xymatrix{
% 	G
% 	\ar[d]
% 	\ar[r]
% 	& G/\Phi(G)
% 	\\
% 	G/[N,N]\ar@{-->}[ur]
% 	}
%    \xymatrix{ & P\ar[d]^f\ar@{-->}[ld]_h\\ M\ar[r]^g & N\ar[r] & 0 }
%    \]
Hence $G$ is nilpotent by Theorem~\ref{theorem:G/phi(G)}.
\end{proof}

\begin{definition}
	A {\em minimal generating set} of a group $G$ is a subset $X$ of $G$ such that $G=\langle X\rangle$ and $G\neq \langle Y\rangle$ for every proper subset $Y$ of $X$.
\end{definition}

Note that the minimal number of generators of a group $G$ can be less than the number of element of a minimal generating set of $G$.
	
\begin{example}
	Let $G=\langle g\mid g^6=1\rangle$.  If $a=g^2$ and
	$b=g^3$, then $\{a,b\}$ is a minimal generating set of $G$,
	but the minimal number of generators of $G$ is $1$.
\end{example}
	
\begin{lemma}
	\label{lemma:Burnside:minimal}
	Let $p$ be a prime and 
	$G$ a $p$-group. Then $G/\Phi(G)$ is a vector space over the field $\F_p$ of $p$ elements.
\end{lemma}

\begin{proof}
	Let $K$ be a maximal subgroup of $G$. Since $G$ is nilpotent, by Theorem~\ref{theorem:minmax_nilpotent}, 
	$K$ is normal in $G$ of index $p$. Hence $G/K$ is a cyclic group of order $p$. 
	
    Let $K_1,\dots,K_m$ be the distinct maximal subgroups of $G$.
    Let $f\colon G\rightarrow G/K_1\times\cdots\times G/K_m$ be the map defined by $f(g)=(gK_1,\dots ,gK_m)$, for all $g\in G$.
    It is clear that $f$ is a homomorphism of groups and $\ker(f)=\Phi(G)$. Hence $G/\Phi(G)$ is isomorphic to $\im(f)$, a subgroup of 
    the $\F_p$-vector space $G/K_1\times\cdots\times G/K_m$. Therefore, the result follows.
\end{proof}

\begin{theorem}[Burnside]
	\label{theorem:Burnside:basis}
	Let $p$ be a prime and  $G$ a $p$-group. If $X$ is a minimal generating set of $G$, then $|X|=\dim_{\F_p} (G/\Phi(G))$. 
\end{theorem}

%%% TODO: explicar mejor la demostración

\begin{proof}
	By the previous lemma, we know that $G/\Phi(G)$ is an $\F_p$-vector space. Let $\pi\colon G\to G/\Phi(G)$ be the canonical map
	and $\{x_1,\dots,x_n\}$ a minimal generating set of $G$, with $|\{x_1,\dots,x_n\}|=n$.
	We shall see that $\pi(x_1),\dots,\pi(x_n)$ are linearly independent in $G/\Phi(G)$.  Suppose that
	$\pi(x_1),\dots,\pi(x_n)$ are not linearly independent. Without loss of generality, we may assume that
	$\pi(x_1)\in\langle \pi(x_2),\dots,\pi(x_n)\rangle$. Then there exists $y\in
	\langle x_2,\dots,x_n\rangle$ such that $x_1y^{-1}\in\Phi(G)$. Since $G=\langle x_1y^{-1},x_2,\dots,x_n\rangle$ and $x_1y^{-1}\in\Phi(G)$, by lemma~\ref{lemma:nongenerators}, we have that $G=\langle x_2,\dots, x_n\rangle$, a contradiction.
	Therefore $n=\dim_{\F_p}(G/\Phi(G))$.
\end{proof}

% TODO: agregar una aplicación (teorema de Hall). Ver Passman permutation groups, 11.7, pag 47

%\begin{corollary}
%	Sea $p$ un número primo y sea $G$ un $p$-grupo finito.  Todo elemento de
%	$\Phi(G)$ pertenece a algún conjunto minimal de generadores.
%\end{corollary}

\section*{Fitting subgroup}

\begin{definition}
	\index{$p$-radical!group}
	Let $G$ be a finite group and $p$ a prime. The
	{\em $p$-radical} of $G$ is the subgroup
	\[
		O_p(G)=\bigcap_{P\in\Syl_p(G)}P.
	\]
\end{definition}

The following result is an easy consequence of Sylow's theorem.  
\begin{lemma}
	\label{lemma:core:Op(G)}
	Let $G$ be a finite group and $p$ a prime. 
	\begin{enumerate}
		\item $O_p(G)$ is normal in $G$.
		\item If $N$ is a normal $p$-subgroup of $G$, then $N\subseteq O_p(G)$.
	\end{enumerate}
\end{lemma}

%\begin{proof}
%	Let $P\in\Syl_p(G)$ y hagamos actuar a $G$ en $G/P$ por multiplicación a
%	izquierda. Tenemos entonces un morfismo $\rho\colon G\to\Sym_{G/P}$ con
%	núcleo
%	\begin{align*}
%		\ker\rho&=\{x\in G:\rho_x=\id\}
%		=\{x\in G:xgP=gP\;\forall g\in G\}\\
%		&=\{x\in G:x\in gPg^{-1}\;\forall g\in G\}=\bigcap_{g\in G}gPg^{-1}=O_p(G).
%	\end{align*}
%	Luego $O_p(G)$ es normal en $G$.
%
%	Sea ahora $N$ un subgrupo normal de $G$ tal que $N\subseteq P$. Como para
%	todo $g\in G$ se tiene $N=gNg^{-1}\subseteq gPg^{-1}$, se concluye que
%	$N\subseteq O_p(G)$.
%\end{proof}

\begin{definition}
	\index{Fitting's! subgroup}
	Let $G$ be a finite group and let $p_1,\dots,p_k$ be the distinct prime divisors of	$|G|$.  
	The {\em Fitting's subgroup} of $G$ is the subgroup
	\[
		F(G)=O_{p_1}(G)\cdots O_{p_k}(G)
	\]
\end{definition}

Note that $F(G)$ is a characteristic subgroup of $G$.

% \begin{svgraybox}
% 	Sea $f\in\Aut(G)$ y sea $p$ un primo. Como $f$ 
% 	permuta los $p$-subgrupos de Sylow de $G$, $f(O_p(G))=O_p(G)$. Luego
% 	$f(F(G))=F(G)$.
% \end{svgraybox}


\begin{example}
	Let $G=\Sym_3$. It is easy to see that $O_2(G)=\{1\}$ and $O_3(G)=\langle
	(123)\rangle$. Thus $F(G)=\langle (123)\rangle$.
\end{example}

\begin{theorem}[Fitting]
	\label{theorem:Fitting}
	\index{Fitting's theorem}
	Let $G$ be a finite group. The Fitting subgroup $F(G)$ is nilpotent and it is normal in $G$. 
	Furthermore $F(G)$ contains all normal nilpotent subgroups of
	$G$.
\end{theorem}

\begin{proof}
	Note that for every prime $p$, $O_p(G)$ is a Sylow $p$-subgroup of $F(G)$. Since $O_p(G)$ is normal in $G$, also is normal in $F(G)$. Hence by Theorem~\ref{thm:nilpotente:eq}, $F(G)$ is
	nilpotent. 

	Let $N$ be a normal nilpotent subgroup of $G$ and let $P\in\Syl_p(N)$. Since
	$N$ is nilpotent, $P$ is normal in $N$. Then $P$ is the unique
	Sylow $p$-subgrup of $N$. Hence $P$ is a characteristic subgroup of $N$ and thus
	$P$ is normal in $G$. By Lemma~\ref{lemma:core:Op(G)}, $P\subseteq O_p(G)$. By Theorem~\ref{thm:nilpotente:eq}, $N$ is the direct product of its Sylow subgroups. Hence $N\subseteq F(G)$.
\end{proof}

\begin{corollary}
	\label{corollary:Z(G)subsetF(G)}
	Let $G$ be a finite group. Then $Z(G)\subseteq F(G)$.
\end{corollary}

\begin{proof}
	Since $Z(G)$ is a normal nilpotent subgroup of $G$, by Fitting's theorem,
	$Z(G)\subseteq F(G)$. 
	%~\ref{theorem:Fitting}.
\end{proof}

\begin{corollary}[Fitting]
	\label{corollary:Fitting}
	Let $K$ and $L$ be normal nilpotent subgroups of a finite group $G$.
	Then $KL$ is nilpotent.
\end{corollary}

\begin{proof}
	By Fitting's theorem, $K$ and $L$ are subgroups of $F(G)$. Hence $KL$ also is a subgroup of $F(G)$. By Fitting's theorem, 
	$F(G)$ is nilpotent. Therefore  $KL$ also is nilpotent.
\end{proof}

\begin{corollary}
	\label{corollary:McapF(G)}
	Let $G$ be a finite group and $N$ a normal subgroup of $G$. Then
	$N\cap F(G)=F(N)$.
\end{corollary}

\begin{proof}
	Since $F(N)$ is a characteristic subgroup of $N$, $F(N)$ is normal in $G$. By Fitting's theorem, $F(N)$ is nilpotent and
	$F(N)\subseteq N\cap F(G)$. 
	On the other hand, since 
	$F(G)$ is normal in $G$, we have that $F(G)\cap N$ is normal in $N$. By Fitting's theorem $F(G)$ is nilpotent and thus $F(G)\cap N$
	also is nilpotent. Again by Fitting's theorem, $F(G)\cap N\subseteq F(N)$. Therefore the result follows. 
\end{proof}


\begin{theorem}
	Let $G$ be a finite group. Then
    $\Phi(G)\subseteq F(G)$ and $F(G)/\Phi(G)\cong F(G/\Phi(G))$.
\end{theorem}

\begin{proof}
	By Theorem~\ref{theorem:Frattini}, $\Phi(G)$ is a normal nilpotent subgroup of $G$. Hence, by Fitting's theorem,
	$\Phi(G)\subseteq F(G)$.

    Let $\pi\colon G\to G/\Phi(G)$ be the canonical map. By Fitting's theorem $F(G)$ is nilpotent, thus $\pi(F(G))$ also is 
    nilpotent. Since $\Phi(G)\subseteq F(G)$ and $F(G)$ is normal in $G$, we have that $\pi(F(G))=F(G)/\Phi(G)$ 
    is normal in $G/\Phi(G)$. By Fitting's theorem,
	\[
	\pi(F(G))\subseteq F(G/\Phi(G)).
	\]
    Let	$H=\pi^{-1}(F(G/\Phi(G)))$. We know that $H$ i a normal  subgroup
	of $G$ such that $\Phi(G)\subseteq F(G)\subseteq H$. Let $P\in\Syl_p(H)$. Then
	$\pi(P)=(P\Phi(G))/\Phi(G)\cong P/P\cap \Phi(G)$ is a
	$p$-subgroup of $\pi(H)=F(G/\Phi(G))$. Since 
	\[
	(\pi(H):\pi(P))
	=\frac{|\pi(H)|}{|\pi(P)|}
	=\frac{|H/\Phi(G)|}{|P/P\cap \Phi(G)|}
	=\frac{(H:P)}{(\Phi(G):P\cap\Phi(G))}
	\]
	is a divisor of $(H:P)$, we have that $\pi(P)\in \Syl_p(\pi(H))$. By Fitting's theorem $\pi(H)$ is nilpotent.
	Hence $\pi(P)$ is a characteristic subgroup of $\pi(H)$, and thus  $P\Phi(G)=\pi^{-1}(\pi(P))$ is
	normal in $G$. Since $P\in\Syl_p(P\Phi(G))$, by Frattini's argument, $G=P\Phi(G)N_G(P)=\Phi(G)N_G(P)$. By Lemma~\ref{lem:G=HPhi(G)}, $G=N_G(P)$, and thus $P$ is normal in $G$. Hence $P$ is normal in $H$. By Theorem \ref{thm:nilpotente:eq}, $H$ is nilpotent. By Fitting's theorem, $H\subseteq F(G)$. Hence
	$F(G/\Phi(G))=\pi(H)\subseteq \pi(F(G))$, and the result follows.
\end{proof}

%\begin{exercise}
%	Sea $G$ un grupo finito. Demuestre que
%	$F(G)/Z(G)\simeq F(G/Z(G))$.
%\end{exercise}
%
%\begin{svgraybox}
%	Sea $\pi\colon G\to G/Z(G)$ el morfismo canónico. 
%	Como $Z(G)$ es abeliano, $\pi(Z(G))$ es nilpotente y luego $\pi(Z(G))\subseteq F(G/Z(G))$. 
%\end{svgraybox}<++>

\section*{Exercises}
% Frattini para braces
% Qué podemos decir de Fitting?


\begin{prob}
	Prove that a group is nilpotent if and only if it admits a central series.
\end{prob}

\begin{prob}
\label{xca:nilpotente_central}
	Let $G$ be a group. Prove that if $K$ is a central subgroup of $G$ such that 
	$G/K$ is nilpotent, then $G$ is nilpotent.
\end{prob}

\begin{prob}
\label{xca:nilpotente_minimalnormal}
	Let $G$ be a nilpotent group and $M$ be a minimal normal subgroup of $G$. 
	Prove that $M\subseteq Z(G)$.
\end{prob}

%\begin{exercise}
%	\label{xca:truco}
%	Let $G$ be a finite group. If $P\in\Syl_p(G)$ and $M$ is a subgroup of $G$ such that 
%	$N_G(P)\subseteq M$, then $M=N_G(M)$. 
%\end{exercise}


\begin{prob}
	\label{xca:normalizadora}
	Let $G$ be a finite group. Prove that the following statements are equivalent:  
	\begin{enumerate}
		\item $G$ is nilpotent.
		\item If $H\subsetneq G$ is a subgroup, then $H\subsetneq N_G(H)$.
		\item Every maximal subgroup of $G$ is normal in $G$.
	\end{enumerate}
\end{prob}

% \begin{svgraybox}
% 	Para demostrar que $(1)\implies(2)$ simplemente usamos el
% 	lema~\ref{lemma:normalizadora}. Para demostrar que $(2)\implies(3)$ hacemos
% 	lo siguiente: si $M$ es un subgrupo maximal, como $M\subsetneq N_G(M)$ por
% 	hipótesis, $N_G(M)=G$ por maximalidad. Finalmente demostremos que
% 	$(3)\implies(1)$.  Sea $P\in\Syl_p(G)$. Si $P$ no es normal en $G$,
% 	$N_G(P)\ne G$ y entonces existe un subgrupo maximal $M$ tal que
% 	$N_G(P)\subseteq M$. Como $M$ es normal en $G$, el
% 	ejercicio~\ref{exercise:truco} implica que $M=N_G(M)=G$, una contradicción.
% 	Luego $P$ es normal en $G$ y entonces $G$ es nilpotente por el
% 	teorema~\ref{theorem:nilpotente:eq}.
% \end{svgraybox}

% ejercicio: G finito. Es nilpotente si y solo si dos elementos de ordenes coprimos conmnutan
% 5.41 rotman

\begin{prob}
	Let $G$ be a finite nilpotent group. Prove that if $p$ is a prime dividing the order of $G$, then there
	exists a minimal normal subgroup of order $p$ and there exists a maximal subgroup of index $p$.
\end{prob}

%\begin{proof}
%	Assume that $|G|=p^{\alpha}m$, where $p$ is coprime with $m$.
%	Write $G=P\times H$, where $P\in\Syl_p(G)$.  Since $Z(P)$ is a non-trivial subgroup of 
%	of $P$, subgroup   
%	of $Z(P)$ minimal-normal en $G$ tiene orden $p$ (y esos subgrupos 
%	existen porque $G$ es un grupo finito). Por otro lado, como $P$ contiene un subgrupo
%	de índice $p$, que resulta maximal. Luego $P\times H$ también contiene un
%	subgrupo maximal de índice $p$.
%\end{proof}

\begin{prob}
	\label{xca:pgrupos}
	Let $p$ be a prime and let $G$ be a non-trivial group of order $p^n$. Prove the following statements.
	\begin{enumerate}
		\item $G$ has a normal subgroup of order $p$.
		\item For each $j\in\{0,\dots,n\}$ there exists a normal subgroup of order $p^j$. 
	\end{enumerate}
\end{prob}

% \begin{svgraybox}
% 	\begin{enumerate}
% 		\item Sabemos que $Z(G)\ne1$. Sea $g\in Z(G)$ tal que $g\ne 1$.
% 			Supongamos que el orden de $g$ es $p^k$ para algún $k\geq1$.
% 			Entonces $g^{p^{k-1}}$ tiene orden $p$ y luego genera un subgrupo
% 			central de orden $p$. 
% 		\item Procederemos por inducción en $n$. Si $n=1$ el resultado es
% 			trivial.  Supongamos entonces que el resultado vale para un cierto
% 			$n\geq2$. Por el punto anterior, $G$ posee un subgrupo normal $N$
% 			de orden $p$. Luego $G/N$ tiene orden $p^{n-1}$. Sea $\pi\colon G\to G/N$ el morfismo canónico. 
% 			Por hipótesis
% 			inductiva, para cada $j\in\{0,\dots,n-1\}$. Por el teorema de la
% 			correspondecia, cada subgrupo normal $S_j$ de $G/N$ de orden $p^j$ se
% 			corresponde con un subgrupo $\pi^{-1}(S_j)$ de $G$ de orden $p^{j+1}$ pues, como
% 			$\pi$ es sobreyectiva, se tiene $\pi(\pi^{-1}(S_j))=S_j$, y luego
% 			\[
% 			p^j=|S_j|=|\pi(\pi^{-1}(S_j))|=\frac{|\pi^{-1}(S_j)|}{|\pi^{-1}(S_j)\cap N|}=\frac{|\pi^{-1}(S_j)|}{|N|}=\frac{|\pi^{-1}(S_j)|}{p}.
% 			\]
% 	\end{enumerate}
% \end{svgraybox}

\begin{prob}
\label{xca:nilpotente_equivalencia}
	Let $G$ be a finite group. Prove that the following conditions are equivalent.
	\begin{enumerate}
		\item $G$ is nilpotent.
		\item Any two elements of coprime order commute. 
		\item Every non-trivial quotient of $G$ has non-trivial center.
		\item If $d$ divides $|G|$, there exists a normal subgroup of $G$ of order $d$. 
		\end{enumerate}
\end{prob}

% \begin{svgraybox}
% 	Veamos que $(1)\implies(2)$. Sabemos que $G$ es producto directo de sus
% 	subgrupos de Sylow, digamos $G=\prod_{i=1}^k S_i$, donde los $S_i$ son los
% 	distintos subgrupos de Sylow de $G$.  Sean
% 	$x=(x_1,\dots,x_k),y=(y_1,\dots,y_k)\in G$. Como $|x|$ y $|y|$ son
% 	coprimos, para cada $i\in\{1,\dots,k\}$ se tiene $x_i=1$ o $y_i=1$. Luego
% 	\[
% 		[x,y]=([x_1,y_1],[x_2,y_2],\dots,[x_k,y_k])=1. 
% 	\]
% 	Demostremos ahora que $(2)\implies(1)$. Supongamos que
% 	$|G|=p_1^{\alpha_1}\cdots p_k^{\alpha_k}$, donde los $p_j$ son primos
% 	distintos y para cada $j$ sea $P_j\in\Syl_{p_j}(G)$. Como elementos de
% 	órdenes coprimos conmutan, la función $P_1\times\cdots\times P_k\to G$,
% 	$(x_1,\dots,x_k)\mapsto x_1\cdots x_k$, es un morfismo inyectivo de grupos.
% 	Como entonces $G\simeq P_1\times\cdots P_k$, y cada $P_j$ es nilpotente,
% 	$G$ es nilpotente. 

% 	Para demostrar que $(1)\implies(3)$ simplemente hay que observar que todo
% 	cociente de $G$ es nilpotente y luego utilizar el
% 	teorema~\ref{theorem:Z(nilpotent)}. Demostremos que $(3)\implies(1)$. Como
% 	todo cociente no trivial de $G$ tiene centro no trivial, en particular
% 	$Z_1=Z(G)$ es no trivial. Si $Z_1=G$ entonces $G$ es abeliano y no hay nada
% 	para demostrar. Si $Z_1\ne G$ entonces $G/Z_1\ne 1$; luego $Z(G/Z_1)\ne 1$.
% 	Si $\pi_1\colon G\to G/Z_1$ es el morfismo canónico,
% 	$Z_2=\pi_1^{-1}(Z(G/Z_1))$. Inductivamente, si tenemos construido el
% 	subgrupo $Z_i$, $Z_i\ne G$ y  $\pi_i\colon G\to G/Z_{i}$ es el morfismo
% 	canónico, se define el subgrupo $Z_{i+1}=\pi_i^{-1}(Z(G/Z_i))$. Por
% 	construcción, $Z_i\subseteq Z_{i+1}$ para todo $i$. Como $G$ es finito,
% 	existe $k$ tal que $Z_k=G$ y luego $G$ es nilpotente.

% 	Demostremos que $(1)\implies(4)$. Esta implicación es consecuencia
% 	inmediata del ejercicio~\ref{exercise:pgrupos}. 
% 	Como $G$ es nilpotente, $G$ producto
% 	directo de sus $p$-grupos de Sylow. Si $d=p_1^{\alpha_1}\cdots
% 	p_k^{\alpha_k}$ es un divisor del orden de $G$, basta tomar
% 	$H=H_1\times\cdots\times H_k$, 
% 	donde cada $H_j$ es un subgrupo normal del $p_j$-subgrupo de Sylow de $G$
% 	de orden $p_j^{\alpha_j}$. Para demostrar que $(4)\implies(1)$ simplemente
% 	se aplica la hipótesis a cada $p$-subgrupo de $G$ de orden maximal.
% \end{svgraybox}

\begin{prob}
Let $G$ be a group and $x,y\in G$.  Prove the following statements.
\begin{enumerate}
	\item If $[x,y]\in C_G(x)\cap C_G(y)$, then
	$[x,y]^n=[x^n,y]=[x,y^n]$
	for all $n\in\Z$.
	\item If $G$ is nilpotent of class two, then $(xy)^n=[y,x]^{n(n-1)/2}x^ny^n$ 
	for all positive integer $n$.
\end{enumerate}	
\end{prob}

\begin{prob}
	Let $p$ be an odd prime number and  
	let $P$ be a $p$-group of nilpotency class $\leq2$. Prove the following statements.
	\begin{enumerate}
	\item 	If $[y,x]^p=1$ for all $x,y\in P$, then $P\to [P,P]$,
	$x\mapsto x^p$, is a group homomorphism. 
	\item $\{x\in P:x^p=1\}$ is a subgroup of $P$. 
	\end{enumerate}
\end{prob}

%\begin{exercise}
%\label{xca:Phi(G)char}
%	If $G$ is a group, then $\Phi(G)$ is characteristic in $G$.
%\end{exercise}




\section*{Notes}
