\chapter{Derivations}
\label{cocycles}

An \textbf{extension} of $K$ by $Q$ is a
short exact sequence 
\[
\begin{tikzcd}
	1 & K & G & Q & 1
	\arrow[from=1-1, to=1-2]
	\arrow[from=1-2, to=1-3]
	\arrow[from=1-3, to=1-4]
	\arrow[from=1-4, to=1-5]
\end{tikzcd}
\]
This means that $G$ is a group with a normal subgroup 
$N$ isomorphic to $K$ such that $G/N\simeq Q$. 

\begin{example}
	$C_6$ and $\Sym_3$ are both extensions of $C_3$ by $C_2$.
\end{example}

\begin{example}
	$C_6$ is an extension of $C_2$ by $C_3$.
\end{example}

\begin{example}
    The direct product $K\times Q$ of the groups $K$ and $Q$ 
    is an extension of $K$ by $Q$ and an extension of $Q$ by $K$. 
\end{example}

\begin{example}
Let $G$ be an extension of $K$ by $Q$. If $L$ is a subgroup of $G$ containing $K$, 
then $L$ is an extension
of $K$ by $L/K$.
\end{example}

\index{Lifting} 
Let $E:
\begin{tikzcd}
	1 & K & G & Q & 1
	\arrow[from=1-1, to=1-2]
	\arrow[from=1-2, to=1-3]
	\arrow[from=1-3, to=1-4]
	\arrow[from=1-4, to=1-5]
\end{tikzcd}$
be an extension. A \textbf{lifting} of $E$ is a map $\ell\colon
Q\to G$ such that $p(\ell(x))=x$ for all $x\in Q$. 

\begin{exercise}
	\label{xca:lifting}
	Let $E:
	\begin{tikzcd}
	1 & K & G & Q & 1
	\arrow[from=1-1, to=1-2]
	\arrow[from=1-2, to=1-3]
	\arrow["p", from=1-3, to=1-4]
	\arrow[from=1-4, to=1-5]
    \end{tikzcd}$
	be an extension. 
	\begin{enumerate}
		\item If $\ell\colon Q\to G$ is a lifting, then $\ell(Q)$
			is a transversal of $\ker p$ in $G$.
		\item Each transversal of $\ker p$ in $G$ induces a lifting $\ell\colon
			Q\to G$.
		\item If $\ell\colon Q\to G$ is a lifting, then 
			$\ell(xy)\ker p=\ell(x)\ell(y)\ker p$.
	\end{enumerate}
\end{exercise}

%\begin{definition}
%	\index{Acoplamiento}
%	Un morfismo de grupos $\chi\colon Q\to\Out(K)$ se denomina un
%	\textbf{acoplamiento} de $Q$ en $K$.
%\end{definition}
%
%\begin{theorem}
%	\label{theorem:coupling}
%	Toda extensión $1\to K\xrightarrow{\iota}G\xrightarrow{p} Q\to1$ determina
%	un acoplamiento de $Q$ en $K$.
%\end{theorem}
%
%\begin{proof}
%	Sea $N=\ker p$. Como $N$ es normal en $G$, para cada $x\in Q$ se tiene un
%	automorfismo $\gamma_{\ell(x)}$ dado por $n\mapsto \ell(x)n\ell(x)^{-1}$ de
%	$N$. Como $\iota\colon K\to\iota(K)=N$ es un isomorfismo, tenemos un automorfismo
%	$\lambda_x\in\Aut(K)$ que hace conmutar al diagrama
%    \[
%    \xymatrix{
%    N
%	\ar[r]^-{\gamma_{\ell(x)}}
%    & N
%	\ar[d]^{\iota^{-1}}
%    \\
%    K
%	\ar[u]^{\iota}
%	\ar[r]_{\lambda_x}
%    & K
%    }
%    \]
%	es decir:	
%	\[
%	\iota(\lambda_x(k))=\ell(x)\iota(k)\ell(x)^{-1},\quad
%	k\in K.
%	\]
%	Estudiemos cómo depende 
%	esta función del levantamiento elegido. Si $\ell_1\colon Q\to G$
%	es un levantamiento, existe $n\in
%	N$ tal que $\ell(x)=\ell_1(x)n$. Si $k\in K$ y $l\in K$ es tal
%	que $\iota(l)=n$, entonces $\lambda_x$ y $(\lambda_1)_x$ pertenecen a la
%	misma coclase módulo $\Inn(K)$ pues 
%	\begin{align*}
%		\iota(\lambda_x(k))&=\ell(x)\iota(k)\iota(x)^{-1}
%		=\ell_1(x)n\iota(k)n^{-1}\ell_1(x)^{-1}\\
%		&=\ell_1(x)\iota(lkl^{-1})\ell_1(x)^{-1}
%		=\iota((\lambda_1)_x(lkl^{-1})).
%	\end{align*}
%	Queda bien definida entonces la función $\lambda\colon Q\to\Out(K)$,
%	$x\mapsto\lambda_x$. 
%	
%	Veamos que $\lambda$ es morfismo de grupos. Por el ejercicio~\ref{exercise:lifting}  
%	existe $n\in N$ tal que
%	$\ell(xy)=\ell(x)\ell(y)n$. Si escribimos $n=\iota(l)$ para algún $l\in K$
%	entonces vemos que $\lambda_x\lambda_y=\lambda_{xy}\gamma_l$ y luego
%	$\lambda(x)\lambda(y)=\lambda(xy)$. 
%\end{proof}
%
%\begin{exercise}
%	Demuestre que extensiones equivalentes dan el mismo acoplamiento.
%\end{exercise}
%
%\begin{svgraybox}
%	Si las extensiones 
%	\[
%	E\colon 1\to K\xrightarrow{\iota}G\xrightarrow{p} Q\to1,
%	\quad
%	E_1\colon 1\to K_1\xrightarrow{\iota_1}G_1\xrightarrow{p_1} Q\to1,
%	\]
%	son equivalentes entonces el diagrama 
%	\[
%	\xymatrix{
%	0\ar[r] 
%	& K
%	\ar@{=}[d]
%	\ar[r]^-{\iota}
%	& G
%	\ar[r]^-{p}
%	\ar[d]^\beta
%	& Q\ar[r]
%	\ar@{=}[d]
%	& 0
%	\\
%	0\ar[r] 
%	& K
%	\ar[r]^-{\iota_{1}}
%	& G_1
%	\ar[r]^-{p_{1}}
%	& Q\ar[r]
%	& 0
%	}
%	\]
%	es conmutativo. Si $\ell\colon Q\to G$ es un levantamiento para $E$
%	entonces $\beta\ell$ es un levantamiento para $E_1$ pues
%	$p_1(\beta\ell)=(p_1\beta)\ell=p\ell=\id$. 
%
%	Si $x\in Q$ y $k\in K$ entonces
%	$\iota(\lambda_x(k))=\ell(x)\iota(k)\ell(x)^{-1}$. Al aplicar $\beta$ y
%	usar la conmutatividad del diagrama:
%	\begin{align*}
%		\iota_1(\lambda_x(k))&=\beta\iota\lambda_x(k)
%		=\beta(\ell(x))\beta\iota(k)\beta(\ell(x)^{-1})\\
%		&=\beta\ell(x)\iota_1(k)(\beta\ell(x))^{-1}
%		=\iota_1((\lambda_1)_x(k)).
%	\end{align*}
%	Como $\iota_1$ es inyectiva,
%	$\chi(x)(k)=\lambda_x(k)=(\lambda_1)_x(k)=\chi_1(x)(k)$.
%\end{svgraybox}

\index{Split!extension}
An extension $E$ \textbf{splits} if there is a lifting of $E$ that it is a group homomorphism. 

%\begin{lemma}
%	\label{lemma:split}
%	Si una extensión $E\colon 1\to K\xrightarrow{\iota}G\xrightarrow{p} Q\to1$
%	se parte entonces $N=\iota(K)$ admite un complemento en $G$.
%\end{lemma}
%
%\begin{proof}
%	
%\end{proof}

%\begin{definition}
%	Sea $E\colon 1\to K\xrightarrow{\iota}G\xrightarrow{p} Q\to1$ una
%	extensión. Diremos que un $\gamma\in\Aut(G)$ \textbf{estabiliza} a $E$ si 
%	el diagrama 
%	\[
%	\xymatrix{
%	0\ar[r] 
%	& K
%	\ar@{=}[d]
%	\ar[r]^-{\iota}
%	& G
%	\ar[r]^-{p}
%	\ar[d]^\gamma
%	& Q\ar[r]
%	\ar@{=}[d]
%	& 0
%	\\
%	0\ar[r] 
%	& K
%	\ar[r]^-{\iota}
%	& G
%	\ar[r]^-{p}
%	& Q\ar[r]
%	& 0
%	}
%	\]
%	es conmutativo. El \textbf{estabilizador} de la extensión $E$ es el
%	conjunto de $\gamma\in\Aut(G)$ que estabilizan a $E$.
%\end{definition}
%
%\begin{theorem}
%	\label{theorem:estabilizador_abeliano}
%	El estabilizador de una extensión 
%	$E\colon 1\to K\xrightarrow{\iota}G\xrightarrow{p} Q\to1$ 
%	es un grupo abeliano.
%\end{theorem}
%
%\begin{proof}
%	Sea $S$ el estabilizador de la extensión $E$ y sea $\gamma\in S$. 
%	Sea $T$ un transversal a $N=\ker p$ en $G$. Si $g\in G$ existen $n,n_1\in N$ y
%	$t,t_1\in T$ tales que $g=nt$ y $\gamma(g)=n_1t_1$. 
%	Como entonces $tN=t_1N$ pues 
%	\[
%		p(t_1)=p(n_1t_1)=p(\gamma(g))=p(g)=p(nt)=p(n)p(t)=p(t),
%	\]
%	se concluye que $\gamma(g)=n_1t$. Luego 
%	\[
%	g\gamma(g)^{-1}=nt(n_1t)^{-1}=ntt^{-1}n_1=nn_1\in N.
%	\]
%
%	Veamos ahora que $g\gamma(g)^{-1}\in Z(N)$. Si $n\in N$ entonces
%	$\gamma(n)=n$ pues $n=\iota(k)$ para
%	algún $k\in K$ y entonces $\gamma(n)=\gamma\iota(k)=\iota(k)=n$. Luego 
%	\[
%	[g\gamma(g)^{-1},n]=g\gamma(g)^{-1}n\gamma(g)g^{-1}n^{-1}=g\gamma(g^{-1}ng)g^{-1}n^{-1}=1.
%	\]
%
%	Sea 
%	\[
%	\Psi\colon S\to\prod_{g\in G}Z(N),
%	\quad
%	\gamma\mapsto (g^{-1}\gamma(g))_{g\in G}. 
%	\]
%	
%	
%
%\end{proof}
%\section{Derivaciones y complementos}


\index{Derivation}
\index{$1$-cocyclo}
Let $Q$ and $K$ be groups. Assume that $Q$ acts by automorphism on $K$.
A map $\varphi\colon Q\to K$ is said to be a $1$-\textbf{cocycle} (or a derivation) if
\[
		\varphi(xy)=\varphi(x)(x\cdot\varphi(y))
\]
for all $x,y\in Q$.  The set of 1-cocycles $Q\to K$ is defined as 
\[
\Der(Q,K)=Z^1(Q,K)=\{\delta\colon Q\to K:\text{$\delta$ is $1$-cocycle}\}.
\]

\begin{example}
	Let $Q$ acts on $K$ by automorphisms. For each $k\in K$, the map 
	$Q\to K$, $x\mapsto [k,x]=kxk^{-1}x^{-1}$, is a derivation. 
\end{example}

% \begin{svgraybox}
% 	Para $k\in K$ y $x\in Q$ escribimos $\delta_k(x)=[k,x]$. Entonces 
% 	\[
% 	\delta_k(x)(x\delta_k(y)x^{-1})
% 	=kxk^{-1}x^{-1}xkyk^{-1}y^{-1}x^{-1}
% 	=k(xy)k^{-1}(xy)^{-1}
% 	=\delta_k(xy).
% 	\]
% \end{svgraybox}

\begin{exercise}
	\label{xca:1cocycle}
	Let $\varphi\colon Q\to K$ be a $1$-cocycle. 
	\begin{enumerate}
		\item $\varphi(1)=1$.
		\item $\varphi(y^{-1})=(y^{-1}\cdot\phi(y))^{-1}=y^{-1}\cdot\phi(y)^{-1}$.
		\item The set $\ker\varphi=\{x\in Q:\varphi(x)=1\}$ is a subgroup of $Q$. 
	\end{enumerate}
\end{exercise}

A subgroup $K$ of $G$ admits a \textbf{complement} $Q$ if $G$ admits an exact factorization 
through $K$ and $Q$, i.e. $G=KQ$ with $K\cap Q=\{1\}$. 
A classical example is the semidirect product $G=K\rtimes Q$, where $K$ is a normal subgroup of $G$ 
and $Q$ is a subgroup of $G$ such that $K\cap Q=\{1\}$. 

\begin{theorem}
	\label{theorem:complements}
	Let $Q$ acts by automorphism on $K$. Then there exists a bijective correspondence between
	the set $\mathcal{C}$ of complements $K$ in $K\rtimes Q$ and the set 
    $\Der(Q,K)$ of $1$-cocycles $Q\to K$.
\end{theorem}

\begin{proof}
	Since $Q$ acts by conjugation on $K$, it follows that $\delta\in\Der(Q,K)$ if and only if 
	$\delta(xy)=\delta(x)x\delta(y)x^{-1}$ for all $x,y\in Q$. In this case, 
	one obtains that 
	$\delta(1)=1$ and $\delta(x^{-1})=x^{-1}\delta(x)^{-1}x$.
	
	Let 
	$C\in\mathcal{C}$. If $x\in Q$, then there exist unique elements  
	$k\in K$ and $c\in C$ such that $x=k^{-1}c$. Hence the he map 
	$\delta_C\colon Q\to K$, $x\mapsto k$, is well-defined. Moreover, 
	$\delta(x)x=c\in C$. 
	
	We claim that $\delta_C\in\Der(Q,K)$. If $x,x_1\in Q$, we write $x=k^{-1}c$
	and $x_1=k_1^{-1}c_1$ for $k,k_1\in K$ and $c,c_1\in C$. Since $K$ is a normal subgroup of 
	the semidirect product $K\rtimes Q$, we can write $xx_1$ as $xx_1=k_2c_2$, where 
	$k_2=k^{-1}(ck_1^{-1}c^{-1})\in K$, $c_2=cc_1\in C$. Thus  
	$\delta(xx_1)xx_1=cc_1=\delta(x)x\delta(x_1)x_1$ 
	implies that $\delta(xx_1)=\delta(x)x\delta(x_1)x^{-1}$. 
	So there is a map $F\colon\mathcal{C}\to\Der(Q,K)$, $F(C)=\delta_C$.

	We now construct a map $G\colon\Der(Q,K)\to\mathcal{C}$. 
	For each 
	$\delta\in\Der(Q,K)$ we find a complement $\Delta$ of $K$ in $K\rtimes Q$. Let 
	$\Delta=\{\delta(x)x:x\in Q\}$. 
	We claim that $\Delta$ is a subgroup of $K\rtimes Q$. Since $\delta(1)=1$,
	$1\in X$. If $x,y\in Q$, then 
	$\delta(x)x\delta(y)y=\delta(x)x\delta(y)x^{-1}xy=\delta(xy)xy\in \Delta$.
	Finally, if $x\in Q$, then 
	\[
	(\delta(x)x)^{-1}=x^{-1}\delta(x)^{-1}xx^{-1}=\delta(x^{-1})x^{-1}.
	\]
	
	Let us prove that $\Delta\cap K=\{1\}$. If $x\in Q$ is such that $\delta(x)x\in K$, then 
    since $\delta(x)\in K$, it follows that $x\in K\cap Q=1$. If $g\in G$, then 
	there are unique $k\in K$ and $x\in Q$ such that $g=kx$. We write 
	$g=k\delta(x)^{-1}\delta(x)x$. Since $k\delta(x)^{-1}\in K$ and $\delta(x)x\in
	\Delta$, we conclude that $G=K\Delta$. Thus there is a well-defined map 
	$G\colon\Der(Q,K)\to\mathcal{C}$, $G(\delta)=\Delta$.

	We claim that $G\circ F=\id_{\mathcal{C}}$. 
	Let $C\in\mathcal{C}$. Then  
	\[
	G(F(C))=G(\delta_C)=\{\delta_C(x)x:x\in
	Q\}=C,
	\]
	by construction. (We know that $\delta_C(x)x\in C$. Conversely, if $c\in
	C$, we write $c=kx$ for unique elements $k\in K$ and $x\in Q$. Thus $x=k^{-1}c$
	and hence $c=\delta_c(x)x$.)

	Finally, we prove that $F\circ G=\id_{\Der(Q,K)}$. Let $\delta\in\Der(Q,K)$.
    Then	
    \[
	F(G(\delta))=F(\Delta)=\delta_{\Delta}.
	\]
	Finally, we need to show that $\delta_\Delta=\delta$.  Let $x\in Q$. There exists 
	$\delta(y)y\in\Delta$ for some $y\in Q$ such that $x=k^{-1}\delta(y)y$.
	Thus $\delta_{\Delta}(x)x=\delta(y)y$ and hence $\delta(x)=\delta(y)$ by
	the uniqueness. 
\end{proof}

\index{Derivación!interior}
\index{$1$-coborde}
Sean $Q$ y $K$ grupos. Supongamos que $Q$ actúa por automorfismos en $K$.
Un $\delta\in\Der(Q,K)$ se dice \textbf{interior} si existe $k\in K$ tal
que $\delta(x)=[k,x]$ para todo $x\in Q$. El conjunto de
\textbf{derivaciones interiores} será denotado por
\[
		\Inn(Q,K)=B^1(Q,K)=\{\delta\in\Der(Q,K):\text{$\delta$ es interior}\}.
\]

Una derivación interior también se llama \textbf{$1$-coborde}.

\begin{theorem}[Sysak]
	\index{Teorema!de Sysak}
	\index{Sysak, Y.}
	\label{theorem:Sysak}
	Sean $Q$ y $K$ grupos tales que $Q$ actúa por automorfismos en $K$. Sea
	$\delta\in\Der(Q,K)$.
	\begin{enumerate}
		\item $\Delta=\{\delta(x)x:x\in Q\}$ es un complemento para $K$ en $K\rtimes Q$.
		\item $\delta\in\Inn(Q,K)$ si y sólo si $Q$ y $\Delta$ son conjugados en
			$K$.
		\item $\ker\delta=Q\cap\Delta$.
		\item $\delta$ es sobreyectiva si y sólo si $K\rtimes Q=\Delta Q$.
	\end{enumerate}
\end{theorem}

\begin{proof}
	Vimos en la demostración del teorema~\ref{theorem:complementos} que
	el conjunto $\Delta$ es un complemento para $K$ en $K\rtimes Q$. 

	Demostremos la segunda afirmación. Si suponemos que $\delta$ es interior,
	existe $k\in K$ tal que $\delta(x)=[k,x]=kxk^{-1}x^{-1}$ para todo $x\in
	Q$. Como $\delta(x)x=kxk^{-1}$ para todo $x\in Q$,  $\Delta=kQk^{-1}$.
	Recíprocamente, si existe $k\in K$ tal que $\Delta=kQk^{-1}$, para cada
	$x\in Q$ existe $y\in Q$ tal que $\delta(x)x=kyk^{-1}$. Como
	$[k,y]=kyk^{-1}y^{-1}\in K$, $\delta(x)\in K$ y $\delta(x)x=[k,y]y\in KQ$,
	se concluye que $x=y$ y luego $\delta(x)=[k,x]$. 

	Demostremos la tercera afirmación. Si $x\in Q$ es tal que $\delta(x)x=y\in
	Q$ entonces $\delta(x)=yx^{-1}\in K\cap Q=\{1\}$. Recíprocamente, si $x\in Q$
	es tal que $\delta(x)=1$ entonces $x=\delta(x)x\in Q\cap\Delta$. 

	Demostremos la cuarta afirmación. Si $\delta$ es sobreyectiva, para cada
	$k\in K$ existe $y\in Q$ tal que $\delta(y)=k$. Luego $K\rtimes Q\subseteq
	\Delta Q$ pues $kx=\delta(y)x=(\delta(y)y)y^{-1}x\in \Delta Q$. Además
	$\Delta Q\subseteq K\rtimes Q$ pues si $\delta(x)\in K$ para todo $x\in Q$.
	Recíprocamente, si $k\in K$ y $x\in Q$ existen 
	$y,z\in Q$ tales que $kx=\delta(y)yz$; en particular, 
	por la unicidad de la escritura de $K\rtimes Q$,
	$k=\delta(y)$. 
\end{proof}

Un caso importante de grupos que admiten factorización
es el siguiente: 

\begin{definition}
	Un grupo $G$ admite una \textbf{factorización triple} si tiene subgrupos
	$A$, $B$ y $M$ tales que $G=MA=MB=AB$ y $A\cap M=B\cap M=\{1\}$.
\end{definition}

%%% TODO:
%%% ejemplos de factorizacion triple
%%% caracterizacion de 1-cociclos biyectivos
%%% ejemplo con anillo radical
%%% braces?

An immediate consequence of Sysak's theorem:

\begin{corollary}
	If the group $Q$ acts by automorphisms on $K$ and 
	$\delta\in\Der(Q,K)$ is surjective, then $G=K\rtimes Q$ admits a triple factorization. 
\end{corollary}

% \begin{proof}
% 	Es consecuencia inmediata del teorema~\ref{theorem:Sysak}. 
% \end{proof}
Another consequence: 

\begin{exercise}
	\label{xca:ker1cocycle}
	Let $\delta\in\Der(Q,K)$. 
	\begin{enumerate}
	\item Prove that $\delta$ is injective if and only if 
	$\ker\delta=\{1\}$.
	\item If $\delta$ is bijective, then  
	$K$ admits a complement 
	$\Delta$ in $K\rtimes Q$ such that $K\rtimes Q=K\rtimes\Delta=\Delta Q$ and 
	$Q\cap\Delta=\{1\}$.
	\end{enumerate}
\end{exercise}

% \begin{proof}
% 	Sean $x,y\in Q$ tales que $\delta(x)=\delta(y)$. Como $\delta(x^{-1}y)=1$
% 	pues 
% 	\[
% 	\delta(x^{-1}y)=\delta(x^{-1})(x^{-1}\delta(y)x)=\delta(x^{-1})x^{-1}\delta(x)x=\delta(x^{-1}x)=\delta(1)=1
% 	\]
% 	y $\delta$ es inyectiva, $x^{-1}y=1$. La afirmación recíproca es trivial.
% \end{proof}

% \begin{corollary}
% 	Si $\delta\in\Der(Q,K)$ es biyectivo entonces $K$ admite un complemento
% 	$\Delta$ en $K\rtimes Q$ tal que $K\rtimes Q=K\rtimes\Delta=\Delta Q$ y
% 	$Q\cap\Delta=1$.
% \end{corollary}

% \begin{proof}
% 	Vimos en el teorema de Sysak que $\delta$ es sobreyectiva si y
% 	sólo si $K\rtimes Q=\Delta Q$ y que $\ker\delta=Q\cap\Delta$.
% \end{proof}

%\section{Aplicación: subespacios invariantes}
%
%Sea $A$ un grupo que actúa por automorfismos en un grupo $G$. Definimos
%\[
%C_G(A)=\{g\in G:g\cdot a=a\text{ para todo $a\in A$}\}.
%\]
%
%Como aplicación de la teoría de Schur--Zassenhaus vamos a demostrar los
%teoremas de Sylow para subespacios $A$-invariantes.
%Necesitamos el siguiente lema:
%
%\begin{lemma}
%	\label{lemma:Glauberman}
%	Sean $A$ y $G$ grupos finitos de órdenes coprimos. Supongamos que $A$ actúa
%	por automorfismos en $G$ y que $A$ o $G$ es resoluble. Supongamos que $A$
%	actúa en un conjunto $X$ y que $G$ actúa transitivamente en $X$ de forma tal que
%	\begin{equation}
%		\label{equation:Glauberman:compatibilidad}
%		a\cdot (g\cdot x)=(aga^{-1})\cdot (a\cdot x)
%	\end{equation}
%	para todo $a\in A$, $g\in G$, $x\in X$. Valen las siguientes afirmaciones:
%	\begin{enumerate}
%		\item Existe un $x\in X$ invariante por la acción de $A$.
%		\item Si $x,y\in X$ son invariantes por la acción de $A$ entonces
%			existe $c\in C_G(A)$ tal que $c\cdot x=y$.
%	\end{enumerate}
%\end{lemma}
%
%\begin{proof}
%	Sea $\Gamma=G\rtimes A$ el producto semidirecto. Todo $\gamma$ se escribe
%	en forma única como $\gamma=ga$ con $g\in G$, $a\in A$. Veamos que $\Gamma$
%	actúa en $X$ por
%	\[
%		\gamma\cdot x=(ga)\cdot x=g\cdot (a\cdot x).
%	\]
%	Es fácil ver que es una acción pues la igualdad
%	\[
%	(ga)\cdot ((hb)\cdot x)=((ga)(hb))\cdot x=(gaha^{-1})\cdot ((ab)\cdot x)
%	\]
%	es consecuencia de la relación de
%	compatibilidad~\eqref{equation:Glauberman:compatibilidad}.\framebox{completar}
%
%\end{proof}
%\begin{theorem}
%	\label{theorem:Sylow_Ainv}
%\end{theorem}
%
%

Let $A$ be an additive group and $G$ be a group and let 
$G\times A\to A$, $(g,a)\mapsto g\cdot a$,
is a left action of $G$ on $A$ by automorphisms. This means that the action of $G$ on $A$ satisfies 
$g\cdot (a+b)=g\cdot a+g\cdot b$ for all $g\in G$ and $a,b\in A$.
A \emph{bijective
$1$-cocyle} is a bijective map $\pi\colon G\to A$ such that 
\begin{equation}
    \label{eq:1cocycle}
    \pi(gh)=\pi(g)+g\cdot \pi(h)
\end{equation}
for all $g,h\in G$. 
We now prove the equivalence between braces and bijective 1-cocycles. 

\begin{theorem}
	\label{thm:1cocycle}
    Over any additive group $A$ the following data are equivalent:
    \begin{enumerate}
        \item A group $G$ and a bijective
            $1$-cocycle $\pi\colon G\to A$. 
        \item A brace structure over $A$. 
    \end{enumerate}

    \begin{proof}
        Consider on $A$ a second group structure given by 
        \[
		a\circ b=\pi(\pi^{-1}(a)\pi^{-1}(b))=a+\pi^{-1}(a)\cdot b
		\]
		for all
        $a,b\in A$.  Since $G$ acts on $A$ by
        automorphisms, 
        \begin{align*}
            a\circ (b+c)&=\pi(\pi^{-1}(a)\pi^{-1}(b+c))=a+\pi^{-1}(a)\cdot (b+c)\\
            &=a+ \pi^{-1}(a)\cdot b+\pi^{-1}(a)\cdot c
            =a\circ b-a+a\circ c
        \end{align*}
        holds for all $a,b,c\in A$.
        
        Conversely, assume that the additive group $A$ has a brace structure. Let $G$ be the multiplicative group of $A$
        and $\pi=\id$. By
        Proposition~\ref{pro:lambda}, $a\mapsto\lambda_a$, is a group homomorphism and 
        hence $G$ acts on $A$ by automorphisms. Then~\eqref{eq:1cocycle} holds
        and therefore $\pi\colon G\to A$ is a bijective $1$-cocycle. 
    \end{proof}
\end{theorem}

The construction of Theorem~\ref{thm:1cocycle} is functorial, see Exercise~\ref{prob:1cocycle}.

\begin{example}
	\label{exa:d8q8}
	Let 
	\[
	D_4=\langle r,s:r^4=s^2=1,srs=r^{-1}\rangle
	\]
	be the dihedral group of eight elements and let
	\[
	Q_8=\{1,-1,i,-i,j,-j,k,-k\}
	\]
	be the quaternion group of eight elements.  Let
	$\pi:Q_8\to D_4$ be given by 
	\begin{align*}
		1\mapsto 1 &, & -1\mapsto r^2 &,  & -k\mapsto r^3s &,&  k\mapsto rs &,\\
		i\mapsto s &, & -i\mapsto r^2s &, &  j\mapsto r^3 &, & -j\mapsto r &.
	\end{align*}
	Since $\pi$ is bijective, 
	a straightforward calculation shows that $D_4$ with 
	\[
	  x+y=xy,\quad 
	  x\circ y=\pi(\pi^{-1}(x)\pi^{-1}(y))
	\]
	is a two-sided brace with additive group isomorphic to $D_4$ and multiplicative group
	isomorphic to $Q_8$. 
\end{example}

\section*{Exercises}

\begin{prob}
\label{prob:1cocycle}
Let $\pi\colon G\to A$ and $\eta\colon H\to B$ be bijective $1$-cocycles.  A
\emph{homorphism} between these bijective $1$-cocycles is a pair $(f,g)$ of
group homomorphisms  $f\colon G\to H$, $g\colon A\to B$ such that
\begin{align*}
&\eta\circ f=g\circ \pi,\\
&g(h\cdot a)=f(h)\cdot g(a),&&a\in A,\;h\in G.
\end{align*}
Bijective $1$-cocycles and homomorphisms form a category. 
For a given additive group $A$ 
the full subcategory of the category of bijective $1$-cocycles with objects
$\pi\colon G\to A$ is equivalent to the full subcategory of the category of
braces with additive group $A$. 
\end{prob}

\section*{Open problems}

\section*{Notes}

In the case of braces of abelian type, Theorem~\ref{thm:1cocycle} is implicit in the work of Rump, see \cite{MR2278047,MR3291816} or~\cite{MR3177933}. Similar results appear 
in the work of Etingof, Schedler and Soloviev~\cite{MR1722951}, Lu, Yan and Zhu~\cite{MR1769723} 
and Soloviev~\cite{MR1809284}.
In~\cite{MR1653340} Etingof and Gelaki give a method of constructing finite-dimensional complex semisimple triangular Hopf algebras. They show how any non-abelian group which admits a bijective 1-cocycle gives rise to a semisimple minimal triangular Hopf algebra which is not a group algebra.