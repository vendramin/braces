\chapter{Complements}
\label{cocycles}

\section*{A}

% explicar producto semidirecto.
% equivalencia de complementos (Rotman)
% demostrar Schur-Zassenhaus

An \textbf{extension} of $K$ by $Q$ is a
short exact sequence 
\[
\begin{tikzcd}
	1 & K & G & Q & 1
	\arrow[from=1-1, to=1-2]
	\arrow[from=1-2, to=1-3]
	\arrow[from=1-3, to=1-4]
	\arrow[from=1-4, to=1-5]
\end{tikzcd}
\]
This means that $G$ is a group with a normal subgroup 
$N$ isomorphic to $K$ such that $G/N\simeq Q$. 

\begin{example}
	$C_6$ and $\Sym_3$ are both extensions of $C_3$ by $C_2$.
\end{example}

\begin{example}
	$C_6$ is an extension of $C_2$ by $C_3$.
\end{example}

\begin{example}
    The direct product $K\times Q$ of the groups $K$ and $Q$ 
    is an extension of $K$ by $Q$ and an extension of $Q$ by $K$. 
\end{example}

\begin{example}
Let $G$ be an extension of $K$ by $Q$. If $L$ is a subgroup of $G$ containing $K$, 
then $L$ is an extension
of $K$ by $L/K$.
\end{example}

\index{Lifting} 
Let $E:
\begin{tikzcd}
	1 & K & G & Q & 1
	\arrow[from=1-1, to=1-2]
	\arrow[from=1-2, to=1-3]
	\arrow[from=1-3, to=1-4]
	\arrow[from=1-4, to=1-5]
\end{tikzcd}$
be an extension. A \textbf{lifting} of $E$ is a map $\ell\colon
Q\to G$ such that $p(\ell(x))=x$ for all $x\in Q$. 

\begin{exercise}
	\label{xca:lifting}
	Let $E:
	\begin{tikzcd}
	1 & K & G & Q & 1
	\arrow[from=1-1, to=1-2]
	\arrow[from=1-2, to=1-3]
	\arrow["p", from=1-3, to=1-4]
	\arrow[from=1-4, to=1-5]
    \end{tikzcd}$
	be an extension. 
	\begin{enumerate}
		\item If $\ell\colon Q\to G$ is a lifting, then $\ell(Q)$
			is a transversal of $\ker p$ in $G$.
		\item Each transversal of $\ker p$ in $G$ induces a lifting $\ell\colon
			Q\to G$.
		\item If $\ell\colon Q\to G$ is a lifting, then 
			$\ell(xy)\ker p=\ell(x)\ell(y)\ker p$.
	\end{enumerate}
\end{exercise}

%\begin{definition}
%	\index{Acoplamiento}
%	Un morfismo de grupos $\chi\colon Q\to\Out(K)$ se denomina un
%	\textbf{acoplamiento} de $Q$ en $K$.
%\end{definition}
%
%\begin{theorem}
%	\label{theorem:coupling}
%	Toda extensión $1\to K\xrightarrow{\iota}G\xrightarrow{p} Q\to1$ determina
%	un acoplamiento de $Q$ en $K$.
%\end{theorem}
%
%\begin{proof}
%	Sea $N=\ker p$. Como $N$ es normal en $G$, para cada $x\in Q$ se tiene un
%	automorfismo $\gamma_{\ell(x)}$ dado por $n\mapsto \ell(x)n\ell(x)^{-1}$ de
%	$N$. Como $\iota\colon K\to\iota(K)=N$ es un isomorfismo, tenemos un automorfismo
%	$\lambda_x\in\Aut(K)$ que hace conmutar al diagrama
%    \[
%    \xymatrix{
%    N
%	\ar[r]^-{\gamma_{\ell(x)}}
%    & N
%	\ar[d]^{\iota^{-1}}
%    \\
%    K
%	\ar[u]^{\iota}
%	\ar[r]_{\lambda_x}
%    & K
%    }
%    \]
%	es decir:	
%	\[
%	\iota(\lambda_x(k))=\ell(x)\iota(k)\ell(x)^{-1},\quad
%	k\in K.
%	\]
%	Estudiemos cómo depende 
%	esta función del levantamiento elegido. Si $\ell_1\colon Q\to G$
%	es un levantamiento, existe $n\in
%	N$ tal que $\ell(x)=\ell_1(x)n$. Si $k\in K$ y $l\in K$ es tal
%	que $\iota(l)=n$, entonces $\lambda_x$ y $(\lambda_1)_x$ pertenecen a la
%	misma coclase módulo $\Inn(K)$ pues 
%	\begin{align*}
%		\iota(\lambda_x(k))&=\ell(x)\iota(k)\iota(x)^{-1}
%		=\ell_1(x)n\iota(k)n^{-1}\ell_1(x)^{-1}\\
%		&=\ell_1(x)\iota(lkl^{-1})\ell_1(x)^{-1}
%		=\iota((\lambda_1)_x(lkl^{-1})).
%	\end{align*}
%	Queda bien definida entonces la función $\lambda\colon Q\to\Out(K)$,
%	$x\mapsto\lambda_x$. 
%	
%	Veamos que $\lambda$ es morfismo de grupos. Por el ejercicio~\ref{exercise:lifting}  
%	existe $n\in N$ tal que
%	$\ell(xy)=\ell(x)\ell(y)n$. Si escribimos $n=\iota(l)$ para algún $l\in K$
%	entonces vemos que $\lambda_x\lambda_y=\lambda_{xy}\gamma_l$ y luego
%	$\lambda(x)\lambda(y)=\lambda(xy)$. 
%\end{proof}
%
%\begin{exercise}
%	Demuestre que extensiones equivalentes dan el mismo acoplamiento.
%\end{exercise}
%
%\begin{svgraybox}
%	Si las extensiones 
%	\[
%	E\colon 1\to K\xrightarrow{\iota}G\xrightarrow{p} Q\to1,
%	\quad
%	E_1\colon 1\to K_1\xrightarrow{\iota_1}G_1\xrightarrow{p_1} Q\to1,
%	\]
%	son equivalentes entonces el diagrama 
%	\[
%	\xymatrix{
%	0\ar[r] 
%	& K
%	\ar@{=}[d]
%	\ar[r]^-{\iota}
%	& G
%	\ar[r]^-{p}
%	\ar[d]^\beta
%	& Q\ar[r]
%	\ar@{=}[d]
%	& 0
%	\\
%	0\ar[r] 
%	& K
%	\ar[r]^-{\iota_{1}}
%	& G_1
%	\ar[r]^-{p_{1}}
%	& Q\ar[r]
%	& 0
%	}
%	\]
%	es conmutativo. Si $\ell\colon Q\to G$ es un levantamiento para $E$
%	entonces $\beta\ell$ es un levantamiento para $E_1$ pues
%	$p_1(\beta\ell)=(p_1\beta)\ell=p\ell=\id$. 
%
%	Si $x\in Q$ y $k\in K$ entonces
%	$\iota(\lambda_x(k))=\ell(x)\iota(k)\ell(x)^{-1}$. Al aplicar $\beta$ y
%	usar la conmutatividad del diagrama:
%	\begin{align*}
%		\iota_1(\lambda_x(k))&=\beta\iota\lambda_x(k)
%		=\beta(\ell(x))\beta\iota(k)\beta(\ell(x)^{-1})\\
%		&=\beta\ell(x)\iota_1(k)(\beta\ell(x))^{-1}
%		=\iota_1((\lambda_1)_x(k)).
%	\end{align*}
%	Como $\iota_1$ es inyectiva,
%	$\chi(x)(k)=\lambda_x(k)=(\lambda_1)_x(k)=\chi_1(x)(k)$.
%\end{svgraybox}

\index{Split!extension}
An extension $E$ \textbf{splits} if there is a lifting of $E$ that it is a group homomorphism. 

%\begin{lemma}
%	\label{lemma:split}
%	Si una extensión $E\colon 1\to K\xrightarrow{\iota}G\xrightarrow{p} Q\to1$
%	se parte entonces $N=\iota(K)$ admite un complemento en $G$.
%\end{lemma}
%
%\begin{proof}
%	
%\end{proof}

%\begin{definition}
%	Sea $E\colon 1\to K\xrightarrow{\iota}G\xrightarrow{p} Q\to1$ una
%	extensión. Diremos que un $\gamma\in\Aut(G)$ \textbf{estabiliza} a $E$ si 
%	el diagrama 
%	\[
%	\xymatrix{
%	0\ar[r] 
%	& K
%	\ar@{=}[d]
%	\ar[r]^-{\iota}
%	& G
%	\ar[r]^-{p}
%	\ar[d]^\gamma
%	& Q\ar[r]
%	\ar@{=}[d]
%	& 0
%	\\
%	0\ar[r] 
%	& K
%	\ar[r]^-{\iota}
%	& G
%	\ar[r]^-{p}
%	& Q\ar[r]
%	& 0
%	}
%	\]
%	es conmutativo. El \textbf{estabilizador} de la extensión $E$ es el
%	conjunto de $\gamma\in\Aut(G)$ que estabilizan a $E$.
%\end{definition}
%
%\begin{theorem}
%	\label{theorem:estabilizador_abeliano}
%	El estabilizador de una extensión 
%	$E\colon 1\to K\xrightarrow{\iota}G\xrightarrow{p} Q\to1$ 
%	es un grupo abeliano.
%\end{theorem}
%
%\begin{proof}
%	Sea $S$ el estabilizador de la extensión $E$ y sea $\gamma\in S$. 
%	Sea $T$ un transversal a $N=\ker p$ en $G$. Si $g\in G$ existen $n,n_1\in N$ y
%	$t,t_1\in T$ tales que $g=nt$ y $\gamma(g)=n_1t_1$. 
%	Como entonces $tN=t_1N$ pues 
%	\[
%		p(t_1)=p(n_1t_1)=p(\gamma(g))=p(g)=p(nt)=p(n)p(t)=p(t),
%	\]
%	se concluye que $\gamma(g)=n_1t$. Luego 
%	\[
%	g\gamma(g)^{-1}=nt(n_1t)^{-1}=ntt^{-1}n_1=nn_1\in N.
%	\]
%
%	Veamos ahora que $g\gamma(g)^{-1}\in Z(N)$. Si $n\in N$ entonces
%	$\gamma(n)=n$ pues $n=\iota(k)$ para
%	algún $k\in K$ y entonces $\gamma(n)=\gamma\iota(k)=\iota(k)=n$. Luego 
%	\[
%	[g\gamma(g)^{-1},n]=g\gamma(g)^{-1}n\gamma(g)g^{-1}n^{-1}=g\gamma(g^{-1}ng)g^{-1}n^{-1}=1.
%	\]
%
%	Sea 
%	\[
%	\Psi\colon S\to\prod_{g\in G}Z(N),
%	\quad
%	\gamma\mapsto (g^{-1}\gamma(g))_{g\in G}. 
%	\]
%	
%	
%
%\end{proof}
%\section{Derivaciones y complementos}


\index{Derivation}
\index{$1$-cocyclo}
Let $Q$ and $K$ be groups. Assume that $Q$ acts by automorphism on $K$.
A map $\varphi\colon Q\to K$ is said to be a \textbf{1-cocycle} (or a derivation) if
\[
		\varphi(xy)=\varphi(x)(x\cdot\varphi(y))
\]
for all $x,y\in Q$.  The set of 1-cocycles $Q\to K$ is defined as 
\[
\Der(Q,K)=Z^1(Q,K)=\{\delta\colon Q\to K:\text{$\delta$ is $1$-cocycle}\}.
\]

\begin{example}
	Let $Q$ acts on $K$ by automorphisms. For each $k\in K$, the map 
	$Q\to K$, $x\mapsto [k,x]=kxk^{-1}x^{-1}$, is a derivation. 
\end{example}

% \begin{svgraybox}
% 	Para $k\in K$ y $x\in Q$ escribimos $\delta_k(x)=[k,x]$. Entonces 
% 	\[
% 	\delta_k(x)(x\delta_k(y)x^{-1})
% 	=kxk^{-1}x^{-1}xkyk^{-1}y^{-1}x^{-1}
% 	=k(xy)k^{-1}(xy)^{-1}
% 	=\delta_k(xy).
% 	\]
% \end{svgraybox}

\begin{exercise}
	\label{xca:1cocycle}
	Let $\varphi\colon Q\to K$ be a 1-cocycle. 
	\begin{enumerate}
		\item $\varphi(1)=1$.
		\item $\varphi(y^{-1})=(y^{-1}\cdot\phi(y))^{-1}=y^{-1}\cdot\phi(y)^{-1}$.
		\item The set $\ker\varphi=\{x\in Q:\varphi(x)=1\}$ is a subgroup of $Q$. 
	\end{enumerate}
\end{exercise}

A subgroup $K$ of $G$ admits a \textbf{complement} $Q$ if $G$ admits an exact factorization 
through $K$ and $Q$, i.e. $G=KQ$ with $K\cap Q=\{1\}$. 
A classical example is the semidirect product $G=K\rtimes Q$, where $K$ is a normal subgroup of $G$ 
and $Q$ is a subgroup of $G$ such that $K\cap Q=\{1\}$. 

\begin{theorem}
	\label{thm:complements}
	Let $Q$ acts by automorphism on $K$. Then there exists a bijective correspondence between
	the set $\mathcal{C}$ of complements $K$ in $K\rtimes Q$ and the set 
    $\Der(Q,K)$ of 1-cocycles $Q\to K$.
\end{theorem}

\begin{proof}
	Since $Q$ acts by conjugation on $K$, it follows that $\delta\in\Der(Q,K)$ if and only if 
	$\delta(xy)=\delta(x)x\delta(y)x^{-1}$ for all $x,y\in Q$. In this case, 
	one obtains that 
	$\delta(1)=1$ and $\delta(x^{-1})=x^{-1}\delta(x)^{-1}x$.
	
	Let 
	$C\in\mathcal{C}$. If $x\in Q$, then there exist unique elements  
	$k\in K$ and $c\in C$ such that $x=k^{-1}c$. Hence the he map 
	$\delta_C\colon Q\to K$, $x\mapsto k$, is well-defined. Moreover, 
	$\delta(x)x=c\in C$. 
	
	We claim that $\delta_C\in\Der(Q,K)$. If $x,x_1\in Q$, we write $x=k^{-1}c$
	and $x_1=k_1^{-1}c_1$ for $k,k_1\in K$ and $c,c_1\in C$. Since $K$ is a normal subgroup of 
	the semidirect product $K\rtimes Q$, we can write $xx_1$ as $xx_1=k_2c_2$, where 
	$k_2=k^{-1}(ck_1^{-1}c^{-1})\in K$, $c_2=cc_1\in C$. Thus  
	$\delta(xx_1)xx_1=cc_1=\delta(x)x\delta(x_1)x_1$ 
	implies that $\delta(xx_1)=\delta(x)x\delta(x_1)x^{-1}$. 
	So there is a map $F\colon\mathcal{C}\to\Der(Q,K)$, $F(C)=\delta_C$.

	We now construct a map $G\colon\Der(Q,K)\to\mathcal{C}$. 
	For each 
	$\delta\in\Der(Q,K)$ we find a complement $\Delta$ of $K$ in $K\rtimes Q$. Let 
	$\Delta=\{\delta(x)x:x\in Q\}$. 
	We claim that $\Delta$ is a subgroup of $K\rtimes Q$. Since $\delta(1)=1$,
	$1\in X$. If $x,y\in Q$, then 
	$\delta(x)x\delta(y)y=\delta(x)x\delta(y)x^{-1}xy=\delta(xy)xy\in \Delta$.
	Finally, if $x\in Q$, then 
	\[
	(\delta(x)x)^{-1}=x^{-1}\delta(x)^{-1}xx^{-1}=\delta(x^{-1})x^{-1}.
	\]
	
	We claim that $\Delta\cap K=\{1\}$. If $x\in Q$ is such that $\delta(x)x\in K$, then 
    since $\delta(x)\in K$, it follows that $x\in K\cap Q=\{1\}$. If $g\in G$, then 
	there are unique $k\in K$ and $x\in Q$ such that $g=kx$. We write 
	$g=k\delta(x)^{-1}\delta(x)x$. Since $k\delta(x)^{-1}\in K$ and $\delta(x)x\in
	\Delta$, we conclude that $G=K\Delta$. Thus there is a well-defined map 
	$G\colon\Der(Q,K)\to\mathcal{C}$, $G(\delta)=\Delta$.

	We claim that $G\circ F=\id_{\mathcal{C}}$. 
	Let $C\in\mathcal{C}$. Then  
	\[
	G(F(C))=G(\delta_C)=\{\delta_C(x)x:x\in
	Q\}=C,
	\]
	by construction. (We know that $\delta_C(x)x\in C$. Conversely, if $c\in
	C$, we write $c=kx$ for unique elements $k\in K$ and $x\in Q$. Thus $x=k^{-1}c$
	and hence $c=\delta_c(x)x$.)

	Finally, we prove that $F\circ G=\id_{\Der(Q,K)}$. Let $\delta\in\Der(Q,K)$.
    Then	
    \[
	F(G(\delta))=F(\Delta)=\delta_{\Delta}.
	\]
	Finally, we need to show that $\delta_\Delta=\delta$.  Let $x\in Q$. There exists 
	$\delta(y)y\in\Delta$ for some $y\in Q$ such that $x=k^{-1}\delta(y)y$.
	Thus $\delta_{\Delta}(x)x=\delta(y)y$ and hence $\delta(x)=\delta(y)$ by
	the uniqueness. 
\end{proof}

\index{Derivación!interior}
\index{$1$-coborde}
Let the group $Q$ acts by automorphism on $K$.
A derivation $\delta\in\Der(Q,K)$ is said to be \textbf{inner} if there exists $k\in K$ 
such that $\delta(x)=[k,x]$ for all $x\in Q$. The set of 
\textbf{inner derivations} will be denoted by 
\[
		\Inn(Q,K)=B^1(Q,K)=\{\delta\in\Der(Q,K):\text{$\delta$ is inner}\}.
\]
An inner derivation is also called a \textbf{1-coboundary}.

\begin{theorem}[Sysak]
	\index{Teorema!de Sysak}
	\index{Sysak, Y.}
	\label{theorem:Sysak}
	Sean $Q$ y $K$ grupos tales que $Q$ actúa por automorfismos en $K$. Sea
	$\delta\in\Der(Q,K)$.
	\begin{enumerate}
		\item $\Delta=\{\delta(x)x:x\in Q\}$ es un complemento para $K$ en $K\rtimes Q$.
		\item $\delta\in\Inn(Q,K)$ si y sólo si $Q$ y $\Delta$ son conjugados en
			$K$.
		\item $\ker\delta=Q\cap\Delta$.
		\item $\delta$ es sobreyectiva si y sólo si $K\rtimes Q=\Delta Q$.
	\end{enumerate}
\end{theorem}

\begin{proof}
	In the proof of Theorem~\ref{thm:complements} we 
	found that $\Delta$ is a complement of $K$ in $K\rtimes Q$. 

	Let us prove the second statement. If $\delta$ is inner, then there exists 
    $k\in K$ such that $\delta(x)=[k,x]=kxk^{-1}x^{-1}$ for all $x\in
	Q$. Since $\delta(x)x=kxk^{-1}$ for all $x\in Q$,  $\Delta=kQk^{-1}$.
	Conversely, if there exists $k\in K$ such that $\Delta=kQk^{-1}$, for each 
	$x\in Q$ there exists $y\in Q$ such that $\delta(x)x=kyk^{-1}$. Since
	$[k,y]=kyk^{-1}y^{-1}\in K$, $\delta(x)\in K$ and $\delta(x)x=[k,y]y\in KQ$,
	we conclude that  $x=y$ and hence $\delta(x)=[k,x]$. 

	Let us prove the third statement. If $x\in Q$ is such that $\delta(x)x=y\in
	Q$, then \[
	\delta(x)=yx^{-1}\in K\cap Q=\{1\}.
	\]
	Conversely, if $x\in Q$
	is such that $\delta(x)=1$, then $x=\delta(x)x\in Q\cap\Delta$. 

	Finally we prove the fourth statement. If $\delta$ is surjective, then for each 
	$k\in K$ there exits $y\in Q$ such that $\delta(y)=k$. Thus $K\rtimes Q\subseteq
	\Delta Q$, as 
	\[
	kx=\delta(y)x=(\delta(y)y)y^{-1}x\in \Delta Q.
	\]
	Moreover, 
	$\Delta Q\subseteq K\rtimes Q$, as $\delta(x)\in K$ for all $x\in Q$.
	Conversely, if $k\in K$ y $x\in Q$ there exist  
	$y,z\in Q$ such that $kx=\delta(y)yz$. Then it follows that 
	$k=\delta(y)$. 
\end{proof}

A group $G$ admits a \textbf{triple factorization} if there are subgroups 
$A$, $B$ and $M$ such that $G=MA=MB=AB$ y $A\cap M=B\cap M=\{1\}$.
The following result is An immediate consequence of Sysak's theorem:

%%% TODO:
%%% ejemplos de factorizacion triple
%%% caracterizacion de 1-cociclos biyectivos
%%% ejemplo con anillo radical
%%% braces?


\begin{corollary}
	If the group $Q$ acts by automorphisms on $K$ and 
	$\delta\in\Der(Q,K)$ is surjective, then $G=K\rtimes Q$ admits a triple factorization. 
\end{corollary}

% \begin{proof}
% 	Es consecuencia inmediata del teorema~\ref{theorem:Sysak}. 
% \end{proof}
Another consequence: 

\begin{exercise}
	\label{xca:ker1cocycle}
	Let $\delta\in\Der(Q,K)$. 
	\begin{enumerate}
	\item Prove that $\delta$ is injective if and only if 
	$\ker\delta=\{1\}$.
	\item Prove that if $\delta$ is bijective, then  
	$K$ admits a complement 
	$\Delta$ in $K\rtimes Q$ such that $K\rtimes Q=K\rtimes\Delta=\Delta Q$ and 
	$Q\cap\Delta=\{1\}$.
	\end{enumerate}
\end{exercise}

% \begin{proof}
% 	Sean $x,y\in Q$ tales que $\delta(x)=\delta(y)$. Como $\delta(x^{-1}y)=1$
% 	pues 
% 	\[
% 	\delta(x^{-1}y)=\delta(x^{-1})(x^{-1}\delta(y)x)=\delta(x^{-1})x^{-1}\delta(x)x=\delta(x^{-1}x)=\delta(1)=1
% 	\]
% 	y $\delta$ es inyectiva, $x^{-1}y=1$. La afirmación recíproca es trivial.
% \end{proof}

% \begin{corollary}
% 	Si $\delta\in\Der(Q,K)$ es biyectivo entonces $K$ admite un complemento
% 	$\Delta$ en $K\rtimes Q$ tal que $K\rtimes Q=K\rtimes\Delta=\Delta Q$ y
% 	$Q\cap\Delta=1$.
% \end{corollary}

% \begin{proof}
% 	Vimos en el teorema de Sysak que $\delta$ es sobreyectiva si y
% 	sólo si $K\rtimes Q=\Delta Q$ y que $\ker\delta=Q\cap\Delta$.
% \end{proof}

%\section{Aplicación: subespacios invariantes}
%
%Sea $A$ un grupo que actúa por automorfismos en un grupo $G$. Definimos
%\[
%C_G(A)=\{g\in G:g\cdot a=a\text{ para todo $a\in A$}\}.
%\]
%
%Como aplicación de la teoría de Schur--Zassenhaus vamos a demostrar los
%teoremas de Sylow para subespacios $A$-invariantes.
%Necesitamos el siguiente lema:
%
%\begin{lemma}
%	\label{lemma:Glauberman}
%	Sean $A$ y $G$ grupos finitos de órdenes coprimos. Supongamos que $A$ actúa
%	por automorfismos en $G$ y que $A$ o $G$ es resoluble. Supongamos que $A$
%	actúa en un conjunto $X$ y que $G$ actúa transitivamente en $X$ de forma tal que
%	\begin{equation}
%		\label{equation:Glauberman:compatibilidad}
%		a\cdot (g\cdot x)=(aga^{-1})\cdot (a\cdot x)
%	\end{equation}
%	para todo $a\in A$, $g\in G$, $x\in X$. Valen las siguientes afirmaciones:
%	\begin{enumerate}
%		\item Existe un $x\in X$ invariante por la acción de $A$.
%		\item Si $x,y\in X$ son invariantes por la acción de $A$ entonces
%			existe $c\in C_G(A)$ tal que $c\cdot x=y$.
%	\end{enumerate}
%\end{lemma}
%
%\begin{proof}
%	Sea $\Gamma=G\rtimes A$ el producto semidirecto. Todo $\gamma$ se escribe
%	en forma única como $\gamma=ga$ con $g\in G$, $a\in A$. Veamos que $\Gamma$
%	actúa en $X$ por
%	\[
%		\gamma\cdot x=(ga)\cdot x=g\cdot (a\cdot x).
%	\]
%	Es fácil ver que es una acción pues la igualdad
%	\[
%	(ga)\cdot ((hb)\cdot x)=((ga)(hb))\cdot x=(gaha^{-1})\cdot ((ab)\cdot x)
%	\]
%	es consecuencia de la relación de
%	compatibilidad~\eqref{equation:Glauberman:compatibilidad}.\framebox{completar}
%
%\end{proof}
%\begin{theorem}
%	\label{theorem:Sylow_Ainv}
%\end{theorem}
%
%

\section*{B}

\begin{lemma}
	\label{lem:1cocycle}
	Let $G$ be a group and $N$ be a normal subgroup of $G$. If $G$ acts on $N$ by conjugation and  
	$\varphi\colon G\to N$ is a 1-cocycle with kernel $K$, then 
	$\varphi(x)=\varphi(y)$ if and only if $xK=yK$. In particular,
	$(G:K)=|\varphi(G)|$. 
\end{lemma}

\begin{proof}
	If $\varphi(x)=\varphi(y)$, then, since 
	\[
		\varphi(x^{-1}y)
		=\varphi(x^{-1})(x^{-1}\cdot\varphi(y))
		=\varphi(x^{-1})(x^{-1}\cdot\varphi(x))
		=\varphi(x^{-1}x)=\varphi(1)
		=1,
	\]
	we obtain that $xK=yK$. Conversely, if $x^{-1}y\in K$, then, since 
	\[
	1=\varphi(x^{-1}y)=\varphi(x^{-1})(x^{-1}\cdot \varphi(y)),
	\]
	we conclude that $\varphi(y)=x\cdot\varphi(x^{-1})^{-1}$. Thus
	$\varphi(x)=\varphi(y)$.

	The second claim is now trivial, as $\varphi$ is constant in each coset of $K$ and
	there are $(G:K)$ different possible values. 
\end{proof}

\begin{lemma}
	\label{lem:d}
	Let $G$ be a finite group, $N$ be a normal abelian subgroup of $G$ and $S$, $T$ and $U$
	be transversal of$N$ in $G$. Let  
	\[
	d(S,T)=\prod st^{-1}\in N,
	\]
	where the product is taken over all $s\in S$ and $t\in T$ such that 
	$sN=tN$. The following statements hold:
	\begin{enumerate}
		\item $d(S,T)d(T,U)=d(S,U)$.
		\item $d(gS,gT)=gd(S,T)g^{-1}$ for all $g\in G$.
		\item $d(nS,S)=n^{(G:N)}$ for all $n\in N$.
	\end{enumerate}
\end{lemma}

\begin{proof}
	If $s\in S$, $t\in T$ and $u\in U$ are such that $sN=tN=uN$, then, since $N$ is
	abelian and $(st^{-1})(tu^{-1})=su^{-1}$, 
	\[
		d(S,T)d(T,U)=\prod (st^{-1})(tu^{-1})=\prod su^{-1}=d(S,U).
	\]

	Since $sN=tN$ if and only if $gsN=gtN$ for all $g\in G$, 
	\[
	g\left(\prod st^{-1}\right)g^{-1}=\prod gst^{-1}g^{-1}=\prod (gs)(gt)^{-1}=d(gS,gT).
	\]

	Finally, since $N$ is normal, $nsN=sN$ for all $n\in N$. Thus 
	\[
		d(nS,S)=\prod (ns)s^{-1}=n^{(G:N)}.\qedhere
	\]
\end{proof}

We now prove the first version of Schur--Zassenhaus' theorem. 

\begin{theorem}[Schur--Zassenhaus]
	\index{Schur--Zassenhaus!theorem}
	\label{thm:SchurZassenhaus:abelian}
	Let $G$ be a finite group and $N$ be an abelian normal subgroup of $G$. If 
	$|N|$ and $(G:N)$ are coprime, then $N$ admits a complement in $G$. In this case, all complements
	of $N$ are conjugate in $G$.  
\end{theorem}

\begin{proof}
	Let $T$ be a transversal of $N$ in $G$. Let $\theta\colon G\to N$,
	$\theta(g)=d(gT,T)$. Since $N$ is abelian, Lemma~\ref{lem:d} implies that 
	$\theta$ is a 1-cocycle, where $G$ acts on $N$ by conjugation: 
	\begin{align*}
		\theta(xy)&=d(xyT,T)
		=d(xyT,xT)d(xT,T)\\
		&=(xd(yT,T)x^{-1})d(xT,T)=(x\cdot\theta(y))\theta(x).
	\end{align*}

	\begin{claim}
		$\theta|_N\colon N\to N$ is surjective.
	\end{claim}

	If $n\in N$, then  
	$\theta(n)=d(nT,T)=n^{(G:N)}$ by Lemma~\ref{lem:d}. Since $|N|$ and $(G:N)$ are coprime, 
	there exist $r,s\in\Z$ such that $r|N|+s(G:N)=1$. Thus
	\[
		n=n^{r|N|+s(G:N)}=(n^s)^{(G:N)}=\theta(n^s).
	\]

	Let $H=\ker\theta$. We claim that $H$ is a complement for $N$. 
	We know that $H$ is a subgroup of $G$. Since 
	\[
		|N|=|\theta(G)|=(G:H)=\frac{|G|}{|H|}
	\]
	by Lemma~\ref{lem:1cocycle}, it follows that $N\cap H=\{1\}$ because 
	$|N|$ and $(G:N)=|H|$ are coprime. Since $|NH|=|N||H|=|G|$, we conclude that 
	$G=NH$ and hence $H$ is a complement of $N$ in $G$.

	We now prove that two complements of $N$ in $G$ are conjugate. Let  
	$K$ be a complement of $N$ in $G$. Since $NK=G$ and $N\cap K=\{1\}$, it follows that 
	$K$ is a transversal of $N$ in $G$. Let $m=d(T,K)\in N$. Since $\theta|_N$ is surjective, 
	there exists $n\in N$ such that $\theta(n)=m$. By 
	Lemma~\ref{lem:d}, for each $k\in K$, 
	\[
	kmk^{-1}=kd(T,K)k^{-1}=d(kT,kK)=d(kT,K)=d(kT,T)d(T,K)=\theta(k)m
	\]
	holds. Since $N$ is abelian,
	$\theta(n^{-1})=m^{-1}$ and hence  
	\begin{align*}
		\theta(nkn^{-1})&=\theta(n)n\theta(kn^{-1})n^{-1}
		=m\theta(kn^{-1})\\
		&=m\theta(k)k\theta(n^{-1})k^{-1}
		=m\theta(k)km^{-1}k^{-1}=1.
	\end{align*}
	Therefore $nKn^{-1}\subseteq H=\ker\theta$. Since
	$|K|=(G:N)=|H|$, we conclude that $nKn^{-1}=H$.
\end{proof}

The general version of 
Schur--Zassenhaus' theorem does not need $N$ to be abelian. 

% todo: necesito frattini
% lo quito? pongo solamente una referencia?

\begin{theorem}[Schur--Zassenhaus]
	\index{Schur--Zassenhaus!theorem}
	\label{thm:SchurZassenhaus}
	Let $G$ be a finite group and $N$ be a normal subgroup of $G$. If 
	$|N|$ and $(G:N)$ are coprime, then $N$ admits a complement in $G$. 
\end{theorem}

\begin{proof}
	We proceed by induction on $|G|$. If there exists a proper subgroup $K$ of
	$G$ such that $NK=G$, then, since $(K:K\cap N)=(G:N)$ is coprime with $|N|$, it coprime with  
	$|K\cap N|$. Moreover, $K\cap N$ is normal in $K$. By inductive hypothesis, 
	$K\cap N$ admits a complement in $K$. Hence there exists a subgroup 
    $H$ of $K$ such that 
    \[
    |H|=(K:K\cap N)=(G:N).
    \] 
	
	Supongamos entonces que no existe un subgrupo propio $K$ de $G$ tal que
	$NK=G$.  Podemos suponer que $N\ne1$ (de lo contrario, basta tomar $G$ como
	complemento de $N$ en $G$).  Como $N$ está contenido en todo subgrupo
	maximal de $G$ (pues si existe un maximal $M\subsetneq G$ tal que
	$N\not\subseteq M$ entonces $NM=G$), se tiene que $N\subseteq\Phi(G)$. Por
	el teorema de Frattini ~\ref{theorem:Frattini}, $\Phi(G)$ es nilpotente y
	luego $N$ es nilpotente; en particular, $Z(N)\ne\{1\}$. Sea $\pi\colon G\to
	G/Z(N)$ el morfismo canónico. Como $N$ es normal en $G$ y $Z(N)$ es
	característico en $N$, $Z(N)$ es normal en $G$.  Además 
	\[
	(\pi(G):\pi(N))=\frac{|\pi(G)|}{|\pi(N)|}=\frac{|G/Z(N)|}{|N/N\cap Z(N)|}=(G:N)
	\]
	es coprimo con $|N|$, y entonces es también coprimo con $|\pi(N)|$. Por hipótesis
	inductiva, $\pi(N)$ admite un complemento en $G/Z(N)$, digamos $\pi(K)$
	para algún subgrupo $K$ de $G$. Luego $G=NK$ pues 
	$\pi(G)=\pi(N)\pi(K)=\pi(NK)$. 
	Como entonces $K=G$ (pues sabíamos que no existe $K$ tal que $G=NK$), 
	$\pi(N)$ es abeliano pues 
	\[
		\pi(Z(N))=\pi(N)\cap\pi(K)=\pi(N)\cap\pi(G)=\pi(N).
	\]
	Luego $N\subseteq Z(N)$ es abeliano y entonces, por el
	teorema~\ref{theorem:SchurZassenhaus:abeliano}, el subgrupo $N$ admite un
	complemento. 
\end{proof}

\begin{theorem}
	\label{them:SchurZassenhaus:conjugacion}
	Let $G$ be a fintie group and $N$ be a normal subgroup of $G$ such that $|N|$ and 
	$(G:N)$ are coprime. If either $N$ or $G/N$ is solvable, then all complements of $N$ in $G$ 
	are conjugate. 
\end{theorem}

%\begin{proof}
%	Sea $G$ un contraejemplo minimal, es decir: existen complementos $K_1$ y
%	$K_2$ a $N$ en $G$ que no son conjugados.
%
%	\begin{claim}
%		$N$ es minimal en $G$.
%	\end{claim}
%
%	Si $M\subseteq N$ es minimal normal en $G$, $M\ne1$ pues $N\ne1$. Sea
%	$\pi\colon G\to G/M$ el morfismo canónico. El grupo $\pi(G)$ contiene un
%	subgrupo normal $\pi(N)$ de índice coprimo con $|\pi(N)|$. Además
%	$\pi(K_1)$ y $\pi(K_2)$ complementan a $\pi(N)$. Como $|G|$ es minimal,
%	$\pi(K_1)$ y $\pi(K_2)$ son conjugados en $\pi(G)$, es decir: existe $x\in G$ tal que 
%	$\pi(K_1)=\pi(xK_2x^{-1})$.
%
%\end{proof}

\begin{proof}
	Let $G$ be a minimal counterexample, so there are complements $K_1$ and 
	$K_2$ of $N$ in $G$ such that $K_1$ and $K_2$ are not conjugate and $|G|$ is minimal with this
	property. 
	
	\begin{claim}
		Each subgroup $U$ of $G$ satiasfies the hypotheses of the theorem with
		respect to the normal subgroup $U\cap N$.
%		Sea $U$ un subgrupo de $G$. Entonces $U$ satisface las hipótesis del
%		teorema con respecto al subgrupo normal $U\cap N$. Si $U$ contiene un
%		complemento $H$ para $N$ en $G$, entonces $H$ complementa a $U\cap N$
%		en $U$.
	\end{claim}
	
	Since $N$ is normal in $G$, the subgroup $U\cap N$ is normal in $U$. Moreover, $|U\cap N|$ and 
	$(U:U\cap N)$ are coprime, as $|U\cap N|$ divides $|N|$ and $(U:U\cap
	N)=(UN:N)$ divides $(G:N)$. If $G/N$ is solvable, then $U/U\cap N$
	is solvable since $U/U\cap N$ is isomorphic to a subgroup of $G/N$. If $N$ is
	solvable, the subgroup $U\cap N$ is solvable. 
	
%	Como $|H|$ divide a $|U|$ y $|H|$ es coprimo con $|U\cap N|$, se tiene que
%	$|H|$ divide a $(U:U\cap N)$. Como además $(U:U\cap N)$ divide a
%	$(G:N)=|H|$, se concluye que $|H|=|U:U\cap N|$. Luego $H$ complementa a
%	$U\cap N$ en $U$.

	\begin{claim}
		If thre is a normal subgroup $L$ of $G$ such that $\pi(N)$ is normal in 
		$\pi(G)$, where $\pi\colon G\to G/L$ is the canonical map, then 
		$\pi(G)$ satifies the hypotheses of the theorem with respect to $\pi(N)$.
		In this case, if $H$ is a complement of $N$ in $G$, then $\pi(H)$ is a complement
		of $\pi(N)$ in $\pi(G)$.
	\end{claim}

	If $N$ is solvable, then $\pi(N)$ is solvable. If $G/N$ is solvable, then the group 
	$\pi(G)/\pi(N)\simeq G/NL$ is solvable. Moreover, 
	$(\pi(G):\pi(N))=\frac{|G/L|}{|N/N\cap L|}$ divides the index $(G:N)$ of $N$ in $G$. 
	
	If $H$ is a complement of $N$ in $G$, $|\pi(H)|$ and $|\pi(N)|$ are coprime.
	Thus $\pi(H)$ is a complement of $\pi(N)$, as 
	$\pi(G)=\pi(N)\pi(H)=\pi(NH)$ and 
	$\pi(N)\cap\pi(H)=\{1\}$. 

	\begin{claim}
		$N$ is minimal normal in $G$.
	\end{claim}

	Let $M\ne\{1\}$ be a normal subgroup of $G$ such that $M\subseteq N$. Let $\pi\colon G\to G/M$
	be the canonical map. The group $\pi(G)$ satiafies the hypotheses of the theorem
	with respect to the normal subgroup $\pi(N)$. By the minimality of $|G|$, there exists 
	$x\in G$ such that $\pi(xK_1x^{-1})=\pi(K_2)$. The subgroup 
	$U=MK_2$ satisfies the hypotheses of the theorem with respect to the normal subgroup 
	$U\cap N$. Since $xK_1x^{-1}\cup K_2\subseteq U$,
	we conclude that $xK_1x^{-1}$ and $K_2$ are complements of $U\cap N$ in $U$.
	Thus $MK_2=G$, as $xK_1x^{-1}$ and $K_2$ are not conjugate and $|G|$ is minimal. 
	Therefore $M=N$, as  
	\[
		\frac{|K_2|}{|M\cap K_2|}=(MK_2:M)=(G:M)=\frac{|NK_2|}{|M|}=(N:M)|K_2|.
	\]

	\begin{claim}
		$N$ is not solvable and $G/N$ is solvable. 
	\end{claim}
	
	Otherwise, by Lemma~\ref{lemma:minimal_normal}, 	$N$ is minimal normal
	and hence abelian. This yields a contradiction, as 
	the previous version of the Schur--Zassenhaus's theorem implies that
	$K_1$ and $K_2$ are conjugate. 
	
	\medskip
	Let $p\colon G\to G/N$ be the canonical map and $S$ be such that $p(S)$
	is minimal normal in $p(G)=G/N$.  By Lemma~\ref{lemma:minimal_normal},
	$p(S)$ is a $p$-group for some prime number $p$.  Since $G=NK_1=NK_2$ and $N\subseteq
	S$, Dedekind's lemma implies that  
	\[
	S=N(S\cap K_1)=N(S\cap K_2).
	\]
	Thus $S\cap K_1$ and $S\cap K_2$
	are complements of $N$ in $S$. Since 
	\[
	p(S)=p(S\cap K_1)=p(S\cap K_2)
	\]
	is a $p$-group,
	$p$ divides $|S|$. The group $S$ 
	satisafies the hypotheses of the theorem 
	with respect to the normal subgroup $N$,
	so $|N|$ and $(S:N)$ are coprime. If $p\mid |N|$, then  
	$p\nmid (S:N)=|p(S)|$, a contradiction. Therefore $p\nmid |N|$. 
	This implies that $S\cap K_1$ and $S\cap K_2$ are
	Sylow $p$-subgroups of $S$, as 
	\[
		|S\cap K_1|=(S:N)=|S\cap K_2|.
	\]
	By the second Sylow's theorem, there exists $s\in
	S$ such that 
	\[
	S\cap sK_1s^{-1}=S\cap K_2.
	\]
	In particular, $S\ne G$ by the minimality of $|G|$.
	Let  
	\[
		L=S\cap K_2=S\cap sK_1s^{-1}\ne\{1\}.
	\]
	Since $S$ is normal in $G$, it follows that $sK_1s^{-1}\cup K_2\subseteq N_G(L)$ (because $L$
	is normal both in $sK_1s^{-1}$ and in $K_2$). The subgroups $sK_1s^{-1}\subseteq
	N_G(L)$ and $K_2\subseteq N_G(L)$ are complements of $N\cap N_G(L)$ in $N_G(L)$. Thus 
	$N_G(L)=G$ by the minimality of $|G|$ (if $N_G(L)\ne G$, then both 
	$sK_1s^{-1}$ and $K_2$ are conjugate in $G$ because they are conjugate in $N_G(L)$). Therefore 
	$L$ is normal in $G$. 
	
	Let $\pi_L\colon G\to G/L$ be the canonical map. Since both 
	$\pi_L(K_1)$ and $\pi_L(K_2)$ are complements of $\pi_L(N)$ in $\pi_L(G)$, the minimality 
	of $|G|$ implies that there exists $g\in G$ such that $\pi_L(gK_1g^{-1})=\pi_L(K_2)$, so  
	there exists $g\in G$ such that $(gK_1g^{-1})L=K_2L$.  Thus $gK_1g^{-1}\cup
	K_2\subseteq \langle K_2,L\rangle=K_2$, as $L\subseteq K_2$. In conclusion,	
	$gK_1g^{-1}=K_2$, a contradiction to the minimality of $|G|$. 
\end{proof}

By Feit--Thompson´s theorem, we do not need to assume that
$N$ or $G/N$ is solvable. Indeed, since every group of odd order is solvable and
$|N|$ and $(G:N)$ are coprime, it follows that either $|N|$ or $(G:N)$ is odd. 

\section*{C}
%
%As an application of Schur--Zassenhaus' theorem we present
%Hall's theory of $\pi$-groups.  

\index{$\pi$-number}
\index{$\pi$-group}
\index{$\pi$-subgroup}
Let $G$ be a finite group and $\pi$ be a set of (positive) prime numbers. We say that
$G$ is a $\pi$-group if all prime divisors of $|G|$ belong to $\pi$.
A $\pi$-subgroup of $G$ is a subgroup of $G$ that is also a $\pi$-group. 
A $\pi$-number is an integer with all prime divisors in 
$\pi$. The complement of $\pi$ is the set of prime numbers will be denoted by 
$\pi'$. A $\pi'$-number is then an integer not divisible by the primes 
of $\pi$.

\medskip
\index{Hall!subgroup}
Let $\pi$ be a set of primes.A subgroup $H$ of a group $G$ is a 
\textbf{Hall $\pi$-subgroup} if $H$ is a $\pi$-subgroup of $G$ and the index $(G:H)$
is a $\pi'$-number.

\begin{theorem}[Hall]
	\index{Hall's!theorem}
	\label{thm:HallE}
	Let $\pi$ be a set of primes and $G$ be a finite solvable group. Then 
	$G$ admits a Hall $\pi$-subgroup. 
\end{theorem}

\begin{proof}
	Assume that $|G|=nm>1$ with $\gcd(n,m)=1$. We prove by induction on  
	$|G|$ that there exists a subgroup of order $m$. Let $K$ 
	be a minimal normal subgroup of $G$ 
	and let $\pi\colon G\to G/K$ be the canonical map. (We are using $\pi$ for a fixed set
	of primes and for the canonical map $G\to G/K$, but hopefully no confusion will arise.) 
	Since $G$ is solvable, $K$ is an abelian $p$-group by Lemma~\ref{lemma:minimal_normal}.
	
	There are two cases to consider. Assume first that $p$ divides $m$. Since 
	$|G/K|<|G|$, the inductive hypothesis and the correspondence theorem imply that there exists 
	a subgroup $J$ of $G$ containing $K$ such that $\pi(J)$ is a subgroup of 
	$\pi(G)=G/K$ of order $m/|K|$. Thus $|J|=m$ since  
	\[
	m/|K|=|\pi(J)|=\frac{|J|}{|K\cap J|}=(J:K).
	\]

	Assume now tht $p$ does not divide $m$. The inductive hypothesis and the correspondence theorem 
	imply that there exists a subgroup $H$ of $G$ containing $K$ such that 
	$\pi(H)$ is a subgroup of $G/K$ of order $m$.  Since $|H|=m|K|$, $K$ 
	is normal in $H$ and $|K|$ is coprime with $|H:K|$, 
	Schur--Zassenhaus's theorem (Theorem~\ref{thm:SchurZassenhaus:abelian}) implies 
	that there exists a complement 
	$J$ of $K$ in $H$. Thus $J$ is a subgroup of $G$ of order 
	$|J|=m$.
\end{proof}

\begin{example}
	The group $\Alt_5$ contains a Hall $\{2,3\}$-subgroup isomorphic to 	$\Alt_4$.
\end{example}

\begin{example}
	The simple group $\PSL_3(2)$ of order $168$ does not contain Hall $\{2,7\}$-subgroups. 
\end{example}


\begin{theorem}[Hall]
	\index{Hall's!theorem}
	\label{thm:HallC}
	Let $G$ be a finite solvable group and $\pi$ be a set of primes. All 
	Hall $\pi$-subgroups of $G$ are conjugate. 
\end{theorem}

\begin{proof}
	Podemos suponer que $G\ne\{1\}$. Procederemos por inducción en $|G|$.  Sean $H$
	y $K$ dos $\pi$-subgrupos de Hall de $G$. Sea $M$ un subgrupo de $G$
	minimal-normal y sea $\pi\colon G\to G/M$ el morfismo canónico. Como $G$ es
	resoluble, el lema~\ref{lemma:minimal_normal} implica que  $M$ es un
	$p$-grupo para algún primo $p$.  Como $\pi(H)$ y $\pi(K)$ son
	$\pi$-subgrupos de Hall de $G/M$, los subgrupos $\pi(H)$ y $\pi(K)$ con
	conjugados en $G/M$. Luego existe $g\in G$ tal que $gHMg^{-1}=KM$. 

	Hay dos casos a considerar. Supongamos primero que $p\in\pi$. Como $|HM|$ y
	$|KM|$ son $\pi$-números y $|H|=|K|$ es el mayor $\pi$-número que divide al
	orden de $G$, se concluye que $H=HM$ y $K=KM$. En particular, $gHg^{-1}=K$. 

	Supongamos ahora que $p\not\in\pi$. Es evidente que $K$ complementa a $M$ en
	$KM$ pues $K\cap M=1$. Veamos que $gHg^{-1}$ complementa a $M$ en $KM$:
	como $M$ es normal en $G$, 
	\[
	(gHg^{-1})M=gHMg^{-1}=KM,
	\]
	y $gHg^{-1}\cap M=1$ ya que $p\not\in\pi$. Estos complementos tienen que
	ser conjugados por el teorema de
	Schur--Zassenhaus~\ref{theorem:SchurZassenhaus:conjugacion}.
\end{proof}

\begin{corollary}
	Sea $G$ un grupo finito y sea $N$ un subgrupo normal de $G$ de orden $n$.
	Supongamos que $N$ o $G/N$ es resoluble. Si $|G:N|=m$ es coprimo con $n$ y
	$m_1$ divide a $m$, todo subgrupo de $G$ de orden $m_1$ está contenido en
	algún subgrupo de orden $m$.
\end{corollary}

\begin{proof}
	Sea $H$ un complemento para $N$ en $G$. Entonces $|H|=m$. Sea $H_1$
	subgrupo de $G$ tal que $|H_1|=m_1$. 
	Como $n$ y $m$ son coprimos, $m_1=|H_1|=|H\cap NH_1|$ pues
	\[
	\frac{|H||N||H_1|}{|H\cap NH_1|}=
	\frac{|H||NH_1|}{|H\cap NH_1|}=|H(NH_1)|=|G|=|NH|=|N||H|.
	\]
	Como $H_1$ y $H\cap NH_1$ son complementos para $N$ en $NH_1$, ambos de
	orden coprimo con $n$, existe $g\in G$ tal que $H_1=g(H\cap NH_1)g^{-1}$. Luego 
	$H_1\subseteq gHg^{-1}$ y entonces $|gHg^{-1}|=m$. 
\end{proof}

\section*{D}

Let $A$ be an additive group and $G$ be a group and let 
$G\times A\to A$, $(g,a)\mapsto g\cdot a$,
is a left action of $G$ on $A$ by automorphisms. This means that the action of $G$ on $A$ satisfies 
$g\cdot (a+b)=g\cdot a+g\cdot b$ for all $g\in G$ and $a,b\in A$.
A \emph{bijective
$1$-cocyle} is a bijective map $\pi\colon G\to A$ such that 
\begin{equation}
    \label{eq:1cocycle}
    \pi(gh)=\pi(g)+g\cdot \pi(h)
\end{equation}
for all $g,h\in G$. 
We now prove the equivalence between braces and bijective 1-cocycles. 

\begin{theorem}
	\label{thm:1cocycle}
    Over any additive group $A$ the following data are equivalent:
    \begin{enumerate}
        \item A group $G$ and a bijective
            1-cocycle $\pi\colon G\to A$. 
        \item A brace structure over $A$. 
    \end{enumerate}

    \begin{proof}
        Consider on $A$ a second group structure given by 
        \[
		a\circ b=\pi(\pi^{-1}(a)\pi^{-1}(b))=a+\pi^{-1}(a)\cdot b
		\]
		for all
        $a,b\in A$.  Since $G$ acts on $A$ by
        automorphisms, 
        \begin{align*}
            a\circ (b+c)&=\pi(\pi^{-1}(a)\pi^{-1}(b+c))=a+\pi^{-1}(a)\cdot (b+c)\\
            &=a+ \pi^{-1}(a)\cdot b+\pi^{-1}(a)\cdot c
            =a\circ b-a+a\circ c
        \end{align*}
        holds for all $a,b,c\in A$.
        
        Conversely, assume that the additive group $A$ has a brace structure. Let $G$ be the multiplicative group of $A$
        and $\pi=\id$. By
        Proposition~\ref{pro:lambda}, $a\mapsto\lambda_a$, is a group homomorphism and 
        hence $G$ acts on $A$ by automorphisms. Then~\eqref{eq:1cocycle} holds
        and therefore $\pi\colon G\to A$ is a bijective 1-cocycle. 
    \end{proof}
\end{theorem}

The construction of Theorem~\ref{thm:1cocycle} is functorial, see Exercise~\ref{prob:1cocycle}.

\begin{example}
	\label{exa:d8q8}
	Let 
	\[
	D_4=\langle r,s:r^4=s^2=1,srs=r^{-1}\rangle
	\]
	be the dihedral group of eight elements and let
	\[
	Q_8=\{1,-1,i,-i,j,-j,k,-k\}
	\]
	be the quaternion group of eight elements.  Let
	$\pi:Q_8\to D_4$ be given by 
	\begin{align*}
		1\mapsto 1 &, & -1\mapsto r^2 &,  & -k\mapsto r^3s &,&  k\mapsto rs &,\\
		i\mapsto s &, & -i\mapsto r^2s &, &  j\mapsto r^3 &, & -j\mapsto r &.
	\end{align*}
	Since $\pi$ is bijective, 
	a straightforward calculation shows that $D_4$ with 
	\[
	  x+y=xy,\quad 
	  x\circ y=\pi(\pi^{-1}(x)\pi^{-1}(y))
	\]
	is a two-sided brace with additive group isomorphic to $D_4$ and multiplicative group
	isomorphic to $Q_8$. 
\end{example}

\section*{Exercises}

\begin{prob}
\label{prob:1cocycle}
Let $\pi\colon G\to A$ and $\eta\colon H\to B$ be bijective 1-cocycles.  A
\emph{homorphism} between these bijective 1-cocycles is a pair $(f,g)$ of
group homomorphisms  $f\colon G\to H$, $g\colon A\to B$ such that
\begin{align*}
&\eta\circ f=g\circ \pi,\\
&g(h\cdot a)=f(h)\cdot g(a),&&a\in A,\;h\in G.
\end{align*}
Bijective 1-cocycles and homomorphisms form a category. 
For a given additive group $A$ 
the full subcategory of the category of bijective 1-cocycles with objects
$\pi\colon G\to A$ is equivalent to the full subcategory of the category of
braces with additive group $A$. 
\end{prob}

\section*{Open problems}

\section*{Notes}

In the case of braces of abelian type, Theorem~\ref{thm:1cocycle} is implicit in the work of Rump, see \cite{MR2278047,MR3291816} or~\cite{MR3177933}. Similar results appear 
in the work of Etingof, Schedler and Soloviev~\cite{MR1722951}, Lu, Yan and Zhu~\cite{MR1769723} 
and Soloviev~\cite{MR1809284}.
In~\cite{MR1653340} Etingof and Gelaki give a method of constructing finite-dimensional complex semisimple triangular Hopf algebras. They show how any non-abelian group which admits a bijective 1-cocycle gives rise to a semisimple minimal triangular Hopf algebra which is not a group algebra.