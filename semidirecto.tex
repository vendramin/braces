

\chapter{}

We will use the following theorem of Kegel and Wielandt:

\begin{theorem}[Kegel--Wielandt]
\end{theorem}

\begin{theorem}

Let $A$ be a finite skew left brace with nilpotent multiplicative group. Then $A$ is of solvable type.
\end{theorem}

\begin{proof}
...
\end{proof}


\chapter{Producto semidirecto}

Si $A$ y $B$ son brazas, una acción de $A$ en $B$ se define
como un morfismo de grupos $\sigma\colon (B,\circ)\to\Aut_{Br}(A)$.

\begin{definition}
\index{Producto!semidirecto de brazas}
Sean $A$ y $B$ brazas y supongamos que $B$ actúa en $A$.  Se 
define el producto semidirecto 
$A\rtimes_{\sigma}B$ como la estructura de braza 
en el producto cartesiano $A\times B$ 
dada por las operaciones
\begin{align*}
&(a_1,b_1)+(a_2,b_2)=(a_1+a_2,b_1+b_2),\\
&(a_1,b_1)\circ (a_2,b_2)=(a_1\circ\sigma(b_1)(a_2),b_1\circ b_2),
\end{align*}
donde $a_1,a_2\in A$ y $b_1,b_2\in B$.
\end{definition}

Un cálculo directo nos permite demostrar que
\begin{align}
\lambda_{(a_1,b_1)}(a_2,b_2)&=(\lambda_{a_1}(\sigma(b_1)(a_2)),\lambda_{b_1}(b_2)),
\end{align}

\begin{exercise}
Demuestre que en el producto semidirecto $A\rtimes_{\sigma}B$ vale 
\[
(a_1,b_1)*(a_2,b_2)=(\lambda_{a_1}(\sigma(b_1)(a_2))-a_2,b_1*b_2).
\]
\end{exercise}

\begin{theorem}
	Sean $A$ y $B$ brazas. El producto semidirecto
	$A\rtimes_{\sigma}B$ es nilpotente a derecha si y sólo si $A$ y $B$ son nilpotentes a derecha. 
\end{theorem}

\begin{proof}
	Sea $P=A\rtimes_{\sigma}B$. 
	Si $P^{(n)}=0$ para algún $n$, entonces $A^{(n)}=0$ y $B^{(n)}=0$. 
	Recíprocamente, si $A^{(k)}=0$ y $B^{(l)}=0$, entonces $P^{(k+l)}=0$.
	\framebox{FIXME}
\end{proof}

% FIXME: un resultado similar vale para producto corona!

Para un subconjunto $X$ del producto semidirecto $A\rtimes_{\sigma}B$, definimos
\begin{align*}
    &\pi_A(X)=\{a\in A:(a,b)\in X\text{ para algún $b\in B$}\},\\
    &\pi_B(X)=\{b\in B:(a,b)\in X\text{ para algún $a\in A$}\}.
\end{align*}

\begin{lemma}
Si $I$ es un ideal del producto semidirecto $A\rtimes_{\sigma}B$, entonoces 
$\pi_B(I)$ es un ideal de $B$.
\end{lemma}

\begin{proof}
    Veamos que $(\pi_B(I),+)$ es un subgrupo de $(B,+)$. Como $(0,0)\in I$, entonces $0\in\pi_B(I)$. Además si $b_1,b_2\in B$, sean 
    $a_1,a_2\in A$ tales que $(a_j,b_j)\in I$, $j\in\{1,2\}$. Como entonces
    \[
    (a_1,b_1)-(a_2,b_2)=(a_1-a_2,b_1-b_2)\in I,
    \]
    y además $a_1-a_2\in A$, se concluye que $b_1-b_2\in \pi_B(I)$. 
    
    Veamos ahora que $(\pi_B(I),+)$ es normal en $(B,+)$. Sean $b\in \pi_B(I)$ y $a\in A$ 
    tal que $(a,b)\in I$. Si $y\in B$, entonces, como $(I,+)$ es normal en $(A\rtimes_{\sigma}B,+)$, tenemos que
    \[
    (0,y)+(a,b)-(0,y)=(a,y+b-y)\in I,
    \]
    que implica que $y+b-y\in \pi_B(I)$.
    
    Veamos que si $y\in B$, entonces $\lambda_y(\pi_B(I))\subseteq \pi_B(I)$. Sea $b\in\pi_B(I)$ y sea $a\in A$ tal que $(a,b)\in I$. Entonces
    \[
    (\sigma(y)(a),\lambda_y(b))=(\lambda_0(\sigma(y)(a),\lambda_y(b))=\lambda_{(0,y)}(a,b)\in I
    \]
    pues $(a,b)\in I$ y sabemos que $I$ es un ideal. Como $\sigma(y)(a)\in A$, se concluye que
    $\lambda_y(b)\in \pi_B(I)$.

    Queda demostrar que $(\pi_B(I),\circ)$ es normal en $(B,\circ)$. Si $y\in B$ y $b\in \pi_B(I)$, 
    sea $a\in A$ tal que $(a,b)\in I$. Sabemos que existe $x\in A$ tal que 
    \[
    (0,y)\circ (a,b)\circ (0,y)'=(x,y\circ b\circ y').
    \]
    Como $I$ es un ideal, $(x,y\circ b\circ y')\in I$ y luego $y\circ b\circ y'\in \pi_B(I)$. 
\end{proof}

\begin{lemma}
Si $I$ es un ideal del producto semidirecto $A\rtimes_{\sigma}B$ tal que 
$\pi_B(I)=0$, entonces $\pi_A(I)$ es un ideal de $A$.
\end{lemma}

\begin{proof}
    Como hicimos en el lema anterior, vemos que $(\pi_A(I),+)$ es un subgrupo normal de $(A,+)$. Si $x\in A$ y $a\in \pi_A(I)$, entonces
    \[
    (\lambda_x(a),b)=(\lambda_x(\sigma(0)(a)),\lambda_0(b))=\lambda_{(x,0)}(a,b)\in I
    \]
    y luego $\lambda_x(a)\in\pi_A(I)$. Para ver que $(\pi_A(I),\circ)$ es normal en $(A,\circ)$ basta
    observar que si $x\in A$ y $a\in\pi_A(I)$ entonces
    \[
    (x,0)\circ(a,b)\circ(x,0)'=(x\circ a\circ x',b)
    \]
    pues si $b\in B$ es tal que $(a,b)\in I$, entonces $x\circ a\circ x'\in \pi_A(I)$.
\end{proof}

\begin{definition}
    \index{Ideal!semiprimo}
    Sea $A$ una braza. Diremos que $A$ es \textbf{semiprima} si el único ideal $I$ de $A$ 
    tal que $I*I=0$ es el ideal nulo.
\end{definition}

\begin{theorem}
\label{thm:sd_semiprime}
Si $A$ y $B$ son brazas semiprimas, entonces el producto semidirecto
$A\rtimes_{\sigma}B$ es también una braza semiprima.
\end{theorem}

\begin{proof}
    Sea $I$ un ideal de $A\rtimes_\sigma B$ tal que $I*I=0$. Para ver que $I=0$ basta
    con demostrar que $\pi_A(I)=0$ y que $\pi_B(I)=0$

    Primero vamos a demostrar que $\pi_B(I)=0$. Como $\pi_B(I)$ es un ideal de $B$, 
    y $B$ es semiprima, basta ver que $\pi_B(I)*\pi_B(I)=0$. Sean $b_1,b_2\in \pi_B(I)$ y sean 
    $a_1,a_2\in A$ tales que $(a_j,b_j)\in I$ para todo $j\in\{1,2,\}$. Como 
    \[
    (x,b_1*b_2)=(a_1,b_1)*(a_2,b_2)\in I*I=0
    \]
    para algún $x\in A$, se tiene que $b_1*b_2=0$.
    
    Vamos a demostrar ahora que $\pi_A(I)=0$. 
    Como $\pi_B(I)=0$, sabemos que $\pi_A(I)$ es un ideal de $A$. Como $A$ es semiprimo, para ver que $\pi_A(I)=0$ basta entonces ver que $\pi_A(I)*\pi_A(I)=0$. Sean $a_1,a_2\in \pi_A(I)$ y sean
    $b_1,b_2\in B$ tales que $(a_j,b_j)\in I$ para todo $j\in\{1,2\}$. Como $b_j\in \pi_B(I)=0$ para todo $j$, se concluye que $(a_j,0)\in I$ para todo $j$. Luego
    \[
    (a_1*a_2,0)=(a_1,0)*(a_2,0)\in I*I=0
    \]
    y entonces $a_1*a_2=0$. 
\end{proof}

\begin{lemma}
    Sean $A$ y $B$ brazas. Demuestre que
    \[
    W=\{f\colon B\to A\text{ tal que $|\{b\in B:f(b)\ne 0\}|<\infty$}\}
    \]
    es una braza con las operaciones
    \begin{align*}
        &(f_1+f_2)(b)=f_1(b)+f_2(b),
        \\
        &(f_1\circ f_2)(b)=f_1(b)\circ f_2(b).
    \end{align*}
\end{lemma}

\begin{lemma}
    If $I$ is an ideal of $W$ and $b\in B$, then
    \[
    J_b=\{a\in A:f(b)=a\text{ for some $f\in I$}\}
    \]
    is an ideal of $A$.
\end{lemma}

\begin{proof}
    Observemos que la función nula $0_W\colon B\to A$ pertenece a $W$. Es fácil
    ver que $(J_b,+)$ es un subgrupo de $(W,+)$ pues $0_W\in W$ y además
    $a_1-a_2=(f_1-f_2)(b)$ si $f_1(b)=a_1$ y $f_2(b)=a_2$. 
    
    Dado $x\in A$ definimos la función $\delta_x\colon B\to A$ como
    \[
    \delta_x(y)=\begin{cases}
    x & \text{if $y=b$},\\
    0 & \text{otherwise}.
    \end{cases}
    \]
    Como $\delta_x(y)\ne 0$ si y sólo si $y=b$, se tiene que $\delta_x\in W$. Para ver
    que $(J_b,+)$ es normal en $(W,+)$ basta obvervar que si $x\in A$ y $a\in J_b$, digamos
    con $f(b)=a$ para algún cierto $f\in I$, entonces 
    \[
    x+a-x=(\alpha_x+f-\alpha_x)(b)\in J_b
    \]
    pues $\alpha_x+f-\alpha_x\in I$. Similarmente $\lambda_x(J_b)\subseteq J_b$ pues
    \[
    \lambda_x(a)=(-\alpha_x+\alpha_x\circ f)(b)
    \]
    y sabemos que $-\alpha_x+\alpha_x\circ f\in I$ pues $I$ es un ideal y $f\in I$. Por último,
    la normalidad de $(J_b,\circ)$ en $(W,\circ)$ es similar pues podemos escribir
    \[
    x\circ a\circ x'=(\alpha_x\circ f\circ\alpha_x^{-1})(b)
    \]
    y sabemos que $\alpha_x\circ f\alpha_x^{-1}\in I$.
\end{proof}

\begin{proposition}
Si $A$ es una braza semiprima, entonces $W$ también es semiprima.
\end{proposition}

\begin{proof}
Sea $I$ un ideal de $W$ tal que $I*I=0$. Sabemos que para cada $b\in B$, 
el conjunto $J_b$ es un ideal de $A$. Para demostrar que $I=0$ alcanza con 
demostrar que todos los $J_b$ son cero. Fijemos $b\in B$
y sean $a_1,a_2\in J_b$. Entonces existen $f_1,f_2\in I$ tales que $f_i(b)=a_i$
para todo $i\in\{1,2\}$. Como
\[
a_1*a_2=f_1(b)*f_2(b)=(f_1*f_2)(b)
\]
y $f_1*f_2\in I*I=0$, se tiene que $J_b*J_b=0$. Como $A$ es semiprimo, $J_b=0$. 
Veamos ahora que $I=0$. Sea $f\in I$. Si $b\in B$, entonces $f(b)\in J_b=0$. 
\end{proof}

\begin{definition}
    \index{Producto!corona de brazas}
    Sean $A$ y $B$ brazas. Se define el \textbf{producto corona} $A\wr B$ como la braza
    dada por el producto semidirecto $W\rtimes_{\sigma} B$, donde
    $\sigma\colon B\to\Aut(W)$ está definido por
    \[
    \sigma(b)(f)(y)=f(by)
    \]
    para todo $y,b\in B$ y $f\in W$.
\end{definition}

\begin{theorem}
\label{thm:wreath_semiprime}
Si $A$ y $B$ son brazas semiprimas, entonces $A\wr B$ es semiprima.
\end{theorem}

\begin{proof}
Como $A$ es semiprima, la braza $W$ es también semiprima. El producto corona 
$A\wr B=W\rtimes_{\sigma}B$ es también una braza semiprima por ser producto semidirecto de brazas semiprimas.
\end{proof}

\begin{question}
Sean $A$ y $B$ brazas tales que el producto semidirecto $A\rtimes_{\sigma}B$ es semiprimo. ¿Es cierto que entonces $A$ es una braza semiprima?
\end{question}

\section*{Notas}

Los teoremas~\ref{thm:sd_semiprime} y~\ref{thm:wreath_semiprime} 
fueron demostrados por Patrick Kinnear.