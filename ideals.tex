\chapter{Ideals}
\label{ideals}

\begin{definition}
\index{Subbrace}
Let $A$ be a brace. A \emph{subbrace} of $A$ is a non-empty 
subset $B$ of $A$ such that $(B,+)$ is a subgroup of $(A,+)$ and $(B,\circ)$ is a subgroup of $(A,\circ)$. 
\end{definition}

\begin{definition}
    \index{Left!ideal}
    \index{Strong!left ideal}
    Let $A$ be a brace. A \emph{left ideal} of $A$ is a subgroup $(I,+)$ of
	$(A,+)$ such that $\lambda_a(I)\subseteq I$ for all $a\in A$, i.e. $\lambda_a(x)\in I$ for all $a\in A$ and $x\in I$. A \emph{strong left ideal} of $A$ 
	is a left ideal $I$ of $A$ such that $(I,+)$ is a normal subgroup of $(A,+)$. 
\end{definition}

\begin{proposition}
    A left ideal $I$ of a brace $A$ is a subbrace of $A$. 
\end{proposition}

\begin{proof}
    We need to prove that $(I,\circ)$ is a subgroup of $(A,\circ)$. Clearly $I$ is non-empty, 
    as it is an additive subgroup of $A$. If $x,y\in I$, then
    $x\circ y=x+\lambda_x(y)\in I+I\subseteq I$ and $x'=-\lambda_x(x)\in I$. 
\end{proof}

\begin{example}
    Let $A$ be a brace. Then 
    \[
    \Fix(A)=\{a\in A:\lambda_x(a)=a\text{ for all $x\in A$}\}
    \]
    is a left ideal of $A$. 
\end{example}

\begin{definition}
    \index{Ideal}
    An \emph{ideal} of $A$ is a strong left ideal $I$ of $A$ such that 
	$(I,\circ)$ is a normal subgroup of $(A,\circ)$. 
\end{definition}

In general 
\[
\{\text{subbraces}\}\subsetneq \{\text{left ideals}\}\subsetneq\{\text{strong left ideals}\}\subsetneq\{\text{ideals}\}.
\]
For example, $\Fix(A)$ is not a strong left ideal of $A$.

\begin{example}
    Consider the semidirect product $A=\Z/(3)\rtimes \Z/(2)$ of the
    trivial braces $\Z/(3)$ and $\Z/(2)$
    via the non-trivial action of $\Z/(2)$ over $\Z/(3)$.
    Then 
    \[
    \lambda_{(x,y)}(a,b)=(x,y)(a,b)-(x,y)=(x+(-1)^ya,y+b)-(x,y)=((-1)^ya,b).
    \]
    Then $\Fix(A)=\{(0,0),(0,1)\}$ is not a 
    normal subgroup of $(A,\circ)$ and hence $\Fix(A)$ is not a strong left 
    ideal of $A$.
\end{example}

\begin{example}
    \index{Kernel}
	Let $f\colon A\to B$ be a brace homomorphism. Then $\ker f$ 
	is an ideal of $A$.
\end{example}

Let $I$ and $J$ be ideals
of a $A$. Then $I\cap J$ is an ideal of $A$, see Exercise~\ref{prob:sum_ideals}.  
The sum $I+J$ of $I$ and $J$ is defined as the
additive subgroup of $A$ generated by all the 
elements of the form
$u+v$, $u\in I$ and $v\in J$. 

\begin{proposition}
Let $A$ be a brace and let
$I$ and $J$ be ideals of $A$. Then $I+J$ is an ideal of $A$.
\end{proposition}

\begin{proof}
    Let $a\in A$, $u\in I$ and $v\in J$. Then $\lambda_a(u+v)\in I+J$ and
    hence it follows that $\lambda_a(I+J)\subseteq I+J$. Moreover, 
    \[
        (u+v)*a=(u\circ\lambda^{-1}_u(v))*a
        =u*(\lambda^{-1}_u(v)*a)+\lambda^{-1}_u(v)*a+u*a\in I+J.
    \]
    This formula implies that  
    \[
        a\circ (u+v)\circ a'=a+\lambda_a((u+v)+(u+v)*a')-a\in I+J.
    \]
    Thus it follows that $a\circ (I+J)\circ a'\subseteq I+J$.
    
    Finally $I+J$ is a normal subgroup of $(A,+)$ since
    \[
        a+\left(\sum_{k} u_k+v_k\right)-a=\sum_k ((a+u_k-a)+(a+v_k-a))\in I+J
    \]
    whenever $u_k\in I$ and $v_k\in J$ for all $k$. 
\end{proof}


\begin{definition}
	\index{Socle}
	Let $A$ be a brace. The subset 
	$\Soc(A)=\ker\lambda\cap Z(A,+)$
	is the \emph{socle} of $A$.
\end{definition}

\begin{lemma}
    \label{lem:socle}
    Let $A$ be a brace and $a\in\Soc(A)$. Then 
    \[
    b+b\circ a=b\circ a+b\quad\text{and}\quad
    \lambda_b(a)=b\circ a\circ b'
    \]
    both hold 
    for all $b\in A$.
\end{lemma}

\begin{proof}
    Let $b\in A$. Since
    $b'\circ (b\circ a+b)=a-b'$ and
    $b'\circ (b+b\circ a)=-b'+a$, the first claim follows since
    $a\in Z(A,+)$.
    Now we prove the second claim:
    \[
    b\circ a\circ b'=b\circ (a\circ b')=b\circ (a+b')=b\circ a-b=-b+b\circ
    a=\lambda_b(a).\qedhere
    \]
\end{proof}

\begin{proposition}
	\label{pro:socle}
	Let $A$ be a brace. Then $\Soc(A)$ is an ideal of $A$.
	
	\begin{proof}
		Clearly $0\in\Soc(A)$, since $\lambda$ is a group homomorphism. Let $a,b\in\Soc(A)$ and $c\in A$. Since 
		$b\circ (-b)=b+(-b)=0$, it follows that 
		$b'=-b\in\Soc(A)$. The calculation 
		\[
		\lambda_{a-b}(c)=\lambda_{a\circ b'}(c)=\lambda_a\lambda^{-1}_b(c)=c,
		\]
 		implies that $a-b\in\ker\lambda$. Since $a-b\in Z(A,+)$, it follows that 
        $(\Soc(A),+)$ is a normal subgroup of $(A,+)$. 
        
        % Since $a\in\Soc(A)\subseteq Z(A,+)$, 
        % \[
        % c\circ a-c=c\circ (a+c')=c\circ (c'+a)=-c+c\circ a.
        % \]
        % From this we obtain 
        % \[
        % \lambda_c(a)=-c+c\circ a=c\circ a-c=c\circ (a+c')=c\circ a\circ c'.
        % \]
        For each $d\in A$, $a+c'\circ d=c'\circ d+a$ by Lemma~\ref{lem:socle}. Then 
        \begin{align*}
        d+\lambda_c(a) &= d-c+c\circ a
        =c\circ (c'\circ d+a)\\
        &= c\circ (a+c'\circ d)
        = c\circ a-c+d
        = -c+c\circ a+d
        = \lambda_c(a)+d,
        \end{align*}
        that is $\lambda_c(a)$ is central in $(A,+)$. Moreover, 
        \begin{align*}
            \lambda_c(a)+d &= -c+c\circ a+d 
            = c\circ a-c+d\\
            &= c\circ (a+(c'\circ d)
            = c\circ a\circ c'\circ d=\lambda_c(a)\circ d
        \end{align*}
        and hence 
        \[
        \lambda_{\lambda_c(a)}(d)=-\lambda_c(a)+\lambda_c(a)\circ d=-\lambda_c(a)+\lambda_c(a)+d=d.
        \]
        Therefore $\Soc(A)$ is a strong left ideal of $A$. In fact, $\Soc(A)$ is an ideal of $A$,
        as $c\circ a\circ c'=\lambda_c(a)\in\Soc(A)$.  
	\end{proof}
\end{proposition}

As a corollary we obtain that the socle of a brace $A$ is a trivial brace of abelian type. 
In particular, if $a\in\Soc(A)$, then $a$ is a central element such that  
$a\circ b=a+b$ for all $b\in B$. 

\begin{proposition}
    \label{pro:soc_kernels}
    Let $A$ be a brace. Then $\Soc(A)=\ker\lambda\cap\ker\mu$.
\end{proposition}

\begin{proof}
    Let $a\in\Soc(A)$. Then $\lambda_a=\id$ and $a\in Z(A,+)$. Let $c=\mu_a(b)=\lambda_b(a)'\circ b\circ a$. Then
    $b\circ a=\lambda_b(a)\circ c=(-b+b\circ a)\circ c$. Since this is equivalent to 
    %\shortintertext{which is equivalent to}
    \begin{align*}
        a\circ c'=b'\circ (-b+b\circ a)=b'\circ (-b)-b'+a=b'+a=a+b'=a\circ b',
    \end{align*}
    it follows that $c'=b'$ and therefore $c=b$. Thus $a\in\ker\lambda\cap\ker\mu$. 
    
    Conversely, let $a\in\ker\lambda\cap\ker\mu$ and $b\in A$. Then $b'=\mu_a(b')=\lambda_{b'}(a)'\circ b'\circ a$, so
    $\lambda_{b'}(a)=b'\circ a\circ b$. Now 
    \[
    b+a=b\circ\lambda^{-1}_b(a)=b\circ\lambda_{b'}(a)=b\circ b'\circ a\circ b=a\circ b=a+\lambda_a(b)=a+b
    \]
    implies that $a\in\Soc(A)$. 
\end{proof}

Another important ideal was defined in~\cite{MR3917122}.

\begin{definition}
\index{Annihilator}
Let $A$ be a brace. The \emph{annihilator} of $A$ is 
defined as the set $\Ann(A)=\Soc(A)\cap Z(A,\circ)$. 
\end{definition}

Note that $\Ann(A)\subseteq\Fix(A)$. 

\begin{proposition}
The annihilator of a brace $A$ is an ideal of $A$. 
\end{proposition}

\begin{proof}
    Let $a\in A$ and $x\in\Ann(A)$. Since $\Ann(A)\subseteq Z(A,+)\cap Z(A,\circ)$, 
    we only need to note that $\lambda_a(x)=x\in\Ann(A)$. 
\end{proof}


If $X$ and $Y$ are subsets of a brace $A$, $X*Y$ is defined as the 
subgroup of $(A,+)$ generated by elements of the form $x*y$, $x\in X$ and $y\in Y$, i.e.
\[
X*Y=\langle x*y:x\in X\,,y\in Y\rangle_+.
\]

\begin{proposition}
    \label{pro:A*I}
    Let $A$ be a brace. A subgroup $I$ of $(A,+)$ is 
    a left ideal of $A$ if and only if $A*I\subseteq I$.
\end{proposition}

\begin{proof}
    Let $a\in A$ and $x\in I$. If $I$ is a
    left ideal, then $a*x=\lambda_a(x)-x\in I$. Conversely, if $A*I\subseteq
    I$, then $\lambda_a(x)=a*x+x\in I$.
\end{proof}

\begin{proposition}
    \label{pro:I*A}
    Let $A$ be a brace. A normal subgroup $I$ of $(A,+)$
    is an ideal of $A$ if and only $\lambda_a(I)\subseteq I$ for all $a\in A$ and
    $I*A\subseteq I$.
\end{proposition}

\begin{proof}
    Let $x\in I$ and $a\in A$.  Assume first that $I$ is invariant under the
    action of $\lambda$ and that $I*A\subseteq I$. Then
    \begin{equation}
    \label{eq:trick:I*A}
        \begin{aligned}
        a\circ x\circ a' &=a+\lambda_a(x\circ a')\\
        &=a+\lambda_a(x+\lambda_x(a'))
        =a+\lambda_a(x)+\lambda_a\lambda_x(a')+a-a\\
        &=a+\lambda_a(x+\lambda_x(a')-a')-a
        =a+\lambda_a(x+x*a')-a
    \end{aligned}
    \end{equation}
    and hence $I$ is an ideal.

    Conversely, assume that $I$ is an ideal. Then $I*A\subseteq I$ since
    \begin{align*}
        x*a&=-x+x\circ a-a\\
        &=-x+a\circ(a'\circ x\circ a)-a
        =-x+a+\lambda_a(a'\circ x\circ a)-a\in I.\qedhere
    \end{align*}
\end{proof}

%Clearly $\Soc(A)=\ker(\lambda)\cap Z(A,+)$. In \cite[Lemma~2.5]{MR3647970} it
%is proved that $\Soc(A)$ is an ideal of $A$.

\index{Quotient brace}
If $A$ is a brace and $I$ is an ideal of $A$, then $a+I=a\circ I$ for all $a\in A$. Indeed, 
$a\circ x=a+\lambda_a(x)\in a+I$ and 
$a+x=a\circ\lambda_a^{-1}(x)=a\circ\lambda_{a'}(x)\in a\circ I$ 
for all $a\in A$ and $x\in I$. 
This allows us to prove that there exists a unique brace structure over $A/I$ such that
the map 
\[
\pi\colon A\to A/I,
\quad
a\mapsto a+I=a\circ I,
\]
is a brace homomorphism. The brace $A/I$ is the \emph{quotient brace} of $A$ modulo $I$. It is possible
to prove the isomorphism theorems for braces, see Exercises~\ref{prob:iso1},~\ref{prob:iso2}, \ref{prob:iso3} and
\ref{prob:correspondence}.



\section*{Exercises}

\begin{prob}
\label{prob:radical}
Recall that two-sided braces are equivalent to radical rings. Prove that under this equivalence, 
(left) ideals of the radical ring correspond to (left) ideals of the associated brace. 
\end{prob}

\begin{prob}
\label{prob:sum_ideals}
Prove that the intersection of ideals is an ideal. 
\end{prob}

\begin{prob}
Let $A$ be a brace and $I$ be a characteristic subgroup of the additive group of $A$. Prove that
$I$ is a left ideal of $A$. 
\end{prob}

\begin{prob}
Let $A=...$. Prove that $A$ has only three ideals:... Let $I$ be the ideal of $A$ of size four. Prove that
$A*I$ has size two and hence it is not an ideal of $A$. 
\end{prob}

\begin{prob}
\label{prob:Bachiller1}
Prove that the socle of a brace $A$ is the kernel of the 
group homomorphism $(A,\circ)\to\Aut(A,+)\times\Sym_A$, $a\mapsto (\lambda_a,\mu_a^{-1})$. 
\end{prob}

\begin{prob}
\label{prob:Bachiller2}
Prove that the socle of a brace $A$ is the kernel of the 
group homomorphism $(A,\circ)\to\Aut(A,+)\times\Aut(A,+)$, $a\mapsto (\lambda_a,\xi_a)$, where
$\xi_a(b)=a+\lambda_a(b)-a$. 
\end{prob}

\begin{prob}
\label{prob:iso1}
    Let $f\colon A\to B$ be a brace homomorphism. Prove that $A/\ker f\simeq f(A)$. 
\end{prob}

\begin{prob}
\label{prob:iso2}
    Let $A$ be a brace and $B$ be a subbrace of $A$. If $I$ is an ideal of $B$, 
    then $B\circ I$ is a subbrace of $B$, 
    $B\cap I$ is an ideal of $B$ and $(B\circ I)/I\simeq B/(B\cap I)$. 
\end{prob}

\begin{prob}
\label{prob:iso3}
Let $A$ be a brace and $I$ and $J$ be ideals of $A$. If $I\subseteq J$, then
$A/J\simeq (A/I)/(J/I)$. 
\end{prob}

\begin{prob}
\label{prob:correspondence}
Let $A$ be a brace and $I$ be an ideal of $A$. There is a bijective correspondence between (left) ideals 
of $A$ containing $I$ and (left) ideals of $A/I$. 
\end{prob}

\section*{Notes}

Exercise~\ref{prob:2sided} comes from~\cite{MR3177933}. 
Exercises~\ref{prob:Bachiller1} and~\ref{prob:Bachiller2} appear in~\cite{MR3835326}.

The socle was defined by Rump in~\cite{MR2278047}. The annihilator first appeared in the work~\cite{MR3917122}  
of Catino, Colazzo and Stefanell. 