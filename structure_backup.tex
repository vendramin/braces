\chapter{The structure of braces}

Let $\pi$ be a set of primes. A $\pi$-group is a finite group $G$ such that all prime divisors of $|G|$ belong to $\pi$. 
A $\pi$-subgroup of $G$ is a subgroup of $G$ that is a $\pi$-group. 
A Hall $\pi$-subgroup of $G$ of index involving no prime of $\pi$. A $p$-complement 
of $G$ is a Hall $\pi$-subgroup, where $\pi=\{q:q\ne p\}$. 

\begin{theorem}[Hall]
Let $G$ be a finite group. If there exists a $p$-complement 
for all prime divisors $p$ of $|G|$, then 
$G$ is solvable. 
\end{theorem}

\begin{proof}
    See for example~\cite[Theorem 3.15]{MR2426855}. 
\end{proof}

\begin{theorem}
\label{thm:add_nilpotent}
Let $A$ be a finite brace of nilpotent type. Then 
the multiplicative group of $A$ is solvable.
\end{theorem}

\begin{proof}
    Let $K$ be the additive group of $A$ and $G$ be the multiplicative group of $A$. Assume
    that $|A|=p_1^{\alpha_1}\cdots p_n^{\alpha}$ for different primes numbers $p_1,\dots,p_n$. 
    Since $K$ is nilpotent, each $K_i\in\Syl_{p_j}(K)$ is normal in $K$, so 
    each $K_i$ is a left ideal of $A$. It follows that for each $i\in\{1,\dots,n\}$ both $K_i$ and 
    $\prod_{j\ne i}K_j$ are braces of coprime order. In particular, for 
    each $i\in\{1,\dots,n\}$ there exists a subgroup of $G$ of order coprime with $p$. 
    Then $G$ is solvable by Hall's theorem. 
\end{proof}

The previous theorem is not true in the case of finite braces of non-nilpotent type. 

\begin{theorem}
\label{thm:Smoktunowicz}
    Let $A$ be a finite skew left brace of nilpotent type. Then $A$ is
    left nilpotent if and only if the multiplicative group of $A$ is nilpotent.
\end{theorem}

\begin{proof}
    Proposition~\ref{pro:left_p} and Theorem~\ref{thm:left_p} prove the theorem.
\end{proof}

%The following theorem was found by Smoktunowicz in the case of braces of abelian type. 



\begin{theorem}
Let $A$ be a finite brace with abelian multiplicative group. Then 
the additive group of $A$ is solvable.
\end{theorem}

\begin{proof}
    
\end{proof}

 The proof 
uses a well-known theorem of Kegel and Wielandt.

\begin{theorem}[Kegel--Wielandt]
    Let $G$ be a finite group such that $G=AB$ for nilpotent subgroups $A$ and $B$. Then $G$ is solvable. 
\end{theorem}

\begin{proof}
    See~\cite[Theorem 2.4.3]{MR1211633}.
\end{proof}


\section*{Exercises}

\begin{prob}
\label{prob:G(X,r)solvable}
Let $(X,r)$ be a finite involutive solution. Prove that $G(X,r)$ is solvable. 
\end{prob}

\begin{prob}
Is the result of Exercise~\ref{prob:G(X,r)solvable} true in the case of non-involutive finite solutions?
\end{prob}

\section*{Notes}

In~\cite{MR1722951}, Etingof, Schedler and Soloviev proved that the structure group of a finite involutive
solution is always solvable. The proof can translated into the language of braces 
to obtain Theorem~\ref{thm:add_nilpotent}.
%~\cite{MR3763907}.


