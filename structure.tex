\chapter{The structure brace of a solution}
\label{structure_brace}

% hacerlo con el guitar map

To prove that the structure group $G(X,r)$ of a solution $(X,r)$ is a skew brace, we follow the 
proof of Lu, Yan and Zhu. They use the language of braided groups.

\section{Braided groups and skew braces}

\begin{definition}
\index{Braided group}
A \emph{braided group} is a pair $(G,r)$, where 
$G$ is a group with operation $m\colon G\times G\to G$, $m(x,y)=xy$, and 
$r\colon G\times G\to G\times G$ is a bijective map such that
\begin{enumerate}
\item $r(xy,z)=(\id\times m)r_1r_2(x,y,z)$ for all $x,y,z\in G$,
\item $r(x,yz)=(m\times\id)r_2r_1(x,y,z)$ for all $x,y,z\in G$,
\item $r(1,x)=(x,1)$ and $r(x,1)=(1,x)$ for all $x\in G$, and 
\item $m\circ r=m$.
\end{enumerate}
The map $r$ is called a \emph{braiding operator} on $G$. 
\end{definition}


\begin{theorem}
    Let $G$ be a group. Then there is a bijection between the set of braiding operators on $G$ and the set of skew brace structures on the set $G$ with multiplicative group $G$. 
\end{theorem}

\begin{proof}
    Let $\mathcal{BO}(G)$ be the set of all the braiding operators on $G$. Let $\mathcal{B}(G)$ be the set of all the skew brace structures $(G,+,\circ)$ on the group $G$, where $x\circ y=xy$, for all $x,y\in G$. For $r\in \mathcal{BO}(G)$, we write $r(x,y)=(x\rhd_r y,x\lhd_r y)$, for all $x,y\in G$. We define
    \[ f\colon \mathcal{BO}(G)\rightarrow \mathcal{B}(G)\]
    by $f(r)=(G,+_r,\circ)$, where $x+_r y=x(x^{-1}\rhd_r y)$, for all $x,y\in G$ and $r\in \mathcal{BO}(G)$. By the definition of a braiding operator, we have that
    \begin{align*}
        r(xy,z)=& (\id\times m)r_1r_2(x,y,z)\\
        =&(\id\times m)r_1(x,y\rhd_r z,y\lhd_r z)\\
        =&(\id\times m)(x\rhd_r (y\rhd_r z),x\lhd_r (y\rhd_r z),y\lhd_r z)\\
        =&(x\rhd_r (y\rhd_r z),(x\lhd_r (y\rhd_r z))(y\lhd_r z)),
        \end{align*}
    and thus $(xy)\rhd_r z=(x\rhd_r (y\rhd_r z)$. Since $r(1,x)=(x,1)$, we have that the map $\xi\colon G\rightarrow \Sym_G$ defined by $\xi(x)(y)=x\rhd_r y$, for all $x,y\in G$ is a left action of $G$ on itself.
    By the definition of a braiding operator, we have that
    \begin{align*}
        r(x,yz)=& (m\times\id)r_2r_1(x,y,z)\\
        =&(m\times\id)r_2(x\rhd_r y,x\lhd_r y,z)\\
        =&(m\times\id)(x\rhd_r y,(x\lhd_r y)\rhd_r z, (x\lhd_r y)\lhd_r z)\\
        =&((x\rhd_r y)((x\lhd_r y)\rhd_r z), (x\lhd_r y)\lhd_r z),
        \end{align*}
    and thus $x\lhd_r (yz)=(x\lhd_r y)\lhd_r z$. Since $r(x,1)=(1,x)$, we have that the map $\eta\colon G\rightarrow \Sym_G$ defined by $\eta(x)(y)=y\lhd_r x$, for all $x,y\in G$ is a right action of $G$ on itself. Furthermore, $xy=(x\rhd_r y)(x\lhd_r y)$, for all $x,y\in G$. Hence, by Theorem \ref{thm:LYZ}, $(G,r)$ is a solution to the YBE. Now we have
    \begin{align}\label{LYZequa}
        x\rhd_r (y+_r z)=&x\rhd_r(y(y^{-1}\rhd_r z))\nonumber\\
        =&(x\rhd_r y)((x\lhd_r y)\rhd_r (y^{-1}\rhd_r z))\nonumber\\
        =&(x\rhd_r y)(((x\lhd_r y)y^{-1})\rhd_r z))\nonumber\\
        =&(x\rhd_r y)(((x\rhd_r y)^{-1}x)\rhd_r z))\nonumber\\
        =&(x\rhd_r y)+(x\rhd_r z).
    \end{align}
    Note that $1+_r x=1(1^{-1}\rhd_r x)=x$ and
    $x+_r 1=x(x^{-1}\rhd_r 1)=x1=x$,  for all $x\in G$. For every $x\in G$, we have
    \[x+_r (x\rhd x^{-1})=x(x^{-1}\rhd_r(x\rhd x^{-1}))=xx^{-1}=1\]
and
\begin{align*}
    (x\rhd x^{-1})+_r x=&(x\rhd x^{-1})((x\rhd x^{-1})^{-1}\rhd x)\\
    =&(x\rhd x^{-1})((x\lhd x^{-1})\rhd x)\\
    =&x\rhd_r(x^{-1}x)=1.
\end{align*}
By (\ref{LYZequa}), we have that
\begin{align*}
    (x+_r y)+_r z=&x(x^{-1}\rhd_r y)(((x^{-1}\rhd_r y)^{-1}x^{-1})\rhd_r z)\\
    =&x((x^{-1}\rhd_r y)+_r(x^{-1}\rhd_r z))\\
    =&x(x^{-1}\rhd_r(y+_r z))=x+_r (y+_r z),
\end{align*}
for all $x,y,z\in G$. Hence $(G,+_r)$ is a group. By (\ref{LYZequa}), we have
\begin{align*}
    x(y+_r z)=&x+_r(x\rhd_r (y+_r z))\\
    =&x+_r (x\rhd_r y)+_r(x\rhd_r z)\\
    =&xy-_r x+_r xz,
\end{align*}
for all $x,y,z\in G$. Hence $(G,+_r,\circ)$ is a skew brace and $f$ is a well-defined map. Note that the lambda map of this skew brace is defined by
$\lambda_x(y)=-_r x+_r xy=x\rhd_r y$, for all $x,y\in G$.

Now we define $g\colon \mathcal{B}(G)\rightarrow \mathcal{BO}(G)$, by $g(G,+,\circ)=r_+$ and
\[ r_+(x,y)=(-x+xy,(-x+xy)^{-1}xy)=(\lambda_x(y),\mu_y(x)),\]
for all $(G,+,\circ)\in\mathcal{B}(G)$ and $x,y\in G$. We shall prove that $r_+$ is a braiding operator on $G$. Note that, by Proposition \ref{pro:lambda}, 
\begin{align*}
    r_+(xy,z)=&(\lambda_{xy}(z),\mu_z(xy))\\
    =&(\lambda_x\lambda_y(z),\lambda_x(\lambda_y(z))^{-1}xyz)\\
    =&(\lambda_x\lambda_y(z),\lambda_x(\lambda_y(z))^{-1}x\lambda_y(z)\lambda_y(z)^{-1}yz)\\
    =&(\lambda_x\lambda_y(z),\mu_{\lambda_y(z)}(x)\mu_z(y))\\
    =&(\id\times m)(r_+)_1(r_+)_2(x,y,z).
\end{align*}
By Propositions \ref{pro:lambda} and \ref{pro:mu},
\begin{align*}
    r_+(x,yz)=&(\lambda_{x}(yz),\mu_{yz}(x))\\
    =&(\lambda_x(y+\lambda_y(z)),\mu_z\mu_y(x))\\
    =&(\lambda_x(y)+\lambda_x\lambda_y(z),\mu_z\mu_y(x))\\
    =&(\lambda_x(y)\lambda_{\lambda_x(y)^{-1}xy}(z),\mu_z\mu_y(x))\\
    =&(\lambda_x(y)\lambda_{\mu_y(x)}(z),\mu_z\mu_y(x))\\
    =&(m\times \id)(r_+)_2(r_+)_1(x,y,z).
\end{align*}
We also have that
\[ r_+(1,x)=(\lambda_1(x),\mu_x(1))=(x,1),\]
\[ r_+(x,1)=(\lambda_x(1),\mu_1(x))=(1,x),\]
and
\[ \lambda_x(y)\mu_y(x)=\lambda_x(y)\lambda_x(y)^{-1}xy=xy,\]
for all $x,y\in G$. Hence $r_+$ is a braiding operator on $G$, thus $g$ is a well-defined map. Now it is easy to check that $f\circ g=\id_{\mathcal{B}(G)}$ and $g\circ f=\id_{\mathcal{BO}(G)}$.
\end{proof}

\section{The structure group of a solution}
\index{Solution!structure group}
Let $(X,r)$ be asolution to the YBE. Write as usual $r(x,y)=(\sigma_x(y),\tau_y(x))$. The {\em structure group} of $(X,r)$ is the group
\[ G(X,r)=\gr(X : xy=\sigma_x(y)\tau_y(x), \text{ for all }x,y\in X).\]
In this section we shall see that there is a natural braiding operator on $G(X,r)$ induced by $r$ that satisfies a nice universal property. 

To do so we prove first that there is an interesting extension of the solution $(X,r)$.

\begin{proposition}
\label{prop:extendsol}
Let $(X,r)$ be a solution of the YBE. Let $X'=\{x'\mid x\in X\}$ be a copy of $X$ and let $Y$ be the disjoint union of $X$ and $X'$. We write $r(x,y)=(\sigma_x(y),\tau_y(x))$ and $r^{-1}(x,y)=(\widehat{\sigma}_x(y),\widehat{\tau}_y(x))$. We define $r'\colon Y\times Y\rightarrow Y\times Y$ by $r'(u,v)=(\sigma'_u(v),\tau'_v(u))$, for all $u,v\in Y$, where
\[ \sigma'_x(y)=\sigma_x(y),\; \sigma'_{x'}(y)=\sigma^{-1}_x(y),\; \sigma'_{x}(y')=\widehat{\tau}^{-1}_x(y)',\; \sigma'_{x'}(y')=\widehat{\tau}_x(y)'\]
and
\[ \tau'_x(y)=\tau_x(y),\; \tau'_{x'}(y)=\tau^{-1}_x(y),\; \tau'_{x}(y')=\widehat{\sigma}^{-1}_x(y)',\; \tau'_{x'}(y')=\widehat{\sigma}_x(y)',\]
for all $x,y\in X$. Then $(Y,r')$ is a solution of the YBE.
\end{proposition}

\begin{proof}
Let $x,y\in X$. Since 
\[ r(\tau^{-1}_y(x),y)=(\sigma_{\tau^{-1}_y(x)}(y),x) \text{ and }
r(y,\sigma^{-1}_y(x))=(x,\tau_{\sigma^{-1}_y(x)}(y)),\] we have that
\[ \widehat{\tau}^{-1}_x(y)=\sigma_{\tau^{-1}_y(x)}(y)\text{ and } \widehat{\sigma}^{-1}_x(y)=\tau_{\sigma^{-1}_y(x)}(y).\]
Similarly one can see that
\[ \tau^{-1}_x(y)=\widehat{\sigma}_{\widehat{\tau}^{-1}_y(x)}(y)\text{ and } \sigma^{-1}_x(y)=\widehat{\tau}_{\widehat{\sigma}^{-1}_y(x)}(y).\]
Now it is easy to check that $r'$ is bijective and
\[ (r')^{-1}(x,y)=(\widehat{\sigma}_x(y),\widehat{\tau}_y(x)),\; (r')^{-1}(x',y)=(\widehat{\sigma}^{-1}_x(y),\sigma^{-1}_y(x)'),\] 
\[ (r')^{-1}(x,y')=(\tau^{-1}_x(y)',\widehat{\tau}^{-1}_y(x)), \; (r')^{-1}(x',y')=(\sigma_x(y)',\tau_y(x)').\]
By Lemma \ref{lem:YB} it is enough to prove that
\begin{itemize}
    \item[(a)] $\sigma'_{u}\sigma'_{v}=\sigma'_{\sigma'_u(v)}\sigma'_{\tau'_v(u)}$,
    \item[(b)]$\sigma'_{\tau'_{\sigma'_u(v)}(w)}\tau'_v(u)=\tau'_{\sigma'_{\tau'_u(w)}(v)}\sigma'_w(u)$,
    \item[(c)] $\tau'_{u}\tau'_{v}=\tau'_{\tau'_u(v)}\tau'_{\sigma'_v(u)}$,
\end{itemize}
for all $u,v,w\in Y$.

Let $x,y,z\in X$. Since $(X,r^{-1})$ is a solution of the YBE, by Lemma \ref{lem:YB}, we have that
\begin{align*}
    \sigma'_{x}\sigma'_{y}(z)=&\sigma_{x}\sigma_{y}(z)=\sigma_{\sigma_x(y)}\sigma_{\tau_y(x)}(z)\\
    =&\sigma'_{\sigma'_x(y)}\sigma'_{\tau'_y(x)}(z),\\
        \sigma'_{x}\sigma'_{y}(z')=&\sigma'_{x}(\widehat{\tau}^{-1}_{y}(z)')=\widehat{\tau}^{-1}_{x}\widehat{\tau}^{-1}_{y}(z)'\\
        =&\widehat{\tau}^{-1}_{\sigma_x(y)}\widehat{\tau}^{-1}_{\tau_y(x)}(z)'\\
    =&\sigma'_{\sigma'_x(y)}\sigma'_{\tau'_y(x)}(z').
    \end{align*}
Hence $\sigma'_x\sigma'_y=\sigma'_{\sigma'_x(y)}\sigma'_{\tau'_y(x)}$. We also have that   
\begin{align*}
    \sigma'_{x}\sigma'_{y'}(z)=&\sigma_{x}\sigma^{-1}_{y}(z)=\sigma^{-1}_{\sigma_{\tau^{-1}_y(x)}(y)}\sigma_{\tau^{-1}_y(x)}(z)\\
    =&\sigma^{-1}_{\widehat{\tau}^{-1}_x(y)}\sigma_{\tau^{-1}_y(x)}(z)=\sigma'_{\sigma'_x(y')}\sigma'_{\tau'_{y'}(x)}(z),\\
        \sigma'_{x}\sigma'_{y'}(z')=&\sigma'_{x}(\widehat{\tau}_{y}(z)')=\widehat{\tau}^{-1}_{x}\widehat{\tau}_{y}(z)'\\
        =&\widehat{\tau}_{\widehat{\tau}^{-1}_x(y)}\widehat{\tau}^{-1}_{\sigma_{\tau^{-1}_x(y)}(x)}(z)'=\widehat{\tau}_{\widehat{\tau}^{-1}_x(y)}\widehat{\tau}^{-1}_{\widehat{\tau}^{-1}_y(x)}(z)'\\
    =&\sigma'_{\sigma'_x(y')}\sigma'_{\tau'_{y'}(x)}(z').
    \end{align*}
Hence $\sigma'_x\sigma'_{y'}=\sigma'_{\sigma'_x(y')}\sigma'_{\tau'_{y'}(x)}$. Similarly one can check that
$\sigma'_{x'}\sigma'_{y}=\sigma'_{\sigma'_{x'}(y)}\sigma'_{\tau'_{y}(x')}$ and $\sigma'_{x'}\sigma'_{y'}=\sigma'_{\sigma'_{x'}(y')}\sigma'_{\tau'_{y'}(x')}$. Hence (a) follows.
By a symmetric argument (c) follows.

Now we prove (b). By Lemma \ref{lem:YB}, we have
\begin{align*}
    \sigma'_{\tau'_{\sigma'_{x'}(y)}(z)}\tau'_y(x')=&
    \sigma'_{\tau'_{\sigma^{-1}_{x}(y)}(z)}\left(\widehat{\sigma}^{-1}_y(x)'\right)\\
    =&
    \sigma'_{\tau_{\sigma^{-1}_{x}(y)}(z)}\left(\tau_{\sigma^{-1}_x(y)}(x)'\right)\\
    =&
    \left(\widehat{\tau}^{-1}_{\tau_{\sigma^{-1}_{x}(y)}(z)}\tau_{\sigma^{-1}_x(y)}(x)\right)'\\
     =&
    \left(\sigma_{\tau^{-1}_{\tau_{\sigma^{-1}_x(y)}(x)}\tau_{\sigma^{-1}_{x}(y)}(z)}\tau_{\sigma^{-1}_x(y)}(x)\right)'\\
    =&
    \left(\sigma_{\tau_{y}\tau^{-1}_{x}(z)}\tau_{\sigma^{-1}_x(y)}(x)\right)'\\
    =&
    \left(\sigma_{\tau_{\sigma_x\sigma^{-1}_x(y)}\tau^{-1}_{x}(z)}\tau_{\sigma^{-1}_x(y)}(x)\right)'\\
    =&
    \left(\tau_{\sigma_{\tau_x\tau^{-1}_x(z)}\sigma^{-1}_{x}(y)}\sigma_{\tau^{-1}_x(z)}(x)\right)'\\
    =&\tau'_{\sigma'_{\tau'_{x'}(z)}(y)}\sigma'_z(x'),
\end{align*}
where the last equality follows by a symmetric argument. By similar calculations one get that
\[\tau'_{\sigma'_{\tau'_x(y')}(z')}\sigma'_{y'}(x)=\sigma'_{\tau'_{\sigma'_x(z')}(y')}\tau'_{z'}(x).\]
Note that, 
\begin{align*}
    \sigma'_{\tau'_{\sigma'_x(y')}(z)}\tau'_{y'}(x)=&\sigma'_{\tau'_{\widehat{\tau}^{-1}_x(y)'}(z)}\tau^{-1}_y(x)\\
    =&\sigma_{\tau^{-1}_{\sigma_{\tau^{-1}_y(x)}(y)}(z)}\tau^{-1}_y(x)
\end{align*}
and
\begin{align*}
    \tau'_{\sigma'_{\tau'_x(z)}(y')}\sigma'_{z}(x)
    =&\tau'_{\widehat{\tau}^{-1}_{\tau_x(z)}(y)'}\sigma_z(x)\\
    =&\tau^{-1}_{\sigma_{\tau^{-1}_y\tau_x(z)}(y)}\sigma_z(x)\\
    =&\tau^{-1}_{\sigma_{\tau_{\tau^{-1}_y(x)}\tau^{-1}_{\sigma_{\tau^{-1}_y(x)}(y)}(z)}(y)}\sigma_z(x),
\end{align*}
by Lemma \ref{lem:YB}.

Since, by Lemma \ref{lem:YB},
\begin{align*}
\tau_{\sigma_{\tau_{\tau^{-1}_y(x)}\tau^{-1}_{\sigma_{\tau^{-1}_y(x)}(y)}(z)}(y)}&\sigma_{\tau^{-1}_{\sigma_{\tau^{-1}_y(x)}(y)}(z)}\tau^{-1}_y(x)\\
=&\sigma_{\tau_{\sigma_{\tau^{-1}_y(x)}(y)}\tau^{-1}_{\sigma_{\tau^{-1}_y(x)}(y)}(z)} \tau_y{\tau^{-1}_y}(x)\\
=&\sigma_z(x), 
\end{align*}
we have that
\[\sigma'_{\tau'_{\sigma'_x(y')}(z)}\tau'_{y'}(x)=\tau'_{\sigma'_{\tau'_x(z)}(y')}\sigma'_{z}(x).\]
By a symmetric argument one can prove that
\[\tau'_{\sigma'_{\tau'_x(y')}(z)}\sigma'_{y'}(x)=\sigma'_{\tau'_{\sigma'_x(z)}(y')}\tau'_{z}(x).\]
By a similar calculations one get that
\[\tau'_{\sigma'_{\tau'_{x'}(y)}(z')}\sigma'_{y}(x')=\sigma'_{\tau'_{\sigma'_{x'}(z')}(y)}\tau'_{z'}(x')\]
and
\[\sigma'_{\tau'_{\sigma'_{x'}(y)}(z')}\tau'_{y}(x')=\tau'_{\sigma'_{\tau'_{x'}(z')}(y)}\sigma'_{z'}(x').\]
Hence, by Lemma \ref{lem:YB}, (c) follows. Therefore $(Y,r')$ is a solution of the YBE.
\end{proof}

We also need the following result. We use the following notation. For every solution $(X,r)$ of the YBE,
\[r_k=\id^{\times (k-1)}\times r\times\id^{\times (n-k-1)}\colon X^{n}\rightarrow X^n\]
for all $1\leq k<n$ and $n\geq 3$.
\begin{theorem}
\label{thm:solfreemonoid}
Let $(X,r)$ be a solution of the YBE. Let $\mathrm{FM}(X)$ be the free monoid on $X$. Let $m_k\colon X^{k}\rightarrow \mathrm{FM}(X)$ be the map defined by $m_k(x_1,\dots ,x_k)=x_1\cdots x_k$ for every positive integer $k$. Let $r_{FM}\colon \mathrm{FM}(X)\times \mathrm{FM}(X)\rightarrow \mathrm{FM}(X)\times \mathrm{FM}(X)$ be the map defined by $r_{FM}(x_1\cdots x_n,\, y_1\cdots y_k)$
\[=(m_k\times m_n)(r_k\cdots r_2r_1)(r_{k+1}\cdots r_3r_2)\cdots(r_{n+k-1}\cdots r_{n+1}r_n)(x_1,\dots ,x_n,y_1,\dots ,y_k)\]
for all $x_1,\dots ,x_n,y_1,\dots ,y_k\in X$ and all positive integers $n,k$,   $r_{FM}(1,a)=(a,1)$ and $r_{FM}(a,1)=(1,a)$ for all $a\in \mathrm{FM}(X)$. Then $(\mathrm{FM}(X),r_{FM})$ is a solution of the YBE.
\end{theorem}

\begin{proof}
Note that if we write $r_{FM}(a,b)=(\sigma_a(b),\tau_b(a))$ for all $a,b\in \mathrm{FM}(X)$, then
\begin{align*}
(\id\times r_{FM})(r_{FM}\times \id)&(\id\times r_{FM})(1,a,b)\\
=&(\id\times r_{FM})(r_{FM}\times \id)(1,\sigma_a(b),\tau_b(a))\\
=&(\id\times r_{FM})(\sigma_a(b),1,\tau_b(a))=(\sigma_a(b),\tau_b(a),1)
\end{align*}
and
\begin{align*}
(r_{FM}\times \id)(\id\times r_{FM})&(r_{FM}\times \id)(1,a,b)\\
=&(r_{FM}\times \id)(\id\times r_{FM})(a,1,b)\\
=&(r_{FM}\times \id)(a,b,1)=(\sigma_a(b),\tau_b(a),1)
\end{align*}
Similarly one can check that
\[(\id\times r_{FM})(r_{FM}\times \id)(\id\times r_{FM})(a,1,b)=
(r_{FM}\times \id)(\id\times r_{FM})(r_{FM}\times \id)(a,1,b)\]
and
\[(\id\times r_{FM})(r_{FM}\times \id)(\id\times r_{FM})(a,b,1)=
(r_{FM}\times \id)(\id\times r_{FM})(r_{FM}\times \id)(a,b,1).\]
Hence
to prove that
\[(\id\times r_{FM})(r_{FM}\times \id)(\id\times r_{FM})=(r_{FM}\times \id)(\id\times r_{FM})(r_{FM}\times \id)\]
we need to show that
\begin{align*}
    (r_{k+t}\cdots r_{1+t})&\cdots (r_{k+t+n-1}\cdots r_{t+n})(r_{t}\cdots r_2r_{1})\cdots (r_{n+t-1}\cdots r_{n})\\
    &\cdot (r_{n+t}\cdots r_{n+1})\cdots (r_{n+k+t-1}\cdots r_{n+k})
    (x_1,\dots ,x_n,y_1,\dots ,y_k,z_1,\dots,z_t)\\
    =&(r_{t}\cdots r_2r_{1})\cdots (r_{k+t-1}\cdots r_{k})(r_{k+t}\cdots r_{k+1})\cdots (r_{n+k+t-1}\cdots r_{n+k})\\
    &\cdot (r_{k}\cdots r_2r_{1})\cdots (r_{n+k-1}\cdots r_{n})(x_1,\dots ,x_n,y_1,\dots ,y_k,z_1,\dots,z_t)
\end{align*}
for all positive integers $n,k,t$ and $x_1,\dots ,x_n,y_1,\dots ,y_k,z_1,\dots ,z_t\in X$.

Note that 
\begin{equation}\label{eq:SFcom}
r_{l+m}r_{m}=r_mr_{l+m}
\end{equation}
for all $1\leq m<n+k+t-2$ and $2\leq l\leq n+k+t-m-1$. Hence
\begin{align}\label{eq:solfree}
r_m(r_{m+l}\cdots r_{m+1}r_m)=&r_{m+l}\cdots r_{m+2}r_mr_{m+1}r_m\nonumber\\
&(r_{m+l}\cdots r_{m+2}r_{m+1}r_{m})r_{m+1}.
\end{align}
By using (\ref{eq:SFcom}) and (\ref{eq:solfree}), we have that
\begin{align*}
    (r_{t+k}\cdots r_{t+1})&\cdots (r_{t+k+n-1}\cdots r_{t+n})(r_{t}\cdots r_2r_{1})\cdots (r_{n+t-1}\cdots r_{n})\\
    &\cdot (r_{n+t}\cdots r_{n+1})\cdots (r_{n+k+t-1}\cdots r_{n+k})\\
    =&(r_{t+k}\cdots r_{t+1}r_t\cdots r_2r_1)\cdots (r_{t+k+n-1}\cdots r_{t+n}r_{n+t-1}\cdots r_{n})\\
    &\cdot (r_{n+t}\cdots r_{n+1})\cdots (r_{n+k+t-1}\cdots r_{n+k})\\
     =&(r_t\cdots r_2r_1)(r_{t+k}\cdots r_{t+1}r_t\cdots r_2r_1)\\
     &\cdots (r_{t+k+n-1}\cdots r_{t+n}r_{n+t-1}\cdots r_{n})\\
    &\cdot (r_{n+t+1}\cdots r_{n+2})\cdots (r_{n+k+t-1}\cdots r_{n+k})\\
    =&(r_t\cdots r_2r_1)\cdots (r_{k+t-1}\cdots r_{k})\\
    &\cdot (r_{t+k}\cdots r_{t+1}r_t\cdots r_2r_1)\cdots (r_{t+k+n-1}\cdots r_{t+n}r_{n+t-1}\cdots r_{n})\\
    =&(r_t\cdots r_2r_1)\cdots (r_{k+t-1}\cdots r_{k})\\
    &\cdot (r_{k+t}\cdots r_{k+1}r_k\cdots r_2r_1)\cdots (r_{n+k+t-1}\cdots r_{n+k}r_{n+k-1}\cdots r_{n})\\
    =&(r_t\cdots r_2r_1)\cdots (r_{k+t-1}\cdots r_{k})\\
    &\cdot (r_{k+t}\cdots r_{k+1})\cdots (r_{n+k+t-1}\cdots r_{n+k})(r_k\cdots r_2r_1)\cdots (r_{n+k-1}\cdots r_{n}).
\end{align*}
Hence
\[(\id\times r_{FM})(r_{FM}\times \id)(\id\times r_{FM})=(r_{FM}\times \id)(\id\times r_{FM})(r_{FM}\times \id).\]
By the definition of the free monoid on $X$, every $m_k$ is bijective. Hence by the definition of $r_{FM}$, since $r$ is bijective, we have that $r_{FM}$ is bijective.

Note that $\sigma_1=\id$ and $\tau_1=\id$. We know that the restrictions of $\sigma_x$ and $\tau_x$ on $X$ for all $x\in X$ are bijective.
We shall prove, by induction on $k$, that the restrictions of $\sigma_x$ and $\tau_x$ on the set $X_k$ of words of length $k$ for all $x\in X$ are bijective. For $k=0$, $\sigma_x(1)=1$ and $\tau_x(1)=1$, by the definition of $r_{FM}$. Suppose that $k\geq 1$ and that the restrictions of $\sigma_x$ and $\tau_x$ on the set $X_l$ of words of length $l\leq k$ for all $x\in X$ are bijective. Let $x,y_1,\dots ,y_{k+1}\in X$. By the definition of $r_{FM}$ we have that
\[\sigma_x(y_1\cdots y_{k+1})=\sigma_x(y_1\cdots y_k)\sigma_{\tau_{y_k}\cdots \tau_{y_1}(x)}(y_{k+1}).\]
Let $z_1,\dots,z_{k+1}\in X$  such that $\sigma_x(y_1\cdots y_{k+1})=\sigma_x(z_1\cdots z_{k+1})$. Hence
\[\sigma_x(y_1\cdots y_{k+1})=\sigma_x(z_1\cdots z_{k+1}) \text{ and }
\sigma_{\tau_{y_k}\cdots \tau_{y_1}(x)}(y_{k+1})=\sigma_{\tau_{z_k}\cdots \tau_{z_1}(x)}(z_{k+1}).\]
By the inductive hypothesis the restriction of $\sigma_x$ to $X_k$ is bijective. Hence $y_1\cdots y_k=z_1\cdots z_k$, and thus $y_i=z_i$ for all $1\leq i\leq k$. Now
\[\sigma_{\tau_{y_k}\cdots \tau_{y_1}(x)}(y_{k+1})=\sigma_{\tau_{z_k}\cdots \tau_{z_1}(x)}(z_{k+1})=\sigma_{\tau_{y_k}\cdots \tau_{y_1}(x)}(z_{k+1})\]
and thus $y_{k+1}=z_{k+1}$. Therefore the restriction of $\sigma_x$ on $X_{k+1}$ is injective. Let $c_1,\dots ,c_{k+1}\in X$. By the inductive hypothesis there exist $x_1,\dots ,x_k\in X$ such that
\[\sigma_x(x_1\cdots x_k)=c_1\cdots c_k.\]
There exists $x_{k+1}\in X$ such that
\[\sigma_{\tau_{x_k}\cdots \tau_{x_1}(x)}(x_{k+1})=z_{k+1}.\]
Hence
\[\sigma_x(x_1\cdots x_{k+1})=c_1\cdots c_{k+1},\]
and therefore the restriction of $\sigma_x$ on $X_{k+1}$ is bijective.




\end{proof}

\begin{theorem}
\label{thm:LYZ9}
Let $(X,r)$ be a solution to the YBE. Let $i\colon X\rightarrow G(X,r)$ be the natural map. Then There exists a unique braiding operator $r_G$ on $G(X,r)$ such that $r_G\circ (i\times i)=(i\times i)\circ r$. Furthermore, if $(H,s)$ is a braided group and $j\colon X\rightarrow H$ is a map such that
$s\circ (j\times j)=(j\times j)\circ r$, then there exists a unique group homomorphism $f\colon G(X,r)\rightarrow H$ such that $s\circ (f\times f)=(f\times f)\circ r_G$ and $j=f\circ i$.
\end{theorem}

\section*{Exercises}

\begin{prob}
Prove that a braiding operator is a solution. 
\end{prob}

