\chapter{The structure skew brace of a solution}
\label{structure_brace}

% hacerlo con el guitar map

To prove that the structure group $G(X,r)$ of a solution $(X,r)$ \textcolor{red}{has a natural structure of} skew brace, we follow the 
proof of Lu, Yan and Zhu. They use the language of braided groups.

\section{Braided groups and skew braces}

\begin{definition}\label{braidedgroup}
\index{Braided group}
A \emph{braided group} is a pair $(G,r)$, where 
$G$ is a group with operation $m\colon G\times G\to G$, $m(x,y)=xy$, and 
$r\colon G\times G\to G\times G$ is a bijective map such that
\begin{enumerate}
\item $r(xy,z)=(\id\times m)r_1r_2(x,y,z)$ for all $x,y,z\in G$,
\item $r(x,yz)=(m\times\id)r_2r_1(x,y,z)$ for all $x,y,z\in G$,
\item $r(1,x)=(x,1)$ and $r(x,1)=(1,x)$ for all $x\in G$, and 
\item $m\circ r=m$.
\end{enumerate}
\index{Braiding operator}
The map $r$ is called a \emph{braiding operator} on $G$. 
\end{definition}

\begin{lemma}
\label{lem:braidedsolYBE}
Let $(G,r)$ be a braided group. We write $r(x,y)=(x\rhd_r y,x\lhd_r y)$, for all $x,y\in G$. Let $\xi\colon G\rightarrow \Sym_G$ be the map defined by $\xi(x)(y)=x\rhd_r y$ for all $x,y\in G$. Let $\eta\colon G\rightarrow \Sym_G$ be the map defined by $\eta(x)(y)=y\lhd_r x$ for all $x,y\in G$. Then $\xi$ is a left action of $G$ on itself, $\eta$ is a right action of $G$ on itself and $(G,r)$ is a solution to the YBE. 
\end{lemma}

\begin{proof}
By the definition of a braiding operator, we have that
    \begin{align*}
        r(xy,z)=& (\id\times m)r_1r_2(x,y,z)\\
        =&(\id\times m)r_1(x,y\rhd_r z,y\lhd_r z)\\
        =&(\id\times m)(x\rhd_r (y\rhd_r z),x\lhd_r (y\rhd_r z),y\lhd_r z)\\
        =&(x\rhd_r (y\rhd_r z),(x\lhd_r (y\rhd_r z))(y\lhd_r z)),
        \end{align*}
    and thus $(xy)\rhd_r z=(x\rhd_r (y\rhd_r z)$. Since $r(1,x)=(x,1)$, we have that the map $\xi$ is a left action of $G$ on itself.
    By the definition of a braiding operator, we have that
    \begin{align}\label{LYZequa1}
        r(x,yz)=& (m\times\id)r_2r_1(x,y,z)\nonumber\\
        =&(m\times\id)r_2(x\rhd_r y,x\lhd_r y,z)\nonumber\\
        =&(m\times\id)(x\rhd_r y,(x\lhd_r y)\rhd_r z, (x\lhd_r y)\lhd_r z)\nonumber\\
        =&((x\rhd_r y)((x\lhd_r y)\rhd_r z), (x\lhd_r y)\lhd_r z),
        \end{align}
    and thus $x\lhd_r (yz)=(x\lhd_r y)\lhd_r z$. Since $r(x,1)=(1,x)$, we have that the map $\eta$ is a right action of $G$ on itself. Furthermore, $xy=(x\rhd_r y)(x\lhd_r y)$ for all $x,y\in G$. Hence, by Theorem \ref{thm:LYZ}, $(G,r)$ is a solution to the YBE.
\end{proof}

\begin{theorem}\label{thm:braidedbraces}
    Let $G$ be a group. Then there is a bijection between the set of braiding operators on $G$ and the set of skew brace structures on the set $G$ with multiplicative group $G$. 
\end{theorem}

\begin{proof}
    Let $\mathcal{BO}(G)$ be the set of all the braiding operators on $G$. Let $\mathcal{B}(G)$ be the set of all the skew brace structures $(G,+,\circ)$ on the group $G$, where $x\circ y=xy$ for all $x,y\in G$. For $r\in \mathcal{BO}(G)$, we write $r(x,y)=(x\rhd_r y,x\lhd_r y)$ for all $x,y\in G$. We define
    \[ f\colon \mathcal{BO}(G)\rightarrow \mathcal{B}(G)\]
    by $f(r)=(G,+_r,\circ)$, where $x+_r y=x(x^{-1}\rhd_r y)$ for all $x,y\in G$ and $r\in \mathcal{BO}(G)$. Now, by (\ref{LYZequa1}) and Lemma \ref{lem:braidedsolYBE}, we have
    \begin{align}\label{LYZequa}
        x\rhd_r (y+_r z)=&x\rhd_r(y(y^{-1}\rhd_r z))\nonumber\\
        =&(x\rhd_r y)((x\lhd_r y)\rhd_r (y^{-1}\rhd_r z))\nonumber\\
        =&(x\rhd_r y)(((x\lhd_r y)y^{-1})\rhd_r z))\nonumber\\
        =&(x\rhd_r y)(((x\rhd_r y)^{-1}x)\rhd_r z))\nonumber\\
        =&\textcolor{red}{(x\rhd_r y)((x\rhd_r y)^{-1}\rhd_r(x\rhd_r z))}\nonumber\\
        =&(x\rhd_r y)+(x\rhd_r z).
    \end{align}
    Note that $1+_r x=1(1^{-1}\rhd_r x)=x$ and
    $x+_r 1=x(x^{-1}\rhd_r 1)=x1=x$  for all $x\in G$. By Lemma \ref{lem:braidedsolYBE},  for every $x\in G$, we have
    \[x+_r (x\rhd_r x^{-1})=x(x^{-1}\rhd_r(x\rhd_r x^{-1}))=xx^{-1}=1\]
and by (\ref{LYZequa1}),
\begin{align*}
    (x\rhd_r x^{-1})+_r x=&(x\rhd_r x^{-1})((x\rhd_r x^{-1})^{-1}\rhd_r x)\\
    =&(x\rhd_r x^{-1})((x\lhd_r x^{-1})\rhd_r x)\\
    =&x\rhd_r(x^{-1}x)=1.
\end{align*}
By Lemma \ref{lem:braidedsolYBE} and (\ref{LYZequa}), we have that
\begin{align*}
    (x+_r y)+_r z=&x(x^{-1}\rhd_r y)(((x^{-1}\rhd_r y)^{-1}x^{-1})\rhd_r z)\\
    =&x(x^{-1}\rhd_r y)((x^{-1}\rhd_r y)^{-1}\rhd_r (x^{-1}\rhd_r z))\\
    =&x((x^{-1}\rhd_r y)+_r(x^{-1}\rhd_r z))\\
    =&x(x^{-1}\rhd_r(y+_r z))=x+_r (y+_r z)
\end{align*}
for all $x,y,z\in G$. Hence $(G,+_r)$ is a group. By Lemma \ref{lem:braidedsolYBE}, 
\[x+(x\rhd_r y)=x(x^{-1}\rhd_r(x\rhd_r y))=xy\]
for all $x,y\in G$.
By (\ref{LYZequa}), we have
\begin{align*}
    x(y+_r z)=&x+_r(x\rhd_r (y+_r z))\\
    =&x+_r (x\rhd_r y)+_r(x\rhd_r z)\\
    =&xy-_r x+_r xz
\end{align*}
for all $x,y,z\in G$. Hence $(G,+_r,\circ)$ is a skew brace and $f$ is a well-defined map. Note that the lambda map of this skew brace is defined by
$\lambda_x(y)=-_r x+_r xy=x\rhd_r y$ for all $x,y\in G$.

Now we define $g\colon \mathcal{B}(G)\rightarrow \mathcal{BO}(G)$, by $g(G,+,\circ)=r_+$ and
\[ r_+(x,y)=(-x+xy,(-x+xy)^{-1}xy)=(\lambda_x(y),\mu_y(x))\]
for all $(G,+,\circ)\in\mathcal{B}(G)$ and $x,y\in G$. We shall prove that $r_+$ is a braiding operator on $G$. Note that, by Proposition \ref{pro:lambda}, 
\begin{align*}
    r_+(xy,z)=&(\lambda_{xy}(z),\mu_z(xy))\\
    =&(\lambda_x\lambda_y(z),\lambda_x(\lambda_y(z))^{-1}xyz)\\
    =&(\lambda_x\lambda_y(z),\lambda_x(\lambda_y(z))^{-1}x\lambda_y(z)\lambda_y(z)^{-1}yz)\\
    =&(\lambda_x\lambda_y(z),\mu_{\lambda_y(z)}(x)\mu_z(y))\\
    =&(\id\times m)(r_+)_1(r_+)_2(x,y,z).
\end{align*}
By Propositions \ref{pro:lambda} and \ref{pro:mu},
\begin{align*}
    r_+(x,yz)=&(\lambda_{x}(yz),\mu_{yz}(x))\\
    =&(\lambda_x(y+\lambda_y(z)),\mu_z\mu_y(x))\\
    =&(\lambda_x(y)+\lambda_x\lambda_y(z),\mu_z\mu_y(x))\\
    =&(\lambda_x(y)\lambda_{\lambda_x(y)^{-1}xy}(z),\mu_z\mu_y(x))\\
    =&(\lambda_x(y)\lambda_{\mu_y(x)}(z),\mu_z\mu_y(x))\\
    =&(m\times \id)(r_+)_2(r_+)_1(x,y,z).
\end{align*}
We also have that
\[ r_+(1,x)=(\lambda_1(x),\mu_x(1))=(x,1),\]
\[ r_+(x,1)=(\lambda_x(1),\mu_1(x))=(1,x),\]
and
\[ \lambda_x(y)\mu_y(x)=\lambda_x(y)\lambda_x(y)^{-1}xy=xy\]
for all $x,y\in G$. Hence $r_+$ is a braiding operator on $G$, thus $g$ is a well-defined map. Now it is easy to check that $fg=\id_{\mathcal{B}(G)}$ and $gf=\id_{\mathcal{BO}(G)}$.
\end{proof}

\section{The structure group of a solution}
\index{Solution!structure group}
\index{Structure group}
Let $(X,r)$ be a solution to the YBE. Write as usual $r(x,y)=(\sigma_x(y),\tau_y(x))$. The {\em structure group} of $(X,r)$ is the group
\[ G(X,r)=\gr(X : xy=\sigma_x(y)\tau_y(x), \text{ for all }x,y\in X).\]
In this section we shall see that there is a natural braiding operator on $G(X,r)$ induced by $r$ that satisfies a nice universal property. 

To do so we prove first that there are interesting extensions of the solution $(X,r)$.

\begin{proposition}
\label{prop:extendsol}
Let $(X,r)$ be a solution to the YBE. Let $X'=\{x'\mid x\in X\}$ be a copy of $X$ and let $Y$ be the disjoint union of $X$ and $X'$. We write $r(x,y)=(\sigma_x(y),\tau_y(x))$ and $r^{-1}(x,y)=(\widehat{\sigma}_x(y),\widehat{\tau}_y(x))$. We define $r'\colon Y\times Y\rightarrow Y\times Y$ by $r'(u,v)=(\sigma'_u(v),\tau'_v(u))$, for all $u,v\in Y$, where
\[ \sigma'_x(y)=\sigma_x(y),\; \sigma'_{x'}(y)=\sigma^{-1}_x(y),\; \sigma'_{x}(y')=\widehat{\tau}^{-1}_x(y)',\; \sigma'_{x'}(y')=\widehat{\tau}_x(y)'\]
and
\[ \tau'_x(y)=\tau_x(y),\; \tau'_{x'}(y)=\tau^{-1}_x(y),\; \tau'_{x}(y')=\widehat{\sigma}^{-1}_x(y)',\; \tau'_{x'}(y')=\widehat{\sigma}_x(y)',\]
for all $x,y\in X$. Then $(Y,r')$ is a solution to the YBE. Furthermore, $(Y,r')$ is involutive if and only if $(X,r)$ is involutive.
\end{proposition}

\begin{proof}
Let $x,y\in X$. Since 
\[ r(\tau^{-1}_y(x),y)=(\sigma_{\tau^{-1}_y(x)}(y),x) \text{ and }
r(y,\sigma^{-1}_y(x))=(x,\tau_{\sigma^{-1}_y(x)}(y)),\] we have that
\[ \widehat{\tau}^{-1}_x(y)=\sigma_{\tau^{-1}_y(x)}(y)\text{ and } \widehat{\sigma}^{-1}_x(y)=\tau_{\sigma^{-1}_y(x)}(y).\]
Similarly one can see that
\[ \tau^{-1}_x(y)=\widehat{\sigma}_{\widehat{\tau}^{-1}_y(x)}(y)\text{ and } \sigma^{-1}_x(y)=\widehat{\tau}_{\widehat{\sigma}^{-1}_y(x)}(y).\]
Now it is easy to check that $r'$ is bijective and
\[ (r')^{-1}(x,y)=(\widehat{\sigma}_x(y),\widehat{\tau}_y(x)),\; (r')^{-1}(x',y)=(\widehat{\sigma}^{-1}_x(y),\sigma^{-1}_y(x)'),\] 
\[ (r')^{-1}(x,y')=(\tau^{-1}_x(y)',\widehat{\tau}^{-1}_y(x)), \; (r')^{-1}(x',y')=(\sigma_x(y)',\tau_y(x)')\]
for all $x,y\in X$.

Hence, to show that $(Y,r')$ is a solution to the YBE,  
by Lemma \ref{lem:YB} it is enough to prove that
\begin{itemize}
    \item[(a)] $\sigma'_{u}\sigma'_{v}=\sigma'_{\sigma'_u(v)}\sigma'_{\tau'_v(u)}$,
    \item[(b)]$\sigma'_{\tau'_{\sigma'_u(v)}(w)}\tau'_v(u)=\tau'_{\sigma'_{\tau'_u(w)}(v)}\sigma'_w(u)$,
    \item[(c)] $\tau'_{u}\tau'_{v}=\tau'_{\tau'_u(v)}\tau'_{\sigma'_v(u)}$,
\end{itemize}
for all $u,v,w\in Y$.

Let $x,y,z\in X$. Since $(X,r^{-1})$ is a solution to the YBE, by Lemma \ref{lem:YB}, we have that
\begin{align*}
    \sigma'_{x}\sigma'_{y}(z)=&\sigma_{x}\sigma_{y}(z)=\sigma_{\sigma_x(y)}\sigma_{\tau_y(x)}(z)\\
    =&\sigma'_{\sigma'_x(y)}\sigma'_{\tau'_y(x)}(z),\\
        \sigma'_{x}\sigma'_{y}(z')=&\sigma'_{x}(\widehat{\tau}^{-1}_{y}(z)')=\widehat{\tau}^{-1}_{x}\widehat{\tau}^{-1}_{y}(z)'\\
        =&\widehat{\tau}^{-1}_{\sigma_x(y)}\widehat{\tau}^{-1}_{\tau_y(x)}(z)'\\
    =&\sigma'_{\sigma'_x(y)}\sigma'_{\tau'_y(x)}(z').
    \end{align*}
Hence $\sigma'_x\sigma'_y=\sigma'_{\sigma'_x(y)}\sigma'_{\tau'_y(x)}$. We also have that   
\begin{align*}
    \sigma'_{x}\sigma'_{y'}(z)=&\sigma_{x}\sigma^{-1}_{y}(z)=\sigma^{-1}_{\sigma_{\tau^{-1}_y(x)}(y)}\sigma_{\tau^{-1}_y(x)}(z)\\
    =&\sigma^{-1}_{\widehat{\tau}^{-1}_x(y)}\sigma_{\tau^{-1}_y(x)}(z)=\sigma'_{\sigma'_x(y')}\sigma'_{\tau'_{y'}(x)}(z),\\
        \sigma'_{x}\sigma'_{y'}(z')=&\sigma'_{x}(\widehat{\tau}_{y}(z)')=\widehat{\tau}^{-1}_{x}\widehat{\tau}_{y}(z)'\\
        =&\widehat{\tau}_{\widehat{\tau}^{-1}_x(y)}\widehat{\tau}^{-1}_{\sigma_{\tau^{-1}_x(y)}(x)}(z)'=\widehat{\tau}_{\widehat{\tau}^{-1}_x(y)}\widehat{\tau}^{-1}_{\widehat{\tau}^{-1}_y(x)}(z)'\\
    =&\sigma'_{\sigma'_x(y')}\sigma'_{\tau'_{y'}(x)}(z').
    \end{align*}
Hence $\sigma'_x\sigma'_{y'}=\sigma'_{\sigma'_x(y')}\sigma'_{\tau'_{y'}(x)}$. Similarly one can check that
$\sigma'_{x'}\sigma'_{y}=\sigma'_{\sigma'_{x'}(y)}\sigma'_{\tau'_{y}(x')}$ and $\sigma'_{x'}\sigma'_{y'}=\sigma'_{\sigma'_{x'}(y')}\sigma'_{\tau'_{y'}(x')}$. Hence (a) follows.
By a symmetric argument (c) follows.

Now we prove (b). By Lemma \ref{lem:YB}, we have
\begin{align*}
    \sigma'_{\tau'_{\sigma'_{x'}(y)}(z)}\tau'_y(x')=&
    \sigma'_{\tau'_{\sigma^{-1}_{x}(y)}(z)}\left(\widehat{\sigma}^{-1}_y(x)'\right)\\
    =&
    \sigma'_{\tau_{\sigma^{-1}_{x}(y)}(z)}\left(\tau_{\sigma^{-1}_x(y)}(x)'\right)\\
    =&
    \left(\widehat{\tau}^{-1}_{\tau_{\sigma^{-1}_{x}(y)}(z)}\tau_{\sigma^{-1}_x(y)}(x)\right)'\\
     =&
    \left(\sigma_{\tau^{-1}_{\tau_{\sigma^{-1}_x(y)}(x)}\tau_{\sigma^{-1}_{x}(y)}(z)}\tau_{\sigma^{-1}_x(y)}(x)\right)'\\
    =&
    \left(\sigma_{\tau_{y}\tau^{-1}_{x}(z)}\tau_{\sigma^{-1}_x(y)}(x)\right)'\\
    =&
    \left(\sigma_{\tau_{\sigma_x\sigma^{-1}_x(y)}\tau^{-1}_{x}(z)}\tau_{\sigma^{-1}_x(y)}(x)\right)'\\
    =&
    \left(\tau_{\sigma_{\tau_x\tau^{-1}_x(z)}\sigma^{-1}_{x}(y)}\sigma_{\tau^{-1}_x(z)}(x)\right)'\\
    =&\tau'_{\sigma'_{\tau'_{x'}(z)}(y)}\sigma'_z(x'),
\end{align*}
where the last equality follows by a symmetric argument. By similar calculations one \textcolor{red}{gets} that
\[\tau'_{\sigma'_{\tau'_x(y')}(z')}\sigma'_{y'}(x)=\sigma'_{\tau'_{\sigma'_x(z')}(y')}\tau'_{z'}(x).\]
Note that, 
\begin{align*}
    \sigma'_{\tau'_{\sigma'_x(y')}(z)}\tau'_{y'}(x)=&\sigma'_{\tau'_{\widehat{\tau}^{-1}_x(y)'}(z)}\tau^{-1}_y(x)\\
    =&\sigma_{\tau^{-1}_{\sigma_{\tau^{-1}_y(x)}(y)}(z)}\tau^{-1}_y(x)
\end{align*}
and
\begin{align*}
    \tau'_{\sigma'_{\tau'_x(z)}(y')}\sigma'_{z}(x)
    =&\tau'_{\widehat{\tau}^{-1}_{\tau_x(z)}(y)'}\sigma_z(x)\\
    =&\tau^{-1}_{\sigma_{\tau^{-1}_y\tau_x(z)}(y)}\sigma_z(x)\\
    =&\tau^{-1}_{\sigma_{\tau_{\tau^{-1}_y(x)}\tau^{-1}_{\sigma_{\tau^{-1}_y(x)}(y)}(z)}(y)}\sigma_z(x),
\end{align*}
by Lemma \ref{lem:YB}.

Since, by Lemma \ref{lem:YB},
\begin{align*}
\tau_{\sigma_{\tau_{\tau^{-1}_y(x)}\tau^{-1}_{\sigma_{\tau^{-1}_y(x)}(y)}(z)}(y)}&\sigma_{\tau^{-1}_{\sigma_{\tau^{-1}_y(x)}(y)}(z)}\tau^{-1}_y(x)\\
=&\sigma_{\tau_{\sigma_{\tau^{-1}_y(x)}(y)}\tau^{-1}_{\sigma_{\tau^{-1}_y(x)}(y)}(z)} \tau_y{\tau^{-1}_y}(x)\\
=&\sigma_z(x), 
\end{align*}
we have that
\[\sigma'_{\tau'_{\sigma'_x(y')}(z)}\tau'_{y'}(x)=\tau'_{\sigma'_{\tau'_x(z)}(y')}\sigma'_{z}(x).\]
By a symmetric argument one can prove that
\[\tau'_{\sigma'_{\tau'_x(y')}(z)}\sigma'_{y'}(x)=\sigma'_{\tau'_{\sigma'_x(z)}(y')}\tau'_{z}(x).\]
By a similar calculations one \textcolor{red}{gets} that
\[\tau'_{\sigma'_{\tau'_{x'}(y)}(z')}\sigma'_{y}(x')=\sigma'_{\tau'_{\sigma'_{x'}(z')}(y)}\tau'_{z'}(x')\]
and
\[\sigma'_{\tau'_{\sigma'_{x'}(y)}(z')}\tau'_{y}(x')=\tau'_{\sigma'_{\tau'_{x'}(z')}(y)}\sigma'_{z'}(x').\]
Hence, by Lemma \ref{lem:YB}, \textcolor{red}{(b)} follows. Therefore $(Y,r')$ is a solution to the YBE. 

By the definition of $r'$, it is clear that if $(r')^2=\id_{Y^2}$, then $r^2=\id_{X^2}$. Suppose that $r^2=\id_{X^2}$. We have that
\[ (r')^2(x,y)=r^2(x,y)=(x,y)\]
for all $x,y\in X$. Since
\[ r'(x',y)=(\sigma'_{x'}(y),\tau'_y(x'))=(\sigma^{-1}_x(y),\sigma^{-1}_y(x)')\]
and 
\[ r'(x,y')=(\sigma'_{x}(y'),\tau'_{y'}(x))=(\tau^{-1}_x(y)',\tau^{-1}_y(x))\]
for all $x,y\in X$, we have that
\[ (r')^2(x',y)=(\tau^{-1}_{\sigma^{-1}_x(y)}\sigma^{-1}_y(x)',\tau^{-1}_{\sigma^{-1}_y(x)}\sigma^{-1}_x(y))\]
and
\[ (r')^2(x,y')=(\sigma^{-1}_{\tau^{-1}_x(y)}\tau^{-1}_y(x),\sigma^{-1}_{\tau^{-1}_y(x)}\tau^{-1}_x(y))'\]
for all $x,y\in X$. Note that
\[\sigma_x\tau_{\sigma^{-1}_y(x)}(y)=\sigma_x\sigma^{-1}_{\sigma_y(\sigma^{-1}_y(x))}(y)=y\]
and
\[
\tau_x\sigma_{\tau^{-1}_y(x)}(y)=\tau_x\tau^{-1}_{\tau_y(\tau^{-1}_y(x))}(y)=y
\]
for all $x,y\in X$. Hence
\[ 
(r')^2(x',y)=(x',y)\; \text{ and }\; (r')^2(x,y')=(x,y')
\]
for all $x,y\in X$.
Finally, 
\begin{align*}
(r')^2(x',y')=&r'(\sigma'_{x'}(y'),\tau'_{y'}(x'))
=r'(\tau_x(y)',\sigma_y(x)')\\
=&(\tau_{\tau_{x}(y)}\sigma_y(x)',\sigma_{\sigma_y(x)}\tau_x(y)')\\
=&(\tau_{\tau_{x}(y)}\tau^{-1}_{\tau_x(y)}(x)',\sigma_{\sigma_y(x)}\sigma^{-1}_{\sigma_y(x)}(y)')=(x',y')
\end{align*}
for all $x,y\in X$. Therefore $(r')^2=\id_{Y^2}$, and the result follows.
\end{proof}

We use the following notation. For every solution $(X,r)$ of the YBE,
\[r_k=\id^{\times (k-1)}\times r\times\id^{\times (n-k-1)}\colon X^{n}\rightarrow X^n\]
for all $1\leq k<n$ and $n\geq 3$.
\begin{theorem}
\label{thm:solfreemonoid}
Let $(X,r)$ be a solution of the YBE. Let $\mathrm{FM}(X)$ be the free monoid on $X$. Let $m_k\colon X^{k}\rightarrow \mathrm{FM}(X)$ be the map defined by $m_k(x_1,\dots ,x_k)=x_1\cdots x_k$ for every positive integer $k$. Let $r_{FM}\colon \mathrm{FM}(X)\times \mathrm{FM}(X)\rightarrow \mathrm{FM}(X)\times \mathrm{FM}(X)$ be the map defined by $r_{FM}(x_1\cdots x_n,\, y_1\cdots y_k)$
\[=(m_k\times m_n)(r_k\cdots r_2r_1)(r_{k+1}\cdots r_3r_2)\cdots(r_{n+k-1}\cdots r_{n+1}r_n)(x_1,\dots ,x_n,y_1,\dots ,y_k)\]
for all $x_1,\dots ,x_n,y_1,\dots ,y_k\in X$ and all positive integers $n,k$,   $r_{FM}(1,a)=(a,1)$ and $r_{FM}(a,1)=(1,a)$ for all $a\in \mathrm{FM}(X)$. Then $(\mathrm{FM}(X),r_{FM})$ is a solution to the YBE. Furthermore, $(\mathrm{FM}(X),r_{FM})$ is involutive if and only if $(X,r)$ is involutive.
\end{theorem}

\begin{proof}
Note that $r_{FM}(x,y)=r_1(x,y)=r(x,y)$ for all $x,y\in X$. Hence if $(r_{FM})^2=\id_{\mathrm{FM}(X)^2}$, the $r^2=\id_{X^2}$.

Note that 
\begin{equation}\label{eq:SFcom}
r_{l+m}r_{m}=r_mr_{l+m}
\end{equation}
for all $1\leq m$ and $2\leq l$. Hence
\begin{align}\label{eq:solfree}
r_m(r_{m+l}\cdots r_{m+1}r_m)=&r_{m+l}\cdots r_{m+2}r_mr_{m+1}r_m\nonumber\\
=&(r_{m+l}\cdots r_{m+2}r_{m+1}r_{m})r_{m+1}.
\end{align}

Suppose that
$r^2=\id_{X^2}$. By using (\ref{eq:SFcom}) and (\ref{eq:solfree}), we have that
\begin{align*}
    &(r_k\cdots r_2r_1)(r_{k+1}\cdots r_3r_2)\cdots(r_{n+k-1}\cdots r_{n+1}r_n)\\
    &\cdot (r_n\cdots r_2r_1)(r_{n+1}\cdots r_3r_2)\cdots(r_{k+n-1}\cdots r_{k+1}r_k)\\
    &=(r_k\cdots r_2r_1)(r_{k+1}\cdots r_3r_2)\cdots(r_{n+k-1}\cdots r_{n+1})\\
    &\cdot (r_{n-1}\cdots r_2r_1)(r_{n+1}\cdots r_3r_2)\cdots(r_{k+n-1}\cdots r_{k+1}r_k)\\
    &=(r_k\cdots r_2r_1)(r_{k+1}\cdots r_3r_2)\cdots(r_{n+k-2}\cdots r_{n-1})\\
    &\cdot (r_{n-1}\cdots r_2r_1)(r_{n+k-1}\cdots r_{n+1})(r_{n+1}\cdots r_3r_2)\cdots(r_{k+n-1}\cdots r_{k+1}r_k)\\
    &=\dots =(r_k\cdots r_2)(r_{k+1}\cdots r_3)\cdots(r_{n+k-1}\cdots r_{n+1})\\
    &\cdot (r_{n+1}\cdots r_3r_2)\cdots(r_{k+n-1}\cdots r_{k+1}r_k)\\
    &=\dots =(r_k)(r_{k+1})\cdots(r_{n+k-1})\\
    &\cdot (r_{k+n-1}\cdots r_{k+1}r_k)=\id_{X^{k+n}}
\end{align*}
for positive integers $n,k$.
Hence $(r_{FM})^2=\id_{\mathrm{FM}(X)^2}$.


 We write $r_{FM}(a,b)=(\sigma_a(b),\tau_b(a))$ for all $a,b\in \mathrm{FM}(X)$. We have that
\begin{align*}
(\id\times r_{FM})(r_{FM}\times \id)&(\id\times r_{FM})(1,a,b)\\
=&(\id\times r_{FM})(r_{FM}\times \id)(1,\sigma_a(b),\tau_b(a))\\
=&(\id\times r_{FM})(\sigma_a(b),1,\tau_b(a))=(\sigma_a(b),\tau_b(a),1)
\end{align*}
and
\begin{align*}
(r_{FM}\times \id)(\id\times r_{FM})&(r_{FM}\times \id)(1,a,b)\\
=&(r_{FM}\times \id)(\id\times r_{FM})(a,1,b)\\
=&(r_{FM}\times \id)(a,b,1)=(\sigma_a(b),\tau_b(a),1)
\end{align*}
Similarly one can check that
\[(\id\times r_{FM})(r_{FM}\times \id)(\id\times r_{FM})(a,1,b)=
(r_{FM}\times \id)(\id\times r_{FM})(r_{FM}\times \id)(a,1,b)\]
and
\[(\id\times r_{FM})(r_{FM}\times \id)(\id\times r_{FM})(a,b,1)=
(r_{FM}\times \id)(\id\times r_{FM})(r_{FM}\times \id)(a,b,1).\]
Hence
to prove that
\[(\id\times r_{FM})(r_{FM}\times \id)(\id\times r_{FM})=(r_{FM}\times \id)(\id\times r_{FM})(r_{FM}\times \id)\]
we need to show that
\begin{align*}
    (r_{k+t}\cdots r_{1+t})&\cdots (r_{k+t+n-1}\cdots r_{t+n})(r_{t}\cdots r_2r_{1})\cdots (r_{n+t-1}\cdots r_{n})\\
    &\cdot (r_{n+t}\cdots r_{n+1})\cdots (r_{n+k+t-1}\cdots r_{n+k})
    (x_1,\dots ,x_n,y_1,\dots ,y_k,z_1,\dots,z_t)\\
    =&(r_{t}\cdots r_2r_{1})\cdots (r_{k+t-1}\cdots r_{k})(r_{k+t}\cdots r_{k+1})\cdots (r_{n+k+t-1}\cdots r_{n+k})\\
    &\cdot (r_{k}\cdots r_2r_{1})\cdots (r_{n+k-1}\cdots r_{n})(x_1,\dots ,x_n,y_1,\dots ,y_k,z_1,\dots,z_t)
\end{align*}
for all positive integers $n,k,t$ and $x_1,\dots ,x_n,y_1,\dots ,y_k,z_1,\dots ,z_t\in X$.

By using (\ref{eq:SFcom}) and (\ref{eq:solfree}), we have that
\begin{align*}
    (r_{t+k}\cdots r_{t+1})&\cdots (r_{t+k+n-1}\cdots r_{t+n})(r_{t}\cdots r_2r_{1})\cdots (r_{n+t-1}\cdots r_{n})\\
    &\cdot (r_{n+t}\cdots r_{n+1})\cdots (r_{n+k+t-1}\cdots r_{n+k})\\
    =&(r_{t+k}\cdots r_{t+1}r_t\cdots r_2r_1)\cdots (r_{t+k+n-1}\cdots r_{t+n}r_{n+t-1}\cdots r_{n})\\
    &\cdot (r_{n+t}\cdots r_{n+1})\cdots (r_{n+k+t-1}\cdots r_{n+k})\\
     =&(r_t\cdots r_2r_1)(r_{t+k}\cdots r_{t+1}r_t\cdots r_2r_1)\\
     &\cdots (r_{t+k+n-1}\cdots r_{t+n}r_{n+t-1}\cdots r_{n})\\
    &\cdot (r_{n+t+1}\cdots r_{n+2})\cdots (r_{n+k+t-1}\cdots r_{n+k})\\
    =&(r_t\cdots r_2r_1)\cdots (r_{k+t-1}\cdots r_{k})\\
    &\cdot (r_{t+k}\cdots r_{t+1}r_t\cdots r_2r_1)\cdots (r_{t+k+n-1}\cdots r_{t+n}r_{n+t-1}\cdots r_{n})\\
    =&(r_t\cdots r_2r_1)\cdots (r_{k+t-1}\cdots r_{k})\\
    &\cdot (r_{k+t}\cdots r_{k+1}r_k\cdots r_2r_1)\cdots (r_{n+k+t-1}\cdots r_{n+k}r_{n+k-1}\cdots r_{n})\\
    =&(r_t\cdots r_2r_1)\cdots (r_{k+t-1}\cdots r_{k})\\
    &\cdot (r_{k+t}\cdots r_{k+1})\cdots (r_{n+k+t-1}\cdots r_{n+k})(r_k\cdots r_2r_1)\cdots (r_{n+k-1}\cdots r_{n}).
\end{align*}
Hence
\[(\id\times r_{FM})(r_{FM}\times \id)(\id\times r_{FM})=(r_{FM}\times \id)(\id\times r_{FM})(r_{FM}\times \id).\]
By the definition of the free monoid on $X$, every $m_k$ is bijective. Hence by the definition of $r_{FM}$, since $r$ is bijective, we have that $r_{FM}$ is bijective.

Note that $\sigma_1=\id$ and $\tau_1=\id$. We know that the restrictions of $\sigma_x$ and $\tau_x$ on $X$ for all $x\in X$ are bijective.
We shall prove, by induction on $k$, that the restrictions of $\sigma_x$ and $\tau_x$ on the set $X_k$ of words of length $k$ for all $x\in X$ are bijective. For $k=0$, $\sigma_x(1)=1$ and $\tau_x(1)=1$, by the definition of $r_{FM}$. Suppose that $k\geq 1$ and that the restrictions of $\sigma_x$ and $\tau_x$ on the set $X_l$ of words of length $l\leq k$ for all $x\in X$ are bijective. Let $x,y_1,\dots ,y_{k+1}\in X$. By the definition of $r_{FM}$ we have that
\[\sigma_x(y_1\cdots y_{k+1})=\sigma_x(y_1\cdots y_k)\sigma_{\tau_{y_k}\cdots \tau_{y_1}(x)}(y_{k+1}).\]
Let $z_1,\dots,z_{k+1}\in X$  such that $\sigma_x(y_1\cdots y_{k+1})=\sigma_x(z_1\cdots z_{k+1})$. Hence
\[\sigma_x(y_1\cdots y_{k})=\sigma_x(z_1\cdots z_{k}) \text{ and }
\sigma_{\tau_{y_k}\cdots \tau_{y_1}(x)}(y_{k+1})=\sigma_{\tau_{z_k}\cdots \tau_{z_1}(x)}(z_{k+1}).\]
By the inductive hypothesis the restriction of $\sigma_x$ on $X_k$ is bijective. Hence $y_1\cdots y_k=z_1\cdots z_k$, and thus $y_i=z_i$ for all $1\leq i\leq k$. Now
\[\sigma_{\tau_{y_k}\cdots \tau_{y_1}(x)}(y_{k+1})=\sigma_{\tau_{z_k}\cdots \tau_{z_1}(x)}(z_{k+1})=\sigma_{\tau_{y_k}\cdots \tau_{y_1}(x)}(z_{k+1})\]
and thus $y_{k+1}=z_{k+1}$. Therefore the restriction of $\sigma_x$ on $X_{k+1}$ is injective. Let $c_1,\dots ,c_{k+1}\in X$. By the inductive hypothesis there exist $x_1,\dots ,x_k\in X$ such that
\[\sigma_x(x_1\cdots x_k)=c_1\cdots c_k.\]
There exists $x_{k+1}\in X$ such that
\[\sigma_{\tau_{x_k}\cdots \tau_{x_1}(x)}(x_{k+1})=z_{k+1}.\]
Hence
\[\sigma_x(x_1\cdots x_{k+1})=c_1\cdots c_{k+1},\]
and thus the restriction of $\sigma_x$ on $X_{k+1}$ is bijective.
Therefore, by induction, $\sigma_x$ is bijective for all $x\in X$.

By the definition of $r_{FM}$ we have that
\[\tau_x(y_1\cdots y_{k+1})=\tau_{\sigma_{y_2}\cdots \sigma_{y_{k+1}}(x)}(y_{1})\tau_x(y_2\cdots y_{k+1}).\]
Let $z_1,\dots,z_{k+1}\in X$  such that $\tau_x(y_1\cdots y_{k+1})=\tau_x(z_1\cdots z_{k+1})$. Hence
\[\tau_x(y_2\cdots y_{k+1})=\tau_x(z_2\cdots z_{k+1}) \text{ and }
\tau_{\sigma_{y_2}\cdots \sigma_{y_{k+1}}(x)}(y_{1})=\tau_{\sigma_{z_2}\cdots \sigma_{z_{k+1}}(x)}(z_{1}).\]
By the inductive hypothesis the restriction of $\tau_x$ on $X_k$ is bijective. Hence $y_2\cdots y_{k+1}=z_2\cdots z_{k+1}$, and thus $y_i=z_i$ for all $2\leq i\leq k+1$. Now
\[\tau_{\sigma_{y_2}\cdots \sigma_{y_{k+1}}(x)}(y_{1})=\tau_{\sigma_{z_2}\cdots \sigma_{z_{k+1}}(x)}(z_{1})=\tau_{\sigma_{y_2}\cdots \sigma_{y_{k+1}}(x)}(z_{1})\]
and thus $y_{1}=z_{1}$. Therefore the restriction of $\tau_x$ on $X_{k+1}$ is injective. Let $c_1,\dots ,c_{k+1}\in X$. By the inductive hypothesis there exist $x_2,\dots ,x_{k+1}\in X$ such that
\[\tau_x(x_2\cdots x_{k+1})=c_2\cdots c_{k+1}.\]
There exists $x_{1}\in X$ such that
\[\tau_{\sigma_{x_2}\cdots \sigma_{x_{k+1}}(x)}(x_{1})=z_{1}.\]
Hence
\[\tau_x(x_1\cdots x_{k+1})=c_1\cdots c_{k+1},\]
and thus the restriction of $\tau_x$ on $X_{k+1}$ is bijective.
Therefore, by induction, $\tau_x$ is bijective for all $x\in X$.

Let $x_1,\dots ,x_{n+1},y_1,\dots ,y_{k+1}\in X$. We shall prove that
\[\sigma_{x_1\cdots x_n}(y_1\cdots y_k)=\sigma_{x_1}\cdots \sigma_{x_n}(y_1\cdots y_k)\] 
and 
\[\tau_{x_1\cdots x_n}(y_1\cdots y_k)=\tau_{x_n}\cdots \tau_{x_1}(y_1\cdots y_k)\]
by induction on $n+k$.
Note that by the definition of $r_{FM}$,
\[\sigma_{x_1\cdots x_n}(y)=\sigma_{x_1}\cdots \sigma_{x_n}(y)\text{ and }\tau_{x_1\cdots x_n}(y)=\tau_{x_n}\cdots \tau_{x_1}(y)\]
for all positive integer $n$ and all $y\in \{1\}\cup X_1$.  Suppose that $k,n\geq 1$ and
\[\sigma_{x_1\cdots x_l}(y_1\cdots y_t)=\sigma_{x_1}\cdots \sigma_{x_l}(y_1\cdots y_t)\]
and
\[\tau_{x_1\cdots x_l}(y_1\cdots y_t)=\tau_{x_l}\cdots \tau_{x_1}(y_1\cdots y_t)\]
for all positive integers $l,t$ such that $l+t\leq n+k$.
Clearly we may assume that $n>1$. By the definition of $r_{FM}$ and the inductive hypothesis, we have that
\begin{align*}
    \sigma_{x_1\cdots x_n}(y_1\cdots y_{k+1})=&\sigma_{x_1\cdots x_n}(y_1\cdots y_k)\sigma_{\tau_{y_{k}}\cdots \tau_{y_1}(x_1\cdots x_n)}(y_{k+1})\\
    =&\sigma_{x_1\cdots x_n}(y_1\cdots y_k)\sigma_{\tau_{y_{1}\cdots y_k}(x_1\cdots x_n)}(y_{k+1})\\
    =&\sigma_{x_1\cdots x_n}(y_1\cdots y_k)\sigma_{\tau_{\sigma_{x_2}\cdots\sigma_{x_n}(y_{1}\cdots y_k)}(x_1)\tau_{y_1\cdots y_k}(x_2\cdots x_n)}(y_{k+1})\\
    =&\sigma_{x_1}\sigma_{x_2\cdots x_n}(y_1\cdots y_k)\sigma_{\tau_{\sigma_{x_2}\cdots\sigma_{x_n}(y_{1}\cdots y_k)}(x_1)}\sigma_{\tau_{y_1\cdots y_k}(x_2\cdots x_n)}(y_{k+1})\\
    =&\sigma_{x_1}\left(\sigma_{x_2\cdots x_n}(y_1\cdots y_k)\sigma_{\tau_{y_1\cdots y_k}(x_2\cdots x_n)}(y_{k+1})\right)\\
    =&\sigma_{x_1}\left(\sigma_{x_2\cdots x_n}(y_1\cdots y_k)\sigma_{\tau_{y_k}\cdots \tau_{y_1}(x_2\cdots x_n)}(y_{k+1})\right)\\
    =&\sigma_{x_1}\sigma_{x_2\cdots x_n}(y_1\cdots y_{k+1})\\
    =&\sigma_{x_1}\cdots\sigma_{x_n}(y_1\cdots y_{k+1})
    \end{align*}
and
\begin{align*}
    \tau_{x_1\cdots x_n}(y_1\cdots y_{k+1})=&\tau_{\sigma_{y_{2}}\cdots \sigma_{y_{k+1}}(x_1\cdots x_n)}(y_{1})\tau_{x_1\cdots x_n}(y_2\cdots y_{k+1})\\
    =&\tau_{\sigma_{y_{2}\cdots y_{k+1}}(x_1\cdots x_n)}(y_{1})\tau_{x_1\cdots x_n}(y_2\cdots y_{k+1})\\
    =&\tau_{\sigma_{y_2\cdots y_{k+1}}(x_1\cdots x_{n-1})\sigma_{\tau_{x_{n-1}}\cdots\tau_{x_{1}}(y_{2}\cdots y_{k+1})}(x_n)}(y_{1})\tau_{x_1\cdots x_n}(y_2\cdots y_{k+1})\\
    =&\tau_{\sigma_{\tau_{x_{1}\cdots x_{n-1}}(y_{2}\cdots y_{k+1})}(x_n)}\tau_{\sigma_{y_2\cdots y_{k+1}}(x_1\cdots x_{n-1})}(y_{1})\\
    &\cdot\tau_{x_n}\tau_{x_1\cdots x_{n-1}}(y_2\cdots y_{k+1})\\
    =&\tau_{x_n}\left(\tau_{\sigma_{y_2\cdots y_{k+1}}(x_1\cdots x_{n-1})}(y_{1})\tau_{x_1\cdots x_{n-1}}(y_2\cdots y_{k+1})\right)\\
    =&\tau_{x_n}\left(\tau_{\sigma_{y_2}\cdots\sigma_{y_{k+1}}(x_1\cdots x_{n-1})}(y_{1})\tau_{x_1\cdots x_{n-1}}(y_2\cdots y_{k+1})\right)\\
    =&\tau_{x_n}\tau_{x_1\cdots x_{n-1}}(y_1\cdots y_{k+1})\\
    =&\tau_{x_n}\cdots\tau_{x_1}(y_1\cdots y_{k+1}).
    \end{align*}
Therefore, by induction,
\[\sigma_{x_1\cdots x_n}=\sigma_{x_1}\cdots\sigma_{x_n}\text{ and }\tau_{x_1\cdots x_n}=\tau_{x_n}\cdots \tau_{x_1}\]
for all positive integer $n$. Hence $\sigma_a$ and $\tau_a$ are bijective for all $a\in FM(X)$, and thus $(FM(X),r_{FM})$ is a solution to the YBE.
\end{proof}

\begin{theorem}
\label{thm:LYZ9}
Let $(X,r)$ be a solution to the YBE. Let $i\colon X\rightarrow G(X,r)$ be the natural map. Then there exists a unique braiding operator $r_G$ on $G(X,r)$ such that $r_G (i\times i)=(i\times i) r$. Furthermore, if $(H,s)$ is a braided group and $j\colon X\rightarrow H$ is a map such that
$s (j\times j)=(j\times j) r$, then there exists a unique group homomorphism $f\colon G(X,r)\rightarrow H$ such that $s (f\times f)=(f\times f) r_G$ and $j=f i$. Furthermore, if $r^2=\id_{X^2}$, then $r_G^2=\id_{G(X,r)^2}$.
\end{theorem}

\begin{proof}
We use the notation of Proposition \ref{prop:extendsol}. Thus, by Proposition \ref{prop:extendsol}, $(Y,r')$ is a solution to the YBE. By Theorem \ref{thm:solfreemonoid}, we extend $(Y,r')$ to a solution $(FM(Y),r'_{FM})$ to the YBE. We write $r'_{FM}(a,b)=(\sigma'_a(b),\tau'_b(a))$ for all $a,b\in FM(Y)$. Note that, by the definition of $r'_{FM}$,
\[\sigma'_{ab}=\sigma'_a\sigma'_b,\quad \tau'_{ab}=\tau'_b\tau'_a,\]
\[\sigma'_c(ab)=\sigma'_c(a)\sigma'_{\tau'_a(c)}(b)\quad\text{and}\quad
\tau'_c(ab)=\tau'_{\sigma'_b(c)}(a)\tau'_c(b)\]
for all $a,b,c\in FM(Y)$. Hence it is easy to check that $r'_{FM}$ satisfies conditions (1), (2) and (3) of the Definition \ref{braidedgroup}. 

Note that $G(X,r)=FM(Y)/\sim$, where $\sim$ is the equivalence relation on $FM(Y)$ generated by $axx'b\sim ab$, $ax'xb\sim ab$ and $axyb\sim a\sigma_x(y)\tau_y(x)b$ for all $a,b\in FM(Y)$ and $x,y\in X$.

We define $r_G(\overline{a},\overline{b})=(\overline{\sigma'_a(b)},\overline{\tau'_b(a)})$ for all $a,b\in FM(Y)$, where $\overline{a}\in G(X.r)$ is \textcolor{red}{the} $\sim$-class of $a\in FM(Y)$. By Lemma \ref{lem:YB}, \[\sigma'_{xy}=\sigma'_{x}\sigma'_{y}=\sigma'_{\sigma_x(y)}\sigma'_{\tau_y(x)}=\sigma'_{\sigma_x(y)\tau_y(x)}\]
and
\[\tau'_{xy}=\tau'_{y}\tau'_{x}=\tau'_{\tau_y(x)}\tau'_{\sigma_x(y)}=\tau'_{\sigma_x(y)\tau_y(x)}\]
for all $x,y\in X$. Hence, to prove that $r_G$ is well-defined it is enough to check that \[\id=\sigma'_{xx'}=\sigma_{x'x}=\tau'_{xx'}=\tau'_{x'x},\] 
\[ \sigma'_{a}(bxx'c)\sim\sigma'_{a}(bc)\sim \sigma'_a(bx'xc),\]
\[ \tau'_{a}(bxx'c)\sim\tau'_{a}(bc)\sim \tau'_a(bx'xc),\]
\[\sigma'_{a}(bxyc)\sim \sigma'_a(b\sigma_x(y)\tau_y(x)c)\]
and 
\[\tau'_{a}(bxyc)\sim \tau'_a(b\sigma_x(y)\tau_y(x)c)\]
for all $a,b,c\in FM(Y)$ and $x,y\in X$.
We shall prove that $\sigma'_{xx'}(a)=a$ for all $x\in X$ and $a\in FM(Y)$ by induction on the length of $a$. Let $x,y\in X$. We have that
\[\sigma'_{xx'}(y)=\sigma'_x\sigma'_{x'}(y)=\sigma_x\sigma^{-1}_x(y)=y,\]
\[\sigma'_{xx'}(y')=\sigma'_x\sigma'_{x'}(y')=\sigma'_x(\widehat{\tau}_x(y)')=\widehat{\tau}^{-1}_x\widehat{\tau}_x(y)'=y'.\]
Let $n$ be a positive integer and let $y_1,\dots ,y_{n+1}\in Y$. Suppose that $\sigma_{xx'}(a)=a$ for all $x\in X$ and all $a\in FM(Y)$ of length $\leq n$. By the inductive hypothesis, we have that
\begin{align*}
    \sigma'_{xx'}(y_1\cdots y_{n+1})=&\sigma'_{xx'}(y_1\cdots y_n)\sigma'_{\tau'_{y_n}\cdots \tau'_{y_1}(xx')}(y_{n+1})\\
    =&y_1\cdots y_n\sigma'_{\tau'_{y_n}\cdots \tau'_{y_1}(xx')}(y_{n+1}).
\end{align*}
Note that $\tau'_{y}(xx')=\tau'_{\sigma'_{x'}(y)}(x)\tau'_y(x')=\tau_{\sigma^{-1}_x(y)}(x)\widehat{\sigma}_y^{-1}(x)'=\widehat{\sigma}_y^{-1}(x)\widehat{\sigma}_y^{-1}(x)'$ and
\begin{align*}\tau'_{y'}(xx')=&\tau'_{\sigma'_{x'}(y')}(x)\tau'_{y'}(x')
=\tau'_{\widehat{\tau}_x(y)'}(x)\widehat{\sigma}_y(x)'\\
=&\tau^{-1}_{\widehat{\tau}_x(y)}(x)\widehat{\sigma}_y(x)'=\widehat{\sigma}_{\widehat{\tau}^{-1}_x(\widehat{\tau}_x(y))}(x)\widehat{\sigma}_y(x)'\\
=&\widehat{\sigma}_y(x)\widehat{\sigma}_y(x)'
\end{align*} 
for all $x,y\in X$. Hence $\sigma'_{xx'}(y_1\cdots y_{n+1})=y_1\cdots y_{n+1}$, and thus, by induction, $\sigma'_{xx'}(a)=a$ for all $x\in X$ and all $a\in FM(Y)$. 
Similarly one can check that $\sigma'_{x'x}=\id=\tau'_{xx'}=\tau'_{x'x}$ for all $x\in X$. 

Let $x\in X$ and $a,b,c\in FM(Y)$. We have
\begin{align*}
    \sigma'_a(bxx'c)=&\sigma'_a(bxx')\sigma'_{\tau'_{bxx'}(a)}(c)\\
    =&\sigma'_a(b)\sigma'_{\tau'_{b}(a)}(xx')\sigma'_{\tau'_{xx'}\tau'_b(a)}(c)\\
    =&\sigma'_a(b)\sigma'_{\tau'_b(a)}(xx')\sigma'_{\tau'_b(a)}(c)
\end{align*}
and
\[ \sigma'_a(bc)=\sigma'_a(b)\sigma'_{\tau'_{b}(a)}(c).
\]
Note that $\sigma'_{y'}(xx')=\sigma'_{y'}(x)\sigma'_{\tau'_{x}(y')}(x')
=\sigma^{-1}_{y}(x)\sigma'_{\widehat{\sigma}^{-1}_{x}(y)'}(x')
=\sigma^{-1}_{y}(x)\widehat{\tau}_{\widehat{\sigma}^{-1}_{x}(y)}(x)'
=\sigma^{-1}_{y}(x)\sigma^{-1}_{y}(x)'$ and \begin{align*}\sigma'_{y}(xx')=&\sigma'_{y}(x)\sigma'_{\tau'_{x}(y)}(x')=\sigma_y(x)\sigma'_{\tau_x(y)}(x')\\
=&\sigma_y(x)\widehat{\tau}^{-1}_{\tau_x(y)}(x)'=\sigma_y(x)\sigma_{\tau_x^{-1}(\tau_x(y))}(x)'\\
=&\sigma_y(x)\sigma_{y}(x)'
\end{align*} 
for all $x,y\in X$. Hence
\[\sigma'_a(bxx'c)\sim\sigma'_a(bc)\]
for all $x\in X$ and $a,b,c\in FM(Y)$. Similarly one can check that
\[ \sigma'_{a}(bx'xc)\sim\sigma'_{a}(bc)\]
and 
\[ \tau'_{a}(bxx'c)\sim\tau'_{a}(bc)\sim \tau'_a(bx'xc)\]
for all $x\in X$ and $a,b,c\in FM(Y)$.

Let $x,y\in X$ and $a,b,c\in FM(Y)$. We have 
\begin{align*}
    \sigma'_a(bxyc)=&\sigma'_a(bxy)\sigma'_{\tau'_{bxy}(a)}(c)\\
    =&\sigma'_a(b)\sigma'_{\tau'_b(a)}(xy)\sigma'_{\tau'_{bxy}(a)}(c)
\end{align*}
and
\begin{align*}
    \sigma'_a(b\sigma_x(y)\tau_y(x)c)=&\sigma'_a(b\sigma_x(y)\tau_y(x))\sigma'_{\tau'_{b\sigma_x(y)\tau_y(x)}(a)}(c)\\
    =&\sigma'_a(b)\sigma'_{\tau'_b(a)}(\sigma_x(y)\tau_y(x))\sigma'_{\tau'_{\sigma_x(y)\tau_y(x)}\tau'_b(a)}(c)\\
    =&\sigma'_a(b)\sigma'_{\tau'_b(a)}(\sigma_x(y)\tau_y(x))\sigma'_{\tau'_{xy}\tau'_b(a)}(c)\\
    =&\sigma'_a(b)\sigma'_{\tau'_b(a)}(\sigma_x(y)\tau_y(x))\sigma'_{\tau'_{bxy}(a)}(c).
\end{align*}
Note that
\begin{align}\label{eq:LYZ1}
    \sigma'_z(xy)=&\sigma'_z(x)\sigma'_{\tau'_x(z)}(y)\nonumber\\
    =&\sigma_z(x)\sigma_{\tau_x(z)}(y),
\end{align}
\begin{align}\label{eq:LYZ2}
    \sigma'_{z'}(xy)=&\sigma'_{z'}(x)\sigma'_{\tau'_x(z')}(y)\nonumber\\
    =&\sigma^{-1}_z(x)\sigma_{\widehat{\sigma}^{-1}_x(z)'}(y)\nonumber\\
    =&\sigma^{-1}_z(x)\sigma^{-1}_{\widehat{\sigma}^{-1}_x(z)}(y)
\end{align}

and by Lemma \ref{lem:YB},
\begin{align}\label{eq:LYZ3}
    \sigma'_z(\sigma_x(y)\tau_y(x))=&\sigma'_z(\sigma_x(y))\sigma'_{\tau'_{\sigma_x(y)}(z)}(\tau_y(x))\nonumber\\
    =&\sigma_z\sigma_x(y)\sigma_{\tau_{\sigma_x(y)}(z)}(\tau_y(x))\nonumber\\
    =&\sigma_{\sigma_z(x)}\sigma_{\tau_x(z)}(y)\tau_{\sigma_{\tau_x(z)}(y)}(\sigma_z(x))
\end{align}
and 
\begin{align}\label{eq:LYZ4}
    \sigma'_{z'}(\sigma_x(y)\tau_y(x))=&\sigma'_{z'}(\sigma_x(y))\sigma'_{\tau'_{\sigma_x(y)}(z')}(\tau_y(x))\nonumber\\
    =&\sigma^{-1}_z\sigma_x(y)\sigma'_{\tau'_{\sigma'_x(y)}(z')}(\tau'_y(x))\nonumber\\
    =&\sigma_{\sigma^{-1}_z(x)}\sigma^{-1}_{\tau_{\sigma^{-1}_z(x)}(z)}(y)\tau'_{\sigma'_{\tau'_x(z')}(y)}(\sigma'_{z'}(x))\nonumber\\
    =&\sigma_{\sigma^{-1}_z(x)}\sigma^{-1}_{\widehat{\sigma}^{-1}_x(z)}(y)\tau_{\sigma^{-1}_{\widehat{\sigma}^{-1}_x(z)}(y)}(\sigma^{-1}_{z}(x))
\end{align}
for all $x,y,z\in X$. 
Hence, by (\ref{eq:LYZ1}), (\ref{eq:LYZ2}), (\ref{eq:LYZ3}) and (\ref{eq:LYZ4}), we have that 
\[\sigma'_{a}(bxyc)\sim \sigma'_a(b\sigma_x(y)\tau_y(x)c)\]
for all $x,y\in X$ and $a,b,c\in FM(Y)$. Similarly one can check that
\[\tau'_{a}(bxyc)\sim \tau'_a(b\sigma_x(y)\tau_y(x)c)\]
for all $x,y\in X$ and $a,b,c\in FM(Y)$. Therefore $r_G$ is well-defined. Now, by the definition of $r_G$, Proposition \ref{prop:extendsol} and Theorem \ref{thm:solfreemonoid}, if $r^2=\id_{X^2}$, then $r_G^2=\id_{G(X,r)^2}$. 

Since $r'_{FM}$ satisfies conditions (1), (2) and (3) of the Definition \ref{braidedgroup}, $r_G$ also satisfies these conditions.  To prove that $r_G$ is a braiding operator on $G(X,r)$, it is enough to show that
\begin{equation}\label{eq:LYZ5}\overline{ab}=\overline{\sigma'_a(b)\tau'_b(a)}\end{equation}
for all $a,b\in FM(Y)$. 
We shall prove (\ref{eq:LYZ5}) by induction on the length of $ab$. For $b=1$, clearly $a=\sigma'_a(1)\tau'_1(a)$, and thus (\ref{eq:LYZ5}) holds in this case. For $a=1$, we have that $b=\sigma'_1(b)\tau'_b(1)$, and thus (\ref{eq:LYZ5}) holds in this case. Hence we may assume that $a\neq 1$ and $b\neq 1$. Note that
\begin{align*}
    \sigma'_x(y)\tau'_y(x)=& \sigma_x(y)\tau_y(x)\sim xy,\\
    \sigma'_x(y')\tau'_{y'}(x)=& \widehat{\tau}^{-1}_x(y)'\tau^{-1}_y(x)
    \sim  \widehat{\tau}^{-1}_x(y)'\tau^{-1}_y(x)yy'\\
    \sim & \widehat{\tau}^{-1}_x(y)'\sigma_{\tau^{-1}_y(x)}(y)\tau_y(\tau^{-1}_y(x))y'\\
    =& \widehat{\tau}^{-1}_x(y)'\widehat{\tau}^{-1}_x(y)xy'
    \sim xy',\\
    \sigma'_{x'}(y)\tau'_{y}(x')=& \sigma^{-1}_x(y)\widehat{\sigma}^{-1}_y(x)'
    \sim  x'x\sigma^{-1}_x(y)\widehat{\sigma}^{-1}_y(x)'\\
    \sim & x'\sigma_x(\sigma^{-1}_x(y))\tau_{\sigma^{-1}_x(y)}(x)\widehat{\sigma}^{-1}_y(x)'\\
    =& x'y\widehat{\sigma}^{-1}_y(x)\widehat{\sigma}^{-1}_y(x)'
    \sim x'y,\\
    \sigma'_{x'}(y')\tau'_{y'}(x')=& \widehat{\tau}_x(y)'\widehat{\sigma}_y(x)'
    \sim  x'x\widehat{\tau}_x(y)'\widehat{\sigma}_y(x)'\\
    \sim & x'\sigma'_x(\widehat{\tau}_x(y)')\tau'_{\widehat{\tau}_x(y)'}(x)\widehat{\sigma}_y(x)'\\
    =& x'y'\tau^{-1}_{\widehat{\tau}_x(y)}(x)\widehat{\sigma}_y(x)'=x'y'\widehat{\sigma}_{\widehat{\tau}^{-1}_x(\widehat{\tau}_x(y))}(x)\widehat{\sigma}_y(x)'\\
    = &x'y'\widehat{\sigma}_{y}(x)\widehat{\sigma}_y(x)'
    \sim x'y'
    \end{align*}
for all $x,y\in X$. Hence (\ref{eq:LYZ5}) holds for all $a,b\in FM(Y)$ such that the length of $ab$ is $\leq 2$. Let $a,b\in FM(Y)$ be elements such that the length of $ab$ is $n>2$ and suppose that
\[cd\sim \sigma'_c(d)\tau'_d(c)\] 
for all $c,d\in FM(Y)$ such that the length of $cd$ is $<n$. We may assume that $a\neq 1$ and $b\neq 1$. Hence there exist $y,z\in Y$ and $c,d\in FM(Y)$ such that $a=cy$ and $b=zd$. Suppose that $c=1$. In this case, by the inductive hypothesis, we have
\begin{align*}ab=&yzd\sim \sigma'_y(z)\tau'_z(y)d\sim \sigma'_y(z)\sigma'_{\tau'_z(y)}(d)\tau'_d(\tau'_z(y))\\
=&\sigma'_y(zd)\tau'_{zd}(y)=\sigma'_a(b)\tau'_b(a).
\end{align*}
Suppose that $c\neq 1$. In this case, by the inductive hypothesis, we have 
\begin{align*} ab=& cyb\sim c\sigma'_y(b)\tau'_b(y)\sim \sigma'_c(\sigma'_y(b))\tau'_{\sigma'_{y}(b)}(c)\tau'_{b}(y)\\
=&\sigma'_{cy}(b)\tau'_{b}(cy)=\sigma'_a(b)\tau'_b(a).
\end{align*}
Hence, by induction, (\ref{eq:LYZ5}) follows. Therefore $r_G$ is a braiding operator on $G(X,r)$.

Note that $r_G(i\times i)(x,y)=r_G(\bar x,\bar y)=(\overline{\sigma_x(y)},\overline{\tau_y(x)})=(i\times i)r(x,y)$ for all $x,y\in X$. Let $(H,s)$ be a braided group and let $j\colon X\rightarrow H$ be a map such that $s(j\times j)=(j\times j)r$.
We write $s(\bar a,\bar b)=(\bar a\rhd_s \bar b, \bar a\lhd_s \bar b)$ for all $a,b\in FM(Y)$. Since 
\[ j(x)j(y)=(j(x)\rhd_s j(y))(j(x)\lhd_s j(y))=j(\sigma_x(y))j(\tau_y(x))\]
for all $x,y\in Y$, there exists a unique group homomorphism $f\colon G(X,r)\rightarrow H$ such that $fi=j$. We shall prove that
\[ s(f\times f)(\bar a,\bar b)=(f\times f)r_G(\bar a, \bar b)\]
for all $a,b\in FM(Y)$ by induction on the length of $ab$.
We know that
\[ s(f\times f)(\bar a,1)=s(f(\bar a),1)=(1,f(\bar a))=(f\times f)(1,\bar a)=(f\times f)r_G(\bar a ,1)\]
and
\[ s(f\times f)(1,\bar a)=s(1,f(\bar a))=(f(\bar a),1)=(f\times f)(\bar a,1)=(f\times f)r_G(1,\bar a)\]
for all $a\in FM(Y)$. Note that
\[s(f\times f)(\bar x,\bar y)=s(j(x),j(y))=(j(\sigma_x(y),\tau_y(x))=(f\times f)r_G(\bar x,\bar y)\]
for all $x,y\in X$.
Since 
\begin{align*}
\left(f(\bar y),f\left(\overline{\widehat{\sigma}^{-1}_y(x)}\right)\right)=&(f\times f)(i\times i)r( x,\sigma^{-1}_x(y))\\
=&(j\times j)r( x,\sigma^{-1}_x(y))=s(j\times j)( x,\sigma^{-1}_x(y))\\
=&(j(x)\rhd_s j(\sigma^{-1}_x(y)),j(x)\lhd_s j(\sigma^{-1}_x(y))),
\end{align*}
we have that
\[ f(\bar x)\rhd_s f\left(\overline{\sigma^{-1}_x(y)}\right)=j(x)\rhd_s j(\sigma^{-1}_x(y))=f(\bar y),\]
and thus 
\[f\left(\overline{x'}\right)\rhd_s f(\bar y)=f(\bar x)^{-1}f(\bar y)=f\left(\overline{\sigma^{-1}_x(y)}\right)=f\left(\overline{\sigma'_{x'}(y)}\right)\]
for all $x,y\in X$. Furthermore
\begin{align*}
    f\left(\overline{\sigma'_{x'}(y)}\right)\left(f\left(\overline{x'}\right)\lhd_s f(\bar y)\right)=&
    \left( f\left(\overline{x'}\right)\rhd_s f(\bar y)\right)\left(f\left(\overline{x'}\right)\lhd_s f(\bar y)\right)\\
    =&f\left(\overline{x'}\right)f(\bar y)=f\left(\overline{x'y}\right)\\
    =&f\left(\overline{\sigma'_{x'}(y)\tau'_y(x')}\right)\\
    =&f\left(\overline{\sigma'_{x'}(y)}\right)f\left(\overline{\tau'_y(x')}\right).
\end{align*}
Hence
\[f\left(\overline{x'}\right)\lhd_s f(\bar y)=f\left(\overline{\tau'_y(x')}\right),\]
and thus
\[s(f\times f)\left(\overline{x'},\bar y\right)=(f\times f)r_G\left(\overline{x'},\bar y\right)\]
for all $x,y\in X$. Similarly one can check that
\[s(f\times f)\left(\bar x,\overline{y'}\right)=(f\times f)r_G\left(\bar x,\overline{y'}\right)\]
for all $x,y\in X$. 


Since 
\begin{align*}
\left(f\left(\overline{y'}\right),f\left(\overline{\tau'_{(\sigma')^{-1}_x(y')}(x)}\right)\right)=&(f\times f)r_G\left( \bar x,\overline{(\sigma')^{-1}_x(y')}\right)\\
=&s(f\times f)\left(\bar x,\overline{(\sigma')^{-1}_x(y')}\right)\\
=&\left(f(\bar x)\rhd_s f\left(\overline{(\sigma')^{-1}_x(y')}\right),f(\bar x)\lhd_s f\left(\overline{(\sigma')^{-1}_x(y')}\right)\right),
\end{align*}
we have that
\[ f(\bar x)\rhd_s f\left(\overline{(\sigma')^{-1}_x(y')}\right)
=f\left(\overline{y'}\right),\]
and thus 
\[f\left(\overline{x'}\right)\rhd_s f\left(\overline{y'}\right)=f(\bar x)^{-1}f\left(\overline{y'}\right)=f\left(\overline{(\sigma')^{-1}_x(y')}\right)=f\left(\overline{\sigma'_{x'}(y')}\right)\]
for all $x,y\in X$. Furthermore
\begin{align*}
    f\left(\overline{\sigma'_{x'}(y')}\right)\left(f\left(\overline{x'}\right)\lhd_s f\left(\overline{y'}\right)\right)=&
    \left( f\left(\overline{x'}\right)\rhd_s f\left(\overline{y'}\right)\right)\left(f\left(\overline{x'}\right)\lhd_s f\left(\overline{y'}\right)\right)\\
    =&f\left(\overline{x'}\right)f\left(\overline{y'}\right)=f\left(\overline{x'y'}\right)\\
    =&f\left(\overline{\sigma'_{x'}(y')\tau'_{y'}(x')}\right)\\
    =&f\left(\overline{\sigma'_{x'}(y')}\right)f\left(\overline{\tau'_{y'}(x')}\right).
\end{align*}
Hence
\[f\left(\overline{x'}\right)\lhd_s f\left(\overline{y'} \right)=f\left(\overline{\tau'_{y'}(x')}\right),\]
and thus
\[s(f\times f)\left(\overline{x'},\overline{y'}\right)=(f\times f)r_G\left(\overline{x'},\overline{y'}\right)\]
for all $x,y\in X$.
Hence 
\[s(f\times f)\left(\bar a,\bar b\right)=(f\times f)r_G\left(\bar a,\bar b\right)\]
for all $a,b\in FM(Y)$ such that the length of $ab$ is $\leq 2$. Let $a,b\in FM(Y)$ be elements such that the length of $ab$ is $n>2$ and suppose that
\[s(f\times f)\left(\bar c,\bar d\right)=(f\times f)r_G\left(\bar c,\bar d\right)\]
for all $c,d\in FM(Y)$ such that the length of $cd$ is $<n$. We may assume that $a\neq 1$ and $b\neq 1$. Hence there exist $y,z\in Y$ and $c,d\in FM(Y)$ such that $a=cy$ and $b=zd$. Suppose that $c\neq 1$. In this case, by the inductive hypothesis,
\begin{align*}
    s(f\times f)(\bar a,\bar b)=&s(f(\bar c)f(\bar y),f(\bar b))\\
    =&(\id\times m)s_1s_2(f(\bar c), f(\bar y), f(\bar b))\\
    =&(\id \times m)(f\times f\times f)(r_G)_1(r_G)_2(\bar c,\bar y,\bar b)\\
    =&(f\times f)(\id\times m)(r_G)_1(r_G)_2(\bar c,\bar y,\bar b)\\
    =&(f\times f)r_G(\bar c\bar y,\bar b)=(f\times f)r_G(\bar a,\bar b).
\end{align*}
Suppose that $c=1$. In this case, by the inductive hypothesis,
\begin{align*}
    s(f\times f)(\bar a,\bar b)=&s(f(\bar a),f(\bar z)f(\bar d))\\
    =&(m\times \id)s_2s_1(f(\bar a), f(\bar z), f(\bar d))\\
    =&(m \times \id)(f\times f\times f)(r_G)_2(r_G)_1(\bar a,\bar z,\bar d)\\
    =&(f\times f)(m\times \id)(r_G)_2(r_G)_1(\bar a,\bar z,\bar d)\\
    =&(f\times f)r_G(\bar a,\bar z\bar d)=(f\times f)r_G(\bar a,\bar b).
\end{align*}
Hence, by induction, we have that
\[s(f\times f)=(f\times f)r_G.\]

The uniqueness of $r_G$ follows by the above argument taking $H=G(X,r)$ and $j=i$, because in this case $f=\id$, and thus $s=r_G$.
\end{proof}

\begin{theorem}\label{thm:GVbraces}
Let $(X,r)$ be a solution to the YBE. Let $i\colon X\rightarrow G(X,r)$ be the natural map. Then there exists a unique structure of skew brace on $G(X,r)$ with multiplicative group the structure group $G(X,r)$ such that $\lambda_{i(x)}(i(y))=i(\sigma_x(y))$ for all $x,y\in X$. Furthermore, if $(B,+,\circ)$ is a skew brace and $j\colon X\rightarrow B$ is a map such that $\lambda_{j(x)}(j(y))=j(\sigma_x(y))$ and $j(x)j(y)=j(\sigma_x(y))j(\tau_y(x))$ for all $x,y\in X$, then there exists a unique homomorphism of skew braces $f\colon G(X,r)\rightarrow B$ such that $fi=j$.  
\end{theorem}

\begin{proof}
By Theorem \ref{thm:LYZ9}, there exists a unique braiding operator $r_G$ on $G(X,r)$ such that $r_G(i\times i)=(i\times i)r$. Consider the structure of skew brace on $G(X,r)$ associated to the braided group $(G(X,r),r_G)$ by the bijective correspondence described in the proof of Theorem \ref{thm:braidedbraces}. Thus the addition on $G(X,r)$ is defined by $\bar a+\bar b=\bar a\overline{(\sigma')^{-1}_a(b)}$ for all $a,b\in FM(Y)$. Hence
\begin{align*}\lambda_{i(x)}(i(y))=&-i(x)+i(x)i(y)=i(x)(i(x)^{-1}+i(y))\\
=&i(x)i(x)^{-1}\overline{(\sigma')^{-1}_{x'}(y)}\\
=&\overline{\sigma_x(y)}=i(\sigma_x(y))
\end{align*}
for all $x,y\in X$.

Let $(B,+,\circ)$ be a skew brace and $j\colon X\rightarrow B$ is a map such that $\lambda_{j(x)}(j(y))=j(\sigma_x(y))$ and $j(x)j(y)=j(\sigma_x(y))j(\tau_y(x))$ for all $x,y\in X$. By the definition of $G(X,r)$ there exists a unique group homomorphism $f\colon G(X,r)\rightarrow B$ such that $fi=j$. Let $r_B\colon B\times B\rightarrow B\times B$ be the map defined by $r_B(u,v)=(\lambda_u(v),\lambda_u(v)^{-1}uv)$. By Theorem \ref{thm:YB}, $(B,r_B)$ is a solution to the YBE. Furthermore, it is easy to check that $r_B$ is a braiding operator on the multiplicative group of $B$. Note that
\begin{align*} r_B(j\times j)(x,y)=&(\lambda_{j(x)}j(y),\lambda_{j(x)}j(y)^{-1}j(x)j(y))\\
=&(j(\sigma_x(y)),j(\tau_y(x)))=(j\times j)r(x,y)
\end{align*}
for all $x,y\in X$. By Theorem \ref{thm:LYZ9}, $r_B(f\times f)=(f\times f)r_G$. In particular, $f\left(\lambda_{\bar a}(\bar b)\right)=\lambda_{f(\bar a)}(f(\bar b))$ for all $a,b\in FM(Y)$. Hence
\begin{align*}f(\bar a+\bar b)=& f\left(\bar a\overline{(\sigma')^{-1}_a(b)}\right)\\
=&f(\bar a)f\left(\overline{(\sigma')^{-1}_a(b)}\right)\\
=&f(\bar a)f\left(\lambda_{(\bar a)^{-1}}(\bar b)\right)\\
=&f(\bar a)\lambda_{f(\bar a)^{-1}}(f(\bar b))=f(\bar a)+f(\bar b)
\end{align*}
for all $a,b\in FM(Y)$. Hence $f$ is a homomorphism of skew braces.
\end{proof}

\begin{definition}
\index{structure skew brace of a solution}
Given a solution $(X,r)$ to the YBE, its structure skew brace is the skew brace structure on the group $G(X,r)$ given in Theorem \ref{thm:GVbraces}.
\end{definition}

\begin{theorem}\label{thm:involstruct}
Let $(X,r)$ be an involutive solution of the YBE. Then the additive group of the structure skew brace $G(X,r)$ is the additive free abelian group on $X$.
\end{theorem}

\begin{proof}
The structure group $G(X,r)$ is
\[ G(X,r)=\gr(X : xy=\sigma_x(y)\tau_y(x)\text{ for all }x,y\in X ).\]
Since $xy=x+\lambda_x(y)=x+\sigma_x(y)$ and $(X,r)$ is involutive, we have that $\sigma_x(y)\tau_y(x)=\sigma_x(y)+\lambda_{\sigma_x(y)}(\tau_y(x)=\sigma_x(y)+\sigma_{\sigma_x(y)}(\tau_y(x))=\sigma_x(y)+x$,
for all $x,y\in X$. Hence the additive group of $G(X,r)$ is generated by $X$ and it is abelian.

Let $\Z^{(X)}$ be the additive free abelian group with basis $X$. Thus the elements of $\Z^{(X)}$ are finite sums of the form
\[ z_1x_1+\dots +z_nx_n,\]
with $z_1,\dots ,z_n\in\Z$ and $x_1,\dots ,x_n\in X$. Furthermore, if 
\[ z_1x_1+\dots +z_nx_n=0\]
and $x_1,\dots ,x_n\in X$ are $n$ distinct elements, then $z_1=\dots =z_n=0$. Consider the semidirect product $\Z^{(X)}\rtimes \Sym_X$, with respect to the natural action of $\Sym_X$ on $\Z^{(X)}$ given by the rule
\[ 
\sigma(z_1x_1+\dots +z_nx_n)=z_1\sigma(x_1)+\dots +z_n\sigma(x_n)
\]
for all $\sigma\in\Sym_X$, $z_1,\dots ,z_n\in\Z$ 
and $x_1,\dots,x_n\in X$. Let $i\colon X\to G(X,r)$ be the natural map.
We define a map $j\colon X\to \Z^{(X)}\rtimes \Sym_X$ by $j(x)=(x,\sigma_x)$ for all $x\in X$. Note that, since $(X,r)$ is involutive, 
\begin{align*}
    j(x)j(y)=&(x,\sigma_x)(y,\sigma_y)=(x+\sigma_x(y),\sigma_x\sigma_y)\\
    =&(\sigma_x(y)+x,\sigma_{\sigma_x(y)}\sigma_{\tau_y(x)})\\
    =&(\sigma_x(y)+\sigma_{\sigma_x(y)}(\tau_y(x)),\sigma_{\sigma_x(y)}\sigma_{\tau_y(x)})\\
    =&(\sigma_x(y),\sigma_{\sigma_x(y)})(\tau_y(x),\sigma_{\tau_y(x)})\\
    =&j(\sigma_x(y))j(\tau_y(x))
\end{align*}
for all $x,y\in X$. Hence there exists a unique homomorphism of groups  
$f\colon G(X,r)\to \Z^{(X)}\rtimes \Sym_X$, from the structure group of $(X,r)$ (the multiplicative group of $G(X,r)$) to the semidirect product $\Z^{(X)}\rtimes \Sym_X$, such that $f(i(x))=j(x)=(x,\sigma_x)$ for all $x\in X$. In particular, the map $i$ is injective and thus we identify $x$ and $i(x)$. Note that with this identification, for every $a\in G(X,r)$, the restriction of $\lambda_a$ to $X$ is in the subgroup $\langle \sigma_x :x\in X\rangle$ of $\Sym_X$. We denote by $\sigma_a$ the restriction of $\lambda_a$ to $X$ for all $a\in G(X,r)$.  Since
 $f(x^{-1})=(x,\sigma_x)^{-1}=(-\sigma_x^{-1}(x),\sigma_x^{-1})$ and
 $\sigma_x^{-1}=\sigma_{x^{-1}}=\sigma_{-\sigma_{x^{-1}}(x)}$, we have that $f(x^{-1})=(-\sigma^{-1}_x(x),\sigma_{-\sigma^{-1}_x(x)})$. We shall prove that
 \begin{equation}\label{eq:keyinv}
     f(\epsilon_1x_1+\dots +\epsilon_nx_n)=(\epsilon_1x_1+\dots +\epsilon_nx_n,\sigma_{\epsilon_1x_1+\dots +\epsilon_nx_n})
 \end{equation}
for all $x_1,\dots ,x_n\in X$ and $\epsilon_1,\dots ,\epsilon_n\in\{ -1,1\}$, by induction on $n$. Note that in $G(X,r)$ we have that
\[ (-x)^{-1}=-\lambda_{(-x)^{-1}}(-x)=\lambda^{-1}_{(-x)}(x)\in X \]
for all $x\in X$.
Hence 
 \begin{align*}
 f(-x)=&f(\lambda^{-1}_{(-x)}(x))^{-1}=(-\sigma_{\lambda^{-1}_{(-x)}(x)}^{-1}(\lambda^{-1}_{(-x)}(x)),\sigma_{-\sigma_{\lambda^{-1}_{(-x)}(x)}^{-1}(\lambda^{-1}_{(-x)}(x))})\\
 =&(-\sigma_{(-x)^{-1}}^{-1}(\lambda^{-1}_{(-x)}(x)),\sigma_{-\sigma_{(-x)^{-1}}^{-1}(\lambda^{-1}_{(-x)}(x))})\\
 =&(-\lambda_{-x}(\lambda^{-1}_{(-x)}(x)),\sigma_{-\lambda_{-x}(\lambda^{-1}_{(-x)}(x))})\\
 =&(-x,\sigma_{-x})
\end{align*} 
for all $x\in X$. Hence (\ref{eq:keyinv}) holds for $n=1$. Suppose that (\ref{eq:keyinv}) holds for some $n\geq 1$. By the inductive hypothesis, we have that
\begin{align*}
    f(&\epsilon_1x_1+\dots +\epsilon_{n+1}x_{n+1})=f((\epsilon_1x_1+\dots +\epsilon_nx_n)\lambda^{-1}_{\epsilon_1x_1+\dots +\epsilon_nx_n}(\epsilon_{n+1}x_{n+1}))\\
    =&f(\epsilon_1x_1+\dots +\epsilon_nx_n)f(\lambda^{-1}_{\epsilon_1x_1+\dots +\epsilon_nx_n}(\epsilon_{n+1}x_{n+1}))\\
    =&(\epsilon_1x_1+\dots +\epsilon_nx_n,\sigma_{\epsilon_1x_1+\dots +\epsilon_nx_n})(\lambda^{-1}_{\epsilon_1x_1+\dots +\epsilon_nx_n}(\epsilon_{n+1}x_{n+1}),\sigma_{\lambda^{-1}_{\epsilon_1x_1+\dots +\epsilon_nx_n}(\epsilon_{n+1}x_{n+1})})\\
    =&(\epsilon_1x_1+\dots +\epsilon_nx_n,\sigma_{\epsilon_1x_1+\dots +\epsilon_nx_n})(\epsilon_{n+1}\sigma^{-1}_{\epsilon_1x_1+\dots +\epsilon_nx_n}(x_{n+1}),\sigma_{\lambda^{-1}_{\epsilon_1x_1+\dots +\epsilon_nx_n}(\epsilon_{n+1}x_{n+1})})\\
    =&(\epsilon_1x_1+\dots +\epsilon_nx_n+\epsilon_{n+1}x_{n+1},\sigma_{\epsilon_1x_1+\dots +\epsilon_nx_n}\sigma_{\lambda^{-1}_{\epsilon_1x_1+\dots +\epsilon_nx_n}(\epsilon_{n+1}x_{n+1})})\\
    =&(\epsilon_1x_1+\dots +\epsilon_nx_n+\epsilon_{n+1}x_{n+1},\sigma_{(\epsilon_1x_1+\dots +\epsilon_nx_n)\lambda^{-1}_{\epsilon_1x_1+\dots +\epsilon_nx_n}(\epsilon_{n+1}x_{n+1})})\\
    =&(\epsilon_1x_1+\dots +\epsilon_nx_n+\epsilon_{n+1}x_{n+1},\sigma_{\epsilon_1x_1+\dots +\epsilon_nx_n+\epsilon_{n+1}x_{n+1}}).
\end{align*}
Hence (\ref{eq:keyinv}) holds for all $n$ by induction. Therefore
 one can easily see that $\pi_1f\colon G(X,r)\to\Z^{(X)}$, where $\pi_1\colon \Z^{(X)}\rtimes\Sym_X\to\Z^{(X)}$ is the natural map, is an isomorphism from the additive group of $G(X,r)$ to $\Z^{(X)}$, \textcolor{red}{in fact the unique homomorphism of groups $g\colon \Z^{(X)}\to G(X,r)$, such that $g(x)=x$ for all $x\in X$, is the inverse of $\pi_1 f$. Therefore } the result follows. 
\end{proof}

\section{The permutation group of a solution}

Let $(X,r)$ be a solution to the YBE. We write 
\[r(x,y)=(\sigma_x(y),\tau_y(x))\] 
for all $x,y\in X$. Consider the structure group $G(X,r)$ of the solution $(X,r)$. Let $i\colon X\to G(X,r)$ be the natural map. 

\index{Solution!permutation group}
The {\em permutation group} of $(X,r)$ is the subgroup
\[\mathcal{G}(X,r)=\langle (\sigma_x,\tau^{-1}_x) : x\in X\rangle\subseteq\Sym_X\times\Sym_X.\]
Since 
\[ \sigma_x\sigma_y=\sigma_{\sigma_x(y)}\sigma_{\tau_y(x)}\quad\text{and}\quad \tau^{-1}_x\tau^{-1}_y=\tau^{-1}_{\sigma_x(y)}\tau^{-1}_{\tau_y(x)} \]
for all $x,y\in X$, there exists a unique group homomorphism
$h\colon G(X,r)\to \mathcal{G}(x,r)$ such that $hi(x)=(\sigma_x,\tau^{-1}_x)$ for all $x\in X$. We write
\[ h(a)=(\sigma_a,\tau_a^{-1})\]
for all $a\in G(X,r)$.
By Theorem \ref{thm:GVbraces}, $G(X,r)$ has a unique structure of skew brace with multiplicative group the structure group $G(X,r)$ and $\lambda_{i(x)}(i(y))=i(\sigma_x(y))$ for all $x\in X$. We shall see that $\ker h$ is an ideal of the skew brace $G(X,r)$. Note that
\[\mu_{i(y)}(i(x))=\lambda_{i(x)}(i(y))^{-1}i(x)i(y)=i(\sigma_x(y))^{-1}i(x)i(y)=i(\tau_y(x))\]
for all $x,y\in X$, by the defining relations of $G(X,r)$.

\begin{lemma}\label{lem:Kerh}
With the above notation,
\[ \ker h\subseteq \ker\lambda\cap\ker\mu=\Soc(G(X,r)).\]
\end{lemma}

\begin{proof}
Let $a\in \ker h$. There exists $x_1,\dots ,x_n\in X$ and $\epsilon_1,\dots ,\epsilon_n\in\{ -1,1\}$ such that
$a=i(x_1)^{\epsilon_1}\cdots i(x_n)^{\epsilon_n}$. Hence
\[(\id_X,\id_X)=h(a)=(\sigma_{x_1}^{\epsilon_1}\cdots\sigma_{x_n}^{\epsilon_n},\tau_{x_1}^{-\epsilon_1}\cdots \tau_{x_n}^{-\epsilon_n}).\]
Therefore
\[ \lambda_a(i(z))=i(\sigma_{x_1}^{\epsilon_1}\cdots\sigma_{x_n}^{\epsilon_n}(z))=i(z)\]
and
\[ \mu^{-1}_a(i(z))=i(\tau_{x_1}^{-\epsilon_1}\cdots \tau_{x_n}^{-\epsilon_n}(z))=i(z)\]
for all $z\in X$. Since $\lambda_a$ is an automorphism of the additive group of the skew brace $G(X,r)$ and this additive group is generated by $i(X)$, we have that $\lambda_a=\id$. On the other hand, we know that $r_G: G(X,r)\times G(X,r)\to G(X,r)\times G(X,r)$, defined by $r_G(b,c)=(\lambda_b(c),\mu_c(b))$ for all $b,c\in G(X,r)$, is a braiding operator on the structure group $G(X,r)$. Hence $\mu_a(1)=1$ and
\begin{align*}
    1=&\mu_a(i(x)^{-1}i(x))=\mu_{\lambda_{i(x)}(a)}(i(x)^{-1})\mu_a(i(x))\\
    =&\mu_{i(x)a\mu_a(i(x))^{-1}}(i(x)^{-1})i(x)\\
    =&\mu_{i(x)ai(x)^{-1}}(i(x)^{-1})i(x)
    \end{align*}
 for all $x\in X$. Hence $\mu_a(i(x)^{-1})=i(x)^{-1}$ for all $x\in X$ and $a\in \ker h$. Let $b\in G(X,r)$. There exist $y_1,\dots ,y_m\in X$ and $\eta_1, \dots ,\eta_m\in\{ -1,1\}$ such that $b=i(y_1)^{\eta_1}\cdots i(y_m)^{\eta_m}$. We shall prove that $\mu_a(b)=b$ for all $a\in\ker h$, by induction on $m$.
 We have proved that this holds for $m=1$. Suppose that $m>1$ and that
 \[\mu_a(i(y_1)^{\eta_1}\cdots i(y_{m-1})^{\eta_{m-1}})=i(y_1)^{\eta_1}\cdots i(y_{m-1})^{\eta_{m-1}}\]
 for all $a\in\ker h$, $y_1,\dots ,y_{m-1}\in X$ and $\eta_1, \dots ,\eta_{m-1}\in\{ -1,1\}$. 
 By the inductive hypothesis, we have that
 \begin{align*}
     \mu_a(b)=& \mu_{\lambda_{i(y_m)^{\eta_m}}(a)}((y_1)^{\eta_1}\cdots i(y_{m-1})^{\eta_{m-1}})\mu_a(i(y_m)^{\eta_m})\\
     =& \mu_{i(y_m)^{\eta_m}a(\mu_a(i(y_m)^{\eta_m})^{-1}}((y_1)^{\eta_1}\cdots i(y_{m-1})^{\eta_{m-1}})i(y_m)^{\eta_m}\\
     =& \mu_{i(y_m)^{\eta_m}ai(y_m)^{-\eta_m}}((y_1)^{\eta_1}\cdots i(y_{m-1})^{\eta_{m-1}})i(y_m)^{\eta_m}\\
     =& (y_1)^{\eta_1}\cdots i(y_{m-1})^{\eta_{m-1}}i(y_m)^{\eta_m}=b.
 \end{align*}
Hence, by induction, $\mu_a=\id$ for all $a\in\ker h$. Thus 
\[\ker h\subseteq \ker\lambda\cap\ker\mu=\Soc(G(X,r)),\]
by Propositon \ref{pro:soc_kernels}.
\end{proof}

\begin{lemma}\label{lem:kerhideal}
With the above notation, $\ker h$ is an ideal of the skew brace $G(X,r)$.
\end{lemma}

\begin{proof}
By Lemma \ref{lem:Kerh}, $\ker h\subseteq \Soc(G(X,r))$. Hence $\ker h\subseteq Z(G(X,r),+)$ and thus $\ker h$ is a normal subgroup of the additive group of the skew brace $G(X,r)$. It is clear that $\ker h$ also is a normal subgroup of the multiplicative group of $G(X,r)$. Let $a\in\ker h$ and $b,c\in G(X,r)$. By Lemma \ref{lem:Kerh}, $\ker h\subseteq\ker\mu$. Hence we have that
\[\lambda_b(a)=ba(\mu_a(b))^{-1}=bab^{-1}\in\ker h.\]
Therefore $\ker h$ is an ideal of the skew brace $G(X,r)$.
\end{proof}

Let $\tilde h \colon G(X,r)/\ker h\to \mathcal{G}(X,r)$ the isomorphism of groups defined by 
\[\tilde h(a\ker h)=h(a)\] 
for all $a\in G(X,r)$. We define an addition on $\mathcal{G}(X,r)$ by 
\[ g_1+g_2=\tilde h(\tilde h^{-1}(g_1)+\tilde h^{-1}(g_2))\]
for all $g_1,g_2\in\mathcal{G}(X,r)$, i.e.
\[ h(a)+h(b)=h(a+b)\]
for all $a,b\in G(X,r)$. Hence $\mathcal{G}(X,r)$ with this addition is a skew brace and then $h$ and $\tilde h$ are skew brace homomorphisms.

In particular, the lambda map of $\mathcal{G}(X,r)$ satisfies that  
\[ \lambda_{(\sigma_a,\tau^{-1}_a)}(\sigma_b,\tau^{-1}_b)=\lambda_{h(a)}h(b)=h(\lambda_{a}(b))=(\sigma_{\lambda_a(b)},\tau^{-1}_{\lambda_a(b)})\]
for all $a,b\in G(X,r)$.



\section{Structure skew brace of a finite solution}

In this section we study the structure skew brace of a finite solution to the YBE.

\begin{theorem}\label{thm:abelianbyfinite}
Let $(X,r)$ be a finite solution to the YBE. Then $\Soc(G(X,r))$ is a normal subgroup of finite index of the structure group $G(X,r)$. In particular, the structure group $G(X,r)$ is abelian-by-finite. 
\end{theorem} 

\begin{proof}
We write $r(x,y)=(\sigma_x(y),\tau_y(x))$ for all $x,y\in X$.
Let $i \colon X\to G(X,r)$ be the natural map. We have seen that there is a unique group homomorphism $h \colon G(X,r)\to \mathcal{G}(X,r)\subseteq \Sym_X\times \Sym_X$ such that $hi(x)=(\sigma_x,\tau^{-1}_x)$ for all $x\in X$. By Lemma \ref{lem:Kerh}, $\ker h\subseteq \Soc(X,r)$. Since $\mathcal{G}(X,r)$ is finite, the result follows. 
\end{proof}


We now use Hall's theorem to obtain information related to the
structure of finite skew braces of nilpotent type. 


\begin{theorem}
\label{thm:add_nilpotent}
Let $A$ be a finite skew brace of nilpotent type. Then 
the multiplicative group of $A$ is solvable.
\end{theorem}

\begin{proof}
    Let $K$ be the additive group of $A$ and $G$ be the multiplicative group of $A$. Assume
    that $|A|=p_1^{\alpha_1}\cdots p_n^{\alpha_n}$ for distinct primes $p_1,\dots,p_n$ and positive integers $\alpha_1,\dots ,\alpha_n$. 
    Since $K$ is nilpotent and finite, each $K_i\in\Syl_{p_j}(K)$ is normal in $K$, so 
    each $K_i$ is a left ideal of $A$. It follows that for each $i\in\{1,\dots,n\}$ both $K_i$ and 
    $\prod_{j\ne i}K_j$ are subbraces of coprime order. In particular, for 
    each $i\in\{1,\dots,n\}$ there exists a $p_i$-complement of $G$. 
    Then $G$ is solvable by Hall's theorem. 
\end{proof}

\begin{theorem}\label{thm:ESSsolvable}
Let $(X,r)$ be a finite involutive solution to the YBE. Then the structure group $G(X,r)$ is solvable.
\end{theorem}

\begin{proof}
By Theorem \ref{thm:involstruct}, $G(X,r)$ is a skew brace of abelian type. By Theorem \ref{thm:abelianbyfinite}, $\Soc(G(X,r))$ is a normal abelian subgroup of the structure group $G(X,r)$ of finite index. Hence $G(X,r)/\Soc(G(X,r))$ is a finite skew brace of abelian type. Therefore the result follows by Theorem \ref{thm:add_nilpotent}.  
\end{proof}

\section{Exercises}

\begin{prob}
Check the missing steps of the proofs of all the results of this chapter. 
\end{prob}

\section{Notes}