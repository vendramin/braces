\chapter{The structure brace of a solution}
\label{structure_brace}

% hacerlo con el guitar map

The prove that the structure group $G(X,r)$ of a solution $(X,r)$ is a brace, we follow the 
proof of Lu, Yan and Zhu. They use the language of braided groups.

\begin{definition}
\index{Braided group}
A \emph{braided group} is a pair $(G,r)$, where 
$G$ is a group with operation $m\colon G\times G\to G$, $m(x,y)=xy$, and 
$r\colon G\times G\to G\times G$ is a bijective map such that
\begin{enumerate}
\item $r(xy,z)=(\id\times m)r_1r_2(x,y,z)$ for all $x,y,z\in G$,
\item $r(x,yz)=(m\times\id)r_2r_1(x,y,z)$ for all $x,y,z\in G$,
\item $r(1,x)=(x,1)$ and $r(x,1)=(1,x)$ for all $x\in G$, and 
\item $m\circ r=m$.
\end{enumerate}
The map $r$ is called a \emph{braiding operator} on $G$. 
\end{definition}


\begin{proof}

\end{proof}

\begin{theorem}
    Let $G$ be a group. Then $G$ admits a braiding operator if and only if 
    there is a brace structure on the set $G$ with multiplicative group $G$. 
\end{theorem}

\begin{proof}
\end{proof}

\section*{Exercises}

\begin{prob}
Prove that a braiding operator is a solution. 
\end{prob}

