\chapter{Involutive solutions}
\label{Isolutions}

\section*{A}

Let $(X,r)$ be an involutive solution to the YBE. We know that its permutation group $\mathcal{G}(X,r)$ has a natural structure of brace of abelian type. If $(Y,r')$ is a solution isomorphic to $(X,r)$, then $\mathcal{G}(Y,r')\cong\mathcal{G}(X,r)$ as braces. Thus the classification of the involutive solutions can be done in two steps:
\begin{enumerate}
	\item Classify all the braces of abelian type.
	\item For each brace $B$ of abelian type, classify all the involutive solutions $(X,r)$ such that $\mathcal{G}(X,r)\cong B$.
\end{enumerate}

In this chapter we focus on the second step. To this end we will use the next result about extensions of braces.

\begin{proposition}(\cite[Theorem~2.1]{MR3320237})\label{extensions}  Let $H$ be an abelian group and $B$
	be a brace of abelian type. Suppose  that $\sigma\colon
	(B,\circ)\longrightarrow \Aut(H,+)$ is an injective morphism and
	$\eta\colon (H,+)\longrightarrow (B,+)$ is a surjective morphism
	such that they satisfy $\eta(\sigma(g)(h))=\lambda_g(\eta(h))$ for
	all $g\in B$ and $h\in H$. Then, the multiplication in $H$ defined
	by
	$$h_1\cdot h_2:=h_1+\sigma(\eta (h_1))(h_2),$$
	for $h_1,h_2\in H$, defines a structure of brace on $H$
	such that $\eta$ is a morphism of braces  with
	$\Soc(H)=\ker(\eta)$ and $H/\Soc(H)\cong B$ as braces.
	
	Two of these structures, determined by $\sigma$, $\eta$ and
	$\sigma'$, $\eta'$ respectively, are isomorphic if and only if there
	exists an $F\in \Aut(H,+)$ such that
	$$
	\sigma'(\eta'(h))=F^{-1}\circ\sigma(\eta(F(h)))\circ F,
	$$
	for all $h\in H$.
	
	Conversely, suppose that $G$ is a brace. Then, the map
	$\sigma\colon (G/\Soc(G),\circ)\longrightarrow\Aut(G,+)$, induced by
	the map $\lambda\colon (G,\circ)\longrightarrow\Aut(G,+)$, and the
	natural map $\eta\colon G\longrightarrow G/\Soc(G)$ satisfy the
	above properties. 
\end{proposition}

\begin{proof}
We shall see that $(H,\cdot)$ is a group. Let $a,b,c\in H$. We have
	\begin{eqnarray*}
		a\cdot (b\cdot c)&=&a\cdot (b+\sigma(\eta (b))(c))=a+\sigma(\eta (a))(b+\sigma(\eta (b))(c))\\
		&=&a+\sigma(\eta (a))(b)+\sigma(\eta (a))(\sigma(\eta (b))(c))\\
		&=&a+\sigma(\eta(a))(b)+\sigma(\eta(a)\circ \eta(b))(c)\\
		&=&a+\sigma(\eta(a))(b)+\sigma(\eta(a)+\lambda_{\eta(a)}(\eta(b)))(c)\\
		&=&a+\sigma(\eta(a))(b)+\sigma(\eta(a)+\eta(\sigma(\eta(a))(b)))(c)\\
		&=&(a+\sigma(\eta(a))(b))+\sigma(\eta(a+\sigma(\eta(a))(b)))(c)\\
		&=&(a\cdot b)\cdot c.
	\end{eqnarray*}	
Note that, $0\cdot a=0+\sigma(\eta(0))(a)=\sigma(0)(a)=\id(a)=a$ and $a\cdot 0=a+\sigma(\eta(a))(0)=a+0=a$. Furthermore,
$a\cdot (-\sigma(\eta(a)')(a))=a+\sigma(\eta(a))(-\sigma(\eta(a)')(a))=a-\sigma(0)(a)=a-a=0$ and
\begin{eqnarray*}
	(-\sigma(\eta(a)')(a))\cdot a&=&-\sigma(\eta(a)')(a)+\sigma(\eta(-\sigma(\eta(a)')(a)))(a)\\
	&=&-\sigma(\eta(a)')(a)+\sigma(-\eta(\sigma(\eta(a)')(a)))(a)\\
	&=&-\sigma(\eta(a)')(a)+\sigma(-\lambda_{\eta(a)'}(\eta(a)))(a)\\
	&=&-\sigma(\eta(a)')(a)+\sigma(\eta(a)')(a)=0.
\end{eqnarray*}
Hence $(H,\cdot)$ is a group. Now we have
\begin{eqnarray*}
a\cdot (b+c)&=&a+\sigma(\eta(a))(b+c)=a+\sigma(\eta(a))(b)+\sigma(\eta(a))(c)\\
&=&a+\sigma(\eta(a))(b)-a+a+\sigma(\eta(a))(b+c)=a\cdot b-a+a\cdot c.
\end{eqnarray*}
Thus $(H,+,\cdot)$ is a left brace. Note that $\eta(a\cdot b)=\eta(a+\sigma(\eta(a))(b))=\eta(a)+\eta(\sigma(\eta(a))(b))=\eta(a)+\lambda_{\eta(a)}(\eta(b))=\eta(a)\circ\eta(b)$. Hence $\eta$ is a morphism of braces. Since $\sigma$ is injective, 
$$\Soc(H)=\{a\in H\mid a\cdot b=a+b \forall b\in H\}=\{a\in H\mid \sigma(\eta(a))(b)=b\forall b\in H\}=\{ a\in H\mid \eta(a)=0\}=\ker(eta).$$
Therefore, since $\eta$ is surjective, $H/\Soc(\eta)\cong B$ as braces.

Let $\sigma'\colon
(B,\circ)\longrightarrow \Aut(H,+)$ be an injective morphism and
let $\eta'\colon (H,+)\longrightarrow (B,+)$ be a surjective morphism
such that they satisfy $\eta'(\sigma'(g)(h))=\lambda_g(\eta'(h))$ for
all $g\in B$ and $h\in H$. Define the multiplication $\circ$ in $H$ 
by
$$h_1\circ h_2:=h_1+\sigma'(\eta' (h_1))(h_2),$$
for $h_1,h_2\in H$. Note that $F\in\Aut(H,+)$ is an isomorphism from $(H,+,\circ)$ to $(H,+,\cdot)$ if and only if
$$F(h_1\circ h_2)=F(h_1)\cdot F(h_2),$$
which is equivalent to
$$F(h_1+\sigma'(\eta' (h_1))(h_2))=F(h_1)+\sigma(\eta (F(h_1)))(F(h_2)).$$
Hence $F$ is an isomorphism if and only if 
$$F^{-1}\circ\sigma(\eta(h))\circ F=\sigma'(\eta'(h)),$$
for all $h\in H$. This proves the first part of the result. The second part follows easily.
\end{proof}	



The next result gives a first way to construct
involutive solutions to the YBE. 

\begin{proposition}\label{BenDavid}
	Let $B$ be a brace of abelian type, and let $X$ be a set.  Suppose that
	$\eta\colon \Z^{(X)}\longrightarrow (B,+)$ is a surjective morphism,
	and that $\sigma\colon (B,\cdot)\longrightarrow\Aut(\Z^{(X)})$ is an
	injective morphism such that $\sigma(a)\mid_X$ is a bijection of $X$
	for all $a\in B$, and $\eta(\sigma(a)(m))=\lambda_a(\eta(m))$ for
	all $a\in B$ and $m\in\Z^{(X)}$. Let $r$ be the map
	$$
	\begin{array}{cccc}
		r\colon & X\times X &\longrightarrow & X\times X\\
		& (x,~y) &\mapsto &(f_{x}(y),~f^{-1}_{f_{x}(y)}(x)),
	\end{array}
	$$
	where $f_x(y):=\sigma(\eta(x))(y)$. Then $(X,r)$ is an involutive solution to
	the YBE and $\mathcal{G}(X,r)\cong B$ as braces. Moreover, any
	involutive solution $(Z,t)$ of the YBE with $\mathcal{G}(Z,t)\cong B$ is of
	this form.
\end{proposition}
\begin{proof}
	By Proposition~\ref{extensions}, the abelian group
	$\mathbb{Z}^{(X)}$ with the multiplication defined  by
	$$x\cdot y:=x+\sigma(\eta(x))(y),$$
	for $x,y\in \mathbb{Z}^{X}$, is a brace of abelian type and $\eta$ becomes a
	homomorphism of braces. Note that for $x, y\in X\subseteq
	\mathbb{Z}^{(X)}$, we have that
	$$\lambda_x(y)=xy-x=x+\sigma(\eta(x))(y)-x=\sigma(\eta(x))(y)=f_x(y).$$
	Therefore $(X,r)$ is an involutive solution to the YBE because it is the
	restriction of the solution to the YBE associated with the 
	brace $\mathbb{Z}^{(X)}$ (cf. \cite[Lemma~2]{CJO2}). In fact the
	brace $\mathbb{Z}^{(X)}$ is equal to the brace $G(X,r)$.
	Recall that the addition of the brace
	$$\mathcal{G}(X,r)=\{ \sigma(\eta(m))|_{X}\mid m\in \Z^{(X)}\}$$
	is defined by
	$\sigma(\eta(m_1))|_{X}+\sigma(\eta(m_2))|_{X}=\sigma(\eta(m_1+m_1))|_{X}$,
	for all $m_1,m_2\in \Z^{(X)}$.  Therefore, the map $B\longrightarrow
	\mathcal{G}(X,r)$ defined by $a\mapsto \sigma(a)|_{X}$ is an
	isomorphism of  braces.
	
	On the other hand, observe that, if $(Z,t)$ is a solution to the YBE
	and  $\xi\colon B\longrightarrow \mathcal{G}(Z,t)$ is an isomorphism
	of braces, then the unique homomorphism  $\eta\colon
	(\Z^{(Z)},+)\longrightarrow  (B,+)$ such that
	$\eta(z)=\xi^{-1}(\phi(z))$, for all $z\in Z$, where $\phi\colon
	G(Z,t)\longrightarrow \mathcal{G}(Z,t)$ is the natural projection,
	is surjective, and the map
	$\sigma:(B,\cdot)\longrightarrow\Aut(\Z^{(Z)})$, where $\sigma(a)$
	is the unique automorphism of $\Z^{(Z)}$ such that
	$\sigma(a)(z)=\xi(a)(z)$, for all $z\in Z$, is an injective
	homomorphism. Furthermore, for $a\in B$, there exists $m\in
	\Z^{(Z)}$ such that $a=\eta(m)$ and,   since $\phi$
	is a homomorphism of braces,
	\begin{eqnarray*}
		\eta(\sigma(a)(z))&=&\eta(\xi(a)(z))=\xi^{-1}(\phi(\xi(\eta(m))(z)))\\
		&=&\xi^{-1}(\phi(\phi(m)(z)))=\xi^{-1}(\phi(mz-m))=\xi^{-1}(\phi(m)\phi(z)-\phi(m))\\
		&=&\eta(m)\eta(z)-\eta(m)=a\eta(z)-a\\
		&=&\lambda_a(\eta(z)),
	\end{eqnarray*}
	for all $z\in Z$. Therefore $\eta(\sigma(a)(n))=\lambda_a(\eta(n))$,
	for all $a\in B$ and all $n\in \Z^{(Z)}$. Now we have that
	$$\sigma(\eta(x))(y)=\xi(\eta(x))(y)=\phi(x)(y),$$
	for all $x,y\in Z$. Hence $(Z,t)$ is exactly the same solution to
	the YBE  as the solution obtained by the given construction using
	the maps $\eta$ and $\sigma$.
\end{proof}

This proposition  reduces the problem of finding all the involutive solutions
to the YBE to the problem of finding   maps $\eta$ and $\sigma$ with
the required properties. Note that the map $\eta$ is determined by
its restriction to $X$ and, since $\eta$ is surjective, $\eta(X)$
should generate $(B,+)$. Note also that the monomorphism $\sigma$
factors trough the symmetric group $\Sym_X$, that is
$\sigma=\sigma_1\sigma_2$ where $\sigma_2\colon
(B,\cdot)\longrightarrow \Sym_X$ is a monomorphism and
$\sigma_1\colon \Sym_X\longrightarrow \Aut(\mathbb{Z}^{(X)})$ is the
natural map. Therefore the problem of finding all the solutions to
the YBE is reduced to the problem of finding all the faithful
$(B,\cdot)$-sets $X$ and all the homomorphisms of $(B,\cdot)$-sets
$X\longrightarrow (B,+)$ such that the image of $X$ generates
$(B,+)$. In the next sections we give a method that describes how to
solve this problem using only  the structure of the brace $B$.



\section{Construction of solutions}\label{sec3}
As we have seen in the previous section, given a brace $B$ of abelian type, to
construct an involutive solution $(X,r)$ to the YBE such that
$\mathcal{G}(X,r)\cong B$, it is enough to find two  maps
$\eta\colon X\longrightarrow B$ and $\sigma\colon B\longrightarrow
\Sym_X$ satisfying some properties. In this section, we show how to
construct the sets $X$ and the maps $\eta$ and $\sigma$ using only
information about the structure of the brace $B$.

Recall that $\lambda : (B,\cdot ) \rightarrow \Aut(B,+)$ is an
action. The stabiliser of $a\in B$ by the action $\lambda$ is
denoted $\St(a)$. For $a\in B$, let $B_a=\{\lambda_b(a)\mid b\in
B\}$ be the orbit of $a$, and let $\mathcal{O}=\{ B_a\mid a\in B\}$
be the set of orbits of the action $\lambda$.  For each $i\in
\mathcal{O}$, choose an element $a_i\in i$. Let $I$ be a subset of
$\mathcal{O}$, such that $Y=\bigcup_{i\in I}i$  (and thus
$Y=\{\lambda_b(a_i)\mid i\in I,\; b\in B\} )$
satisfies $B=\langle
Y\rangle_+$, the additive subgroup generated by $Y$. For each $i\in
I$, let $J_i$ be a non-empty set and  let $\{ K_{i,j} \}_{j\in J_i}$
be a family of subgroups of $\St(a_i)$ such that
$$\bigcap_{i\in I}\bigcap_{j\in J_i}\bigcap_{b\in B}bK_{i,j}b^{-1}=\{ 1\}.$$
Note that if one of the subgroups $K_{i,j}$ is trivial, then this
last condition is satisfied.


The next result  is the main result of this section.


\begin{theorem}\label{main}
	With the above notation,  let  $X:=\bigsqcup_{i\in I}\bigsqcup_{j\in
		J_i}B/K_{i,j}$ be the disjoint union of the sets of left cosets
	$B/K_{i,j}$. Let $\eta\colon X\longrightarrow B$ be the map
	defined by $\eta(bK_{i,j})=\lambda_b(a_i)$, and define $\sigma\colon
	B\longrightarrow \Sym_{X}$ to be the natural action of $(B,\cdot)$
	on $X$ given by left multiplications on the cosets in $B/K_{i,j}$;
	i.e. $\sigma(c)(bK_{i,j}):=cbK_{i,j}$. Then $\sigma$ is injective
	and $\eta(\sigma(a)(x))=\lambda_a(\eta(x))$, for all $x\in X$.
	Moreover $(X,r)$, where $r$ is the map
	$$
	\begin{array}{cccc}
		r\colon& X\times X &\longrightarrow & X\times X\\
		& (x,y) &\mapsto &(f_{x}(y),~f^{-1}_{f_{x}(y)}(x)),
	\end{array}
	$$
	with
	$f_{x}(y)=\sigma(\eta(x))(y)$, is an involutive solution to the YBE such that
	$\mathcal{G}(X,r)\cong B$ as braces.
	
	Furthermore, any involutive solution $(Z,t)$, with $\mathcal{G}(Z,t)\cong
	B$ as braces, is isomorphic to  such a solution.
\end{theorem}

\begin{proof}
	First we shall prove  $\sigma$ is injective. Let $c\in B$ be an
	element such that $\sigma(c)=\id$. Hence $cbK_{i,j}=bK_{i,j}$, for
	all $b\in B$, $i\in I$ and $j\in J_i$. Thus $c\in \bigcap_{i\in
		I}\bigcap_{j\in J_i}\bigcap_{b\in B} bK_{i,j}b^{-1}=\{ 1\}$.
	Therefore $\sigma$ is injective. Let $a,b\in B$. We have that
	\begin{eqnarray*}\eta(\sigma(a)(bK_{i,j}))&=&\eta(abK_{i,j})=\lambda_{ab}(a_i)\\
		&=&\lambda_{a}(\lambda_{b}(a_i))=\lambda_{a}(\eta(bK_{i,j})).
	\end{eqnarray*}
	Hence $\eta(\sigma(a)(x))=\lambda_a(\eta(x))$, for all $x\in X$.
	Hence, by Proposition~\ref{BenDavid}, $(X,r)$ is an involutive solution to the
	YBE and $\mathcal{G}(X,r)\cong B$ as braces.
	
	
	Let $(Z,t)$ be an involutive solution to the YBE such that
	$\mathcal{G}(Z,t)\cong B$ as braces. Let $\xi\colon
	\mathcal{G}(Z,t)\longrightarrow B$ be an isomorphism of braces.
	Let $\eta=\xi\circ \phi$, where $\phi\colon G(Z,t)\longrightarrow
	\mathcal{G}(Z,t)$ is the natural projection. Then $\eta(Z)$ is a
	subset of $B$, invariant by $\lambda$ which generates $B$
	additively. Let $Y=\eta(Z)$.
	
	We also have an injective morphism $\sigma:
	(B,\cdot)\longrightarrow\Aut(\Z^{(Z)})$, such that
	$\sigma(b)(z)=\xi^{-1}(b)(z)$, for all $b\in B$ and $z\in Z$.
	Therefore $Z$ is a left $B$-set with the action induced by $\sigma$.
	Let $a\in B$ and $z\in Z$. Let $g\in G(Z,t)$  such that
	$\phi(g)=\xi^{-1}(a)$. We have
	\begin{eqnarray*}
		\eta(\sigma(a)(z))&=&\xi(\phi(\xi^{-1}(a)(z)))=\xi(\phi(\phi(g)(z)))\\
		&=&\xi(\phi(gz-g))=\xi(\phi(g))\xi(\phi(z))-\xi(\phi(g))\\
		&=&a\eta(z)-a=\lambda_a(\eta(z)).
	\end{eqnarray*}
	Therefore the restriction $\eta|_Z\colon Z\longrightarrow Y$ of
	$\eta$ is a $B$-map.
	
	Let $Y=\bigcup_{i\in I} i$ be the decomposition of $Y$
	as disjoint union of orbits under the action $\lambda$. Since
	$\eta|_Z$ is a surjective $B$-map, for all $i\in I$, the action
	$\sigma$ splits $(\eta|_Z)^{-1}(i)$ into orbits:
	$(\eta|_Z)^{-1}(i)=\bigcup_{j\in J_i} Z_{i,j}$ and
	$\eta(Z_{i,j})=i$. So we have $Z=\bigcup_{i\in I}\bigcup_{j\in J_i}
	Z_{i,j}$, where $\eta(Z_{i,j})=i$ for all $i,j$.
	
	For each $i\in I$, we choose an element $a_i\in i$, and for each
	$j\in J_i$, we choose $z_{i,j}\in Z_{i,j}$ such that
	$\eta(z_{i,j})=a_i$. Note that  $\St(z_{i,j})\leq \St(a_i)\leq B$,
	since any $b\in \St(z_{i,j})$ satisfies $\sigma(b)(z_{i,j})=z_{i,j}$
	and, applying $\eta$, we obtain
	$a_i=\eta(z_{i,j})=\eta(\sigma(b)(z_{i,j}))=\lambda_b(\eta(z_{i,j}))=\lambda_b(a_i)$,
	so $b\in \St(a_i)$.
	
	Recall that the maps $f_i\colon i\rightarrow B/\St(a_i)$ and
	$f_{i,j}\colon Z_{i,j}\rightarrow B/\St(z_{i,j})$ defined by
	$f_{i}(\lambda_b(a_i))= b\St(a_i)$ and $f_{i,j}(\sigma(b)(z_{i,j}))=
	b\St(z_{i,j})$ are isomorphisms of $B$-sets.
	Then,
	$\eta\mid_{Z_{i,j}}$ is determined by the canonical projection
	$$
	\begin{array}{cccc}
		\pi \colon &B/\St(z_{i,j})&\longrightarrow &B/\St(a_i)\\
		&t\St(z_{i,j})&\mapsto &t\St(a_i),
	\end{array}
	$$
	that is, \begin{eqnarray*}
		\eta(f_{i,j}^{-1}(b\St(z_{i,j})))&=&\eta(\sigma(b)(z_{i,j}))=\lambda_b(\eta(z_{i,j}))\\
		&=&
		\lambda_b(a_{i})=f_i^{-1}(b\St(a_i))\\
		&=&f_i^{-1}(\pi(b\St(z_{i,j})))  .\end{eqnarray*}
	
	Note that $\sigma(c)(z)=z$, for all $z\in Z$, if and only if
	$cb\St(z_{i,j})=b\St(z_{i,j})$, for all $b\in B$ and all $i,j$,
	equivalently,  $c\in\bigcap_{i,j} \bigcap_{b\in
		B}b\St(z_{i,j})b^{-1}$. Since $\sigma$ is injective, we have
	$\bigcap_{i,j} \bigcap_{b\in B}b\St(z_{i,j})b^{-1}=\{ 1\}$.  Put
	$H_i:=\St(a_i)$ and $K_{i,j}:=\St(z_{i,j})$.  Let $(X,r)$ be the
	solution defined as in the statement of the theorem. We shall
	prove that $(X,r)\cong (Z,t)$. Let $f\colon Z\longrightarrow X$ be
	the map defined by $f(z)=f_{i,j}(z)$, for all $z\in Z_{i,j}$.
	Clearly $f$ is bijective. Let $x,z\in Z$. We may assume that $z\in
	Z_{i,j}$ and $x\in Z_{p,q}$. Therefore there exist $b_1,b_2\in B$
	such that $z=\sigma(b_1)(z_{i,j})$ and $x=\sigma(b_2)(z_{p,q})$. We
	have
	\begin{eqnarray*}
		f(\phi(x)(z))&=&f(\sigma(\eta(x))(z))=f_{i,j}(\sigma(\eta(x))(\sigma(b_1)(z_{i,j})))\\
		&=&f_{i,j}(\sigma(\eta(x)b_1)(z_{i,j}))=\eta(x)b_1\St(z_{i,j})\\
		&=&\eta(x)b_1K_{i,j}=\eta(\sigma(b_2)(z_{p,q}))b_1K_{i,j}\\
		&=&\lambda_{b_2}(\eta(z_{p,q}))b_1K_{i,j}=\lambda_{b_2}(a_p)b_1K_{i,j}\\
		&=&f_{b_2K_{p,q}}(b_1K_{i,j})\\
		&=&f_{f(x)}(f(z)).
	\end{eqnarray*}
	Therefore $f$ is an isomorphism of solutions to the YBE.
\end{proof}


As a consequence of Theorem~\ref{main}, given a brace $B$ of abelian type, to
construct all the involutive solutions $(X,r)$ of the YBE such that
$\mathcal{G}(X,r)\cong B$ as braces one can proceed as follows:
\begin{enumerate}
	\item Find the decomposition of $B$ as disjoint union of orbits, $B=\bigcup_{i\in K} B_i$, by the action $\lambda\colon (B,\cdot)\longrightarrow\Aut(B,+)$.
	Then choose one element $a_i$ in each orbit $B_i$ for all $i\in K$.
	
	\item Find all the subsets $I$ of $K$ such that the subset
	$Y=\bigcup_{i\in I}B_i$ generates the additive group of $B$.
	
	\item Given such an $Y$, find for each $i\in I$ a non-empty family $\{K_{i,j}\}_{j\in
		J_i}$ of subgroups of $\St(a_i)$ such that $\bigcap_{i,j}
	\bigcap_{b\in B}bK_{i,j}b^{-1}=\{ 1\}$. Note that the $K_{i,j}$
	could be equal for different $(i,j)$.
	
	\item Construct a solution as in the statement of
	Theorem~\ref{main} using the families $\{K_{i,j}\}_{j\in J_i}$, for
	$i\in I$.
\end{enumerate}

Note that by Theorem~\ref{main}, any involutive solution $(X,r)$ to the YBE
such that $\mathcal{G}(X,r)\cong B$  (as braces) is isomorphic
to one constructed in this way. It could happen that different
solutions of the YBE constructed in this way from a brace $B$ of abelian type
are in fact isomorphic. In the next section we characterize when two
of these solutions are isomorphic.



\section{Isomorphism of solutions}
Let $B$ be a brace of abelian type and let $\mathcal{O}$, $I$, $a_i$, $J_i$,
$K_{i,j}$ be as in the beginning of Section~\ref{sec3}. Let $(X,r)$
be the solution to the YBE of the statement of Theorem~\ref{main}.

Let $I'\subseteq \mathcal{O}$ such that $Y'=\bigcup_{i'\in I'}i'$
satisfy $B=\langle Y'\rangle_+$. For each $i'\in I'$, let $\{
L_{i',j'} \}_{j'\in J'_{i'}}$ be a non-empty family of subgroups of
$\St(a_{i'})$ such that
$$\bigcap_{i'\in I'}\bigcap_{j'\in J'_{i'}}\bigcap_{b\in B}bL_{i',j'}b^{-1}=\{ 1\}.$$
Let $(X',r')$ be the corresponding solution to the YBE defined as in
the statement of Theorem~\ref{main}, that is
\begin{eqnarray*}
	&&r'(b_1L_{i'_1,j'_1},b_2L_{i'_2,j'_2})=(\lambda_{b_1}(a_{i'_1})b_2L_{i'_2,j'_2},\lambda_{\lambda_{b_1}(a_{i'_1})b_2}(a_{i'_2})^{-1}b_1L_{i'_1,j'_1}).
\end{eqnarray*}
We shall characterize when $(X,r)$ and $(X',r')$ are isomorphic in
the following result.

\begin{theorem}\label{isomorphism}
	The solutions $(X,r)$ and $(X',r')$ are isomorphic if and only if
	there exist an automorphism $\psi$ of the left brace $B$, a
	bijective map $\alpha\colon I\rightarrow I'$, a bijective map
	$\beta_i\colon J_i\rightarrow J'_{\alpha(i)}$ and $z_{i,j}\in B$,
	for each $i\in I$ and $j\in J_i$, such that
	$$\psi(a_i)=\lambda_{z_{i,j}}(a_{\alpha(i)})\quad\mbox{and}\quad \psi(K_{i,j})=z_{i,j}L_{\alpha(i),\beta_i(j)}z_{i,j}^{-1},$$
	for all $i\in I_1$ and $j\in J_i$.
\end{theorem}

\begin{proof}
	Suppose that there exist an automorphism $\psi$ of the brace
	$B$, a bijective map $\alpha\colon I\rightarrow I'$, a bijective map
	$\beta_i\colon J_i\rightarrow J'_{\alpha(i)}$
	and $z_{i,j}\in B$, for $i\in I$, and for $j\in J_i$, such that
	$$\psi(a_i)=\lambda_{z_{i,j}}(a_{\alpha(i)})\quad\mbox{and}\quad \psi(K_{i,j})=z_{i,j}L_{\alpha(i),\beta_i(j)}z_{i,j}^{-1}.$$
	Observe that we also have
	$\psi(\lambda_b(a_i))=\lambda_{\psi(b)z_{i,j}}(a_{\alpha(i)})$ for
	every $b\in B$, because $\psi$ is a morphism of braces. We define
	$F\colon X\rightarrow X'$ by
	$F(bK_{i,j})=\psi(b)z_{i,j}L_{\alpha(i),\beta_i(j)}$, for all $i\in
	I$, $j\in J_i$ and $b\in B$. Since
	$\psi(K_{i,j})=z_{i,j}L_{\alpha(i),\beta_i(j)}z_{i,j}^{-1}$,  we get
	that  $F$ is well defined. It is easy to check that $F$ is an
	isomorphism of the solutions $(X,r)$ and $(X',r')$.
	
	Conversely, suppose that there exists an isomorphism $F\colon
	X\rightarrow X'$ of the solutions $(X,r)$ and $(X',r')$. We can
	write $F(bK_{i,j})=\varphi(bK_{i,j})L_{\alpha(b,i,j),\beta(b,i,j)}$,
	for some maps $\varphi\colon X\rightarrow B$, $\alpha\colon
	X\rightarrow I'$ and $\beta \colon X\rightarrow \bigcup_{i'\in
		I'}J'_{i'}$. We shall prove that $\alpha(b,i,j)=\alpha(1,i,k)$ and
	$\beta(b,i,j)=\beta(1,i,j)$, for all $b\in B$, $i\in I$ and $j,k\in
	J_i$. Since $F$ is a morphism of solutions to the YBE, we have
	\begin{eqnarray}\label{F}
		&&F(\lambda_{b_1}(a_{i_1})b_2K_{i_2,j_2})
		=\lambda_{\varphi(b_1K_{i_1,j_1})}(a_{\alpha(b_1,i_1,j_1)})F(b_2K_{i_2,j_2}),\end{eqnarray}
	for all $b_1,b_2\in B$, $i_1,i_2\in I$,  $j_1\in J_{i_1}$ and
	$j_2\in J_{i_2}$. Hence
	\begin{eqnarray*}\lefteqn{\varphi(\lambda_{b_1}(a_{i_1})b_2K_{i_2,j_2})L_{\alpha(\lambda_{b_1}(a_{i_1})b_2,i_2,j_2),\beta(\lambda_{b_1}(a_{i_1})b_2,i_2,j_2)}}
		\\
		&&=\lambda_{\varphi(b_1K_{i_1,j_1})}(a_{\alpha(b_1,i_1,j_1)})\varphi(b_2K_{i_2,j_2})L_{\alpha(b_2,i_2,j_2),\beta(b_2,i_2,j_2)},\end{eqnarray*}
	for all $b_1,b_2\in B$, $i_1,i_2\in I$,  $j_1\in J_{i_1}$ and
	$j_2\in J_{i_2}$. Thus
	$\alpha(\lambda_{b_1}(a_{i_1})b_2,i_2,j_2)=\alpha(b_2,i_2,j_2)$ and
	$\beta(\lambda_{b_1}(a_{i_1})b_2,i_2,j_2)=\beta(b_2,i_2,j_2)$. Since
	$B=\langle Y\rangle_+$ and $Y$ is $B$-invariant (by the action
	$\lambda$), we know that $Y$ also generates the multiplicative group
	of $B$. Therefore $\alpha(b_2,i_2,j_2)=\alpha(1,i_2,j_2)$ and
	$\beta(b_2,i_2,j_2)=\beta(1,i_2,j_2)$. Note also that
	$$\lambda_{\varphi(b_1K_{i_1,j_1})}(a_{\alpha(b_1,i_1,j_1)})F(b_2K_{i_2,j_2})=\lambda_{\varphi(b_1K_{i_1,j})}(a_{\alpha(b_1,i_1,j)})F(b_2K_{i_2,j_2}),$$
	for all $b_1,b_2\in B$, $i_1,i_2\in I$,  $j_1,j\in J_{i_1}$ and
	$j_2\in J_{i_2}$. Since $\bigcap_{i',j'}\bigcap_{b\in
		B}bL_{i',j'}b^{-1}=\{ 1\}$, we have that
	$$\lambda_{\varphi(b_1K_{i_1,j_1})}(a_{\alpha(b_1,i_1,j_1)})=\lambda_{\varphi(b_1K_{i_1,j})}(a_{\alpha(b_1,i_1,j)}),$$
	for all $b_1\in B$, $i_1\in I$ and  $j_1,j\in J_{i_1}$. Therefore
	$a_{\alpha(b_1,i_1,j_1)},a_{\alpha(b_1,i_1,j)}\in \alpha(1,i_1,k),$
	for all $b_1\in B$, $i_1\in I$ and $j_1,j\in J_{i_1}$ and thus
	$\alpha(b,i,j)=\alpha(1,i,k)$, for all $b\in B$, $i\in I_1$ and
	$j,k\in J_{i}$. For each $i\in I$ we choose an element $j_i\in
	J_{i}$. Since $F$ is bijective, the map $I\rightarrow I'$ defined by
	$i\mapsto \alpha(1,i,j_i)$ is bijective and for each $i\in I$ the
	map $J_i\rightarrow J'_{\alpha(1,i,j_i)}$ defined by $j\mapsto
	\beta(1,i,j)$ is bijective. We shall see that there exists an
	automorphism $\psi$ of the left brace $B$ such that
	$$\psi(\lambda_b(a_i))=\lambda_{\psi(b)\varphi(K_{i,j_i})}(a_{\alpha(1,i,j_i)})$$
	and
	$$\psi(K_{i,j})=\varphi(K_{i,j_i})L_{\alpha(1,i,j_i),\beta(1,i,j)}\varphi(K_{i,j_i})^{-1},$$
	for all $b\in B$, $i\in I$ and $j\in J_i$. Let
	$1=\lambda_{b_1}(a_{i_1})^{\varepsilon_1}\cdots
	\lambda_{b_m}(a_{i_m})^{\varepsilon_m}$, for some $b_1,\dots ,b_m\in
	B$, $i_1,\dots ,i_m\in I$ and $\varepsilon_1,\dots ,\varepsilon_m\in
	\{ 1,-1\}$. By (\ref{F}), we have
	\begin{align*}
		F(bK_{i,j})&=F(\lambda_{b_1}(a_{i_1})^{\varepsilon_1}\cdots
		\lambda_{b_m}(a_{i_m})^{\varepsilon_m}bK_{i,j})\\
		&=\lambda_{\varphi(b_1K_{i_1,j_{i_1}})}(a_{\alpha(1,i_1,j_{i_1})})^{\varepsilon_1}F(\lambda_{b_2}(a_{i_2})^{\varepsilon_2}\cdots
		\lambda_{b_m}(a_{i_m})^{\varepsilon_m}bK_{i,j})\\
		&=\lambda_{\varphi(b_1K_{i_1,j_{i_1}})}(a_{\alpha(1,i_1,j_{i_1})})^{\varepsilon_1}\cdots
		\lambda_{\varphi(b_mK_{i_m,j_{i_m}})}(a_{\alpha(1,i_m,j_{i_m})})^{\varepsilon_m}F(bK_{i,j}),
	\end{align*}
	for all $b\in B$, $i\in I$ and $ j\in J_i$. Since
	$\bigcap_{i',j'}\bigcap_{b\in B}bL_{j,l}b^{-1}=\{ 1\}$, we have that
	$\lambda_{\varphi(b_1K_{i_1,j_{i_1}})}(a_{\alpha(1,i_1,j_{i_1})})^{\varepsilon_1}\cdots
	\lambda_{\varphi(b_mK_{i_m,j_{i_m}})}(a_{\alpha(1,i_m,j_{i_m})})^{\varepsilon_m}=1$.
	Therefore there exists a unique morphism $\psi\colon B\rightarrow B$
	of multiplicative groups such that
	$\psi(\lambda_b(a_i))=\lambda_{\varphi(bK_{i,j_{i}})}(a_{\alpha(1,i,j_i)})$.
	Since $Y$ generates the multiplicative group of $B$, by (\ref{F})
	one can see that
	$$\varphi(bK_{i,j})L_{\alpha(1,i,j_i),\beta(1,i,j)}=F(bK_{i,j})=\psi(b)\varphi(K_{i,j})L_{\alpha(1,i,j_i),\beta(1,i,j)}.$$
	Therefore, since $L_{\alpha(1,i,j_i),\beta(1,i,j)}\subseteq
	\St(a_{\alpha(1,i,j_i)})$, we have
	$$\lambda_{\varphi(bK_{i,j})}(a_{\alpha(1,i,j_i)})=\lambda_{\psi(b)\varphi(K_{i,j})}(a_{\alpha(1,i,j_i)}).$$
	Hence
	$\psi(\lambda_b(a_i))=\lambda_{\psi(b)\varphi(K_{i,j_i})}(a_{\alpha(1,i,j_i)})$.
	Now we have that
	\begin{eqnarray*}
		\psi(b+a_i)&=&\psi(b\lambda_{b^{-1}}(a_i))=\psi(b)\psi(\lambda_{b^{-1}}(a_i))\\
		&=&\psi(b)\lambda_{\psi(b)^{-1}\varphi(K_{i,j_i})}(a_{\alpha(1,i,j_i)})\\
		&=&\psi(b)\lambda_{\psi(b)^{-1}}(\lambda_{\varphi(K_{i,j_i})}(a_{\alpha(1,i,j_i)}))\\
		&=&\psi(b)+\lambda_{\varphi(K_{i,j_i})}(a_{\alpha(1,i,j_i)})\\
		&=&\psi(b)+\psi(\lambda_{1}(a_i))=\psi(b)+\psi(a_i).
	\end{eqnarray*}
	Now it is easy to see that $\psi$ is a morphism of  braces.
	Since $F$ is bijective and $F(bb'K_{i,j})=\psi(b)F(b'K_{i,j})$, it
	follows that $\psi$ is bijective. Furthermore $b\in K_{i,j}$ if and
	only if
	\begin{eqnarray*}
		\varphi(K_{i,j})L_{\alpha(1,i,j_i),\beta(1,i,j)}&=&F(K_{i,j})=F(bK_{i,j})=\psi(b)F(K_{i,j})\\
		&=&\psi(b)\varphi(K_{i,j})L_{\alpha(1,i,j_i),\beta(1,i,j)}.
	\end{eqnarray*}
	Therefore the result follows.
\end{proof}

Summarizing, the last theorem says that two solutions constructed as
in Theorem~\ref{main} are isomorphic if we can find an automorphism
of the brace $B$ that brings each $K_{i,j}$ to one $L_{i',j'}$,
taking into account that maybe the $L_{i',j'}$'s are permuted (that
is the reason for the $\alpha$ and $\beta_i$ maps), and that maybe
we have chosen another element of the orbit in the process (that is
the reason why the image $a_i$ is $\lambda_{z_{i,j}}(a_{\alpha(i)})$
and not just $a_{\alpha(i)}$, and it is also the reason why the
$L_{\alpha(i),\beta_i(j)}$ is conjugated by $z_{i,j}$).


The following is an example of how to use  Theorem~\ref{main} and
Theorem~\ref{isomorphism} to compute all the finite solutions
associated to a given finite left brace up to isomorphism. We use
the easiest examples of braces: trivial braces of order $p$, where
$p$ is a prime. Recall that a brace $B$ is trivial if $ab=a+b$
for all $a,b\in B$.

\begin{example}\label{bracesp}
	{\rm Consider the trivial brace  over $G=\Z/(p)$. Then, the orbits
		are $\{\alpha\}$ for every $\alpha\in\Z/(p)$.
		Since any orbit has  only one element,  we have $\St(\alpha)=G$, and the possible $K_{i,j}$'s in this case are $0$ and $G$.
		
		Let $Y$ be a subset of $\Z/(p)$ with at least a nonzero element.
		Let $K_{\alpha,j}=G$ for $\alpha\in Y$ and $j\in\{1,\dots,k_{\alpha}\}$, and let $K'_{\alpha,k}=0$ for
		$\alpha\in Y$ and $k\in\{1,\dots,m_\alpha\}$, where $k_{\alpha}$ and $m_{\alpha}$ are non-negative integers such that $k_{\alpha}+m_{\alpha}>0$.
		Write $G/K_{\alpha,j}=\{y_{\alpha,j}\}$, and
		$G/K'_{\alpha,k}=\{y^1_{\alpha,k},\dots,y^p_{\alpha,k}\}$, where
		$y^{l}_{\alpha,k}=l+K'_{\alpha,k}$. Assume that at least one $m_{\alpha}$ is positive. Then the corresponding solution
		of the YBE is $(X,r)$, where
		$$
		X=\bigcup_{\alpha\in Y}\left( \left(\bigcup_{1\leq j\leq
			k_{\alpha}}\{ y_{\alpha,j}\}\right)\cup \left(\bigcup_{1\leq k\leq
			m_{\alpha}} \{y^1_{\alpha,k},\dots ,y^p_{\alpha,k}\}\right) \right)
		$$
		and $r(x,y)=(\sigma_x(y),\sigma^{-1}_{\sigma_{x}(y)}(x))$, with the
		sigma maps given by
		$$
		\sigma_{y_{\alpha,j}}=\sigma_{y^l_{\alpha,k}}=\tau^\alpha\text{, for
			all } \alpha\in Y,\text{ for all } j,k \text{ and for all }
		l\in\{1,\dots,p\},
		$$
		where $\tau\in \Sym_X$ is the product of all the cycles
		$(y^1_{\alpha,k},y^2_{\alpha,k},\dots,y^p_{\alpha,k})$ for any
		$\alpha\in Y$ and $k\in\{1,\dots,m_\alpha\}$.
		
		Finally observe that, in this case, $\Aut(G,+,\cdot)=\Aut(G,+)\cong
		(\Z/(p))^*$, and the effect of an automorphism of $G$ over a
		solution is to change
		$\sigma_{y_{\alpha,j}}=\sigma_{y^l_{\alpha,k}}=\tau^\alpha$ to the
		isomorphic solution
		$\sigma_{y_{\alpha,j}}=\sigma_{y^l_{\alpha,k}}=\tau^{A\alpha}$,
		where $A\in(\Z/(p))^*$.}
\end{example}






\section*{Exercises}

\begin{prob}
\end{prob}

\begin{prob}
\end{prob}


\begin{prob}
\end{prob}

\begin{prob}
\end{prob}


\section*{Notes}

\index{Bachiller, D.}
\index{Ced\'o, F}
\index{Jespers, E.}
\index{Okni\'{n}ski, J.}
The first example of a simple non-trivial brace of abelian type is due to Bachiller \cite{MR3763276}. 
In the same papar Bachiller introduced the matched product of two left braces of abelian type. One can see that the definition of matched product of two left braces is the same that in the case of braces of abelian type. 
In \cite{MR3812099}, Bachiller, Ced\'o, Jespers and Okni\'{n}ski introduced the matched product of more than two left braces of abelian type (this can be generaliced to arbitrari left braces without changes). In the same paper the authors construct several families of  simple non-trivial left braces of abelian type using matched products.

\index{Catino, F.}
\index{Colazzo, I.}
\index{Stefanelli, P.}
The asymmetric product of left braces of abelian type was intruduced by Catino, Colazzo and Stefanelli in \cite{MR3478858}, and Theorem \ref{ccs} is due to them (\cite[Theorem 3]{MR3478858}).  In \cite{MR4020748} Bachiller, Ced\'o, Jespers and Okni\'{n}ski used the asymmetric product to construct to costruct new families of simple non-trivial left braces of abelian type. In fact, every known simple non-trivial left brace is an asymmetric product (see \cite{MR4020748, MR4161288, MR4122077}).
Theorem \ref{thm:simplebrace} is based on \cite[Theorem 3.6]{MR3812099} and \cite[Theorem 6.2]{MR4020748}. The two concrete constructions of simple non-trivial simple braces of abelian type appear in \cite{MR3812099,  MR4161288}.
