\chapter{Involutive solutions}
\label{Isolutions}



Let $(X,r)$ be an involutive solution to the YBE. We know that its permutation group $\mathcal{G}(X,r)$ has a natural structure of skew brace of abelian type. If $(Y,r')$ is a solution isomorphic to $(X,r)$, then $\mathcal{G}(Y,r')\cong\mathcal{G}(X,r)$ as skew braces. Thus the classification of the involutive solutions can be done in two steps:
\begin{enumerate}
	\item Classify all the skew braces of abelian type.
	\item For each skew brace $B$ of abelian type, classify all the involutive solutions $(X,r)$ to the YBE such that $\mathcal{G}(X,r)\cong B$.
\end{enumerate}

In this chapter we focus on the second step. 

\section{Extensions of skew braces of abelian type}

In this section we study some extensions of skew braces of abelian type and how to construct involutive solutions to the YBE using these extensions.

\begin{proposition}\label{extensions}  Let $H$ be an additive abelian group and $B$
	be a skew brace of abelian type. Suppose  that $\varphi\colon
	(B,\circ)\to\Aut(H,+)$ is an injective homomorphism and
	$\eta\colon (H,+)\to (B,+)$ is a surjective homomorphism
	such that they satisfy $\eta(\varphi(g)(h))=\lambda_g(\eta(h))$ for
	all $g\in B$ and $h\in H$. Then, the multiplication in $H$ defined
	by
	$$h_1\cdot h_2:=h_1+\varphi(\eta (h_1))(h_2),$$
	for $h_1,h_2\in H$, defines a structure of skew brace on $H$
	such that $\eta$ is a homomorphism of skew braces  with
	$\Soc(H)=\ker(\eta)$ and $H/\Soc(H)\cong B$ as skew braces.
	
	Two of these structures, determined by $\varphi$, $\eta$ and
	$\varphi'$, $\eta'$ respectively, are isomorphic if and only if there
	exists an $F\in \Aut(H,+)$ such that
	$$
	\varphi'(\eta'(h))=F^{-1}\varphi(\eta(F(h))) F,
	$$
	for all $h\in H$.
	
	Conversely, suppose that $G$ is a skew brace of abelian type. Then, the map
	$\varphi\colon (G/\Soc(G),\circ)\to\Aut(G,+)$, induced by
	the map $\lambda\colon (G,\circ)\to\Aut(G,+)$, and the
	natural map $\eta\colon G\to G/\Soc(G)$ satisfy the
	above properties. 
\end{proposition}

\begin{proof}
We shall see that $(H,\cdot)$ is a group. Let $a,b,c\in H$. We have
	\begin{eqnarray*}
		a\cdot (b\cdot c)&=&a\cdot (b+\varphi(\eta (b))(c))=a+\varphi(\eta (a))(b+\varphi(\eta (b))(c))\\
		&=&a+\varphi(\eta (a))(b)+\varphi(\eta (a))(\varphi(\eta (b))(c))\\
		&=&a+\varphi(\eta(a))(b)+\varphi(\eta(a)\circ \eta(b))(c)\\
		&=&a+\varphi(\eta(a))(b)+\varphi(\eta(a)+\lambda_{\eta(a)}(\eta(b)))(c)\\
		&=&a+\varphi(\eta(a))(b)+\varphi(\eta(a)+\eta(\varphi(\eta(a))(b)))(c)\\
		&=&(a+\varphi(\eta(a))(b))+\varphi(\eta(a+\varphi(\eta(a))(b)))(c)\\
		&=&(a\cdot b)\cdot c.
	\end{eqnarray*}	
Note that, $0\cdot a=0+\varphi(\eta(0))(a)=\varphi(0)(a)=\id(a)=a$ and $a\cdot 0=a+\varphi(\eta(a))(0)=a+0=a$. Furthermore,
$a\cdot (-\varphi(\eta(a)')(a))=a+\varphi(\eta(a))(-\varphi(\eta(a)')(a))=a-\varphi(0)(a)=a-a=0$ and
\begin{eqnarray*}
	(-\varphi(\eta(a)')(a))\cdot a&=&-\varphi(\eta(a)')(a)+\varphi(\eta(-\varphi(\eta(a)')(a)))(a)\\
	&=&-\varphi(\eta(a)')(a)+\varphi(-\eta(\varphi(\eta(a)')(a)))(a)\\
	&=&-\varphi(\eta(a)')(a)+\varphi(-\lambda_{\eta(a)'}(\eta(a)))(a)\\
	&=&-\varphi(\eta(a)')(a)+\varphi(\eta(a)')(a)=0.
\end{eqnarray*}
Hence $(H,\cdot)$ is a group. Now we have
\begin{eqnarray*}
a\cdot (b+c)&=&a+\varphi(\eta(a))(b+c)=a+\varphi(\eta(a))(b)+\varphi(\eta(a))(c)\\
&=&a+\varphi(\eta(a))(b)-a+a+\varphi(\eta(a))(b+c)=a\cdot b-a+a\cdot c.
\end{eqnarray*}
Thus $(H,+,\cdot)$ is a skew brace of abelian type. Note that $\eta(a\cdot b)=\eta(a+\varphi(\eta(a))(b))=\eta(a)+\eta(\varphi(\eta(a))(b))=\eta(a)+\lambda_{\eta(a)}(\eta(b))=\eta(a)\circ\eta(b)$. Hence $\eta$ is a homomorphism of skew braces. Since $\varphi$ is injective, 
\begin{align*}
    Soc(H)=&\{a\in H : a\cdot b=a+b \text{ for all } b\in H\}\\
    =&\{a\in H : \varphi(\eta(a))(b)=b\text{ for all } b\in H\}\\
    =&\{ a\in H : \eta(a)=0\}=\ker(\eta).
    \end{align*}
Therefore, since $\eta$ is surjective, $H/\Soc(\eta)\cong B$ as skew braces.

Let $\varphi'\colon
(B,\circ)\to \Aut(H,+)$ be an injective homomorphism and
let $\eta'\colon (H,+)\to (B,+)$ be a surjective homomorphism
such that they satisfy $\eta'(\varphi'(g)(h))=\lambda_g(\eta'(h))$ for
all $g\in B$ and $h\in H$. Define the multiplication $\circ$ in $H$ 
by
$$h_1\circ h_2:=h_1+\varphi'(\eta' (h_1))(h_2),$$
for $h_1,h_2\in H$. Note that $F\in\Aut(H,+)$ is an isomorphism from $(H,+,\circ)$ to $(H,+,\cdot)$ if and only if
$$F(h_1\circ h_2)=F(h_1)\cdot F(h_2),$$
which is equivalent to
$$F(h_1+\varphi'(\eta' (h_1))(h_2))=F(h_1)+\varphi(\eta (F(h_1)))(F(h_2)).$$
Hence $F$ is an isomorphism if and only if 
$$F^{-1}\varphi(\eta(F(h))) F=\varphi'(\eta'(h)),$$
for all $h\in H$. This proves the first part of the result. The second part follows easily.
\end{proof}	



The next result gives a first way to construct
involutive solutions to the YBE. 

\begin{proposition}\label{BenDavid}
	Let $B$ be a skew brace of abelian type, and let $X$ be a non-empty set.  Suppose that
	$\eta\colon \Z^{(X)}\to (B,+)$ is a surjective homomorphism,
	and that $\varphi\colon (B,\circ)\to\Aut(\Z^{(X)})$ is an
	injective homomorphism such that $\varphi(a)\mid_X$ is a bijection of $X$
	for all $a\in B$, and $\eta(\varphi(a)(m))=\lambda_a(\eta(m))$ for
	all $a\in B$ and $m\in\Z^{(X)}$. Let $r$ be the map
	$$
	\begin{array}{cccc}
		r\colon & X\times X &\to & X\times X\\
		& (x,~y) &\mapsto &(\sigma_{x}(y),~\sigma^{-1}_{\sigma_{x}(y)}(x)),
	\end{array}
	$$
	where $\sigma_x(y):=\varphi(\eta(x))(y)$. Then $(X,r)$ is an involutive solution to
	the YBE and $\mathcal{G}(X,r)\cong B$ as skew braces. Moreover, any
	involutive solution $(Z,t)$ to the YBE with $\mathcal{G}(Z,t)\cong B$ is of
	this form.
\end{proposition}
\begin{proof}
	By Proposition~\ref{extensions}, the abelian group
	$\mathbb{Z}^{(X)}$ with the multiplication defined  by
	$$a\cdot b:=a+\varphi(\eta(a))(b),$$
	for $a,b\in \mathbb{Z}^{X}$, is a skew brace of abelian type and $\eta$ becomes a
	homomorphism of skew braces. Note that for $x, y\in X\subseteq
	\mathbb{Z}^{(X)}$, we have that
	$$\lambda_x(y)=-x+x\cdot y=-x+x+\varphi(\eta(x))(y)=\varphi(\eta(x))(y)=\sigma_x(y).$$
	Therefore $(X,r)$ is an involutive solution to the YBE because it is the
	restriction of the solution to the YBE associated with the 
	skew brace $\mathbb{Z}^{(X)}$ (cf. Theorem \ref{thm:YB} and Lemma \ref{lem:|r|}). In fact the
	skew brace $\mathbb{Z}^{(X)}$ is equal to the skew brace $G(X,r)$.
	Recall that the addition of the skew brace
	$$\mathcal{G}(X,r)=\{ \varphi(\eta(m))|_{X} : m\in \Z^{(X)}\}$$
	is defined by
	$\varphi(\eta(m_1))|_{X}+\varphi(\eta(m_2))|_{X}=\varphi(\eta(m_1+m_1))|_{X}$,
	for all $m_1,m_2\in \Z^{(X)}$.  Therefore, the map $B\to
	\mathcal{G}(X,r)$ defined by $a\mapsto \varphi(a)|_{X}$ is an
	isomorphism of skew braces.
	
	On the other hand, observe that, if $(Z,t)$ is an involutive solution to the YBE
	and  $\xi\colon B\to \mathcal{G}(Z,t)$ is an isomorphism
	of skew braces, then the unique homomorphism  $\eta\colon
	(\Z^{(Z)},+)\to (B,+)$ such that
	$\eta(z)=\xi^{-1}(\phi(z))$, for all $z\in Z$, where $\phi\colon
	G(Z,t)\to \mathcal{G}(Z,t)$ is the natural projection,
	is surjective, and the map
	$\varphi:(B,\circ)\to\Aut(\Z^{(Z)})$, where $\varphi(a)$
	is the unique automorphism of $\Z^{(Z)}$ such that
	$\varphi(a)(z)=\xi(a)(z)$, for all $z\in Z$, is an injective
	homomorphism. Furthermore, for $a\in B$, there exists $m\in
	\Z^{(Z)}$ such that $a=\eta(m)$ and,   since $\phi$
	is a homomorphism of skew braces,
	\begin{eqnarray*}
		\eta(\varphi(a)(z))&=&\eta(\xi(a)(z))=\xi^{-1}(\phi(\xi(\eta(m))(z)))\\
		&=&\xi^{-1}(\phi(\phi(m)(z)))=\xi^{-1}(\phi(-m+mz))=\xi^{-1}(-\phi(m)+\phi(m)\phi(z))\\
		&=&-\eta(m)+\eta(m)\eta(z)=-a+a\eta(z)\\
		&=&\lambda_a(\eta(z)),
	\end{eqnarray*}
	for all $z\in Z$. Therefore $\eta(\varphi(a)(n))=\lambda_a(\eta(n))$,
	for all $a\in B$ and all $n\in \Z^{(Z)}$. Now we have that
	$$\varphi(\eta(x))(y)=\xi(\eta(x))(y)=\phi(x)(y),$$
	for all $x,y\in Z$. Hence $(Z,t)$ is exactly the same solution to
	the YBE  as the solution obtained by the given construction using
	the maps $\eta$ and $\varphi$.
\end{proof}

This proposition  reduces the problem of finding all the involutive solutions
to the YBE to the problem of finding   maps $\eta$ and $\varphi$ with
the required properties. Note that the map $\eta$ is determined by
its restriction to $X$ and, since $\eta$ is surjective, $\eta(X)$
should generate $(B,+)$. Note also that the monomorphism $\varphi$
factors trough the symmetric group $\Sym_X$, that is
$\varphi=\varphi_1\varphi_2$ where $\varphi_2\colon
(B,\circ)\to \Sym_X$ is a monomorphism and
$\varphi_1\colon \Sym_X\to \Aut(\mathbb{Z}^{(X)})$ is the
natural map. Therefore the problem of finding all the solutions to
the YBE is reduced to the problem of finding all the faithful
$(B,\circ)$-sets $X$ and all the homomorphisms of $(B,\circ)$-sets
$X\to (B,+)$ such that the image of $X$ generates
$(B,+)$. In the next sections we give a method that describes how to
solve this problem using only  the structure of the skew brace $B$.



\section{Construction of solutions}\label{construcsec3}
As we have seen in the previous section, given a skew brace $B$ of abelian type, to
construct an involutive solution $(X,r)$ to the YBE such that
$\mathcal{G}(X,r)\cong B$ as skew braces, it is enough to find two  maps
$\eta\colon X\to B$ and $\varphi\colon B\to \Sym_X$ satisfying some properties. In this section, we show how to
construct the sets $X$ and the maps $\eta$ and $\varphi$ using only
information about the structure of the skew brace $B$.

Recall that $\lambda : (B,\circ ) \to \Aut(B,+)$ is an
action. The stabiliser of $a\in B$ by the action $\lambda$ is
denoted $\St(a)$. For $a\in B$, let $B_a=\{\lambda_b(a)\mid b\in
B\}$ be the orbit of $a$, and let $\mathcal{O}=\{ B_a\mid a\in B\}$
be the set of orbits of the action $\lambda$.  For each $i\in
\mathcal{O}$, choose an element $a_i\in i$. Let $I$ be a subset of
$\mathcal{O}$, such that $Y=\bigcup_{i\in I}i$  (and thus
$Y=\{\lambda_b(a_i)\mid i\in I,\; b\in B\} )$
satisfies $B=\langle
Y\rangle_+$, the additive subgroup generated by $Y$. For each $i\in
I$, let $J_i$ be a non-empty set and  let $\{ K_{i,j} \}_{j\in J_i}$
be a family of subgroups of $\St(a_i)$ such that
$$\bigcap_{i\in I}\bigcap_{j\in J_i}\bigcap_{b\in B}b\circ K_{i,j}\circ b'=\{ 0\}.$$
Note that if one of the subgroups $K_{i,j}$ is trivial, then this
last condition is satisfied.


The next result  is the main result of this section.


\begin{theorem}\label{main}
	With the above notation,  let  $X:=\bigsqcup_{i\in I}\bigsqcup_{j\in
		J_i}B/K_{i,j}$ be the disjoint union of the sets of left cosets
	$B/K_{i,j}$ of the group $(B,\circ)$. Let $\eta\colon X\to B$ be the map
	defined by $\eta(b\circ K_{i,j})=\lambda_b(a_i)$, and define $\varphi\colon
	B\to  \Sym_{X}$ to be the natural action of $(B,\circ)$
	on $X$ given by left multiplications on the cosets in $B/K_{i,j}$;
	i.e. $\varphi(c)(b\circ K_{i,j}):=c\circ b\circ K_{i,j}$. Then $\varphi$ is injective
	and $\eta(\varphi(a)(x))=\lambda_a(\eta(x))$, for all $x\in X$ and $a\in B$.
	Moreover $(X,r)$, where $r$ is the map
	$$
	\begin{array}{cccc}
		r\colon& X\times X &\to & X\times X\\
		& (x,y) &\mapsto &(\sigma_{x}(y),~\sigma^{-1}_{\sigma_{x}(y)}(x)),
	\end{array}
	$$
	with
	$\sigma_{x}(y)=\varphi(\eta(x))(y)$, is an involutive solution to the YBE such that
	$\mathcal{G}(X,r)\cong B$ as skew braces.
	
	Furthermore, any involutive solution $(Z,t)$, with $\mathcal{G}(Z,t)\cong
	B$ as skew braces, is isomorphic to  such a solution.
\end{theorem}

\begin{proof}
	First we shall prove that $\varphi$ is injective. Let $c\in B$ be an
	element such that $\varphi(c)=\id$. Hence $c\circ b\circ K_{i,j}=b\circ K_{i,j}$, for
	all $b\in B$, $i\in I$ and $j\in J_i$. Thus $c\in \bigcap_{i\in
		I}\bigcap_{j\in J_i}\bigcap_{b\in B} b\circ K_{i,j}\circ b'=\{ 0\}$.
	Therefore $\varphi$ is injective. Let $a,b\in B$. We have that
	\begin{eqnarray*}\eta(\varphi(a)(b\circ K_{i,j}))&=&\eta(a\circ b\circ K_{i,j})=\lambda_{a\circ b}(a_i)\\
		&=&\lambda_{a}(\lambda_{b}(a_i))=\lambda_{a}(\eta(b\circ K_{i,j})).
	\end{eqnarray*}
	Hence $\eta(\varphi(a)(x))=\lambda_a(\eta(x))$, for all $x\in X$ and $a\in B$.
	Hence, by Proposition~\ref{BenDavid}, $(X,r)$ is an involutive solution to the
	YBE and $\mathcal{G}(X,r)\cong B$ as skew braces.
	
	
	Let $(Z,t)$ be an involutive solution to the YBE such that
	$\mathcal{G}(Z,t)\cong B$ as skew braces. Let $\xi\colon
	\mathcal{G}(Z,t)\to  B$ be an isomorphism of skew braces.
	Let $\eta=\xi\phi$, where $\phi\colon G(Z,t)\to
	\mathcal{G}(Z,t)$ is the natural projection. Then $\eta(Z)$ is a
	subset of $B$, invariant by $\lambda$ which generates $B$
	additively. Let $Y=\eta(Z)$.
	
	We also have an injective homomorphism $\varphi:
	(B,\circ)\to \Aut(\Z^{(Z)})$, such that
	$\varphi(b)(z)=\xi^{-1}(b)(z)$, for all $b\in B$ and $z\in Z$.
	Therefore $Z$ is a left $B$-set with the action induced by $\varphi$.
	Let $a\in B$ and $z\in Z$. Let $g\in G(Z,t)$  such that
	$\phi(g)=\xi^{-1}(a)$. We have
	\begin{eqnarray*}
		\eta(\varphi(a)(z))&=&\xi(\phi(\xi^{-1}(a)(z)))=\xi(\phi(\phi(g)(z)))\\
		&=&\xi(\phi(-g+gz))=-\xi(\phi(g))+\xi(\phi(g))\circ\xi(\phi(z))\\
		&=&-a+a\circ \eta(z)=\lambda_a(\eta(z)).
	\end{eqnarray*}
	Therefore the restriction $\eta|_Z\colon Z\longrightarrow Y$ of
	$\eta$ is a $B$-map.
	
	Let $Y=\bigcup_{i\in I} i$ be the decomposition of $Y$
	as disjoint union of orbits under the action $\lambda$. Since
	$\eta|_Z$ is a surjective $B$-map, for all $i\in I$, the action
	$\varphi$ splits $(\eta|_Z)^{-1}(i)$ into orbits:
	$(\eta|_Z)^{-1}(i)=\bigcup_{j\in J_i} Z_{i,j}$ and
	$\eta(Z_{i,j})=i$. So we have $Z=\bigcup_{i\in I}\bigcup_{j\in J_i}
	Z_{i,j}$, where $\eta(Z_{i,j})=i$ for all $i,j$.
	
	For each $i\in I$, we choose an element $a_i\in i$, and for each
	$j\in J_i$, we choose $z_{i,j}\in Z_{i,j}$ such that
	$\eta(z_{i,j})=a_i$. Note that  $\St(z_{i,j})\leq \St(a_i)\leq B$,
	since any $b\in \St(z_{i,j})$ satisfies $\varphi(b)(z_{i,j})=z_{i,j}$
	and, applying $\eta$, we obtain
	$a_i=\eta(z_{i,j})=\eta(\varphi(b)(z_{i,j}))=\lambda_b(\eta(z_{i,j}))=\lambda_b(a_i)$,
	so $b\in \St(a_i)$.
	
	Recall that the maps $f_i\colon i\rightarrow B/\St(a_i)$ and
	$f_{i,j}\colon Z_{i,j}\rightarrow B/\St(z_{i,j})$ defined by
	$f_{i}(\lambda_b(a_i))= b\circ \St(a_i)$ and $f_{i,j}(\varphi(b)(z_{i,j}))=
	b\circ \St(z_{i,j})$ are isomorphisms of $B$-sets.
	Then,
	$\eta\mid_{Z_{i,j}}$ is determined by the canonical projection
	$$
	\begin{array}{cccc}
		\pi \colon &B/\St(z_{i,j})&\to  &B/\St(a_i)\\
		&t\circ\St(z_{i,j})&\mapsto &t\circ \St(a_i),
	\end{array}
	$$
	that is, \begin{eqnarray*}
		\eta(f_{i,j}^{-1}(b\circ\St(z_{i,j})))&=&\eta(\varphi(b)(z_{i,j}))=\lambda_b(\eta(z_{i,j}))\\
		&=&
		\lambda_b(a_{i})=f_i^{-1}(b\circ\St(a_i))\\
		&=&f_i^{-1}(\pi(b\circ \St(z_{i,j})))  .\end{eqnarray*}
	
	Note that $\varphi(c)(z)=z$, for all $z\in Z$, if and only if
	$c\circ b\circ \St(z_{i,j})=b\circ\St(z_{i,j})$, for all $b\in B$ and all $i,j$,
	equivalently,  $c\in\bigcap_{i,j} \bigcap_{b\in
		B}b\circ\St(z_{i,j})\circ b'$. Since $\varphi$ is injective, we have
	$\bigcap_{i,j} \bigcap_{b\in B}b\circ\St(z_{i,j})\circ b'=\{ 0\}$.  Put
	$H_i:=\St(a_i)$ and $K_{i,j}:=\St(z_{i,j})$.  Let $(X,r)$ be the
	solution defined as in the statement of the theorem. We shall
	prove that $(X,r)\cong (Z,t)$. Let $f\colon Z\to X$ be
	the map defined by $f(z)=f_{i,j}(z)$, for all $z\in Z_{i,j}$.
	Clearly $f$ is bijective. Let $x,z\in Z$. We may assume that $z\in
	Z_{i,j}$ and $x\in Z_{p,q}$. Therefore there exist $b_1,b_2\in B$
	such that $z=\varphi(b_1)(z_{i,j})$ and $x=\varphi(b_2)(z_{p,q})$. We
	have
	\begin{eqnarray*}
		f(\phi(x)(z))&=&f(\varphi(\eta(x))(z))=f_{i,j}(\varphi(\eta(x))(\varphi(b_1)(z_{i,j})))\\
		&=&f_{i,j}(\varphi(\eta(x)b_1)(z_{i,j}))=\eta(x)\circ b_1\circ \St(z_{i,j})\\
		&=&\eta(x)\circ b_1\circ K_{i,j}=\eta(\varphi(b_2)(z_{p,q}))\circ b_1\circ K_{i,j}\\
		&=&\lambda_{b_2}(\eta(z_{p,q}))\circ b_1\circ K_{i,j}=\lambda_{b_2}(a_p)\circ b_1\circ K_{i,j}\\
		&=&f_{b_2\circ K_{p,q}}(b_1\circ K_{i,j})\\
		&=&f_{f(x)}(f(z)).
	\end{eqnarray*}
	Therefore $f$ is an isomorphism of solutions to the YBE.
\end{proof}


As a consequence of Theorem~\ref{main}, given a skew brace $B$ of abelian type, to
construct all the involutive solutions $(X,r)$ to the YBE such that
$\mathcal{G}(X,r)\cong B$ as skew braces one can proceed as follows:
\begin{enumerate}
	\item Find the decomposition of $B$ as disjoint union of orbits, $B=\bigcup_{i\in K} B_i$, by the action $\lambda\colon (B,\circ)\to \Aut(B,+)$.
	Then choose one element $a_i$ in each orbit $B_i$ for all $i\in K$.
	
	\item Find all the subsets $I$ of $K$ such that the subset
	$Y=\bigcup_{i\in I}B_i$ generates the additive group of $B$.
	
	\item Given such an $Y$, find for each $i\in I$ a non-empty family $\{K_{i,j}\}_{j\in
		J_i}$ of subgroups of $\St(a_i)$ such that $\bigcap_{i,j}
	\bigcap_{b\in B}b\circ K_{i,j}\circ b'=\{ 0\}$. Note that the $K_{i,j}$
	could be equal for different $(i,j)$.
	
	\item Construct a solution as in the statement of
	Theorem~\ref{main} using the families $\{K_{i,j}\}_{j\in J_i}$, for
	$i\in I$.
\end{enumerate}

Note that by Theorem~\ref{main}, any involutive solution $(X,r)$ to the YBE
such that $\mathcal{G}(X,r)\cong B$  (as skew braces) is isomorphic
to one constructed in this way. It could happen that different
solutions to the YBE constructed in this way from a skew brace $B$ of abelian type
are in fact isomorphic. In the next section we characterize when two
of these solutions are isomorphic.



\section{Isomorphism of solutions} \label{construciso}
Let $B$ be a skew brace of abelian type and let $\mathcal{O}$, $I$, $a_i$, $J_i$,
$K_{i,j}$ be as in the beginning of Section~\ref{sec3}. Let $(X,r)$
be the solution to the YBE of the statement of Theorem~\ref{main}.

Let $I'\subseteq \mathcal{O}$ such that $Y'=\bigcup_{i'\in I'}i'$
satisfies $B=\langle Y'\rangle_+$. For each $i'\in I'$, let $\{
L_{i',j'} \}_{j'\in J'_{i'}}$ be a non-empty family of subgroups of
$\St(a_{i'})$ such that
$$\bigcap_{i'\in I'}\bigcap_{j'\in J'_{i'}}\bigcap_{b\in B}b\circ L_{i',j'}\circ b'=\{ 0\}.$$
Let $(X',r')$ be the corresponding solution to the YBE defined as in
the statement of Theorem~\ref{main}, that is
\begin{eqnarray*}
	&&r'(b_1\circ L_{i'_1,j'_1},b_2\circ L_{i'_2,j'_2})=(\lambda_{b_1}(a_{i'_1})\circ b_2\circ L_{i'_2,j'_2},\lambda_{\lambda_{b_1}(a_{i'_1})\circ b_2}(a_{i'_2})'\circ b_1\circ L_{i'_1,j'_1}).
\end{eqnarray*}
We shall characterize when $(X,r)$ and $(X',r')$ are isomorphic in
the following result.

\begin{theorem}\label{isomorphism}
	The solutions $(X,r)$ and $(X',r')$ are isomorphic if and only if
	there exist an automorphism $\psi$ of the skew brace $B$, a
	bijective map $\alpha\colon I\rightarrow I'$, a bijective map
	$\beta_i\colon J_i\rightarrow J'_{\alpha(i)}$ and $z_{i,j}\in B$,
	for each $i\in I$ and $j\in J_i$, such that
	$$\psi(a_i)=\lambda_{z_{i,j}}(a_{\alpha(i)})\quad\mbox{and}\quad \psi(K_{i,j})=z_{i,j}\circ L_{\alpha(i),\beta_i(j)}\circ z_{i,j}',$$
	for all $i\in I_1$ and $j\in J_i$.
\end{theorem}

\begin{proof}
	Suppose that there exist an automorphism $\psi$ of the skew brace
	$B$, a bijective map $\alpha\colon I\rightarrow I'$, a bijective map
	$\beta_i\colon J_i\rightarrow J'_{\alpha(i)}$
	and $z_{i,j}\in B$, for $i\in I$, and for $j\in J_i$, such that
	$$\psi(a_i)=\lambda_{z_{i,j}}(a_{\alpha(i)})\quad\mbox{and}\quad \psi(K_{i,j})=z_{i,j}\circ L_{\alpha(i),\beta_i(j)}\circ z_{i,j}'.$$
	Observe that we also have
	$\psi(\lambda_b(a_i))=\lambda_{\psi(b)\circ z_{i,j}}(a_{\alpha(i)})$ for
	every $b\in B$, because $\psi$ is a homomorphism of skew braces. We define
	$F\colon X\rightarrow X'$ by
	$F(b\circ K_{i,j})=\psi(b)\circ z_{i,j}\circ L_{\alpha(i),\beta_i(j)}$, for all $i\in
	I$, $j\in J_i$ and $b\in B$. Since
	$\psi(K_{i,j})=z_{i,j}\circ L_{\alpha(i),\beta_i(j)}\circ z_{i,j}'$,  we get
	that  $F$ is well defined. It is easy to check that $F$ is an
	isomorphism of the solutions $(X,r)$ and $(X',r')$.
	
	Conversely, suppose that there exists an isomorphism $F\colon
	X\rightarrow X'$ of the solutions $(X,r)$ and $(X',r')$. We can
	write $F(b\circ K_{i,j})=\varphi(b\circ K_{i,j})\circ L_{\alpha(b,i,j),\beta(b,i,j)}$,
	for some maps $\varphi\colon X\rightarrow B$, $\alpha\colon
	X\rightarrow I'$ and $\beta \colon X\rightarrow \bigcup_{i'\in
		I'}J'_{i'}$. We shall prove that $\alpha(b,i,j)=\alpha(0,i,k)$ and
	$\beta(b,i,j)=\beta(0,i,j)$, for all $b\in B$, $i\in I$ and $j,k\in
	J_i$. Since $F$ is a homomorphism of solutions to the YBE, we have
	\begin{eqnarray}\label{F}
		&&F(\lambda_{b_1}(a_{i_1})\circ b_2\circ K_{i_2,j_2})
		=\lambda_{\varphi(b_1\circ K_{i_1,j_1})}(a_{\alpha(b_1,i_1,j_1)})F(b_2\circ K_{i_2,j_2}),\end{eqnarray}
	for all $b_1,b_2\in B$, $i_1,i_2\in I$,  $j_1\in J_{i_1}$ and
	$j_2\in J_{i_2}$. Hence
	\begin{eqnarray*}\lefteqn{\varphi(\lambda_{b_1}(a_{i_1})\circ b_2\circ K_{i_2,j_2})\circ L_{\alpha(\lambda_{b_1}(a_{i_1})\circ b_2,i_2,j_2),\beta(\lambda_{b_1}(a_{i_1})\circ b_2,i_2,j_2)}}
		\\
		&&=\lambda_{\varphi(b_1\circ K_{i_1,j_1})}(a_{\alpha(b_1,i_1,j_1)})\varphi(b_2\circ K_{i_2,j_2})\circ L_{\alpha(b_2,i_2,j_2),\beta(b_2,i_2,j_2)},\end{eqnarray*}
	for all $b_1,b_2\in B$, $i_1,i_2\in I$,  $j_1\in J_{i_1}$ and
	$j_2\in J_{i_2}$. Thus
	$\alpha(\lambda_{b_1}(a_{i_1})\circ b_2,i_2,j_2)=\alpha(b_2,i_2,j_2)$ and
	$\beta(\lambda_{b_1}(a_{i_1})\circ b_2,i_2,j_2)=\beta(b_2,i_2,j_2)$. Since
	$B=\langle Y\rangle_+$ and $Y$ is $B$-invariant (by the action
	$\lambda$), we know that $Y$ also generates the multiplicative group
	of $B$. Therefore $\alpha(b_2,i_2,j_2)=\alpha(0,i_2,j_2)$ and
	$\beta(b_2,i_2,j_2)=\beta(0,i_2,j_2)$. Note also that
	$$\lambda_{\varphi(b_1K_{i_1,j_1})}(a_{\alpha(b_1,i_1,j_1)})\circ F(b_2\circ K_{i_2,j_2})=\lambda_{\varphi(b_1\circ K_{i_1,j})}(a_{\alpha(b_1,i_1,j)})\circ F(b_2\circ K_{i_2,j_2}),$$
	for all $b_1,b_2\in B$, $i_1,i_2\in I$,  $j_1,j\in J_{i_1}$ and
	$j_2\in J_{i_2}$. Since $\bigcap_{i',j'}\bigcap_{b\in
		B}b\circ L_{i',j'}\circ b'=\{ 0\}$, we have that
	$$\lambda_{\varphi(b_1\circ K_{i_1,j_1})}(a_{\alpha(b_1,i_1,j_1)})=\lambda_{\varphi(b_1\circ K_{i_1,j})}(a_{\alpha(b_1,i_1,j)}),$$
	for all $b_1\in B$, $i_1\in I$ and  $j_1,j\in J_{i_1}$. Therefore
	$a_{\alpha(b_1,i_1,j_1)},a_{\alpha(b_1,i_1,j)}\in \alpha(1,i_1,k),$
	for all $b_1\in B$, $i_1\in I$ and $j_1,j\in J_{i_1}$ and thus
	$\alpha(b,i,j)=\alpha(0,i,k)$, for all $b\in B$, $i\in I_1$ and
	$j,k\in J_{i}$. For each $i\in I$ we choose an element $j_i\in
	J_{i}$. Since $F$ is bijective, the map $I\rightarrow I'$ defined by
	$i\mapsto \alpha(0,i,j_i)$ is bijective and for each $i\in I$ the
	map $J_i\rightarrow J'_{\alpha(0,i,j_i)}$ defined by $j\mapsto
	\beta(0,i,j)$ is bijective. We shall see that there exists an
	automorphism $\psi$ of the left brace $B$ such that
	$$\psi(\lambda_b(a_i))=\lambda_{\psi(b)\varphi(K_{i,j_i})}(a_{\alpha(0,i,j_i)})$$
	and
	$$\psi(K_{i,j})=\varphi(K_{i,j_i})\circ L_{\alpha(0,i,j_i),\beta(0,i,j)}\circ \varphi(K_{i,j_i})',$$
	for all $b\in B$, $i\in I$ and $j\in J_i$. Let
	$0=\lambda_{b_1}(a_{i_1})^{\varepsilon_1}\circ \cdots\circ
	\lambda_{b_m}(a_{i_m})^{\varepsilon_m}$, for some $b_1,\dots ,b_m\in
	B$, $i_1,\dots ,i_m\in I$ and $\varepsilon_1,\dots ,\varepsilon_m\in
	\{ 1,-1\}$ (here we understand that $a^{-1}=a'$ for $a\in B$). By (\ref{F}), we have
	\begin{align*}
		F(b\circ K_{i,j})&=F(\lambda_{b_1}(a_{i_1})^{\varepsilon_1}\circ \cdots\circ 
		\lambda_{b_m}(a_{i_m})^{\varepsilon_m}\circ b\circ K_{i,j})\\
		&=\lambda_{\varphi(b_1\circ K_{i_1,j_{i_1}})}(a_{\alpha(0,i_1,j_{i_1})})^{\varepsilon_1}\circ F(\lambda_{b_2}(a_{i_2})^{\varepsilon_2}\circ \cdots\circ 
		\lambda_{b_m}(a_{i_m})^{\varepsilon_m}\circ b\circ K_{i,j})\\
		&=\lambda_{\varphi(b_1\circ K_{i_1,j_{i_1}})}(a_{\alpha(0,i_1,j_{i_1})})^{\varepsilon_1}\circ \cdots\circ 
		\lambda_{\varphi(b_m\circ K_{i_m,j_{i_m}})}(a_{\alpha(0,i_m,j_{i_m})})^{\varepsilon_m}\circ F(b\circ K_{i,j}),
	\end{align*}
	for all $b\in B$, $i\in I$ and $ j\in J_i$. Since
	$\bigcap_{i',j'}\bigcap_{b\in B}b\circ L_{j,l}\circ b'=\{ 0\}$, we have that
	$\lambda_{\varphi(b_1\circ K_{i_1,j_{i_1}})}(a_{\alpha(0,i_1,j_{i_1})})^{\varepsilon_1}\circ \cdots\circ 
	\lambda_{\varphi(b_m\circ K_{i_m,j_{i_m}})}(a_{\alpha(0,i_m,j_{i_m})})^{\varepsilon_m}=0$.
	Therefore there exists a unique homomorphism $\psi\colon B\rightarrow B$
	of multiplicative groups such that
	$\psi(\lambda_b(a_i))=\lambda_{\varphi(b\circ K_{i,j_{i}})}(a_{\alpha(0,i,j_i)})$.
	Since $Y$ generates the multiplicative group of $B$, by (\ref{F})
	one can see that
	$$\varphi(b\circ K_{i,j})\circ L_{\alpha(0,i,j_i),\beta(0,i,j)}=F(b\circ K_{i,j})=\psi(b)\circ \varphi(K_{i,j})\circ L_{\alpha(0,i,j_i),\beta(0,i,j)}.$$
	Therefore, since $L_{\alpha(0,i,j_i),\beta(0,i,j)}\subseteq
	\St(a_{\alpha(0,i,j_i)})$, we have
	$$\lambda_{\varphi(b\circ K_{i,j})}(a_{\alpha(0,i,j_i)})=\lambda_{\psi(b)\varphi(K_{i,j})}(a_{\alpha(0,i,j_i)}).$$
	Hence
	$\psi(\lambda_b(a_i))=\lambda_{\psi(b)\varphi(K_{i,j_i})}(a_{\alpha(0,i,j_i)})$.
	Now we have that
	\begin{eqnarray*}
		\psi(b+a_i)&=&\psi(b\circ \lambda_{b'}(a_i))=\psi(b)\circ \psi(\lambda_{b'}(a_i))\\
		&=&\psi(b)\lambda_{\psi(b)'\circ \varphi(K_{i,j_i})}(a_{\alpha(0,i,j_i)})\\
		&=&\psi(b)\circ \lambda_{\psi(b)'}(\lambda_{\varphi(K_{i,j_i})}(a_{\alpha(0,i,j_i)}))\\
		&=&\psi(b)+\lambda_{\varphi(K_{i,j_i})}(a_{\alpha(0,i,j_i)})\\
		&=&\psi(b)+\psi(\lambda_{0}(a_i))=\psi(b)+\psi(a_i).
	\end{eqnarray*}
	Now it is easy to see that $\psi$ is a homomorphism of  skew braces.
	Since $F$ is bijective and $F(b\circ c\circ K_{i,j})=\psi(b)\circ F(c\circ K_{i,j})$, it
	follows that $\psi$ is bijective. Furthermore $b\in K_{i,j}$ if and
	only if
	\begin{eqnarray*}
		\varphi(K_{i,j})\circ L_{\alpha(0,i,j_i),\beta(0,i,j)}&=&F(K_{i,j})=F(b\circ K_{i,j})=\psi(b)\circ F(K_{i,j})\\
		&=&\psi(b)\circ \varphi(K_{i,j})\circ L_{\alpha(0,i,j_i),\beta(0,i,j)}.
	\end{eqnarray*}
	Therefore the result follows.
\end{proof}

Summarizing, the last theorem says that two solutions constructed as
in Theorem~\ref{main} are isomorphic if we can find an automorphism
of the brace $B$ that brings each $K_{i,j}$ to one $L_{i',j'}$,
taking into account that maybe the $L_{i',j'}$'s are permuted (that
is the reason for the $\alpha$ and $\beta_i$ maps), and that maybe
we have chosen another element of the orbit in the process (that is
the reason why the image $a_i$ is $\lambda_{z_{i,j}}(a_{\alpha(i)})$
and not just $a_{\alpha(i)}$, and it is also the reason why the
$L_{\alpha(i),\beta_i(j)}$ is conjugated by $z_{i,j}$).


The following is an example of how to use  Theorem~\ref{main} and
Theorem~\ref{isomorphism} to compute all the finite solutions
associated to a given finite left brace up to isomorphism. We use
the easiest examples of skew braces of abelian type: trivial skew braces of order $p$, where
$p$ is a prime. 
for all $a,b\in B$.

\begin{example}\label{bracesp}
	{\rm Consider the trivial skew brace  over $G=\Z/(p)$. Then, the orbits
		are $\{\alpha\}$ for every $\alpha\in\Z/(p)$.
		Since any orbit has  only one element,  we have $\St(\alpha)=G$, and the possible $K_{i,j}$'s in this case are $0$ and $G$.
		
		Let $Y$ be a subset of $\Z/(p)$ with at least a nonzero element.
		Let $K_{\alpha,j}=G$ for $\alpha\in Y$ and $j\in\{1,\dots,k_{\alpha}\}$, and let $K'_{\alpha,k}=0$ for
		$\alpha\in Y$ and $k\in\{1,\dots,m_\alpha\}$, where $k_{\alpha}$ and $m_{\alpha}$ are non-negative integers such that $k_{\alpha}+m_{\alpha}>0$.
		Write $G/K_{\alpha,j}=\{y_{\alpha,j}\}$, and
		$G/K'_{\alpha,k}=\{y^1_{\alpha,k},\dots,y^p_{\alpha,k}\}$, where
		$y^{l}_{\alpha,k}=l+K'_{\alpha,k}$. Assume that at least one $m_{\alpha}$ is positive. Then the corresponding solution
		of the YBE is $(X,r)$, where
		$$
		X=\bigcup_{\alpha\in Y}\left( \left(\bigcup_{1\leq j\leq
			k_{\alpha}}\{ y_{\alpha,j}\}\right)\cup \left(\bigcup_{1\leq k\leq
			m_{\alpha}} \{y^1_{\alpha,k},\dots ,y^p_{\alpha,k}\}\right) \right)
		$$
		and $r(x,y)=(\sigma_x(y),\sigma^{-1}_{\sigma_{x}(y)}(x))$, with the
		sigma maps given by
		$$
		\sigma_{y_{\alpha,j}}=\sigma_{y^l_{\alpha,k}}=\tau^\alpha\text{, for
			all } \alpha\in Y,\text{ for all } j,k \text{ and for all }
		l\in\{1,\dots,p\},
		$$
		where $\tau\in \Sym_X$ is the product of all the cycles
		$(y^1_{\alpha,k},y^2_{\alpha,k},\dots,y^p_{\alpha,k})$ for any
		$\alpha\in Y$ and $k\in\{1,\dots,m_\alpha\}$.
		
		Finally observe that, in this case, $\Aut(G,+,\circ)=\Aut(G,+)\cong
		(\Z/(p))^*$, and the effect of an automorphism of $G$ over a
		solution is to change
		$\sigma_{y_{\alpha,j}}=\sigma_{y^l_{\alpha,k}}=\tau^\alpha$ to the
		isomorphic solution
		$\sigma_{y_{\alpha,j}}=\sigma_{y^l_{\alpha,k}}=\tau^{A\alpha}$,
		where $A\in(\Z/(p))^*$.}
\end{example}






\section{Exercises}

\begin{prob} Let $p$ and $q$ be two distinct prime numbers. Describe all the finite involutive solutions $(X,r)$ to the YBE such that $\mathcal{G}(X,r)$ is the trivial skew brace over the cyclic group $\Z/(pq)$.
\end{prob}

\begin{prob}
Let $p$ be a prime number.
Give an example of a finite non-involutive solution $(X,r)$ to the YBE such that $\mathcal{G}(X,r)$ is the trivial skew brace over the cyclic group $\Z/(p)$.
\end{prob}




\section{Notes}
Proposition \ref{extensions} is due to Bachiller (see \cite[Theorem~2.1]{MR3320237}).

A
cohomological version of Proposition \ref{BenDavid} appears in \cite[Theorem~4.6]{MR3500774} 
and a first version without a formal statement in \cite[pages
182--183]{MR1722951}. 

Sections \ref{construcsec3} and \ref{construciso} are based on \cite{MR3527540}.

In \cite{MR3835326} Bachiller generalized the results of this chapter to general solutions to the YBE. Hence the construction of all (finite) solutions to the YBE is reduced to the construction of all (finite) skew braces.