\chapter{Skew braces}
\label{braces}

% Braces were introduced by Rump in~\cite{MR2278047} to study set-theoretical involutive
% solutions of the Yang--Baxter equation. The following definition of~\cite{MR3647970} 
% generalizes braces to the non-involutive setting.
\section{Basic definitions}
%Skew braces and the Yang--Baxter equation}

By convention, an additive group $A$ will be a (not necessarily abelian) group 
with a binary operation $(a,b)\mapsto a+b$. The 
identity of $A$ will be denoted by $0$ and the inverse of an element $a$ will be denoted by $-a$. 

\begin{definition}
    \label{def:brace}
	\index{Skew brace}
	\index{Skew brace!multiplicative group}
	\index{Skew brace!additive group}
	A \emph{skew left brace} is a triple $(A,+,\circ)$, where $(A,+)$ and $(A,\circ)$ 
	are (not necessarily abelian) 
	groups and 
	\begin{equation}
	    \label{eq:compatibility}
	    a\circ(b+c)=(a\circ b)-a+(a\circ c)
	\end{equation}
	holds for all $a,b,c\in A$. The groups 
	$(A,+)$ and $(A,\circ)$ are respectively 
	the \emph{additive} and \emph{multiplicative} group
	of the skew left brace $A$.
\end{definition}

As usual, multiplication is granted a higher precedence than addition. This means that one can write
\eqref{eq:compatibility} simply as 
$a\circ(b+c)=a\circ b-a+a\circ c$. 
We write $a'$ to denote the inverse of $a$ with respect to the circle operation $\circ$. 

Skew right braces are defined similarly, one needs 
to replace~\eqref{eq:compatibility} by 
\[
(a+b)\circ c=a\circ c-c+b\circ c.
\]
There is a bijective correspondence between skew left braces and skew right braces, 
see Exercise~\ref{prob:left_right}. For that reason, 
a skew brace will always mean a skew left brace. 

\index{Skew brace!of $\mathcal{X}$-type}
\index{Skew brace!of abelian type}
\begin{definition}
    Let $\mathcal{X}$ be a family of groups. A skew brace $A$ is said to be
    of $\mathcal{X}$-type if its additive group belongs to $\mathcal{X}$.
\end{definition}

One particularly interesting family of skew braces is the family of \emph{skew braces of abelian type}, 
that is skew braces with abelian additive group. 
Skew braces of abelian type were introduced by Rump in~\cite{MR2278047} to study involutive solutions to the Yang--Baxter equation. 
\index{Left brace}
In the literature, skew braces of abelian type are called \emph{left braces}. 

\begin{example}
	\label{exa:trivial}
	\index{Skew brace!trivial}
	Let $A$ be an additive group. Then $A$ is a skew brace with
	$a\circ b=a+b$ for all $a,b\in A$. 
	A skew brace $(A,+,\circ)$ such that $a\circ b=a+b$ for all $a,b\in A$ is
    said to be \emph{trivial}. 
	Similarly, the
   operation $a\circ b=b+a$ turns $A$ into a skew brace. 
\end{example}

\begin{example}
	\label{exa:times}
	\index{Direct product!of skew braces}
	Let $A$ and $B$ be skew braces. Then $A\times B$ with 
	\[
		(a,b)+(a_1,b_1)=(a+a_1,b+b_1),\quad
		(a,b)\circ (a_1,b_1)=(a\circ a_1,b\circ b_1),
	\]
	is a skew brace. This is the {\em direct product} of the skew braces $A$ and $B$. 
\end{example}

\begin{example}
	\label{exa:sd}
	Let $A$ and $M$ be additive groups and let $\alpha\colon A\to\Aut(M)$ be a
	group homomorphism. Then $M\times A$ with 
	\[
	(x,a)+(y,b)=(x+y,a+b),
	\quad
	(x,a)\circ (y,b)=(x+\alpha_a(y),a+b)
	\]
	is a skew brace. Similarly, $M\times A$ with
	\[
	(x,a)+(y,b)=(x+\alpha_a(y),a+b),\quad
	(x,a)\circ (y,b)=(x+y,b+a)
	\]
	is a skew brace. 
\end{example}

\begin{example}
  \label{exa:s3c6}
Let $A$ be the symmetric
  group $\Sym_3$ written additively. For example,
  $(12)+(13)=(132)$.
  Let $\lambda\colon A\to\Aut(A)$ 
  be the map given by
  \begin{align*}
    &\lambda_{\id}=\lambda_{(123)}=\lambda_{(132)}=\id,\\
    &\lambda_{(12)}=\lambda_{(23)}=\lambda_{(13)}=\text{conjugation by $(23)$}.
  \end{align*}
  The operation $a\circ b=a+\lambda_a(b)$ 
  turns $A$ into a skew brace with
  additive group $\Sym_3$ and 
  multiplicative group cyclic of order six. The multiplicative
  group of $A$ is generated by $(12)$. 
\end{example}



\begin{example}
    \label{exa:WX}
   Let $A$ be an additive group
	and $B$ and $C$ be subgroups of $A$ such that $B\cap C=\{ 0\}$ and $A=B+C$. In this case, one says that $A$ admits an {\em exact factorization} through the subgroups $B$ and $C$.  Thus each $a\in A$ can be written in a unique
	way as $a=b+c$, for some $b\in B$ and $c\in C$.  The map
    \[
        f\colon B\times C^{\mathrm{op}}\to A,\quad
		(b,c)\mapsto b+c,
	\]
	is bijective. Since $f$ is bijective, 
	$A$ is a group with the operation
	\begin{equation}
	\label{eq:factorization}
	a\circ a_1=f\left(f^{-1}(a)f^{-1}(a_1)\right).
	\end{equation}
	To compute \eqref{eq:factorization} write $a=b+c$ and 
	$a_1=b_1+c_1$ for $b,b_1\in B$ and $c,c_1\in C$. 
	Then 
	\begin{align*}
	a\circ a_1&=f\left((b,c)(b_1,c_1)\right)\\
	&=f\left((b+b_1,c_1+c)\right)
	=b+(b_1+c_1)+c=b+a_1+c.
	\end{align*}
	It follows that $(A,\circ)$ is a group 
	isomorphic to $B\times C^{\mathrm{op}}\textcolor{red}{\cong} B\times C$. Moreover, if $x,y\in A$, 
	then 
	\begin{align*}
	a\circ x-a+a\circ y=b+x+c-(b+c)+b+y+c=b+x+y+c=a\circ (x+y),
	\end{align*}
	and therefore $(A,+,\circ)$ is a skew brace. 
\end{example}

% \begin{proof} The map $\eta\colon B\times C\to A$, $\eta(b,c)=bc^{-1}$, is
%   bijective.  Since $\eta$ is bijective and $a\circ
%   a'=\eta(\eta^{-1}(a)\eta^{-1}(a'))$, it follows that $(A,\circ)$ is a group
%   isomorphic to the direct product $B\times C$. To prove that $A$ is a skew
%   brace it remains to show~\eqref{eq:compatibility}. Let $a=bc\in BC$ and
%   $a',a''\in A$. Then \begin{align*} (a\circ a')a^{-1}(a\circ a'')
%     &=(ba'c)a^{-1}(ba''c)\\ &=ba'c(c^{-1}b^{-1})ba''c\\ &=ba'a''c\\ &=a\circ
%     (a'a'').  \end{align*} This completes the proof.  \end{proof}

We now give concrete some examples of the previous construction. 

\begin{example}
  \label{exa:QR}
  Let $n$ be a positive integer.  For  
  \begin{gather*}
  A=\begin{pmatrix}
  a_{1,1}&a_{1,2}&\cdots &a_{1,n}\\
  a_{2,1}&a_{2,2}&\cdots&a_{2,n}\\
  \vdots&\vdots&\ddots & \vdots\\
  a_{n,1}&a_{n,2}&\cdots&a_{n,n}
  \end{pmatrix}
  \shortintertext{let}
  A^*=\bar{A}^t=
  \begin{pmatrix}
  \bar a_{1,1}&\bar a_{2,1}&\cdots &\bar a_{n,1}\\
  \bar a_{1,2}&\bar a_{2,2}&\cdots &\bar a_{n,2}\\
  \vdots&\vdots&\ddots & \vdots\\
  \bar a_{n,1}&\bar a_{2,n}&\cdots &\bar a_{n,n}
  \end{pmatrix},
  \end{gather*}
  where $\overline{a+bi}=a-bi$ for all $a,b\in\R$.
  The group $\GL_n(\C)$ admits an
  exact factorization through the subgroups $U(n)$ and $T(n)$, where $U(n)=\{ A\in\GL_n(\C): AA^*=I\}$
  is the unitary group and $T(n)$ is the group of upper triangular matrices
  with positive diagonal entries.  
  Therefore there exists a skew brace with additive group
  isomorphic to $\GL_n(\C)$ and multiplicative group isomorphic to $U(n)\times T(n)$.  
\end{example}

The following examples appeared in the theory of 
Hopf--Galois structures, see~\cite[Corollary 1.1]{MR3425626}.

\begin{example} 
	\label{exa:a5a4c5}
	The alternating simple group $\Alt_5$ admits an exact factorization
  through the subgroups 
  $A=\langle (123),(12)(34)\rangle\textcolor{red}{\cong}\Alt_4$ and 
  $B=\langle(12345)\rangle\textcolor{red}{\cong} C_5$.  
  There exists a skew brace with additive group isomorphic to $\Alt_5$ and multiplicative
  group isomorphic to $\Alt_4\times C_5$. 
\end{example}

\begin{example} 
	\label{exa:PSL27S4C7}
  The simple group $\PSL_2(7)=\SL_2(7)/Z(\SL_2(7))$ admits an exact factorization through the subgroups 
  \[
  A=\left\langle\left(\begin{array}{cc}
  0&1\\
  -1&-1\end{array}\right)Z(\SL_2(7)),\left(\begin{array}{cc}
  1&-2\\
  1&-1\end{array}\right)Z(\SL_2(7))\right\rangle\textcolor{red}{\cong}\Sym_4\] and \[B=\left\langle\left(\begin{array}{cc}
  1&1\\
  0&1\end{array}\right)Z(\SL_2(7))\right\rangle\textcolor{red}{\cong} C_7.
  \]
  Hence there exists a skew brace with additive
  group isomorphic to $\PSL_2(7)$ and multiplicative group isomorphic to 
  $\Sym_4\times C_7$.  
\end{example}

\begin{lemma}
    \label{lem:basic}
	Let $A$ be a skew brace. Then the following properties hold:
    \begin{enumerate}
        \item $0=1$.  
        \item $a\circ(-b+c)=a-(a\circ b)+(a\circ c)$, for all $a,b,c\in A$.
        \item $a\circ(b-c)=(a\circ b)-(a\circ c)+a$, for all $a,b,c\in A$.
    \end{enumerate}
\end{lemma}

\begin{proof}
		The first claim follows from the compatibility condition~\eqref{eq:compatibility} with
		$c=1$.  To prove the second claim let $d=b+c$.
		Then~\eqref{eq:compatibility} becomes 
		\[
		a\circ d =a\circ b-a+a\circ (-b+d)
		\]
		and the claim follows. The third claim is
		proved similarly.
\end{proof}

\begin{proposition}
\label{pro:lambda}
    Let $A$ be a skew brace. For each $a\in A$, the map
    \[
        \lambda_a\colon A\to A,\quad
        b\mapsto -a+(a\circ b),
    \]
    is an automorphism of $(A,+)$. Moreover, the map 
    $\lambda\colon (A,\circ)\to\Aut(A,+)$, $a\mapsto\lambda_a$, is a group homomorphism. 
\end{proposition}

\begin{proof}
The inverse of $\lambda_a$ is given by $\lambda^{-1}_a\colon A\to A$, $b\mapsto a'\circ (a+b)$. To prove
that $\lambda_a\in\Aut(A,+)$ we note that
\[
\lambda_a(b+c)=-a+a\circ(b+c)=-a+a\circ b-a+a\circ c=\lambda_a(b)+\lambda_a(c).
\]
Note that $\lambda_a(b)=-a+a\circ b=a\circ (a'+b)$, for all $a,b\in A$. Hence 
%To prove that $\lambda$ is a group homomorphism, we 
%use Lemma~\ref{lem:basic} to obtain
\begin{align*}
\lambda_a(\lambda_b(c))&=a\circ (a'+b\circ (b'+c))=-a+a\circ b\circ (b'+c)\\
&=-a+a-a\circ b+a\circ b\circ c=-a\circ b+a\circ b\circ c=\lambda_{a\circ b}(c).\qedhere
\end{align*}
\end{proof}

If $A$ is a skew brace, then 
the map $\lambda$ in the previous proposition yields a left action from $(A,\circ)$ on $(A,+)$ by automorphisms. 

\index{Commutator identities!for braces}
The following formulas 
\begin{align}
\label{eq:formulas}
&a\circ b = a+\lambda_a(b),
&&a+b=a\circ \lambda^{-1}_a(b),
&&\lambda_a(a')=-a
\end{align}
hold for $a,b\in A$. Moreover, if 
\[
    a*b=\lambda_a(b)-b=-a+a\circ b-b,
\]
then the following identities are easily verified:
\begin{align}
\label{eq:commutator1}&a*(b+c)=a*b+b+a*c-b,\\
\label{eq:commutator2}&(a\circ b)*c=(a*(b*c))+b*c+a*c.
\end{align}
These last two identities are similar to the usual
\textcolor{red}{commutator identities \ref{eq:comident1} and \ref{eq:comident2}.}

There is also a right action of $(A,\circ)$ on the set $A$.


\begin{proposition}
\label{pro:mu}
    Let $A$ be a skew brace. For each $a\in A$, the map
    \[
        \mu_a\colon A\to A,\quad
        b\mapsto \lambda_b(a)'\circ b\circ a,
    \]
    is bijective. Moreover, the map 
    $\mu\colon (A,\circ)\to\Sym_A$, $a\mapsto\mu_a$, satisfies $\mu_b\circ\mu_a=\mu_{a\circ b}$, for all $a,b\in A$. 
\end{proposition}

\begin{proof}
    Note that
    \[\mu_a(b)=\lambda_b(a)'\circ b\circ a=(b\circ (b'+a))'\circ b\circ a=(b'+a)'\circ a,\]
    for all $a,b\in A$. Hence $\mu_a$ is bijective and
    \[\mu_a^{-1}(b)=((b\circ a')'-a)'=(a\circ b'-a)'=(b'+a')'\circ a',\]
    for all $a,b\in A$. Now we have
    \begin{align*}
        \mu_b(\mu_a(c))&=\mu_b((c'+a)'\circ a)=(a'\circ (c'+a)+b)'\circ b\\
        &=(a'\circ c'-a'+b)'\circ b=(a'\circ (c'+a\circ b))'\circ b\\
        &=(c'+a\circ b)'\circ a\circ b=\mu_{a\circ b}(c),
    \end{align*}
    for all $a,b,c\in A$. Therefore the result follows.
%    Let $a,b,c\in A$. To prove that $\mu$ is a group anti-homorphism, we compute
%    \begin{align*}
%    &\mu_{b\circ a}(c)=\lambda_c(b\circ a)'\circ c\circ b\circ a
%    \shortintertext{and}
%    &\mu_a\mu_b(c)=\mu_a(\lambda_c(b)'\circ c\circ b)
%    =\lambda_{\lambda_c(b)'\circ c\circ b}(a)'\circ \lambda_c(b)'\circ c\circ b\circ a.
%    \end{align*}
%    Using the formulas~\eqref{eq:formulas} and Proposition \ref{pro:lambda}, 
%    \begin{align*}
%    \lambda_c(b\circ a)'&=\lambda_c(b+\lambda_b(a))'=(\lambda_c(b)+\lambda_{c\circ b}(a))'\\
%    &=(\lambda_c(b)\circ %\lambda^{-1}_{\lambda_c(b)}\lambda_{c\circ b}(a))'
%    =\lambda_{\lambda_c(b)'\circ c\circ b}(a)'\circ \lambda_c(b)',
%    \end{align*}
%    which proves that $\mu$ is an anti-homomorphism. 
%    
%    To compute the inverse of $\mu_b$ we proceed as follows. Since $a'\circ (-a)=2a$ by Lemma~\ref{lem:basic}, 
%    \begin{align*}
%    (\lambda_a(b)'\circ a\circ b)'&=b'\circ (a'\circ \lambda_a(b))\\
%    &=b'\circ (a'\circ (-a+a\circ b))
%    =b'\circ (a'+b)=b'\circ a'-b.
%    \end{align*}
%    From this one immediately obtains that $\mu_b^{-1}(a)=(b\circ a'-b)'$. 
\end{proof}

 \begin{definition}
 	\index{Homomorphism!of skew braces}
 	A map $f\colon A\to B$ between two skew braces $A$ and $B$ is a {\em homomorphism of skew braces} if $f(x\circ y)=f(x)\circ f(y)$ and $f(x+y)=f(x)+f(y)$, for all $x,y\in A$.  The \emph{kernel} of $f$ is
     \[
         \ker f=\{a\in A:f(a)=0\}.
     \]
 \end{definition}

\index{Isomorphism!of skew braces}
We write $\Hom_b(A,B)$ to denote the set
of skew brace homomorphisms $A\to B$.
A bijective homomorphism of skew braces is an isomorphism. An automorphism of a skew brace $A$ is an isomorphism from the skew brace $A$ to it self. 
We write $\Aut_b(A)$ to denote the group of skew brace 
automorphisms of $A$.
Two skew braces $A$ and $B$ are isomorphic if there exist an isomorphism $f\colon A\rightarrow B$. We write \textcolor{red}{$A\cong B$} to denote that the skew braces $A$ and $B$ are isomorphic.

\begin{proposition}
\label{prop:semidirect} 
\index{Skew brace!semidirect product} Let $A,B$ be skew braces. Let $\alpha\colon (B,\circ)\rightarrow \Aut_b(A)$ be a homomorphism of groups. Define two operations on $A\times B$ by
\[ (a_1,b_1)+(a_2,b_2)=(a_1+a_2,b_1+b_2)\text{ and }(a_1,b_1)\circ (a_2,b_2)=(a_1\circ\alpha_{b_1}(a_2),b_1\circ b_2),\]
where $\alpha_{b_1}=\alpha(b_1)$, for all $a_1,a_2\in A$ and $b_1,b_2\in B$. Then $(A\times B,+,\circ)$ is a skew brace. This skew brace is the {\em semidirect product} of the skew brace $A$ by $B$ via $\alpha$, and it is denoted by $A\rtimes_{\alpha}B$.
\end{proposition}

\begin{proof}
    Note that $(A\times B,+)$ is the direct product of the groups $(A,+)$ and $(B,+)$, and $(A\times B,\circ)$ is the semidirect product of the group $(A,\circ)$ by $(B,\circ)$ via $\alpha$. Hence it is enough to check the compatibility condition.
    Let $a_1,a_2,a_3\in A$ and $b_1,b_2,b_3\in B$. We have
    \begin{align*}
        (a_1,b_1)\circ &((a_2,b_2)\textcolor{red}{+}(a_3,b_3))\\
        =&(a_1,b_1)\circ (a_2+a_3,b_2+b_3)\\
        =&(a_1\circ \alpha_{b_1}(a_2+a_3),b_1\circ (b_2+b_3))\\
        =&(a_1\circ (\alpha_{b_1}(a_2)+\alpha_{b_1}(a_3)),b_1\circ b_2-b_1+b_1\circ b_3)\\
        =&(a_1\circ \alpha_{b_1}(a_2)-a_1+a_1\circ\alpha_{b_1}(a_3)),b_1\circ b_2-b_1+b_1\circ b_3)\\
        =&(a_1\circ \alpha_{b_1}(a_2),b_1\circ b_2)-(a_1,b_1)+(a_1\circ\alpha_{b_1}(a_3)),b_1\circ b_3)\\
        =&(a_1,b_1)\circ (a_2,b_2)-(a_1,b_1)+(a_1,b_1)\circ (a_2,b_3).
    \end{align*}
Therefore the result follows.
\end{proof}
%Braces and brace homomorphisms form a category.  

%\begin{example}
%        \label{exa:trivial}
%        Let $(A,\cdot)$ be a group. Then $A$ is a skew left brace with
%        $a\circ b=ab$ for all $a,b\in A$.  Similarly, $a\star b=ba$ defines a
%        skew left brace structure over $A$. These braces are isomorphic
%        if and only if $(A,\cdot)$ is abelian.
%\end{example}
%
%\begin{example}
%	\label{exa:sd}
%    Let $A$ and $B$ be groups and let $\alpha\colon A\to\Aut(B)$ be a
%    group homomorphism. Assume that $A$ is abelian. Then $A\times B$ has a
%    skew left brace structure given by
%    \begin{align*}
%        &(a,b)(a',b')=(aa',b\alpha_a(b')),\\
%        &(a,b)\circ(a',b')=(aa',bb'),
%    \end{align*}
%    where $a,a'\in A$ and $b,b'\in B$. 
%\end{example}
%
%\begin{example}
%    \label{exa:WX}
%    This example is motivated by the paper of Weinstein and
%	Xu on the Yang--Baxter equation, see~\cite{MR1178147}. Let $A$ be a group
%	and $A_+,A_-$ be subgroups of $A$ such that $A$ admits a \emph{unique
%	factorization} as $A=A_+A_-$. Thus each $a\in A$ can be written in a unique
%	way as $a=a_+a_-$ for some $a_+\in A_+$ and $a_-\in A_-$.  The map
%	\[
%		A_+\times A_-\to A,\quad
%		(a_+,a_-{)}\mapsto a_+(a_{-})^{-1},
%	\]
%	is bijective. Using this map we transport the group structure of the direct
%	product $A_+\times A_-$ into the set $A$. For $a=a_+a_-\in A$ and
%	$b=b_+b_-\in A$ let 
%	\begin{align*}
%		a\circ b&=a_+ba_-.
%	\end{align*}
%	Then $(A,\circ)$ is a group. Furthermore, $A$ is a
%	skew left brace.
%\end{example}



% \begin{exa}
%   \label{exa:trivial}
%   Let $A$ be a group. Then $a\circ b=ab$ gives a skew brace. Similarly, the
%   operation $a\circ b=ba$ turns $A$ into a skew brace. 
% \end{exa}



% Lemma~\ref{lem:lambda} justifies the following definition:

% \begin{defn}
% 	\label{def:structure}
% 	Let $A$ be a skew brace. The \emph{crossed group} of $A$ is defined as
% 	the group $\Gamma(A)=(A,\cdot)\rtimes(A,\circ)$ with multiplication 
%     \[
%     (a,x)(b,y)=(a\lambda_x(b),x\circ y).
%     \]
% \end{defn}

% The following two propositions were proved by Bachiller in the case of braces of abelian type, 
% see \cite[Lemma 2.4]{Bachiller3} and~\cite[Proposition 2.3]{MR3465351}.

% \begin{proposition}
% 	\label{lem:mu}
% 	Let $A$ be a skew brace and let 
% 	\[
% 	\mu\colon\Mul{A}\to\Sym_A,\quad
% 	\mu_b(a)=\overline{\lambda_a(b)}\circ a\circ b. 
% 	\]
% 	Then 
% 	$\mu_1=\id$ and $\mu_{a\circ b}=\mu_b\mu_a$ for all $a,b\in A$.
% \end{proposition}

% \begin{proof}
    
% \end{proof}

% \begin{proposition}
% \label{lem:Bachiller}
% Let $A$ be a group and $\lambda\colon A\to\Aut(A)$ be a map such that
% \begin{equation}
% \label{eq:lambda}
% \lambda_{a\lambda_a(b)}=\lambda_a\lambda_b,\quad a,b\in A.
% \end{equation}
% Then $A$ with $a\circ b=a\lambda_a(b)$ is a skew brace. 
% \end{proposition}

% %\begin{lem}
% %	The following statement hold:
% %	\begin{enumerate}
% %		\item Let $A$ be a skew brace. The map 
% %			\[
% %			\lambda\colon\Mul{A}\to\Aut\Add{A},\quad
% %			\lambda_a(b)=a^{-1}(a\circ b),
% %			\]
% %			is a group homomorphism.
% %		\item Let $A$ be a group and $\lambda\colon A\to\Aut(A)$ be a map such that
% %			\begin{equation}
% %				\label{eq:lambda}
% %				\lambda_{a\lambda_a(b)}=\lambda_a\lambda_b,\quad a,b\in A.
% %			\end{equation}
% %			Then $A$ with $a\circ b=a\lambda_a(b)$ is a skew brace. 
% %	\end{enumerate}
% %\end{lem}

% \begin{proof}
% 	The first claim is~\cite[Corollary 1.10]{MR3647970}. For the second claim see~\cite[Lemma 1.1.17]{BachillerTESIS}.
% \end{proof}

% The following lemma provides another useful tool for constructing skew braces.

% \begin{proposition}
% 	\label{lem:dual}
% 	Let $\Mul{A}$ be a group and $\lambda\colon A\to\Sym_A$ be a group
% 	homomorphism.  Assume that 
% 	$\lambda_a(1)=1$ for all $a\in A$ and that
% 	\begin{equation}
% 		\label{eq:dual}
% 		\lambda_a(b\circ\lambda^{-1}_b(c))=\lambda_a(b)\circ\lambda^{-1}_{\lambda_a(b)}\lambda_a(c)
% 	\end{equation}
% 	for all $a,b,c\in A$. Then $A$  with 
% 	$ab=a\circ\lambda^{-1}_a(b)$ is a skew brace. 
% \end{proposition}

% \begin{proof}
% 	Note that Equation~\eqref{eq:dual} is equivalent to
% 	\begin{equation}
% 		\label{eq:better}
% 		\lambda_a^{-1}(bc)=\lambda_a^{-1}(b)\lambda_a^{-1}(c).
% 	\end{equation}
% 	We prove that the operation is associative:
% 	\begin{align*}
% 		a(bc) &= a\circ\lambda^{-1}_a(bc)
% 		=a\circ(\lambda^{-1}_a(b)\lambda^{-1}_a(c))\\
% 		&=a\circ\lambda^{-1}_a(b)\circ \lambda^{-1}_{a\circ \lambda^{-1}_a(b)}(c)
% 		=(ab)\circ\lambda^{-1}_{ab}(c)=(ab)c.
% 	\end{align*}

% 	The neutral element $1$ of $A$ is a right identity: $a1=a\circ\lambda^{-1}_a(1)=a\circ 1=a$. 
% 	The element $a^{-1}=\lambda_a(\overline{a})$ is a right inverse of $A$ since 
% 	\[
% 	aa^{-1}=a\circ\lambda^{-1}_a(a^{-1})=a\circ\lambda^{-1}_a\lambda_a(\overline{a})=a\circ\overline{a}=1.
% 	\]
% 	Therefore $\Add{A}$ is a group by~\cite[\S1.1.2]{MR1357169}. 

% 	The brace compatibility condition follows from Equation~\eqref{eq:better}:
% 	\begin{align*}
% 		&(a\circ b)a^{-1}(a\circ c)=(a\circ b)\lambda_a(c)=a\lambda_a(b)\lambda_a(c)=a\lambda_a(bc)=a\circ (bc).
% 	\end{align*}
% 	The lemma is proved. 
% \end{proof}

%\begin{exa}
%	Let $A_1,\dots,A_n$ be groups. Assume that 
%	for each $i\in\{2,\dots,n-1\}$ there is a group homomorphism 
%	$\rho_{i}\colon A_{i-1}\ltimes_{\rho_{i-1}} A_i\to\Aut(A_{i+1})$. 
%	Then the set $A_1\times\cdots\times A_n$ with
%	additive structure given by
%	\[
%	(a_1,\dots,a_n)\cdot(a_1',\dots,a_n')=(a_1a_1',\dots,a_n a_n')
%	\]
%	and multiplicative structure given by 
%	\begin{multline*}
%		(a_1,\dots,a_n)\circ (a_1',\dots,a_n')\\
%		=(a_1a_1',a_2\rho_1(a_1)(a_2'),a_3\rho_2(a_1,a_2)(a_3'),\dots,a_n\rho_{n-1}(a_1,\dots,a_{n-1})(a_n'))    
%	\end{multline*}
%	is a skew brace. This brace will be denoted by
%	\[
%	A_1\ltimes_{\rho_1} A_2\ltimes_{\rho_2}A_3\ltimes_{\rho_3}\cdots\ltimes_{\rho_{n-1}}A_n.
%	\]
%\end{exa}

\begin{lemma}\label{lem:homlambda}
    Let $A$ and $B$ be skew braces. A map $f\colon A\to B$ is a homomorphism of skew braces if and only if $f(a\circ b)=f(a)\circ f(b)$ and $f(\lambda_a(b))=\lambda_{f(a)}(f(b))$, for all $a,b\in A$.
\end{lemma}

\begin{proof}
    Let $a,b\in A$. Suppose that $f$ is a homomorphism of skew braces. Then
    \[ f(\lambda_a(b))=f(-a+a\circ b)=-f(a)+f(a)\circ f(b)=\lambda_{f(a)}(f(b)).\]
    Conversely, suppose that $f(x\circ y)=f(x)\circ f(y)$ and $f(\lambda_x(y))=\lambda_{f(x)}(f(y))$, for all $x,y\in A$. We have that 
    \[f(a+b)=f(a\circ\lambda_{a'}(b))=f(a)\circ f(\lambda_{a'}(b))=f(a)\circ \lambda_{f(a)'}(f(b))=f(a)+f(b).\]
    Therefore the result follows. 
\end{proof}



\begin{definition}
    \index{Skew brace!two sided}
	A skew brace $A$ is said to be a \emph{skew two-sided brace} if 
	\begin{equation}
	\label{eq:right_compatibility}
	(a+b)\circ c=a\circ c-c+b\circ c
	\end{equation}
	holds for all $a,b,c\in A$. 
\end{definition}

If $A$ is a skew two-sided brace, then 
\begin{align}
\label{eq:2sided}
&a\circ(-b)=a-a\circ b+a,
&&(-a)\circ b=b-a\circ b+b    
\end{align}
hold for all $a,b\in A$. The first equality holds for every skew brace and follows from Lemma~\ref{lem:basic}. 
The second equality follows from~\eqref{eq:right_compatibility}. 

\begin{example}
  Any skew brace with abelian multiplicative group is 
  two-sided.
\end{example}

\begin{example}
  Let $n\in\N$ be such that $n=p_1^{a_1}\cdots p_k^{a_k}$, where the $p_j$ are
  distinct primes, all $a_j\in\{1,2\}$ and $p_i^m\not\equiv 1\bmod{p_j}$ for
  all $i,j,m$ with $1\leq m\leq a_i$. Then every brace of size $n$ is a skew
  two-sided brace of abelian type, since every group of order $n$ is abelian, see for
  example~\cite{MR1786236}.  
\end{example}

\index{Jacobson!radical ring}
\index{Radical ring}
%Skew two-sided braces of abelian type form an interesting family of rings without unit. 
% Those rings, known as radical rings, 
% were introduced by Jacobson in~\cite{MR12271}. In any ring $R$ the \emph{circle operation} 
% \[
% x\circ y=x+xy+y
% \]
% is always associative 
% and such that $x\circ 0=0\circ x=x$ for all $x\in R$. A non-unital ring (or \emph{rng}, for short) 
% $R$ is said to be a \emph{radical ring} if $(R,\circ)$ is a group. 
% In this case, following Jacobson's notation, the inverse of an element $x$ with respect to the circle operation is denoted by $x'$. 
% There are other characterizations of radical rings, see for example~\cite{MR3308118}.

% \begin{example}
%     The ring 
%     $R=\begin{pmatrix}
%     0 & \R & \R\\
%     0 & 0 & \R\\
%     0 & 0 & 0
%     \end{pmatrix}$ 
%     is  radical ring. 
% \end{example}

% \begin{example}
%     The ring $R=\left\{\frac{2x}{2y+1}:x,y\in\Z\right\}$ is a radical ring. In fact, 
%     \[
%     \left(\frac{2x}{2y+1}\right)'=\frac{2(-x)}{2(x+y)+1}.
%     \]
% \end{example}

%Skew braces are a far reaching generalizations of radical rings. The following result was proved by Rump in~\cite{MR2278047}.

\begin{theorem}
\label{thm:radical}
    A skew brace of abelian type $(A,+,\circ)$ 
    is two-sided if and only if $(A,+,*)$ it is a radical ring.
\end{theorem}

\begin{proof}
    Assume first that $A$ is a skew two-sided brace of abelian type. Then $(A,+)$ is an abelian group. 
    Let us prove that the operation
    \[
    a*b=-a+a\circ b-b
    \]
    turns $A$ into a radical ring. Left distributivity follows from the compatibility condition and the fact that the addition is commutative:
    \begin{align*}
    a*(b+c)&=-a+a\circ (b+c)-(b+c)
    =-a+a\circ b-a+a\circ c-c-b=a*b+a*c.
    \end{align*}
    Similarly, since $A$ is two-sided, one proves $(a+b)*c=a*c+b*c$. It remains to show that the operation $*$
    is associative. On the one hand, using the first equality of~\eqref{eq:2sided} 
    and the compatibility condition, we write
    \begin{align*}
    a*(b*c)&=a*(-b+b\circ c-c)\\
    &=-a+a\circ(-b+b\circ c-c)-(-b+b\circ c-c)\\
    &=-a+a\circ (-b)-a+a\circ(b\circ c)-a+a\circ (-c)+c-b\circ c+b\\
    &=a\circ (b\circ c)-a\circ b-a\circ c-b\circ c+a+b+c,
    \end{align*}
    since the group $(A,+)$ is abelian. On the other hand, the second equality of~\eqref{eq:2sided} and
    Equality~\eqref{eq:right_compatibility} imply that
    \begin{align*}
    (a*b)*c &= (-a+a\circ b-b)*c=-(-a+a\circ b-b)+(-a+a\circ b-b)\circ c-c\\
    &=b-a\circ b+a+(-a)\circ c-c+(a\circ b)\circ c-c+(-b)\circ c-c\\
    &=(a\circ b)\circ c-a\circ b-a\circ c-b\circ c+a+b+c.
    \end{align*}
    It then follows that the operation $*$ is associative. 
    
    Conversely, if $A$ is a radical ring, say with ring multiplication $(a,b)\mapsto ab$, 
    then $a\circ b=a+ab+b$ turns $A$ into a skew two-sided brace 
    of abelian type. In fact, since $A$ is a radical ring, then 
    $(A,+)$ is an abelian group and $(A,\circ)$ is a group. Moreover, 
    \begin{align*}
        a\circ (b+c)=a+a(b+c)+(b+c)=a+ab+ac+b+c=a\circ b-a+a\circ c.
    \end{align*}
    Similarly ones proves $(a+b)\circ c=a\circ c-c+b\circ c$.
\end{proof}

\index{Skew brace!associative}
A skew brace is said to be \emph{associative} if the operation $(x,y)\mapsto
x*y=\lambda_x(y)-y$ is associative. In \cite[Question 2.1(2)]{MR3818285}, 
Cedó, Gateva--Ivanova and Smoktunowicz asked whether associative skew braces of abelian type are always radical rings. 
To answer this question, we need some lemmas. 

\begin{lemma}\label{lem:CGIS}
    If $A$ is an associative skew brace of abelian type, then 
    \[
    (-a)*b=-(a*b)
    \]
	holds for all $a,b\in A$. In particular, $(-a)\circ b=2b-(a\circ b)$ for all $a,b\in A$. 
\end{lemma}

\begin{proof}
    Since $A$ is of abelian type, 
    Identity \eqref{eq:commutator2} implies that
    \begin{align*}
 	    (a*(-a))*b &= 
 	    (a*(-a)+a+(-a))*b\\
 	    &=(a\circ (-a))*b\\
 	    &=a*( (-a)*b)+(-a)*b+a*b.
 	 \end{align*}
     The associativity of $A$ then implies that 
     $(-a)*b=-(a*b)$.
\end{proof}

If $A$ is a skew brace of abelian type, then one proves by induction that 
\begin{align}
\label{eq:Lau}
    a\circ \left(\sum_{i=1}^n b_i-\sum_{j=1}^mc_j\right)
    =\sum_{i=1}^n a\circ b_i-\sum_{j=1}^ma\circ c_j+(m-n+1)a
\end{align}
holds for all $a,b,c\in A$, see Exercise~\ref{prob:Lau}. 

% \begin{lemma}
% If $A$ is a brace of abelian type, then 
% % \[
% %     a*\left(\sum_{i=1}^nb_i-\sum_{j=1}^mc_j\right)=\sum_{i=1}^na*b_i-\sum_{j=1}^ma*c_j,
% % \]
% \begin{align}
% \label{eq:Lau}
% a\circ \left(\sum_{i=1}^n b_i-\sum_{j=1}^mc_j\right)
% =\sum_{i=1}^n a\circ b_i-\sum_{j=1}^ma\circ c_j+(m-n+1)a.
% \end{align}
% \end{lemma}

% \begin{proof}

% \end{proof}

%The following result 
%answers \cite[Question 2.1(2)]{MR3818285}. 

\begin{theorem}
    \label{thm:Lau}
	If $(A,+,\circ)$ is an associative skew brace of abelian type, 
	then $(A,+,*)$ is a radical ring. 
\end{theorem}

\begin{proof}
    We need to prove that the right compatibility condition holds. 
    Since $A$ is associative, $(a*b)*c=a*(b*c)$ for all $a,b,c\in A$. Write the associativity condition between $a,b,c\in A$ as
 	\[
 	(a\circ b-a-b)\circ c-(a\circ b-a-b)-c
 	=a\circ (b\circ c-b-c)-a-(b\circ c-b-c),
 	\]
 	which is equivalent to 
 	\[
 	a'\circ ( (a\circ b-a-b)\circ c-a\circ b)
 	=a'\circ ( a\circ (b\circ c-b-c)-a-a-b\circ c+2c).
 	\]
 	By Lemma \ref{lem:CGIS}, this is equivalent to
 	\[ a'\circ ( (a\circ b-a-b)\circ c-a\circ b)
 	=a'\circ ( a\circ (b\circ c-b-c)-a-a+(-b)\circ c).
 	\]
    By using the formula~\eqref{eq:Lau}, this is equivalent to  
 	\[
 	(b+a'\circ(-b))\circ c-b+a'=b\circ c-b-c+a'\circ (-b)\circ c+a'
 	\]
 	Hence the associativity of $(A,*)$ is equivalent to 
 	\begin{equation}
 	    \label{eq:asociatividad}
 		(b+a'\circ (-b))\circ c+c=b\circ c+a'\circ(-b)\circ c.
 	\end{equation}
 	Let $b,c\in A$. If $d\in A$, then there exists $a\in A$ such that $d=a'\circ (-b)$. Equality~\eqref{eq:asociatividad} 
    implies that 
 	\[
 	(b+d)\circ c+c=b\circ c+d\circ c.\qedhere
 	\]
\end{proof}


In the database of \cite{MR3647970} one finds associative skew braces (of size 16) that are not two-sided, see for example \cite{MR4223285}.


% \begin{example}
% {\bf Ferran: si el ejemplo es demasiado técnico, entonces quizás es mejor escribir ``There are associative skew braces that are not two-sided, see for example \cite{}.''}
% \end{example}

% \begin{example}
% 	\label{exa:simple}
% 	Let $G$ be the group generated by the 
% 	permutations 
% 	\[
% 	(1263)(48ba)(57c9),
% 	\quad
% 	(145)(278)(39a)(6bc).
% 	\]
% 	Then $G$ is a group of order twelve isomorphic to $C_3\rtimes C_4$. 
% 	Let $\pi\colon G\to\Alt_4$ be the bijective map given by
% 	\begin{align*}
% 		&\id\mapsto\id, && (16)(23)(4b)(5c)(79)(8a)\mapsto (14)(23),\\
% 		& (145)(278)(39a)(6bc)\mapsto (234), && (1b564c)(29837a)\mapsto (143),\\
% 		& (154)(287)(3a9)(6cb)\mapsto (243), && (1c465b)(2a7389)\mapsto (142),\\
% 		& (1362)(4ab8)(59c7)\mapsto (13)(24), && (1263)(48ba)(57c9)\mapsto (12)(34),\\
% 		& (1a68)(253c)(49b7)\mapsto (132), && (186a)(2c35)(47b9)\mapsto (124),\\
% 		& (1967)(243b)(5ac8)\mapsto (134), && (1769)(2b34)(58ca)\mapsto (123).
% 	\end{align*}
% 	A straightforward calculation shows that $\Alt_4$
% 	with the operation
% 	\[
% 		\sigma\circ\tau=\pi(\pi^{-1}(\sigma)\pi^{-1}(\tau))
% 	\]
% 	is a brace. 
	
% 	Let $a=(14)(23)$ and $b=c=(234)$. Then 
% 	\[
% 		(12)(34)=(ab)\circ c\ne (a\circ c)c^{-1}(a\circ b)=(123),
% 	\]
% 	hence it is not two-sided.
% \end{example}

%\textcolor{blue}{
%A \emph{rack} is a set $X$ with a map $X\times X\to X$, $(x,y)\mapsto x\triangleleft
%y$, such that for each $y\in X$ the map $x\mapsto x\triangleleft y$ is
%bijective and $\lambda_a^{-1}(b^{-1}\partial(x)b)$ 
%$(x\triangleleft y)\triangleleft z=(x\triangleleft
%z)\triangleleft (y\triangleleft z)$ 
%holds for all $x,y,z\in X$.}

In Proposition~\ref{pro:Rump} we used radical rings to produce examples of solutions of the YBE. 
A natural question arises: Does one need radical rings? Surprisingly, 
radical rings are just the tip of the iceberg. 

\begin{theorem}
\label{thm:YB}
Let $A$ be a skew brace. Then 
$(A,r)$, where 
\[
r\colon A\times A\to A\times A,\quad
r(x,y)=( -x+x\circ y,(-x+x\circ y)'\circ x\circ y),
\]
is a solution to the YBE. 
\end{theorem}

\begin{proof}
    By Theorem~\ref{thm:LYZ}, 
    since $x\circ y=(-x+x\circ y)\circ ((-x+x\circ y)'\circ x\circ y)$ for all $x,y\in A$, 
    we only need to check that 
    $x\rhd y=\lambda_x(y)=-x+x\circ y$ 
    is a left action of $(A,\circ)$ on the set $A$ 
    and that $x\lhd y=\mu_y(x)=(-x+x\circ y)'\circ x\circ y$ 
    is a right action of $(A,\circ)$ on the set $A$. For the left action we use 
    Proposition~\ref{pro:lambda} and for the right action we use Proposition~\ref{pro:mu}.
\end{proof}

In Theorem~\ref{thm:YB} it is possible to prove that the solution 
is involutive if and only if the additive group of the brace is abelian. 
The next result generalizes this fact. We shall need a lemma.

% \begin{defn} 
%   Let $A$ be a skew brace with additive group $G$. The \emph{depth}
%   of $A$ is defined as the exponent of the group $G/Z(G)$.  
% \end{defn}

% \begin{exa} 
%   Classical braces have depth one.  
% \end{exa}

% To study the depth of a skew brace we need the following lemma. 

\begin{lemma}
\label{lem:|r|}
Let $A$ be a skew brace and $r$ be its associated solution.  Then
  \begin{align} 
  \nonumber
  r^{2n}(a,b)&=(-n(a\circ b)+a+n(a\circ
    b),\\
    \label{eq:r^2n}
    &\phantom{=(-n(a\circ b)+}(-n(a\circ b)+a+n(a\circ b))'\circ a\circ b),\\
  \nonumber
  r^{2n+1}(a,b)&=(-n(a\circ b)-a+(n+1)(a\circ
    b),\\
    \label{eq:r^2n+1}
    &\phantom{=(-n(a\circ b)+}(-n(a\circ b)-a+(n+1)(a\circ b))'\circ a\circ b),
    \end{align} 
    for all $n\geq0$.  Moreover, the following statements hold:
  \begin{enumerate} 
  \item $r^{2n}=\id$ if and only if $a+nb=nb+a$ for all $a,b\in A$.  
      \item $r^{2n+1}=\id$ if and only if $\lambda_a(b)=n(a\circ
	b)+a-n(a\circ b)$ for all $a,b\in A$.  
	\end{enumerate} 
\end{lemma}

\begin{proof} 
First we shall prove~\eqref{eq:r^2n} and~\eqref{eq:r^2n+1} by induction on $n$. The case $n=0$ is trivial for~\eqref{eq:r^2n}
  and~\eqref{eq:r^2n+1}. Assume that the claim holds for some $n\geq 0$. By applying the map $r$ to Equation~\eqref{eq:r^2n+1} 
  we obtain that 
  \begin{align*} 
  r^{2(n+1)}(a,b) &=r\left( -n(a\circ b)-a+(n+1)(a\circ b),\right.\\
    &\phantom{=(-n(a\circ b)+} \left. (-n(a\circ b)-a+(n+1)(a\circ
    b))'\circ a\circ b\right)\\
    &=\left( -(n+1)(a\circ b)+a+(n+1)(a\circ b),\right.\\
    &\phantom{=(-n(a\circ b)+} \left. (-(n+1)(a\circ b)+a+(n+1)(a\circ
    b))'\circ a\circ b\right).
    \end{align*} 
    By applying $r$ again to this equality, we get 
    \begin{align*} 
  r^{2(n+1)+1}(a,b) &= r\left(-(n+1)(a\circ b)+a+(n+1)(a\circ
    b),\right.\\
    &\phantom{=(-n(a\circ b)+} \left. (-(n+1)(a\circ b)+a+(n+1)(a\circ b))'\circ a\circ b\right)\\
    &=\left( -(n+1)(a\circ b)-a+(n+2)(a\circ b),\right.\\
    &\phantom{=(-n(a\circ b)+} \left. (-(n+1)(a\circ b)-a+(n+2)(a\circ
    b))'\circ a\circ b\right).
    \end{align*} 
   Thus Equations~\eqref{eq:r^2n} and~\eqref{eq:r^2n+1} hold by induction.  The other claims follow easily from
    Equations~\eqref{eq:r^2n} and~\eqref{eq:r^2n+1}.
\end{proof}

%\begin{thm} \label{pro:depth_even} Let $A$ be a skew brace of finite depth
%with more than one element and let $r_A$ be its associated solution. Then the
%order of $r_A$ is an even number.  \end{thm}
%
%\begin{proof} Let $n$ be such that $r^{2n+1}=\id$. By applying
%Lemma~\ref{lem:depth} one obtains that $a^{-1}(a\circ b)^{n+1}=(a\circ b)^na$
%for all $a,b\in A$. In particular, if $b=1$, then $a=1$.  \end{proof}

Recall that the (minimal) \emph{exponent} $\exp(G)$ of a 
finite group $G$ is the least positive integer $n$ such that 
$g^n=1$ for all $g\in G$. 

\begin{theorem} 
\label{thm:|r|} 
  Let $A$ be a finite skew brace with more than one
  element and let $G$ be the additive group of $A$. 
  If $r$ is the solution associated with $A$, 
  then $r$ has order $2\exp(G/Z(G))$.
\end{theorem}

\begin{proof} 
  Assume that $r$ has order $2n+1$. Since $r^{2n+1}=\id$, by applying
  Lemma~\ref{lem:|r|}, one obtains that $-a+(n+1)(a\circ b)=n(a\circ b)+a$
  for all $a,b\in A$. In particular, for $b=0$, we get $a=0$, for all $a\in A$, a contradiction. 
  Therefore we may assume that the order of the permutation $r$ is
  $2n$, where 
  \[
  n=\min\{k\in\Z: k>0\text{ and }kb+a=a+kb\;\text{ for all }a,b\in A\}.
  \]
  Now one computes
  \begin{align*} 
  n&=\min\{k\in\Z: k>0\text{ and }kb\in Z(G)\text{ for all }b\in A\}\\ 
  &=\min\{k\in\Z: k>0\text{ and }k(b+Z(G)) = Z(G)\text{ for all }b\in A\} =\exp(G/Z(G)).\qedhere
  \end{align*}
\end{proof}

An inmmediate consequence is the following result.

\begin{corollary}
    Let $A$ be a finite skew brace and $r$ be its associated solution. Then 
    $r$ is involutive if and only if $A$ is of abelian type.
    \end{corollary}

% \begin{proof}
%     It follows immediately from Theorem~\ref{thm:|r|}.
% \end{proof}

%\begin{exa} A skew brace has depth one if and only if its additive group is
%  abelian.  \end{exa}

% \begin{exa} 
%   \label{exa:2p} 
%   Let $p$ be an odd prime number and let $A$ be a non-classical skew brace of
%   size $2p$. Then the additive group of $A$ is isomorphic to the dihedral group
%   $\D_{2p}$ of size $2p$.  Since $Z(\D_{2p})=1$ and the exponent of $\D_{2p}$
%   is $2p$, the order of $r_A$ is $4p$.
% \end{exa}

\section{Subbraces and ideals}

\begin{definition}
\index{Subbrace}
Let $A$ be a skew brace. A \emph{subbrace} of $A$ is a 
subset $B$ of $A$ such that $(B,+)$ is a subgroup of $(A,+)$ and $(B,\circ)$ is a subgroup of $(A,\circ)$. 
\end{definition}

\begin{definition}
    \index{Left ideal!of a skew brace}
    \index{Strong left ideal!of a skew brace}
    Let $A$ be a skew brace. A \emph{left ideal} of $A$ is a subgroup $(I,+)$ of
	$(A,+)$ such that $\lambda_a(I)\subseteq I$ for all $a\in A$, i.e. $\lambda_a(x)\in I$ for all $a\in A$ and $x\in I$. A \emph{strong left ideal} of $A$ 
	is a left ideal $I$ of $A$ such that $(I,+)$ is a normal subgroup of $(A,+)$. 
\end{definition}

\begin{proposition}
    A left ideal $I$ of a skew brace $A$ is a subbrace of $A$. 
\end{proposition}

\begin{proof}
    We need to prove that $(I,\circ)$ is a subgroup of $(A,\circ)$. Clearly $I$ is non-empty, 
    as it is an additive subgroup of $A$. If $x,y\in I$, then
    $x\circ y=x+\lambda_x(y)\in I+I\subseteq I$ and $x'=-\lambda_{x'}(x)\in I$. 
\end{proof}

\begin{example}
    Let $A$ be a skew brace. Then 
    \[
    \Fix(A)=\{a\in A:\lambda_x(a)=a\text{ for all $x\in A$}\}
    \]
    is a left ideal of $A$. 
\end{example}

\begin{definition}
    \index{Ideal of a skew brace}
    An \emph{ideal} of a skew brace $A$ is a strong left ideal $I$ of $A$ such that 
	$(I,\circ)$ is a normal subgroup of $(A,\circ)$. 
\end{definition}

In general, left ideals, strong left ideals and ideals are
different notions, see Exercise \ref{prob:exleftidealnotstrong}. 

\begin{example}
    Consider the semidirect product $A=\Z/(3)\rtimes \Z/(2)$ of the
    trivial skew braces $\Z/(3)$ and $\Z/(2)$
    via the non-trivial action of $\Z/(2)$ over $\Z/(3)$.
    Then 
    \[
    \lambda_{(x,y)}(a,b)=-(x,y)+(x,y)\circ(a,b)=-(x,y)+(x+(-1)^ya,y+b)=((-1)^ya,b)
    \]
    and hence $\Fix(A)=\{(0,0),(0,1)\}$ is not a 
    normal subgroup \textcolor{red}{of $(A,\circ)$.} In particular, $\Fix(A)$ is not 
    an ideal of $A$.
\end{example}

\begin{example}
    \index{Kernel}
	Let $f\colon A\to B$ be a homomorphism of skew braces. Then $\ker f$ 
	is an ideal of $A$.
\end{example}

If $X$ and $Y$ are subsets of a brace $A$, $X*Y$ is defined as the 
subgroup of $(A,+)$ generated by elements of the form $x*y$, $x\in X$ and $y\in Y$, i.e.
\[
X*Y=\langle x*y:x\in X\,,y\in Y\rangle_+.
\]

\begin{proposition}
    \label{pro:A*I}
    Let $A$ be a skew brace. A subgroup $I$ of $(A,+)$ is 
    a left ideal of $A$ if and only if $A*I\subseteq I$.
\end{proposition}

\begin{proof}
    Let $a\in A$ and $x\in I$. If $I$ is a
    left ideal, then $a*x=\lambda_a(x)-x\in I$. Conversely, if $A*I\subseteq
    I$, then $\lambda_a(x)=a*x+x\in I$.
\end{proof}

\begin{proposition}
    \label{pro:I*A}
    Let $A$ be a skew brace. A normal subgroup $I$ of $(A,+)$
    is an ideal of $A$ if and only $\lambda_a(I)\subseteq I$, for all $a\in A$, and
    $I*A\subseteq I$.
\end{proposition}

\begin{proof}
    Let $x\in I$ and $a\in A$.  Assume first that $I$ is invariant under the
    action of $\lambda$ and that $I*A\subseteq I$. Then
    \begin{equation}
    \label{eq:trick:I*A}
        \begin{aligned}
        a\circ x\circ a' &=a+\lambda_a(x\circ a')\\
        &=a+\lambda_a(x+\lambda_x(a'))
        =a+\lambda_a(x)+\lambda_a\lambda_x(a')+a-a\\
        &=a+\lambda_a(x+\lambda_x(a')-a')-a
        =a+\lambda_a(x+x*a')-a\in I,
    \end{aligned}
    \end{equation}
    and hence $I$ is an ideal.

    Conversely, assume that $I$ is an ideal. Then $I*A\subseteq I$ since
    \begin{align*}
        x*a&=-x+x\circ a-a\\
        &=-x+a\circ(a'\circ x\circ a)-a
        =-x+a+\lambda_a(a'\circ x\circ a)-a\in I
    \end{align*}
   for all $x\in I$ and $a\in A$.
\end{proof}


Let $I$ and $J$ be ideals
of a skew brace $A$. Then $I\cap J$ is an ideal of $A$, see Exercise~\ref{prob:sum_ideals}.  
The sum $I+J$ of $I$ and $J$ is defined as the
additive subgroup of $A$ generated by all the 
elements of the form
$u+v$, $u\in I$ and $v\in J$. 

\begin{proposition}
Let $A$ be a skew brace and let
$I$ and $J$ be ideals of $A$. Then $I+J$ is an ideal of $A$.
\end{proposition}

\begin{proof}
    Since $I$ and $J$ are normal \textcolor{red}{subgroups of $(A,+)$, we have that
    \[ I+J=\{ u+v \mid u\in I,\; v\in J \}\]
    is a normal subgroup of $(A,+)$.}
    Let $a\in A$, $u\in I$ and $v\in J$. Then 
    \[
    \lambda_a(u+v)=\lambda_a(u)+\lambda_a(v)\in I+J
    \]
    and
    hence it follows that $\lambda_a(I+J)\subseteq I+J$. Moreover, by \textcolor{red}{Proposition~\ref{pro:I*A}},
        \[
        (u+v)*a=(u\circ\lambda^{-1}_u(v))*a
        =u*(\lambda^{-1}_u(v)*a)+\lambda^{-1}_u(v)*a+u*a\in I+J.
    \]
    Hence $(I+J)*A\subseteq I+J$. Therefore the result follows by Proposition~\ref{pro:I*A}.
\end{proof}


\begin{definition}
	\index{Socle of a skew brace}
	Let $A$ be a skew brace. The subset 
	$\Soc(A)=\ker\lambda\cap Z(A,+)$
	is the \emph{socle} of $A$.
\end{definition}

\begin{lemma}
    \label{lem:socle}
    Let $A$ be a skew brace and $x\in\Soc(A)$. Then 
    \[
    a+a\circ x=a\circ x+a\quad\text{and}\quad
    \lambda_a(x)=a\circ x\circ a'
    \]
    hold 
    for all $a\in A$. In particular, $x\circ a=x+a$ 
    for all $x\in\Soc(A)$ and $a\in A$, 
    and
    $x'=-x$ for all $x\in\Soc(A)$.
\end{lemma}

\begin{proof}
    Let $a\in A$ and $x\in\Soc(A)$. Since
    \[
    a'\circ (a\circ x+a)=x-a'\quad\text{and}\quad
    a'\circ (a+a\circ x)=-a'+x,
    \]
    the first claim follows since
    $x\in Z(A,+)$.
    Now we prove the second claim:
    \[
    a\circ x\circ a'=a\circ (x\circ a')=a\circ (x+a')=a\circ x-a=-a+a\circ
    x=\lambda_a(x).\qedhere
    \]
\end{proof}

\begin{proposition}
	\label{pro:socle}
	Let $A$ be a skew brace. Then $\Soc(A)$ is an ideal of $A$.
	\end{proposition}

	\begin{proof}
		We first prove that $(\Soc(A),+)$ is 
		a normal subgroup of $(A,+)$. 
		Clearly $0\in\Soc(A)$, since $\lambda$ is a group homomorphism. Let $x,y\in\Soc(A)$ and $a\in A$.  
		By Lemma \ref{lem:socle}, 
		$x-y=x\circ(-y)=x\circ y'$ and hence 
		\[
		\lambda_{x-y}(a)=\lambda_{x\circ y'}(a)=\lambda_x\lambda^{-1}_y(a)=a.
		\]
		Thus $x-y\in\Soc(A)$ and therefore the claim follows, as 
		$\Soc(A)$ is central in $(A,+)$.
		
		
		We now prove that $\lambda_a(x)\in\Soc(A)$ for all $a\in A$ and $x\in\Soc(A)$. 
		First we note that $\lambda_a(x)$ is central, as
	    \begin{align*}
	        b+\lambda_a(x)&=b-a+a\circ x=a\circ (a'\circ b+x)\\
	        &=a\circ (x+a'\circ b)=a\circ x-a+b
	        =-a+a\circ x+b=\lambda_a(x)+b
	    \end{align*} 	
	    for all $b\in A$, by Lemma \ref{lem:socle}. Moreover, 
	    by Lemma \ref{lem:socle},
	    \[
	    \lambda_{\lambda_a(x)}(b)=\lambda_{a\circ x\circ a'}(b)
	    =\lambda_a\lambda_x\lambda^{-1}_a(b)=b
	    \]
	    for all $b\in A$. Since $x*a=-x+x\circ a-a=-x+x+a-a=0\in\Soc(A)$, 
	    the claim follows from Proposition \ref{pro:I*A}.
	\end{proof}

As a corollary we obtain that the socle of a skew brace $A$ is a trivial skew brace of abelian type. 
%In particular, if $a\in\Soc(A)$, then $a$ is a central element of $(A,+)$ such that  
%$a\circ b=a+b$, for all $b\in A$. 

\begin{proposition}
    \label{pro:soc_kernels}
    Let $A$ be a skew brace. Then $\Soc(A)=\ker\lambda\cap\ker\mu$.
\end{proposition}

\begin{proof}
    Let $a\in\Soc(A)$ and $b\in A$. Then $\lambda_a=\id$ and $a\in Z(A,+)$. By Lemma~\ref{lem:socle}, \[\mu_a(b)=\lambda_b(a)'\circ b\circ a=(b\circ a\circ b')'\circ b\circ a=b.\]  Thus $a\in\ker\lambda\cap\ker\mu$. 
    
    Conversely, let $a\in\ker\lambda\cap\ker\mu$ and $b\in A$. Then $b'=\mu_a(b')=\lambda_{b'}(a)'\circ b'\circ a$, so
    $\lambda_{b'}(a)=b'\circ a\circ b$. By Lemma \ref{lem:socle}
    and \eqref{eq:formulas},
    \[
    b+a=b\circ\lambda^{-1}_b(a)=b\circ\lambda_{b'}(a)=b\circ b'\circ a\circ b=a\circ b=a+\lambda_a(b)=a+b.
    \]
    Hence $a\in\Soc(A)$. 
\end{proof}

%Another important ideal was defined in~\cite{MR3917122}.

\begin{definition}
\index{Annihilator!of a skew brace}
Let $A$ be a skew brace. The \emph{annihilator} of $A$ is 
defined as the set $\Ann(A)=\Soc(A)\cap Z(A,\circ)$. 
\end{definition}

Note that $\Ann(A)\subseteq\Fix(A)$. 

\begin{proposition}
The annihilator of a skew brace $A$ is an ideal of $A$. 
\end{proposition}

\begin{proof}
    Let $x,y\in\Ann(A)$. Note that $x-y=x\circ y'\in Z(A,\circ)$. Hence $\Ann(A)$ is a subbrace of $A$. Since $\Ann(A)\subseteq Z(A,+)\cap Z(A,\circ)$, 
    we only need to note that $\lambda_a(x)=x\in\Ann(A)$, for all $a\in A$. 
\end{proof}



%Clearly $\Soc(A)=\ker(\lambda)\cap Z(A,+)$. In \cite[Lemma~2.5]{MR3647970} it
%is proved that $\Soc(A)$ is an ideal of $A$.

\index{Quotient skew brace}
If $A$ is a skew brace and $I$ is an ideal of $A$, then $a+I=a\circ I$ for all $a\in A$. Indeed, 
$a\circ x=a+\lambda_a(x)\in a+I$ and 
$a+x=a\circ\lambda_a^{-1}(x)=a\circ\lambda_{a'}(x)\in a\circ I$ 
for all $a\in A$ and $x\in I$. 
This allows us to prove that there exists a unique skew brace structure over $A/I$ such that
the map 
\[
\pi\colon A\to A/I,
\quad
a\mapsto a+I=a\circ I,
\]
is a homomorphism of skew braces. The \textcolor{red}{skew brace} $A/I$ is the \emph{quotient \textcolor{red}{skew brace}} of $A$ modulo $I$. It is possible
to prove the isomorphism theorems for skew braces, see Exercises~\ref{prob:iso1},~\ref{prob:iso2}, \ref{prob:iso3} and
\ref{prob:correspondence}.


\section{Exercises}

% \begin{prob}
% 	Let $(A,\lambda^A)$ and $(B,\lambda^B)$ be skew braces. Then $A$ and $B$
% 	are isomorphic if and only if there is a group homomorphism $\alpha\colon
% 	A\to B$ such that $\alpha\lambda^A_a\alpha^{-1}=\lambda^B_{\alpha(a)}$ for
% 	all $a\in A$.
% \end{prob}

\begin{prob}
\label{prob:left_right}
    Prove that there exists a bijective correspondence between skew left braces and skew right braces. 
\end{prob}

\begin{prob}
Let $p$ be a prime number. Prove that $\Z/(p^2)$ is a skew brace of abelian type with
the operation $x\circ y=x+y+pxy$. 
\end{prob}

\begin{prob}
Let $A$ be a skew brace. 
Prove that 
\[
\mu_b(a)=\lambda^{-1}_{\lambda_a(b)}(-a\circ b+a+a\circ b).
\]
\end{prob}

\begin{prob}
\label{prob:star}
Let $A$ be an additive (not necessarily abelian) group.  
Prove that a structure of skew brace over $A$ is equivalent to an operation $A\times A\to
A$, $(a,b)\mapsto a*b$, such that 
\[
a*(b+c)=a*b+b+a*c-b
\]
holds for all $a,b,c\in A$, and the operation $a\circ b=a+a*b+b$ turns $A$
into a group. 
\end{prob}

\begin{prob}
\label{prob:equivalences}
	Let $(A,+,\circ)$ be a triple, where 
	$(A,+)$ and $(A,\circ)$ are
	groups, and let $\lambda\colon A\to\Sym_A$, $a\mapsto\lambda_a$, $\lambda_a(b)=-a+a\circ b$. 
	Prove that the following statements are equivalent:
	\begin{enumerate}
		\item $A$ is a skew brace.
		\item $\lambda_{a\circ b}(c)=\lambda_a\lambda_b(c)$ for all $a,b,c\in A$.
		\item $\lambda_a(b+c)=\lambda_a(b)+\lambda_a(c)$ for all $a,b,c\in A$.
	\end{enumerate}
\end{prob}

\begin{prob}
Let $A$ and $B$ be skew braces. Prove that $f\colon A\rightarrow B$ is a homomorphism of skew braces if and only if $f(a+b)=f(a)+f(b)$ and \textcolor{red}{$f(\lambda_a(b))=\lambda_{f(a)}(f(b))$,} for all $a,b\in A$.
\end{prob}


\begin{prob}
\label{prob:2sided}
	Let $A$ be a \textcolor{red}{skew brace} such that $\lambda_a(a)=a$ for all $a\in A$.
	Prove that $A$ is two-sided. \textcolor{red}{(Hint: Prove first that $a'=-a$ for all $a\in A$.)}
\end{prob}

\begin{prob}
\label{prob:Lau}
    Prove Equality~\eqref{eq:Lau}.
\end{prob}

\begin{prob}
\label{prob:Landau}
Let $k,n\in\N$ and let $A$ be a finite skew brace of size $n$ and $k$ orbits. 
Then $|k|\geq\frac{\log\log n}{\log 4}$. 
\end{prob}

% orbits = lambda-orbits

\begin{prob}
\label{prob:radical}
Recall that skew two-sided braces of abelian type are equivalent to radical rings. Prove that under this equivalence, 
(left) ideals of the radical ring correspond to (left) ideals of the associated brace. 
\end{prob}

\begin{prob}
\label{prob:sum_ideals}
Prove that the intersection of ideals of a skew brace $A$ is an ideal of $A$. 
\end{prob}

\begin{prob}
Let $A$ be a skew brace and $I$ be a characteristic subgroup of the additive group of $A$. Prove that
$I$ is a left ideal of $A$. 
\end{prob}

%\begin{prob}
%Let $A=...$. Prove that $A$ has only three ideals:... Let $I$ be the ideal of $A$ of size four. Prove that
%$A*I$ has size two and hence it is not an ideal of $A$. 
%\end{prob}

\begin{prob}\label{prob:exleftidealnotstrong}
\textcolor{red}{Give an example of a left ideal which is not a strong left ideal of a skew brace. Give an example of a strong left ideal which is not an ideal of a skew brace.} 
\end{prob}

\begin{prob}
\label{prob:Bachiller1}
Prove that the socle of a skew brace $A$ is the kernel of the 
group homomorphism $(A,\circ)\to\Aut(A,+)\times\Sym_A$, $a\mapsto (\lambda_a,\mu_a^{-1})$. 
\end{prob}

\begin{prob}
\label{prob:Bachiller2}
Prove that the socle of a skew brace $A$ is the kernel of the 
group homomorphism $(A,\circ)\to\Aut(A,+)\times\Aut(A,+)$, $a\mapsto (\lambda_a,\xi_a)$, where
$\xi_a(b)=a+\lambda_a(b)-a$. 
\end{prob}

\begin{prob}
\label{prob:iso1}
    Let $f\colon A\to B$ be a homomorphism of skew braces. Prove that $A/\ker f\simeq f(A)$. 
\end{prob}

\begin{prob}
\label{prob:iso2}
    Let $A$ be a skew brace and let $B$ be a subbrace of $A$. Prove that if $I$ is an ideal of $A$, 
    then $B\circ I$ is a subbrace of $A$, 
    $B\cap I$ is an ideal of $B$ and $(B\circ I)/I\simeq B/(B\cap I)$. 
\end{prob}

\begin{prob}
\label{prob:iso3}
Let $A$ be a skew brace and $I$ and $J$ be ideals of $A$. Prove that if $I\subseteq J$, then
$A/J\simeq (A/I)/(J/I)$. 
\end{prob}

\begin{prob}
\label{prob:correspondence}
Let $A$ be a skew brace and let $I$ be an ideal of $A$. Prove that there is a bijective correspondence between (left) ideals 
of $A$ containing $I$ and (left) ideals of $A/I$. 
\end{prob}

%\begin{prob}
%\label{prob:G(X,r)solvable}
%Let $(X,r)$ be a finite involutive solution. Prove that $G(X,r)$ is solvable. 
%\end{prob}


\section{Notes}

\index{Rump, W.}
\index{Ced\'o, F.}
\index{Jespers, E.}
\index{Okni\'nski, J.}
Skew braces of abelian type (i.e. left braces) were introduced by Rump in~\cite{MR2278047} for studying involutive
solutions to the YBE. 
Rump's definition was reformulated  
by Ced\'o, Jespers and Okni\'nski in~\cite{MR3177933}. With this definition at hand, 
Guarnieri and Vendramin introduced 
arbitrary skew braces in~\cite{MR3647970}. 

\index{Weinstein, A.}
\index{Xu, P.}
\textcolor{blue}{Example is motivated by the paper~\cite{MR1178147}.}

Theorem~\ref{thm:radical} was proved by Rump in~\cite{MR2278047}. 

\index{Lau, I.}
\index{Kinyon, M.}
\index{Ced\'o, F.}
\index{Smoktunowicz, A.}
Theorem~\ref{thm:Lau} was proved by Lau~\cite{MR4136750} and independently by Kinyon (unpublished). It answers 
a question of Ced\'o, Gateva--Ivanova and Smoktunowicz, see~\cite{MR3818285}. 

\index{Vendramin, L.}
\index{Guarnieri, L.}
\index{Smoktunowicz, A.}
Theorem~\ref{thm:YB} for skew braces of abelian type appears implicit in the work~\cite{MR2278047} of Rump, see
also~\cite{MR3177933}. The general case was proved by Guarnieri and Vendramin in~\cite{MR3647970}. 

Theorem~\ref{thm:|r|} was proved by Smoktunowicz and Vendramin in~\cite{MR3763907}. 

\index{Rump, W.}
\index{Catino, F.}
\index{Colazzo, I.}
\index{Stefanelli, P.}
The socle was defined by Rump in~\cite{MR2278047}. The annihilator first appeared in the work~\cite{MR3917122}  
of Catino, Colazzo and Stefanelli. 

\index{Byott, N. P.}
Theorem \ref{thm:add_nilpotent} was proved by Byott in the context of Hopf-Galois extensions in \cite{MR3425626}. 


\index{Bachiller, D.}
\index{Rump, W.}
\index{Gateva--Ivanova, T.}
Exercise~\ref{prob:equivalences} combines results of Bachiller, Rump~\cite{MR2278047} and
Gateva--Ivanova~\cite{MR3861714}. 
Exercise~\ref{prob:2sided}  \textcolor{red}{for skew braces of abelian type} comes from~\cite{MR3177933}. 
Exercises~\ref{prob:Bachiller1} and~\ref{prob:Bachiller2} appear in~\cite{MR3835326}.

Exercise~\ref{prob:Landau} is a direct translation into the context of braces 
(or more precisely, the context of groups acting by automorphisms on groups)   
of Newmann's result based on Landau's method. 

