
The Yang--Baxter equation (YBE) arose from Yang's work on statistic mechanics. In 1967 Yang tried to find the eigenfunctions of a one-dimensional fermion gas with delta function inteaction. This was
a rather difficult problem. He solved it and showed that a crucual identity in the intermediate steps was a matrix equation
\[
A(u)B(u+v)A(v)=B(v)A(u+v)B(u).
\]
Later, Baxter, in his solution of another problem in physics, the 8-vertex model again used the YBE. 
In 1980 Faddeev coined the term "Yang-Baxter Equation". A number of exciting developmenents in 
physics and mathematics have led to the conclusion that the YBE is a fundamental mathematical structure with connections to various subfields of mathematics such as 
knot theory, braid theory, operator theory, Hopf algebras, quantum groups, 3-manifolds, the monondromy of differential equations...

\begin{quote}
    I got the feeling that the YBE is the next pervasive algebraic equation after the Jacobi identity.
\end{quote}

\framebox{Compiled: \today,~\currenttime.}

Thanks: Jingpeng Shen
