
The Yang--Baxter equation (YBE) arose from Yang's work on statistic mechanics. In 1967 Yang tried to find the eigenfunctions of a one-dimensional fermion gas with delta function inteaction. This was
a rather difficult problem. He solved it and showed that a crucual identity in the intermediate steps was a matrix equation
\[
A(u)B(u+v)A(v)=B(v)A(u+v)B(u).
\]
Later, Baxter, in his solution of another problem in physics, the 8-vertex model again used the YBE. 
In 1980 Faddeev coined the term "Yang-Baxter Equation". A number of exciting developmenents in 
physics and mathematics have led to the conclusion that the YBE is a fundamental mathematical structure with connections to various subfields of mathematics such as 
knot theory, braid theory, operator theory, Hopf algebras, quantum groups, 3-manifolds, the monondromy of differential equations...

\begin{quote}
    I got the feeling that the YBE is the next pervasive algebraic equation after the Jacobi identity.
\end{quote}

\framebox{Compiled: \today,~\currenttime.}

\medskip
In Chapter~\ref{YB} basic definitions and examples of solutions introduced. 
The main result of the chapter is Theorem~\ref{thm:LYZ}, where the deep relationship between
solutions and group actions is studied. 

The first part of Chapter~\ref{radical} provides an introductions to the theory of radical rings.
After recalling basic definitions and stating basic properties, the the theory of the Jacobson
radical is explored. The second part of the chapter is devoted to involutive solutions. One of the
main results of this chapter is Rump's theorem, which states that radical rings produce solutions. In this
chapter we also introduce cycle sets, which are structures that turns out to be equivalent
to involutive solutions. 

In Chapter~\ref{braces} we introduce the theory of braces. One of the main results of this 
chapter is Theorem~\ref{thm:}, which proves that braces produce arbitrary solutions. We introduce
skew cycle sets and prove in Theorem... that skew cycle sets and arbitrary solutions are equivalent. 

In Chapter~\ref{cocycles} we study 1-cocycles. 
In Theorem... 
Sysak's theorem states that...

Chapter~\ref{lem:factorization} is about factorization of groups and braces. First
we prove It\^o's theorem in the case of groups: Every group that admits a factorization through two 
abelian subgroups is meta-abelian. 

Thanks: Jingpeng Shen

\medskip
Versión compilada el \today~a las~\currenttime.

\begin{flushright}
Leandro Vendramin\\Buenos Aires, Argentina\par
\end{flushright}
