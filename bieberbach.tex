\chapter{Bieberbach groups}
\label{Bieberbach}

\section{Left ordered groupss}


\index{Group!left ordered}
A group $G$ is {\em left ordered} if there is a total ordering $\leq$ on $G$ 
such that $x\leq y$ implies $zx\leq zy$ for all $x,y,z\in G$. In this case we say that $G$ is a left ordered group with respect to the total order $\leq$. Similarly one
defines right ordered groups. 

\index{Group!ordered}
A group $G$ is {\em ordered} if there is a total ordering $\leq$ on $G$ 
such that $x\leq y$ implies $zx\leq zy$ and $xz\leq yz$ for all $x,y,z\in G$.

\begin{example}
The group $\Z$ is an ordered group with respect to the natural order. 
\end{example}

\begin{example}
	If $G$ is a left ordered group and $H$ is a subgroup of $G$, then $H$ is left ordered.
\end{example}

\begin{remark}
Let $G$ be a left ordered group with respect to the total order $\leq$. We define another total order $\leq'$ on $G$ by
\[ x\leq' y\text{ if and only if }x^{-1}\leq y^{-1}\]
for all $x,y\in G$. Note that if $x\leq' y$ then
\[ (xz)^{-1}=z^{-1}x^{-1}\leq z^{-1}y^{-1}=(yz)^{-1},\]
and thus $xz\leq' yz$ for all $z\in G$. Hence $G$ is a right ordered group with respect to the total order $\leq'$.
\end{remark}

\begin{proposition}\label{prop:LOgroup1}
Let $G$ be a group and let $N\unlhd G$. If $N$ and $G/N$ are left ordered groups, then $G$ is left ordered. 
\end{proposition}

\begin{proof}
Suppose that $N$ is a left ordered group with respect to the total order $\leq_N$ and $G/N$ is a left ordered group with respect to the total order $\leq_{\bar G}$.	We define 
	\[
	x\leq y\Longleftrightarrow
	\begin{cases}
		1\leq_N x^{-1}y & \text{if $xN=yN$},\\
		xN\leq_{\bar G}yN & \text{otherwise},		
	\end{cases}
	\]
for all $x,y\in G$.	A straightforward computation shows that then $G$ is a left ordered group with respect to the total order $\leq$. 
\end{proof}

\begin{example}
	Let us show that $G=\gr( x,y:xyx^{-1}=y^{-1})$ is left-orderable. Let 
	$f\colon G\to\Z$ be given by $x\mapsto 1$ and $y\mapsto 0$. Then $\ker f=\langle y\rangle$. 
	 The map
\[
\{ x,y\}\to\GL_2(\C),
\quad 
x\mapsto\begin{pmatrix}
-1&0\\
0&1	
\end{pmatrix},
\quad
y\mapsto\begin{pmatrix}
1&1\\
0&1	
\end{pmatrix},
\]
induces a group homomorphism $G\to \GL_2(\C)$. 
In particular, $y$ has infinite order and hence $\langle y\rangle\cong\Z$. Hence $\ker f$ and $G/\ker f$ are left ordered groups and, by Proposition \ref{prop:LOgroup1}, $G$ is a left ordered group. We shall see that $G$ is not an ordered group. Suppose that $G$ is an ordered group with respect to the total order $\leq$. Suppose that $1\leq y$. Then we have
\[ 1=xx^{-1}\leq xyx^{-1}=y^{-1},\]
and thus $y\leq yy^{-1}=1$, a contradiction because $y\neq 1$. Hence $y\leq 1$. But then we have that
\[ y^{-1}=xyx^{-1}\leq xx^{-1}=1,\]
and thus $1=yy^{-1}\leq  y$, a contradiction. Therefore $G$ is not an ordered group.
\end{example}

\index{Baumslag--Solitar's group}
The previous example is the Baumslag--Solitar group $B(1,-1)$. Recall that for $n,m\in\Z$
the Baumslag--Solitar's group is defined as the 
group \[
B(m,n)=\gr(a,b:ba^mb^{-1}=a^n).
\]
The map 
\[
\{ a,b\}\to\GL_2(\C),\quad
a\mapsto\begin{pmatrix}
\frac{m}{n}&1\\
0&1	
\end{pmatrix}\quad\text{ and }\quad  
b\mapsto\begin{pmatrix}
-1&0\\
0&1	
\end{pmatrix}
\] 
Induces a group homomorphism $B(m,n)\to \GL_2(\C)$.



\index{Group!poly-$\Z$}
A group $G$ is said to be {\em poly-$\Z$} if it has a finite subnormal series
\[ \{ 1\}=N_0\unlhd N_1\unlhd\dots\unlhd N_n=G\]
such that $N_{i+1}/N_{i}\cong \Z$ for all $1\leq i<n$.
By Proposition \ref{prop:LOgroup1} and induction on $n$, it is easy to see that every poly-$\Z$ group is left ordered. 

\index{Group!indicable}
\index{Group!locally indicable}
A group $G$ is said to be {\em indicable} if there exists a non-trivial 
group homomorphism $G\to\Z$, and $G$ is said to be {\em locally indicable} 
if every finitely generated subgroup of $G$ is indicable.  

%\framebox{FIXME}
% hay que completar todo esto! 

\begin{theorem}[Burns--Hale]
\index{Burns, R. G.}
\index{Hale, V. W. D.}
\index{Burns--Hale's theorem}
Let $G$ be a group. Then $G$ is left-orderable if and only if 
for each finitely generated non-trivial subgroup $H$ of $G$ there exists a left-ordered group $L$ 
and a non-trivial group homomorphism $H\to L$.  
\end{theorem}

\begin{proof}
	If $G$ is left-orderable, take $L=G$. 
	
	Conversely, suppose that for each finitely generated non-trivial subgroup $H$ of $G$ there exists a left-ordered group $L$ 
and a non-trivial group homomorphism $H\to L$. We claim that for all $\{x_1,\dots,x_n\}\subseteq G\setminus\{1\}$ 
	there exist $\epsilon_1,\dots,\epsilon_n\in\{-1,1\}$ such that 
	\[
	1\not\in S(x_1^{\epsilon_1},\dots,x_n^{\epsilon_n}),
	\]
	where $S(x_1^{\epsilon_1},\dots,x_n^{\epsilon_n})$ denotes the semigroup generated by 
	the set $\{x_1^{\epsilon_1},\dots,x_n^{\epsilon_n}\}$. 
	We proceed by induction on $n$. If $n=1$, then $x_1\in G\setminus\{1\}$. Let $\epsilon_1=1$. If
	$1\in S(x_1)$, then $x_1$ is an element of finite order and hence $\langle x_1\rangle\to L$ is the trivial homomorphism for all left ordered group $L$. Hence $1\notin S(x_1)$.
    Now assume that the claim holds for some $n\geq 1$. Let $\{x_1,\dots,x_{n+1}\}\subseteq G\setminus\{1\}$. 
	By assumption, there exists a non-trivial group homomorphism 
	$h\colon\langle x_1,\dots,x_{n+1}\rangle\to L$ for some left ordered group $L$. In particular, $h(x_i)\ne 1$ for some $i\in\{1,\dots,n\}$. Without loss
	of generality we may assume that there exists an integer $1\leq k\leq n+1$ such that $h(x_j)\ne 1$ for all $j\in\{1,\dots,k\}$ and 
	$h(x_j)=1$ for all $j>k$. Suppose that $L$ is left ordered with respect to a total order $\leq$. Since $h(x_j)\ne 1$ for all $j\leq k$, there
	are elements $\epsilon_j\in\{-1,1\}$ such that $1\leq h(x_j^{\epsilon_j})$ for all $j\leq k$. By the inductive hypothesis, 
	there are elements $\epsilon_{k+1},\dots,\epsilon_{n+1}\in\{-1,1\}$ such that 
	$1\not\in S(x_{k+1}^{\epsilon_{k+1}},\dots,x_{n+1}^{\epsilon_{n+1}})$. Note that for every  $x\in S(x_1^{\epsilon_1},\dots,x_{n+1}^{\epsilon_{n+1}})\setminus S(x_{k+1}^{\epsilon_{k+1}},\dots,x_{n+1}^{\epsilon_{n+1}})$, $1\leq h(x)\neq 1$. Hence $1\notin S(x_1^{\epsilon_1},\dots ,x_{n+1}^{\epsilon_{n+1}})$, and the claim follows by induction.
	
	Consider the set 
	
	$\mathcal{F}=\{ (F,f) : F$ is a finite subset  of $G\setminus\{ 1\}$ and   $f\colon A\to\{ -1, 1\}$ such that for every finite subset $B$ of $G\setminus\{ 1\}$ containing $F$, there exists a map  $g\colon B\to \{ -1,1\}$ such that  $1\notin S(a^{g(a)} : a\in B)$ and $g(x)=f(x)$ for all $x\in F\}$. 
	
	Let $\mathcal{C}=\{ (A,f) : A\subseteq G\setminus\{ 1\}$ and $f\colon A\to \{ -1,1\}$ such that  $(F,f|_F)\in\mathcal{F}$ for all finite subset $F\text{ of }A \}$. We define and order on $\mathcal{C}$ by $(A,f)\leq (B,g)$ if and only if $A\subseteq B$ and $g(a)=f(a)$ for all $a\in A$, i. e. $f=g|_A$. Note that there is a unique map $f_{\emptyset}\colon \emptyset\to \{-1,1\}$. We have shown that $(\emptyset,f_{\emptyset})\in\mathcal{C}$. Hence $\mathcal{C}\neq \emptyset$. Furthermore, it is easy to see that every chain of elements in $\mathcal{C}$ has an upper bound. Thus, by Zorn's lemma, there exists a maximal element $(A,f)\in\mathcal{C}$. Suppose that $A\neq G\setminus\{ 1\}$. Let $x\in G\setminus (A\cup\{ 1\})$. Let $g_1\colon A\cup\{ x\}\to\{ -1,1\}$ and $g_{-1}\colon A\cup\{ x\}\to\{ -1,1\}$ be the maps defined $g_i(a)=f(a)$ for all $a\in A$ and $g_i(x)=i$ for $i\in\{ -1,1\}$. By the maximality of $(A,f)$, we have that $(A\cup\{ x\}, g_i)\notin\mathcal{C}$. Hence there exists a finite subsets $F_1$ and $F_{-1}$ of $A\cup \{ x\}$ and finite subsets $B_1$ and $B_{-1}$ such that $F_1\subseteq B_1$, $F_{-1}\subseteq B_{-1}$, $1\in S(a^{h_1(a)}: a\in B_1)$ and $1\in S(a^{h_{-1}(a)}:a\in B_2)$ for all $h_1\colon B_1\to\{ -1,1\}$ and all $h_{-1}\colon B_{-1}\to \{ -1,1\}$ such that $g_i(a)=h_i(a)$ for all $a\in F_i$. Let $C=\bigcup_{i\in\{ -1,1\}}(A\cap F_i)$. Note that $C\cup\{ x\}=F_1\cupF_{-1}\subseteq B_1\cup B_{-1}$. Since $(C,f|_{C})\in \mathcal{F}$, there exists $h\colon B_1\cup B_{-1}\to\{ -1,1\}$ such that $1\notin S(a^{h(a)}: a\in B_1\cup B_{-1})$ and $h(a)=f(a)$ for all $a\in C$. Let $i=h(x)\in\{ -1,1\}$. We have that $h(x)=g_i(x)$, and thus $h(a)=g_i(a)$ for all $a\in F_i$, a contradiction because $S(a^{h(a)}:a\in B_i)\subseteq S(a^{h(a)}: a\in B_1\cup B_{-1})$.
	Therefore $A=G\setminus\{ 1\}$. 
	
	Let $P=\{a\in G\setminus \{1\} : f(a)=1\}$. Note that if $b\in G\setminus\{ 1\}$ then 
	\[1\notin S(b^{f(b)},(b^{-1})^{f(b^{-1})}).\] 
	Hence, $G$ is the disjoint union of $P$, $P^{-1}=\{ a^{-1} : a\in P\}$ and $\{ 1\}$. Note that for all $a,b\in P$, $1\notin S(a,b,(ab)^{f(ab)})$. Hence $f(ab)=1$ and thus $ab\in P$. This proves that $P$ is a subsemigroup of $G$. We define a binary relation $\leq$ on $G$ by, for all $a,b\in G$
	\[ a\leq b\text{ if and only if }a^{-1}b\in P\cup\{ 1\}.\]
	It is straightforward to check that $\leq$ is a total order on $G$ and that $G$ is a left ordered group with respect to $\leq$.
\end{proof}

An inmediate corollary:

\begin{corollary}
	Locally indicable groups are left-orderable. 
\end{corollary}

Another consequence of Burns--Hale's Theorem:

\begin{exercise}
	Let $G$ be a group and $\{N_\alpha:\alpha\}$ be a collection of normal subgroups of $G$ such that
	$\cap_{\alpha}N_\alpha=\{1\}$. If $G/N_{\alpha}$ is left-ordererable for all $\alpha$, then 
	$G$ is left-orderable. 
\end{exercise}

\index{Group!bi-orderable}
A group $G$ is \textbf{bi-orderable} if there exists a total ordering $<$ in $G$ 
such that $x<y$ implies $xz<yz$ and $zx<zy$ for all $z\in G$. 

\index{Group!with the unique product property}
\index{Unique product property}
A group $G$ satisfies the \textbf{unique product property} if there are non-empty
finite subsets $A$ and $B$ such that $|gA\cap B|=1$ for some $g\in G$. Thus $G$ 
satisfies the unique product property if and only if for all finite non-empty subsets
$A$ and $B$ there exists $g\in G$ such that $g=ab$ for unique elements 
$a\in A$ and $b\in B$.  

\begin{proposition}
	A group with the unique product property is torsion-free.	
\end{proposition}

\begin{proof}
	Assume that $G$ has torsion and let $g\in G$ be an element of order $n\geq2$. 
	Let $A=B=\{1,g,g^2,\dots,g^{n-1}\}$. Then $g$ admits more than one representation
	of the form $g=ab$ for $a\in A$ and $b\in B$, so $G$ cannot have the unique product property. 
\end{proof}

\begin{proposition}
	A left-orderable group satisfies the unique product property.
\end{proposition}

\begin{proof}
	Let $G$ be a group and 
    $A=\{a_1,\dots,a_n\}$ and $B=\{b_1,\dots,b_m\}$ be non-empty 
    subsets of $G$. We may assume that $a_1<a_2<\cdots<a_n$ and
    $b_1<b_2<\cdots<b_m$. Let $g\in G$ be such that $ga_n=b_1$. Then
    $ga_1<ga_2<\cdots<ga_n=b_1<b_2<\cdots<b_m$. 
\end{proof}

A group $G$ satisfies the \textbf{double unique product property} if for any two given finite non-empty 
subsets $A$ and $B$ of $G$ such that $|A|+|B|>2$ there exist at least two unique products in $AB$. 

\begin{theorem}[Strojnowski]
	\label{thm:Strojnowski}
	\index{Strojonowski's!Theorem}
	Sea $G$ un grupo. Las siguientes afirmaciones son equivalentes:
	\begin{enumerate}
		\item $G$ tiene la propiedad del doble producto único.
		\item Para todo subconjunto $A\subseteq G$ finito y no vacío, existe al
			menos un producto único en $AA=\{a_1a_2:a_1,a_2\in A\}$.
		\item $G$ tiene la propiedad del producto único.
	\end{enumerate}
\end{theorem}

\begin{proof}
	La implicación $(1)\implies(2)$ es trivial.  Demostremos que vale
	$(2)\implies(3)$. Si $G$ no tiene la propiedad del producto único, existen
	subconjuntos $A,B\subseteq G$ finitos y no vacíos tales que todo elemento
	de $AB$ admite al menos dos representaciones. Sea $C=AB$. Todo $c\in C$ es
	de la forma $c=(a_1b_1)(a_2b_2)$ con $a_1,a_2\in A$ y $b_1,b_2\in B$. Como
	$a_2^{-1}b_1^{-1}\in AB$, existen $a_3\in A\setminus\{a_2\}$ y $b_3\in B\setminus\{b_1\}$ tales que
	$a_2^{-1}b_1^{-1}=a_3^{-1}b_3^{-1}$. Luego $b_1a_2=b_3a_3$ y entonces
	\[
	c=(a_1b_1)(a_2b_2)=(a_1b_3)(a_3b_2)
	\]
	son dos representaciones distintas de $c$ en $AB$.
	pues $a_2\ne a_3$ y $b_1\ne b_3$.

	Demostremos ahora que $(3)\implies(1)$. Si $G$ tiene la propiedad del
	producto único pero no tiene la propiedad del doble producto único, existen
	subconjuntos $A,B\subseteq G$ finitos y no vacíos con $|A|+|B|>2$ tales que
	en $AB$ existe un único elemento $ab$ con una única representación en $AB$.
	Sean $C=a^{-1}A$ y $D=Bb^{-1}$. Entonces $1\in C\cap D$ y el elemento
	neutro $1$ admite una única representación en $CD$ (pues si $1=cd$ con
	$c=a^{-1}a_1\ne 1$ y $d=b_1b^{-1}\ne 1$, entonces $ab=a_1b_1$ con $a\ne
	a_1$ y $b\ne b_1)$. Sean $E=D^{-1}C$ y $F=DC^{-1}$. Todo elemento de $EF$
	se escribe como $(d_1^{-1}c_1)(d_2c_2^{-1})$. Si $c_1\ne 1$ o $d_2\ne 1$
	entonces $c_1d_2=c_3d_3$ para algún $c_3\in C\setminus\{c_1\}$ y algún
	$d_3\in D\setminus\{d_2\}$. Entonces
	$(d_1^{-1}c_1)(d_2c_2^{-1})=(d_1^{-1}c_3)(d_3c_2^{-1})$ son dos
	representaciones distintas para $(d_1^{-1}c_1)(d_2c_2^{-1})$. Si $c_2\ne 1$
	o $d_1\ne 1$ entonces $c_2d_1=c_4d_4$ para algún $d_4\in D\setminus\{d_1\}$
	y algún $c_4\in C\setminus\{c_2\}$ y entonces, como
	$d_1^{-1}c_2^{-1}=d_4^{-1}c_4^{-1}$,
	$(d_1^{-1}1)(1c_2^{-1})=(d_4^{-1}1)(1c_4^{-1})$.  Como $|C|+|D|>2$, $C$ o
	$D$ contienen algún $c\ne1$, y entonces $(1\cdot 1)(1\cdot 1)=(1\cdot
	c)(1\cdot c^{-1})$. Demostramos entonces que todo elemento de $EF$ tiene al
	menos dos representaciones. 
\end{proof}

% passman lema 1.9 pag 589
\begin{exercise}
	Demuestre que si $G$ es un grupo que satisface la propiedad del producto
	único, entonces $K[G]$ tiene solamente unidades triviales.
\end{exercise}

En general es muy difícil verificar si un grupo posee la propiedad del producto
único. Una propiedad similar es la de ser un grupo difuso. Si $G$ es un grupo
libre de torsión y $A\subseteq G$ es un subconjunto, diremos que $A$ es
antisimétrico si $A\cap A^{-1}\subseteq\{1\}$, donde $A^{-1}=\{a^{-1}:a\in
A\}$. El conjunto de \textbf{elementos extremales} de $A$ se define como
$\Delta(A)=\{a\in A:Aa^{-1}\text{ es antisimétrico}\}$. Luego
\[
	a\in A\setminus\Delta(A)
	\Longleftrightarrow
	\text{existe $g\in G\setminus\{1\}$ tal que $ga\in A$ y $g^{-1}a\in A$}.
\]

\begin{definition}
	\index{Grupo!difuso}
	Un grupo $G$ se dice \textbf{difuso} si para todo subconjunto $A\subseteq
	G$ tal que $2\leq |A|<\infty$ se tiene $|\Delta(A)|\geq2$.
\end{definition}

\begin{lemma}
	Si $G$ es ordenable a derecha, entonces $G$ es difuso.	
\end{lemma}

\begin{proof}
	Supongamos que $A=\{a_1,\dots,a_n\}$ y $a_1<a_2<\cdots<a_n$. Vamos a
	demostrar que $\{a_1,a_n\}\subseteq\Delta(A)$. Si $a_1\in
	A\setminus\Delta(A)$, existe $g\in G\setminus\{1\}$ tal que $ga_1\in A$ y
	$g^{-1}a_1\in A$. Esto implica que $a_1\leq ga_1$ y $a_1\leq g^{-1}a_1$, de
	donde se concluye que $1\leq g$ y $1\leq g^{-1}$, una contradicción. De la
	misma forma se demuestra que $a_n\in \Delta(A)$.
\end{proof}

\begin{lemma}
	\label{lemma:difuso=>2up}
	Si $G$ es difuso, entonces $G$ tiene la propiedad del doble producto único.	
\end{lemma}

\begin{proof}
	Supongamos que $G$ no tiene la propiedad del doble producto único. Existen
	entonces subconjuntos finitos $A,B\subseteq G$ con $|A|+|B|>2$ tales que
	$C=AB$ tiene a lo sumo un producto único. Luego $|C|\geq2$. Como $G$ es
	difuso, $|\Delta(C)|\geq2$. Si $c\in\Delta(C)$, entonces $c$ tiene una
	única expresión como $c=ab$ con $a\in A$ y $b\in B$ (de lo contrario, si
	$c=a_0b_0=a_1b_1$ con $a_0\ne a_1$ y $b_0\ne b_1$. Si $g=a_0a_1^{-1}$,
	entonces $g\ne 1$, $gc=a_0a_1^{-1}a_1b_1=a_0b_1\in C$ y además
	$g^{-1}c=a_1a_0^{-1}a_0b_0=a_1b_0\in C$. Luego $c\not\in\Delta(c)$, una
	contradicción.
\end{proof}



%Un grupo $G$ se dice \textbf{débilmente difuso} si para todo subconjunto
%finito $A\subseteq G$ no vacío se tiene $\Delta(A)\ne\emptyset$. La técnica
%usada para demostrar el lema~\ref{lemma:difuso=>2up} puede usarse para
%demostrar que un grupo débilmente difuso posee la propiedad del producto
%único. El teorema~\ref{theorem:Strojnowski} sugiere entonces la siguiente
%pregunta: 
%
%\begin{problem}
%	¿Existe un grupo débilmente difuso que no sea difuso?
%\end{problem}
%
%\section{El grupo de Promislow}
%
%Veremos un ejemplo concreto de un grupo sin torsión que no es ordenable, no es
%difuso y no tiene la propiedad del producto único.
%
%\begin{exercise}
%	\label{exercise:Dinfty}
%	Demuestre que $G=\langle x,y:x^2=y^2=1\rangle$ es isomorfo al grupo diedral infinito.
%\end{exercise}
%
%\begin{definition}
%	Se define el grupo de Promislow como 
%	\[
%		G=\langle x,y:x^{-1}y^2x=y^{-2},\,y^{-1}x^2y=x^{-2}\rangle.
%	\]
%\end{definition}
%
%\begin{proposition}
%	\label{proposition:Promislow}
%	El grupo de Promislow es libre de torsión y no satisface la propiedad del
%	producto único. 
%\end{proposition}
%
%\begin{proof}
%	
%\end{proof}



\section*{B}
