\chapter{Bieberbach groups}
\label{Bieberbach}

\section{Left ordered groups}


\index{Group!left ordered}
A group $G$ is {\em left ordered} if there is a total ordering $\leq$ on $G$ 
such that $x\leq y$ implies $zx\leq zy$ for all $x,y,z\in G$. In this case we say that $G$ is a left ordered group with respect to the total order $\leq$. Similarly one
defines right ordered groups. 

\index{Group!ordered}
A group $G$ is {\em ordered} if there is a total ordering $\leq$ on $G$ 
such that $x\leq y$ implies $zx\leq zy$ and $xz\leq yz$ for all $x,y,z\in G$.

\begin{example}
The group $\Z$ is an ordered group with respect to the natural order. 
\end{example}

\begin{example}
	If $G$ is a left ordered group and $H$ is a subgroup of $G$, then $H$ is left ordered.
\end{example}

\begin{remark}
Let $G$ be a left ordered group with respect to the total order $\leq$. We define another total order $\leq'$ on $G$ by
\[ x\leq' y\text{ if and only if }x^{-1}\leq y^{-1}\]
for all $x,y\in G$. Note that if $x\leq' y$ then
\[ (xz)^{-1}=z^{-1}x^{-1}\leq z^{-1}y^{-1}=(yz)^{-1},\]
and thus $xz\leq' yz$ for all $z\in G$. Hence $G$ is a right ordered group with respect to the total order $\leq'$.
\end{remark}

\begin{proposition}\label{prop:LOgroup1}
Let $G$ be a group and let $N\unlhd G$. If $N$ and $G/N$ are left ordered groups, then $G$ is left ordered. 
\end{proposition}

\begin{proof}
Suppose that $N$ is a left ordered group with respect to the total order $\leq_N$ and $G/N$ is a left ordered group with respect to the total order $\leq_{\bar G}$.	We define 
	\[
	x\leq y\Longleftrightarrow
	\begin{cases}
		1\leq_N x^{-1}y & \text{if $xN=yN$},\\
		xN\leq_{\bar G}yN & \text{otherwise},		
	\end{cases}
	\]
for all $x,y\in G$.	A straightforward computation shows that then $G$ is a left ordered group with respect to the total order $\leq$. 
\end{proof}

\begin{example}
	Let us show that $G=\gr( x,y:xyx^{-1}=y^{-1})$ is left ordered. Let 
	$f\colon G\to\Z$ be given by $x\mapsto 1$ and $y\mapsto 0$. Then $\ker f=\langle y\rangle$. 
	 The map
\[
\{ x,y\}\to\GL_2(\C),
\quad 
x\mapsto\begin{pmatrix}
-1&0\\
0&1	
\end{pmatrix},
\quad
y\mapsto\begin{pmatrix}
1&1\\
0&1	
\end{pmatrix},
\]
induces a group homomorphism $G\to \GL_2(\C)$. 
In particular, $y$ has infinite order and hence $\langle y\rangle\cong\Z$. Hence $\ker f$ and $G/\ker f$ are left ordered groups and, by Proposition \ref{prop:LOgroup1}, $G$ is a left ordered group. We shall see that $G$ is not an ordered group. Suppose that $G$ is an ordered group with respect to the total order $\leq$. Suppose that $1\leq y$. Then we have
\[ 1=xx^{-1}\leq xyx^{-1}=y^{-1},\]
and thus $y\leq yy^{-1}=1$, a contradiction because $y\neq 1$. Hence $y\leq 1$. But then we have that
\[ y^{-1}=xyx^{-1}\leq xx^{-1}=1,\]
and thus $1=yy^{-1}\leq  y$, a contradiction. Therefore $G$ is not an ordered group.
\end{example}

\index{Baumslag--Solitar's group}
The previous example is the Baumslag--Solitar group $B(1,-1)$. Recall that for $n,m\in\Z$
the Baumslag--Solitar's group is defined as the 
group \[
B(m,n)=\gr(a,b:ba^mb^{-1}=a^n).
\]
The map 
\[
\{ a,b\}\to\GL_2(\C),\quad
a\mapsto\begin{pmatrix}
1&1\\
0&1	
\end{pmatrix}\quad\text{ and }\quad  
b\mapsto\begin{pmatrix}
\frac{1}{m}&0\\
0&\frac{1}{n}	
\end{pmatrix}
\] 
induces a group homomorphism $B(m,n)\to \GL_2(\C)$.



\index{Group!poly-$\Z$}
A group $G$ is said to be {\em poly-$\Z$} if it has a finite subnormal series
\[ \{ 1\}=N_0\unlhd N_1\unlhd\dots\unlhd N_n=G\]
such that $N_{i+1}/N_{i}\cong \Z$ for all $1\leq i<n$.
By Proposition \ref{prop:LOgroup1} and induction on $n$, it is easy to see that every poly-$\Z$ group is left ordered. 

\index{Group!indicable}
\index{Group!locally indicable}
A group $G$ is said to be {\em indicable} if there exists a non-trivial 
group homomorphism $G\to\Z$, and $G$ is said to be {\em locally indicable} 
if every non-trivial finitely generated subgroup of $G$ is indicable.  

%\framebox{FIXME}
% hay que completar todo esto! 

\begin{theorem}[Burns--Hale]
\index{Burns, R. G.}
\index{Hale, V. W. D.}
\index{Burns--Hale's theorem}
Let $G$ be a group. Then $G$ is left ordered if and only if 
for each finitely generated non-trivial subgroup $H$ of $G$ there exists a left ordered group $L$ 
and a non-trivial group homomorphism $H\to L$.  
\end{theorem}

\begin{proof}
	If $G$ is a left ordered group, take $L=G$. 
	
	Conversely, suppose that for each finitely generated non-trivial subgroup $H$ of $G$ there exists a left ordered group $L$ 
and a non-trivial group homomorphism $H\to L$. We claim that for all $\{x_1,\dots,x_n\}\subseteq G\setminus\{1\}$ 
	there exist $\epsilon_1,\dots,\epsilon_n\in\{-1,1\}$ such that 
	\[
	1\not\in S(x_1^{\epsilon_1},\dots,x_n^{\epsilon_n}),
	\]
	where $S(x_1^{\epsilon_1},\dots,x_n^{\epsilon_n})$ denotes the semigroup generated by 
	the set $\{x_1^{\epsilon_1},\dots,x_n^{\epsilon_n}\}$. 
	We proceed by induction on $n$. If $n=1$, then $x_1\in G\setminus\{1\}$. Let $\epsilon_1=1$. If
	$1\in S(x_1)$, then $x_1$ is an element of finite order and hence $\langle x_1\rangle\to L$ is the trivial homomorphism for every left ordered group $L$. Hence $1\notin S(x_1)$.
    Now assume that the claim holds for some $n\geq 1$. Let $\{x_1,\dots,x_{n+1}\}\subseteq G\setminus\{1\}$. 
	By assumption, there exists a non-trivial group homomorphism 
	$h\colon\langle x_1,\dots,x_{n+1}\rangle\to L$ for some left ordered group $L$. In particular, $h(x_i)\ne 1$ for some $i\in\{1,\dots,n\}$. Without loss
	of generality we may assume that there exists an integer $1\leq k\leq n+1$ such that $h(x_j)\ne 1$ for all $j\in\{1,\dots,k\}$ and 
	$h(x_j)=1$ for all $j>k$. Suppose that $L$ is left ordered with respect to a total order $\leq$. Since $h(x_j)\ne 1$ for all $j\leq k$, there
	are elements $\epsilon_j\in\{-1,1\}$ such that $1\leq h(x_j^{\epsilon_j})$ for all $j\leq k$. By the inductive hypothesis, 
	there are elements $\epsilon_{k+1},\dots,\epsilon_{n+1}\in\{-1,1\}$ such that 
	$1\not\in S(x_{k+1}^{\epsilon_{k+1}},\dots,x_{n+1}^{\epsilon_{n+1}})$. Note that for every  $x\in S(x_1^{\epsilon_1},\dots,x_{n+1}^{\epsilon_{n+1}})\setminus S(x_{k+1}^{\epsilon_{k+1}},\dots,x_{n+1}^{\epsilon_{n+1}})$, $1\leq h(x)\neq 1$. Hence $1\notin S(x_1^{\epsilon_1},\dots ,x_{n+1}^{\epsilon_{n+1}})$, and the claim follows by induction.
	
	Consider the set 
	
	$\mathcal{F}=\{ (F,f) : F$ is a finite subset  of $G\setminus\{ 1\}$ and   $f\colon F\to\{ -1, 1\}$ such that for every finite subset $B$ of $G\setminus\{ 1\}$ containing $F$, there exists a map  $g\colon B\to \{ -1,1\}$ such that  $1\notin S(a^{g(a)} : a\in B)$ and $g(x)=f(x)$ for all $x\in F\}$. 
	
	Let $\mathcal{C}=\{ (A,f) : A\subseteq G\setminus\{ 1\}$ and $f\colon A\to \{ -1,1\}$ such that  $(F,f|_F)\in\mathcal{F}$ for all finite subset $F\text{ of }A \}$. We define and order on $\mathcal{C}$ by $(A,f)\leq (B,g)$ if and only if $A\subseteq B$ and $g(a)=f(a)$ for all $a\in A$, i. e. $f=g|_A$. Note that there is a unique map $f_{\emptyset}\colon \emptyset\to \{-1,1\}$. We have shown that $(\emptyset,f_{\emptyset})\in\mathcal{C}$. Hence $\mathcal{C}\neq \emptyset$. Furthermore, it is easy to see that every chain of elements in $\mathcal{C}$ has an upper bound in $\mathcal{C}$. Thus, by Zorn's lemma, there exists a maximal element $(A,f)\in\mathcal{C}$. Suppose that $A\neq G\setminus\{ 1\}$. Let $x\in G\setminus (A\cup\{ 1\})$. Let $g_1\colon A\cup\{ x\}\to\{ -1,1\}$ and $g_{-1}\colon A\cup\{ x\}\to\{ -1,1\}$ be the maps defined $g_i(a)=f(a)$ for all $a\in A$ and $g_i(x)=i$ for $i\in\{ -1,1\}$. By the maximality of $(A,f)$, we have that $(A\cup\{ x\}, g_i)\not\in\mathcal{C}$. Hence there exist finite subsets $F_1$ and $F_{-1}$ of $A\cup \{ x\}$ and finite subsets $B_1$ and $B_{-1}$ of $G\setminus\{ 1\}$ such that $F_1\subseteq B_1$, $F_{-1}\subseteq B_{-1}$, $1\in S(a^{h_1(a)}: a\in B_1)$ and $1\in S(a^{h_{-1}(a)}:a\in B_2)$ for all $h_1\colon B_1\to\{ -1,1\}$ and all $h_{-1}\colon B_{-1}\to \{ -1,1\}$ such that $g_i(a)=h_i(a)$ for all $a\in F_i$. Let $C=\bigcup_{i\in\{ -1,1\}}(A\cap F_i)$. Note that $C\cup\{ x\}=F_1\cup F_{-1}\subseteq B_1\cup B_{-1}$. Since $(C,f|_{C})\in \mathcal{F}$, there exists $h\colon B_1\cup B_{-1}\to\{ -1,1\}$ such that $1\notin S(a^{h(a)}: a\in B_1\cup B_{-1})$ and $h(a)=f(a)$ for all $a\in C$. Let $i=h(x)\in\{ -1,1\}$. We have that $h(x)=g_i(x)$, and thus $h(a)=g_i(a)$ for all $a\in F_i$, a contradiction because $S(a^{h(a)}:a\in B_i)\subseteq S(a^{h(a)}: a\in B_1\cup B_{-1})$.
	Therefore $A=G\setminus\{ 1\}$. 
	
	Let $P=\{a\in G\setminus \{1\} : f(a)=1\}$. Note that if $b\in G\setminus\{ 1\}$ then 
	\[1\notin S(b^{f(b)},(b^{-1})^{f(b^{-1})}),\] 
	and thus $f(b)f(b^{-1})=-1$. Hence $G$ is the disjoint union of $P$, $P^{-1}=\{ a^{-1} : a\in P\}$ and $\{ 1\}$. Note that for all $a,b\in P$, $1\notin S(a,b,(ab)^{f(ab)})$. Hence $f(ab)=1$ and thus $ab\in P$. This proves that $P$ is a subsemigroup of $G$. We define a binary relation $\leq$ on $G$ by, for all $a,b\in G$,
	\[ a\leq b\text{ if and only if }a^{-1}b\in P\cup\{ 1\}.\]
	It is straightforward to check that $\leq$ is a total order on $G$ and that $G$ is a left ordered group with respect to $\leq$.
\end{proof}

An immediate corollary:

\begin{corollary}\label{cor:LIimpliesLO}
	Locally indicable groups are left ordered groups. 
\end{corollary}

%Another consequence of Burns--Hale's Theorem:
%
%\begin{exercise}
%	Let $G$ be a group and $\{N_\alpha:\alpha\}$ be a collection of normal subgroups of $G$ such that
%	$\cap_{\alpha}N_\alpha=\{1\}$. If $G/N_{\alpha}$ is left-ordererable for all $\alpha$, then 
%	$G$ is left-orderable. 
%\end{exercise}

\section{The unique product property and diffuse groups}

\index{Group!with the unique product property}
\index{Unique product property}
A group $G$ satisfies the {\em unique product property} if for all finite non-empty subsets
$A$ and $B$ there exists $g\in G$ such that $g=ab$ for unique elements 
$a\in A$ and $b\in B$.  

\begin{proposition}
	A group with the unique product property is torsion-free.	
\end{proposition}

\begin{proof}
	Assume that $G$ has torsion and let $g\in G$ be an element of order $n\geq2$. 
	Let $A=B=\langle g\rangle$. Then $g^{i}g^{j}=g^{i+1}g^{j-1}$ for all $i,j$, and $g^{i}\neq g^{i+1}$, so $G$ cannot have the unique product property. 
\end{proof}

\begin{proposition}
	A left ordered group satisfies the unique product property.
\end{proposition}

\begin{proof}
	Let $G$ be a left ordered group with respect to a total order $\leq$ and 
    $A=\{a_1,\dots,a_n\}$ and $B=\{b_1,\dots,b_m\}$ be non-empty 
    subsets of $G$. We may assume that $a_1<a_2<\cdots<a_n$ and
    $b_1<b_2<\cdots<b_m$. Let $a_i$ be the unique element in $A$ such that $a_ib_m$ is the largest element in $AB$. Note that $a_jb_k\leq a_jb_m\leq a_ib_m$. Furthermore, if $a_jb_k=a_ib_m$, then $a_jb_k=a_jb_m$ and thus $b_k=b_m$ and $a_j=a_i$. Hence $G$ satisfies the unique product property.
\end{proof}

A group $G$ satisfies the {\em double unique product property} if for any two given finite non-empty 
subsets $A$ and $B$ of $G$ such that $|A|+|B|>2$ there exist at least two unique products in $AB$, i. e. there exist two distinct elements $g_1,g_2\in G$ such that $g_1=a_1b_1$ for unique elements $a_1\in A$ and $b_1\in B$, and $g_2=a_2b_2$ for unique elements $a_2\in A$ and $b_2\in B$. 

\begin{theorem}[Strojnowski]
	\label{thm:Strojnowski}
	\index{Strojonowski's!Theorem}
	Let $G$ be a group. the following conditions are equivalent:
	\begin{enumerate}
		\item $G$ satisfies the double unique product property.
		\item For all finite non-empty subset $A\subseteq G$ there exists at least a unique product in $AA=\{a_1a_2:a_1,a_2\in A\}$.
		\item $G$ satisfies the unique product property.
	\end{enumerate}
\end{theorem}

\begin{proof}
	The implication $(1)\implies(2)$ is trivial.  
	
	We shall prove 	$(2)\implies(3)$. Suppose that $G$ satisfies $(2)$. Let  $A,B$ be finite non-empty subsets of $G$. Let $C=BA$.  By $(2)$, there exist a unique $g\in G$ such that $g=(b_1a_1)(b_2a_2)$ for unique $b_1a_1,b_2a_2\in C$, where $a_1,a_2\in A$ and $b_1,b_2\in B$. Note that this implies that $a_1b_2=ab$ for $a\in A$ and $b\in B$ if and only if $a=a_1$ and $b=b_2$. Hence $G$ satisfies the unique product property.
	
	We shall prove $(3)\implies(1)$. Suppose that  $G$ satisfies the unique product property, but it does not satisfy the double unique product property. Thus there exist finite non-empty subsets $A,B\subseteq G$ with $|A|+|B|>2$ and there is a unique $g\in G$ such that $g=ab$ for unique elements $a\in A$ and $b\in B$.
	Let $C=a^{-1}A$ and $D=Bb^{-1}$. Then $1\in C\cap D$. Note that if $c\in C$, $d\in D$ and $cd\neq 1$, then there exist $a_1\in A$ and $b_1\in B$ such that $c=a^{-1}a_1$, $d=b_1b^{-1}$ and $ab\neq a_1b_1$. Hence there exist $a_2\in A\setminus\{ a_1\}$ and $b_2\in B\setminus\{ b_1\}$ such that $a_1b_1=a_2b_2$. Let $c_1=a^{-1}a_2$ and $d_1=b_2b^{-1}$. We have that $c\neq c_1$, $d\neq d_1$ and 
	\[ cd=a^{-1}a_1b_1b^{-1}=a^{-1}a_2b_2b^{-1}=c_1d_1.\]
	Let $E=D^{-1}C$ and $F=DC^{-1}$. Every element of $EF$ is of the form $(d_1^{-1}c_1)(d_2c_2^{-1})$, where $c_1,c_2\in C$ and $d_1,d_2\in D$. Suppose that $c_1d_2\ne 1$. We have seen that then there exist $c_3\in C\setminus \{ c_1\}$ and $d_3\in D\setminus\{ d_2\}$ such that $c_1d_2=c_3d_3$. Hence $d_1^{-1}c_3\in E\setminus \{ d_1^{-1}c_1\}$, $d_3c_2^{-1}\in F\setminus\{ d_2c_2^{-1}\}$
	and
	\[ (d_1^{-1}c_1)(d_2c_2^{-1})=(d_1^{-1}c_3)(d_3c_2^{-1}).\]
	Suppose that $c_2d_1\neq 1$. Then there exist $c_4\in C\setminus \{ c_2\}$ and $d_4\in D\setminus\{ d_1\}$ such that $c_2d_1=c_4d_4$. Hence $d_4^{-1}\cdot 1\in E\setminus \{ d_1^{-1}\cdot 1\}$, $1\cdot c_4^{-1}\in F\setminus\{ 1\cdot c_2^{-1}\}$
	and
	\[ (d_1^{-1}\cdot 1)(1\cdot c_2^{-1})=(d_4^{-1}\cdot 1)(1\cdot c_4^{-1}).\]
	Since $|C|+|D|=|A|+|B|>2$, either there exists $c\in C\setminus\{ 1\}$ or there exists $d\in D\setminus\{ 1\}$. In the first case, we have
	\[ (1\cdot 1)(1\cdot 1)=(1\cdot c)(1\cdot c^{-1}),\]
	and in the second case, we have 
		\[ (1\cdot 1)(1\cdot 1)=(d^{-1} \cdot 1)(d\cdot 1).\]
	Thus, we have found two finite non-empty subsets $E,F\subseteq G$ such that for every $e\in E$ and $f\in F$, there exist $e_1\in E\setminus\{ e\}$ and $f_1\in F\setminus\{ f\}$ such that
	$ef=e_1f_1$, a contradiction, because $G$ satisfies the unique product property. Therefore $G$ also satisfies the double unique product property.
\end{proof}


In general, it is difficult to check whether a group satisfies the unique product property. 
%Una propiedad similar es la de ser un grupo difuso. Si $G$ es un grupo
%libre de torsión y $A\subseteq G$ es un subconjunto, diremos que $A$ es
%antisimétrico si $A\cap A^{-1}\subseteq\{1\}$, donde $A^{-1}=\{a^{-1}:a\in
%A\}$. El conjunto de \textbf{elementos extremales} de $A$ se define como
%$\Delta(A)=\{a\in A:Aa^{-1}\text{ es antisimétrico}\}$. Luego
%\[
%	a\in A\setminus\Delta(A)
%	\Longleftrightarrow
%	\text{existe $g\in G\setminus\{1\}$ tal que $ga\in A$ y $g^{-1}a\in A$}.
%\]

\begin{definition}
	\index{Group!diffuse}
	\index{Extreme point}
	A group $G$ is said to be {\em diffuse} if, given any finite non-empty subset $A\subseteq G$, 
	there exists $a\in A$ such that for every $g\in G\setminus\{ 1\}$ either 
	$ga\notin A$ or $g^{-1}a\notin A$. Such an element $a\in A$ satisfying this property 
	is called an {\em extreme point} of $A$.  
\end{definition}

\begin{proposition}[Linnell,Witte Morris]
Let $G$ be a diffuse group. Then, given any finite subset $A\subseteq G$ such that $|A|\geq 2$, there exist two distinct extreme points of $A$. 
\end{proposition}

\begin{proof}
Suppose that there exists a finite subset $A\subseteq G$ with a unique extreme point $a\in A$ and $|A|\geq 2$.
Let $B=a^{-1}A$. We shall see that $1$ is the unique extreme point of $B$. Let $b\in B$ be an extreme point of $B$. Thus, for every $g\in G\setminus\{ 1\}$, either $gb\notin B$ or $g^{-1}b\notin B$. We have that $b=a^{-1}a_1$ for some $a_1\in A$. Hence either $ga^{-1}a_1\notin a^{-1}A$ or $g^{-1}a^{-1}a_1\notin a^{-1}A$, and thus, either $aga^{-1}a_1\notin A$ or $ag^{-1}a^{-1}a_1\notin A$. Hence $a_1$ is an extreme point of $A$, and thus $a_1=a$. This proves that $b=a^{-1}a_1=1$ is the only extreme point of $B$. Let $C=B\cup B^{-1}$, where $B^{-1}=\{ b^{-1} : b\in B\}$. Let $b\in B\setminus\{ 1\}$. Since $b$ is not  an extreme point of $C$, there exists $g\in G\setminus\{ 1\}$ such that $gb,g^{-1}b\in B\subseteq C$. Hence $b$ is not an extreme point of $C$. Note that
\[ (b^{-1}g^{-1}b)b^{-1}=(gb)^{-1}\in B^{-1}\subseteq C\text{ and }(b^{-1}gb)b^{-1}=(g^{-1}b)^{-1}\in B^{-1}\subseteq C.\]
Hence $b^{-1}$ is not an extreme point of $C$. Since $|B|=|A|\geq 2$, there exists $b\in B\setminus\{ 1\}$. Since $b\cdot 1,b^{-1}\cdot 1\in C$, we have that $1$ is not an extreme point of $C$. Hence $C$ has no extreme points, a contradiction. Therefore the result follows. 
\end{proof}


\begin{lemma}\label{lem:LOimpliesdiffuse}
	Let $G$ be a left ordered group. Then $G$ is diffuse.	
\end{lemma}

\begin{proof}
	Suppose that $G$ is left ordered with respect to a total order $\leq$. Let $A\subseteq G$ be a finite non-empty subset. Let $a_1\in A$ such that $a_1\leq a$ for all $a\in A$. We shall prove that $a_1$ is an extreme point of $A$. Suppose that there exists $g\in G\setminus\{ 1\}$ such that $ga_1,g^{-1}a_1\in A$. Hence $a_1\leq ga_1$ and $a_1\leq g^{-1}a_1$. Hence
	\[ g^{-1}a_1\leq g^{-1}ga_1=a_1\leq g^{-1}a_1,\]
	and thus $g^{-1}a_1=a_1$, a contradiction because $g\neq 1$. Thus $a_1$ is an extreme point of $A$. Therefore $G$ is diffuse. 
\end{proof}

\begin{lemma}
	\label{lemma:difuso=>2up}
	Let $G$ be a diffuse group. Then $G$ satisfies the unique product property.	
\end{lemma}

\begin{proof}
	Suppose that $G$ does not satisfy the unique product property. Thus, there exist finite non-empty subsets $A,B\subseteq G$ 
	such that for all $(a,b)\in A\times B$, there exists $(a_1,b_1)\in A\times B\setminus\{ (a,b)\}$ such that $ab=a_1b_1$.
	Let $C=AB=\{ ab : a\in A,\; b\in B\}$. Let $c\in C$. There exist two distinct elements $(a,b),(a_1,b_1)\in A\times B$ such that 
	$c=ab=a_1,b_1$. Note that $aa_1^{-1}\neq 1$ and
	\[ aa_1^{-1}c=aa_1^{-1}a_1b_1=ab_1\in C \text{ and } (aa_1^{-1})^{-1}c=a_1a^{-1}ab=a_1b\in C.\]
	Hence $c$ is not an extreme point of $C$. Therefore the result follows.
\end{proof}

Note that the above result shows that every diffuse group $G$ is torsion-free.


%Un grupo $G$ se dice \textbf{débilmente difuso} si para todo subconjunto
%finito $A\subseteq G$ no vacío se tiene $\Delta(A)\ne\emptyset$. La técnica
%usada para demostrar el lema~\ref{lemma:difuso=>2up} puede usarse para
%demostrar que un grupo débilmente difuso posee la propiedad del producto
%único. El teorema~\ref{theorem:Strojnowski} sugiere entonces la siguiente
%pregunta: 
%
%\begin{problem}
%	¿Existe un grupo débilmente difuso que no sea difuso?
%\end{problem}
%
%\section{El grupo de Promislow}
%
%Veremos un ejemplo concreto de un grupo sin torsión que no es ordenable, no es
%difuso y no tiene la propiedad del producto único.
%
%\begin{exercise}
%	\label{exercise:Dinfty}
%	Demuestre que $G=\langle x,y:x^2=y^2=1\rangle$ es isomorfo al grupo diedral infinito.
%\end{exercise}
%
%\begin{definition}
%	Se define el grupo de Promislow como 
%	\[
%		G=\langle x,y:x^{-1}y^2x=y^{-2},\,y^{-1}x^2y=x^{-2}\rangle.
%	\]
%\end{definition}
%
%\begin{proposition}
%	\label{proposition:Promislow}
%	El grupo de Promislow es libre de torsión y no satisface la propiedad del
%	producto único. 
%\end{proposition}
%
%\begin{proof}
%	
%\end{proof}

\section{The transfer map}

\index{Transfer map}
Let $G$ be a group and $H$ be a finite index subgroup. We will define
a group homomorphism $G\to H/[H,H]$, known as the {\em transfer map} of $G$
on $H$. 

\begin{lemma}
	\label{lem:sigma}
	Let $G$ be a group and $H$ be a subgroup of finite index $n=(G:H)$. Let
	$S=\{s_1,\dots,s_n\}$ and $T=\{t_1,\dots,t_n\}$ be left transversals of $H$ in $G$.
	If $g\in G$, there exist unique $h_{1,g},\dots,h_{n,g}\in H$ and a permutation 
	$\sigma_g\in\Sym_n$  
	such that
		\[
		gt_i=s_{\sigma_g(i)}h_{i,g},\quad
		i\in\{1,\dots,n\}.
	\]
	Furthermore, if $s_i=t_i$ for all $i$, then $\sigma_{g_1g_2}=\sigma_{g_1}\sigma_{g_2}$ and $h_{i,g_1g_2}=h_{\sigma_{g_2}(i),g_1}h_{i,g_2}$ for all $g_1,g_2\in G$ and all $i$.
\end{lemma}

\begin{proof}
	If $i\in\{1,\dots,n\}$, then there exists a unique $j\in\{1,\dots,n\}$ such that $gt_i\in
	s_jH$. Thus there exists a unique $h_{i,g}\in H$ such that $gt_i=s_jh_{i,g}$. Take 
	$\sigma_g(i)=j$ and thus there is a well-defined map 
	$\sigma_g\colon\{1,\dots,n\}\to\{1,\dots,n\}$.  To prove that 
	$\sigma_g\in\Sym_n$ it is enough to check that $\sigma_g$ is injective. If
	$\sigma_g(i)=\sigma_g(k)=j$, since $gt_i=s_jh_{i,g}$ and $gt_k=s_jh_{k,g}$, it follows that 
	$t_i^{-1}t_k=h_{i,g}^{-1}h_{k,g}\in H$. Hence $i=k$, as $t_iH=t_kH$.
	
	Suppose that $s_i=t_i$ for all $i$. Let $g_1,g_2\in G$. We have that
	\[t_{\sigma_{g_1g_2}(i)}h_{i,g_1g_2}=g_1g_2t_i=g_1t_{\sigma_{g_2}(i)}h_{i,g_2}=t_{\sigma_{g_1}\sigma_{g_2}(i)}h_{\sigma_{g_2}(i),g_1}h_{i,g_2}.\]
	Hence $\sigma_{g_1g_2}(i)=\sigma_{g_1}\sigma_{g_2}(i)$ and  $h_{i,g_1g_2}=h_{\sigma_{g_2}(i),g_1}h_{i,g_2}$, thus the result follows.
\end{proof}

%\begin{exercise} 
%	Demuestre que la acción de $G$ en el conjunto de coclases $H\backslash G$
%	dada por $(Hx)\cdot g=H(xg)$ induce una acción a derecha de $G$ en $T$. 
%\end{exercise}
%
%\begin{svgraybox}
%	Si $t\in T$ y $g\in G$ existe un único $t\cdot g\in T$ tal que $(Ht)\cdot
%	g=H(t\cdot g)$. Como $G$ actúa en $H\backslash G$ por multiplicación a
%	derecha, $H(t\cdot (g_1g_2))=H((t\cdot g_1)\cdot g_2)$ para todo $t\in T$,
%	$g_1,g_2\in G$.
%\end{svgraybox}
%
%\begin{exercise} 
%	Demuestre que si $t\in T$ y $g\in G$ entonces $tg(t\cdot g)^{-1}\in H$.
%\end{exercise}
%
%\begin{svgraybox}
%	Sean $t\in T$ y $g\in G$. Entonces $t\cdot g\in H$ es el único elemento de
%	$H$ tal que $H(tg)=H(t\cdot g)$. Luego $(tg)(t\cdot g)^{-1}\in H$.
%\end{svgraybox}


Let $G$ be a group and $H$ be a subgroup of $G$ of finite index $n$. If
$T=\{t_1,\dots,t_n\}$ is a left transversal of $H$ in $G$, we define the map 
	\[
		\nu_T\colon G\to H/[H,H],\quad
		\nu_T(g)=\prod_{i=1}^n \bar h_i
	\]
where $gt_i=t_jh_i$ and $\bar h=h[H,H]\in H/[H,H]$. Note that the product is well-defined since $H/[H,H]$ is an abelian group. 
We now prove that the map does not depend on the 
transversal. 

\begin{lemma}
	\label{lem:nu_T}
	Let $G$ be a group and $H$ be a subgroup of $G$ of finite index. if $T$ 
	and $S$ are left transversals of $H$ in $G$, then $\nu_T=\nu_S$.
\end{lemma}

\begin{proof}
	Assume that $n=(G:H)$, $T=\{ t_1,\dots ,t_n\}$ and $S=\{ s_1,\dots ,s_n\}$, where $s_i=t_ik_i$, for some $k_i\in H$. Let $g\in G$. There exist $\sigma_{g}\in \Sym_n$ and $h_{i,g}\in H$ such that $gs_i=s_{\sigma_g(i)}h_{i,g}$ for all $i$. Let $l_{i,g}=k_{\sigma_g(i)}h_{i,g}k_i^{-1}\in H$. Then 
	\[
	gt_i=gs_ik_i^{-1}=s_{\sigma_g(i)}h_{i,g}k_i^{-1}=t_{\sigma_g(i)}k_{\sigma_g(i)}h_{i,g}k_i^{-1}=t_{\sigma_g(i)}l_{i,g}
	\]
	for all $i\in\{1,\dots,n\}$. Moreover,  
	\[
			s_{\sigma_g(i)}^{-1}gs_i=k_{\sigma_g(i)}^{-1}t_{\sigma_g(i)}^{-1}gt_ik_i.
	\]
	Since $H/[H,H]$ is abelian, 
	\begin{align*}
		\nu_S(g)
		&=\prod_{i=1}^n \overline{s_{\sigma_g(i)}^{-1}gs_i}
		=\prod_{i=1}^n \bar k_{\sigma_g(i)}^{-1}\overline{t_{\sigma_g(i)}^{-1}gt_i}\bar k_i\\
		&=\prod_{i=1}^n \bar k_{\sigma(i)}^{-1}\prod_{i=1}^n \bar k_i\prod_{i=1}^n \overline{t_{\sigma(i)}^{-1}gt_i}
		=\prod_{i=1}^n \overline{t_{\sigma(i)}^{-1}gt_i}
		=\nu_T(g).\qedhere
	\end{align*}
\end{proof}

By Lemma~\ref{lem:nu_T}, if $H$ is a finite-index subgroup of $G$, the map 
\[
\nu\colon G\to H/[H,H],
\quad
\nu(g)=\nu_T(g),
\]
where $T$ is some left transversal of $H$ in $G$, is well-defined. 

\begin{theorem}
	\label{theorem:transfer}
	Let $G$ be a group and $H$ be a finite-index subgroup of $G$. Then $\nu(xy)=\nu(v)\nu(y)$ 
	for all $x,y\in G$.
\end{theorem}

\begin{proof}
	Let $T=\{t_1,\dots,t_n\}$ be a left transversal of $H$ in $G$, where $n=(G:H)$.  By 
	Lemma~\ref{lem:sigma}, for every $g\in G$ there exist unique elements $h_{1,g},\dots,h_{n,g}\in H$ and 
	a permutation $\sigma_g\in\Sym_n$ such that $gt_i=t_{\sigma_g(i)}h_{i,g}$. Furthermore $\sigma_{xy}=\sigma_x\sigma_y$ and $h_{i,xy}=h_{\sigma_y(i),x}h_{i,y}$ for all $x,y\in G$ and all $i$. Since $H/[H,H]$ is abelian, 
	\[
		\nu(xy)=\prod_{i=1}^nh_{i,xy}=\prod_{i=1}^n h_{\sigma_y(i),x}h_{i,y}=\prod_{i=1}^n h_{\sigma_y(i),x}\prod_{i=1}^n h_{i,y}=\nu(x)\nu(y).\qedhere
	\]
\end{proof}

If $G$ is a group and $H$ is a finite-index subgroup of $G$, the 
{\em transfer homomorphism} is the group homomorphism $\nu\colon G\to H/[H,H]$,
$\nu(g)=\nu_T(g)$, for some left transversal $T$ of $H$ in $G$.
	
%\begin{theorem}
%	\label{theorem:P_noabeliano}
%	Sea $G$ un grupo finito. Sea $p$ un primo que divide al orden de $[G,G]\cap
%	Z(G)$. Si $P\in\Syl_p(G)$ entonces $P$ es no abeliano.
%\end{theorem}
%
%\begin{proof}
%	Supongamos que $P$ es abeliano y sea $T=\{t_1,\dots,t_n\}$ un transversal
%	de $P$ en $G$. Como $[G,G]\cap Z(G)$ es un subgrupo normal de $G$, podemos
%	suponer que $P\cap [G,G]\cap Z(G)\ne1$. Sea $z\in P\cap [G,G]\cap Z(G)$ tal
%	que $z\ne1$. 
%
%	Sea $\nu\colon G\to P$ el morfismo de transferencia. Vamos a calcular
%	$\nu(z)$ con el lema~\ref{lemma:sigma}. Para cada $i\in\{1,\dots,n\}$ sean
%	$x_1,\dots,x_n\in P$ y sea $\sigma\in\Sym_n$ tales que
%	$zt_i=t_{\sigma(i)}x_i$. Como $z\in Z(G)$, se tiene
%	$t_i=t_{\sigma(i)}x_iz^{-1}$ y luego la unicidad del lema~\ref{lemma:sigma}
%	implica que $\sigma=\id$ y $x_i=z$ para todo $i$. Luego 
%	\[
%	\nu(z)=z^{|T|}=z^{(G:P)}. 
%	\]
%
%	Como $P$ es abeliano, $[G,G]\subseteq\ker\nu$. Luego $\nu(z)=1$. Esto es
%	una contradicción pues $1\ne z\in P$ y $z^{(G:P)}=1$ implica que $z$ tiene
%	orden no divisible por $p$. 
%\end{proof}

% rotman 7.47, 7.48

\begin{lemma}
	\label{lem:evaluation}
	Let $G$ be a group and $H$ be a subgroup of $G$ with $(G:H)=n$. Let 
	$T=\{t_1,\dots,t_n\}$ be a left transversal of $H$ in $G$.  
	For each $g\in G$ there exist a positive integer
	$m$,  $m$ distinct
	elements $s_{1},\dots,s_{m}\in T$ 
	and positive integers $n_1,\dots,n_m$
	such that 
	\[
	s_i^{-1}g^{n_i}s_i\in H,
	\quad
	n_1+\cdots+n_m=n\quad\text{and}\quad   
	\nu(g)=\prod_{i=1}^m s_i^{-1}g^{n_i}s_i.
	\]
\end{lemma}

\begin{proof}
	Let $g\in G$. For each $i$ there exist $h_1,\dots,h_n\in H$ and $\sigma\in\Sym_n$ such that 
	$gt_i=t_{\sigma(i)}h_i$. Write $\sigma$ as a product 
	\[
		\sigma=\alpha_{k+1}\cdots\alpha_m
	\]
	of disjoint cycles and $|\{ j :\sigma(j)=j\}|=k$. Note that if $\sigma(j)=j$, then $gt_j=t_jh_j$, and thus $t_j^{-1}gt_j\in H$. Thus we take $\{ s_1,\dots, s_k\}=\{ t_j :\sigma(j)=j\}$ and $n_1=\dots =n_k=1$.  

	Fix $i\in\{k+1,\dots,m\}$ and write  
	$\alpha_i=(j_{1}\cdots j_{n_i})$. Since  
	\[
		g t_{j_k}=t_{\sigma(j_k)}h_{j_k}=\begin{cases}
			t_{j_1}h_{n_k} & \text{if $k=n_i$},\\
			t_{j_{k+1}}h_{k} & \text{otherwise},
		\end{cases}
	\]
	it follows that 
	\begin{align*}
	t_{j_1}^{-1}g^{n_i}t_{j_1}
	&=t_{j_1}^{-1}g^{n_i-1}gt_{j_1}\\
	&=t_{j_1}^{-1}g^{n_i-1}t_{j_2}h_{j_1}\\
	&=t_{j_1}^{-1}g^{n_i-2}gt_{j_2}h_{j_1}\\
	&=t_{j_1}^{-1}g^{n_i-2}t_{j_3}h_{j_2}h_{j_1}\\
	&\phantom{=}\vdots\\
	&=t_{j_1}^{-1}gt_{j_{n_i}}h_{j_{n_{i}-1}}\cdots h_{j_2}h_{j_1}\\
	&=t_{j_1}^{-1}t_{j_1}h_{j_{n_i}}\cdots h_{j_2}h_{j_1}\in H. 	
	\end{align*}
	So we let $s_i=t_{j_1}\in T$. Now the claim follows, 
	since $\nu(g)=\bar h_1\cdots \bar h_{n}$.
\end{proof}

\begin{proposition}
	\label{prop:v(g)=g^n}
	Let $G$ be a group and $H$ be a central subgroup of index $n$. Then 
	$\nu(g)=g^n$ for all $g\in G$.
\end{proposition}

\begin{proof}
	Let $g\in G$. Let $T$ be a left transversal of $H$ in $G$. By Lemma~\ref{lem:evaluation}, there exist $s_1,\dots,s_m\in
	T$ and positive integers $n_1,\dots ,n_m$ such that $\sum_{i=1}^mn_i=n$, $s_i^{-1}g^{n_i}s_i\in H$ and $\nu(g)=\prod_{i=1}^m
	s_i^{-1}g^{n_i}s_i$.  Since $H$ is central in $G$, then it is normal in $G$. Thus 
	\[
	g^{n_i}=s_i(s_i^{-1}g^{n_i}s_i)s_i^{-1}\in H\subseteq Z(G)
	\]
	and hence  
	\[
		\nu(g)
		=\prod_{i=1}^m s_i^{-1}g^{n_i}s_i
		=\prod_{i=1}^m g^{n_i}
		=g^{\sum_{i=1}^m n_i}
		=g^n.\qedhere
	\]
\end{proof}
%
%\begin{exercise}
%	Let $G$ be a group with a central subgroup $H$ of index $n$. Then 
%	$g\mapsto g^n$ is a group homomorphism. 
%\end{exercise}

The next result is an application to finite groups. 

\begin{proposition}
	\label{prop:semidirecto}
	Let $G$ be a finite group and $H$ a central subgroup of index $n$, where 
	$n$ is coprime with $|H|$. Then
	$G\cong N\times H$.
\end{proposition}

\begin{proof} Let $\nu \colon G\to H$ be the transfer homomorphism.
	 By Proposition~\ref{prop:v(g)=g^n}, 
		$\nu(g)=g^n$ for all $g\in G$. In particular, since the order of $h\in H$ and $n$ are coprime, there exists a positive integer $k$ such that $h^{kn}=h$. Hence $\nu$ is surjective. Let $N=\ker \nu$. Then $|G|=|N||H|=n|H|$. Hence $|N|=n$. Since $n$ and $|H|$ are coprime, we have that $N\cap H=\{ 1\}$ and $G=NH\cong N\times H$, because $N$ and $H$ are normal subgroups of $G$. 
\end{proof}


%\begin{proof}
%	Es consecuencia inmediata del corolario~\ref{corollary:semidirecto} pues
%	$H$ es normal por ser un subgrupo central.
%\end{proof}

% TODO: Transitivity of the transfer

% serre, 7.12
An application to infinite groups. 

\begin{theorem}\label{thm:Zfiniteindex}
	Let $G$ be a torsion-free group that contains a finite-index subgroup isomorphic to  
	$\Z$. Then $G\simeq\Z$.
\end{theorem}

\begin{proof}
	We may assume that $G$ contains a finite-index normal subgroup isomorphic to $\Z$. Indeed, 
	if $H$ is a finite-index subgroup of $G$ such that $H\simeq\Z$, and $T$ is a left transversal of $H$ in $G$, then 
	$K=\bigcap_{x\in T}xHx^{-1}$ is a finite-index normal subgroup of $G$. Since $K\subseteq H\subseteq G$  and $(G:K)=(G:H)(H:K)$ is finite, we have that $K\cong\Z$.
	The action of $G$ on $K$ by conjugation induces a group homomorphism  
	$\epsilon\colon G\to\Aut(K)$. Since $\Aut(K)\cong\Aut(\Z)=\{-1,1\}$, 
	there are two cases to consider.
	
	Assume first that $\epsilon(g)=\id$ for all $g\in G$. In this case, $K\subseteq Z(G)$, let
	$\nu\colon G\to K$ be the transfer homomorphism. By
	Proposition~\ref{prop:v(g)=g^n}, $\nu(g)=g^n$, where $n=(G:K)$. Since
	$G$ has no torsion, $\nu$ is injective. Thus
	$G\cong\Z$ because it is isomorphic to a non-trivial subgroup of $K$.

	Assume now that there exists $g\in G$ such that $\epsilon(g)\ne\id$. Let $N=\ker\epsilon\ne G$. Since
	$K\cong\Z$ is abelian, $K\subseteq N$. The result proved in the previous paragraph 
	applied to $\epsilon|_N$ implies that $N\cong\Z$, as 
	$N$ contains a finite-index subgroup isomorphic to $\Z$. Let $g\in G\setminus N$. 
	Since $N$ is normal in $G$, $G$ acts by conjugation on $N$ and hence 
	there exists a group homomorphism $c_g\in\Aut(N)\simeq\{-1,1\}$ defined by $c_g(n)=gng^{-1}$ for all $n\in N$. Since
	$K\subseteq N$ and $g$ acts non-trivially on $K$, 
	\[
	c_g(n)=gng^{-1}=n^{-1}
	\]
	for all $n\in N$.  Since 
	$g^2\in N$, 
	\[
		g^2=gg^2g^{-1}=g^{-2}.
	\]
	Therefore $g^4=1$, a contradiction since $g\ne1$ and $G$ has no torsion. Therefore $\epsilon(g)=\id$ for all $g\in G$ and the result follows.
\end{proof}

\section{Bieberbach groups}

We begin this section with an elementary property of the subgroups of finite index. 

\begin{lemma} \label{lem:fgfisubgroup}
Let $G$ be a group.  Let $H$ be a subgroup of finite index of $G$. Then $H$ is finitely generated if and only if $G$ is finitely generated.
\end{lemma}

\begin{proof}
Suppose that $H$ is finitely generated. Let $A$ be a finite subset of $H$ such that $H=\langle A\rangle$.  
Let $T$ be a left transversal of $H$ in $G$. It is clear that $G=\langle T\cup A\rangle$, and thus $G$ is finitely generated.

Conversely, suppose that $G$ is finitely generated. Assume that  $1\in T$. Let $X$ be a finite subset of $G$ such that $G=\langle X\rangle$. We assume that $x\in X$ if and only if $x^{-1}\in X$. Let $F=T\cup X=\{ f_1,f_2,\dots ,f_n\}$, with $f_1=1$. For $f_i,f_j\in F$, there exist $t_{i,j}\in T$ and $h_{i,j}\in  H$ such that $f_if_j=t_{i,j}h_{i,j}$. Let $W=\langle h_{i,j} : 1\leq i,j\leq n\rangle$.  Let $g\in G$. There exist $f_{i_1},\dots ,f_{i_m}\in X\subseteq F$ such that $g=f_{i_1}\cdots f_{i_m}$. We shall prove that $g\in\bigcup_{t\in T}tW$, by induction on $m$. If $m=1$, then $g=f_{i_1}=t_{i_1,1}h_{i_1,1}\in \bigcup_{t\in T}tW$. Suppose that $m>1$ and that every product of $m-1$ elements of $F$ belongs to $\bigcup_{t\in T}tW$. We have that
\begin{align*}
    g=&f_{i_1}\cdots f_{i_m}\\
    =&f_{i_1}\cdots f_{i_{m-2}}t_{i_{m-1},i_m}h_{i_{m-1},i_m}\in\bigcup_{t\in T}tW,
\end{align*}
by the inductive hypothesis. By induction, $G=\bigcup_{t\in T}tW$. Since
$tW\subseteq tH$ and $G=\bigcup_{t\in T}tH$ is a disjoint union, we have that $tW=tH$ for all $t\in T$. In particular $H=W$ is finitely generated.
\end{proof}

The following result is due to Schur.
\begin{lemma}\label{lem:Schurcenter}
Let $G$ be a group such that $(G:Z(G))=n<\infty$. Then $[G,G]$ is finite and $|[G,G]|\leq n^{n(n-1)}$.
\end{lemma}

\begin{proof}
Let $\nu \colon G\to Z(G)$ be the transfer homomorphism. By Proposition \ref{prop:v(g)=g^n}, $\nu(g)=g^n$ for all $g\in G$. Since $[G,G]\in\ker\nu$, we have that $h^n=1$ for all $h\in [G,G]$. Let $T=\{ t_1,\dots ,t_n\}$ be a left transversal of $Z(G)$ in G. Let
\[ A=\{ [g,h] : g,h\in G\}.\]
Note that
\[ A=\{ [t_i,t_j] : 1\leq i,j\leq n\} .\]
Since $[t_i,t_j]^{-1}=[t_j,t_i]$, every element of $[G,G]$ is a product of  elements of $A$. Note that $|A\setminus \{ 1\}|=k\leq n(n-1)$. Let
$A\setminus\{ 1\}= \{ a_1,\dots ,a_k\}$. We shall prove that
\[ [G,G]=\{ a_1^{n_1}\cdots a_k^{n_k} : 0\leq n_i<n \text{ for all } i=1,\dots ,k\}.\]
Let $B=\{ a_1^{n_1}\cdots a_k^{n_k} : 0\leq n_i<n \text{ for all } i=1,\dots ,k\}$. Let $h\in [G,G]$. There exist $a_{i_1},\dots , a_{i_m}\in A\setminus\{ 1\}$ such that $h=a_{i_1}\cdots a_{i_m}$. We shall prove that $h=a_1^{n_1}\cdots a_k^{n_k}\in B$, for some non-negative integers $n_1,\dots ,n_k$ such that $n_1+\dots +n_k\leq m$, by induction on $m$. For $m=1$, it is clear
that $h=a_{i_1}\in B$. Suppose that $m>1$ and that
$a_{j_1}\cdots a_{j_l}=a_1^{n_1}\cdots a_k^{n_k}\in B$, for some non-negative integers $n_1,\dots ,n_k<n$ such that $n_1+\dots +n_k\leq l$, for all $l<m$. Since 
\begin{equation}\label{eq:Bieberbach1}
[x,y][u,v]=[u,v][u,v]^{-1}[x,y][u,v]=[u,v][[u,v]^{-1}x[u,v],[u,v]^{-1} y[u,v]]\end{equation}
for all $x,y,u,v\in G$, we may assume that $i_1$ is the smallest positive integer in all the representations of $h$ as a product of $m$ elements in $A\setminus\{ 1\}$. By the inductive hypothesis
\[a_{i_2}\cdots a_{i_m}=a_1^{n_1}\cdots a_k^{n_k}\in B,\]
for some non-negative integers $n_1,\dots ,n_k<n$ such that $n_1+\dots +n_k\leq m-1$. Hence, by the choice of $i_1$ and (\ref{eq:Bieberbach1}), we have that $h=a_{i_1}a_1^{n_1}\cdots a_k^{n_k}=a_{i_1}^{n_{i_1}+1}a_{i_1+1}^{n_{i_1+1}}\cdots a_k^{n_k}$. Note that $n_{i_1}+1\leq n$ and  $n_{i_1}+1+n_{i_1+1}+\dots +n_k\leq m$.
If $n_{i_1}+1<n$, then clearly $h\in B$. If $n_{i_1}+1=n$, then
\[ h=a_{i_1+1}^{n_{i_1+1}}\cdots a_k^{n_k}\in B.\]
Hence, by induction, $[G,G]=B$. Clearly $|[G,G]|\leq n^k\leq n^{n(n-1)}$, and the result follows.
\end{proof}

\index{Group|$FC$-center}
The {\em $FC$-center} of a group $G$ is the set
\[ \Delta(G)=\{ g\in G : (G:C_G(g))<\infty\} .\]
Note that $\Delta(G)$ is the set of all elements in $G$ that have finitely many conjugates, i. e.
\[ \Delta(G)=\{ g\in G : |\{xgx^{-1} : x\in G\} |<\infty\}.\]


\begin{lemma}\label{lem:Delta(G)}
Let $G$ be a group. Then $\Delta(G)$ and $T(\Delta(G))=\{ g\in\Delta(G) : o(g)<\infty\}$
are a characteristic  subgroups of $G$ and $\Delta(G)/T(\Delta(G))$ is an abelian torsion-free group.
\end{lemma}

\begin{proof}
Note that $1\in T(\Delta(G))$. Let $g,h\in \Delta(G)$. Then $xgh^{-1}x^{-1}=(xgx^{-1})(xhx^{-1})^{-1}$ for all $x\in G$. Since $g$ and $h$ have finitely many conjugates, we get that $gh^{-1}$ also have finitely many conjugates.  Hence $\Delta(G)$ is a subgroup of $G$. Let $\alpha\in \Aut(G)$. Then $x\alpha(g)x^{-1}=\alpha(\alpha^{-1}(x)g\alpha^{-1}(x)^{-1})$, for all $x\in G$. Hence $\alpha(g)\in \Delta(G)$ and thus $\Delta(G)$ is a characteristic subgroup of $G$. Let $H=\langle g,h\rangle$. Note that $\Delta(H)=H$ and $Z(H)=C_H(g)\cap C_H(h)$. Thus $(H:Z(H))<\infty$. By Lemma \ref{lem:Schurcenter}, $[H,H]$ is finite. Hence $T(H)$ is a finite subgroup. In particular, if $g,h\in T(\Delta(G))$, then $\langle g,h\rangle\subseteq T(\Delta(G))$, and this shows that $T(\Delta(G))$ is a subgroup of $G$. Clearly $T(\Delta(G))$ is characteristic in $\Delta(G)$. Since $\Delta (G)$ is characteristic in $G$, it follows that $T(\Delta(G))$ is characteristic in $G$. Since $[g,h]\in T(\Delta(G))$ for all $g,h\in \Delta(G)$, it is clear that $\Delta(G)/T(\Delta(G))$ is torsion-free abelian.  
\end{proof}
\index{Group!Bieberbach}

A group $G$ is said to be a {\em Bieberbach} group if it is a finitely generated torsion-free group with an abelian group of finite index.


\begin{proposition}\label{prop:Bieberbach2}
Let $G$ be a Bieberbach group. Then $\Delta(G)$ is a torsion-free finitely generated abelian normal subgroup of $G$ of finite index. Futhermore $A\subseteq \Delta(G)$ for every abelian subgroup $A$ of $G$ of finite index.
\end{proposition}

\begin{proof}
Let $A$ be an abelian subgroup of $G$ of finite index. Note that for every $a\in A$, $A\subseteq C_G(a)$. Hence $A\subseteq \Delta(G)$. By Lemmas \ref{lem:Delta(G)} and \ref{lem:fgfisubgroup},  the result follows. 
\end{proof}

The {\em dimension} of a Bieberbach group $G$ is the rank of $\Delta(G)$.

\begin{lemma}\label{lem:SubBieberbeach}
Let $G$ be a Bieberbach group. Then every subgroup $H$ of $G$ is a Bieberbach group and $\dim(H)\leq\dim(G)$.
\end{lemma}

\begin{proof}
Let $H$ be a subgroup of $G$. Then $(H: H\cap\Delta(G))<\infty$. Let $n=\dim(G)$. Thus $\Delta(G)\cong \Z^n$. Hence $H\cap\Delta(G)$ is a finitely generated subgroup of $\Delta(G)$, and thus $H\cap \Delta(G)\cong\Z^m$ for some $m\leq n$. Since $(H:H\cap\Delta(G))<\infty$ and $H\cap\Delta(G)$ is finitely generated, we have that $H$ also is finitely generated by Lemma \ref{lem:fgfisubgroup}. Hence $H$ is a Bieberbach group. Note that $H\cap\Delta(G)\subseteq \Delta(H)$. Since $(\Delta(H):H\cap\Delta(G))<\infty$, the rank of $\Delta(H)$ is $m$. Thus $\dim(H)=m\leq n=\dim(G)$.  
\end{proof}

\begin{lemma}\label{lem:semidirectBieberbach}
Let $G$ be a group with a torsion-free abelian normal subgroup $A$ of finite index. Then $G$ is isomorphic to a subgroup $H$ of the semidirect product $A\rtimes_{\alpha}G/A$, where the action $\alpha$ is naturally induced by conjugation. Furthermore
\[ H\cap (A\times\{ A\})=\{a^n : a\in A\}\times\{ A\},\]
where $n=(G:A)$.
\end{lemma}

\begin{proof}
Let $n=(G:A)$ and let $T$ be a transversal of $A$ in $G$ such that $1\in T$. For each $g\in G$ there exist a unique $a_g\in A$ and a unique $t_g\in T$ such that $g=a_gt_g$. We denote $a_{t_gt_h}$ by $c(t_g,t_h)$ for all $g,h\in G$. Hence
\[ t_gt_h=c(t_g,t_h)t_{gh}\]
for all $g,h\in G$. Note that $c(t_1,t)=c(1,t)=1=c(t,1)=c(t,t_1)$ for all $t\in T$. The associativity of $G$ implies that
\[c(t_{g_1},t_{g_2})c(t_{g_1g_2},t_{g_3})=t_1c(t_{g_2},t_{g_3})t_1^{-1}c(t_{g_1},t_{g_2g_3})\]
for all $g_1,g_2,g_3\in G$. Since $A$ is abelian and $n=(G:A)$, multiplying this equality for all $g_3\in T$, we have that
\begin{equation}\label{eq:SemiB1}
c(t_{g_1},t_{g_2})^n\prod_{t\in T}c(t_{g_1g_2},t)=t_{g_1}\left(\prod_{t\in T}c(t_{g_2},t)\right)t_{g_1}^{-1}\prod_{t\in T}c(t_{g_1},t).
\end{equation}
for all $g_1,g_2\in G$. Let $f\colon G\to A\rtimes_{\alpha} G/A$ be the map defined by
\[ f(g)=f(a_gt_g)=\left( a_g^n\prod_{t\in T}c(t_g,t), gA\right)\]
for all $g\in G$. It is clear that $f(g)\in A\times\{ A\}$ if and only if $t_g=1$, and in this case $f(g)=(g^n,A)$. Note that for $g_1,g_2\in G$ we have
\begin{align*}
    f(g_1g_2)=&f(a_{g_1}t_{g_1}a_{g_2}t_{g_2})\\
    =&f(a_{g_1}t_{g_1}a_{g_2}t_{g_1}^{-1}c(t_{g_1},t_{g_2})t_{g_1g_2})\\
    =&\left(a_{g_1}^nt_{g_1}a_{g_2}^nt_{g_1}^{-1}c(t_{g_1},t_{g_2})^n\prod_{t\in T}c(t_{g_1g_2},t),g_1g_2A\right)
\end{align*}
and
\begin{align*}
    f(g_1)f(g_2)=&f(a_{g_1}t_{g_1})f(a_{g_2}t_{g_2})\\
    =&\left(a_{g_1}^n\prod_{t\in T}c(t_{g_1},t),g_1A\right)\left(a_{g_2}^n\prod_{t\in T}c(t_{g_2},t),g_2A\right)\\
    =&\left(a_{g_1}^n\prod_{t\in T}c(t_{g_1},t)t_{g_1}a_{g_2}^nt_{g_1}^{-1}t_{g_1}\left(\prod_{t\in T}c(t_{g_2},t)\right)t_{g_1}^{-1},g_1g_2A\right).
\end{align*}
By (\ref{eq:SemiB1}), we get that $f(g_1g_2)=f(g_1)f(g_2)$. Hence $f$ is a homomorphism of groups. Let $g\in \ker f$. Then $gA=A$, and thus $t_g=1$. Hence
\[ (1,A)=f(g)=\left( a_g^n\prod_{t\in T}c(1,t), A\right)=(a_{g}^n,A),\]
and, since $A$ is torsion-free, we get that $a_g=1$. Hence $f$ is injective and the result follows.
\end{proof}

We denote by $\Isom(\R^n)$ the group of isometries of the euclidean space $\R^n$. Let $T(\R^n)$ be the subgroup of translations of $\Isom(\R^n)$ and let $O(\R^n)$ be the subgroup of orthogonal automorphism of $\R^n$. It is known that $T(\R^n)$ is normal in $\Isom(\R^n)$, $\Isom(\R^n)=T(\R^n)O(\R^n)$ and $T(\R^n)\cap O(\R^n)=\{\id\}$. Hence every $g\in \Isom(\R^n)$ can be written uniquely as $g=tf$ with $t\in T(\R^n)$ and $f\in O(\R^n)$.

\begin{theorem}\label{thm:GeomBieberbach}
Let $G$ be an $n$-dimensional Bieberbach group. Then $G$ is isomorphic to a subgroup $H$ of the group $\Isom(\R^n)$ such that $\Delta(G)$ corresponds to the translations of $H$, that is $\Delta(H)=H\cap T(\R^n)$. 
\end{theorem}

\begin{proof} By Lemma \ref{lem:semidirectBieberbach}, it is enough to show that $\Delta(G)\rtimes G/\Delta(G)$ is isomorphic to a subgroup $H$ of $\Isom(\R^n)$ and $\Delta(H)=H\cap T(\R^n)$.
Since $G$ is an $n$-dimensional Bieberbach group, there is an isomorphism $f\colon \Delta(G)\to \Z^n$. Let $\sigma\colon G\to \Aut(\Z^n)$ be the map defined by $\sigma(g)=\sigma_g$ and $\sigma_g(a_1,\dots a_n)=f(gf^{-1}(a_1,\dots ,a_n)g^{-1})$ for all $g\in G$ and $(a_1,\dots ,a_n)\in\Z^n$. It is straightforward to check that $\sigma_g\in\Aut(\Z^n)$ and that $\sigma$ is a homomorphism of groups.
Note that $\ker\sigma=C_G(\Delta(G))$. By Proposition \ref{prop:Bieberbach2}, $\Delta(G)=C_G(\Delta(G))$. Hence $\sigma(G)\cong G/\Delta(G)$. Note that every automorphism of $\Z^n$ can be naturally extended to an automorphism of $\R^n$. Hence we can think $\sigma(G)$ as a finite subgroup of $\Aut(\R^n)$. 

Let $\langle u\mid v\rangle$ be the standard inner product of $u,v\in\R^n$. We define a new inner product on $\R^n$ by
\[ \langle u\mid v\rangle '=\sum_{g\in \sigma(G)}\langle g(u)\mid g(v)\rangle\]
for all $u,v\in\R^n$. Let $e'_1,\dots ,e'_n$ be an orthonormal basis for $\langle .\mid .\rangle '$. Let $\alpha\in \Aut(\R^n)$ be the map defined by $\alpha(a_1,\dots ,a_n)=\sum_{i=1}^na_ie'_i$. Note that
\[ \langle u\mid v\rangle=\langle \alpha (u)\mid \alpha (v)\rangle '\]
for all $u,v\in\R^n$. If $g\in \sigma(G)$, then
\begin{align*}
    \langle \alpha^{-1}g\alpha(u), \alpha^{-1}g\alpha(v)\rangle =&
    \langle g\alpha(u), g\alpha(v)\rangle '\\
    =&\langle \alpha(u), \alpha(v)\rangle '=\langle u, v\rangle 
\end{align*}
for all $u,v\in \R^n$. Hence $\alpha^{-1}g\alpha$ is in the orthogonal group of $\R^n$ for all $g\in \sigma(G)$.

Let $\Psi\colon \Delta(G)\rtimes G/\Delta(G)\to\Isom(\R^n)$ be the map defined by
\[ \Psi(h,g\Delta(G))(a_1,\dots ,a_n)=\alpha^{-1}(f(h))+\alpha^{-1}\sigma(g)\alpha(a_1,\dots ,a_n)\]
for all $h\in \Delta(g)$, $g\in G$ and $(a_1,\dots ,a_n)\in\R^n$. For $h_1,h_2\in \Delta(G)$, $g_1,g_2\in G$ and $(a_1,\dots ,a_n)\in \R^n$, we have that
\begin{align*}
    \Psi(h_1,g_1\Delta(G))&\Psi(h_2,g_2\Delta(G))(a_1,\dots ,a_n)\\
    =&\Psi(h_1,g_1\Delta(G))(\alpha^{-1}(f(h_2))+\alpha^{-1}\sigma(g_2)\alpha(a_1,\dots ,a_n))\\
    =&\alpha^{-1}(f(h_1))+\alpha^{-1}\sigma(g_1)\alpha(\alpha^{-1}(f(h_2))+\alpha^{-1}\sigma(g_2)\alpha(a_1,\dots ,a_n))\\
    =&\alpha^{-1}(f(h_1))+\alpha^{-1}\sigma(g_1)(f(h_2))+\alpha^{-1}\sigma(g_1)\sigma(g_2)\alpha(a_1,\dots ,a_n))\\
    =&\alpha^{-1}(f(h_1))+\alpha^{-1}(f(g_1h_2g_1^{-1}))+\alpha^{-1}\sigma(g_1g_2)\alpha(a_1,\dots ,a_n))\\
    =&\alpha^{-1}(f(h_1g_1h_2g_1^{-1}))+\alpha^{-1}\sigma(g_1g_2)\alpha(a_1,\dots ,a_n))\\
    =&\Psi(h_1g_1h_2g_1^{-1},g_1g_2\Delta(G))(a_1,\dots ,a_n)\\
    =&\Psi((h_1,g_1\Delta(G))(h_2,g_2\Delta(G))(a_1,\dots ,a_n).
\end{align*}
Hence $\Psi$ is a homomorphism of groups. Let $h\in \Delta(G)$ and $g\in G$ such that $\Psi(h,g\Delta(G))=\id$. In particular,
\[ (0,\dots,0)=\Psi(h,g\Delta(G))(0,\dots 0)=\alpha^{-1}(f(h)),\]
and thus $h=1$. Now we have that
\[\alpha^{-1}(f(x))=\Psi(1,g\Delta(G))(\alpha^{-1}(f(x)))=\alpha^{-1}\sigma_g(f(x))=\alpha^{-1}(f(gxg^{-1}))\]
for all $x\in \Delta(G)$. Hence $g\in C_G(\Delta(G))=\Delta(G)$. Therefore $g\Delta(G)=\Delta(G)$.
This shows that $\Psi$ is injective, and the result follows.
\end{proof}

\begin{theorem}\label{thm:leftorderedBieberbach}
Let $G$ be a Bieberbach group. Then the following conditions are equivalent.
\begin{enumerate}
    \item Every non-trivial subgroup of $G$ has non-trivial center.
    \item $G$ is locally indicable.
    \item $G$ is a poly-$\Z$ group.
    \item $G$ is left ordered.
    \item $G$ is diffuse.
\end{enumerate}
\end{theorem}

\begin{proof}
$(1)\implies (2).$ Suppose that every non-trivial subgroup of $G$ has a non-trivial center. Let $H$ be a non-trivial finitely generated subgroup of $G$. Let $z\in Z(H)\setminus \{ 1\}$. By Lemma \ref{lem:SubBieberbeach}, $H$ is a Bieberbach group.  Let $\nu\colon H\to\Delta(H)$ be the transfer homomorphism. Let $n=(H:\Delta(H))$. Then $\nu(z)=z^n\neq 1$. Hence $\nu(H)$ is an infinite finitely generated abelian group. Therefore there exists a surjective homomorphism of groups $f\colon \nu(H)\to\Z$ and thus $f\nu \colon H\to\Z$ is a surjective homomorphism of groups. This shows that $G$ is locally indicable. 

$(2)\implies (3).$ Suppose that $G$ is locally indicable. We shall show that $G$ is a poly-$\Z$ group by induction on the dimension of $G$. If $G$ has dimension $1$, then by Theorem \ref{thm:Zfiniteindex}, $G\cong \Z$. Suppose that $\dim(G)=n>1$ and that every locally indicable Bieberbach group of dimension $n-1$ is a poly-$\Z$ group. Since $G$ is locally indicable and finitely generated, there exists a surjective homomorphism $f\colon G\to\Z$. Let $H=\ker f$. Let $g\in G$ be an element such that $f(g)=1$. Note that $G=H\langle g\rangle$ and $H\cap \langle g\rangle=\{ 1\}$. By Lemma \ref{lem:SubBieberbeach}, $H$ is a Bieberbach group. Let $A_1=H\cap\Delta (G)$ and $A_2=\langle g\rangle \cap\Delta(G)$. Note that $A_1A_2$ is an abelian subgroup of finite index in $G$. Since $A_1$ is a subgroup of finite index of $H$, we have that the rank of $A_1$ is the dimension of $H$. Since $A_2$ is a non-trivial subgroup of $\langle g\rangle$, we have that $A_2\cong \Z$.
Hence $\dim(G)=\dim (H)+1$. Therefore $H$ has dimension $n-1$ an it is loccally indicable. By the inductive hypothesis, $H$ is a poly-$\Z$ group. Therefore, $G$ also is a poly-$\Z$ group. Thus the result follows by induction. 

$(3)\implies (4).$ Since $\Z$ is an ordered group, the result is an easy consequence of Proposition \ref{prop:LOgroup1}.

$(4)\implies (5).$ This follows by Lemma \ref{lem:LOimpliesdiffuse}

$(5)\implies (1).$ Suppose that there exists a non-trivial subgroup $H$ of $G$ such that $Z(H)=\{ 1\}$. We shall prove that $H$ is not diffuse, and therefore $G$ is not diffuse and the result follows.

By Lemma \ref{lem:SubBieberbeach}, $H$ is a Bieberbach group. Let $n=\dim(H)$.
By Theorem \ref{thm:GeomBieberbach}, $H$ is isomorphic to a subgroup $H_1$ of $\Isom(\R^n)$ such that $\Delta(H)$ corresponds to the translations of $H_1$. 

Note that if $u\in \R^n$ and $r\in\R$ is positive, then
\[ B(u,r)=\{ t\in\Delta(H_1) : \| t(0)-u\|<r\}\]
is finite. Since $(H_1:\Delta(H_1))=n$,
\[B'(0,r)=\{ h\in H :  \| h(0)\|<r\}\]
is finite. We shall prove that $B'(0,r)$ has no extreme points for  all sufficiently large $r>0$. 

Let $r_0>0$ such that $B(u,r_0)\neq\emptyset$ for all $u\in\R^n$. Note that every element $h\in H_1$ can be uniquely written as $h=t_hf_h$ with $t_h\in T(\R^n)$ and $f_h\in O(\R^n)$. Furthermore $H_1/\Delta(H_1)\cong (T(\R^n)H_1)/T(\R^n)\cong \langle f_h :h\in H_1\rangle=\{ f_h : h\in H_1\}$. 

Let $F=\{ v\in \R^n : f_g(v)=v$ for all $ g\in H_1\}$. Let $v\in F$. Let 
\[w=\sum_{t\in B(v,r_0)}t(0)=\left(\prod_{t\in B(v,r_0)}t\right)(0).\] 
Suppose that $v=(v_1,\dots ,v_n)\neq (0,\dots ,0)$. Since $sv\in F$ for all $s\in \R$, we may assume that $v_i>r_0$. If $t\in B(v,r_0)$ and $t(0)=(a_1,\dots ,a_n)$, then $0<v_i-r_0<a_i<v_i+r_0$. Hence $w\neq 0$. Note that if $t\in B(v,r_0)$, then
\[\|f_gtf_g^{-1}(0)-v\|=\|f_gt(0)-f_g(v)\|=\|t(0)-v\|<r_0\]
for all $g\in H_1$.
Now we have,
\begin{align*}
    f_g(w)=&\sum_{t\in B(v,r_0)}f_g(t(0))\\
    =&\sum_{t\in B(v,r_0)}f_gtf_g^{-1}(0)\\
    =&\sum_{t\in B(v,r_0)}t(0)=w\neq 0.
\end{align*}
Furthermore,
\begin{align*}g\left(\prod_{t\in B(v,r_0)}t\right)g^{-1}(0)=&f_g\left(\prod_{t\in B(v,r_0)}t\right)f_g^{-1}(0)\\
=&\sum_{t\in B(v,r_0)}f_gtf_g^{-1}(0)\\
    =&\sum_{t\in B(v,r_0)}t(0)\\
=&\left(\prod_{t\in B(v,r_0)}t\right)(0)\neq 0,
\end{align*}
for all $g\in H_1$. Hence $\prod_{t\in B(v,r_0)}t\in Z(H_1)\setminus\{\id\}$, a contradiction because $Z(H_1)=\{ \id\}$. Hence $F=\{ (0,\dots ,0)\}$.

Hence, for every nonzero vector $u\in\R^n$, there exists $h_u\in H_1$ such that $f_{h_u}(u)\neq u$, and thus
\[ \|f_{h_u}(u)+u\|<2\|u\|.\]
Hence, there exists $0<\delta_u<1$ such that
\[\|f_{h_u}(u)+u\|\leq 2\delta_u\|u\|. \]
Let $S=\{ u\in \R^n : \| u\|=1\}$. Let $V_u=\{ v\in\R^n :\|u-v\|<\frac{1-\delta_u}{2} \}$. Note that
\begin{align*}\|f_{h_u}(v)+v \|\leq &\|f_{h_u}(v)-f_{h_u}(u)\|+\|f_{h_u}(u)+u\|+\|v-u\|\\
<& 1-\delta_u+2\delta_u=1+\delta_u.
\end{align*}
Since $S$ is compact and $S\subseteq\bigcup_{u\in S}V_u$, there exists a finite subset $K\subseteq S$ such that 
\[S\subseteq\bigcup_{u\in K}V_u.\]
Let $\delta=\max\{ \frac{1+\delta_u}{2} :u\in K\}$. Note that $0<\delta<1$. Then it is easy to check that for all $u\in \R^n$ there exists $h_u\in H_1\setminus\Delta(H_1)$ such that
\[ \|f_{h_u}(u)+u\|\leq 2\delta\| u\|.\]

Let 
\[ r> \frac{r_0}{1-\delta}.\]
Let $h\in B'(0,r)$. Hence $\|h(0)\|<r$.
Let $u=h(0)$. We have seen that there exists $h\in H_1\setminus \Delta(H_1)$ such that
\[ \|f_{h_u}(u)+u\|\leq 2\delta\| u\|.\]
Let $v=\frac{1}{2}(-f_{h_u}(u)+u)$. Let $t\in B(v-h_u(0),r_0)$. Since $\| u\|=\|f_{h_u}(u)\|<r$, we have that
\[ \| u-t(0)-h_u(0)\|\leq \|u-v\|+\|v-t(0)-h_u(0))\|\leq\delta r+r_0<r\]
and
\[ \| f_{h_u}(u)+t(0)+h_u(0)\|\leq \|f_{h_u}(u)+v\|+\|-v+t(0)+h_u(0)\|\leq\delta r+r_0<r.\]
Let $g=th_u$. Note that $g\in H_1$. Since $f_{h_u}\neq \id$, we have that $g\neq 1$.
Note that
\[    \|gh(0)\|=\|g(u)\|=\|t(0)+h_u(0)+f_{h_u}(u)\|<r
\]
and
\begin{align*}    \|g^{-1}h(0)\|=&\|h_u^{-1}(0)-f_{h_u}^{-1}t(0)+f_{h_u}^{-1}(u)\|\\
=&\|-f_{h_u}h_u(0)-f_{h_u}^{-1}t(0)+f_{h_u}^{-1}(u)\|=\|-h_u(0)-t(0)+u\|<r.
\end{align*}
Hence $gh,g^{-1}h\in B'(0,r)$, and the result follows.
\end{proof}

\begin{theorem}\label{thm:finiteinvol}
Let $(X,r)$ be a finite involutive solution of the YBE. Then the structure group $G(X,r)$ is a solvable Bieberbach group.
\end{theorem}

\begin{proof}
Since $\sigma_x\sigma_y=\sigma_{\sigma_x(y)}\sigma_{\tau_y(x)}$ for all $x,y\in X$, there exists a unique homomorphism of groups
$\phi\colon G(X,r)\to\langle \sigma_x : x\in X\rangle\subseteq\Sym_X$ such that $\phi(x)=\sigma_x$ for all $x\in X$. By Theorem \ref{thm:involstruct}, the additive group of $G(X,r)$ is free with basis $X$ (the natural map $i\colon X\to G(X,r)$ is injective and we identify $x$ with $i(x)$ for all $x\in X$. Hence $\ker(\phi)=\Soc(G(X,r))$ is an abelian subgroup of $G(X,r)$ and $G(X,r)/\Soc(G(X,r))$ is a finite skew brace of abelian type. By Theorem \ref{thm:add_nilpotent}, it follows that $G(X,r)$ is solvable.

Since $G(X,r)$ has an abelian subgroup of finite index, to prove that $G(X,r)$ is a Bieberbach group, it is enough to show that $G(X,r)$ is torsion free.
Suppose that $G(X,r)$ is not torsion free. Hence there exists $g\in G(X,r)$ of prime order $p\leq |X|=n$. Since
$g^1=g+\lambda_g(g)+\dots +\lambda_{g^{n-1}}(g)$ and the additive group of the skew brace $G(X,r)$ is torsion free, we have that $\lambda_g(g)\neq g$. Hence $\lambda_g$ has order $p$. Let $X=\{ x_1,\dots ,x_n\}$. The restriction of $\lambda_g$ to $X$ is a product of cycles of length $p$ in $\Sym_X$. Hence we may assume that there exists a non-negative integer $m$ such that
\[\lambda_g(x_{kp+i}=\left\{\begin{array}{ll}
x_{kp+i+1}&\text{ if }1\leq i<p\\
x_{kp+1}&\text{ if }i=p\end{array}\right.\]
for all $0\leq k\leq m$ and $\lambda_g(x_t)=x_t$ for all $mp+p<t\leq n$.
Since $g^p=1$, in the additive group of the skew brace $G(X,r)$ we have that $g=z_1x_1+\dots +z_nx_n$ for some $z_1,\dots ,z_n\in\Z$ and 
\[g+\lambda_g(g)+\dots +\lambda_{g^{n-1}}(g)=0.\]
Hence
\[\sum_{i=1}^pz_{kp+1}=0\quad\text{and}\quad z_t=0\]
for all $0\leq k\leq m$ and $mp+p<t\leq n$. Thus $g=\sum_{i=1}^{mp+p}z_ix_i$. Let
\[ h_k=\sum_{j=1}^{p-1}\left(\sum_{i=1}^jz_{kp+i}\right)x_{kp+j}\]
for all $0\leq k\leq m$. Let $h=\sum_{k=0}^mh_k$. We have that
\begin{align*}
    gh=&g+\lambda_g(h)=\sum_{k=0}^m\sum_{j=1}^pz_{kp+j}x_{kp+j}+\sum_{k=0}^m\sum_{j=1}^{p-1}\left(\sum_{i=1}^jz_{kp+i}\right)\lambda_g(x_{kp+j})\\
    =&\sum_{k=0}^m\sum_{j=1}^pz_{kp+j}x_{kp+j}+\sum_{k=0}^m\sum_{j=1}^{p-1}\left(\sum_{i=1}^jz_{kp+i}\right)x_{kp+j+1}\\
    =&\sum_{k=0}^mz_{kp+1}x_{kp+1}+\sum_{k=0}^m\sum_{j=1}^{p-1}\left(\sum_{i=1}^{j+1}z_{kp+i}\right)x_{kp+j+1}\\
    =&\sum_{k=0}^m\sum_{j=0}^{p-1}\left(\sum_{i=1}^{j+1}z_{kp+i}\right)\x_{kp+j+1}\\
    =&h+\sum_{k=0}^m\left(\sum_{i=1}^pz_{kp+i}\right)x_{kp+p}\\
    =&h.
\end{align*}
Hence $g=1$, a contradiction. Therefore the structure group $G(X,r)$ is torsion free and the result follows.
\end{proof}




\section{Exercises}

\begin{prob}
	Let $G$ be a group and $\{N_\alpha:\alpha\}$ be a collection of normal subgroups of $G$ such that
	$\cap_{\alpha}N_\alpha=\{1\}$. Prove that if $G/N_{\alpha}$ is a left ordered group for all $\alpha$, then 
	$G$ is a left ordered group. 
\end{prob}

% passman lema 1.9 pag 589
\begin{prob}
	Let $K$ be a field. Prove that if $G$ is a group satisfying the unique product property, then every invertible element of $K[G]$ is of the form $ag$ for $a\in K\setminus\{ 0\}$ and $g\in G$ (i. e. $K[G]$ has only trivial units).
\end{prob}

\begin{prob}
	\label{xca:[x,y]^n=1}
	Let $G$ be a group such that $(G:Z(G))=n$. Prove that $[x,y]^n=1$ for all $x,y\in G$. 
\end{prob}

\begin{prob}
	Let $H$ be a central subgroup of a finite group $G$. Prove that if $|H|$
	and $|G/H|$ are coprime, then $G\cong H\times G/H$.
\end{prob}

\begin{prob}
Let $G$ be a group.
Prove that $\Delta(G)$ is a characteristic subgroup of the group $G$.
\end{prob}

\section{Open problems}
\begin{problem}
Let $G$ be a group satisfying the unique product property. Is $G$ diffuse?
\end{problem}


\section*{Notes}