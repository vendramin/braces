\chapter{Multipermutation solutions}
\label{MP}

\section*{A}

If $X$ is a solution, we 
consider over $X$ the relation
\[
x\sim y\Longleftrightarrow \sigma_x=\sigma_y\text{ and }\tau_x=\tau_y.
\]
Then $\sim$ is an equivalence relation. 
Let $\overline{X}$ be the set of equivalence classes 
and $[x]$ 
%$\overline{x}$ 
denote the equivalence class of $x$. 

\begin{proposition}
Let $(X,r)$ be a solution. Then $(\overline{X},\overline{r})$, where 
%$\overline{r}(\overline{x},\overline{y})=(\overline{\sigma_x(y)},\overline{\tau_y(x)})$, 
\[
\overline{r}([x],[y])=([\sigma_x(y)],[\tau_y(x)]),
\]
is a solution. 
\end{proposition}

\begin{proof}
    We first prove that $\overline{r}$ is well-defined. 
    Let $x,y\in X$ be such that $x\sim y$ and let $z\in X$. Since $(X,r)$ is a solution, 
    Lemma~\ref{lem:YB} implies that 
    \[
    \sigma_{\sigma_x(z)}\circ\sigma_{\sigma_z(x)}=\sigma_x\circ\sigma_z=\sigma_y\circ\sigma_z=\sigma_{\sigma_y(z)}\circ\sigma_{\sigma_z(x)},
    \]
    it follows that $\sigma_{\sigma_z(x)}=\sigma_{\sigma_z(y)}$ and hence $\sigma_z(x)\sim\sigma_z(y)$. 
    Similarly $\tau_{\tau_z(x)}=\tau_{\tau_z(y)}$ and therefore 
    $\overline{r}$ is well-defined. 
    
    We now prove that $\overline{r}$ is invertible. 
    
    ...
\end{proof}

In the case of involutive solutions, it follows from Proposition~\ref{pro:T} 
that $\sigma_x=\sigma_y$ if and only if $\tau_x=\tau_y$. 

\begin{definition}
\index{Retraction!of a solution}
Let $(X,r)$ be a solution. The solution $\Ret(X,r)=(\overline{X},\overline{r})$ induced
by the equivalence relation $\sim$ is the \emph{retraction} of $(X,r)$.
\end{definition}

We define inductively $\Ret^0(X,r)=(X,r)$, $\Ret^1(X,r)=\Ret(X,r)$ and
\[
\Ret^{n+1}(X,r)=\Ret(\Ret^n(X,r))\quad n\geq1.
\]


\begin{definition}
    \index{Solution!multipermutation}
    A solution $(X,r)$ is said to be of \emph{multipermutation level} $n$ 
    if $n$ is the smallest non-negative integer 
    such that $|\Ret^n(X,r)|=1$. The solution $(X,r)$ is said to be 
    \emph{irretractable} if $\Ret(X,r)=(X,r)$. 
\end{definition}

The trivial solution over the set with one element is a multipermutation of level zero. 
Permutation solutions are multipermutation solutions of level one. 

\begin{example}

\end{example}

\begin{example}

\end{example}

\begin{table}[H]
\centering
\caption{Involutive solutions of size $\leq10$.}
\begin{tabular}{|r|ccccccccc|}
\hline
$n$ & 2 & 3 & 4 & 5 & 6 & 7 & 8 & 9 & 10\tabularnewline
\hline 
solutions & 2 & 5 & 23 & 88 & 595 & 3456 & 34530 & 321931 & 4895272\tabularnewline
multipermutation & 2 & 5 & 21 & 84 & 554 & 3295 & 32155 & 305916 & 4606440\tabularnewline
irretractable & 0 & 0 & 2 & 4 & 9 & 13 & 191 & 685 & 3590\tabularnewline
\hline
\end{tabular}
\label{tab:INV_mp}
\end{table}

For size $\leq7$ the numbers of Table~\ref{tab:INV_mp} coincide with those in~\cite{MR1722951}
but there are some differences for solutions of size eight. 

\begin{table}[H]
\centering
\caption{Non-involutive solutions of size $\leq8$.}
\begin{tabular}{|r|ccccccc|}
\hline
$n$ & 2 & 3 & 4 & 5 & 6 & 7 & 8 \tabularnewline
\hline 
solutions & 2 & 21 & 230 & 3519 & 100071 & 4602720  & 422449480 \tabularnewline
multipermutation &  & 15 & 206 & 3165 & 95517 & 4461805 & 416725250 \tabularnewline
irretractable & & 6 & 24 & 98 & 514 & 2659 & 17370\tabularnewline
\hline
\end{tabular}
\label{tab:mp}
\end{table}
%[ 422449480, 416725250, 17370, 1069 ]


\begin{theorem}
\label{thm:CJKAV}
    Let $(X,r)$ be a finite multipermutation solution. If $|X|>1$, then $r$ has even order. 
\end{theorem}

\begin{proof}
    Since $(X,r)\to\Ret(X,r)$, $x\mapsto[x]$ is a homomorphism of solutions, 
    it follows that the order of the solution $\overline{r}$ divides de orden of $r$. 
    Assume that $(X,r)$ has multipermutation level $n$. 
    There exists a homomorphism of solutions $(X,r)\to\Ret^{n-1}(X,r)$, thus 
    it is enough to prove the theorem in the case where
    $r(x,y)=(\sigma(y),\tau(x))$ for commuting permutations $\sigma$ and $\tau$, i.e. 
    multipermutation solutions of level one. If $r$ has order $2k+1$, then 
    \[
    (x,y)=r^{2k+1}(x,y)=(\sigma^{k+1}\tau^k(y),\sigma^k\tau^{k+1}(x)).
    \]
    This implies that $\sigma^{k+1}\tau^k(y)=x$ for all $x,y\in X$. This equality in particular 
    implies that $x=y$ because $\sigma^{k+1}\tau^k$ is a permutation, a contradiction. 
\end{proof}

The connection between the socle of a brace and the 
retract of a solution was discovered by Rump in the case of 
involutive solutions and braces of abelian type, see~\cite{MR2278047}. 

\begin{proposition}
\label{pro:add_cyclic}
Let $A$ be a brace and $(A,r)$ be its associated solution. 
Then the retraction $\Ret(A,r)$ is the canonical solution associated with the quotient brace $A/\Soc(A)$. 
\end{proposition}

\begin{proof}
    The equivalence relation $\sim$ on $A$ is defined as $a\sim b$ if and only if $\lambda_a=\lambda_b$ and 
    $\mu_a=\mu_b$. Let $\overline{A}$ be the set of equivalence classes. 
    The equivalence class of an element $a$ is then
    \begin{align*}
    [a]&=\{b\in A:a\sim b\}
    =\{b\in A:\lambda_a=\lambda_b,\,\mu_a=\mu_b\}\\
    &=\{b\in A:a'\circ b\in \ker\lambda\cap\ker\mu\}=\{b\in A:a'\circ b\in\Soc(A)\},
    \end{align*}
    by Proposition~\ref{pro:soc_kernels}. This means that 
    $[a]=[b]$ if and only if $\pi(a)=\pi(b)$, where $\pi\colon A\to A/\Soc(A)$, $x\mapsto x\circ\Soc(A)$,  
    is the canonical brace homomorphism. Moreover, $A/\Soc(A)=\overline{A}$ as sets. 
    Now we compute the retraction of $(A,r)$:
    \begin{align*}
        \overline{r}([a],[b]) &= ([\lambda_a(b)],[\mu_b(a)])
        =(\pi(\lambda_a(b)),\pi(\mu_b(a)))\\
        &=\left(\lambda_{\pi(a)}(\pi(b)),\mu_{\pi(b)}(\pi(a))\right)
        =\left(\lambda_{[a]}([b]),\mu_{[b]}([a])\right).
    \end{align*}
    Therefore $\Ret(A,r)=(A/\Soc(A),\overline{r})$. 
\end{proof}

Now...

\begin{proposition}
\label{pro:mpl}
Let $(X,r)$ and $(Y,s)$ be solutions. Each surjective homomorphism 
of solutions $f\colon (X,r)\to (Y,s)$ 
induces a surjective homomorphism 
of solutions $\Ret(X,r)\to\Ret(Y,s)$. 
\end{proposition}

\begin{proof}
Write $r(x,y)=(\sigma_x(y),\tau_y(x))$ and $s(x,y)=(\lambda_x(y),\mu_y(x))$. 
Let $x,x_1\in X$ be such that $x\sim x_1$. If $z\in X$, then 
\[
\lambda_{f(x)}f(z)=f(\sigma_x(z))=f(\sigma_{x_1}(z))=\lambda_{f(x_1)}f(z).
\]
Since $f$ is surjective, it follows that $\lambda_{f(x)}=\lambda_{f(x_1)}$. A similar calculation proves 
that $\mu_{f(x)}=\mu_{f(x_1)}$. If $\pi\colon (Y,s)\to\Ret(Y,r)$, $y\mapsto [y]$, is the canonical map, 
the composition $\pi\circ f\colon (X,r)\to\Ret(Y,s)$ is a surjective homomorphism of solutions. Therefore 
the map $\Ret(X,r)\to\Ret(Y,s)$, $[x]\mapsto\pi(f(x))$, is then a well-defined surjective
homomorphism of solutions. 
\end{proof}

\begin{proposition}
\label{pro:mpl_subsol}
Let $(X,r)$ be a solution of finite multipermutation level $m$ 
and $Y\subseteq X$ be such that $r(Y\times Y)\subseteq Y\times Y$. 
Then the \emph{subsolution} $(Y,r|_{Y\times Y})$ is of finite multipermutation level $\leq m$. 
\end{proposition}

\begin{proof}

\end{proof}

\begin{theorem}
Let $(X,r)$ be a solution. The following statements are equivalent:
\begin{enumerate}
    \item $(X,r)$ has finite multipermutation level.
    \item $\left(\mathcal{G}(X,r),r_{\mathcal{G}(X,r)}\right)$ has finite multipermutation level.
    \item $\left(G(X,r),r_{G(X,r)}\right)$ has finite multipermutation level.
    %\item The brace $G(X,r)$ is right nilpotent.
    %\item The brace $\mathcal{G}(X,r)$ is right nilpotent. 
\end{enumerate}
\end{theorem}

\begin{proof}
    Let us first prove that (2) implies (1). 
    The map $X\to\mathcal{G}(X,r)$, $x\mapsto(\lambda_x,\mu_x^{-1})$, is a homomorphism of solutions that induces
    an injective homomorphism of solutions $\Ret(X,r)\to\left(\mathcal{G}(X,r),r_{\mathcal{G}(X,r)}\right)$. 
    Since $\left(\mathcal{G}(X,r),r_{\mathcal{G}(X,r)}\right)$ has finite multipermutation level,
    $(X,r)$ has finite multipermutation level by Proposition~\ref{pro:mpl_subsol}.
    
    Let us now prove that (3) implies (2). The canonical map $G(X,r)\to\mathcal{G}(X,r)$ yields a surjective
    homomorphism of solutions. Then Proposition~\ref{pro:mpl} applies. 

    ...
\end{proof}

The following result appeared in~\cite{MR1722951}.

\begin{proposition}
Let $(X,r)$ be a finite involutive solution. If the additive group 
of the brace $\mathcal{G}(X,r)$ is cyclic, then $(X,r)$ 
is multipermutation.
\end{proposition}

\begin{proof}
    Let $(X,r)$ be a counterexample of minimal cardinality. If $K$ is the additive 
    group of $\mathcal{G}(X,r)$, then $K$ is finite and cyclic. 
    Write $G$ for the multiplicative group of $\mathcal{G}(X,r)$. 
    Since $|\Aut(K)|=\varphi(|K|)<|K|$, where $\varphi$ is the Euler function, 
    the group homomorphism $\lambda\colon G\to\Aut(K)$ 
    has a non-trivial kernel, so $\Soc(\mathcal{G}(X,r))$ is non-zero.  
    This implies that $(X,r)$ is retractable. 
    Since $\mathcal{G}(X,r)/\Soc(\mathcal{G}(X,r))$ is a brace with cyclic 
    additive group and $\Ret(X,r)$ is an involutive solution, the 
    minimality of $|X|$ implies that 
    $\Ret(X,r)$ is a multipermutation solution, and hence so is $(X,r)$, 
    a contradiction. 
\end{proof}

The converse of the previous proposition does not hold. 

\begin{example}
	Let $X=\{1,2,3,4\}$ and $r(x,y)=(\varphi_x(y),\varphi_y(x))$, where
	\[
		\varphi_1=\varphi_2=\id,\quad
		\varphi_3=(34),\quad
		\varphi_4=(12)(34).
 	\]
	Then $(X,r)$ is an involutive multipermutation solution. One easily checks
	that $\mathcal{G}(X,r)\simeq C_2\times C_2$.
\end{example}

A similar idea proves the following result:

\begin{theorem}
\label{thm:mul_cyclic}
Let $(X,r)$ be a finite involutive solution. If the multiplicative group 
of the brace $\mathcal{G}(X,r)$ is cyclic, then $(X,r)$ 
is multipermutation.
\end{theorem}

\begin{proof}
    Let $(X,r)$ be a counterexample of minimal cardinality. Write $K$ for the additive 
    group of $\mathcal{G}(X,r)$ and 
    $G=\langle g\rangle$ for the multiplicative group of $\mathcal{G}(X,r)$. Since 
    the image of the group homomorphism $\lambda\colon G\to\Aut(K)$ is cyclic generated by $\lambda_g$ and 
    $|\lambda_g|<|G|$ by Horosevskii's theorem, see~\cite[Corollary 3.3]{MR2426855}, it follows that 
    $\lambda$ has a non-trivial kernel, so $\Soc(\mathcal{G}(X,r))$ is non-zero.  
    This implies that $(X,r)$ is retractable. 
    Since $\mathcal{G}(X,r)/\Soc(\mathcal{G}(X,r))$ is a brace with cyclic 
    additive group and $\Ret(X,r)$ is an involutive solution, the 
    minimality of $|X|$ implies that 
    $\Ret(X,r)$ is a multipermutation solution, and hence so is $(X,r)$, 
    a contradiction. 
\end{proof}

The previous result does not hold in the case of arbitrary solutions.

\begin{example}
Let $X=\{1,2,3,4,5,6\}$ and $r(x,y)=(\sigma_x(y),\tau_y(x))$, where
\begin{align*}
    \sigma_1 &= \id, & \sigma_2&=\id, & \sigma_3 &= \id,\\
    \sigma_4&=(23)(56), &\sigma_5 &= (23)(56), & \sigma_6&=(23)(56),\\
    \tau_1 &= \id, & \tau_2&=(456), & \tau_3 &= (465),\\
    \tau_4&=\id, &\tau_5 &= (465), & \tau_6&=(456).
\end{align*}
The brace $\mathcal{G}(X,r)$ has multiplicative group isomorphic to $\Sym_3$ and
additive group isomorphic to the cyclic group of order six. 
\end{example}

We will see later that Theorem~\ref{thm:mul_cyclic}
is true for braces of nilpotent type. 
The following example appears in the work of Rump~\cite{MR2278047}.

\begin{proposition}
\label{pro:radical}
Let $A$ be a finite non-trivial radical ring. Then $\Soc(A)\ne\{0\}$ and 
$(A,r_A)$ is an involutive multipermutation solution. 
\end{proposition}

\begin{proof}
Let $A$ be a counterexample of minimal size. This means that $\Soc(A)=\{0\}$ 
and all two-sided braces of abelian type of size $<|A|$ have non-trivial socle. 
Since $A$ is finite, there exists a non-zero minimal left ideal $I$ of $A$.
Recall that $A$ is a radical ring with product $a*b=\lambda_a(b)-b$. 
Since $A$ is a radical ring, $A$ is a nil ring, which implies by 
Nakayama's lemma that $I*A=\{0\}$. This means that if $x\in I$, 
then $x\in\Soc(A)$, as $0=x*a=\lambda_x(a)-a$ for all $a\in A$. 
In particular, $\Soc(A)\ne\{0\}$, a contradiction. 
\end{proof}

Proposition~\ref{pro:radical} has a nice application. The results appeared first in~\cite{MR2652212}. 
The proof presented here is from~\cite{MR3177933}. 

\begin{theorem}
\label{thm:CJO_abelian}
    Let $(X,r)$ be a finite involutive solution. If the multiplicative group of the brace 
    $\mathcal{G}(X,r)$ is abelian, then $(X,r)$ is multipermutation. 
\end{theorem}

\begin{proof}
    
\end{proof}



In~\cite{MR2095675}, Gateva--Ivanova conjectured that finite involutive 
square-free solutions are retractable.


In~\cite{MR3861714} Gateva--Ivanova asked when...

Right nilpotency...

The following theorem characterizes multipermutation involutive solutions in 
terms of left orderability of groups. A group $G$ is said to be 
\emph{left ordered} if it admits a total ordering $<$ such that 
\[
x<y\implies zx<zy
\]
for all $x,y,z\in G$. Torsion-free abelian groups, 
free groups and braid groups are left ordered groups. 
See~\cite{MR3560661} for more information on ordered groups. 

 \begin{theorem}
\label{thm:BCV}
	Let $(X,r)$ be a finite involutive solution. The following statements are equivalent:
	\begin{enumerate}
		\item $(X,r)$ is a multipermutation solution.
		\item $G(X,r)$ is poly-$\Z$.
		\item $G(X,r)$ is left orderable.
		\item $G(X,r)$ is diffuse.
	\end{enumerate}
\end{theorem}

\begin{proof}

\end{proof}

\index{Group!with the unique product property}
Recall that a group $G$ has the \emph{unique product property} if 
for all finite non-empty subsets $A$ and $B$ of $G$ there exists $x\in G$ 
that can be written uniquely as $x = ab$ with $a\in A$ and $b\in B$. 
% We refer to~\cite{MR798076} for more information related to 
% the unique product property. 

It is natural to ask when $G(X,r)$ has the unique product property. By Theorem~\ref{thm:BCV}, if $(X,r)$ is an involutive
multipermutation solution, then $G(X,r)$ has the unique product property since $G(X,r)$ is left orderable. 

\begin{example}
    \label{pro:4-19}
	Let $X=\{1,2,3,4\}$ and $r(x,y)=(\sigma_x(y),\tau_y(x))$ be the irretractable involutive solution given by 
	\begin{align*}
		&\sigma_1=(12), && \sigma_2=(1324), && \sigma_3=(34), && \sigma_4=(1423),\\
		&\tau_1=(14), &&\tau_2=(1243), && \tau_3=(23), && \tau_4=(1342).
	\end{align*}
	We claim that the group $G(X,r)$ with 
	generators
	$x_1,x_2,x_3,x_4$ and relations
	\begin{align*}
		& x_1^2=x_2x_4,
		&& x_1x_3=x_3x_1,
		&& x_1x_4=x_4x_3,\\
		& x_2x_1=x_3x_2,
		&& x_2^2=x_4^2,
		&& x_3^2=x_4x_2.
	\end{align*}
	does not have the unique product property. Let $x=x_1x_2^{-1}$ and $y=x_1x_3^{-1}$ and 
    \begin{multline}
    \label{eq:Promislow}
    S=\{ x^2y,
    y^2x,
    xyx^{-1},
    (y^2x)^{-1},
    (xy)^{-2},
    y,
    (xy)^2x,
    (xy)^2,\\
    (xyx)^{-1},
    yxy,
    y^{-1},
    x,
    xyx, 
    x^{-1}
	\}.
    \end{multline}
    To prove that $G(X,r)$ does not have
    the unique product property it is enough to prove that 
    each $s\in S^2=\{s_1s_2:s_1,s_2\in S\}$ admits at least two different decompositions 
    of the form $s=ab=uv$ for $a,b,u,v\in S$. To perform these calculations we 
    use the injective group homomorphism $G\to\GL(5,\Z)$ given by  
	\begin{align*}
	&x_1\mapsto\left(\begin{smallmatrix}
	0 & 1 & 0 & 0 & 1\\
	1 & 0 & 0 & 0 & 0\\
	0 & 0 & 1 & 0 & 0\\
	0 & 0 & 0 & 1 & 0\\
	0 & 0 & 0 & 0 & 1
  	\end{smallmatrix}\right),
  	&&
	x_2\mapsto\left(\begin{smallmatrix}
	0 & 0 & 0 & 1 & 0\\
	0 & 0 & 1 & 0 & 1\\
	1 & 0 & 0 & 0 & 0\\
	0 & 1 & 0 & 0 & 0\\
	0 & 0 & 0 & 0 & 1
  	\end{smallmatrix}\right),
  	\\
  	&x_3\mapsto\left(\begin{smallmatrix}
	1 & 0 & 0 & 0 & 0\\
	0 & 1 & 0 & 0 & 0\\
	0 & 0 & 0 & 1 & 1\\
	0 & 0 & 1 & 0 & 0\\
	0 & 0 & 0 & 0 & 1
  	\end{smallmatrix}\right),
  	&&
  	x_4\mapsto\left(\begin{smallmatrix}
	0 & 0 & 1 & 0 & 0\\
	0 & 0 & 0 & 1 & 0\\
	0 & 1 & 0 & 0 & 0\\
	1 & 0 & 0 & 0 & 1\\
	0 & 0 & 0 & 0 & 1
  	\end{smallmatrix}\right).
  	\end{align*}
  	This faithful representation of $G(X,r)$ allows us to compute all possible products of the form $s_1s_2$ for all $s_1,s_2\in S$. 
  	By inspection, each element of $S^2$ admits at least two different representations.
\end{example}

\begin{theorem}
\label{thm:CJO_mp}
    Let $A$ be a finite brace of abelian type with multiplicative group $G$. Then there exists a finite solution $(X,r)$ 
    such that $\Ret(X,r)$ is isomorphic to $(A,r_A)$ and $\mathcal{G}(X,r)\simeq G$.
\end{theorem}

\begin{proof}
    Let $X=A\times\Z/(2)$. For $a,b\in A$, let 
    \begin{align*}
        &\varphi_{(a,0)}(b,0)=(b,0), && \varphi_{(a,0)}(b,1)=(b,1),\\
        &\varphi_{(a,1)}(b,0)=(a'\circ b,0), && \varphi_{(a,1)}(b,1)=(\lambda^{-1}_a(b),1).
    \end{align*}
    The maps $\varphi_{(a,\epsilon)}$ are invertible 
    for all $a\in A$ and $\epsilon\in\Z/(2)$. In fact, 
    \[
    \varphi_{(a,0)}^{-1}=\id,
    \quad
    \varphi_{(a,1)}^{-1}(b,\epsilon)=\begin{cases}
    a\circ b & \text{if $\epsilon=0$,}\\
    \lambda_a(b) & \text{if $\epsilon=1$.}
    \end{cases}
    \]
    So we need to check that 
    \begin{align}
    \label{eq:CJO_tocheck}
    ((a,\epsilon_1)\cdot (b,\epsilon_2))\cdot ((a,\epsilon_1)\cdot (c,\epsilon_3))
    =((b,\epsilon_2)\cdot (a,\epsilon_1))\cdot ((b,\epsilon_2)\cdot (c,\epsilon_3))
    % \shortintertext{and}
    % \varphi_{(a,\epsilon_1)}((-h,y)+(k,z)+(h,y))=-\varphi_{(g,x)}(h,y)+\varphi_{(g,x)}(k,z)+\varphi_{(g,x)}(h,y).
    \end{align}
    holds for all $a,b,c\in A$ and $\epsilon_1,\epsilon_2,\epsilon_3\in\Z/(2)$. There are several cases to consider. 
    Let us assume first
    that $(\epsilon_1,\epsilon_2,\epsilon_3)=(1,1,0)$. Then~\eqref{eq:CJO_tocheck} turns out to be 
    \[
    (\lambda^{-1}_a(b)'\circ a'\circ c,0)=(\lambda^{-1}_b(a)'\circ b'\circ c,0),
    \]
    which holds for all $a,b\in A$, as  
    \[
    \lambda^{-1}_a(b)'\circ a'\circ c=(a\circ \lambda^{-1}_a(b))'\circ c=(a+b)'\circ c=(b+a)'\circ c=\lambda^{-1}_b(a)'\circ b'\circ c
    \]
    because $A$ is of abelian type. Let us now deal with the case $(\epsilon_1,\epsilon_2,\epsilon_3)=(1,1,1)$. In this case 
    Equality~\eqref{eq:CJO_tocheck}
    turns out to be equivalent to 
    \[
    (\lambda^{-1}_a(b),1)\cdot (\lambda^{-1}_a(c),1)
    =(\lambda^{-1}_b(a),1)\cdot (\lambda^{-1}_b(c),1).
    \]
    Since $A$ is of abelian type,
    \begin{align*}
        (\lambda^{-1}_a(b),1)\cdot (\lambda^{-1}_a(c),1)
        &= \lambda^{-1}_{\lambda^{-1}_a(b)}\lambda^{-1}_a(c)
        = \lambda^{-1}_{a\circ\lambda^{-1}_a(b)}(c)
        = \lambda^{-1}_{a+b}(c)\\
        &= \lambda^{-1}_{b+a}(c)
        = \lambda^{-1}_{\lambda^{-1}_b(a)}\lambda^{-1}_b(c)
        = (\lambda^{-1}_b(a),1)\cdot (\lambda^{-1}_b(c),1).
    \end{align*}
    The other cases are easier and require straightforward calculations. 
    
    Let $\psi\colon G\to\Sym_X$, $\psi(g)=\varphi_{(g',1)}$. Since 
    \[
    \psi(g)(0,0)=\varphi_{(g',1)}(0,0)=(g\circ 0,0)=(g,0),
    \]
    it follows that $\psi$ is injective. Moreover, $\psi$ is a group homomorphism, as
    \begin{align*}
        \psi(a)\psi(b)(c,0) &= \psi(a)\varphi_{(b,1)}(c,0)
        =\psi(a)(b\circ c,0)\\
        &=\varphi_{(a,1)}(b\circ c,0)
        =(a\circ b\circ c,0)
        =\varphi_{(a\circ b,1)}(c,0)=\psi_{(a\circ b)}(c,0)
    \shortintertext{and}
        \psi(a)\psi(b)(c,1) &= \varphi_{(a',1)}\varphi_{(b',1)}(c,1)
        =\varphi_{(a',1)}(\lambda^{-1}_{b'}(c),1)\\
        &=(\lambda^{-1}_{a'}\lambda^{-1}_{b'}(c),1)
        =(\lambda_{a\circ b}(c),1)
        =\varphi_{(a\circ b,1)}(c,1)
        =\psi(a\circ b)(c,1).
    \end{align*}
    Since $\psi$ is an injective group homomorphism, 
    \[
    G\simeq\psi(G)\simeq\langle \psi(a):a\in A\rangle=\langle\varphi_{(a,1)}:a\in A\rangle\simeq\mathcal{G}(X,r).
    \]
    
    Consider the equivalence relation on $X$ given by $x\sim y$ if and only if $\varphi_x=\varphi_y$. 
    As usual $[x]$ denotes the equivalence class of the element $x\in X$ and $\overline{X}$ is the set of equivalence classes. 
    A straightforward computation shows that 
    $(a,0)\sim (0,1)$ for all $a\in A$. This implies that 
    \[
    \overline{p}\colon \overline{X}\to A, 
    \quad
    \overline{p}([(a,\epsilon)])=\begin{cases}
    0 & \text{if $\epsilon=0$},\\
    a & \text{if $\epsilon=1$},
    \end{cases}
    \]
    is a well-defined surjective map. 
    We claim that $\overline{p}$ is injective. Let $(a,\epsilon_1)\in\overline{X}$ and
    $(b,\epsilon_2)\in\overline{X}$ be such that $\overline{p}([(a,\epsilon_1)])=\overline{p}([(b,\epsilon_2)])$. 
    Since $[(a,0)]=[(0,1)]$ for all $a\in A$, we only need to consider the case where $\epsilon_1=\epsilon_2=1$. 
    In this case, 
    \[
    a=\overline{p}([(a,\epsilon_1)])=\overline{p}([(b,\epsilon_2)])=b.
    \]
    Thus $\overline{p}$ is bijective. Now 
    \begin{align*}
        r_A(\overline{p}[(a,1)],\overline{p}[(b,1)]) 
        &= r_A(a,b)
        = (\lambda_a(b),\mu_b(a))
        = (\overline{p}[\lambda_a(b),1],\overline{p}[\mu_b(a),1])\\
        &= (\overline{p}\varphi_{(a,1)}(b,1),\overline{p}...).
    \end{align*}
\end{proof}

\begin{theorem}
    Let $A$ be...
\end{theorem}

\begin{proof}

\end{proof}

\section*{Exercises}

\begin{prob}
\label{prob:bounded_mpl}
    Let $(X,r)$ be a solution of finite multipermutation level $m$. Prove that any homomorphic image of $(X,r)$ 
    is a solution of finite multipermutation level $\leq m$. 
\end{prob}

\begin{prob}
	\label{prob:4-13}
	Let $X=\{1,2,3,4\}$ and $r(x,y)=(\sigma_x(y),\tau_y(x))$ be the irretractable involutive solution given by
	\begin{align*}
		&\sigma_1=(34), && \sigma_2=(1324), && \sigma_3=(1423), && \sigma_4=(12),\\
		&\tau_1=(24), &&\tau_2=(1432), && \tau_3=(1234), && \tau_4=(13).
	\end{align*}
	Prove that $G(X,r)$ 
% 	The structure group $G(X,r)$ with generators
% 	$x_1,x_2,x_3,x_4$ and relations
% 	\begin{align*}
% 		& x_1x_2=x_2x_4,
% 		&& x_1x_3=x_4x_2,
% 		&& x_1x_4=x_3^2,\\
% 		& x_2x_1=x_3x_4,
% 		&& x_2^2=x_4x_1,
% 		&& x_3x_1=x_4x_3.
% 	\end{align*}
	does not have the unique product property.
\end{prob}

\section*{Open problems}

\begin{problem}
    Does the group $G(X,r)$...
    A linear representation of this group is...
\end{problem}

\section*{Notes}

Multipermutation involutive solutions were introduced in~\cite{MR1722951}. 
The notion was extended to the non-involutive case in~\cite{MR3974961}. 

% \index{Ced\'o, F.}
% \index{Jespers, E.}
% \index{Van Antwerpen, A.}
% \index{Kubat, \L.}
% \index{Verwimp, C.}
Theorem~\ref{thm:CJKAV} was proved in... %by Ced\'o, Jespers, Kubat, van Antwerpen and Verwimp. 

Proposition~\ref{pro:mpl} was proved in~\cite{MR3177933} 
for involutive solutions. The general case goes back to... 

Theorem~\ref{thm:BCV} combines several results. The implication... 

Theorem~\ref{thm:CJO_mp} appears in~\cite{MR3177933}. 

Non-involutive multipermutation solutions...

The set~\eqref{eq:Promislow} appears in the work of Promislow~\cite{MR940281}.
Exercise~\ref{prob:4-13} appears in the book of Jespers and Okni\'nski, see~\cite[Example 8.2.14]{MR2301033}. 
