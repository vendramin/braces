\chapter{Multipermutation solutions}
\label{MP}


\section{The retraction of a solution}

Let $(X,r)$ be a solution to the YBE. We write
\[ r(x,y)=(\sigma_x(y),\tau_y(x))\quad\text{and}\quad r^{-1}(x,y)=(\widehat{\sigma}_x(y),\widehat{\tau}_y(x))\]
for all $x,y\in X$. 
Since $\sigma_x\sigma_y=\sigma_{\sigma_x(y)}\sigma_{\tau_y(x)}$ for all $x,y\in X$, the map
$\sigma\colon X\to\Sym_X$, defined by $\sigma(x)=\sigma_x$ for all $x\in X$,  induces
a unique homomorphism $\sigma\colon G(X,r)\to\Sym_X$, such that
$\sigma(a)=\sigma_a$ and $\sigma_{i(x)}(y)=\sigma_x(y)$ for all $x,y\in X$ and $a\in G(X,r)$, where $i \colon X\to G(X,r)$ is the natural map.
Similarly, the map
$\widehat{\sigma}\colon X\to\Sym_X$, defined by $\widehat{\sigma}(x)=\widehat{\sigma}_x$ for all $x\in X$,  induces
a unique homomorphism $\widehat{\sigma}\colon G(X,r)\to\Sym_X$, such that $\widehat{\sigma}(a)=\widehat{\sigma}_a$ and
$\widehat{\sigma}_{i(x)}(y)=\widehat{\sigma}_x(y)$ for all $x,y\in X$ and $a\in G(X,r)$, the map $\tau\colon X\to\Sym_X$, defined by $\tau (x)=\tau_x$ for all $x\in X$,  induces a unique anti-homomorphism $\tau\colon G(X,r)\to\Sym_X$, such that
$\tau(a)=\tau_a$ and $\tau_{i(x)}(y)=\tau_x(y)$ for all $x,y\in X$ and $a\in G(X,r)$, and the map $\widehat{\tau}\colon X\to\Sym_X$, defined by $\widehat{\tau}(x)=\widehat{\tau}_x$ for all $x\in X$,  induces a unique anti-homomorphism $\widehat{\tau}\colon G(X,r)\to\Sym_X$, such that $\widehat{\tau}(a)=\widehat{\tau}_a$ and
$\widehat{\tau}_{i(x)}(y)=\widehat{\tau}_x(y)$ for all $x,y\in X$ and $a\in G(X,r)$.

By Theorem \ref{thm:LYZ9}, there exists a unique braiding operator $r_G$ on $G(X,r)$ such that $r_G(i\times i)=(i\times i)r$.
We write 
\[r_G(a,b)=(\lambda_a(b),\mu_b(a))\quad\text{and}\quad r_G^{-1}(a,b)=(\widehat{\lambda}_a(b),\widehat{\mu}_b(a))\] 
for all $a,b\in G(X,r)$. 

\begin{lemma}\label{lem:permutationlemma}
	Let $(X,r)$ be a solution to the YBE. Then
 	\begin{alignat*}{2}
		\sigma_a^{-1}(x) &=\widehat{\tau}_{\widehat{\lambda}_x^{-1}(a)}(x),\quad
		& \widehat{\sigma}_a^{-1}(x) &=\tau_{\lambda_{i(x)}^{-1}(a)}(x),\\
		\tau_a^{-1}(x) &=\widehat{\sigma}_{\widehat{\mu}_{i(x)}^{-1}(a)}(x),\quad
		& \hat{\tau}_a^{-1}(x) &=\sigma_{\mu_{i(x)}^{-1}(a)}(x)
 	\end{alignat*}
 	for all $a\in G(X,r)$ and $x\in X$.
 	\begin{proof}
		Let $\tilde{g}\colon G(X,r)\to\Sym_X$ be the map defined by $\tilde{g}(a)=\tilde{g}_a$
		and $\tilde{g}_a(x)=\tau_{\lambda_{i(x)}^{-1}(a)}(x)$ for all $a\in G(X,r)$ and $x\in X$. We shall prove that $\tilde{g}$ is an anti-homomorphism
		of groups. 
		Let $b_1,b_2\in G(X,r)$ and $x\in X$. We have that
		\begin{align*}
			\tilde{g}_{b_2b_1}(x)=&\tau_{\lambda^{-1}_{i(x)}(b_2b_1)}(x)=\tau_{\lambda_{i(x)^{-1}}(b_2b_1)}(x)\\
			=&\tau_{\lambda_{i(x)^{-1}}(b_2)\lambda_{\mu_{b_2}(i(x)^{-1})}(b_1)}(x)\\
			=&\tau_{\lambda_{\mu_{b_2}(i(x)^{-1})}(b_1)}\tau_{\lambda_{i(x)^{-1}}(b_2)}(x)\\
			=&\tau_{\lambda_{\mu_{\lambda_{i(x)^{-1}}(b_2)}(i(x))^{-1}}(b_1)}\tau_{\lambda_{i(x)^{-1}}(b_2)}(x)\\
			=&\tau_{\lambda^{-1}_{\mu_{\lambda^{-1}_{i(x)}(b_2)}(i(x))}(b_1)}\tau_{\lambda^{-1}_{i(x)}(b_2)}(x)\\
			=&\tau_{\lambda^{-1}_{i(\tau_{\lambda^{-1}_{i(x)}(b_2)}(x))}(b_1)}\tau_{\lambda^{-1}_{i(x)}(b_2)}(x)\\
			=&\tilde{g}_{b_1}(\tau_{\lambda^{-1}_{i(x)}(b_2)}(x))\\
			=&\tilde{g}_{b_1}\tilde{g}_{b_2}(x).
		\end{align*}
		Hence $\tilde{g}$ is an anti-homomorphism of groups.
		
		
		
		Similarly, one verifies that the map $\tilde{f}\colon G(X,r)\to\Sym_X$, defined by $\tilde{f}(a)=\tilde{f}_a$ for all $a\in G(X,r)$,
		where $\tilde{f}_a(x)=\sigma_{\mu_{i(x)}^{-1}(a)}(x)$ for all $a\in G(X,r)$ and $x\in X$, is a homomorphism of groups.
		Note that the map $\widehat{\sigma}^{-1}\colon G(X,r)\to\Sym(X)$ defined by $\widehat{\sigma}^{-1}(a)=\widehat{\sigma}_a^{-1}$ is an anti-homomorphism
		of groups. Since 
		\[\widehat{\sigma}_{i(x)}^{-1}(y)=\widehat{\sigma}_{x}^{-1}(y)=\tau_{\sigma_{y}^{-1}(x)}(y)=\tau_{i(\sigma_{y}^{-1}(x))}(y)=\tau_{\lambda_{i(y)}^{-1}(i(x))}(y)=\tilde{g}_{i(x)}(y)\] for all $x,y\in X$, we have that
		$\widehat{\sigma}^{-1}=\tilde{g}$, i.e., 
		\[\widehat{\sigma}_a^{-1}(x)=\tau_{\lambda_{i(x)}^{-1}(a)}(x)\] for all $a\in G(X,r)$
		and $x\in X$. We also have that the map $\widehat{\tau}^{-1}\colon G(X,r)\to\Sym(X)$, defined by $\widehat{\tau}^{-1}(a)=\widehat{\tau}_a^{-1}$ for all $a\in G(X,r)$, is a homomorphism
		of groups. Since 
		\[\widehat{\tau}_{i(x)}^{-1}(y)=\widehat{\tau}_{x}^{-1}(y)=\sigma_{i(\tau_{y}^{-1}(x))}(y)=\sigma_{\mu_{i(y)}^{-1}(i(x))}(y)=\tilde{f}_{i(x)}(y)\] for all $x,y\in X$, we have that
		$\widehat{\tau}^{-1}=\tilde{f}$, i.e., \[\widehat{\tau}_a^{-1}(x)=\sigma_{\mu_{i(x)}^{-1}(a)}(x)\] for all $a\in G(X,r)$ and
		$x\in X$. This proves two of the equalities in the statement of the result. The other two equalities follow similarly. 
 	\end{proof}
\end{lemma}

\begin{lemma}\label{lem:Permutation}
	Let $(X,r)$ be a solution to the YBE. Then 
	\[\ker\sigma\cap\ker\tau=\ker\sigma\cap\ker\widehat{\sigma}=\ker\widehat{\sigma}\cap\ker\widehat{\tau}=\ker\tau\cap\ker\widehat{\tau}.\]
	\begin{proof}
	By the proof of Lemma \ref{lem:Kerh}, $\ker\sigma\subseteq\ker\lambda$ and $\ker\tau\subseteq\ker\mu$.
		
		First we shall see that $\mu_{i(x)}(\ker\sigma)=\ker\sigma$ for all $x\in X$. To do so, let us fix
		$a\in\ker\sigma$ and $x\in X$. We have
		\[\sigma_{i(x)}=\sigma_a\sigma_{i(x)}=\sigma_{\lambda_a(i(x))}\sigma_{\mu_{i(x)}(a)}=\sigma_{i(x)}\sigma_{\mu_{i(x)}(a)}\]
		and \[\sigma_{i(x)^{-1}}=\sigma_a\sigma_{i(x)^{-1}}=\sigma_{\lambda_a(i(x)^{-1})}\sigma_{\mu_{i(x)^{-1}}(a)}=
		\sigma_{i(x)^{-1}}\sigma_{\mu_{i(x)^{-1}}(a)}.\] Hence $\sigma_{\mu_{i(x)}(a)}=\id$ and
		$\sigma_{(\mu_i(x))^{-1}(a)}=\id$, and thus $\mu_{i(x)}(\ker\sigma)=\ker\sigma$.
		
		Similarly one proves that
		\[\lambda_{i(x)}(\ker\tau)=\ker\tau,\quad \widehat{\mu}_{i(x)}(\ker\widehat{\sigma})=\ker\widehat{\sigma},\quad
		\widehat{\lambda}_{i(x)}(\ker\widehat{\tau})=\ker\widehat{\tau}\] for all $x\in X$.
		
		Now, we shall see that \[\ker\sigma\cap\ker\widehat{\sigma}=\ker\sigma\cap\ker\tau=\ker\tau\cap\ker\widehat{\tau}.\]  
		Indeed, let $a\in\ker\sigma\cap\ker\widehat{\sigma}$. For every $x\in X$,
		we have $\widehat{\sigma}_{\widehat{\mu}_{i(x)}^{-1}(a)}=\id$ and $\sigma_{\mu_{i(x)}^{-1}(a)}=\id$. Thus by
		Lemma~\ref{lem:permutationlemma}, 
		\[\tau_a^{-1} (x)=\widehat{\sigma}_{\widehat{\mu}_{i(x)}^{-1}(a)}(x)=x\quad\text{and}\quad
		\widehat{\tau}_a^{-1}(x)=\sigma_{\mu_{i(x)}^{-1}(a)}(x)=x\] for all $x\in X$. This shows that $\ker\sigma\cap\ker\widehat{\sigma}\subseteq\ker\tau\cap\ker\widehat{\tau}$. The other inclusion follows by a symmetric argument. Hence $\ker\sigma\cap\ker\widehat{\sigma}=\ker\tau\cap\ker\widehat{\tau}$.
		
		Note that we have also proven that $\ker\sigma\cap\ker\widehat{\sigma}\subseteq\ker\sigma\cap\ker\tau$. Let $b\in\ker\sigma\cap\ker\widehat{\tau}$.
		As $\ker\tau$ is $\lambda_{i(x)}$-invariant, for every $x\in X$, we have that $\tau_{\lambda_{i(x)}^{-1}(b)}=\id$.
		Therefore $\widehat{\sigma}_b^{-1}(x)=\tau_{\lambda_{i(x)}^{-1}(b)}(x)=x$ for all $x\in X$. This shows that
		$\ker\sigma\cap\ker\tau\subseteq\ker\sigma\cap\ker\widehat{\sigma}$, and thus $\ker\sigma\cap\ker\tau=\ker\sigma\cap\ker\widehat{\sigma}$. The other equality follows by symmetry between $(X,r)$ and $(X,r^{-1})$.
	\end{proof}
\end{lemma}


Note that in case the solution $(X,r)$ is involutive, we have $\sigma=\widehat{\sigma}$ and $\tau=\widehat{\tau}$ and therefore \[\ker \sigma=\ker\sigma\cap\ker\tau=\ker\tau.\] 



Let $(X,r)$ be a solution to the YBE. 
We consider over $X$ the relation
\[x\sim y\Longleftrightarrow \sigma_x=\sigma_y\text{ and }\tau_x=\tau_y.\]
Then $\sim$ is an equivalence relation. Let $\overline{X}$ be the set of equivalence classes and $[x]$ denote the equivalence class of $x$. 


\begin{proposition}\label{prop:retractissolution}
	Let $(X,r)$ be a solution to the YBE. Then $r$ induces a solution $\overline{r}$ to the YBE on $\overline{X}$, by
	\[\overline{r}([x],[y])=([\sigma_x(y)],[\tau_y(x)])\]
	for all $x,y\in X$. 
	%This solution $(\overline{X},\overline{r})$ is denoted $\Ret(X,r)$
	%and is called the {\em retract} of $(X,r)$. One says that $(X,r)$ is {\em retractable} if $\sim$ is not the trivial relation.
	\end{proposition}

	\begin{proof}
		Let $x,y,z\in X$ be elements such that $x\sim y$. Note that
		\begin{align*}
			\sigma_{\sigma_x(z)}\sigma_{\tau_z(x)} &
			=\sigma_x\sigma_z=\sigma_y\sigma_z=\sigma_{\sigma_y(z)}\sigma_{\tau_z(y)}
			=\sigma_{\sigma_x(z)}\sigma_{\tau_z(y)},\\
			\tau_{\tau_x(z)}\tau_{\sigma_z(x)} &=\tau_x\tau_z=\tau_y\tau_z
			=\tau_{\tau_y(z)}\tau_{\sigma_z(y)}=\tau_{\tau_x(z)}\tau_{\sigma_z(y)},\\
			\sigma_{\sigma_z(x)}\sigma_{\tau_x(z)} &=\sigma_z\sigma_x
			=\sigma_z\sigma_y=\sigma_{\sigma_z(y)}\sigma_{\tau_y(z)}
			=\sigma_{\sigma_z(y)}\sigma_{\tau_x(z)},\\
			\tau_{\tau_z(x)}\tau_{\sigma_x(z)} &=\tau_z\tau_x=\tau_z\tau_y
			=\tau_{\tau_z(y)}\tau_{\sigma_y(z)}=\tau_{\tau_z(y)}\tau_{\sigma_x(z)}.
		\end{align*}
		Hence $\sigma_z(x)\sim\sigma_z(y)$ and $\tau_z(x)\sim\tau_z(y)$. Therefore $\overline{r}$ is well-defined.
		Note that $x\sim y$ if and only if $i(x)i(y)^{-1}\in\ker\sigma\cap\ker\tau\subseteq G(X,r)$. It follows from Lemma~\ref{lem:Permutation}
		that $\widehat{\sigma}_z(x)\sim\widehat{\sigma}_z(y)$ and $\widehat{\tau}_z(x)\sim\widehat{\tau}_z(y)$, and so the map
		$s:\overline{X}^2\to\overline{X}^2$, defined by $s([x],[y])=([\widehat{\sigma}_x(y)],[\widehat{\tau}_y(x)])$ for all $x,y\in X$,	is well-defined. Clearly $s=\overline{r}^{-1}$. It is clear that $\overline{r}$ is a bijective set-theoretic
	 	solution to the Yang--Baxter equation. We will prove that it is non-degenerate. Define
		$\sigma_{[x]}\colon\overline{X}\to\overline{X}$ and $\tau_{[x]}\colon \overline{X}\to\overline{X}$
		by $\sigma_{[x]}([y])=[\sigma_x(y)]$ and $\tau_{[x]}([y])=[\tau_x(y)]$ for all $x,y\in X$.
		Clearly, in order to prove that $\sigma_{[x]}$ is bijective, it is enough to prove that it is injective (surjectivity
		of $\sigma_{[x]}$ is an immediate consequence of surjectivity of $\sigma_x$).
	 	
		
		Let $x,y,z\in X$ be elements such that $\sigma_z(x) \sim \sigma_z(y)$. We have that
		\[x=\sigma_z^{-1}\sigma_z(x)=\widehat{\tau}_{\widehat{\sigma}^{-1}_{\sigma_z(x)}(z)}(\sigma_z(x))
		\sim\widehat{\tau}_{\widehat{\sigma}^{-1}_{\sigma_z(x)}(z)}(\sigma_z(y))
		=\widehat{\tau}_{\widehat{\sigma}^{-1}_{\sigma_z(y)}(z)}(\sigma_z(y))=y.\]
		Hence $\sigma_{[x]}$ is injective and thus it is bijective. Similarly one can prove that
		$\tau_{[x]}$ is bijective. Hence $\overline{r}$ is non-degenerate, and the result follows.
	\end{proof}


%Let $(X,r)$ be a  solution to the YBE. We 
%consider over $X$ the relation
%\[
%x\sim y\Longleftrightarrow \sigma_x=\sigma_y\text{ and }\tau_x=\tau_y.
%\]
%Then $\sim$ is an equivalence relation. 
%Let $\overline{X}$ be the set of equivalence classes 
%and $[x]$ 
%$\overline{x}$ 
%denote the equivalence class of $x$. 

%\begin{proposition}
%Let $(X,r)$ be a solution to the YBE. Then $(\overline{X},\overline{r})$, where 
%$\overline{r}(\overline{x},\overline{y})=(\overline{\sigma_x(y)},\overline{\tau_y(x)})$, 
%\[
%\overline{r}([x],[y])=([\sigma_x(y)],[\tau_y(x)]),
%\]
%is a solution to the YBE. 
%\end{proposition}

%\begin{proof}
%    We define $s\colon \overline{X}\times \overline{X}\to \overline{X}\times \overline{X}$ by $s([x],[y])=([\widehat{\sigma}_x(y)],[\widehat{\tau}_y(x)])$.
%    We first prove that $\overline{r}$ and $s$ are well-defined. 
%    Let $x,y,u,v\in X$ be such that $x\sim u$ and $y\sim v$. Let $h\colon G(X,r)\to \mathcal{G}(X,r)$ be the homomorphism of skew braces defined by $h(a)=(\sigma_a,\tau^{-1}_a)$. Note that 
%    \[ x\sim u \Longleftrightarrow h(i(x))=h(i(u))\]
%    where $i\colon X\to G(X,r)$ is the natural map. Let $r_G\colon G(X,r)\times G(X,r)\to G(X,r)\times G(X,r)$ be the braiding operator on $G(X,r)$ defined by $r_G(a,b)=(\lambda_a(b),\mu_b(a))$, for all $a,b\in G(X,r)$. We write $r_G^{-1}(a,b)=(\widehat{\lambda}_a(b),\widehat{\mu}_b(a))$ for all $a,b\in G(X,r)$.
%    Now we have
%    \begin{align*}
%        h(i(\sigma_x(y)))=&) h(i(\sigma_u(y)))=h(i(u)i(y)(i(\tau_y(u)))^{-1})\\
%        =& h(i(u))h(i(y))h(i(\tau_v(u)))^{-1}\\
%        =& h(i(u))h(i(v))h(i(\tau_v(u)))^{-1}\\
%        =& h(i(u)i(v)(i(\tau_v(u))^{-1})=h(i(\sigma_u(v)),
%    \end{align*}
%    \begin{align*}
%        h(i(\tau_y(x)))=& h(i(\tau_v(x)))=h((i(\sigma_x(v)))^{-1}i(x)i(v))\\
%        =& h(i(\sigma_u(v)))^{-1} h(i(x))h(i(v))\\
%        =& h(i(\sigma_u(v)))^{-1} h(i(u))h(i(v))\\
%        =& h(i(\sigma_u(v))^{-1}i(u)i(v))=h(i(\tau_v(u)),
%    \end{align*}
%    and since $\ker h\subseteq\ker\lambda\cap\ker\mu$,
%    \begin{align*}
%        h(i(\widehat{\sigma}_x(y)))=& h(i(\widehat{\sigma}_x(y)))
%        =h(\widehat{\lambda}_{i(x)}i(y))\\
%        =& h(\widehat{\lambda}_{i(x)^{-1}}^{-1}i(y))=h(\mu_{\lambda^{-1}_{i(y)}(i(x)^{-1})}i(y))\\
%        =&h(\mu_{\lambda^{-1}_{i(v)}(i(x)^{-1})}i(y))\\
%        =&h(\mu_{\lambda_{i(v)^{-1}}(i(x)^{-1})}i(y))\\
%        =&h(\mu_{i(v)^{-1}i(x){-1}(\mu_{i(x)^{-1}}(i(v)^{-1}))^{-1}}i(y))\\
%        =&h(\mu_{i(v)^{-1}i(u)^{-1}(\mu_{i(u)^{-1}}(i(v)^{-1}))^{-1}}i(y))\\
%        =&h(\mu_{\lambda^{-1}_{i(v)}(i(u)^{-1})}i(y))\\
%        =&h((\lambda_{i(y)}\lambda^{-1}_{i(v)}(i(u)^{-1}))^{-1}i(y)\lambda^{-1}_{i(v)}(i(u)^{-1}))\\
%        =&h((\lambda_{i(v)}\lambda^{-1}_{i(v)}(i(u)^{-1}))^{-1}i(v)\lambda^{-1}_{i(v)}(i(u)^{-1}))\\
%        =&h(\mu_{\lambda^{-1}_{i(v)}(i(u)^{-1})}i(v))\\
%        =& h(\widehat{\lambda}^{-1}_{i(u)^{-1}}i(v))\\
%        =&h(\widehat{\lambda}_{i(u)}i(v))=h(i(\widehat{\sigma}_u(v)))
%    \end{align*}
%    and similarly
%    \begin{align*}
%        h(i(\widehat{\tau}_y(x)))=& h(i(\widehat{\tau}_v(u))). 
%    \end{align*}
%    
%    Hence $[\sigma_x(y)]=[\sigma_u(v)]$, $[\tau_y(x)]=[\tau_v(u)]$, $[\widehat{\sigma}_x(y)]=[\widehat{\sigma}_u(v)]$ and $[\widehat{\tau}_y(x)]=[\widehat{\tau}_v(u)]$, and thus $\overline{r}$ and $s$ are well-defined. Furthermore, it is clear that  $\overline{r}s=s\overline{r}=\id$, and hence $\overline{r}$ is bijective.
    
%    As above, one can check that if $x\sim u$ and $y\sim v$, then
%    $[\sigma^{-1}_x(y)]=[\sigma^{-1}_u(v)]$ and $[\tau^{-1}_x(y)]=[\tau^{-1}_u(v)]$. Hence the maps $\sigma_{[x]}\colon \overline{X}\to\overline{X}$ and $\tau_{[x]}\colon\overline{X}\to\overline{X}$ defined by
%    \[ \sigma_{[x]}([y])=[\sigma_x(y)] \quad\text{and}\quad \tau_{[x]}([y])=[\tau_x(y)],\]
%    for all $x,y\in X$, are bijective and
%    \[ \sigma^{-1}_{[x]}([y])=[\sigma^{-1}_x(y)] \quad\text{and}\quad \tau^{-1}_{[x]}([y])=[\tau^{-1}_x(y)]\]
%    for all $x,y\in X$. It is straightforward to check that 
%    \[\overline{r}_1\overline{r}_2\overline{r}_1=\overline{r}_2\overline{r}_1\overline{r}_2.\]
%    Hence $(\overline{X},\overline{r})$ is a solution to the YBE.
%\end{proof}

In the case of involutive solutions, it follows from Proposition~\ref{pro:T} 
that $\sigma_x=\sigma_y$ if and only if $\tau_x=\tau_y$. In particular, if $(X,r)$ is involutive, then
\begin{align*}\mathcal{G}(X,r) &\cong\langle \sigma_x : x\in X\rangle=\{ \sigma_a : a\in G(X,r)\}\\
&\cong\{ \lambda_a : a\in G(X,r)\}\cong G(X,r)/\Soc(G(X,r)).
\end{align*}

\begin{definition}
\index{Retraction!of a solution}
Let $(X,r)$ be a solution. The solution $\Ret(X,r)=(\overline{X},\overline{r})$ induced
by the equivalence relation $\sim$ is the \emph{retraction} of $(X,r)$.
\end{definition}

We define inductively $\Ret^0(X,r)=(X,r)$, $\Ret^1(X,r)=\Ret(X,r)$ and
\[
\Ret^{n+1}(X,r)=\Ret(\Ret^n(X,r))\quad n\geq1.
\]


\begin{definition}
    \index{Solution!multipermutation}
    A solution $(X,r)$ is said to be {\em multipermutation} solution of \emph{level} $n$ 
    if $n$ is the smallest non-negative integer 
    such that $|\Ret^n(X,r)|=1$. The solution $(X,r)$ is said to be 
    \emph{irretractable} if $\Ret(X,r)=(X,r)$. 
\end{definition}

The trivial solution over the set with one element is a multipermutation solution of level zero. 
Permutation solutions are multipermutation solutions of level one. 

\begin{example}

\end{example}

\begin{example}

\end{example}

\begin{table}[H]
\centering
\caption{Involutive solutions of size $\leq10$.}
\begin{tabular}{|r|ccccccccc|}
\hline
$n$ & 2 & 3 & 4 & 5 & 6 & 7 & 8 & 9 & 10\tabularnewline
\hline 
solutions & 2 & 5 & 23 & 88 & 595 & 3456 & 34530 & 321931 & 4895272\tabularnewline
multipermutation & 2 & 5 & 21 & 84 & 554 & 3295 & 32155 & 305916 & 4606440\tabularnewline
irretractable & 0 & 0 & 2 & 4 & 9 & 13 & 191 & 685 & 3590\tabularnewline
\hline
\end{tabular}
\label{tab:INV_mp}
\end{table}

For size $\leq7$ the numbers of Table~\ref{tab:INV_mp} coincide with those in~\cite{MR1722951}
but there are some differences for solutions of size eight. 

\begin{table}[H]
\centering
\caption{Non-involutive solutions of size $\leq8$.}
\begin{tabular}{|r|ccccccc|}
\hline
$n$ & 2 & 3 & 4 & 5 & 6 & 7 & 8 \tabularnewline
\hline 
solutions & 2 & 21 & 230 & 3519 & 100071 & 4602720  & 422449480 \tabularnewline
multipermutation &  & 15 & 206 & 3165 & 95517 & 4461805 & 416725250 \tabularnewline
irretractable & & 6 & 24 & 98 & 514 & 2659 & 17370\tabularnewline
\hline
\end{tabular}
\label{tab:mp}
\end{table}
%[ 422449480, 416725250, 17370, 1069 ]



\begin{theorem}
\label{thm:CJKAV}
    Let $(X,r)$ be a finite multipermutation solution to the YBE. If $|X|>1$, then $r$ has even order. 
\end{theorem}

\begin{proof}
    Since $(X,r)\to\Ret(X,r)$, $x\mapsto[x]$ is a homomorphism of solutions, 
    it follows that the order of the solution $\overline{r}$ divides de orden of $r$. 
    Assume that $(X,r)$ has multipermutation level $n$. 
    There exists a homomorphism of solutions $(X,r)\to\Ret^{n-1}(X,r)$, thus 
    it is enough to prove the theorem in the case where
    $r(x,y)=(\sigma(y),\tau(x))$ for commuting permutations $\sigma$ and $\tau$, i.e. 
    multipermutation solutions of level one. If $r$ has order $2k+1$, then 
    \[
    (x,y)=r^{2k+1}(x,y)=(\sigma^{k+1}\tau^k(y),\sigma^k\tau^{k+1}(x)).
    \]
    This implies that $\sigma^{k+1}\tau^k(y)=x$ for all $x,y\in X$. This equality in particular 
    implies that $x=y$ because $\sigma^{k+1}\tau^k$ is a permutation, a contradiction. 
\end{proof}

The connection between the socle of a skew brace and the 
retract of a solution was discovered by Rump in the case of 
involutive solutions and skew braces of abelian type, see~\cite{MR2278047}. 

\begin{proposition}
\label{pro:add_cyclic}
Let $A$ be a skew brace and $(A,r)$ be its associated solution. 
Then the retraction $\Ret(A,r)$ is the canonical solution associated with the quotient brace $A/\Soc(A)$. 
\end{proposition}

\begin{proof}
    The equivalence relation $\sim$ on $A$ is defined as $a\sim b$ if and only if $\lambda_a=\lambda_b$ and 
    $\mu_a=\mu_b$. Let $\overline{A}$ be the set of equivalence classes. 
    The equivalence class of an element $a$ is then
    \begin{align*}
    [a]&=\{b\in A:a\sim b\}
    =\{b\in A:\lambda_a=\lambda_b,\,\mu_a=\mu_b\}\\
    &=\{b\in A:a'\circ b\in \ker\lambda\cap\ker\mu\}=\{b\in A:a'\circ b\in\Soc(A)\},
    \end{align*}
    by Proposition~\ref{pro:soc_kernels}. This means that 
    $[a]=[b]$ if and only if $\pi(a)=\pi(b)$, where $\pi\colon A\to A/\Soc(A)$, $x\mapsto x\circ\Soc(A)$,  
    is the canonical skew brace homomorphism. Moreover, $A/\Soc(A)=\overline{A}$ as sets. 
    Now we compute the retraction of $(A,r)$:
    \begin{align*}
        \overline{r}([a],[b]) &= ([\lambda_a(b)],[\mu_b(a)])
        =(\pi(\lambda_a(b)),\pi(\mu_b(a)))\\
        &=\left(\lambda_{\pi(a)}(\pi(b)),\mu_{\pi(b)}(\pi(a))\right)
        =\left(\lambda_{[a]}([b]),\mu_{[b]}([a])\right).
    \end{align*}
    Therefore $\Ret(A,r)=(A/\Soc(A),\overline{r})$, and the result follows. 
\end{proof}

Note that a skew brace $A$ has finite multipermutation level (see Definition \ref{def:bracempl}) if and only if its associated solution $(A,r_A)$ is a multipermutation solution.

\begin{proposition}
\label{pro:bracefinmpl}
Let $A$ be a skew brace of nilpotent type. Then its associated solution $(A,r_A)$ is a multipermutation solution if and only if $A$ is right nilpotent.
\end{proposition}

\begin{proof}
This follows by Theorem \ref{thm:mpl&right_nilpotent}.
\end{proof}

\section{Multipermutation solutions}

In this section we study some properties of multipermutation solutions and its structure group.

\begin{proposition}
\label{pro:mpl}
Let $(X,r)$ and $(Y,s)$ be solutions to the YBE. Each surjective homomorphism 
of solutions $f\colon (X,r)\to (Y,s)$ 
induces a surjective homomorphism 
of solutions $\Ret(X,r)\to\Ret(Y,s)$. In particular, if $(X,r)$ is a multipermutation solution of level $m$ and $(Y,s)$ is an homomorphic image of $(X,r)$, then $(Y,s)$ is a multipermutation solution of level $\leq m$. 
\end{proposition}

\begin{proof}
Write $r(x,y)=(\sigma_x(y),\tau_y(x))$ and $s(z,t)=(\sigma'_z(t),\tau'_t(z))$. 
Let $x,x_1\in X$ be such that $x\sim x_1$. If $y\in X$, then 
\[
\sigma'_{f(x)}f(y)=f(\sigma_x(y))=f(\sigma_{x_1}(y))=\sigma'_{f(x_1)}f(y).
\]
Since $f$ is surjective, it follows that $\sigma'_{f(x)}=\sigma'_{f(x_1)}$. A similar calculation proves 
that $\tau'_{f(x)}=\tau'_{f(x_1)}$. If $\pi\colon (Y,s)\to\Ret(Y,r)$, $y\mapsto [y]$, is the canonical map, 
the composition $\pi f\colon (X,r)\to\Ret(Y,s)$ is a surjective homomorphism of solutions. Therefore 
the map $\Ret(X,r)\to\Ret(Y,s)$, $[x]\mapsto\pi(f(x))$, is then a well-defined surjective
homomorphism of solutions. This proves the first part of the result. The second part follows easily from the first part and induction on $m$.
\end{proof}

\begin{proposition}
\label{pro:mpl_subsol}
Let $(X,r)$ be a solution to the YBE of finite multipermutation level $m$ 
and $Y\subseteq X$ be such that $r(Y\times Y)\subseteq Y\times Y$. 
Then the \emph{subsolution} $(Y,r|_{Y\times Y})$ is of finite multipermutation level $\leq m$.  
\end{proposition}

\begin{proof}
We shall prove the result by induction on $m$. For $m=0$, $|X|=1$ and thus $Y=X$ and the result holds in this case. 
Suppose that $m\geq 1$ and that the result holds for all solutions to the YBE of multipermutation level $<m$.
Let $\Ret(X,r)=(\overline{X},\overline{r})$. Let $\pi\colon (X,r)\to\Ret(X,r)$ be the natural map. Clearly $(\pi(Y), \overline{r}|_{\pi(Y)^{2}})$ is a subsolution of $\Ret(X,r)$. Since $\Ret(X,r)$ has multipermutaion level $m-1$, 
by the inductive hypothesis, $(\pi(Y), \overline{r}|_{\pi(Y)^{2}})$ has multipermutation level $\leq m-1$.  Let $\sim_X$ be the retract relation on $X$ and $\sim_Y$ the retract relation on $Y$. For $y\in Y$, we denote by $[y]_X$ the $\sim_X$-class of $y$ and by $[y]_Y$ the $\sim_Y$-class of $y$, i.e.
\[ [y]_X=\{ x\in X : x\sim_X y\}\quad \text{and}\quad [y]_Y=\{ z\in Y :z\sim_Y y\}.\]
Note that if $x,y\in Y$ and $x\sim_X y$, then $x\sim_Y y$. Thus the map $f\colon (\pi(Y),\overline{r}|_{\pi(Y)^2})\to\Ret(Y,r|_{Y^2})$, defined by $f([y]_X)=[y]_Y$ for all $y\in Y$, is a surjective homomorphism of solutions. By Proposition \ref{pro:mpl}, $\Ret(Y,r|_{y^2})$ has multipermutation level $\leq m-1$. Hence $(Y,r|_{Y^2})$ has multipermutation level $\leq m$ and the result follows by induction. 
\end{proof}

\begin{theorem}\label{thm:multiG}
Let $(X,r)$ be a solution. The following statements are equivalent:
\begin{enumerate}
    \item $(X,r)$ has finite multipermutation level.
    \item $\left(\mathcal{G}(X,r),r_{\mathcal{G}(X,r)}\right)$ has finite multipermutation level.
    \item $\left(G(X,r),r_{G(X,r)}\right)$ has finite multipermutation level.
    %\item The brace $G(X,r)$ is right nilpotent.
    %\item The brace $\mathcal{G}(X,r)$ is right nilpotent. 
\end{enumerate}
\end{theorem}

\begin{proof}
    Let us first prove that (2) implies (1). 
    The map $X\to\mathcal{G}(X,r)$, $x\mapsto(\lambda_x,\mu_x^{-1})$, is a homomorphism of solutions that induces
    an injective homomorphism of solutions $j\colon\Ret(X,r)\to\left(\mathcal{G}(X,r),r_{\mathcal{G}(X,r)}\right)$. 
    Since $\left(\mathcal{G}(X,r),r_{\mathcal{G}(X,r)}\right)$ has finite multipermutation level,
    $(X,r)$ has finite multipermutation level by Proposition~\ref{pro:mpl_subsol}.
    
    Let us now prove that (3) implies (2). The canonical map $G(X,r)\to\mathcal{G}(X,r)$ yields a surjective
    homomorphism of solutions. Then Proposition~\ref{pro:mpl} applies. 

    Finally we shall prove that (1) implies (3). In fact, we shall prove that if $(X,r)$ is a multipermutation solution to the YBE of level $m$, then $(G(X,r),r_{G(X,r)})$ is a multipermutation solution of level $\leq m+1$, by induction on $m$. For $m=0$, we have that $|X|=1$ and $G(X,r)$ is the trivial skew brace of abelian type $(\Z,+)$. Hence $\left(G(X,r),r_{G(X,r)}\right)$ has multipermutation level $1$ in this case. Suppose that $m\geq 1$ and $\left(G(Y,s),r_{G(Y,s)}\right)$ has finite multipermutation level $\leq k+1$ for every multipermutation solution $(Y,s)$ to the YBE of level $k<m$. By the inductive hypothesis, $\left(G(\Ret(X,r)),r_{G(\Ret(X,r))}\right)$ has finite multipermutation level $\leq m$. By Lemmas \ref{lem:Kerh} and \ref{lem:kerhideal}, there exists a surjective homomorphism of skew braces $\pi\colon \mathcal{G}(X,r)\to G(X,r)/\Soc(G(X,r))$. By Proposition \ref{pro:add_cyclic}, $\Ret(G(X,r),r_{G(X,r)})=\left( G(X,r)/\Soc(G(X,r)),r_{G(X,r)/\Soc(G(X,r))}\right)$. Hence $\pi$ also is a surjective homomorphism of solutions
    \[ \pi\colon \left(\mathcal{G}(X,r), r_{\mathcal{G}(X,r)}\right)\to \Ret(G(X,r)).\]
    By Theorem \ref{thm:LYZ9}, the injective homomorphism of solutions $j\colon\Ret(X,r)\to\left(\mathcal{G}(X,r),r_{\mathcal{G}(X,r)}\right)$ induces a homomorphism of braided groups \[\varphi\colon \left(G(\Ret(X,r)),r_{G(\Ret(X,r))}\right)\to \left(\mathcal{G}(X,r),r_{\mathcal{G}(X,r)}\right),\] such that $\varphi i([x])=j([x])=(\sigma_x,\tau^{-1}_x)$ for all $x\in X$, where \[i\colon \Ret(X,r)\to G(\Ret(X,r))\] is the natural map. Clearly $\varphi$ is surjective and thus,
    $\pi\varphi$ is a surjective homomorphism of solutions. Since $\left(G(\Ret(X,r)),r_{G(\Ret(X,r))}\right)$ has finite multipermutation level $\leq m$, by Proposition \ref{pro:mpl}, $\Ret(G(X,r))$ has finite multipermutation level $\leq m$. Hence $\left(G(X,r),r_{G(X,r)}\right)$ has finite multipermutation level $\leq m+1$. Therefore the result follows by induction.
    \end{proof}
    
\begin{proposition}\label{pro:multiperbrace}
Let $B$ be a skew brace such that its associated solution $(B,r_B)$ to the YBE has finite multipermutation level $m$.
Then $(B,\circ)$ is solvable of derived length $\leq m$ and $(B,+)$ is nilpotent of nilpotency class $\leq m$.
\end{proposition}

\begin{proof}
By Proposition \ref{pro:add_cyclic}, $\Ret(B,r_B)=(B/\Soc(B),r_{B/\Soc(B)})$. Since $\Soc(B)$ is a trivial skew brace an $(\Soc(B),+)$ is central in $(B,+)$, the result follows easily by induction on $m$.
\end{proof}    
An easy consequence of the proof of Theorem \ref{thm:multiG} and Proposition \ref{pro:multiperbrace} is the following result.

\begin{theorem}
    Let $(X,r)$ be a multipermutation solution to the YBE of level $m$. Then the multiplicative group of the skew brace $G(X,r)$ is solvable of derived legnth $\leq m+1$ and $(G(X,r),+)$ is nilpotent of nilpotency class $\leq m+1$.
\end{theorem}

\begin{proof}
  By the proof of Theorem \ref{thm:multiG}, $\left( G(X,r),r_{G(X,r)}\right)$ is a multipermutation solution of level $\leq m+1$. The result follows by Proposition \ref{pro:multiperbrace}.
\end{proof}

\section{Finite multipermutation solutions} 

The following result appeared in~\cite{MR1722951}.

\begin{proposition}
Let $(X,r)$ be a finite involutive solution to the YBE. If the additive group 
of the skew brace $\mathcal{G}(X,r)$ is cyclic, then $(X,r)$ 
is multipermutation.
\end{proposition}

\begin{proof}
    Let $(X,r)$ be a counterexample of minimal cardinality. If $K$ is the additive 
    group of $\mathcal{G}(X,r)$, then $K$ is finite and cyclic. 
    Write $G$ for the multiplicative group of $\mathcal{G}(X,r)$. 
    Since $|\Aut(K)|=\varphi(|K|)<|K|$, where $\varphi$ is the Euler function, 
    the group homomorphism $\lambda\colon G\to\Aut(K)$ 
    has a non-trivial kernel, so $\Soc(\mathcal{G}(X,r))$ is non-zero.  
    This implies that $(X,r)$ is retractable. 
    Since $\mathcal{G}(X,r)/\Soc(\mathcal{G}(X,r))$ is a brace with cyclic 
    additive group and $\Ret(X,r)$ is an involutive solution, the 
    minimality of $|X|$ implies that 
    $\Ret(X,r)$ is a multipermutation solution, and hence so is $(X,r)$, 
    a contradiction. 
\end{proof}

The converse of the previous proposition does not hold. 

\begin{example}
	Let $X=\{1,2,3,4\}$ and $r(x,y)=(\varphi_x(y),\varphi_y(x))$, where
	\[
		\varphi_1=\varphi_2=\id,\quad
		\varphi_3=(34),\quad
		\varphi_4=(12)(34).
 	\]
	Then $(X,r)$ is an involutive multipermutation solution. One easily checks
	that $\mathcal{G}(X,r)\cong C_2\times C_2$.
\end{example}

A similar idea proves the following result:

\begin{theorem}
\label{thm:mul_cyclic}
Let $(X,r)$ be a finite involutive solution to the YBE. If the multiplicative group 
of the brace $\mathcal{G}(X,r)$ is cyclic, then $(X,r)$ 
is multipermutation.
\end{theorem}

\begin{proof}
    Let $(X,r)$ be a counterexample of minimal cardinality. Write $K$ for the additive 
    group of $\mathcal{G}(X,r)$ and 
    $G=\langle g\rangle$ for the multiplicative group of $\mathcal{G}(X,r)$. Since 
    the image of the group homomorphism $\lambda\colon G\to\Aut(K)$ is cyclic generated by $\lambda_g$ and 
    $|\lambda_g|<|G|$ by Horosevskii's theorem, see~\cite[Corollary 3.3]{MR2426855}, it follows that 
    $\lambda$ has a non-trivial kernel, so $\Soc(\mathcal{G}(X,r))$ is non-zero.  
    This implies that $(X,r)$ is retractable. 
    Since $\mathcal{G}(X,r)/\Soc(\mathcal{G}(X,r))$ is a brace with cyclic 
    additive group and $\Ret(X,r)$ is an involutive solution, the 
    minimality of $|X|$ implies that 
    $\Ret(X,r)$ is a multipermutation solution, and hence so is $(X,r)$, 
    a contradiction. 
\end{proof}

The previous result does not hold in the case of arbitrary solutions.

\begin{example}
Let $X=\{1,2,3,4,5,6\}$ and $r(x,y)=(\sigma_x(y),\tau_y(x))$, where
\begin{align*}
    \sigma_1 &= \id, & \sigma_2&=\id, & \sigma_3 &= \id,\\
    \sigma_4&=(23)(56), &\sigma_5 &= (23)(56), & \sigma_6&=(23)(56),\\
    \tau_1 &= \id, & \tau_2&=(456), & \tau_3 &= (465),\\
    \tau_4&=\id, &\tau_5 &= (465), & \tau_6&=(456).
\end{align*}
The brace $\mathcal{G}(X,r)$ has multiplicative group isomorphic to $\Sym_3$ and
additive group isomorphic to the cyclic group of order six. 
\end{example}

We will see later that Theorem~\ref{thm:mul_cyclic}
is true for braces of nilpotent type. 
The following example appears in the work of Rump~\cite{MR2278047}.

\begin{proposition}
\label{pro:radical}
Let $A$ be a finite non-trivial radical ring. Then $\Soc(A)\ne\{0\}$ and 
$(A,r_A)$ is an involutive multipermutation solution. 
\end{proposition}

\begin{proof}
Let $A$ be a counterexample of minimal size. This means that $\Soc(A)=\{0\}$. 
Since $A$ is finite, there exists a non-zero minimal left ideal $I$ of $A$.
Recall that $A$ is a radical ring with product $a*b=\lambda_a(b)-b$. 
Since $A$ is a radical ring, $A$ is a nil ring, which implies by 
Nakayama's lemma that $I*A=\{0\}$. This means that if $x\in I$, 
then $x\in\Soc(A)$, as $0=x*a=\lambda_x(a)-a$ for all $a\in A$. 
In particular, $\Soc(A)\ne\{0\}$, a contradiction. 
\end{proof}

Proposition~\ref{pro:radical} has a nice application. The results appeared first in~\cite{MR3177933}. 

\begin{theorem}
\label{thm:CJO_abelian}
    Let $(X,r)$ be a finite involutive solution to the YBE. If the multiplicative group of the skew brace 
    $\mathcal{G}(X,r)$ is abelian, then $(X,r)$ is multipermutation. 
\end{theorem}

\begin{proof}
    Note that $(\mathcal{G}(X,r), +,*)$ is a finite radical ring. By Proposition~\ref{pro:radical}, $\mathcal{G}(X,r)$  either is zero or has non-zero socle. By Proposition \ref{pro:add_cyclic}, it follows easily that     
    $\left(\mathcal{G}(X,r),r_{\mathcal{G}(X,r)}\right)$ is a multipermutation solution. Therefore, the result follows by Theorem \ref{thm:multiG}.
\end{proof}



%In~\cite{MR2095675}, Gateva--Ivanova conjectured that finite involutive 
%square-free solutions are retractable.


%In~\cite{MR3861714} Gateva--Ivanova asked when...

%Right nilpotency...

The following theorem characterizes finite multipermutation involutive solutions to the YBE in 
terms of its structure group. 

%A group $G$ is said to be 
%\emph{left ordered} if it admits a total ordering $<$ such that 
%\[
%x<y\implies zx<zy
%\]
%for all $x,y,z\in G$. Torsion-free abelian groups, 
%free groups and braid groups are left ordered groups. 
%See~\cite{MR3560661} for more information on ordered groups. 

 \begin{theorem}
\label{thm:BCV}
	Let $(X,r)$ be a finite involutive solution to the YBE. The following statements are equivalent:
	\begin{enumerate}
		\item $(X,r)$ is a multipermutation solution.
		\item Every non-trivial subgroup of $G(X,r)$ has non-trivial center.
		\item $G(X,r)$ is poly-$\Z$.
		\item $G(X,r)$ is left orderable.
		\item $G(X,r)$ is locally incicable.
		\item $G(X,r)$ is diffuse.
	\end{enumerate}
\end{theorem}

\begin{proof}
Let $G=G(X,r)$ and $\mathcal{G}=\mathcal{G}(X,r)$. By Theorem \ref{thm:finiteinvol}, $G$ is a Bieberbach group.
By Theorem \ref{thm:leftorderedBieberbach}, the statemens (2)-(6) are equivalent.

$(1)\implies (3).$ Suppose that $(X,r)$ has multipermutation level $m$. We shall prove (3) by induction on $m$. For $m=0$, $G\cong \Z$ and thus (3) holds. Suppose that $m\geq 1$ and that $G(Y,s)$ is poly-$\Z$ for every finite involutive solution $(Y,s)$ to the YBE of multipermutation level $<m$. Hence, by the inductive hypothesis $G(\Ret(X,r))$ is poly-$\Z$. 
Let $\pi \colon X\to\overline{X}$ be the map defined by $f(x)=[x]$ for all $x\in X$, where $\Ret(X,r)=(\overline{X},\overline{r})$.
By Theorem \ref{thm:GVbraces}, there is a unique homomorphism of skew braces $\overline{f}\colon G\to G(\overline{X}.\overline{r})$ such that $\overline{f}(x)=[x]$ for all $x\in X$. Clearly $\overline{f}$ is surjective.
Note that $xy^{-1}\in\ker\overline{f}$ for all $x,y\in X$ such that $\sigma_x=\sigma_y$, and in this case $\lambda_x=\lambda_y$.
Hence $\{ xy^{-1} : x,y\in X \text{ and }[x]=[y]\}\subseteq\Soc(G)$. Let $I$ be the ideal of $G$ generated by $\{xy^{-1} : x,y\in X \text{ and }[x]=[y]\}$. Let $\pi\colon G\to G/I$ be the natural map. Since $I\subseteq \ker\overline{f}$, there exists a homomorphism of skew braces $\tilde{f}\colon G/I\to G(\overline{X},\overline{r})$ such that $\overline{f}=\tilde{f}\pi$. Note also that the map $\overline{X}\to G/I$ defined by $[x]\mapsto xI$ is a homomorphism of solutions $(\overline{X},\overline{r})\to (G/I,r_{G/I})$. By Theorem \ref{thm:GVbraces}, there exists a unique homomorphism of skew braces $g\colon G(\overline{X},\overline{r})\to G/I$ such that $g([x])=xI$ for all $x\in X$. Since $\tilde{f}(xI)=[x]$ for all $x\in X$, we have that $g$ is bijective and $g^{-1}=\tilde{f}$. 
Since $I\subseteq\Soc(G)$ and $\Soc(G)$ is a finitely generated free abelian group, $I$ also is a finitely generated free abelian group. Hence $G$ is poly-$\Z$. Therefore (3) holds by induction.

$(2)\implies (1).$ Assume that $(X,r)$ is not a multipermutation solution. By Theorem \ref{thm:multiG}, $(G,r_G)$ and $(\mathcal{G},r_{\mathcal{G}})$ are not multipermutation solutions. Since $G$ and $\mathcal{G}$ are skew braces of abelian type, by Proposition \ref{pro:bracefinmpl},
the skew braces $G$ and $\mathcal{G}$ are not right nilpotent, i.e. $G^{(n)}\neq\{ 0\}$ and $\mathcal{G}^{(n)}\neq\{0\}$ for all positive integer $n$. Since $\mathcal{G}$ is finite, there exists a positive integer $n$ such that $\mathcal{G}^{(n+1)}=\mathcal{G}^{(n)}\neq \{0\}$. Let $H=G^{n+1}$. We shall prove that $H$ has trivial center. Let $z\in Z(H)$. Since $\Soc(G)$ has finite index in $G$ and $G$ is torsion free, without loss of generality we may assume that $z\in \Soc(G)$. Notice that if $h\in H$, then
\begin{equation}\label{eq:BCV1}
\lambda_h(z)=-h+hz=zh-h=z+h-h=z.
\end{equation}
Let $X_1,\dots ,X_k$ be the orbits of $X$ under the action of $\mathcal{G}^{(n)}$. These are the orbits of $X$ under the action of $G^{(n)}$ though the map $\lambda$. By Theorem \ref{thm:involstruct}, $(G,+)$ is free abelian with basis $X$. Hence the element $z$ can be uniquely written as
\[ z=z_1+\dots +z_k,\]
where each $z_i\in \langle X_i\rangle_+$. From the uniqueness of the decomposition of $z$ and (\ref{eq:BCV1}), one obtains that $\lambda_h(z_i)=z_i$ for all $i\in \{ 1,\dots ,k\}$ and $h\in H$. Now we write each $z_i$ as
\[ z_i=\sum_{t\in X_i}n_tt,\]
where $n_t\in\Z$. Note that
\begin{align*}
    H=G^{(n+1)}&=G^{(n)}*G=\langle a*b : a\in G^{(n)}, \, b\in G\rangle_+\\
    &=\langle \lambda_a(b)-b : a\in G^{(n)}, \, b\in G\rangle_+\\
    &=\langle \lambda_a(x)-x : a\in G^{(n)}, \, x\in X\rangle_+\\
    &=\langle y-x : x,y\in X_i, \, i\in \{1,\dots ,k\}\rangle_+.
\end{align*}
Hence $\sum_{t\in X_i}n_t=0$ for all $i\in\{ 1,\dots ,k\}$. Let $x,y\in X_i$. Then there exists $g\in G^{(n)}$ such that $\lambda_g(x)=y$. From $\mathcal{G}^{(n+1)}=\mathcal{G}^{(n)}$ it follows that
\[ G^{(n)}=G^{(n+1)}+(\Soc(G)\cap G^{(n)})=H+(\Soc(G)\cap G^{(n)}).\]
Thus $g=g_1+g_2$, where $g_1\in H$ and $g_2\in \Soc(G)\cap G^{(n)}$. Since $g_2\in\Soc(G)$, $g=g_2g_1$. Therefore
\[ y=\lambda_g(x)=\lambda_{g_2g_1}(x)=\lambda_{g_2}\lambda_{g_1}(x)=\lambda_{g_1}(x).\]
Since $z_i=\lambda_{g_1}(z_i)=\sum_{t\in X_i}n_t\lambda_{g_1}(t)$, we conclude that $n_x=n_y$. Hence $n_x=n_y$ for all $x,y\in X_i$. Since $\sum_{t\in X_i}n_t=0$, it follows that $n_t=0$ for all $t\in X_i$ and all $i\in \{ 1,\dots ,k\}$. Therefore $z=0=1$ and the result follows.
\end{proof}

\index{Group!with the unique product property}
Recall that a group $G$ has the \emph{unique product property} if 
for all finite non-empty subsets $A$ and $B$ of $G$ there exists $x\in G$ 
that can be written uniquely as $x = ab$ with $a\in A$ and $b\in B$. 
% We refer to~\cite{MR798076} for more information related to 
% the unique product property. 

It is natural to ask when $G(X,r)$ has the unique product property. By Theorem~\ref{thm:BCV}, if $(X,r)$ is an involutive
multipermutation solution, then $G(X,r)$ has the unique product property since $G(X,r)$ is left orderable. 

\begin{example}
    \label{pro:4-19}
	Let $X=\{1,2,3,4\}$ and $r(x,y)=(\sigma_x(y),\tau_y(x))$ be the irretractable involutive solution given by 
	\begin{align*}
		&\sigma_1=(12), && \sigma_2=(1324), && \sigma_3=(34), && \sigma_4=(1423),\\
		&\tau_1=(14), &&\tau_2=(1243), && \tau_3=(23), && \tau_4=(1342).
	\end{align*}
	We claim that the group $G(X,r)$ with 
	generators
	$x_1,x_2,x_3,x_4$ and relations
	\begin{align*}
		& x_1^2=x_2x_4,
		&& x_1x_3=x_3x_1,
		&& x_1x_4=x_4x_3,\\
		& x_2x_1=x_3x_2,
		&& x_2^2=x_4^2,
		&& x_3^2=x_4x_2.
	\end{align*}
	does not have the unique product property. Let $x=x_1x_2^{-1}$ and $y=x_1x_3^{-1}$ and 
    \begin{multline}
    \label{eq:Promislow}
    S=\{ x^2y,
    y^2x,
    xyx^{-1},
    (y^2x)^{-1},
    (xy)^{-2},
    y,
    (xy)^2x,
    (xy)^2,\\
    (xyx)^{-1},
    yxy,
    y^{-1},
    x,
    xyx, 
    x^{-1}
	\}.
    \end{multline}
    To prove that $G(X,r)$ does not have
    the unique product property it is enough to prove that 
    each $s\in S^2=\{s_1s_2:s_1,s_2\in S\}$ admits at least two different decompositions 
    of the form $s=ab=uv$ for $a,b,u,v\in S$. To perform these calculations we 
    use the injective group homomorphism $G\to\GL(5,\Z)$ given by  
	\begin{align*}
	&x_1\mapsto\left(\begin{smallmatrix}
	0 & 1 & 0 & 0 & 1\\
	1 & 0 & 0 & 0 & 0\\
	0 & 0 & 1 & 0 & 0\\
	0 & 0 & 0 & 1 & 0\\
	0 & 0 & 0 & 0 & 1
  	\end{smallmatrix}\right),
  	&&
	x_2\mapsto\left(\begin{smallmatrix}
	0 & 0 & 0 & 1 & 0\\
	0 & 0 & 1 & 0 & 1\\
	1 & 0 & 0 & 0 & 0\\
	0 & 1 & 0 & 0 & 0\\
	0 & 0 & 0 & 0 & 1
  	\end{smallmatrix}\right),
  	\\
  	&x_3\mapsto\left(\begin{smallmatrix}
	1 & 0 & 0 & 0 & 0\\
	0 & 1 & 0 & 0 & 0\\
	0 & 0 & 0 & 1 & 1\\
	0 & 0 & 1 & 0 & 0\\
	0 & 0 & 0 & 0 & 1
  	\end{smallmatrix}\right),
  	&&
  	x_4\mapsto\left(\begin{smallmatrix}
	0 & 0 & 1 & 0 & 0\\
	0 & 0 & 0 & 1 & 0\\
	0 & 1 & 0 & 0 & 0\\
	1 & 0 & 0 & 0 & 1\\
	0 & 0 & 0 & 0 & 1
  	\end{smallmatrix}\right).
  	\end{align*}
  	This faithful representation of $G(X,r)$ allows us to compute all possible products of the form $s_1s_2$ for all $s_1,s_2\in S$. 
  	By inspection, each element of $S^2$ admits at least two different representations.
\end{example}

\begin{theorem}
\label{thm:CJO_mp}
    Let $A$ be a finite brace of abelian type with multiplicative group $G$. Then there exists a finite solution $(X,r)$ 
    such that $\Ret(X,r)$ is isomorphic to $(A,r_A)$ and $\mathcal{G}(X,r)\cong G$.
\end{theorem}

\begin{proof}
    Let $X=A\times\Z/(2)$. For $a,b\in A$, let 
    \begin{align*}
        &\varphi_{(a,0)}(b,0)=(b,0), && \varphi_{(a,0)}(b,1)=(b,1),\\
        &\varphi_{(a,1)}(b,0)=(a'\circ b,0), && \varphi_{(a,1)}(b,1)=(\lambda^{-1}_a(b),1).
    \end{align*}
    The maps $\varphi_{(a,\epsilon)}$ are invertible 
    for all $a\in A$ and $\epsilon\in\Z/(2)$. In fact, 
    \[
    \varphi_{(a,0)}^{-1}=\id,
    \quad
    \varphi_{(a,1)}^{-1}(b,\epsilon)=\begin{cases}
    a\circ b & \text{if $\epsilon=0$,}\\
    \lambda_a(b) & \text{if $\epsilon=1$.}
    \end{cases}
    \]
    So we need to check that 
    \begin{align}
    \label{eq:CJO_tocheck}
    ((a,\epsilon_1)\cdot (b,\epsilon_2))\cdot ((a,\epsilon_1)\cdot (c,\epsilon_3))
    =((b,\epsilon_2)\cdot (a,\epsilon_1))\cdot ((b,\epsilon_2)\cdot (c,\epsilon_3))
    % \shortintertext{and}
    % \varphi_{(a,\epsilon_1)}((-h,y)+(k,z)+(h,y))=-\varphi_{(g,x)}(h,y)+\varphi_{(g,x)}(k,z)+\varphi_{(g,x)}(h,y).
    \end{align}
    holds for all $a,b,c\in A$ and $\epsilon_1,\epsilon_2,\epsilon_3\in\Z/(2)$. There are several cases to consider. 
    Let us assume first
    that $(\epsilon_1,\epsilon_2,\epsilon_3)=(1,1,0)$. Then~\eqref{eq:CJO_tocheck} turns out to be 
    \[
    (\lambda^{-1}_a(b)'\circ a'\circ c,0)=(\lambda^{-1}_b(a)'\circ b'\circ c,0),
    \]
    which holds for all $a,b\in A$, as  
    \[
    \lambda^{-1}_a(b)'\circ a'\circ c=(a\circ \lambda^{-1}_a(b))'\circ c=(a+b)'\circ c=(b+a)'\circ c=\lambda^{-1}_b(a)'\circ b'\circ c
    \]
    because $A$ is of abelian type. Let us now deal with the case $(\epsilon_1,\epsilon_2,\epsilon_3)=(1,1,1)$. In this case 
    Equality~\eqref{eq:CJO_tocheck}
    turns out to be equivalent to 
    \[
    (\lambda^{-1}_a(b),1)\cdot (\lambda^{-1}_a(c),1)
    =(\lambda^{-1}_b(a),1)\cdot (\lambda^{-1}_b(c),1).
    \]
    Since $A$ is of abelian type,
    \begin{align*}
        (\lambda^{-1}_a(b),1)\cdot (\lambda^{-1}_a(c),1)
        &= \lambda^{-1}_{\lambda^{-1}_a(b)}\lambda^{-1}_a(c)
        = \lambda^{-1}_{a\circ\lambda^{-1}_a(b)}(c)
        = \lambda^{-1}_{a+b}(c)\\
        &= \lambda^{-1}_{b+a}(c)
        = \lambda^{-1}_{\lambda^{-1}_b(a)}\lambda^{-1}_b(c)
        = (\lambda^{-1}_b(a),1)\cdot (\lambda^{-1}_b(c),1).
    \end{align*}
    The other cases are easier and require straightforward calculations. 
    
    Let $\psi\colon G\to\Sym_X$, $\psi(g)=\varphi_{(g',1)}$. Since 
    \[
    \psi(g)(0,0)=\varphi_{(g',1)}(0,0)=(g\circ 0,0)=(g,0),
    \]
    it follows that $\psi$ is injective. Moreover, $\psi$ is a group homomorphism, as
    \begin{align*}
        \psi(a)\psi(b)(c,0) &= \psi(a)\varphi_{(b,1)}(c,0)
        =\psi(a)(b\circ c,0)\\
        &=\varphi_{(a,1)}(b\circ c,0)
        =(a\circ b\circ c,0)
        =\varphi_{(a\circ b,1)}(c,0)=\psi_{(a\circ b)}(c,0)
    \shortintertext{and}
        \psi(a)\psi(b)(c,1) &= \varphi_{(a',1)}\varphi_{(b',1)}(c,1)
        =\varphi_{(a',1)}(\lambda^{-1}_{b'}(c),1)\\
        &=(\lambda^{-1}_{a'}\lambda^{-1}_{b'}(c),1)
        =(\lambda_{a\circ b}(c),1)
        =\varphi_{(a\circ b,1)}(c,1)
        =\psi(a\circ b)(c,1).
    \end{align*}
    Since $\psi$ is an injective group homomorphism, 
    \[
    G\simeq\psi(G)\simeq\langle \psi(a):a\in A\rangle=\langle\varphi_{(a,1)}:a\in A\rangle\simeq\mathcal{G}(X,r).
    \]
    
    Consider the equivalence relation on $X$ given by $x\sim y$ if and only if $\varphi_x=\varphi_y$. 
    As usual $[x]$ denotes the equivalence class of the element $x\in X$ and $\overline{X}$ is the set of equivalence classes. 
    A straightforward computation shows that 
    $(a,0)\sim (0,1)$ for all $a\in A$. This implies that 
    \[
    \overline{p}\colon \overline{X}\to A, 
    \quad
    \overline{p}([(a,\epsilon)])=\begin{cases}
    0 & \text{if $\epsilon=0$},\\
    a & \text{if $\epsilon=1$},
    \end{cases}
    \]
    is a well-defined surjective map. 
    We claim that $\overline{p}$ is injective. Let $(a,\epsilon_1)\in\overline{X}$ and
    $(b,\epsilon_2)\in\overline{X}$ be such that $\overline{p}([(a,\epsilon_1)])=\overline{p}([(b,\epsilon_2)])$. 
    Since $[(a,0)]=[(0,1)]$ for all $a\in A$, we only need to consider the case where $\epsilon_1=\epsilon_2=1$. 
    In this case, 
    \[
    a=\overline{p}([(a,\epsilon_1)])=\overline{p}([(b,\epsilon_2)])=b.
    \]
    Thus $\overline{p}$ is bijective. Now 
    \begin{align*}
        r_A(\overline{p}[(a,1)],\overline{p}[(b,1)]) 
        &= r_A(a,b)
        = (\lambda_a(b),\mu_b(a))
        = (\overline{p}[\lambda_a(b),1],\overline{p}[\mu_b(a),1])\\
        &= (\overline{p}\varphi_{(a,1)}(b,1),\overline{p}...).
    \end{align*}
\end{proof}

\begin{theorem}
    Let $A$ be...
\end{theorem}

\begin{proof}

\end{proof}

\section{Exercises}

\begin{prob}
\label{prob:bounded_mpl}
    Let $(X,r)$ be a solution of finite multipermutation level $m$. Prove that any homomorphic image of $(X,r)$ 
    is a solution of finite multipermutation level $\leq m$. 
\end{prob}

\begin{prob}\label{prob:CJKVV}
    Let $(X,r)$ be a solution to the YBE. We say that $(X,r)$ is square-free if $r(x,x)=(x,x)$ for all $x\in X$. Suppose that $(X,r)$ is a square-free multipermutation solution of level $m>0$. Prove that $\left( \mathcal{G}(X,r),r_{\mathcal{G}(X,r)}\right)$ has multipermutation level $m-1$ and $\left( G(X,r),r_{G(X,r)}\right)$ has multipermutation level $\leq m$.
\end{prob}

\begin{prob}
	\label{prob:4-13}
	Let $X=\{1,2,3,4\}$ and $r(x,y)=(\sigma_x(y),\tau_y(x))$ be the irretractable involutive solution given by
	\begin{align*}
		&\sigma_1=(34), && \sigma_2=(1324), && \sigma_3=(1423), && \sigma_4=(12),\\
		&\tau_1=(24), &&\tau_2=(1432), && \tau_3=(1234), && \tau_4=(13).
	\end{align*}
	Prove that $G(X,r)$ 
% 	The structure group $G(X,r)$ with generators
% 	$x_1,x_2,x_3,x_4$ and relations
% 	\begin{align*}
% 		& x_1x_2=x_2x_4,
% 		&& x_1x_3=x_4x_2,
% 		&& x_1x_4=x_3^2,\\
% 		& x_2x_1=x_3x_4,
% 		&& x_2^2=x_4x_1,
% 		&& x_3x_1=x_4x_3.
% 	\end{align*}
	does not have the unique product property.
\end{prob}

\section{Open problems}

\begin{problem}\label{openprob:LV}
    Let $(X,r)$ be a finite involutive solution to the YBE such that $G(X,r)$ has the unique product property. Is $(X,r)$ a multipermutation solution?
\end{problem}

\begin{problem}
    Does the group $G(X,r)$...
    A linear representation of this group is...
\end{problem}

\section{Notes}

Multipermutation involutive solutions were introduced in~\cite{MR1722951}. 
The notion was extended to the finite non-involutive case in~\cite{MR3974961}, and to the general case in~\cite{MR4457900}.

% \index{Ced\'o, F.}
% \index{Jespers, E.}
% \index{Van Antwerpen, A.}
% \index{Kubat, \L.}
% \index{Verwimp, C.}
Theorem~\ref{thm:CJKAV} was proved for the associated solution to finite skew braces in \cite{MR3763907} and the general case in \cite{MR4457900}. %by Ced\'o, Jespers, Kubat, van Antwerpen and Verwimp. 

Proposition~\ref{pro:mpl} and Lemma \ref{pro:mpl_subsol} were proved in~\cite{MR3177933} 
for involutive solutions. The general case goes back to \cite{MR4457900}. 

Theorem~\ref{thm:BCV} combines several results. The implication $(1)\implies (3)$ was proved in \cite{MR2189580}. The implication $(1)\implies (4)$ was proved in \cite{MR3572046}. The implication $(3)\implies (1)$ was proved in \cite{MR3815290}. 

Theorem~\ref{thm:CJO_mp} appears in~\cite{MR3177933}. 

Non-involutive multipermutation solutions...

The set~\eqref{eq:Promislow} appears in the work of Promislow~\cite{MR940281}.

Exercise \ref{prob:CJKVV} appears in \cite{MR4457900}. 
Exercise~\ref{prob:4-13} appears in the book of Jespers and Okni\'nski, see~\cite[Example 8.2.14]{MR2301033}. 

Open problem \ref{openprob:LV} appears in \cite{MR3974961}.
